%!TEX root = ../main.tex

\section{Hungry Hungry Hippos Layer to Model Discrete Sequences}
\label{sec:method}

To understand the gap between SSMs and attention on language modeling, we
examine two synthetic language modeling tasks.
These tasks motivate our \hthree layer to add a discrete SSM (based on shift matrix) and multiplicative interaction to effectively model discrete sequences.
We then show that the \hthree layer is expressive enough to solve these synthetic tasks, and that this understanding leads to better performance on a real language modeling benchmark.

\subsection{Motivation: Synthetic Language Modeling Tasks\label{sec:synthetics}}

We describe two closely-related synthetic tasks, summarized in Table~\ref{table:synthetic_tasks}. 
Olsson et al.~\citep{olsson2022context} argue that the ability to solve (variants of) these tasks accounts for the majority of the in-context learning capability of Transformers, and more intuition is given in Appendix~\ref{sec:app_exp_details}. 
\begin{table}[h]
    % \captionsetup{font=small}
    \small
    \centering
    \caption{\label{table:synthetic_tasks} Synthetic language modeling tasks.}
    %   \vspace{1em}
    {
        \begin{tabular}{@{}|l|l|l|c|c|@{}}
        %   \specialrule{.15em}{.05em}{.05em}
        \hline
        Task & Input & Output & Sequence Length & Vocab Size  \\ % & Training time \\
        %   \specialrule{.15em}{.05em}{.05em}
        \hline
        Induction Head & \textit{a b c d e $\vdash$ f g h i $\dots$ x y z $\vdash$} & \textit{f} & 30 & 20 \\
        Associative Recall & \textit{a 2 c 4 b 3 d 1 a} & \textit{2} & 20 & 10 \\ \hline
        % Partial Copy & \textit{a b c @ d e f * g h i} & \textit{@ d e f * . . .} & 20 & 10 \\ \hline
        \end{tabular}
    }
\end{table}

The \textbf{Induction Head} task tests how well a model can recall content after a special token (e.g., $\vdash$ in~\cref{table:synthetic_tasks}).
At the end of the sequence, the model must recall the token that appeared immediately after the special token earlier in the sequence. \textbf{Associative Recall}~\citep{ba2016using} is similar to the induction head task, but requires the model to remember multiple key-value pairs.
At the end of the sequence, the model must recall a specific value belonging to a specific key.
\begin{table}[h]
    % \captionsetup{font=small}
    \small
    \centering
    % \vspace{-1em}
    \caption{\label{table:synthetics} Evaluation of 2-layer models on synthetic language tasks.}
    %   \vspace{1em}
    {
        \begin{tabular}{@{}|c|c|ccc|c|@{}}
        %   \specialrule{.15em}{.05em}{.05em}
        \hline
        Task & Random & S4D & Gated State Spaces & H3 & Attention  \\ % & Training time \\
        %   \specialrule{.15em}{.05em}{.05em}
        \hline
        Induction Head & 5.0 & 35.6 & 6.8 & \textbf{100.0} & \textbf{100.0} \\
        Associative Recall & 25.0 & 86.0 & 78.0 & 99.8 & \textbf{100.0}  \\ \hline
        % Gated State Spaces & 78.0 & 6.8 & 88.2 \\
        % LSTM & -- & -- & -- \\
        % H3 & 99.8 & \textbf{100.0} & \textbf{99.6}  \\ \hline
        % Attention & \textbf{100.0} & \textbf{100.0} & 71.0 \\ \hline
        \end{tabular}
    }
    % \vspace{-1.5em}
\end{table}

% \begin{table}[h]
%     % \captionsetup{font=small}
%     \small
%     \centering
%     % \vspace{-1em}
%     \caption{\label{table:synthetics} Evaluation of 2-layer GPT models on in-context learning synthetic tasks.}
%     %   \vspace{1em}
%     {
%         \begin{tabular}{@{}|c|ccc|@{}}
%         %   \specialrule{.15em}{.05em}{.05em}
%         \hline
%         Model & Associative Recall & Induction Head & Partial Copy  \\ % & Training time \\
%         %   \specialrule{.15em}{.05em}{.05em}
%         \hline
%         Random & 50.0 & 5.0 & 5.0 \\
%         S4D & 86.0 & 35.6 & 98.5  \\
%         Gated State Spaces & 78.0 & 6.8 & 88.2 \\
%         % LSTM & -- & -- & -- \\
%         H3 & 99.8 & \textbf{100.0} & \textbf{99.6}  \\ \hline
%         Attention & \textbf{100.0} & \textbf{100.0} & 71.0 \\ \hline
%         \end{tabular}
%     }
%     % \vspace{-1.5em}
% \end{table}
Table~\ref{table:synthetics} (for two-layer models) shows that S4D~\citep{gu2022parameterization} and Gated State Spaces~\citep{mehta2022long} both fail to model these synthetic languages, which suggests they may not have the expressivity for general language.
We argue that these failures suggest two missing capabilities: (i) to remember tokens that appear after a particular event (e.g., the special token in the induction head task), and (ii) to compare tokens across the sequence (e.g., comparing keys to decide which value to recall).
Attention has both these capabilities: it can compare tokens by constructing the \textit{quadratic} attention matrix $\vQ \vK^\top$, and it can recall tokens by direct copying (multiplying $\softmax(\vQ \vK^\top)$ with $\vV$).
In Section~\ref{sec:method_h3}, we design our new layer \hthree to enable these capabilities in SSMs, narrowing the expressivity gap between SSMs and attention.

\subsection{\hthree Layer\label{sec:method_h3}}
\hthree uses SSMs with shift and diagonal matrices, along with multiplicative
operations against projections of the input to capture the missing capabilities identified by the synthetics.

\textbf{High-level Intuition.}
(i) To remember tokens from the past, we want the state $x_i$ to copy from the input $u_i$, and then pass that information to the next state $x_{i+1}$. As $x_{i+1}$ relates to $x_i$by $\vA x_i$, we use a discrete SSM with a shift matrix $\vA$ (described formally below) that shifts the elements of a state vector (e.g., mapping $[a, b, c] \to [0, a, b]$).
(ii) To compare tokens across the sequence, we use multiplicative interaction: the output of an SSM, containing information from previous time steps, is multiplied with the input at the current time steps, thus measuring similarity between tokens.

\hthree is loosely inspired by linear attention (\cref{sec:background}): we project the input $u$ to get three signals $\vQ, \vK, \vV$.
Then we replace the non-linearity $\phi(\vK)$ with an SSM where $\vA$ is a shift matrix ($\mathrm{SSM}_\mathrm{shift}$), and we replace the summation $S_i$ with a SSM with diagonal $\vA$ ($\mathrm{SSM}_\mathrm{diag}$).
The output, for the case of head dimension $d_h = 1$, is:
\begin{equation*}
  \vQ \odot \mathrm{SSM}_\mathrm{diag}(\mathrm{SSM}_\mathrm{shift}(\vK) \odot \vV),
\end{equation*}
where $\odot$ denotes pointwise multiplication.
We can view this form as stacking two SSMs with multiplicative
interaction (each is a ``hungry hippo'', hence the name of our layer).
A more formal connection between linear attention, time-varying systems, and \hthree can be found in Appendix~\ref{app:linear_attention}.

\textbf{Remembering Key Tokens: Shift and Diagonal SSMs.}
The shift and diagonal SSMs are designed to address the capability to log tokens after particular events.
In the shift SSM, we constrain $\vA \in \mathbb{R}^{m \times m}$ to be a shift matrix
$
\vA_{i, j} = 
    \begin{cases}
    1 & \text{for } i - 1 = j\\
    0 & \text{otherwise}
    \end{cases}
$.
The action of this matrix on the hidden state $x_i$ is to shift each coordinate
down by one---thereby creating a ``memory'' of the previous states.
For example, if $\vB = e_1$, the first basis vector, then
$x_i = [u_i, u_{i-1}, \dots, u_{i-m+1}]$ contains the inputs from the previous $m$
time steps.
We learn $\vB$ and $\vC$ ($\vB$ can also be fixed to $e_1$ for
simplicity, in which case the output is a 1D conv.\ with kernel size $m$).

The diagonal SSM constrains $\vA$ to be diagonal and initializes it from the diagonal version of HiPPO (S4D~\citep{gu2022parameterization}).
This parameterization allows the model to remember state over the entire sequence.
The shift SSM can detect when a particular event occurs, and the diagonal SSM can remember a token afterwards for the rest of the sequence.

\textbf{Multiplicative Interaction for Comparison.}
We take the multiplicative interactions from linear attention, but they provide another missing capability when combined with a shift matrix: comparing tokens across the sequence.
The multiplicative interactions between the output of the shift SSM and the $\vV$ projection mimics local multiplicative interactions in linear attention (depending on the size of the hidden state).
Similarly, multiplicative interactions with the $\vQ$ projection and the output of the diagonal SSM allows comparisons between tokens over the entire sequence.

\textbf{\hthree Layer.} The overall layer is given in Algorithm~\ref{alg:h3} and shown schematically in Figure~\ref{fig:banner} (left).
We use the \hthree layer to construct a model in the same style as Transformers by interleaving it with MLPs,
connected by residual connection and layer norm (i.e., pre-norm architecture~\citep{baevski2018adaptive}).
We will also consider a hybrid \hthree-attention model (two attention layers while the rest are \hthree, \cref{sec:expressivity,sec:evaluation}).
\begin{algorithm}[H]
  \algsetup{linenosize=\tiny}
  \caption{\label{alg:h3} H3 Layer}
  \small
  \begin{algorithmic}[1]
    \REQUIRE Input sequence $u \in \mathbb{R}^{N \times d}$ from the previous layer, weight
    matrices $\vW_Q, \vW_K, \vW_V, \vW_O \in \mathbb{R}^{d \times d}$, a shift SSM $\mathrm{SSM}_\mathrm{shift}$, a diagonal SSM $\mathrm{SSM}_{\mathrm{diag}}$, head dimension $d_h$.
    \STATE Compute $\vQ = u \vW_Q, \vK = u \vW_K, \vV = u \vW_V \in \mathbb{R}^{N \times d}$.
    \STATE Pass $\vK$ through the shift SSM: $\overline{\vK} = \mathrm{SSM}_\mathrm{shift}(\vK) \in \mathbb{R}^{N \times d}$.
    \STATE Split $\vQ, \overline{\vK}, \vV$ into $H$ ``heads'' ($\vQ^{(h)}, {\overline{\vK}}^{(h)}, \vV^{(h)}$
    for $h = 1, \dots, H$), each a sequence of $N$ vectors of size $d_{h} = d / H$.
    \FOR{$1 \leq h \leq H$}
    \STATE Take the batched outer product ${\overline{\vK}}^{(h)} (\vV^{(h)})^\top \in \mathbb{R}^{N \times d_h \times d_h}$ (batched in the $N$-dimension) and pass it through a diagonal SSM: $\vK\vV^{(h)} = \mathrm{SSM}_{\mathrm{diag}}({\overline{\vK}}^{(h)} (\vV^{(h)})^\top) \in \mathbb{R}^{N \times d_{h} \times {d_{h}}}$.
    \STATE Batch-multiply by $\vQ$:  $\vO^{(h)} = [\vQ^{(h)}_1 \vK\vV^{(h)}_1, \dots, \vQ^{(h)}_N \vK\vV^{(h)}_N]\in  \mathbb{R}^{N \times d_h}$ (batched in the $N$-dimension).
    \ENDFOR
    \STATE Concatenate the output $\vO^{(h)}$ of each head, and multiply by the output
    projection matrix $\vW_O \in \mathbb{R}^{d \times d}$.
  \end{algorithmic}
\end{algorithm}

\paragraph{Efficiency}
We show that \hthree scales as $O(N \log N)$ with
sequence length $N$---asymptotically more efficient than attention, which typically requires $O(N^2d)$ time and $O(N^2)$ space\footnote{There are several memory-efficient
algorithms for attention~\citep{rabe2021self,dao2022flashattention}, though
their time complexity is still quadratic in $N$, which is a lower-bound for attention~\citep{keles2022computational}.} (proof in~\cref{sec:app_h3_complexity}).
\begin{proposition}\label{thm:h3_complexity}
  Let $N$ be the sequence length, $d$ be the hidden dimension, and assume that
  the head dimension $d_h$ is of order $O(1)$.
  Then the H3 layer takes $O(d^2N + d N \log N)$ time and $O(dN)$ space to compute.
\end{proposition}


\subsection{Expressivity}
\label{sec:expressivity}

We show that \hthree can model our synthetic languages, as well as natural language on OpenWebText~\citep{Gokaslan2019OpenWeb}.
We also present a hybrid \hthree-attention extension that outperforms Transformers on OpenWebText.

\textbf{Mechanism for Solving Associative Recall with H3.}
\hthree is expressive enough to solve our synthetic language modeling tasks, as shown in Table~\ref{table:synthetics}.
Figure~\ref{fig:banner} (middle) shows a mechanism for a single \hthree layer to solve the associative recall task for a particular key-value pair $(a, 3)$.
The shift SSM and following multiplicative interaction act as a gate on whether to let a value through to the diagonal SSM, based on whether the previous token was key $a$.
The diagonal SSM stores the value $3$ in memory, and continually outputs it.
The final multiplicative interaction gates whether to let the diagonal SSM's output through---based on whether the current input token is the key $a$.
We formally construct the weights of an \hthree layer to solve this task in Appendix~\ref{sec:app_expressivity}.

\subsection{Model Ablations}
\label{sec:exp:ablations}

We perform a series of detailed ablations on components of our model,
focusing on the setting of language modeling with size $\approx 350$M models at Chinchilla token counts (same setting as \cref{fig:lm-scaling}).

\subsubsection{Architecture}
\cref{tab:ablations-arch} investigates the effects of the architecture (block) and its inner SSM layer (\cref{fig:architecture}).
We find that
\begin{itemize}[leftmargin=*,itemsep=0pt]
  \item Among previous non-selective (LTI) SSMs, which are equivalent to global convolutions, performance is very similar.
  \item Replacing the complex-valued S4 variant from previous work with a real-valued one does not affect performance much,
    suggesting that (at least for LM) real-valued SSMs may be a better choice when accounting for hardware efficiency.
  \item Replacing any of these with a selective SSM (S6) significantly improves performance, validating the motivation of \cref{sec:method}.
  \item The Mamba architecture performs similarly to the H3 architecture (and seems slightly better when using a selective layer).
\end{itemize}

We also investigate interleaving the Mamba block with other blocks such as MLP (a traditional architecture) MHA (a hybrid attention architecture)
in \cref{sec:exp-details:lm:scaling-ablations}.

\subsubsection{Selective SSM}

\cref{tab:ablations-variable} ablates the selective SSM layer by considering different combinations of selective $\dt$, $\B$, and $\C$ parameters (\cref{alg:s6}),
showing that $\dt$ is the most important parameter due to its connection to RNN gating (\cref{thm:gating}).

\cref{tab:ablations-init} considers different initializations of the SSM, which have been shown to make a large difference in some data modalities and settings \citep{gu2022efficiently,gu2022parameterization}.
On language modeling, we find that simpler real-valued diagonal initializations (S4D-Real, row 3) instead of more standard complex-valued parameterizations (S4D-Lin, row 1)
perform better.
Random initializations also work well, consistent with findings from prior work~\citep{mehta2023long}.

\cref{tab:ablations-dt} and \cref{tab:ablations-N} consider varying the dimension of the $\dt$ and $(\B, \C)$ projections respectively.
Changing them from static to selective provides the most benefit,
while increasing the dimensions further generally improves performance modestly with a small increase in parameter count.


\begin{table}
  \caption{
    (\textbf{Ablations: Architecture and SSM layer}.)
    The Mamba block performs similarly to H3 while being simpler.
    In the inner layer, there is little difference among different parameterizations of LTI models,
    while selective SSMs (S6) provide a large improvement.
    More specifically, the S4 (real) variant is S4D-Real and the S4 (complex) variant is S4D-Lin.
  }
  \centering
  \begin{tabular}{@{}llll@{}}
    \toprule
    \textsc{Model} & \textsc{Arch.} & \textsc{SSM Layer} & \iftoggle{arxiv}{\sc{Perplexity}}{\sc{Ppl}} \\
    \midrule
    Hyena          & H3             & Hyena              & $10.24$ \\ %
    H3             & H3             & S4 (complex)       & $10.30$ \\ %
    -              & H3             & S4 (real)          & $10.34$ \\ %
    -              & H3             & S6                 & $\mathbf{8.95}$ \\ %
    \bottomrule
  \end{tabular}
  \qquad
  \begin{tabular}{@{}llll@{}}
    \toprule
    \textsc{Model} & \textsc{Arch.} & \textsc{SSM Layer} & \iftoggle{arxiv}{\sc{Perplexity}}{\sc{Ppl}} \\
    \midrule
    -              & Mamba          & Hyena              & $10.75$ \\ %
    -              & Mamba          & S4 (complex)       & $10.54$ \\ %
    -              & Mamba          & S4 (real)          & $10.56$ \\ %
    Mamba          & Mamba          & S6                 & $\mathbf{8.69}$ \\ %
    \bottomrule
  \end{tabular}
  \label{tab:ablations-arch}
\end{table}


\begin{figure}[!t]
  \begin{minipage}{.5\linewidth}
    \centering
    \captionsetup{type=table}
    \caption{
      (\textbf{Ablations: Selective parameters}.)
      $\dt$ is the most important parameter (\cref{thm:gating}), but using multiple selective parameters together synergizes.
    }
    \begin{tabular}{@{}llll@{}}
      \toprule
      \sc{Selective} $\dt$ & \sc{Selective} $\B$ & \sc{Selective} $\C$ & \iftoggle{arxiv}{\sc{Perplexity}}{\sc{Ppl}} \\
      \midrule
      \xmark               & \xmark              & \xmark              & 10.93 \\ %
      \xmark               & \cmark              & \xmark              & 10.15 \\ %
      \xmark               & \xmark              & \cmark              & 9.98 \\ %
      \cmark               & \xmark              & \xmark              & 9.81 \\ %
      \cmark               & \cmark              & \cmark              & 8.71 \\
      \bottomrule
    \end{tabular}
    \label{tab:ablations-variable}
  \end{minipage}
\hfill
\begin{minipage}{.45\linewidth}
  \centering
  \captionsetup{type=table}
  \caption{
    (\textbf{Ablations: Parameterization of $\A$}.)
    The more standard initializations based on S4D-Lin~\citep{gu2022parameterization} perform worse than S4D-Real or a random initialization,
    when the SSM is selective.
  }
  \begin{tabular}{@{}lll@{}}
    \toprule
    $\A_n$ \sc{Initialization}          & \sc{Field} & \iftoggle{arxiv}{\sc{Perplexity}}{\sc{Ppl}} \\
    \midrule
    $\A_n = -\frac{1}{2} + n i$         & Complex    & 9.16 \\
    $\A_n = -1/2$                       & Real       & 8.85 \\
    $\A_n = -(n+1)$                     & Real       & 8.71 \\
    $\A_n \sim \exp(\mathcal{N}(0, 1))$ & Real       & 8.71 \\
    \bottomrule
  \end{tabular}
  \label{tab:ablations-init}
\end{minipage}
\end{figure}

\begin{figure}[!t]
  \begin{minipage}{.32\linewidth}
    \captionsetup{type=table}
    \caption{
      (\textbf{Ablations: Expressivity of $\dt$}.)
      The selection mechanism of $\dt$ constructs it with a projection of the input.
      Projecting it even to dim.\ $1$ provides a large increase in performance;
      increasing it further provides further improvements at the cost of a modest increase in parameters.
      State size fixed to $N=16$.
    }
    \centering
    \begin{tabular}{@{}lll@{}}
      \toprule
      \sc{Size of $\dt$ proj.} & \sc{Params (M)} & \iftoggle{arxiv}{\sc{Perplexity}}{\sc{Ppl}} \\
      \midrule
      -                        & 358.9      & 9.12 \\ %
      $1$                      & 359.1      & 8.97 \\ %
      $2$                      & 359.3      & 8.97 \\ %
      $4$                      & 359.7      & 8.91 \\ %
      $8$                      & 360.5      & 8.83 \\ %
      $16$                     & 362.1      & 8.84 \\ %
      $32$                     & 365.2      & 8.80 \\ %
      $64$                     & 371.5      & 8.71 \\ %
      \bottomrule
    \end{tabular}
    \label{tab:ablations-dt}
  \end{minipage}
  \hfill
  \begin{minipage}{.65\linewidth}
    \centering
    \captionsetup{type=table}
    \caption{
      (\textbf{Ablations: SSM state dimension}.)
      (\emph{Top}) Constant $\B$ and $\C$
      (\emph{Bottom}) Selective $\B$ and $\C$.
      Increasing the SSM state dimension $N$, which can be viewed as an expansion factor on the dimension of the recurrent state, can significantly improve performance for a negligible cost in parameters/FLOPs, but only when $\B$ and $\C$ are also selective.
      Size of $\dt$ projection fixed to $64$.
    }
    \begin{tabular}{@{}lll@{}}
      \toprule
      \sc{State dimension} $N$ & \sc{Params (M)} & \iftoggle{arxiv}{\sc{Perplexity}}{\sc{Ppl}} \\
      \midrule
      $1$                 & 367.1 & 9.88 \\ %
      $2$                 & 367.4 & 9.86 \\ %
      $4$                 & 368.0 & 9.82 \\ %
      $8$                 & 369.1 & 9.82 \\ %
      $16$                & 371.5 & 9.81 \\ %
      \midrule
      $1$                 & 367.1 & 9.73 \\ %
      $2$                 & 367.4 & 9.40 \\ %
      $4$                 & 368.0 & 9.09 \\ %
      $8$                 & 369.1 & 8.84 \\ %
      $16$                & 371.5 & 8.71 \\ %
      \bottomrule
    \end{tabular}
    \label{tab:ablations-N}
  \end{minipage}
\end{figure}

Of particular note is the dramatic improvement of the selective SSM when the state size $N$ is increased, with over a 1.0 perplexity improvement for a cost of only 1\% additional parameters.
This validates our core motivation in \cref{sec:method:motivation,sec:method:scan}.


\textbf{Better Synthetic Language Modeling Translates to Better Natural Language Modeling.}
We validate that when H3 can solve these synthetic tasks, it also improves the modeling capability on natural language (e.g., on the OpenWebText dataset).
As shown in Table~\ref{table:ablations}, \hthree comes within 0.4 perplexity points of Transformers when trained for 50B tokens on OpenWebText, and performs much better than existing SSM variants (S4D, GSS), by $3-3.9$ points.

\textbf{Extension: H3-attention Hybrid Model.}
A simple hybrid \hthree-attention language model surprisingly outperforms Transformers on OpenWebText by 1.0 point.
Our hybrid model simply retains two self-attention layers: one in the second layer, and one in the middle (layer $2 + N / 2$ for an $N$-layer model, $N$ even).
The \hthree-attention hybrid also outperforms the GSS-attention hybrid~\citep{mehta2022long}.


\section{\fastfft: Efficiently Training SSMs\label{sec:efficiency}}

To improve the efficiency of SSMs on modern hardware, we propose \fastfft.
\fastfft fuses the FFT, pointwise multiply, and inverse FFT to reduce memory
reads/writes.
It also uses a block FFT algorithm to make use of specialized matrix multiply units (e.g., tensor cores on A100) for sequence lengths up to 8K.
For sequences longer than 8K, the computation no longer fits in GPU SRAM\footnote{SRAM, or on-chip memory, is much faster than off-chip GPU memory, but usually much smaller, on the order of around 100KB for each streaming processor.}, so we
propose a novel state-passing algorithm that splits the sequence into chunks to
compute the FFT convolution one chunk at a time.
\fastfft can speed up any SSMs (not just \hthree).


\subsection{Fused Block FFTConv}
\label{sec:systolic_fft}

We deploy two techniques to speed up the FFT-based convolution for sequences shorter than 8K: kernel fusion and block FFT.
Kernel fusion addresses IO bottlenecks due to reading and writing of intermediate results, while block FFT allows the FFT-based convolution to utilize specialized matrix multiplication units.
These techniques allow us to speed up FFTConv
by 2$\times$ (\cref{sec:eval_efficiency}) for sequences shorter than 8k.

\textbf{Kernel Fusion.}
Naive implementations of FFTConv using standard libraries such as cuFFT are IO-bound due to repeated reading and writing of intermediate results.
The FFT convolution in an SSM with input $u$ and filter $f$ has the form $iFFT(FFT(u) \odot FFT(f))$ (where $\odot$ denotes pointwise multiplication).
It requires reading and writing intermediate results to GPU memory---which can dominate the runtime.
Following \textsc{FlashAttention}~\citep{dao2022flashattention}, we first fuse the entire FFTConv into a single kernel and compute it in SRAM to avoid this overhead.

\textbf{Block FFT.}
To further speed up the computation of FFT-based convolution, we exploit specialized matrix multiplication hardware on modern GPUs
(e.g., Tensor Cores on Nvidia GPUs perform fast $16 \times 16$ matrix multiplication).
We appeal to classical results that show that the FFT can be written as a series of block-diagonal matrix
multiplications interleaved with permutation.
We note that such algorithms are not new, but our setting (fused FFTConv on GPU) introduces new bottlenecks---by removing the IO bottlenecks, compute becomes the bottleneck (note that a single FFT on GPU is usually IO bound).

Suppose that we want to perform an $N$-point FFT, which is equivalent to
multiply by the DFT matrix $\vF_N$.
Suppose that $N = N_1 N_2$ for some integers $N_1, N_2$.
By the Cooley-Tukey decomposition of the DFT~\citep{cooley1965an,
  bailey1990ffts} (also known as the four-step FFT algorithm),
we can write $\vF_N = \vP (\vI_{N_2} \otimes \vF_{N_1}) \vP^\top \vD (\vI_{N_1} \otimes \vF_{N_2}) \vP$,
where $\vP$ denotes a fixed permutation that reshapes the input as a $N_1 \times N_2$
array and then transpose it, $\otimes$ denotes Kroneker product, $\vD$ is a $N \times N$
diagonal matrix (called the twiddle factors)~\citep{dao2022monarch}, and $\vI_{N_i}$ and $\vF_{N_i}$ are the identity and DFT matrix of size $N_i \times N_i$.
As $\vI_{N_2} \otimes \vF_{N_1}$ and $\vI_{N_1} \otimes \vF_{N_2}$ are just block-diagonal matrices,
we can make use of specialized matmul units to perform these multiplications.
Similarly, if $N = N_1 N_2 N_3$ then we can decompose the $N$-point FFT into a
series of (block) FFT of size $N_1$, $N_2$, and $N_3$, interleaved by
permutation.

The block FFT algorithm incurs $O(N r \log N / \log r)$ FLOPs for a sequence length $N$, if $N$ can be written as $r^p$ for two integers $r, p$.
This incurs more FLOPs than standard FFT $(O(N\log N))$, but can run faster when
we using specialized matrix multiplication hardware.

\subsection{State-Passing}
However, the fused kernel cannot run if the sequence is too long to fit into GPU SRAM (longer than 8K on A100).
We show how to exploit the particular form of the FFT in SSM to speed it up for long
sequences.

The recurrent nature of SSMs allows us to split the FFTConv of a length-$N$ sequence into chunks of size $N'$ each ($N'$ is the longest FFT we can fit into SRAM), assuming $N$ is a multiple of $N'$).
We use FFTConv to compute each chunk and use a recurrence to connect the chunks.
In particular, we split the inputs $u$ into $C = N/N'$ chunks $u^{(c)} \in \mathbb{R}^{N'}$ for $c=1, \dots, C$.
Similarly, split the states $x$ into $x^{(c)} \in \mathbb{R}^{N' \times m}$ and the output $y$ into $y^{(c)} \in \mathbb{R}^{N'}$ for $i = 1, \dots, C$.
We will only need the end-state $x_{N'}^{(c)}$ of each chunk $c$.

Let $f = [\vC \vB, \vC\vA\vB, \vC\vA^2\vB, \dots, \vC\vA^{N'-1}\vB]$ be the SSM filter.
Recall from \cref{sec:background} that for each chunk $c$, $y_i^{(c)} = \vC\vA^i\vB x_{N'}^{(c-1)} + (f \ast u^{(c)})_i + \vD u_i^{(c)}$, since $x_{N'}^{(c-1)}$, the end-state of the previous chunk $(c-1)$ is the initial condition for the current chunk $c$.
In vector notation, $y^{(c)} = \vM_{xy} x_{N'}^{(c-1)} + f \ast u^{(c)} + \vD u^{(c)}$ for some matrix $\vM_{xy} \in \mathbb{R}^{N' \times m}$.
Additionally we need to update the end-state of each chunk with $x_{N'}^{c} = \vA^{N'} x_{N'}^{(c-1)} + \vM_{ux} u^{(c)}$ for some matrix $\vM_{ux}^{m \times N'}$ (derivation in Appendix~\ref{sec:state-passing-matrices}).
In essence, we can compute the output for each chunk with FFT-based convolution as long as we remember the end-state of the previous chunk, and the end-state of each chunk can be updated recurrently.
This yields a state-passing algorithm for long sequences, where we only compute FFT of length $N'$, and update some hidden state each iteration.

Let \textsc{BlockFFTConv} refer to our fused block FFTConv kernel.
Then, the state-passing algorithm for 1D input is given by Algorithm~\ref{alg:statepassing}.
For inputs of dimension $d$ where we stack $d$ SSMs, we simply batch~\cref{alg:statepassing} along the $d$-dimension.
\begin{algorithm}[h]
  \algsetup{linenosize=\tiny}
  \caption{\label{alg:statepassing} State Passing Algorithm}
  \small
  \begin{algorithmic}[1]
    \REQUIRE Input $u \in \mathbb{R}^{N}$, SSM parameterized by matrices $\vA \in \mathbb{R}^{m \times m}$, $\vB \in \mathbb{R}^{m \times 1}$, $\vC \in \mathbb{R}^{1 \times m}$, $\vD \in \mathbb{R}^{1 \times 1}$, chunk size $N'$ where $N$ is a multiple of $N'$.
    \STATE Precompute $\vA^{N'} \in \mathbb{R}^{m \times m}$, $\vM_{ux} = [\vA^{N'-1}\vB, \dots, \vB] \in \mathbb{R}^{m \times N'}$, $\vM_{xy} = [\vC, \vC\vA, \dots, \vC\vA^{N'-1}] \in \mathbb{R}^{N' \times m}$.
    \STATE Split the inputs $u_{1:N}$ into $C = N/N'$ chunks $u_{1:N'}^{(c)}$ for $c=1, \dots, C$.
    \STATE Let the initial state be $x_{N'}^{(0)} = 0 \in \mathbb{R}^m$.
    \FOR{$1 \leq c \leq C$}
      \STATE Compute $y^{(c)} = \vM_{xy} x_{N'}^{(c-1)} + $ \textsc{BlockFFTConv}($f$, $u_j$) $+ \vD u^{(c)} \in \mathbb{R}^{N'}$.
      \STATE Update state: $x_{N'}^{(c)} = \vA^{N'} x_{N'}^{(c-1)} + \vM_{ux} u^{(c)}$.
    \ENDFOR
    \STATE Return $y = [y^{(1)} \dots y^{(C)}]$.
  \end{algorithmic}
\end{algorithm}

We prove that~\cref{alg:statepassing} yields the same output as if one has computed the SSM using a large FFT of size $N$ (proof in~\cref{sec:app_statepassing_correctness_proof}):
\begin{proposition}\label{thm:statepassing_correctness}
  For input $u \in \mathbb{R}^N$ and matrices $\vA, \vB, \vC, \vD$, the output $y \in \mathbb{R}^N$ returned by~\cref{alg:statepassing} is the same as the output defined by the SSM parameterized by $\vA, \vB, \vC, \vD$.
\end{proposition}

%%% Local Variables:
%%% mode: latex
%%% TeX-master: "../main"
%%% End:
