\section{\hthree Evaluation}
\label{sec:evaluation}

\begin{table}[t]
    % \captionsetup{font=small}
    \small
    \centering
    % \vspace{-1em}
    \caption{\label{table:gpt} Perplexity (lower is better) of models on the Pile, OpenWebText and
      WikiText-103. GPT-Neo and hybrid \hthree are trained on the Pile, while GPT2 is
      trained on WebText. All models use the same GPT2 tokenizer. We report the
      perplexity of GPT-2 models on the Pile ($^*$) for context, though the performance is not directly comparable since they were trained on different data.}
  %   \vspace{1em}
    {
      \begin{tabular}{@{}|c|c|cc|@{}}
      %   \specialrule{.15em}{.05em}{.05em}
      \hline
        Model & Pile & OpenWebText & WikiText103 \\ % & Training time \\
      %   \specialrule{.15em}{.05em}{.05em}
        \hline
        % GPT-2 small (125M) & 9.8 & 18.2 & -- \\ % & 4.7 days \\
        % H3 (126M) & 18.7 & 4.5 days \\ \hline
        % OPT-125M & -- & -- & -- \\
        GPT-2 small (125M) & 19.0* & 22.6 & 29.9 \\
        GPT-Neo-125M & 9.4 & 22.6 & 26.3 \\
        % GPT3-125M (our replication) & - & - & 24.6 \\
        \textbf{Hybrid H3-125M} & \textbf{8.8} & \textbf{20.9} & \textbf{23.7} \\ \hline %2.7 days \\ \hline
        GPT-2 medium (355M) & 13.9* & 17.0 & 21.8 \\ % & 11.5 days \\
        % H3 (361M) &  &  \\ \hline
        % OPT-350M & -- & -- & -- \\
        % GPT3-355M (our replication) & - & - & 16.6 \\
        \textbf{Hybrid H3-355M} & \textbf{7.1} & \textbf{15.9} & \textbf{16.9} \\ \hline
        GPT-2 XL (1.5B) & 12.4* & 12.9 & 17.0 \\
        % OPT-1.3B & -- & -- & -- \\
        GPT-Neo-1.3B & 6.2 & 13.1 & 13.3 \\
        \textbf{Hybrid H3-1.3B} & \textbf{6.0} & \textbf{12.4} & \textbf{12.5} \\
        \hline
        GPT-Neo-2.7B & 5.7 & 11.7 & 11.5 \\
        \textbf{Hybrid H3-2.7B} & \textbf{5.4} & \textbf{11.0} & \textbf{10.6} \\
        \hline
      \end{tabular}
    }
    % \vspace{-1.5em}
  \end{table}

\begin{table}[t]
    % \captionsetup{font=small}
    \scriptsize
    \centering
    % \vspace{-1em}
    \caption{\label{table:superglue_zeroshot_logit} Zero-shot acc.\ on SuperGLUE with logit scoring. Best results in bold, second best underline. }
    %   \vspace{1em}
    {
        \begin{tabular}{@{}|c|cccccccc|c|@{}}
            \hline
        %   \specialrule{.15em}{.05em}{.05em}
        Model & WSC & WIC & RTE & CB & MultiRC & ReCoRD & BoolQ & COPA & Average \\ % & Training time \\
        %   \specialrule{.15em}{.05em}{.05em}
        \hline
        % GPT-2 small (125M) & -- & -- & -- & -- & -- & -- & -- & -- & -- \\ % & 4.7 days \\
        % H3 (126M) & 18.7 & 4.5 days \\ \hline
        OPT-125M & \textbf{39.4} & \underline{52.0} & 48.7 & 37.4 & \underline{58.9} & \underline{44.9} & \underline{59.6} & \underline{60.0} & 50.1 \\
        GPT-Neo-125M & \underline{36.5} & \textbf{53.6} & \underline{53.1} & \underline{41.1} & \textbf{59.9} & 39.6 & \textbf{62.2} & \underline{60.0} & \underline{50.8} \\
        % \textbf{\hthree-125M} & 0.0 & 0.0 & 47.3 & 8.9 & 0.0 & -- & 37.8 & 51.0 & 20.7 \\ %2.7 days \\ \hline
        \textbf{Hybrid \hthree-125M} & \textbf{39.4} & 51.4 & \textbf{59.2} & \textbf{48.2} & 51.4 & \textbf{55.0} & \underline{59.6} & \textbf{67.0} & \textbf{53.9} \\ \hline %2.7 days \\ \hline
        GPT-2 medium (355M) & \underline{50.0} & \textbf{52.0} & 51.3 & 28.6 & \textbf{59.5} & \underline{53.3} & \underline{61.0} & \underline{65.0} & 52.6 \\
        % H3 (361M) &  &  \\ \hline
        OPT-350M & \textbf{53.5} & 50.8 & \underline{53.4} & \underline{35.7} & \underline{58.9} & 51.4 & 60.9 & 60.0 & \underline{53.1} \\
        \textbf{Hybrid \hthree-355M} & 37.5 & \underline{51.7} & \textbf{55.2} & \textbf{41.1} & \textbf{59.5} & \textbf{62.3} & \textbf{61.5} & \textbf{69.0} & \textbf{54.7} \\ \hline
        % GPT-2 (1.5B) & 14.4 & \textbf{36.5} & \underline{49.1} & \underline{8.9} & 0.5 & \underline{44.0} & \underline{46.6} & 59.0 & 32.4 \\
        OPT-1.3B & 36.5 & 49.5 & \textbf{53.4} & \textbf{39.3} & \textbf{58.3} & \underline{61.8} & 55.0 & \underline{69.0} & \underline{52.9} \\
        GPT-Neo-1.3B & \underline{41.3} & \underline{50.0} & \underline{52.3} & \underline{33.9} & 57.9 & 55.5 & \underline{59.9} & 66.0 & 52.1 \\
        \textbf{Hybrid \hthree-1.3B} & \textbf{52.9} & \textbf{50.3} & \textbf{53.4} & \underline{33.9} & \underline{58.2} & \textbf{67.8} & \textbf{61.7} & \textbf{74.0} & \textbf{56.5} \\ \hline
        OPT-2.7B & \textbf{51.0} & \underline{50.8} & 50.5 & \underline{41.1} & 57.4 & \underline{65.9} & 60.9 & 66.0 & \underline{55.5} \\
        GPT-Neo-2.7B & \underline{37.5} & 50.0 & \underline{52.3} & \textbf{50.0} & \textbf{59.1} & 60.0 & \textbf{61.1} & \underline{67.0} & 54.6 \\
        \textbf{Hybrid \hthree-2.7B} & 36.5 & \textbf{51.3} & \textbf{57.0} & 37.5 & \underline{58.7} & \textbf{71.3} & \textbf{61.1} & \textbf{81.0} & \textbf{56.8} \\ \hline
        \end{tabular}
    }
    % \vspace{-1.5em}
\end{table}
\begin{table}[t]
    % \captionsetup{font=small}
    \scriptsize
    \centering
    % \vspace{-1em}
    \caption{\label{table:superglue_fewshot_logit} 3-shot acc.\ on SuperGLUE with logit scoring. Best results in bold, second best underline. }
    %   \vspace{1em}
    {
        \begin{tabular}{@{}|c|cccccccc|c|@{}}
            \hline
        %   \specialrule{.15em}{.05em}{.05em}
        Model & WSC & WIC & RTE & CB & MultiRC & ReCoRD & BoolQ & COPA & Average \\ % & Training time \\
        %   \specialrule{.15em}{.05em}{.05em}
        \hline
        % GPT-2 small (125M) & -- & -- & -- & -- & -- & -- & -- & -- & -- \\ % & 4.7 days \\
        % H3 (126M) & 18.7 & 4.5 days \\ \hline
        OPT-125M & 36.5 & \textbf{50.2} & 47.3 & \underline{44.6} & \textbf{57.9} & \underline{44.9} & 41.9 & 60.0 & \underline{47.9} \\
        GPT-Neo-125M & \underline{38.5} & \underline{50.0} & \underline{53.1} & 17.9 & \underline{56.3} & 39.6 & \textbf{62.1} & \underline{60.0} & 47.2 \\
        \textbf{Hybrid \hthree-125M} & \textbf{43.3} & 49.1 & \textbf{58.1} & \textbf{51.8} & 48.9 & \textbf{55.0} & \underline{56.1} & \textbf{67.0} & \textbf{53.7} \\ \hline %2.7 days \\ \hline
        GPT-2 medium (355M) & 36.5 & \textbf{50.5} & \underline{48.0} & 8.9 & 43.5 & \underline{53.3} & 58.8 & \underline{65.0} & 45.6 \\
        % H3 (361M) &  &  \\ \hline
        OPT-350M & \underline{37.5} & \underline{50.0} & 45.8 & \textbf{44.6} & \underline{49.8} & 51.4 & \textbf{61.7} & 60.0 & \underline{50.1} \\
        \textbf{Hybrid \hthree-355M} & \textbf{42.3} & 47.5 & \textbf{50.5} & \underline{28.6} & \textbf{59.7} & \textbf{62.3} & \underline{60.5} & \textbf{69.0} & \textbf{52.6} \\ \hline
        % GPT-2 (1.5B) & \underline{36.5} & 36.5 & \textbf{56.7} & 41.1 & 6.1 & 43.8 & \underline{53.7} & 61.0 & 41.9 \\
        OPT-1.3B & \textbf{44.2} & \textbf{51.1} & \underline{53.4} & 16.1 & \textbf{59.9} & \underline{62.1} & 38.3 & \underline{70.0} & 49.4 \\
        GPT-Neo-1.3B & 35.6 & \underline{50.6} & 47.3 & \textbf{32.1} & \textbf{59.9} & 55.7 & \textbf{61.2} & 67.0 & \underline{51.2} \\
        \textbf{Hybrid \hthree-1.3B} & \underline{36.5} & 49.2 & \textbf{55.2} & \underline{23.2} & \underline{59.3} & \textbf{67.6} & \underline{56.9} & \textbf{76.0} & \textbf{53.0} \\ \hline
        OPT-2.7B & \underline{44.2} & \underline{50.5} & \textbf{53.4} & 17.9 & \underline{59.2} & \underline{66.0} & \textbf{62.0} & \underline{71.0} & \underline{53.0} \\
        GPT-Neo-2.7B & \textbf{49.0} & \textbf{51.9} & \underline{51.6} & \underline{21.4} & 57.0 & 60.0 & 56.0 & 68.0 & 51.9 \\
        \textbf{Hybrid \hthree-2.7B} & 36.5 & 45.6 & 47.3 & \textbf{46.4} & \textbf{59.4} & \textbf{71.1} & \underline{60.6} & \textbf{77.0} & \textbf{55.5} \\ \hline
        \end{tabular}
    }
    % \vspace{-1.5em}
\end{table}

To understand how well capturing the synthetics in Section~\ref{sec:synthetics} translates to language modeling, we train two hybrid hybrid \hthree-attention language models at sizes 125M, 355M, 1.3B, and 2.7B, and we evaluate their performance against Transformers.
The hybrid models match or exceed the quality of Transformers in perplexity and zero/few-shot learning.
We also validate that \hthree models retain strong performance on non-text sequence modeling.
Appendix~\ref{sec:app_additional_experiments} contains additional experiments on more datasets, length extrapolation, and scaling with data.

\subsection{Language Modeling}
\label{subsec:language_modeling}

We compare hybrid \hthree-attention language models against Transformer-based language models.
We evaluate language modeling performance using perplexity, zero-shot learning, and few-shot learning performance.
Hybrid \hthree models outperform Transformers, which suggests that closing the gap between SSMs and attention on the synthetic languages translates to real language modeling capabilities.
We also report the generation speed of hybrid \hthree models compared to Transformers; since SSMs are recurrent models, they can generate tokens \num{2.4$\times$} faster than Transformers.
Appendix~\ref{sec:app_additional_experiments} shows performance of pure \hthree language models on these same evaluation metrics.

\paragraph{Setup}
We train hybrid models at sizes 125M, 355M, 1.3B, and 2.7B on the Pile~\citep{gao2020pile} for 400B tokens.
We compare against checkpoints of equivalent sizes from
Open-AI~\citep{radford2019language} and GPT-Neo\footnote{There
  is no pretrained GPT-Neo at the 350M size.}~\citep{gpt-neo},
from HuggingFace~\citep{wolf-etal-2020-transformers}.


\paragraph{Perplexity}
Table \ref{table:gpt} shows perplexity on the Pile~\citep{gao2020pile}, OpenWebText~\citep{Gokaslan2019OpenWeb}, and WikiText-103~\citep{merity2016pointer}. 
On the Pile, our 125M hybrid model outperforms GPT-Neo, which was also trained on the Pile.
Our hybrid models also outperform GPT-Neo models and GPT-2 models on zero-shot transfer to OpenWebText and WikiText103.
We report the PPL of GPT-2 models for context, though the performance is not directly comparable since they were trained on different data.

\paragraph{Zero- and Few-shot Performance}
We compare the zero- and few-shot performance of hybrid \hthree language models against OPT~\citep{zhang2022opt}, GPT-Neo, and GPT-2 models, where public checkpoints are available.
We report performance with rank classification on the logits of the possible choices (see Appendix~\ref{sec:app_generation} for raw generation).
Table~\ref{table:superglue_zeroshot_logit} reports zero-shot performance on the SuperGLUE benchmark, and Table~\ref{table:superglue_fewshot_logit} reports the 3-shot performance.
Following the perplexity results, the hybrid language models outperform or match the best the Transformer baseline on more than half the tasks.

\begin{table}[t]
\centering
% \begin{table}[h]
    % \captionsetup{font=small}
    \small
    \centering
    % \vspace{-1em}
    \caption{\label{table:training_time} Inference throughput on A100 80GB, 1.3B models.
    Batch size 64, prompt length 512, 1024, or 1536, and generating 128 tokens
    per sequence in the batch (i.e., 64 $\times$ 128 tokens in a batch). Hybrid
    \hthree is up to 2.4$\times$ faster than a Transformer of similar size in inference. The
    difference is larger for longer sequences.}
    %   \vspace{1em}
    {
        \begin{tabular}{@{}|c|c|c|c|@{}}
            \hline
        %   \specialrule{.15em}{.05em}{.05em}
        Tokens/s & Prompt length 512 & Prompt length 1024 & Prompt length 1536 \\ % & Training time \\
        %   \specialrule{.15em}{.05em}{.05em}
        \hline
        Transformer-1.3B & 1340 & 770 & 520 \\
        % GPT-2 Small (FlashAttention~\citep{dao2022flashattention}) & -- & -- \\ 
        % GPT-2 Small (FasterTransformer~\citep{}) & N/A & -- \\ \hline
        Hybrid \hthree-1.3B & 1980 & 1580 & 1240 \\ \hline
        \end{tabular}
    }
    % \vspace{-1.5em}
% \end{table}
\end{table}
\paragraph{Language Modeling Inference}
Finally, since SSMs are recurrent models, they admit faster text generation than Transformers.
Table~\ref{table:training_time} shows inference throughput of a 1.3B-parameter hybrid model compared to a Transformer.
The hybrid model has up to \num{2.4$\times$} higher throughput.



%%% Local Variables:
%%% mode: latex
%%% TeX-master: "../main"
%%% End:
