%!TEX root = ../main.tex

State space models (SSMs) have demonstrated state-of-the-art sequence modeling performance in some modalities, but underperform attention in language modeling.
Moreover, despite scaling nearly linearly in sequence length instead of quadratically, SSMs are still slower than Transformers due to poor hardware utilization.
In this paper, we make progress on understanding the expressivity gap between SSMs and attention in language modeling, and on reducing the hardware barrier between SSMs and attention.
First, we use synthetic language modeling tasks to understand the gap between SSMs and attention.
We find that existing SSMs struggle with two capabilities: recalling earlier tokens in the sequence and comparing tokens across the sequence.
To understand the impact on language modeling, we propose a new SSM layer, \hthree, that is explicitly designed for these abilities.
\hthree matches attention on the synthetic languages and comes within \num{0.4} PPL of Transformers on OpenWebText.
Furthermore, a hybrid 125M-parameter \hthree-attention model that retains two attention layers
surprisingly outperforms Transformers on OpenWebText by \num{1.0} PPL.
Next, to improve the efficiency of training SSMs on modern hardware,
we propose \fastfft.
\fastfft uses a fused block FFT algorithm to improve efficiency on sequences up to 8K, and introduces a novel state passing algorithm that exploits the recurrent properties of SSMs to scale to longer sequences.
\fastfft yields 2$\times$ speedup on the long-range arena benchmark and allows hybrid language models to generate text \num{2.4$\times$} faster than Transformers.
Using \fastfft, we scale hybrid \hthree-attention language models up to 2.7B parameters on the Pile and find promising initial results, achieving lower perplexity than Transformers and outperforming Transformers in zero- and few-shot learning on a majority of tasks in the SuperGLUE benchmark.

%%% Local Variables:
%%% mode: latex
%%% TeX-master: "../main"
%%% End: