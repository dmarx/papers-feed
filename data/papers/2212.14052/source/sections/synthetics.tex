\section{Synthetic Language Modeling Tasks}
\begin{table}[h]
    % \captionsetup{font=small}
    \small
    \centering
    \caption{\label{table:synthetic_tasks} Synthetic language modeling tasks.}
    %   \vspace{1em}
    {
        \begin{tabular}{@{}|l|l|l|c|c|@{}}
        %   \specialrule{.15em}{.05em}{.05em}
        \hline
        Task & Input & Output & Sequence Length & Vocab Size  \\ % & Training time \\
        %   \specialrule{.15em}{.05em}{.05em}
        \hline
        Induction Head & \textit{a b c d e $\vdash$ f g h i $\dots$ x y z $\vdash$} & \textit{f} & 30 & 20 \\
        Associative Recall & \textit{a 2 c 4 b 3 d 1 a} & \textit{2} & 20 & 10 \\ \hline
        % Partial Copy & \textit{a b c @ d e f * g h i} & \textit{@ d e f * . . .} & 20 & 10 \\ \hline
        \end{tabular}
    }
\end{table}
We describe two synthetic tasks we use to understand the gap between SSMs and attention in language modeling, summarized in Table~\ref{table:synthetic_tasks}. The \textbf{Induction Head} task, inspired by the analysis of~\citet{olsson2022context}, tests how well a model can recall content after a special token.
At the end of the sequence, the model recall the token that appeared immediately after a special token earlier in the sequence. \textbf{Associative Recall}~\citep{ba2016using} tests how well a model can associate specific values to keys. The model is shown a series of key-value pairs, and must recall an associated value at the end.

\begin{table}[h]
    % \captionsetup{font=small}
    \small
    \centering
    % \vspace{-1em}
    \caption{\label{table:synthetics} Evaluation of 2-layer models on synthetic language tasks.}
    %   \vspace{1em}
    {
        \begin{tabular}{@{}|c|c|ccc|c|@{}}
        %   \specialrule{.15em}{.05em}{.05em}
        \hline
        Task & Random & S4D & Gated State Spaces & H3 & Attention  \\ % & Training time \\
        %   \specialrule{.15em}{.05em}{.05em}
        \hline
        Induction Head & 5.0 & 35.6 & 6.8 & \textbf{100.0} & \textbf{100.0} \\
        Associative Recall & 25.0 & 86.0 & 78.0 & 99.8 & \textbf{100.0}  \\ \hline
        % Gated State Spaces & 78.0 & 6.8 & 88.2 \\
        % LSTM & -- & -- & -- \\
        % H3 & 99.8 & \textbf{100.0} & \textbf{99.6}  \\ \hline
        % Attention & \textbf{100.0} & \textbf{100.0} & 71.0 \\ \hline
        \end{tabular}
    }
    % \vspace{-1.5em}
\end{table}

% \begin{table}[h]
%     % \captionsetup{font=small}
%     \small
%     \centering
%     % \vspace{-1em}
%     \caption{\label{table:synthetics} Evaluation of 2-layer GPT models on in-context learning synthetic tasks.}
%     %   \vspace{1em}
%     {
%         \begin{tabular}{@{}|c|ccc|@{}}
%         %   \specialrule{.15em}{.05em}{.05em}
%         \hline
%         Model & Associative Recall & Induction Head & Partial Copy  \\ % & Training time \\
%         %   \specialrule{.15em}{.05em}{.05em}
%         \hline
%         Random & 50.0 & 5.0 & 5.0 \\
%         S4D & 86.0 & 35.6 & 98.5  \\
%         Gated State Spaces & 78.0 & 6.8 & 88.2 \\
%         % LSTM & -- & -- & -- \\
%         H3 & 99.8 & \textbf{100.0} & \textbf{99.6}  \\ \hline
%         Attention & \textbf{100.0} & \textbf{100.0} & 71.0 \\ \hline
%         \end{tabular}
%     }
%     % \vspace{-1.5em}
% \end{table}
Table~\ref{table:synthetics} shows the performance of two-layer GPT-style models with attention or continuous SSMs on these synthetic tasks.
These failures suggest two missing capabilities: to remember tokens that appear after a particular event (e.g., the special token in the induction head task), and to compare tokens across the sequence (e.g., comparing keys to decide which value to recall).
In Section~\ref{sec:method}, we design our new layer \hthree to capture these capabilities.