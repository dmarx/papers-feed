
\documentclass{article}

% Recommended, but optional, packages for figures and better typesetting:
\usepackage{microtype}
\usepackage{graphicx}
% \usepackage{subfigure}
\usepackage{subcaption} 
\usepackage{booktabs} % for professional tables

\usepackage{url}
\usepackage{enumitem} 
\usepackage{tabularx} 
\usepackage{centernot}
\usepackage[parfill]{parskip}
\usepackage{quiver}
\usepackage{bbm} 
\usepackage{graphicx} 
\usepackage{scalerel}
\usepackage{nicefrac} 
\usepackage{bbm} 

% hyperref makes hyperlinks in the resulting PDF.
% If your build breaks (sometimes temporarily if a hyperlink spans a page)
% please comment out the following usepackage line and replace
% \usepackage{icml2024} with \usepackage[nohyperref]{icml2024} above.
\usepackage{hyperref}


% Attempt to make hyperref and algorithmic work together better:
\newcommand{\theHalgorithm}{\arabic{algorithm}}

% Use the following line for the initial blind version submitted for review:
\usepackage{icml2024}

% If accepted, instead use the following line for the camera-ready submission:
% \usepackage[accepted]{icml2024}

% For theorems and such
\usepackage{amsmath}
\usepackage{amssymb}
\usepackage{mathtools}
\usepackage{amsthm}

% if you use cleveref..
\usepackage[capitalize,noabbrev]{cleveref}

%%%%%%%%%%%%%%%%%%%%%%%%%%%%%%%%
% THEOREMS
%%%%%%%%%%%%%%%%%%%%%%%%%%%%%%%%
\theoremstyle{plain}
\newtheorem{theorem}{Theorem}[section]
\newtheorem{proposition}[theorem]{Proposition}
\newtheorem{lemma}[theorem]{Lemma}
\newtheorem{corollary}[theorem]{Corollary}
\theoremstyle{definition}
\newtheorem{definition}[theorem]{Definition}
\newtheorem{assumption}[theorem]{Assumption}
\theoremstyle{remark}
\newtheorem{remark}[theorem]{Remark}
\theoremstyle{remark} 
\newtheorem{example}[theorem]{Example} 

% Todonotes is useful during development; simply uncomment the next line
%    and comment out the line below the next line to turn off comments
%\usepackage[disable,textsize=tiny]{todonotes}
\usepackage[textsize=tiny]{todonotes}


% The \icmltitle you define below is probably too long as a header.
% Therefore, a short form for the running title is supplied here:
\icmltitlerunning{Submission and Formatting Instructions for ICML 2024}

\begin{document}

\twocolumn[
\icmltitle{Analyzing GFlowNets: Limitations, Countermeasures, and Assessment}

% It is OKAY to include author information, even for blind
% submissions: the style file will automatically remove it for you
% unless you've provided the [accepted] option to the icml2024
% package.

% List of affiliations: The first argument should be a (short)
% identifier you will use later to specify author affiliations
% Academic affiliations should list Department, University, City, Region, Country
% Industry affiliations should list Company, City, Region, Country

% You can specify symbols, otherwise they are numbered in order.
% Ideally, you should not use this facility. Affiliations will be numbered
% in order of appearance and this is the preferred way.
\icmlsetsymbol{equal}{*}

\begin{icmlauthorlist}
\icmlauthor{Firstname1 Lastname1}{equal,yyy}
\icmlauthor{Firstname2 Lastname2}{equal,yyy,comp}
\icmlauthor{Firstname3 Lastname3}{comp}
\icmlauthor{Firstname4 Lastname4}{sch}
\icmlauthor{Firstname5 Lastname5}{yyy}
\icmlauthor{Firstname6 Lastname6}{sch,yyy,comp}
\icmlauthor{Firstname7 Lastname7}{comp}
%\icmlauthor{}{sch}
\icmlauthor{Firstname8 Lastname8}{sch}
\icmlauthor{Firstname8 Lastname8}{yyy,comp}
%\icmlauthor{}{sch}
%\icmlauthor{}{sch}
\end{icmlauthorlist}

\icmlaffiliation{yyy}{Department of XXX, University of YYY, Location, Country}
\icmlaffiliation{comp}{Company Name, Location, Country}
\icmlaffiliation{sch}{School of ZZZ, Institute of WWW, Location, Country}

\icmlcorrespondingauthor{Firstname1 Lastname1}{first1.last1@xxx.edu}
\icmlcorrespondingauthor{Firstname2 Lastname2}{first2.last2@www.uk}

% You may provide any keywords that you
% find helpful for describing your paper; these are used to populate
% the "keywords" metadata in the PDF but will not be shown in the document
\icmlkeywords{Machine Learning, ICML}

\vskip 0.3in
]

% this must go after the closing bracket ] following \twocolumn[ ...

% This command actually creates the footnote in the first column
% listing the affiliations and the copyright notice.
% The command takes one argument, which is text to display at the start of the footnote.
% The \icmlEqualContribution command is standard text for equal contribution.
% Remove it (just {}) if you do not need this facility.

%\printAffiliationsAndNotice{}  % leave blank if no need to mention equal contribution
\printAffiliationsAndNotice{\icmlEqualContribution} % otherwise use the standard text.

\begin{abstract}
Generative Flow Networks (GFlowNets) are powerful samplers for distributions over compositional objects (e.g., graphs). Despite their increasing popularity, GFlowNets are still in their infancy, and many fundamental questions about them remain unexplored. In this work, we analyze GFlowNets from three fundamental perspectives: expressiveness, stability, and model evaluation.  


Regarding expressiveness, we consider GFlowNets for graph generation. We prove that, given a suitable state graph, GFlowNets can accurately learn any distribution supported over trees. Strikingly, however, we show simple combinations of state graphs and reward functions that cause GFlowNets to fail --- i.e., for which balance is unattainable. We propose leveraging embeddings of children's states to circumvent this limitation and thus increase the expressiveness of GFlowNets, provably.

For stability, we analyze tree-structured state graphs and discuss how fluctuations in balance conditions impact the accuracy of GFlowNets, i.e., the gap between the sampling and target distributions. Our theoretical results suggest that i) flow imbalances near the initial state have a higher impact than those near terminal ones; ii) imbalance in a single edge may be catastrophically propagated within the state graph, severely damaging the approximation to the target distribution.

Finally, we revisit evaluation procedures for GFlowNets and highlight the pitfalls of broadly used procedures. We provide guidelines for a more reliable evaluation of GFlowNets.


\end{abstract}

\section{Introduction} 

We contribute by 

\begin{enumerate} 
    \item providing flow-based bounds on the total variation distance between the target distribution and the distribution learned by a GFlowNet (\autoref{sec:flows}); % . 
    \item establishing the distributional limits of GFlowNets with GNN-based forward policies and showing that using a non-inductively biased neural network significantly hinders training convergence due to an exponential  increase in the state graph's size (\autoref{sec:rep}); 
    \item designing efficient and robust diagnostics for assessing the training convergence of GFlowNets (\autoref{sec:cov}); 
    \item rigorously comparing, through the proposed diagnostics, the rate of convergence induced by each of the previously proposed training objectives for GFlowNets in a comprehensive set of experiments (\autoref{sec:experiments}).        
\end{enumerate} 

\newpage 
% a
% \newpage 


\section{Background and related works} 

\paragraph{Notations and definitions.} Let $\mathcal{X}$ be a finite set and $\tilde{\pi}$ be a (possibly) unnormalized distribution over $\mathcal{X}$ with \textit{partition function} $Z = \sum_{x \in \mathcal{X}} \tilde{\pi}(x)$. Moreover, let $\mathcal{S} \supseteq \mathcal{X}$ be an extension of $\mathcal{X}$ and $\mathcal{G} = (\mathcal{S}, \mathbf{A})$ be a DAG over $\mathcal{S}$ with adjacency matrix $\mathbf{A} \in \{1, 0\}^{|\mathcal{S}| \times |\mathcal{S}|}$. We call the elements of $\mathcal{S}$ \textit{states} and assume the existence of a detached state, $s_{o}$, named the \textit{initial state}, which has no incoming edges and from which every other $v \in \mathcal{S}$ is reachable; $\mathcal{G}$ is then called the \textit{state graph} and the elements $x \in \mathcal{X}$, which assumedly have no outgoing edges, are called \textit{terminal states}. In this setting, we define a \textit{flow} over $\mathcal{G}$ as a function $F \colon \mathcal{S} \times \mathcal{S} \rightarrow \mathbb{R}_{+}$ such that $F(u, v) > 0$ if $\mathbf{A}_{uv} = 1$ and $F(u, v) = 0$ otherwise. Similarly, we let the \textit{flow through a node} $v \in \mathcal{S} \setminus \mathcal{X}$, denoted by $F_{N}$, be the sum of flows within all $v$'s outgoing edges, i.e., $F_{N}(v) = \sum_{u \in V} \mathbf{A}_{vu} F(v, u)$; for $x \in \mathcal{X}$, we let $F_{N}(x) = \tilde{\pi}(x)$. In this context, a \textit{forward policy} $p_{F}$ over $\mathcal{G}$ is a flow such that the projection $u \mapsto p_{F}(v, u)$ is a probability distribution over $\mathcal{S}$ for each $v \in \mathcal{S}$. On the other hand, a \textit{backward policy} $p_{B}$ over $\mathcal{G}$ is a forward policy over $\mathcal{G}^{\intercal} = (\mathcal{S}, \mathbf{A}^{\intercal})$. In both cases, note that a policy defines a conditional distribution over the trajectories $\tau = (s_{i})_{i=0}^{M}$ in $\mathcal{G}$ by $p_{F}(\tau | s_{o}) = \prod_{i=1}^{M} p_{F}(s_{i - 1}, s_{i})$ and that it is related to the edge-level flow $F$ by $F(u, v) = p_{F}(u, v) F_{N}(u)$.  

\paragraph{GFlowNets.} A \textit{GFlowNet} is represented as a tuple $(\log p_{F}(\cdot ; \theta_{F}), \log p_{B}(\cdot ; \theta_{B}), \log F_{N}(\cdot ; \theta_{N}), \theta_{\log Z})$ consisting of parametric models for the forward and backward policies and sometimes for flow $\log F_{N}$ and the partition function's logarithm $\theta_{\log Z}$ (when such quantities are not estimated, we simply let $\theta_{N} = \theta_{\log Z} = \emptyset$). Importantly, the unknown functions $\log p_{F}$, $\log p_{B}$ and $\log F_{N}$ are parameterized by neural networks and are trained to ensure that the marginal distribution of $p_{F}$ over $\mathcal{X}$, 
\begin{equation} \label{eq:marg} 
    p_{\intercal}(x ; \theta_{F}) \coloneqq \sum_{\tau \rightsquigarrow x} p_{F}(\tau | s_{o} ; \theta_{F}),  
\end{equation}
matches the normalized target distribution $\pi(x) = \tilde{\pi}(x) / Z$ for each $x \in \mathcal{X}$. Notoriously, contrasting with MCMC-based methods, the quantity $p_{\intercal}(x ; \theta_{F})$ may be unbiasedly estimated by an importance sampling estimator based on $p_{B}$ as a proposal distribution, which we denote by $\hat{p}_{\intercal}$. 

% A GFlowNet is a model parameterized in many distinct ways with allegedly purposeful distributional guarantees. 

\paragraph{Training GFlowNets.} \autoref{eq:marg} sums over a potentially intractable number of terms and it is generally not possible to directly minimize a difference between $p_{\intercal}$ and $\pi$ to estimate the parameters of a GFlowNet. Consequently, the training of GFlowNets is based upon the enforcement of trajectory-level \textit{balance conditions} ensuring that $p_{\intercal} = \pi$; Bengio, for instance, originally proposed the \textit{flow-matching} condition guaranteeing that the flows entering and leaving a state $v \in \mathcal{S}$ are the same, namely, 
\begin{equation}
    \sum_{u \in \mathcal{S}} \mathbf{A}_{uv} F(u, v) = \sum_{w \in \mathcal{S}} \mathbf{A}_{vw} F(v, w) + \mathbbm{1}_{v \in \mathcal{X}} \tilde{\pi}(v),   
\end{equation}
which they proved sufficient for $p_{\intercal} = \pi$ and was practically enforced by the minimization of the expected squared log-ratio between its left- and right-hand sides under a distribution with full-support over $\mathcal{S}$. Nonetheless, enumerating the parents and children of a state is a computationally burdensome endeavor and the flow-matching condition is not a practically useful training objective for GFlowNets; this observation led to the development of alternative and easy-to-verify balance conditions. More specifically, let $\tau = (s_{i})_{i=0}^{M}$ be a trajectory in $\mathcal{G}$ and, for $0 \le m < n \le M$,  
\begin{equation} \label{eq:stbalance}  
    \mathcal{L}_{m, n}(\tau) = \left( \log \frac{F(s_{m}) \prod_{i = m + 1}^{n} p_{F}(s_{i - 1}, s_{i})}{F(s_{n}) \prod_{i = m + 1}^{n} p_{B}(s_{i}, s_{i - 1})} \right)^{2},  % . 
\end{equation}
in which we omitted the dependency of $\mathcal{L}_{m, n}$ on the parameters of $p_{F}$, $p_{B}$ and $F$ for clarity; we further assume that $\log F(s_{o}) = \theta_{\log Z}$ and $F(x) = \tilde{\pi}(x)$ for each $x \in \mathcal{X}$. Then, the \textit{trajectory balance} (TB) objective is defined by $\mathcal{L}_{TB}(\tau) \coloneqq \mathcal{L}_{o, M}(\tau)$. Correspondingly, the \textit{detailed balance} (DB) objective is $\mathcal{L}_{DB}(\tau) = \sum_{m=1}^{M} \mathcal{L}_{m - 1, m}(\tau)$. Finally, the \textit{subtrajectory balance} (SubTB) objective is, for a hyperparameter $\lambda \in [0, 1]$, determined by   
\begin{equation}
    \mathcal{L}_{SubTB}(\tau) = \frac{\sum_{0 \le m < n \le M} \lambda^{n - m} \mathcal{L}_{m, n}(\tau)}{\sum_{0 \le m < n \le M} \lambda^{n - m}}; 
\end{equation}
in practice, $\lambda$ is commonly fixed at $.9$. Notably, it is currently unclear how a deviance from satisfying a balance condition, which is defined within an extension of the target distribution $\pi$'s support, relates to the discrepancy between GFlowNet's marginal distribution $p_{\intercal}$ and $\pi$. Our work is the first step towards  understanding this relationship.  

% Then, the \textit{trajectory balance} condition is defined by $\mathcal{L}_{TB}(\tau) = \mathcal{L}_{0 M} $ 

% we further assume that $F(s_{o}) = \theta_{\log Z}$ and $F(x) = \tilde{\pi}(x)$ for each $x \in \mathcal{X}$. 
% Consequently, the training of GFlowNets is based upon trajectory-level \textit{balance conditions}.  

% Expose the different training objectives for GFlowNets and their different parameterizations 

\paragraph{Diagnosing GFlowNets.} Assessing the goodness-of-fit of a trained GFlowNet is an important problem mostly unaddressed by the literature. Indeed, previous works considered measures such as the average unnormalized target, $\mathbb{E}_{x \sim p_{\intercal}}[\tilde{\pi}(x)]$, and (Pearson) correlation between the $\log p_{\intercal}$ and $\log \tilde{\pi}$, both of which fail by attributing a respectively high or perfect scores to an incorrectly learned $p_{\intercal} \propto \tilde{\pi}^{\alpha}$ for $\alpha > 1$; the \textit{accuracy} of a GFlowNet, defined by 
\begin{equation}
    \min \left\{ \frac{\mathbb{E}_{x \sim p_{\intercal}}\left[ \tilde{\pi}(x) \right]}{\mathbb{E}_{x \sim \pi} \left[ \tilde{\pi}(x) \right]}, 1 \right\}, 
\end{equation}
similarly assigns an inappropriately high value to a learned distribution with excessive probability mass on the target's modes. Alternatively, when the target distribution support's $\mathcal{X}$ is small enough to be enumerated, one may directly compare $\tilde{\pi}$ to $p_{\intercal}$; however, $\mathcal{X}$ is generally very large and this approach is generally unfeasible. In this setting, our work is the first to consider general and correct diagnostics for GFlowNets and to use them to extensively compare the efficacy of previously proposed training objectives.  


\textcolor{red}{\bf If G1 and G2 have loss(G1) < loss (G2), can l1(G1) > l2(G2)?}

% In this scenario, our work is the first to design efficient and correct diagnostics and to fairly compare the GFlowNets' training objectives. 

% In this scenario, our work is the first to propose provably correct and efficiently implementable diagnostic procedures and to fairly 

% Using the unbiased estimator may be a good starting diagnostic (on a fixed heldout set)

% This will require a huge amount of work for subsuming a comprehensive theory 

\section{Flow-based deterministic bounds on the total variation of GFlowNets} \label{sec:flows} 

We start our analysis by quantifying the consequences of a deviance from the flows' balance on the underlying marginal distribution over the terminal states. Our results highlight, on the one hand, that a lack of flow-balance upstream in the generative process is more damaging to the overall distributional accuracy of a GFlowNet than an equivalent imbalance nearer the terminal states; on the other hand, that a disruption of the detailed balance condition in a single edge may be catastrophically propagated within the network and severely damage the approximation to the target distribution. See the appendix for self-contained proofs of our results. 

\begin{figure}[t]
    \center 
    \def\spacing{-15pt}
\[\begin{tikzcd}
	&[\spacing]&[\spacing] {F+\delta} \\
	&[\spacing] {\frac{F}{g}+\delta} &[\spacing]&[\spacing]&[\spacing] {\frac{F}{g}} \\
% 	& \triangle & \triangle && \triangle & \triangle \\
    {\frac{F}{g^h}+\delta_1} &[\spacing] {\frac{F}{g^h}+\delta_2} &[\spacing]&[\spacing] % {\frac{F}{g^h}+\delta_{g^{h-1}}} 
    &[\spacing] {\frac{F}{g^h}} &[\spacing] {\frac{F}{g^h}} % & {\frac{F}{g^h}}
    \arrow["{\text{degree g}}", swap, from=1-3, to=2-2]
	\arrow[from=1-3, to=2-5]
	\arrow[from=2-5, to=3-6]
	\arrow[from=2-5, to=3-5]
	\arrow[from=2-2, to=3-1]
	\arrow[from=2-2, to=3-2] 
    % \arrow[from=3-2, to=4-1]
	% \arrow[from=3-2, to=4-2]
	% \arrow[from=3-3, to=4-3]
	% \arrow[from=3-5, to=4-5]
	% \arrow[from=3-6, to=4-6]
	% \arrow[from=3-6, to=4-7]
\end{tikzcd}\]
\caption{\textbf{Imbalanced flows in a regular tree} with width $g = 2$ and depth $h = 2$. The extra flow within the root's left child breaks the expected uniform distribution over the $g^{h}$ leaves.}
% \caption{\textbf{Imbalanced flow network in a tree.}}
% \caption{A flow network with a extra flow of $\delta$ in one of the branches of the initial state} 
    \label{fig:a} 
    % \label{fig:treesgraphs} 
\end{figure}

% \begin{theorem}[Total variation of the sampling distribution] Let $\delta >0$ and $\sum_{i=1}^{g^{h-1}} \delta_i = \delta$, where $\delta_i \in [0, \delta]$ for all $i \in \{1,2, \dots, g^{h-1}\}$. Suppose that we have the flow network $(G_T, F+\delta)$ abiding by Assumption~\ref{as: gf_tree_unif} besides the first edge from the root to a son where it has a $\delta$ increasing generating a new target distribution $\pi$ (see Figure [ref]).  Under these conditions, the total variation distance between $\pi$ and $\pi^*$ is bounded above and below by 
% \begin{align*}
% & \epsilon(\delta, g) \leq ||\pi - \pi^*||_{\scaleto{\textbf{TV}}{3pt}} \leq \epsilon(\delta, g^h) \quad \text{where}
% \\
% & \epsilon(a,b) := \Big(1 - \frac{1}{b} \Big) \frac{a}{F+a}\,.
% \end{align*}
% \end{theorem}


\paragraph{Bounds on TV for arbitrary state graphs.} We show that the intuition built upon the previous discussion relatively to tree-structured state graphs carries out to general acyclic generative processes, namely, (i) that an imbalanced edge reaching many terminal states has a larger impact on the approximation to the target distribution than an imbalanced edge leading to comparably fewer terminal states; and (ii) the difficult of training a GFlowNet is an increasing function of the target distribution's support. To start with, \autoref{thm:wca} lays out a worst-case analysis of the propagated errors within a flow network and underlines point (ii) above. 

\begin{theorem}[Deterministic flow-based bounds for the TV in general graphs] \label{thm:wca} 
    Let $(\mathcal{G}, F, \tilde{\pi})$ be a imbalanced flow network defined on a DAG $\mathcal{G}$ and $\tilde{\pi}$ be an uniform distribution supported on a space with $n$ objects. Assume that, except for an edge $(u, v)$ in $\mathcal{G}$, i.e., $\mathbf{A}_{uv} = 1$, for which  
    \begin{equation}
        F(s) p_{F}(s, s') - F(s') p_{B}(s', s) = \delta 
    \end{equation}
    with $\delta \ge 0$, the network is balanced. Let $p_{\intercal}^{(\delta)}$ be the corresponding marginal distribution over $\mathcal{X}$ and $d$ be the number of reachable terminal states from state $v$. Then, 
    \begin{equation}
        \frac{\delta (n - d)}{2n (F + \delta)} \le \|p_{\intercal}^{(\delta)} - \pi\|_{\scaleto{\textbf{TV}}{3pt}} \le \frac{\delta (n - 1)}{2n (F + \delta)}, 
    \end{equation}
    in which $\pi \propto \tilde{\pi}$ and $F(s_{o}) = F + \delta$ is the total flow. 
\end{theorem}
% \begin{theorem}[Total variation of the sampling distribution] Let $(G_n, F)$ be a flow network which should generates a target distribution $\pi$ uniform in the number of final vertices. Suppose there exists an edge in $G_n$, that is $s \to s' \in \mathbb{A}$ such that
% \[  F(s)P_{F}(s' | s) - F(s')P_{B}(s|s') = \delta \,,\]
% where $\delta > 0$. Then we have that $(G_n, F)$ generates a probability distribution $\mu_{\delta}$ such that
% \begin{align*}
% & \frac{\delta(n - d)}{2n(F + \delta)} \le ||\mu_{\delta} -\pi||_{\scaleto{\textbf{TV}}{3pt}} \leq \frac{\delta(n + dn - d)}{2n(F + \delta)} \,,
% \end{align*}
% where $d \in \{1,2, \dots, n-1\}$ is the number of final vertices that are descendants of $s'$. 
% \end{theorem}

Remarkably, this result shows that the accuracy of the downstream distribution depends linearly upon the deviation from the detailed balance within a single edge and, therefore, even a small error may lead to significant disruptions in the distributional approximation. Correspondingly, the result underscores the hardening properties of the number $n$ of terminal states regarding the training of GFlowNets: both the (provably tight) lower and upper bounds are increasing functions of $n$. Nonetheless, the preceding statement is poorly informative about the relevance of the particular imbalanced state to the overall generative process, an often observed phenomenon in worst-case analyses. In this context, we show in \autoref{thm:random} that, assuming Dirichlet-distributed downstream flows $\delta_{1}, \dots, \delta_{d}$ as shown in \autoref{fig:a} and an uniform target, the average TV distance between the learned and target distributions is an increasing function of the number of reachable states from the imbalanced node.  

\begin{theorem}[Expected TV under Dirichlet-distributed extra flows]\label{thm:random} 
    Let $(\mathcal{G}, F, \tilde{\pi})$ be the same imbalanced flow network as in \autoref{thm:wca}. Assume that the extra flow $\delta$ is distributed among the imbalanced node's terminal children according to a Dirichlet distribution with concentration parameter $\alpha \in \mathbb{R}^{d}$, i.e., 
    \begin{equation*}
        \left(\nicefrac{\delta_{i}}{\delta}\right)_{1 \le i \le d} \sim \textrm{Dir}\left(\alpha\right) % .   
    \end{equation*}
    (see \autoref{fig:a} for an illustration in trees). Denote $x_{i} = \nicefrac{\delta_{i}}{\delta} \in [0, 1]$ for $1 \le i \le d$ and $\mu_{\mathbf{x}, \delta}$ for the distribution resulted from the corresponding $\delta$-imbalance. Then, 
    \begin{equation*}
        \underset{\mathbf{x} \sim \textrm{Dir}(\alpha)}{\mathbb{E}} \left[ \|\mu_{\mathbf{x}, \delta} - \pi\|_{TV} \right] = \left( d \left( \Lambda - \frac{1}{n} \right) + 1 \right) \cdot \frac{\delta}{2(F + \delta)}, % .   
    \end{equation*}
    with, by letting $F_{a,  b}$ be the CDF of a $\textrm{Beta}(a, b)$ distribution,  
    \begin{equation*}
        \Lambda = \nicefrac{2}{n} F_{\alpha_{i}, 2\bar{\alpha}_{i}} \left( \nicefrac{1}{n} \right) - 2 F_{\alpha_{i} + 1, 2\bar{\alpha}_{i} + 1} \left( \nicefrac{1}{n} \right) + 1 - \nicefrac{1}{n} % . 
    \end{equation*}
    and $\bar{\alpha}_{i} = \sum_{j \neq i} \alpha_{j}$. 
\end{theorem}

However, it is unclear whether $\Lambda > \nicefrac{1}{n}$ and, in particular, whether $d$ positively or negatively influences the expected TV distance: for $n = 1$, $\Lambda = 0$, whereas $\lim_{n \rightarrow \infty} \Lambda = 1$. In this sense, \autoref{thm:random}'s corollary below reveals that, for uniformly distributed $\delta_{i}$, $d$ has a positive effect on the expected approximation error in all practically relevant cases, reiterating that ensuring the flow-balance in nodes from which many terminal states are reachable must be prioritized during training --- as we previously remarked in the context of tree-based state graphs. 

\begin{corollary} \label{col:random} 
    In \autoref{thm:random}, let $\alpha_{i} = 1$ for each $1 \le i \le d$, i.e., $(\nicefrac{\delta_{i}}{\delta})_{i=1}^{d}$ is uniformly distributed within the $(d + 1)$-dimensional simplex. Under these conditions, 
    \begin{equation*}
        \Lambda(d, n) = (d - 1) \left( \frac{1}{2} - \frac{1}{n} + \frac{1}{n^{2}}\right)  
    \end{equation*}
    and the expected TV between $\mu_{\mathbf{x}, \delta}$  and $\tilde{\pi}$, 
    \begin{equation*}
        \frac{\delta}{2(F + \delta)} \left( d \left( \Lambda(d, n) - \frac{1}{n} \right) + 1 \right),   
    \end{equation*}
    is an increasing function of $d \ge 1$ for $n > 2$. 
\end{corollary}
% Remarkably, this is very and very threatening to mankind and to the planet.  

% This is a weighted version of the detailed balance condition, which may be adequate for large state spaces under constraining computational settings

% \paragraph{Transition-decomposable and discriminatory DB loss ($\text{TD}^{3}$).}  

\paragraph{Theoretical analysis for tree-structured state graphs.} To gain some intuition, we first consider the scenario in which the state graph $\mathcal{G}$ is a regular tree with depth of $h$ in which each node, except the leaves representing the terminal states, has $g$ children and that the target distribution is uniform; see \autoref{fig:a}. Under these conditions, \autoref{thm:a} underscores that an improperly estimated flow near the initial state may significantly hinder the distributional approximation. 

\begin{example}[Total variation of the sampling distribution for trees] \label{thm:a} 
    Let $(\mathcal{G}, F, \tilde{\pi})$ be a balanced flow network with state graph $\mathcal{G}$ that is a tree with width $g$ and depth $h$, flow $F$ and uniform target distribution $\tilde{\pi}$.  Define $\delta \ge 0$ and $\sum_{i=1}^{g^{h - 1}} \delta_{i} = \delta$, with $\delta_{i} \in [0, \delta]$ for all $i \in \{1, 2, \dots, g^{h - 1}\}$. Then, assume that the flow within an edge stemming from the root has an increase of $\delta$, leading to an non-uniform marginal distribution $p_{\intercal}$ over the leaves; see \autoref{fig:a}. In this scenario, the total variation distance between $p^{\intercal}$ and $\pi \propto \tilde{\pi}$ is bounded above and below by 
    \begin{equation}
    \label{eq:a} 
    \begin{aligned}
        & \epsilon(\delta, g) \leq \|p_{\intercal} - \pi\|_{\scaleto{\textbf{TV}}{3pt}} \leq \epsilon(\delta, g^{h}), \, \text{with} \\ 
        & \epsilon(a, b) \coloneqq \left(1 - \frac{1}{b}\right) \frac{a}{F + a}.  
    \end{aligned}
    \end{equation} 
\end{example}

Notably, \autoref{eq:a} reveals that the effect of a broken balance of the learned flows over the downstream distribution crucially depends upon the height $h$ and, particularly, of the number of reachable leaves from the imbalanced edge; that is, the upper bound on the TV distance between $p_{\intercal}$ and $\pi$ increases exponentially in $h$ through $g^{h}$. Likewise, the lower bound on the TV increases as a function of the tree's width $g$, implying that the impact of the imbalance of the flows over $p_{\intercal}$ is larger for large state graphs than for comparably small ones. This underlines the advantages of building parsimonious state graphs and rigorously explains why the difficult of training a GFlowNet increases as we endeavor to approximate distributions with larger supports. 

\paragraph{Going beyond uniform distributions.} Our analysis was up to this moment based on the assumption of an uniform target distribution. Although insightful, this approach is somewhat constraining since a practitioner is generally interested in sampling from non-uniform distributions. In this context, \autoref{thm:app:a} in the appendix extends the previous results to multi-modal distributions and shows, similarly to \autoref{thm:a}, the relevance of upstream states and, differently from prior discussion, that the impact on the distributional accuracy of a lack of balance at a state having a large amount of probability mass concentrated among its descendants is larger than the impact corresponding to the imbalance of a state associated to a relatively small probability mass. This justifies, e.g., the empirical success of the replay buffer, which stores trajectories leading to high-probability terminal states and periodically replays them during training to reduce their associated error. 

% \paragraph{Transition-decomposable and discriminatory DB loss ($\text{TD}^{3}$).} 
\paragraph{Transition-decomposable discriminatory DB loss ($\text{TD}^{3}$).} \autoref{thm:a}, \autoref{thm:wca} and \autoref{thm:random} emphasized under different circumstances that, during training, one should prioritize ensuring the flow balance of states that are ancestral to high probability regions of the target distribution. Hence, we build upon this fact to define, for a trajectory $\tau = (s_{o}, s_{1}, \dots, s_{m})$, the loss function 
\begin{equation*}
    \mathcal{L}_{TD}(\tau) = \frac{1}{\sum_{i} \gamma(s_{i})} \sum_{1 \le i \le m} \gamma(s_{i}) \mathcal{L}_{i - 1, i}(\tau)% , 
\end{equation*}
(see \autoref{eq:stbalance}), in which $s \mapsto \gamma(s)$ is a weighting function; we call $\mathcal{L}_{TD}$ the \textit{transition-decomposable discriminatory detailed balance loss} ($\text{TD}^{3}$). Ideally, $\gamma(s)$ would be proportional to the probability mass associated with the $s$'s terminal descendants. In practice, however, it is very difficult to estimate such quantity in general state graphs and we instead consider $\gamma(s) = (T - d(s, s_{o}))^{2}$ as a proxy to this quantity, with $d(s, s_{o})$ as the geodesic distance between $s$ and the state graph's initial state and $T$ as the maximum trajectory length, which is always known. Intuitively, $\gamma$ assigns high weight to nodes upstream in the generative process leading to a large number of terminal states and, thus, related to a high probability region of the target. Moreover, we regard $\gamma$ as a quadratic function of $d(s, s_{o})$ (instead of a linear one) for the expected TV in \autoref{col:random} quadratically depends upon the imbalanced node's terminal children.   

% Moreover, we regard $\gamma$ as a quadratic function of $d(s, s_{o})$ as the expected TV in \autoref{col:random} quadratically depends upon the node's children.    

% Moreover, we regard $\gamma$ as a quadratic function of $d(s, s_{o})$ due to the quadratic dependence 

% Write about the annealing stuff (in the empirical section) and about using square instead of linear weighting 

\begin{figure}[!t] 
    \centering
    \includegraphics[width=.75\linewidth]{figures/avg_l1_gflownets_sets_weighted_db.pdf}
    \caption{\textbf{Weighted DB accelerates training convergence relatively to standard DB.} By weighting each transition proportionally to its closeness to the state graph's root, we notoriously improve upon the DB loss. This empirically supports our theoretical claims regarding the elevated importance of ensuring the balance within states leading to a large number of terminal states.}
    \label{fig:aaa}
\end{figure}

\paragraph{Empirical illustration.} 
We consider the task of generating discrete sets of fixed size with elements extracted from a finite warehouse; each element of this warehouse is assigned to a random positive value and the unnormalized distribution of a set corresponds to the product of its elements' values. 
% We consider the task of generating discrete sets of fixed size with elements extracted from a finite warehouse; each element of this warehouse is assigned to a random positive value and the unnormalized distribution of a set corresponds to the product of its elements' values. 
In this context, \autoref{fig:aaa} shows that $\mathcal{L}_{TD}$ significantly accelerates the training convergence of a GFlowNet relatively to a standard detailed balance loss (corresponding to setting $\gamma \coloneqq 1$), corroborating our rigorously laid analysis regarding the governing effect of imbalanced upstream states on the distributional approximation. Counterpositively, \autoref{fig:aaa} also points out that by weighting each transition proportionally to its distance to the root, thereby assigning larger weight to nodes downstream --- related to regions of relatively smaller probability ---, significantly worsens the resulting GFlowNet's accuracy. Notably, we found it beneficial to consider the tempered weighting function $\gamma_{\beta}(s) = \gamma(s)^{\beta}$ and linearly anneal the parameter $\beta$ to $0$ throughout training. We provide more implementation details in the accompanying supplementary material.  

% We provide more implementation details in the accompanying supplementary material.    

% This justifies, for instance, the empirical success of the replay buffer, a heuristics that consists of storing trajectories associated to high-probability terminal states and periodically replaying them during training to mitigate their deviation from balance.   

% \textcolor{red}{The results concerning multi-modal distributions in the appendix are very difficult to read and doesn't seem to be quite correct.} 

\newpage 

\section{Distributional limits of GNN-based policy networks} \label{sec:rep}  

\begin{figure}[!t] 
    \centering
    \includegraphics[page=1, width=\linewidth]{graphsss.pdf}
    \caption{\textbf{State graph for a GNN-based GFlowNet} generating 3-node graphs. The GNN's permutation invariance ensures each state is uniquely represented and the smallness resulting state graph.}% .}}
    \label{fig:gnns}
    \vspace{-12pt} 
\end{figure}

\paragraph{Advantages of GNN-based parameterizations.} By employing a GNN to parameterize the polices of a GFlowNet, one effectively reduces the size of the state graph $\mathcal{G}$ and significantly eases the optimization problem underlying the model's training, as was previously empirically observed by [ref]. Consider, for example, the generative process for graphs with three unlabeled nodes outlined in Figure [ref], in which we explicitly instantiate the state graph for a GFlowNet parameterized by a GNN (left) and by a neural network that is not permutationally invariant by design (right; e.g., an MLP). Noticeably, $\mathcal{G}$'s size is substantially reduced due to adoption of a permutation-invariant neural network: when generating graphs with $12$ nodes, for instance, this parameterization leads to a significant $5 \cdot 10^{13}$-fold reduction in $\mathcal{G}$'s size.  

\begin{figure}[!t] 
    \centering
    \includegraphics[page=2, width=\linewidth]{graphsss.pdf}
    \caption{\textbf{State graph for a MLP-based GFlowNet} generating 3-node graphs. The absence of an inductive bias for permutation variance leads to a different treatment for the column-wise isomorphic graphs by the MLP. Compare to \autoref{fig:gnns} and note the reduction in size due to using an inductively biased neural network. We omitted some edges from the state graph to avoid cluttering.} % .}}
    \label{fig:mlps}
\end{figure}

% Notably, by parameterizing the policies with a permutation-invariance task, the state graph's size for this task is reduced by a factorial  

% Notably, the state graph's size for this task factorially  

% Using a GNN to parameterize the distributions is very helpful. I am totally not motivated to continue writing this; the bounds of the preceding section are so, so, so loose, so damn loose. 

\begin{figure}
    \centering
    \begin{subfigure}{.49\linewidth}  
        \includegraphics[width=\textwidth]{figures/W&B Chart 1_7_2024, 10_25_46 PM.png}
        \caption{Heterogeneous target.}
    \end{subfigure}
    \begin{subfigure}{.49\linewidth} 
        \includegraphics[width=\textwidth]{figures/W&B Chart 1_7_2024, 10_26_45 PM.png}
        \caption{Homogeneous target.}
    \end{subfigure}
    \caption{\textbf{GNN-based GFlowNets are less expressive than their MLP-based counterparts.} GNN-based GFlowNets can learn to sample some (right), but not all (left), distributions represented by the state graph of \autoref{fig:gnns_state}. An MLP-based GFlowNet, in contrast, is not subject to such constraints and can sample from any target.}
    \label{fig:gnns}
\end{figure}

\paragraph{Limitations of 1-WL GNN-based parameterizations.} Nevertheless, despite its computational advantages, the use of GNN-based policies hinders the expressivity of GFlowNets due to the limited discriminative power of the 1-WL GNNs --- which are unable to distinguish some simple graph structures [ref]. 
Importantly, 1-WL GNNs such as GCN and GIN [ref, ref] are the most common GNN instances in practically used implementations of GFlowNets [ref, ref, ref].
In this context, the next theorem shows that there exist some distributions from which a GFlowNet parameterized by a GNN cannot learn to sample. 

\begin{figure}[!t] 
    \centering
    \includegraphics[page=3, width=\linewidth]{graphsss.pdf}
    \caption{\textbf{Limitations of a GNN-based GFlowNet.} The highlighted graphs are indistinguishable by a 1-WL GNN and, thus, a GFlowNet with such parameterization would learn the same policies at both states. Consequently, such GFlowNet cannot learn to sample from the target due to the heterogeneity of its unnormalized values at the root's left ($3$ and $3$) and right ($3$ and $5$) grandchildren.}
    \label{fig:gnns_state}
\end{figure}

\begin{theorem}[Distributional limits for GNN-based GFlowNets] 
\label{thm:aaa} 
    Let $(\mathcal{G}, p_{F}(\cdot ; \theta_{F}), p_{B}(\cdot ; \theta_{B}), F(\cdot ; \theta_{N}), \theta_{\log Z})$ be a GFlowNet and assume that $p_{F}$ is parameterized by a 1-WL GNN. Then, there is a state graph $\mathcal{G}$ and a target distribution $\tilde{\pi}$ such that there is no $\theta_{F}$ for which the learned marginal distribution matches $\pi \propto \tilde{\pi}$.   
\end{theorem}

Figure [ref] illustrates \autoref{thm:aaa}. As a 1-WL GNN cannot distinguish between the root's children, the GFlowNet will inevitably learn the same forward conditional distribution at both highlighted states. Thus, such GFlowNet is fundamentally unable to learn to sample from any distribution over the leaves in Figure [ref] in which the probabilities of the root's left and right grandchildren differ. 

\paragraph{Empirical illustration.} We validate our results in two different generative tasks. Firstly, Figure [ref] (left) shows that a GNN-based GFlowNet does converge substantially faster than an MLP-based one, both of which with approximately the same number of parameters, for the task of generating graphs with $n = 12$ unlabeled nodes. Secondly, Figure [ref] (right) confirms that, for the generative process depicted in Figure [ref], a GFlowNet employing GIN to parameterize the policies doesn't converge at all to the an heterogeneous target distribution; contrastingly, an MLP-based GFlowNet swiftly converges to the target distribution. The bottom line is that, when implementing GNNs to parameterize the GFlowNets' policies, one should be cautious to avoid limiting GFlowNet's capacity to learn to sample from the target distribution. This problem may be of particular concern in drug discovery --- in which non-insomorphic graphs indistinguishable by the 1-WL test may be commonly found. The discriminative power of GNNs, however, is not a constraining factor for causal discovery, where the sampled causal graphs' nodes are uniquely labeled and GNNs work as universal approximators, nor in phylogenetic inference, where the sampled complete binary trees that can be readily distinguished by the 1-WL test and, thus, by 1-WL GNNs. 

\textcolor{red}{Did the forward-looking trick work?}


\newpage

\section{Convergence diagnostics for GFlowNets} \label{sec:cov}  

In this section, we propose a provably correct metric for verifying the distributional incorrectness of GFlowNets, along with probably approximately correct (PAC) statistical guarantees for the accompanying estimators (\Cref{sec:assgfn}). Then, we apply this metric to two recently published methods for training GFlowNets, namely, LED- and FL-GFlowNets, and note that they are generally incapable of learning to sample from the target distribution (\Cref{sec:rev}). 

\subsection{Assessing GFlowNets} \label{sec:assgfn}

\paragraph{Flow Consistency in Sub-graphs (FCS).}  Let $(p_{F}, p_{B}, R)$ be a GFlowNet trained to sample from a distribution $R$ on $\mathcal{X}$. For each $x \in \mathcal{X}$, we can unbiasedly estimate the marginal probability of $x$ induced by $p_{F}$ through 
\begin{equation}
    p_{T}(x) = \sum_{\tau \rightsquigarrow x} p_{F}(\tau) = \mathbb{E}_{\tau \sim p_{B}(\cdot | x)} \left[ \frac{p_{F}(\tau)}{p_{B}(\tau)}  \right]. % ; % . 
\end{equation}
For frequently implemented benchmark environments for GFlowNets (such as grid world, sequence design and set generation), the preceding expectation may be directly computed by enumerating the (relatively small) trajectories leading to $x$. In this context, we build upon the identity 
\begin{equation}
    \mathbb{P}[X = x | X \in S] = \frac{\mathbb{P}[X = x]}{\mathbb{P}[X \in S]} = \frac{p_{T}(x)}{\sum_{y \in S} p_{T}(y)}  
\end{equation}
(and its analogous for $\pi = \nicefrac{R}{Z}$) for $X \sim  p_{T}$ and $S \subseteq \mathcal{X}$ to show that a correctly trained GFlowNet should satisfy the balance conditions in sub-graphs of the underlying flow network. For this, define for $S \subseteq \mathcal{X}$ 
\begin{equation*}
    \begin{split} 
    p_{T}^{(S)}(x ; \theta) = \frac{p_{T}(x ; \theta)}{\sum_{y \in S} p_{T}(y ; \theta)}, \, \pi^{(S)}(x) = \frac{R(x)}{\sum_{y \in S} R(y)} \\
    \text{ and } e(S, \theta) = \frac{1}{2} \sum_{x \in S} |p_{T}^{(S)}(x ; \theta) - \pi^{(S)}(x)|.
    \end{split} 
\end{equation*}
Finally, let $p$ be a distribution of full-support over fixed-size subsets of $\mathcal{X}$. In this scenario, the metric $\mathbb{E}_{S \sim p}[e(S, \theta)]$ quantifies GFlowNet's correctness in sub-graphs induced by $S \subseteq \mathcal{X}$; we call it the \emph{flow consistency in sub-graphs} (FCS). The following proposition shows that FCS is closely related to the TV distance between the learned and target distributions on $\mathcal{X}$. Nevertheless, contrarily to TV, FCS is computationally tractable since it is not reliant upon an extensive enumeration of the state graph. 

\begin{proposition}[Equivalence between TV and FCS] \label{prop:tv_fsc}  
    Let $p$ be a full-support distribution over $\{S \subseteq \mathcal{X} \colon |S| = B\}$ for some $B \ge 2$. Also, let $d_{TV} = e(\mathcal{X}, \theta)$ be the TV distance between $p_{T}$ and $\pi$ for a GFlowNet parameterized by $\theta$. Then, $d_{TV} = 0$ if and only if $\mathbb{E}_{S \sim p}[e(S, \theta)] = 0$.   
\end{proposition}


% Nevertheless, contrarily to TV, FCS does not require extensively searching the state graph.  


% $\{S_{1}, \dots, S_{N}\} \sim p$ be a random sample of fixed-size subsets of $\mathcal{X}$ according to a full-support distribution $p$. 

\paragraph{PAC statistical guarantees for $e(S ; \theta)$.} Realistically, we use a Monte Carlo approximation of the intractable quantity $\mathbb{E}_{S \sim p}[e(S, \theta)]$ to assess the accuracy of the learned GFlowNet due to the large size of $\mathcal{X}$. In this sense, the next proposition underlines that our estimator is, with high probability, a good approximation to FCS. % and, when learning is accurate, to TV. 

\begin{proposition}[PAC bound for FCS] \label{prop:pac} 
    Let $\{S_{1}, \dots, S_{m}\} \sim p$ be a random sample from the distribution $p$ of \Cref{prop:tv_fsc} and $\delta \in (0, 1)$. Then, 
    \begin{equation*}
        \begin{aligned} 
            \mathbb{P}_{S \sim p}\left[ 
                \underset{S \sim p}{\mathbb{E}} \left[ e(S, \theta) \right] 
                \le \frac{1}{m} \sum_{1 \le i \le m} e(S_{i}, \theta) + \sqrt{\frac{\log \frac{1}{\delta}}{2m}}    
            \right] \ge 1 - \delta.
        \end{aligned} 
    \end{equation*} 
    % and 
    % \begin{equation*}
    %     \begin{split} 
    %         \mathbb{P}_{S \sim p}\left[ d_{TV} \le \frac{1}{m} \sum_{1 \le i \le m} e(S_{i}, \theta) + s \right] \\
    %         \le \exp\left\{\left( d_{TV} - \mathbb{E}_{S \sim p} [ e(S, \theta) ]\right) + \frac{1}{8m} - s \right\}. 
    %     \end{split} 
    % \end{equation*}
\end{proposition}

This is, to the best of our knowledge, the first PAC-style result concerning GFlowNets. We believe that a rigorous analysis of the generalization capabilities of this class of models, which was already hinted by previous works [ref], will greatly benefit the understanding of its potentialities.   

\subsection{Revisiting LED- and FL-GFlowNets} \label{sec:rev} 

\paragraph{LED- and FL-GFlowNets.} 

\paragraph{Experimental setup.} 

\paragraph{Results.} 


% \paragraph{Using the the partition function's variance.} 

% \paragraph{Score-based goodness-of-fit test.} Use something similar to Stein's variational inference 

% \paragraph{Comparing the distribution of objects' features.} Exploit the compositionality of the sampled objects to compare the distributions of the samples' components  

\section{Experiments} \label{sec:experiments} 

\paragraph{Task descriptions.} 

\begin{enumerate}
    \item set generation, 
    \item design of sequences, 
    \item phylogenetic inference, 
    \item grid world, 
    \item linear preference learning, 
    \item bayesian structure learning 
\end{enumerate}

\paragraph{Experimental setup.} 

\paragraph{Comparing TB, SubTB, DB and CB.} 

% \section{Flows for trees and uniform distributions}


ping

\textcolor{blue}{
\textbf{To-do list (sensitivity analysis):}
\begin{enumerate}
    \item Sensitivity analysis for regular trees and uniform distribution 
    \item Generalization for DAGs
    \item Generalization for non-uniform distributions
\end{enumerate}
}
\textcolor{orange}{
\textbf{To-do list (policy networks):}
\begin{enumerate}
    \item Anonymous
    \begin{enumerate}
        \item Balance is impossible for some pairs of pointed DAGs and reward functions
        \item Some characterization of rewards that are particularly hard to approximate
        \item Sequences, Multisets, Anonymous and non-anonymous graphs (directed and undirected) 
\item trade-off between invariances in the networks and built into the state graph    \end{enumerate}
    \item Non-anonymous
\end{enumerate}
    }

\textcolor{red}{
\textbf{To-do list (evaluation and diagnostic of GFLowNets):}
\begin{enumerate}
    \item Current evaluation protocols are crap (usually focus on covering modes rather than approximation). Doing the right thing is also computationally infeasible for larges state-spaces 
    \item Convergence diagnostic based on some estimate of $\delta$, leveraging our theorems in the first part of the paper
    \item some diagonose based on, e.g, the estimates we can get from $R$ based on the trajectory-balance loss (in equilibrium, should be path independent, i.e., a constant)
\end{enumerate}
}

 

\section{Related works} 

\paragraph{Applications of GFlowNets.} 

\paragraph{Limitations of GNNs.} 

\section{Conclusions} 


\bibliographystyle{icml2024} 
\bibliography{bibliography} 

\newpage 
\onecolumn 

\appendix

\subsection{Uniform distributions and uniform flows}
\begin{itemize}
    \item A uniform flow of degree $g$ and height $h$ is a Markovian flow that models a uniform distribution, meaning all the leaf nodes have the same value. The policy of such flow consists in a function that takes the incoming flow and splits it equally between each of the g outgoing nodes.
\end{itemize}

To start let us consider the example of a flow trained on a uniform distribution and policy that takes the incoming flow and split it equally to all outgoing states. The resulting uniform distribution on the terminal nodes density is  represented by $\pi^*(x)=\frac{1}{g^h}$, for each terminal object in the domain $x \in \mathcal{X}$ and $|\mathcal{X}|=g^h$.

Let us consider the case that the policy at the root of the network introduces an error in the flow of size $\delta$, meaning that one children node now will receive a flow $\frac{F}{g}+\delta$ and the other $g-1$ will continue with $\frac{F}{g}$, which is equivalent to the policy with a probability density (after normalizing) that assigns a probability $\frac{F+g\delta}{g(F+\delta)}$ to one branch and $\frac{F}{g(F+\delta)}$ to the other $g-1$ branches. The total variation distance between this new density and the original policy (uniform probability for each $g$ branches) is $\epsilon(\delta, g)=(1-\frac{1}{g})\frac{\delta}{F+\delta}$. Now we denote the resulting sampling distribution induced by this modified flow as $\pi(x)$.


\begin{figure}[h]
    \center 
% https://q.uiver.app/#q=WzAsMTMsWzMsMCwiRitcXGRlbHRhIl0sWzIsMSwiXFxmcmFje0Z9e2d9K1xcZGVsdGEiXSxbNCwxLCJcXGZyYWN7Rn17Z31cXHRleHR7IH1cXHRyaWFuZ2xlIl0sWzIsMiwiXFx0cmlhbmdsZSJdLFswLDMsIlxcZnJhY3tGfXtnXmh9K1xcZGVsdGFfMSJdLFsxLDIsIlxcdHJpYW5nbGUiXSxbNSwyLCJcXHRyaWFuZ2xlIl0sWzQsMiwiXFx0cmlhbmdsZSJdLFsxLDMsIlxcZnJhY3tGfXtnXmh9K1xcZGVsdGFfMlxcdGV4dHsgIH1cXGxkb3RzICJdLFsyLDMsIlxcZnJhY3tGfXtnXmh9K1xcZGVsdGFfe2dee2gtMX19Il0sWzQsMywiXFxmcmFje0Z9e2deaH0iXSxbNSwzLCJcXGZyYWN7Rn17Z15ofSJdLFs2LDMsIlxcZnJhY3tGfXtnXmh9Il0sWzAsMSwiXFx0ZXh0e2RlZ3JlZSBnfSJdLFswLDJdLFsyLDZdLFsyLDddLFsxLDNdLFsxLDVdLFs1LDRdLFs1LDhdLFszLDldLFs3LDEwXSxbNiwxMV0sWzYsMTJdXQ==
\[\begin{tikzcd}
	&&& {F+\delta} \\
	&& {\frac{F}{g}+\delta} && {\frac{F}{g}\text{ }\triangle} \\
	& \triangle & \triangle && \triangle & \triangle \\
	{\frac{F}{g^h}+\delta_1} & {\frac{F}{g^h}+\delta_2\text{  }\ldots } & {\frac{F}{g^h}+\delta_{g^{h-1}}} && {\frac{F}{g^h}} & {\frac{F}{g^h}} & {\frac{F}{g^h}}
	\arrow["{\text{degree g}}", from=1-4, to=2-3]
	\arrow[from=1-4, to=2-5]
	\arrow[from=2-5, to=3-6]
	\arrow[from=2-5, to=3-5]
	\arrow[from=2-3, to=3-3]
	\arrow[from=2-3, to=3-2]
	\arrow[from=3-2, to=4-1]
	\arrow[from=3-2, to=4-2]
	\arrow[from=3-3, to=4-3]
	\arrow[from=3-5, to=4-5]
	\arrow[from=3-6, to=4-6]
	\arrow[from=3-6, to=4-7]
\end{tikzcd}\]
\caption{A flow network with a extra flow of $\delta$ in one of the branches of the initial state} 
    \label{fig:tree_graph} 
\end{figure}

Let $\mu$ and $\nu$ be two probability measures, then we denote $||\mu - \nu||_{\scaleto{\textbf{TV}}{3pt}}$ as the total variation distance between them. 

\begin{assumption}\label{as: gf_tree_unif}
 Let the pair $(G_T, F)$ be a flow network such that $G_T$ is a regular tree with degree $g$ and depth $h$. Furthermore, assume that $F$ spreads uniformly in the edges of $G_T$, then the target distribution $\pi^*$ generated by $(G_T, F)$ is uniform.      
\end{assumption}

\begin{theorem}[Total variation of the sampling distribution] Let $\delta >0$ and $\sum_{i=1}^{g^{h-1}} \delta_i = \delta$, where $\delta_i \in [0, \delta]$ for all $i \in \{1,2, \dots, g^{h-1}\}$. Suppose that we have the flow network $(G_T, F+\delta)$  abiding by Assumption~\ref{as: gf_tree_unif} besides the first edge from the root to a son where it has a $\delta$ increasing generating a new target distribution $\pi$. Then under these conditions describe the total variation distance between $\pi$ and $\pi^*$ is bounded above and below by the following
\begin{align*}
& \epsilon(\delta, g) \leq ||\pi - \pi^*||_{\scaleto{\textbf{TV}}{3pt}} \leq \epsilon(\delta, g^h) \quad \text{where}
\\
& \epsilon(a,b) := \Big(1 - \frac{1}{b} \Big) \frac{a}{F+a}\,.
\end{align*}
\end{theorem}

\begin{proof}
The terminal states of the modified flow network will have two types of nodes, with flow $\frac{F}{g^h}$ and $\frac{F}{g^h}+\delta_{i}$, with $\delta_i \geq 0$ and $\sum_{i=1}^{g^{h-1}} \delta_i = \delta$. We normalize those probabilities to obtain the individual probabilities for each terminal state, which determines the density of each sample. From that, we can proceed to compute the total variation distance between $\pi$ and $\pi^*$.
\begin{align*}
    ||\pi - \pi^*||_{\scaleto{\textbf{TV}}{3pt}} &= \frac{1}{2}\sum_{x \in \mathcal{X}} | \pi(x)- \pi^*(x) | \\
                          &= \frac{1}{2}\left[(g^h-g^{h-1})\left|\frac{F}{g^h}\frac{1}{F+\delta} - \frac{1}{g^h}\right|+ \sum_{i=1}^{g^{h-1}} \left|\frac{F+g^h\delta_i}{g^h}\frac{1}{F+\delta} - \frac{1}{g^h}\right| \right] \\
                          &= \frac{1}{2}\left[\frac{g^h\delta-g^{h-1}\delta+\sum_{i=1}^{g^{h-1}}|g^h\delta_i-\delta|}{g^h(F+\delta)}\right]
\end{align*}

We can lower bound $\sum_{i=1}^{g^{h-1}}|g^h\delta_i-\delta|$, by considering that $\sum_{i=1}^{g^{h-1}}(g^h\delta_i-\delta)=g^{h}\delta-g^{h-1}\delta$, taking the absolute value of the result and each element of the sum to obtain $g^{h}\delta-g^{h-1}\delta \leq \sum_{i=1}^{g^{h-1}}|g^h\delta_i-\delta|$. Thus we obtain the lower bound 
\begin{align*}
\frac{1}{2}\left[\frac{g^h\delta-g^{h-1}\delta+g^{h}\delta-g^{h-1}\delta}{g^h(F+\delta)}\right]&\leq \frac{1}{2}\left[\frac{g^h\delta-g^{h-1}\delta+\sum_{i=1}^{g^{h-1}}|g^h\delta_i-\delta|}{g^h(F+\delta)}\right] \\
                       \left(1-\frac{1}{g}\right)\frac{\delta}{F+\delta} &\leq  ||\pi - \pi^*||_{\scaleto{\textbf{TV}}{3pt}} 
\end{align*}.

This lower bound is reached when all error terms in the terminal states have the same value $\delta_i = \frac{\delta}{g^h}$.

To upper bound $|g^h\delta_i-\delta|$ we apply the triangle inequality, obtaining $|g^h\delta_i-\delta| \leq g^h\delta_i+\delta$ and  $\sum_{i=1}^{g^{h-1}}|g^h\delta_i-\delta| \leq g^h\delta+g^{h-1}\delta$, from which we obtain the upper bound
\begin{align*}
    ||\pi - \pi^*||_{\scaleto{\textbf{TV}}{3pt}} &\leq \frac{1}{2}\left[\frac{g^h\delta-g^{h-1}\delta + g^{h}\delta+g^{h-1}\delta}{g^h(F+\delta)}\right] \\
    ||\pi - \pi^*||_{\scaleto{\textbf{TV}}{3pt}} &\leq \frac{\delta}{F+\delta}
\end{align*}.

To obtain a tighter bound we break the sum $\sum_{i=1}^{g^{h-1}}|g^h\delta_i-\delta|$ by partitioning the sum into the first $I$ terms $S_A=g^h\sum_{i=1}^{I}|\delta_i-\frac{\delta}{g^h}|$ with $\delta_i < \frac{\delta}{g^h}$ and subsequent $g^{h-1}-I$ terms $S_B=g^h\sum_{j=I+1}^{g^{h-1}}|\delta_j-\frac{\delta}{g^h}|$ with $\delta_j \geq \frac{\delta}{g^h}$. By construction, we know that $S_A+g^h\sum_{i=1}^{I}\delta_i+g^h\sum_{j=I+1}^{g^{h-1}}\delta_j-S_B=g^{h-1}\delta$, simplifying to $S_B-S_A=\delta(g^h-g^{h-1})$. We rewrite $S_A + S_B = S_B-S_A+2S_A=\delta(g^h-g^{h-1})+2S_A$, and by triangle inequality on $S_A$, we obtain the upper bound $\sum_{i=1}^{g^{h-1}}|g^h\delta_i-\delta|=S_A+S_B \leq g^h\delta-g^{h-1}\delta+2I\delta $. Setting $I=g^{h-1}-1$ (the biggest value it can have without breaking the constraints on $\delta_i$), it simplifies to $S_A+S_B \leq g^h\delta+g^{h-1}\delta-2\delta $

\begin{align*}
    ||\pi - \pi^*||_{\scaleto{\textbf{TV}}{3pt}} &\leq \frac{1}{2}\left[\frac{g^h\delta-g^{h-1}\delta+\sum_{i=1}^{g^{h-1}}|g^h\delta_i-\delta|}{g^h(F+\delta)}\right] \\
    ||\pi - \pi^*||_{\scaleto{\textbf{TV}}{3pt}} &\leq \frac{1}{2}\left[\frac{g^h\delta-g^{h-1}\delta+g^h\delta+g^{h-1}\delta-2\delta }{g^h(F+\delta)}\right] \\
    ||\pi - \pi^*||_{\scaleto{\textbf{TV}}{3pt}} &\leq \left[\frac{g^h\delta-\delta }{g^h(F+\delta)}\right] \\
    ||\pi - \pi^*||_{\scaleto{\textbf{TV}}{3pt}} &\leq  \left(1-\frac{1}{g^h}\right)\frac{\delta}{F+\delta}
\end{align*}.
\end{proof}

\textcolor{red}{\begin{theorem}[Total variation of the sampling distribution] Let $\delta >0$ and $\sum_{i=1}^{n} \delta_i = \delta$, where $\delta_i \in [0, \delta]$. Suppose that we have the flow network $(G_n, F)$ which generates a target distribution $\pi^*$ uniform in the number of final vertices. Then if we increase the flow $F$ by $\delta$ in the same graph, that is $(G_n, F + \delta)$, generating a new target distribution $\pi$, the total variation distance between $\pi$ and $\pi^*$ is bounded above and below by the following
\begin{align*}
& ||\pi -\pi^*||_{\scaleto{\textbf{TV}}{3pt}} \leq \epsilon(\delta, n) \quad \text{where}
\\
& \epsilon(a,b) := \Big(1 - \frac{1}{b} \Big) \frac{a}{F+a}\,.
\end{align*}
\end{theorem}}




\section{Inherent limitations of policy networks}



\begin{figure}
    \center 
    \includegraphics[scale=.3]{gflownets_wl_test.pdf} 
    \caption{A state graph whose downstream distribution is not learnable by a GFlowNet with a policy network is
    parametrized by a 1-WL GNN.} 
    \label{fig:wl_graphs} 
\end{figure}

\begin{theorem}[Distributional constraints of GFlowNets] 
    Let $\mathcal{G} = \{(\mathbf{X}, \mathbf{A}) \colon \mathbf{A} \in \{0,
    1\}^{N \times N}\}$ be the set of equally featured graphs with adjacency matrix $\mathbf{A}$
    and features $\mathbf{r} \in \mathbb{R}^{d}$ ($\mathbf{X} = \mathbf{1}\mathbf{r}^{T} \in \mathbb{R}^{N \times d}$). Let $F_{\theta} \colon \mathcal{G} \rightarrow \Delta_{2}$ be the
    \textit{policy network} that maps a graph $G \in \mathcal{G}$ to a point within the simplex of action-probabilities $\Delta_{2} =
    \{(a^{(1)}, a^{(2)}) \colon a^{(1)} + a^{(2)} = 1 \text{ and
    } a^{(1)}, a^{(2)} \ge 0\}$. See Figure~\ref{fig:wl_graphs}. Suppose that the policy network is parametrized by an 1-WL GNN with parameters $\theta$. Let $\pi$ be a distribution on the
    graphs $\{G_{i} \colon i \in \{1, 2, 3, 4\}\}$ of Figure~\ref{fig:wl_graphs} with $\pi(G_{1}) = \pi(G_{2}) = \pi(G_{3}) =
    \frac{1}{6}$ and $\pi(G_{4}) = \frac{1}{2}$. In these settings, there does not exist a
    $\theta$ such that the downstream distribution induced by the policy network equals $\pi$. 
\end{theorem}

\begin{proof}
    Let $p_{\theta}(X | S_{o})$ be the marginal transition probability learned by the GFlowNet of reaching the state $X
    \in \mathcal{G}$
    through the generative process characterized by the state graph of Figure~\ref{fig:wl_graphs} and the policy network
    $F_{\theta}$. We will show that
    $p_{\theta}(G_{i} | S_{o})$ is -- for any $\theta$ -- necessarily different of $\pi(G_{i})$ for at least two graphs
    in $\pi$'s support. 

    For this, notice that the Markovity of the stochastic transitions learned by the GFlowNet
    entails that 
    $p_{\theta}(G_{1} | S_{o}) = p_{\theta}(N_{1} | S_{o}) p_{\theta}(G_{1} | N_{1})$, 
    $p_{\theta}(G_{2} | S_{o}) = p_{\theta}(N_{1} | S_{o}) p_{\theta}(G_{2} | N_{1})$, 
    $p_{\theta}(G_{3} | S_{o}) = p_{\theta}(N_{2} | S_{o}) p_{\theta}(G_{3} | N_{2})$ and  
    $p_{\theta}(G_{4} | S_{o}) = p_{\theta}(N_{2} | S_{o}) p_{\theta}(G_{4} | N_{2})$. 
    Notably, the indistinguishability of the graphs $N_{1}$ and $N_{2}$ according to the 1-WL 
    isomorphism test implies that $F_{\theta}(N_{1}) = F_{\theta}(N_{2})$ and hence the transition  
    probabilities must satisfy $p_{\theta}(G_{1} | N_{1}) = p_{\theta}(G_{3} | N_{2})$ and $p_{\theta}(G_{2} | N_{1}) =
    p_{\theta}(G_{4} | N_{2})$. 

    Contradictorily, suppose that there is a $\theta$ such that the policy network $F_{\theta}$ is perfectly adjusted to the target distribution $\pi$.
    Hence, $p_{\theta}(G_{i} | S_{o}) = \pi(G_{i})$ for each $i \in \{1, 2, 3, 4\}$. Nonetheless, the representational equivalence of
    $N_{1}$ and $N_{2}$ and the Markovian assumption imply that 

    \begin{equation*} 
        \begin{split} 
        p_{\theta}(N_{1} | S_{o}) = \frac{\pi(G_{1})}{p_{\theta} (G_{1} | N_{1})} 
        = \frac{\pi(G_{3})}{p_{\theta} (G_{3} | N_{2})} = p_{\theta}(N_{2} | S_{o}) 
        \text{ and that } \\ 
        p_{\theta}(N_{1} | S_{o}) = \frac{\pi(G_{2})}{p_{\theta}(G_{2} | N_{1})} \neq 
        \frac{\pi(G_{4})}{p_{\theta}(G_{4} | N_{2})} = p_{\theta}(N_{2} | S_{o}).
        % p_{\theta}(N_{1} | S_{o}) = 
    \end{split} 
    \end{equation*} 

   \noindent This contradiction guarantees that $p(G_{i} | S_{o})$ is necessarily different from $\pi(G_{i})$ for at
   least a pair of graphs and asseverates that the distribution characterized by the state graph of
   Figure~\ref{fig:wl_graphs} is unlearnable by a GFlowNet parametrized by a 1-WL GNN. 
\end{proof}

\begin{remark}
    The previous theorem states the limitations of a GFlowNet parametrized by a 1-WL GNN. The alternative use of a more
    expressive yet not permutationally invariant flow parametrization would entail a factorially large increase of the
    size of the state graph, as equivalent graphs with different labelling would be treated differently by the flow
    estimator, and lead to a computationally untractable problem. The next theorem characterizes a weak relationship
    between the size of the state graph and the statistical efficiency of a maximally entropic exploratory policy within the state graph.   
\end{remark}

\subsubsection*{Author Contributions}
If you'd like to, you may include  a section for author contributions as is done
in many journals. This is optional and at the discretion of the authors.

\subsubsection*{Acknowledgments}
Use unnumbered third level headings for the acknowledgments. All
acknowledgments, including those to funding agencies, go at the end of the paper.
 

\section{Appendix} \label{appendix}


\subsection{NewYorker Data for evaluation}

\begin{figure}[!ht]
\small
\centering
\includegraphics[width=0.4\textwidth]{figures/length.png}
\caption{\label{lengthdist} Distribution of word count of stories in our test set}
\end{figure}

Table \ref{teststories} shows the data used for conducting our evaluation. The 12 stories shown are taken from The New Yorker and summarized into single-sentence plots. These stories come from highly established literary experts acting as an upper bound for what it means to be creative. These stories span complex themes.

\begin{table*}[!ht]
\centering
\small
\def\arraystretch{1.35}
\begin{tabular}{|l|}
\hline
\begin{tabular}[c]{@{}l@{}}Write a New Yorker-style story given the plot below. Make sure it is atleast \textbf{\color{blue}\{\{word\_count\}\}} words. Directly start with the\\ story, do not say things like `Here's the story {[}...{]}:\end{tabular}                                                                                                                                                                                            \\ \hline\hline
\begin{tabular}[c]{@{}l@{}}You wrote the story I gave you below. I requested a story with \textbf{\color{blue}\{\{word\_count\}\}} words, but the story only has\\ \textbf{\color{blue}\{\{current\_word\_count\}\}} words. Can you rewrite the story to make it longer, and closer to the \textbf{\color{blue}\{\{word\_count\}\}} word target\\ I gave you. Directly start with the story, do not say things like `Here's the story {[}...{]}:`\\ \\ Current story: \{\{story\}\}\end{tabular} \\ \hline
\end{tabular}
\vspace{2ex}
\caption{\label{promptstory}Prompt to write the initial story (Row1) vs Prompt to rewrite the initial story to be longer. word\_count represents the number of words in the human written story on a given plot (P) while current\_word\_count represents the number of words in the LLM generated story on the same plot (P)}
\end{table*}

\begin{table*}[!ht]
\def\arraystretch{1.15}
\small
\begin{tabular}{|l|l|}
\hline
Story                                    & Plot                                                                                                                                                                                                                                                                                                                                                                                                                                                                                                                                   \\ \hline
\href{https://www.newyorker.com/books/flash-fiction/a-triangle}{A Triangle}                               & \begin{tabular}[c]{@{}l@{}}An observer becomes entranced by a seemingly ordinary couple on the street, follows them home, and then \\watches them from outside in the rising floodwaters, drawing an eerie connection between the woman and\\ a discarded, burned chair they'd noticed earlier.\end{tabular}                                                                                                                                                                    \\ \hline\hline
\href{https://www.newyorker.com/books/flash-fiction/barbara-detroit-1966}{\begin{tabular}[c]{@{}l@{}}Barbara\\ Detroit,1966\end{tabular}}                    & \begin{tabular}[c]{@{}l@{}}On Feb 12, 1966, a heavily pregnant woman named Barbara experienced a shocking incident in her synagogue\\in Southfield, Detroit, where a young man shot and killed the renowned Rabbi Adler before turning the gun\\ on himself, and though Barbara tried to reach the shooter, she was swept away by the fleeing crowd.\end{tabular}                                                                              \\ \hline\hline
\href{https://www.newyorker.com/books/flash-fiction/beyond-nature}{Beyond Nature}                           & \begin{tabular}[c]{@{}l@{}}A solitary man walking in a remote mountainous region comes across a car crash, and stays by the side\\ of the lifeless female victim, narrating stories of his past and reflecting on the impermanence of \\events and life itself, while awaiting emergency services amidst the looming presence of wilderness.\end{tabular}                                                                                                                \\ \hline\hline
\href{https://www.newyorker.com/books/flash-fiction/certain-european-movies}{\begin{tabular}[c]{@{}l@{}}Certain European\\ Movies\end{tabular}}                  & \begin{tabular}[c]{@{}l@{}}Two individuals, at a residency together, navigate the complexity of their ephemeral relationship during\\ their final beach trip, framed by misadventures, subtle tensions, unspoken desires, and looming departures.\end{tabular}                                                                                                                                                                                   \\ \hline\hline
\href{https://www.newyorker.com/books/flash-fiction/keys}{Keys}                                     & \begin{tabular}[c]{@{}l@{}}Daniel, struggling with recurring dreams of his ex-wife Rachel and a mysterious unused flat, eventually \\discusses them with his current partner Isabel, sparking various reflections and conversations about their\\ past relationships, until a real-life discovery of old keys triggers a nostalgic memory and helps him find a\\ way to reconnect with his present relationship through canoeing.\end{tabular}                                     \\ \hline\hline
\href{https://www.newyorker.com/books/flash-fiction/listening-for-the-click}{\begin{tabular}[c]{@{}l@{}}Listening For\\ the Click\end{tabular}}                  & \begin{tabular}[c]{@{}l@{}}Navigating a complex social landscape, the protagonist experiences a series of complex relationships \\and emotional turmoil in a student environment, and engages in self-discovery and self-reflection as she\\ interacts with the characters Carl, Martin, Lizzy, and Johan, resulting in a journey of introspection,\\ betrayal, love, and personal growth.\end{tabular}                                                          \\ \hline\hline
\href{https://www.newyorker.com/magazine/2023/05/15/maintenance-hvidovre-fiction-olga-ravn}{\begin{tabular}[c]{@{}l@{}}Maintenance,\\ Hvidovre\end{tabular}}                   & \begin{tabular}[c]{@{}l@{}}A woman experiences a disorienting night in a maternity ward where she encounters other similarly \\disoriented new mothers, leading to an uncanny mix-up where she leaves the hospital with a baby \\that she realizes is not her own, yet accepts the situation with an inexplicable sense of happiness.\end{tabular}                                                                                                  \\ \hline\hline
\href{https://www.newyorker.com/magazine/2022/11/14/returns}{Returns}                                  & \begin{tabular}[c]{@{}l@{}}The narrator visits their elderly mother in her small town, spending a day with her that is filled with \\nostalgia, conversation, and old habits, only to return a month later after her hospitalization due to\\ a sunstroke, finding remnants of their last visit.\end{tabular}                                                                                                                                                                      \\ \hline\hline
\href{https://www.newyorker.com/books/flash-fiction/the-facade-renovation-thats-going-well}{\begin{tabular}[c]{@{}l@{}}The Facade \\Renovation\\ That’s Going Well\end{tabular}} & \begin{tabular}[c]{@{}l@{}}An academic faculty housed in a building with a critical waterproofing layer missing experiences a series\\ of disruptive and problematic construction repairs, causing tension, inconvenience, and health concerns\\ among the tenants, ultimately leading to resignation and endurance in hopes of better future circumstances.\end{tabular}                                                        \\ \hline\hline
\href{https://www.newyorker.com/books/flash-fiction/the-kingdom-that-failed}{\begin{tabular}[c]{@{}l@{}}The Kingdom\\ That Failed\end{tabular}}                  & \begin{tabular}[c]{@{}l@{}}The narrator recounts their college friendship with the seemingly flawless Q, and after a decade apart, \\they accidentally cross paths at a pool, where the narrator anonymously observes Q's failed attempt to \\let down a woman about a work-related issue, demonstrating that Q, too, has his share of difficulties.\end{tabular}                                                                                                \\ \hline\hline
\href{https://www.newyorker.com/magazine/2022/06/13/trash }{Trash}                                    & \begin{tabular}[c]{@{}l@{}}A woman unexpectedly marries the son of a successful, ambitious woman named Miss Emily, finding both \\acceptance and critique from her mother-in-law as she navigates this new relationship and confronts the \\stark contrasts between her former life as a supermarket cashier and her new life as part of a well-off family.\end{tabular}                                                                                                            \\ \hline\hline
\href{https://www.newyorker.com/culture/personal-history/the-last-dance-with-my-dad}{\begin{tabular}[c]{@{}l@{}}The Last Dance\\ with my Dad \end{tabular}}               & \begin{tabular}[c]{@{}l@{}}A young teenager recounts her experiences of fitting into her father's gay lifestyle, highlighted by a\\ seven-day cruise with hundreds of gay men, where she experienced acceptance and connection, had her\\ first genuine interaction with a  boy, and shared a last dance with her terminally ill father.\end{tabular}                                                                                                       \\ \hline
\end{tabular}
\vspace{2ex}
\caption{\label{teststories} Expert-written short stories from the New Yorker along with their human-verified GPT4 generated summary as plots that are included as part of our test data for Creativity Evaluation}
\end{table*}


\subsection{Expert Perception on the TTCW tests}

\begin{figure*}[!ht]
    \centering
     \includegraphics[width=\textwidth]{figures/rel.pdf}
    \caption{\label{relev} Relative Evaluation by Creative Writing Experts within a given group of four stories}
\end{figure*}

\begin{table*}[!ht]
\small
\centering
\begin{tabular}{|l|l|}
\hline
E5 & \begin{tabular}[c]{@{}l@{}}It was a pretty effective rubric! I'm used to being more subjective in my work -- did you like a story? Did it connect with \\you? Did it make sense? Why or why not? It was often challenging to break it down into more regimented segments \\like the rubric asked for -- but I do think that it allowed me to express the subjective feelings in a pretty thorough and\\ structured way!\end{tabular}                                                                                                                                                                 \\ \hline
E3 & \begin{tabular}[c]{@{}l@{}}As for the rubric, I thought it was quite thorough. There were some categories where I would say the story didn’t ``pass,"\\ but which were excellent. This happened often with the categories about multiple points of view, and innovative\\ structure and form. Overall, I think the rubric was helpful in helping me think about the different aspects of storytelling.\end{tabular}                                                                                                                                                                                 \\ \hline
E4 & \begin{tabular}[c]{@{}l@{}}I thought the rubric felt pretty thorough; the only part I felt could be added was that suggestion about consistency in\\ voice \& diction!\end{tabular}                                                                                                                                                         \\ \hline
E2 & \begin{tabular}[c]{@{}l@{}}The rubric seemed great to me! It’s however hard to talk about something like pacing without talking about scene and \\summary, for instance. Or the difference between originality of thought and originality in theme/content—wouldn’t the \\latter make up the former and vice/versa? But it is also comprehensive and I can see the merits of this sort of repetition in\\ teasing out a fuller picture of things\end{tabular} \\ \hline
E1 & \begin{tabular}[c]{@{}l@{}}I thought the rubric was pretty good tbh. I think there is overlap in some of the different elements, like "language \\proficiency \& literary devices" and "originality in thought." it's tricky to use words like "satisfying" and "sophisticated" \\when assessing art, but there's always going to be a great deal of subjectivity in these matters.I think that voice is a crucial \\aspect of high-quality writing that is being overlooked by the rubric, and one that greatly informs how I as a reader\\ evaluate 
and appreciate literary writing.\end{tabular} \\ \hline
\end{tabular}
\vspace{2ex}
\caption{\label{expertfeedbackrubric}Expert perception and feedback on the TTCW tests they conducted as part of our data collection.}
\end{table*}

Since the experts listed in Table \ref{creativeexperts} were not involved in designing the rubric but evaluated several stories based on the rubric we asked them their \textit{overall thought about the rubric and any potentially crucial test we missed out on that they use to discriminate between good and bad writing}.As can be seen in Table \ref{expertfeedbackrubric} in Appendix overall almost every expert agreed on the thorough and effective nature of our rubric. Many of them agreed on the fact that our rubric helped them to think about different aspects of storytelling in a more structured way. One of the difficult things about coming up with a rubric for creativity is ensuring coverage. Even though our rubric covers most aspects of creative writing, some experts such as E1 and E4 emphasized on the utility of \textbf{Consistency of Voice and Diction} as a measurable test. In E4's words \textit{``Inconsistent voice and diction are sometimes/often notable in stories that aren't very good, and when voice \& diction are used beautifully, it enhances a story considerably"}. E1 similarly exclaimed \textit{``One of the most meaningful aspects of high-quality literary writing is voice, which conveys qualities of proficiency, artistry, personality, and identity."}. We hope future work can adapt this as a meaningful test in addition to the tests covered in our rubric. Finally, some of the tests from our rubric can have potential overlaps as pointed out by E2. This is further corroborated by the similar numbers for \textit{Narrative Pacing} and \textit{Scenes vs Exposition} suggesting a strong correlation between the two.
\begin{table*}[!ht]
\small
\centering
% \def\arraystretch{1.3}
\begin{tabular}{|l|l|l|}
\hline
Test & Passing Stories & Failing Stories \\ \hline
\begin{tabular}[c]{@{}l@{}}Originality in\\ Form\end{tabular} & \begin{tabular}[c]{@{}l@{}}Inventive techniques like time jumping, varied \\ perspectives, unconventional punctuation, and\\ delayed revelation of key information\end{tabular} & \begin{tabular}[c]{@{}l@{}}Conventional and linear in its form, language, \\ and narrative, with occasional attempts at \\ innovation that do not significantly contribute to \\ its overall originality or creativity\end{tabular} \\ \hline
\begin{tabular}[c]{@{}l@{}}Originality in\\ Thought\end{tabular} & \begin{tabular}[c]{@{}l@{}}Fresh language, unique plot and characters, subtle\\ emotional resonance, and inventive metaphors. Minor \\ familiar elements, but do not undermine the overall \\ sense of imagination and distinctiveness\end{tabular} & \begin{tabular}[c]{@{}l@{}}Stories relies heavily on cliches \& tired tropes.\\ Language does not feel fresh or original with \\ narrative arc following a predictable trajectory.\\ Metaphors, descriptions, and overall premise \\ cover familiar ground that lacks novelty or nuance\end{tabular} \\ \hline
\begin{tabular}[c]{@{}l@{}}Originality in\\ Theme/Content\end{tabular} & \begin{tabular}[c]{@{}l@{}}Unconventional, dreamlike exploration of emotions\\ such as love and loss, evoking empathy and reflection\\ through its distinct main character perspective, \\ eschewing simplistic meanings for ambiguity, and \\ valuing open-ended questions over singular messages,\\ thus providing a unique reading experience compared\\ to conventional stories.\end{tabular} & \begin{tabular}[c]{@{}l@{}}Disjointed narrative, underdeveloped themes, \\ inconsistent tone, vaguely defined characters, and\\ abrupt context shifts, lack depth and fail to provide \\ substantive insight or originality to the reader.\end{tabular} \\ \hline\hline
\begin{tabular}[c]{@{}l@{}}World Building\\ and Setting\end{tabular} & \begin{tabular}[c]{@{}l@{}}Strategic use of concrete, specific sensory details from\\ a particular character’s perspective balances narrative\\ momentum, making a fictional world feel real, vivid\\ and immersive for readers. Thoughtful depiction of\\ everyday objects, and idiosyncratic elements within\\ narrative and dialogue to balance exposition with \\ vivid scene-setting, creating authenticity and realism \\ that serves the plot and characters\end{tabular} & \begin{tabular}[c]{@{}l@{}}Fictional world is not always convincingly \\established through sensory details and language. \\Stories rely too heavily on overwrought imagery\\ and figurative language without grounding \\the reader in a tangible reality.\end{tabular} \\ \hline
\begin{tabular}[c]{@{}l@{}}Character\\ Development\end{tabular} & \begin{tabular}[c]{@{}l@{}}Fully realized characters with contradictions, \\ motivations, and backstories that make them\\ feel lifelike. Flatter, less developed characters\\ that feel appropriate for the narrative goals \\ and style is not necessarily a weakness\end{tabular} & \begin{tabular}[c]{@{}l@{}}Characters not well rounded. easily resorting to \\stereotypes. Predictable arcs not making them\\memorable. Actions or motivations unclear leading \\to disconnect\end{tabular} \\ \hline
\begin{tabular}[c]{@{}l@{}}Rhetorical\\ Complexity\end{tabular} & \begin{tabular}[c]{@{}l@{}}Rich subtext that emerges through contrasts between\\ characters and settings. Omissions that let readers \\ fill in meaning, metaphors with layered significance, \\ implicit characterization, ambiguous endings, \\ intertextual references, and rich language that goes\\ beyond straightforward storytelling\end{tabular} & \begin{tabular}[c]{@{}l@{}}Straightforward storytelling, lack of complexity,\\ nuance, and ambiguity.The absence of context and \\ ironic tensions, overtelling instead of showing, \\ Use of expository dialogue, blunt symbolism, \\ predictable characters, and definitive endings, all \\of which limit interpretation and undermine\\ their potential resonance\end{tabular} \\ \hline
\end{tabular}
\vspace{2ex}
\caption{\label{expertexpl2}Common themes and issues found in expert explanations for tests focusing on TTCW-Originality and TTCW-Elaboration}
\vspace{-8ex}
\end{table*}

\subsection{Common themes in TTCW of Originality and Elaboration}


Table~\ref{expertexpl2} provides common themes and issues found in the expert explanations for each TTCW in the Originality and Elaboration dimensions.


\subsection{Example LLM-generated and expert-written explanations for a TTCW assessment} 

\begin{table*}[!ht]
\centering
\small
\def\arraystretch{1.35}
\begin{tabular}{|l|l|}
\hline
GPT4 & \begin{tabular}[c]{@{}l@{}}The story incorporates numerous instances of subtext that serve to further develop the characters' internal emotions \\ and relationships, as well as accentuate the atmospheric tension and undertones of longing. The subtext, which is\\ subtly weaved into surface-level interactions, actions, and observations, is eloquently crafted and does not feel \\ forced, enriching both the narrative depth and setting.\\ One key passage with evident subtext is when he is in the ocean, swimming and jumping off the rock, while she\\ observes from the towel. The surface-level interpretation suggests a casual, relaxed beach day. However, the \\ subtext is rife with allusions of emotional undercurrents in their interaction. His unawareness of her going \\ into the water parallels the emotional disconnect between them, insinuating the vast, metaphorical space that \\ separates their respective experiences and understanding of their relationship.\\ Another instance of subtext occurs when he takes a photo of her with her swimsuit top off, indicating his attraction\\ to her. But the meaning deepens when we understand that he will delete the photo before going home, implicitly\\ revealing his guilt or fear of being discovered, and again highlighting the transient, secret nature of their relationship.\\ The line, "As in certain European movies" suggests the fleeting, impermanent element of their relationship and the\\ eventual abruptness and intensity of feelings that often feature in said films. The 'European movies' serve as a \\ metaphor for the unexpected and sudden end of their encounter. \\ In these examples, the subtext does not feel forced but instead flows naturally from the characters and \\ circumstances, subtly conveying deeper meanings that heighten both the narrative tension and emotional depth.\\ \\ So Yes.\end{tabular} \\ \hline
E3   & \begin{tabular}[c]{@{}l@{}}There is rich subtext, as the main character seems continually conflicted about whether she wants to be where she is, \\ doing what she is doing. On the surface, she is carefree, riding to the beach with the guy she met, skipping the ceramics\\ and the museum, and whatever else. And yet, she is unhappy and unsatisfied, longing for a beer, imagining that if their\\ relationship continued they would only hate each other. This tension is maintained throughout the story.\end{tabular}                                                                                                                                                                                                                                                                                                               \\ \hline
E1   & \begin{tabular}[c]{@{}l@{}}This piece has an iceberg of subtext floating underneath it. The entire story is conveyed through the successful \\ integration of subtext and text. The interactions between the protagonist and the man (Did you see me jump of the \\ rock? No, she hadn't.Did he notice she had gone in the water too, that her hair was dripping? No, he hadn't.)convey\\ a profound disconnect that causes the reader to wonder why the protagonist continues to suffer the presence of this\\ man she clearly disdains and seems to view as an incompetent man-child.\end{tabular}
               \\ \hline
E7   & \begin{tabular}[c]{@{}l@{}}Yes!!!!! Again, the idea of the story was fairly simple (the inevitability of age, parting, change), but it was illustrated\\ in a way that felt inspiring re: questioning how these ideas relate and resonate throughout our own lives. It was really \\ beautiful and I was left feeling changed at the end of it :)\end{tabular}                                                                                                                                                                                                                                                            \\ \hline
\end{tabular}
\vspace{2ex}
\caption{\label{llmvsexpertexpl}LLM explanation vs expert explanation for Rhetorical Complexity}
\end{table*}

In Table~\ref{llmvsexpertexpl}, we show examples of explanations that experts wrote in conjunction with a binary TTCW assessment they made on a story, as well as the corresponding LLM-generated explanations.

\subsection{Can non-experts administer TTCW tests?}

Recruiting experts for data annotation purposes is challenging, and costly, and must consider the time constraint put on the experts. Prior work has shown the potential of crowd-sourcing (through platforms such as Amazon Mechanical Turk) and the ability of non-experts to accomplish complex tasks as a crowd \cite{kittur2013future}, when following an appropriate workflow that iterates and validates the work on individual non-experts. Some prior work has even shown the validity of crowd-based feedback for writing tasks \cite{bernstein2010soylent,nebeling2016wearwrite}. 

In this work, we chose to rely on experts for annotation, to maximize the validity of our experiments, and confirm whether experts with domain knowledge would reach satisfying agreement levels when evaluating stories with TTCW. Future work can leverage our open-sourced annotations to explore whether non-experts correlate with experts when performing TTCW evaluation, which could lead to more cost-effective TTCW evaluation.

\subsection{Prompts for TTCW} \label{allprompts}

All the instructions shown to creative writing experts and LLMs are given in the tables below.


\begin{table*}[!ht]
\centering
\small
\begin{tabular}{|l|l|}
\hline
\begin{tabular}[c]{@{}l@{}}Expert \\ Measure\end{tabular}               & Does the manipulation of time in terms of compression or stretching feel appropriate and balanced?                                                                                                                                    \\ \hline
\begin{tabular}[c]{@{}l@{}}Expanded\\ Expert\\ Measure (M)\end{tabular} & \begin{tabular}[c]{@{}l@{}}`Compression/stretching of time' in fiction writing, also known as pacing, refers to the manipulation of time in \\storytelling for dramatic effect, pacing, or other narrative purposes. Essentially, it's about controlling the perceived \\speed and rhythm at which a story unfolds.\\ \\

Compression of time refers to when events that take a long time (hours, days, weeks, or even years) are summarized \\or condensed into a brief narrative span. For example, a writer might compress several years of a character's life \\into a few paragraphs to quickly convey important changes or developments.\\ \\

On the other hand, stretching of time is when a brief moment or event is drawn out over pages or chapters. It's often \\used to create suspense, emphasize details, or delve deeper into a character's thoughts and feelings. For example, \\the few seconds it takes for a dropped glass to hit the floor might be stretched out with detailed descriptions of the\\ action, reactions, and thoughts of characters involved.\\ \\

Storytime refers to the time within the world of the story, while real-world time refers to the time it takes for the \\reader to read the story. A skilled writer can manipulate the relationship between these two to affect the pacing of \\the narrative, either speeding it up (compression) or slowing it down (stretching). This technique plays a crucial role \\in shaping the reader's experience and engagement with the story.\end{tabular} \\ \hline
\begin{tabular}[c]{@{}l@{}}Human\\ Instruction\end{tabular}             & \begin{tabular}[c]{@{}l@{}}\{\{M\}\}\\ \\ Based on the story that you just read, answer the following question.\\ \textit{\color{blue}Does the manipulation of time in terms of compression or stretching feel appropriate and balanced?}\\ -Yes \\ -No \\\\ Reasoning : \end{tabular}                                                                       \\ \hline
\begin{tabular}[c]{@{}l@{}}LLM\\ Instruction\end{tabular}               & \begin{tabular}[c]{@{}l@{}}\{\{M\}\}\\ \\ Given the story above, list out the scenes in the story in which time compression or time stretching is used, and \\argue for each whether it is successfully implemented.  Then overall, give your reasoning about the question below \\and give an answer to it between 'Yes' or 'No' only \\ \\ \textit{\color{blue} Q) Does the manipulation of time in terms of compression or stretching feel appropriate and balanced?}\end{tabular}                                                                                                                                                                                                                    \\ \hline
\end{tabular}
\vspace{2ex}
\caption{\label{prompting}TTCW Fluency1 (Narrative Pacing) }
\vspace{-5ex}
\end{table*}


% ==================================================





\begin{table*}[!ht]
\centering
\small
% \def\arraystretch{1.15}
\begin{tabular}{|l|l|}
\hline
\begin{tabular}[c]{@{}l@{}}Expert \\ Measure\end{tabular}               & \begin{tabular}[c]{@{}l@{}}Does the story have an appropriate balance between scene and summary/exposition or it relies on one\\ of the elements heavily compared to the other?  \end{tabular}                                                                                                                                  \\ \hline
\begin{tabular}[c]{@{}l@{}}Expanded\\ Expert\\ Measure (M)\end{tabular} & \begin{tabular}[c]{@{}l@{}}'Scene' and 'summary/exposition' are two crucial elements of narrative storytelling, and balancing them \\appropriately is an important skill in fiction writing.\\ \\ 

A 'scene' is a moment in the story that is dramatized in real-time. Scenes are usually vivid and engaging, often \\featuring character interaction, dialogue, and action. They are the building blocks of the plot, and through them, \\the story unfolds.\\ \\ 

'Summary' or 'exposition', on the other hand, involves summarizing events or providing information. Instead of \\unfolding in real time, \\summaries compress time and tell the reader what happened. Exposition provides \\necessary background information, like character history, setting details, or prior events. \\ \\ 

A good writer knows when to use scenes to make the story come alive, show character development, or increase \\tension. They also know when to use summary or exposition to move the story forward, fill in background \\information, or bridge gaps between important scenes. \\ \\ 

The right balance between scene and summary/exposition can vary depending on the story, but in general, it's \\essential for maintaining a good pace, keeping the reader engaged, and delivering necessary information. \\A story with too many scenes and not enough summary might feel overwhelming or slow, while a story with \\too much exposition and not enough scenes could feel dry and unengaging.\end{tabular} \\ \hline
\begin{tabular}[c]{@{}l@{}}Human\\ Instruction\end{tabular}             & \begin{tabular}[c]{@{}l@{}}\{\{M\}\}\\ \\ Based on the story that you just read, answer the following question.\\ \textit{\color{blue} Does the story have an appropriate balance between scene and summary/exposition or it relies on one of the elements} \\\textit{\color{blue}heavily compared to the other?} \\ -Yes \\ -No \\\\ Reasoning : \end{tabular}    
\\ \hline
\begin{tabular}[c]{@{}l@{}}LLM\\ Instruction\end{tabular}               & \begin{tabular}[c]{@{}l@{}}\{\{M\}\}\\ \\ Given the story above, answer the following question. Please first explain your reasoning step by step \\and then given an answer between 'Yes' or 'No' only \\ \\ \textit{\color{blue} Does the story have an appropriate balance between scene and summary/exposition or it relies on one of the elements} \\\textit{\color{blue}heavily compared to the other?}\end{tabular}                                                                                                                                                                                                                    \\ \hline
\end{tabular}
\vspace{2ex}
\caption{\label{prompting}TTCW Fluency2 (Scene vs Exposition) }
\vspace{-5ex}
\end{table*}


% ==================================================


\begin{table*}[!ht]
\centering
\small
% \def\arraystretch{1.15}
\begin{tabular}{|l|l|}
\hline
\begin{tabular}[c]{@{}l@{}}Expert \\ Measure\end{tabular}               & Does the story make sophisticated use of idiom or metaphor or literary allusion?                                                                                                                                     \\ \hline
\begin{tabular}[c]{@{}l@{}}Expanded\\ Expert\\ Measure (M)\end{tabular} & \begin{tabular}[c]{@{}l@{}}`Idiom' refers to phrases or expressions that have a figurative, or sometimes literal, meaning that is \\comprehensible to a particular group of people. These can be cultural, regional, or specific to a certain group or \\profession.Sophisticated use of idiom suggests that the writer is skillfully using these expressions to lend \\authenticity to character voices or to convey specific meanings in a concise way.\\\\

`Metaphor' is a figure of speech that describes an object or action in a way that isn't literally true, but helps explain\\ an idea or make a comparison. Sophisticated use of metaphor suggests the
writer could create impactful, original \\comparisons that reveal deeper insights about themes,
characters, or situations in the story.\\\\

`Literary allusion' refers to a brief and indirect reference to a person, place, thing or idea of
historical, cultural,\\ literary, or political significance. It does not describe in detail the person or thing to which it refers. A sophisticated\\ use of literary allusion implies the writer can effectively incorporate these references to enhance the depth\\ and resonance of the story. They can provide contextual richness, evoke a specific tone, or draw parallels between\\ the narrative and the work alluded to.\\\\

Overall, when a writer uses these techniques well, they add depth, interest, and nuanced \\meaning
to their work. It allows for a richer reading experience, where the literal events are \\imbued with deeper symbolic or thematic significance.\end{tabular} \\ \hline
\begin{tabular}[c]{@{}l@{}}Human\\ Instruction\end{tabular}             & \begin{tabular}[c]{@{}l@{}}\{\{M\}\}\\ \\ Based on the story that you just read, answer the following question.\\ \textit{\color{blue}Does the story make sophisticated use of idiom or metaphor or literary allusion?}\\ -Yes \\ -No \\\\ Reasoning: \end{tabular}                                                                       \\ \hline
\begin{tabular}[c]{@{}l@{}}LLM\\ Instruction\end{tabular}               & \begin{tabular}[c]{@{}l@{}}\{\{M\}\}\\ \\ Given the story above, please list out all the metaphors, idioms and literary allusions, and for each decide \\whether it is successful vs it feels forced or too easy.  Then overall, give your reasoning about the question \\below and give an answer to it between 'Yes' or 'No' only\\ \\ \textit{\color{blue} Q) Does the story make sophisticated use of idiom or metaphor or literary allusion?}\end{tabular}                                                                                                                                                                                                                    \\ \hline
\end{tabular}
\vspace{2ex}
\caption{\label{prompting}TTCW Fluency3 (Language Proficiency \& Literary Devices) }
\vspace{-5ex}
\end{table*}


% ==================================================



\begin{table*}[!ht]
\centering
\small
% \def\arraystretch{1.15}
\begin{tabular}{|l|l|}
\hline
\begin{tabular}[c]{@{}l@{}}Expert \\ Measure\end{tabular}               & Does the end of the story feel natural and earned, as opposed to arbitrary or abrupt?                                                                                                                                    \\ \hline
\begin{tabular}[c]{@{}l@{}}Expanded\\ Expert\\ Measure (M)\end{tabular} & \begin{tabular}[c]{@{}l@{}}If the writer ends the piece simply because they are 'tired of writing', the conclusion might feel abrupt, disjointed, \\or unfulfilling to the reader. It suggests a rushed ending, where plot threads might be left unresolved and character \\arcs incomplete.\\ \\ 

Conversely, if the writer concludes because they've reached `the moment the entire piece has been leading readers \\towards', it implies a well-considered and purposeful ending. The events, character development, and themes \\throughout the story have built towards this climactic moment, providing a satisfying resolution to the reader.\\ \\ 

A strong ending offers a sense of closure, ties up the central conflicts or questions of the story, and generally \\leaves the reader feeling that the narrative journey was worthwhile and complete.\end{tabular} \\ \hline
\begin{tabular}[c]{@{}l@{}}Human\\ Instruction\end{tabular}             & \begin{tabular}[c]{@{}l@{}}\{\{M\}\}\\ \\ Based on the story that you just read, answer the following question.\\ \textit{\color{blue}Does the end of the story feel natural and earned, as opposed to arbitrary or abrupt?}\\ -Yes \\ -No \\\\ Reasoning : \end{tabular}                                                                       \\ \hline
\begin{tabular}[c]{@{}l@{}}LLM\\ Instruction\end{tabular}               & \begin{tabular}[c]{@{}l@{}}\{\{M\}\}\\ \\ Given the story above, answer the following question. Please first explain your reasoning step by step \\ and then given an answer between 'Yes' or 'No' only\\ \\ \textit{\color{blue} Q) Does the end of the story feel natural and earned, as opposed to arbitrary or abrupt?}\end{tabular}                                                                                                                                                                                                                    \\ \hline
\end{tabular}
\vspace{2ex}
\caption{\label{prompting}TTCW Fluency4 (Narrative Ending) }
\vspace{-5ex}
\end{table*}



% ==================================================



\begin{table*}[!ht]
\centering
\small
% \def\arraystretch{1.15}
\begin{tabular}{|l|l|}
\hline
\begin{tabular}[c]{@{}l@{}}Expert \\ Measure\end{tabular}               & Do the different elements of the story work together to form a unified, engaging, and satisfying whole?                                                                                                                                     \\ \hline
\begin{tabular}[c]{@{}l@{}}Expanded\\ Expert\\ Measure (M)\end{tabular} & \begin{tabular}[c]{@{}l@{}}A well-crafted story usually follows a logical path, where the events in the beginning set up the middle, which then\\ logically leads to the end. Every scene, character action, and piece of dialogue should serve the story and propel it \\forward. Well-written stories have an underlying the unity that binds the elements together. The themes, plotlines, \\character arcs, and other elements of the story interweave to create a harmonious whole. A story with 'disorder'\\ might feel disjointed, with characters, scenes, etc that don't connect or contribute to the overall narrative.\end{tabular} \\ \hline
\begin{tabular}[c]{@{}l@{}}Human\\ Instruction\end{tabular}             & \begin{tabular}[c]{@{}l@{}}\{\{M\}\}\\ \\ Based on the story that you just read, answer the following question.\\ \textit{\color{blue}Do the different elements of the story work together to form a unified, engaging, and satisfying whole?}\\ -Yes \\ -No \\\\ Reasoning : \end{tabular}                                                                       \\ \hline
\begin{tabular}[c]{@{}l@{}}LLM\\ Instruction\end{tabular}               & \begin{tabular}[c]{@{}l@{}}\{\{M\}\}\\ \\ Given the story above, answer the following question. Please first explain your reasoning step by step and then \\give an answer between 'Yes' or 'No' only\\ \\ \textit{\color{blue} Q) Do the different elements of the story work together to form a unified, engaging, and satisfying whole?}\end{tabular}                                                                                                                                                                                                                                 \\ \hline
\end{tabular}
\vspace{2ex}
\caption{\label{prompting}TTCW Fluency5 (Understandability \& Coherence) }
\vspace{-5ex}
\end{table*}


% ==================================================



\begin{table*}[!ht]
\centering
\small
% \def\arraystretch{1.15}
\begin{tabular}{|l|l|}
\hline
\begin{tabular}[c]{@{}l@{}}Expert \\ Measure\end{tabular}               & \begin{tabular}[c]{@{}l@{}}Does the story provide diverse perspectives, and if there are unlikeable characters, are their perspectives \\presented convincingly and accurately? \end{tabular}                                                                                                                                     \\ \hline
\begin{tabular}[c]{@{}l@{}}Expanded\\ Expert\\ Measure (M)\end{tabular} & \begin{tabular}[c]{@{}l@{}}A good writer can convincingly and accurately depict a wide range of character viewpoints, including those of\\ characters who may be morally ambiguous, difficult, or otherwise unappealing.\\ \\ 

This can involve diving into the mindset of characters who may act or think in ways that the reader, or even \\the writer, finds objectionable or repugnant. It involves understanding their motivations, their beliefs, and the \\reasons behind their actions, and then conveying these elements in a way that is believable and consistent.\\ \\ 

The purpose of doing so is not to justify or endorse these perspectives, but rather to create complex, three-\\dimensional characters who contribute to the richness and depth of the story. This can also serve to \\challenge the reader, provoke thought, and provide insights into different aspects of the human experience.\end{tabular} \\ \hline
\begin{tabular}[c]{@{}l@{}}Human\\ Instruction\end{tabular}             & \begin{tabular}[c]{@{}l@{}}\{\{M\}\}\\ \\ Based on the story that you just read, answer the following question.\\ \textit{\color{blue}Does the story provide diverse perspectives, and if there are unlikeable characters, are their perspectives presented} \\ \textit{\color{blue}convincingly and accurately?}\\ -Yes \\ -No \\\\ Reasoning : \end{tabular}                                                                       \\ \hline
\begin{tabular}[c]{@{}l@{}}LLM\\ Instruction\end{tabular}               & \begin{tabular}[c]{@{}l@{}}\{\{M\}\}\\ \\ Given the story above, answer the following question. Please first explain your reasoning step by step and then \\give an answer between 'Yes' or 'No' only\\ \\ \textit{\color{blue} Q) Does the story provide diverse perspectives, and if there are unlikeable characters, are their perspectives presented}\\\textit{\color{blue} convincingly and accurately?}\end{tabular}                                                                                                                                                                                                                                 \\ \hline
\end{tabular}
\vspace{2ex}
\caption{\label{prompting}TTCW Flexibility1 (Perspective \& Voice Flexibility) }
\vspace{-5ex}
\end{table*}


% ==================================================




\begin{table*}[!ht]
\centering
\small
% \def\arraystretch{1.15}
\begin{tabular}{|l|l|}
\hline
\begin{tabular}[c]{@{}l@{}}Expert \\ Measure\end{tabular}               & \begin{tabular}[c]{@{}l@{}}Does the story achieve a good balance between interiority and exteriority, in a way that feels \\emotionally flexible? \end{tabular}                                                                                                                                     \\ \hline
\begin{tabular}[c]{@{}l@{}}Expanded\\ Expert\\ Measure (M)\end{tabular} & \begin{tabular}[c]{@{}l@{}}`Emotional flexibility' is asking whether the piece of writing effectively balances action and introspection, and \\if it portrays a broad and realistic spectrum of emotions.\\ \\

`Exteriority' refers to the observable actions, behaviors, or dialogue of a character, and the physical or visible \\aspects of the setting, plot, and conflicts.\\ \\

`Interiority', on the other hand, pertains to the inner life of a character — their thoughts, feelings, memories, \\and subjective experiences.\\ \\

A balance between these two aspects is crucial in creating well-rounded characters and compelling narratives. \\If a piece is too heavy on exteriority, it may feel shallow or lack emotional depth. If it leans too much on\\ interiority, it could become overly introspective and potentially lose the momentum of the plot.
\end{tabular} \\ \hline
\begin{tabular}[c]{@{}l@{}}Human\\ Instruction\end{tabular}             & \begin{tabular}[c]{@{}l@{}}\{\{M\}\}\\ \\ Based on the story that you just read, answer the following question.\\ \textit{\color{blue}Does the story achieve a good balance between interiority and exteriority, in a way that feels emotionally flexible?}\\ -Yes \\ -No \\\\ Reasoning : \end{tabular}                                                                       \\ \hline
\begin{tabular}[c]{@{}l@{}}LLM\\ Instruction\end{tabular}               & \begin{tabular}[c]{@{}l@{}}\{\{M\}\}\\ \\ Given the story above, answer the following question. Please first explain your reasoning step by step and \\then give an answer between 'Yes' or 'No' only\\ \\ \textit{\color{blue}Q) Does the story achieve a good balance between interiority and exteriority, in a way that feels} \\\textit{\color{blue}emotionally flexible?}\end{tabular}                                                                                                                                                                                                                                 \\ \hline
\end{tabular}
\vspace{2ex}
\caption{\label{prompting}TTCW Flexibility2 (Emotional Flexibility) }
\vspace{-5ex}
\end{table*}


% ==================================================




\begin{table*}[!ht]
\centering
\small
% \def\arraystretch{1.15}
\begin{tabular}{|l|l|}
\hline
\begin{tabular}[c]{@{}l@{}}Expert \\ Measure\end{tabular}               & \begin{tabular}[c]{@{}l@{}}Does the story contain turns that are both surprising and appropriate? \end{tabular}                                                                                                                                     \\ \hline
\begin{tabular}[c]{@{}l@{}}Expanded\\ Expert\\ Measure (M)\end{tabular} & \begin{tabular}[c]{@{}l@{}}`Surprising': This refers to the element of unpredictability in a narrative. A good story often has plot twists, \\character developments, or thematic revelations that surprise the reader, subverting their expectations in a \\thrilling way.It's about keeping readers engaged and curious, never fully knowing what's going to happen next.\\ \\ 

`Appropriate': Despite the surprises and twists, the turns in the story must also make sense within the established \\context of the story's universe, its characters, and its themes. This means that even though an event might be \\surprising, it should feel appropriate or fitting in hindsight. It shouldn't feel like the writer has broken the rules \\they've set up, or made a character behave inconsistently without reason, simply for the sake of shock value.\\ \\ 

So when someone wonders if a writer can make turns that are 'both surprising and appropriate', they're asking \\if the writer can strike this balance between unexpectedness and coherence, keeping the reader on their toes \\while maintaining a believable, satisfying narrative. \end{tabular} \\ \hline
\begin{tabular}[c]{@{}l@{}}Human\\ Instruction\end{tabular}             & \begin{tabular}[c]{@{}l@{}}\{\{M\}\}\\ \\ Based on the story that you just read, answer the following question.\\ \textit{\color{blue}Does the story contain turns that are both surprising and appropriate?}\\ -Yes \\ -No \\\\ Reasoning: \end{tabular}                                                                       \\ \hline
\begin{tabular}[c]{@{}l@{}}LLM\\ Instruction\end{tabular}               & \begin{tabular}[c]{@{}l@{}}\{\{M\}\}\\ \\ Given the story above, list each element in the story that is intended to be surprising. For each, decide whether the\\ surprising element remains appropriate with respect to the entire story. Then overall, give your reasoning \\about the question below and give an answer to it between 'Yes' or 'No' only\\ \\ \textit{\color{blue} Q) Does the story contain turns that are both surprising and appropriate?}\end{tabular}                                                                                                                                                                                                                                 \\ \hline
\end{tabular}
\vspace{2ex}
\caption{\label{prompting}TTCW Flexibility3 (Structural Flexibility) }
\vspace{-5ex}
\end{table*}


% ==================================================






\begin{table*}[!ht]
\centering
\small
% \def\arraystretch{1.15}
\begin{tabular}{|l|l|}
\hline
\begin{tabular}[c]{@{}l@{}}Expert \\ Measure\end{tabular}               & \begin{tabular}[c]{@{}l@{}}Will an average reader of this story obtain a unique and original idea from reading it? \end{tabular}                                                                                                                                     \\ \hline
\begin{tabular}[c]{@{}l@{}}Expanded\\ Expert\\ Measure (M)\end{tabular} & \begin{tabular}[c]{@{}l@{}}If a story is good, the reader gains new insights, perspectives, or knowledge from it. This doesn't necessarily\\ mean factual information but could relate to a deeper understanding of human nature, cultural insights,\\ unique viewpoints, or even the exploration of new ideas and themes. Essentially, it's about what\\ the reader takes away from the story beyond just the plot.\\ \\ 

A good story has lasting impacts on its readers and the society. It is meant to entertain, inform, provoke \\thought, challenge beliefs, provide comfort, or raise awareness on specific issues.
 \end{tabular} \\ \hline
\begin{tabular}[c]{@{}l@{}}Human\\ Instruction\end{tabular}             & \begin{tabular}[c]{@{}l@{}}\{\{M\}\}\\ \\ Based on the story that you just read, answer the following question.\\ \textit{\color{blue}Will an average reader of this story obtain a unique and original idea from reading it?}\\ -Yes \\ -No \\\\ Reasoning : \end{tabular}                                                                       \\ \hline
\begin{tabular}[c]{@{}l@{}}LLM\\ Instruction\end{tabular}               & \begin{tabular}[c]{@{}l@{}}\{\{M\}\}\\ \\ Given the story above, list out elements that are unique takeaways of this story for the reader. Then overall, \\give your reasoning about the question below and give an answer to it between 'Yes' or 'No' only\\ \\ \textit{\color{blue} Q) Will an average reader of this story obtain a unique and original idea from reading it?}\end{tabular}                                                                                                                                                                                                                                 \\ \hline
\end{tabular}
\vspace{2ex}
\caption{\label{prompting}TTCW Originality1 (Originality in Theme and Content) }
\vspace{-3ex}
\end{table*}


% ==================================================








\begin{table*}[!ht]
\centering
\small
% \def\arraystretch{1.15}
\begin{tabular}{|l|l|}
\hline
\begin{tabular}[c]{@{}l@{}}Expert \\ Measure\end{tabular}               & \begin{tabular}[c]{@{}l@{}}Is the story an original piece of writing without any cliches?\end{tabular}                                                                                                                                     \\ \hline
\begin{tabular}[c]{@{}l@{}}Expanded\\ Expert\\ Measure (M)\end{tabular} & \begin{tabular}[c]{@{}l@{}}A cliche is an idea, expression, character, or plot that has been overused to the point of losing its original \\meaning or impact. They often become predictable and uninteresting for the reader. Originality suggests\\ that the piece isn't cliche.

 \end{tabular} \\ \hline
\begin{tabular}[c]{@{}l@{}}Human\\ Instruction\end{tabular}             & \begin{tabular}[c]{@{}l@{}}\{\{M\}\}\\ \\ Based on the story that you just read, answer the following question.\\ \textit{\color{blue}Is the story an original piece of writing without any cliches?}\\ -Yes \\ -No \\\\ Reasoning: \end{tabular}                                                                       \\ \hline
\begin{tabular}[c]{@{}l@{}}LLM\\ Instruction\end{tabular}               & \begin{tabular}[c]{@{}l@{}}\{\{M\}\}\\ \\ Given the story above, are there any cliches in the story? If so, list out all the elements in this story that \\are cliche. Then overall, give your reasoning if the piece is negatively impacted by the cliches and give \\an answer to the question below between 'Yes' or 'No' only\\ \\ \textit{\color{blue} Q) Is the story an original piece of writing without any cliches?}\end{tabular}                                                                                                                                                                                                                                 \\ \hline
\end{tabular}
\vspace{2ex}
\caption{\label{prompting}TTCW Originality2 (Originality in Thought) }
\vspace{-5ex}
\end{table*}


% ==================================================




\begin{table*}[!ht]
\centering
\small
% \def\arraystretch{1.15}
\begin{tabular}{|l|l|}
\hline
\begin{tabular}[c]{@{}l@{}}Expert \\ Measure\end{tabular}               & \begin{tabular}[c]{@{}l@{}}Does the story show originality in its form?\end{tabular}                                                                                                                                     \\ \hline
\begin{tabular}[c]{@{}l@{}}Expanded\\ Expert\\ Measure (M)\end{tabular} & \begin{tabular}[c]{@{}l@{}}When someone says that a piece of fiction 'shows an innovative use of form/structure', they're referring to\\ how the writer has chosen to shape and organize the story in an unusual, original, or inventive way. This could \\involve a variety of different elements, including:\\ \\ 

Narrative Structure: This could include unconventional timelines (e.g. a non-linear story, a story told in reverse)\\, multiple perspectives or narrators, or unusual narrative voices (e.g. a story told from the perspective of an \\inanimate object).\\ \\ 

Format: This could be something as simple as using unconventional punctuation or capitalization, or as complex \\as telling a story through a series of letters, diary entries, newspaper clippings, or other documents. In recent years,\\ some authors have even experimented with using social media posts or text messages as a form of narrative structure.\\ \\ 

Genre Hybridity: Combining elements from different genres or sub-genres in unexpected ways can also be seen\\ as an innovative use of form such as Horror-Mystery or Comic Fantasy.\\ \\ 

Plot Structure: Deviating from traditional plot structures such as three-act structure, or following them in unexpected\\ ways.For example, telling a story without a clear resolution, incorporating multiple climaxes or using revelation as a \\device where a surprising, and often shocking, development occurs that was previously kept hidden from the \\characters and/or the audience. It's typically designed to provide new context for interpreting what has previously \\occurred in the story. \\ \\ 

Language and Style: Innovative use of form can also come in the form of unique use of language, style, or \\even typography, such as concrete poetry or writing that visually represents its subject matter on the page.\\ \\ 

The goal of this innovation is often to provide a fresh reader experience, challenge conventional reading\\ expectations, or to create a deeper or more complex exploration of the story's themes.

 \end{tabular} \\ \hline
\begin{tabular}[c]{@{}l@{}}Human\\ Instruction\end{tabular}             & \begin{tabular}[c]{@{}l@{}}\{\{M\}\}\\ \\ Based on the story that you just read, answer the following question.\\ \textit{\color{blue}Does the story show originality in its form?}\\ -Yes \\ -No \\\\ Reasoning: \end{tabular}                                                                       \\ \hline
\begin{tabular}[c]{@{}l@{}}LLM\\ Instruction\end{tabular}               & \begin{tabular}[c]{@{}l@{}}\{\{M\}\}\\ \\ Given the story and the devices mentioned above, list each device used with a short explanation of whether it is \\successful or not. Then overall, give your reasoning about the question below and give an answer to it\\ between 'Yes' or 'No' only\\ \\ \textit{\color{blue} Q) Does the story show originality in its form?}\end{tabular}                                                                                                                                                                                                                                 \\ \hline
\end{tabular}
\vspace{2ex}
\caption{\label{prompting}TTCW Originality3 (Originality in Form) }
\vspace{-5ex}
\end{table*}


% ==================================================




\begin{table*}[!ht]
\centering
\small
% \def\arraystretch{1.15}
\begin{tabular}{|l|l|}
\hline
\begin{tabular}[c]{@{}l@{}}Expert \\ Measure\end{tabular}               & \begin{tabular}[c]{@{}l@{}}Does each character in the story feel developed at the appropriate complexity level, ensuring that no character \\feels like they are present simply to satisfy a plot requirement?\end{tabular}                                                                                                                                     \\ \hline
\begin{tabular}[c]{@{}l@{}}Expanded\\ Expert\\ Measure (M)\end{tabular} & \begin{tabular}[c]{@{}l@{}} A `flat character' is typically a minor character who is not thoroughly developed or who does not undergo \\significant change or growth throughout the story. They often embody or represent a single trait or idea, \\and they're used to advance the plot or highlight certain qualities in other characters.\\ \\ 

A `complex character', also known as a round character, has depth in feelings and passions, has a variety \\of traits of a real human being, and evolves over time. They have their strengths, weaknesses, \\and they learn from their experiences. They tend to be more engaging to the reader, as they mirror \\the complexity of real people.\\ \\ 

In good stories, authors take a character who initially appears to be one-dimensional or stereotypical (flat) and \\add depth to them. This could be done by revealing more about their backstory, introducing unexpected traits \\or motivations, or having them grow and change in response to the events of the story. \\This transformation from a flat to a complex character can make the narrative more engaging and believable.
 \end{tabular} \\ \hline
\begin{tabular}[c]{@{}l@{}}Human\\ Instruction\end{tabular}             & \begin{tabular}[c]{@{}l@{}}\{\{M\}\}\\ \\ Based on the story that you just read, answer the following question.\\  \textit{\color{blue} Q) Does each character in the story feel developed at the appropriate complexity level, ensuring that no character} \\ \textit{\color{blue}feels like they are present simply to satisfy a plot requirement?}\\ -Yes \\ -No \\\\ Reasoning: \end{tabular}                                                                       \\ \hline
\begin{tabular}[c]{@{}l@{}}LLM\\ Instruction\end{tabular}               & \begin{tabular}[c]{@{}l@{}}\{\{M\}\}\\ \\ Given the story above, list each character and the level of development. Then overall, give your reasoning \\about the question below and give an answer to it between 'Yes' or 'No' only\\ \\ 
 \textit{\color{blue} Q) Does each character in the story feel developed at the appropriate complexity level, ensuring that no character} \\ \textit{\color{blue}feels like they are present simply to satisfy a plot requirement?}\end{tabular}                                                                                                                                                                                                                                 \\ \hline
\end{tabular}
\vspace{2ex}
\caption{\label{prompting}TTCW Elaboration2 (Character Development) }
\vspace{-5ex}
\end{table*}


% ==================================================



\begin{table*}[!ht]
\centering
\small
% \def\arraystretch{1.15}
\begin{tabular}{|l|l|}
\hline
\begin{tabular}[c]{@{}l@{}}Expert \\ Measure\end{tabular}               & \begin{tabular}[c]{@{}l@{}}Are there passages in the story that involve subtext and when there is subtext, does it enrich the story's setting \\or does it feel forced?\end{tabular}                                                                                                                                     \\ \hline
\begin{tabular}[c]{@{}l@{}}Expanded\\ Expert\\ Measure (M)\end{tabular} & \begin{tabular}[c]{@{}l@{}} `Surface' level: This is the most apparent and straightforward level of a story. It includes the visible actions, \\explicit dialogue, and clear descriptions. This is what literally happens in the plot: the characters' actions, events, \\and the apparent consequences.\\ \\ 

`Subtext' level: This is the underlying or implicit meaning that isn't directly stated but can be inferred from \\the characters'  actions, dialogue, and other elements of the story. Subtext often reveals deeper truths about \\characters, themes, or the overall message of the piece. It could be a hidden motive, an unstated\\ emotion, a cultural commentary, or a symbolic meaning.\\ \\ 

For example, in a conversation between two characters, the surface text might be polite and cordial, but the \\subtext \\discerned from the characters' nonverbal cues, previous interactions, or the context of their conversation\\ — could suggest tension or hostility.\\ \\ 

Effective fiction often operates on both levels. The surface text keeps the reader engaged with the plot and \\characters, while the subtext provides depth, complexity, and additional layers of interpretation, \\contributing to a richer and more rewarding reading experience.
 \end{tabular} \\ \hline
\begin{tabular}[c]{@{}l@{}}Human\\ Instruction\end{tabular}             & \begin{tabular}[c]{@{}l@{}}\{\{M\}\}\\ \\ Based on the story that you just read, answer the following question.\\  \textit{\color{blue} Q) Are there passages in the story that involve subtext and when there is subtext, does it enrich the story's setting} \\ \textit{\color{blue} or does it feel forced?}\\ -Yes \\ -No \\\\ Reasoning: \end{tabular}                                                                       \\ \hline
\begin{tabular}[c]{@{}l@{}}LLM\\ Instruction\end{tabular}               & \begin{tabular}[c]{@{}l@{}}\{\{M\}\}\\ \\ Given the story above, answer the following question. Please first explain your reasoning step by step \\and then give an answer between 'Yes' or 'No' only\\ \\ 
 \textit{\color{blue} Q)Are there passages in the story that involve subtext and when there is subtext, does it enrich the story's setting} \\ \textit{\color{blue} or does it feel forced?}\end{tabular}                                                                                                                                                                                                                                 \\ \hline
\end{tabular}
\vspace{2ex}
\caption{\label{prompting}TTCW Elaboration3 (Rhetorical Complexity) }
\vspace{-5ex}
\end{table*}


% ==================================================
 


\end{document}


% This document was modified from the file originally made available by
% Pat Langley and Andrea Danyluk for ICML-2K. This version was created
% by Iain Murray in 2018, and modified by Alexandre Bouchard in
% 2019 and 2021 and by Csaba Szepesvari, Gang Niu and Sivan Sabato in 2022.
% Modified again in 2023 and 2024 by Sivan Sabato and Jonathan Scarlett.
% Previous contributors include Dan Roy, Lise Getoor and Tobias
% Scheffer, which was slightly modified from the 2010 version by
% Thorsten Joachims & Johannes Fuernkranz, slightly modified from the
% 2009 version by Kiri Wagstaff and Sam Roweis's 2008 version, which is
% slightly modified from Prasad Tadepalli's 2007 version which is a
% lightly changed version of the previous year's version by Andrew
% Moore, which was in turn edited from those of Kristian Kersting and
% Codrina Lauth. Alex Smola contributed to the algorithmic style files.
