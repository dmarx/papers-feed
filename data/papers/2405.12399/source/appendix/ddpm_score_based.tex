\section{Link between DDPM and continuous-time score-based diffusion models}
\label{app:ddpm}

Denoising Diffusion Probabilistic Models (\textsc{ddpm}, \citet{ho2020DDPM}) can be described as a discrete version of the diffusion process introduced in Section \ref{subsec:diffusion}, as described in \citet{song_sde}. The discrete forward process is a Markov chain characterized by a discrete noise schedule $0 < \beta_1, \dots, \beta_i, \dots \beta_N < 1$, and a variance-preserving Gaussian transition kernel,

\begin{equation}
    p(\x^i|\x^{i-1}) = \mathcal{N}(\x^i; \sqrt{1-\beta_i} \x^{i-1}, \beta_i \mathbf{I}).
\end{equation}

In the continuous time limit $N \to \infty$, the Markov chain becomes a diffusion process, and the discrete noise schedule becomes a time-dependent function $\beta(\tau)$. This diffusion process can be described by an SDE with drift coefficient $\mathbf{f}(\x, \tau) = -\frac{1}{2}\beta(\tau)\x$ and diffusion coefficient $g(\tau) = \sqrt{\beta(\tau)}$ \citep{song_sde}. 

