\documentclass[twocolumn,10pt]{article}
% adding this for arxiv
\usepackage[numbers]{natbib}
\usepackage{microtype}
\usepackage{graphicx}
\usepackage{subfigure}
\usepackage{booktabs} % for professional tables
\usepackage{array}
\usepackage{multirow}
\usepackage{bm}

\usepackage[utf8]{inputenc} % allow utf-8 input
\usepackage[T1]{fontenc}    % use 8-bit T1 fonts
\usepackage{hyperref}       % hyperlinks
\usepackage{url}            % simple URL typesetting
\usepackage{booktabs}       % professional-quality tables
\usepackage{amsfonts}       % blackboard math symbols
\usepackage{nicefrac}       % compact symbols for 1/2, etc.
\usepackage{microtype}      % microtypography
\usepackage{xcolor}         % colors
\usepackage{graphicx} 
\usepackage{array}
\usepackage{subfigure}
\usepackage{multirow}
\usepackage{amsmath}
\usepackage{bm}
\usepackage{siunitx}
\newcommand{\todo}[1]{\textbf{\textcolor{red}{TODO: #1}}}

\begin{document}

\title{\Large \bf Mosaic Memory: Fuzzy Duplication in Copyright Traps for Large Language Models
        \vspace*{0.5cm}}

\date{}

\author{
 {\rm Igor Shilov\thanks{Equal contribution}}\\
 \textit{Imperial College London}
 \and
 {\rm Matthieu Meeus\footnotemark[1]}\\
 \textit{Imperial College London}
 \and
 {\rm Yves-Alexandre de Montjoye\footnote{Corresponding author: deMontjoye@imperial.ac.uk.}}\\
 \textit{Imperial College London}
} % end author


\maketitle

\vskip 0.3in

\begin{abstract}

The immense datasets used to develop Large Language Models (LLMs) often include copyright-protected content, typically without the content creator's consent. Copyright traps have been proposed to be injected into the original content, improving content detectability in newly released LLMs. Traps, however, rely on the exact duplication of a unique text sequence, leaving them vulnerable to commonly deployed data deduplication techniques. We here propose the generation of \emph{fuzzy} copyright traps, featuring slight modifications across duplication. When injected in the fine-tuning data of a 1.3B LLM, we show fuzzy trap sequences to be memorized nearly as well as exact duplicates. Specifically, the Membership Inference Attack (MIA) ROC AUC only drops from $0.90$ to $0.87$ when $R=4$ tokens are replaced across the fuzzy duplicates. We also find that selecting replacement positions to minimize the exact overlap between fuzzy duplicates leads to similar memorization, while making fuzzy duplicates highly unlikely to be removed by any deduplication process. Lastly, we argue that the fact that LLMs memorize across fuzzy duplicates challenges the study of LLM memorization relying on naturally occurring duplicates. Indeed, we find that the commonly used training dataset, The Pile, contains significant amounts of fuzzy duplicates. This introduces a previously unexplored confounding factor in post-hoc studies of LLM memorization, and questions the effectiveness of (exact) data deduplication as a privacy protection technique. 
\end{abstract}


%-------------------------------------------------------------------------------
\section{Introduction}
\section{Introduction}
\label{sec:intro}

Neural networks have undergone rapid development in various computer vision tasks such as image classification, detection and segmentation. While their impressive performance has powered many applications, a roaring trend is to pursue fast neural networks with low latency and high throughput for great user experiences, instant responses, safety reasons, etc.

How to be fast? Instead of asking for more costly computing devices, researchers and practitioners prefer to design cost-effective fast neural networks with reduced computational complexity, mainly measured in the number of {\bf fl}oating-point {\bf op}eration{\bf s} (FLOPs)\footnote{We follow a widely adopted definition of FLOPs, as the number of multiply-adds~\cite{zhang2018shufflenet,liu2022convnet}.}. MobileNets~\cite{howard2017mobilenets,sandler2018mobilenetv2,howard2019searching},
ShuffleNets~\cite{zhang2018shufflenet,ma2018shufflenet} and GhostNet~\cite{han2020ghostnet}, among others, leverage the depthwise convolution (DWConv)~\cite{sifre2014rigid} and/or group convolution (GConv)~\cite{krizhevsky2012imagenet} to extract spatial features. However, in the effort to reduce FLOPs, the operators often suffer from the side effect of increased memory access. MicroNet~\cite{li2021micronet} further decomposes and sparsifies the network to push its FLOPs to an extremely low level. Despite its improvement in FLOPs, this approach experiences inefficient fragmented computation. Besides, the above networks are often accompanied by additional data manipulations, such as concatenation, shuffling, and pooling, whose running time tends to be significant for tiny models.

\begin{figure}
    \centering
    \includegraphics[width=1\linewidth]{figures/PConv-cropped.pdf}
    \vspace{-0.2in}
    \caption{Our partial convolution (PConv) is fast and efficient by applying filters on only a few input channels while leaving the remaining ones untouched. PConv obtains lower FLOPs than the regular convolution and higher FLOPS than the depthwise/group convolution.}
    \label{fig: PConv}
    \vspace{-0.05in}
\end{figure}

\begin{figure*}
    \centering
    \includegraphics[width=1\linewidth]{figures/FLOPS_latency_vs_FLOPs-cropped.pdf}
    \vspace{-0.3in}
    \caption{(a) FLOPS under varied FLOPs on CPU. Many existing neural networks suffer from low computational speed issues. Their effective FLOPS are lower than the popular ResNet50. By contrast, our FasterNet attains higher FLOPS. (b) Latency under varied FLOPs on CPU. Our FasterNet obtains lower latency than others with the same amount of FLOPs.}
    \label{fig:FLOPS(latency)_vs_FLOPs}
    \vspace{-0.05in}
\end{figure*}

Apart from the above pure convolutional neural networks (CNNs),  there is an emerging interest in making vision transformers (ViTs)~\cite{dosovitskiy2020image} and multilayer perceptrons (MLPs) architectures~\cite{tolstikhin2021mlp} smaller and faster. For example, MobileViTs~\cite{mehta2021mobilevit,mehta2022separable,wadekar2022mobilevitv3} and MobileFormer~\cite{chen2022mobile} reduce the computational complexity by combining DWConv with a modified attention mechanism. However, they still suffer from the aforementioned issue with DWConv and also need dedicated hardware support for the modified attention mechanism. The use of advanced yet time-consuming normalization and activation layers may also limit their speed on devices.

All these issues together lead to the following question: Are
these ``fast'' neural networks really fast? To answer this, we examine the relationship between latency and FLOPs, which is captured by 
\begin{equation}
  Latency = \frac{FLOPs}{FLOPS},
  \label{eq:latency_FLOPs}
\end{equation}
where FLOPS is short for {\bf fl}oating-point {\bf op}erations per {\bf s}econd, as a measure of the effective computational speed. While there are many attempts to reduce FLOPs, they seldom consider optimizing FLOPS at the same time to achieve truly low latency. To better understand the situation, we compare the FLOPS of typical neural networks on an Intel CPU. The results in~\cref{fig:FLOPS(latency)_vs_FLOPs} show that many existing neural networks suffer from low FLOPS, and their FLOPS is generally lower than the popular ResNet50. With such low FLOPS, these ``fast'' neural networks are actually not fast enough.
Their reduction in FLOPs cannot be translated into the exact amount of reduction in latency. In some cases, there is no improvement, and it even leads to worse latency. For example, CycleMLP-B1~\cite{chen2021cyclemlp} has half of FLOPs of ResNet50~\cite{he2016deep} but runs more slowly (\ie, CycleMLP-B1 \vs ResNet50: 116.1ms \vs 73.0ms). Note that this discrepancy between FLOPs and latency has also been noticed in previous works~\cite{ma2018shufflenet,mehta2021mobilevit} but remains unresolved partially because they employ the DWConv/GConv and various data manipulations with low FLOPS. It is deemed there are no better alternatives available.

This paper aims to eliminate the discrepancy by developing a simple yet fast and effective operator that maintains high FLOPS with reduced FLOPs. Specifically, we reexamine existing operators, particularly  DWConv, in terms of the computational speed -- FLOPS. We uncover that the main reason causing the low FLOPS issue is \emph{frequent memory access}. We then propose a novel partial convolution (PConv) as a competitive alternative that reduces the computational redundancy as well as the number of memory access. \cref{fig: PConv} illustrates the design of our PConv. It takes advantage of redundancy within the feature maps and systematically applies a regular convolution (Conv) on only a part of the input channels while leaving the remaining ones untouched. By nature, PConv has lower FLOPs than the regular Conv while having higher FLOPS than the DWConv/GConv. In other words, PConv better exploits the on-device computational capacity. PConv is also effective in extracting spatial features as empirically validated later in the paper. 


We further introduce FasterNet, which is primarily built upon our PConv, as a new family of networks that run highly fast on various devices. In particular, our FasterNet achieves state-of-the-art performance for classification, detection, and segmentation tasks while having much lower latency and higher throughput. For example, our tiny FasterNet-T0 is $2.8\times$, $3.3\times$, and $2.4\times$ faster than MobileViT-XXS~\cite{mehta2021mobilevit} on GPU, CPU, and ARM processors, respectively, while being 2.9\% more accurate on ImageNet-1k. Our large FasterNet-L achieves 83.5\% top-1 accuracy, on par with the emerging Swin-B~\cite{liu2021swin}, while offering 36\% higher throughput on GPU and saving 37\% compute time on CPU. To summarize, our contributions are as follows:
\begin{itemize}
\itemsep0em 
\item We point out the importance of achieving higher FLOPS beyond simply reducing FLOPs for faster neural networks.
\item We introduce a simple yet fast and effective operator called PConv, which has a high potential to replace the existing go-to choice, DWConv.
\item We introduce FasterNet which runs favorably and universally fast on a variety of devices such as GPU, CPU, and ARM processors.
\item We conduct extensive experiments on various tasks and validate the high speed and effectiveness of our PConv and FasterNet.
\end{itemize}
\label{sec:introduction}
%-------------------------------------------------------------------------------

%-------------------------------------------------------------------------------
\section{Related work}
\section{Related Work}
\label{sec:related}                             

\noindent{\textbf{Text-to-Video Generation.}}~Early video generation models were primarily based on Unet-based latent diffusion models (LDMs) extended from text-to-image models like Stable Diffusion~\cite{rombach2022high}. For example, AnimateDiff~\cite{guo2023animatediff} introduced a temporal attention module to improve temporal consistency across frames. Subsequent video generation models~\cite{wang2023modelscope, cerspense2023zeroscope, chen2023videocrafter1, chen2024videocrafter2, zhang2024moonshot, zhang2023show} adopted an alternating approach with 2D spatial and 1D temporal attention, including works like ModelScope, VideoCrafter, Moonshot, and Show-1. 

With advancements in large language models (LLMs) and the introduction of Sora~\cite{sora2024}, attention shifted from Unet architectures to transformer-based architectures (DiT). DiT-based video generation models, such as Latte~\cite{ma2024latte} and OpenSora~\cite{opensora}, extended the DiT text-to-image (T2I) model~\cite{chen2023pixart} and maintained the 2D and 1D alternating attention approach, achieving promising results. Recently, DiT-based video generation has rapidly progressed, achieving further improvements in quality. Several methods~\cite{yang2024cogvideox, opensoraplan, genmo2024mochi} have moved away from the 2D and 1D alternating approach, instead treating video frames as a single long sequence with 3D positional embeddings for encoding. These approaches also prepend text tokens—processed through a text encoder—to the video sequence, creating a streamlined network that relies solely on full self-attention and feed-forward layers. Our method builds upon these recent open-source transformer-based video generation models.


\vspace{0.5em}
\noindent{\textbf{Video Matting.}}~A straightforward approach for RGBA video generation is to extract the alpha channel from generated RGB content, as done with traditional green screen keying or learning-based video matting expert models~\cite{lin2023omnimatterf, lin2021real, lin2022robust}. OmnimatteRF~\cite{lin2023omnimatterf} introduces a video matting method that combines dynamic 2D foreground layers with a 3D background model, enabling more realistic scene reconstruction for real-world videos. Robust Video Matting (RVM)~\cite{lin2022robust} proposes a real-time, high-quality human video matting method with a recurrent architecture for improved temporal coherence, achieving state-of-the-art results without auxiliary inputs. Another work presents a high-speed, high-resolution background replacement technique with precise alpha matte extraction, supported by the VideoMatte240K and PhotoMatte13K/85 datasets~\cite{lin2021real}. Additionally, many image matting methods~\cite{chen2022pp, li2024matting, yao2024vitmatte, wang2024matting} can be applied for frame-by-frame matting.


Further, several works~\cite{he2024lotus, yang2024depth, ke2024repurposing} in image depth estimation adapt pretrained generation models for prediction tasks, achieving strong performance that often surpasses traditional, scratch-trained expert models. Marigold~\cite{ke2024repurposing} modifies architectures to create image-conditioned generation models, while Lotus~\cite{he2024lotus} explores the role of the diffusion process in this context. Although there is currently no dedicated approach for video matting within video generation models, we replicate and extend these methods to evaluate their performance, allowing us to highlight the limitations of prediction-based pipelines for RGBA generation.

\vspace{0.5em}
\noindent{\textbf{Generation beyond RGB.}}~Another category of methods~\cite{zhang2024transparent, long2024wonder3d, bao2023one, luo2024intrinsicdiffusion, zeng2024rgb, he2024lucidfusion, yang2023defect} explores expanding generation models to simultaneously generate additional channels, though they are not specifically designed for RGBA video generation. 
For instance, LayerDiffusion~\cite{zhang2024transparent} modifies the VAE in latent diffusion models to decode alpha channels. However, VAEs typically lack the semantic understanding required for precise alpha generation, limiting their effectiveness in complex visual scenarios where texture and contour details are critical. 
In contrast, other approaches~\cite{long2024wonder3d, bao2023one, luo2024intrinsicdiffusion, zeng2024rgb} modify the denoising model directly to enable joint generation. Wonder3D~\cite{long2024wonder3d} uses a domain embedding to control the model’s generation modality, while methods like IntrinsicDiffusion~\cite{luo2024intrinsicdiffusion} and RGB\(\leftrightarrow\)X~\cite{zeng2024rgb} adapt the UNet’s input and output layers to jointly produce intrinsic modalities. However, all these methods are designed for image tasks and rely on UNet architectures. When applied to video generation, they face limitations in quality and diversity due to the scarcity of RGBA video data.

\label{sec:related_work}
%-------------------------------------------------------------------------------

%-------------------------------------------------------------------------------
\section{Fuzzy trap sequences}
\section{Method}
%%%%%%%%% Figure: Overall framework
\begin{figure*}[t]
  \centering
   \includegraphics[width=0.85\linewidth]{figures/PoolFormer_overall_architecture.pdf}
   \vspace{-4mm}
   \caption{\textbf{(a) The overall framework of \modelname{}.} Similar to \cite{resnet, pvt, swin}, \modelname{} adopts hierarchical architecture with 4 stages. For a model with L \modelname{} blocks, stage [1, 2, 3, 4] have [L/6, L/6, L/2, L/6] blocks, respectively. The feature dimension $D_i$ of stage $i$ is shown in the figure. \textbf{(b) The architecture of \modelname{} block.} Compared with Transformer block, it replaces attention with extremely simple non-parametric operator, pooling, to conduct only basic token mixing.}
   \label{fig:overall_architecture}
\end{figure*}


%%%%%%%%% Algorithm: Pooling
\begin{algorithm}[t]
\caption{Pooling for PoolFormer, PyTorch-like Code}
\label{alg:code}
\definecolor{codeblue}{rgb}{0.25,0.5,0.5}
\definecolor{codekw}{rgb}{0.85, 0.18, 0.50}
\lstset{
  backgroundcolor=\color{white},
  basicstyle=\fontsize{7.5pt}{7.5pt}\ttfamily\selectfont,
  columns=fullflexible,
  breaklines=true,
  captionpos=b,
  commentstyle=\fontsize{7.5pt}{7.5pt}\color{codeblue},
  keywordstyle=\fontsize{7.5pt}{7.5pt}\color{codekw},
}
\begin{lstlisting}[language=python]
import torch.nn as nn

class Pooling(nn.Module):
    def __init__(self, pool_size=3):
        super().__init__()
        self.pool = nn.AvgPool2d(
            pool_size, stride=1, 
            padding=pool_size//2, 
            count_include_pad=False,
        )
    def forward(self, x):
        """
        [B, C, H, W] = x.shape
        Subtraction of the input itself is added 
        since the block already has a 
        residual connection.
        """
        return self.pool(x) - x
\end{lstlisting}
\end{algorithm}

\subsection{MetaFormer}
We present the core concept ``MetaFormer" for this work at first. As shown in Figure \ref{fig:first_figure}, abstracted from Transformers \cite{transformer}, 
MetaFormer is a general architecture where the token mixer is not specified while the other components are kept the same as Transformers. The input $I$ is first processed by input embedding, such as  patch embedding for ViTs \cite{vit},
\begin{equation}
    X = \mathrm{InputEmb}(I),
\end{equation}
where  $X \in \mathbb{R}^{N \times C}$ denotes the embedding tokens with sequence length $N$ and embedding dimension $C$. 


Then, embedding tokens are fed to repeated MetaFormer blocks, each of which includes two residual sub-blocks. Specifically, the first sub-block mainly contains a token mixer to communicate information among tokens and this sub-block can be expressed as
\begin{equation}
    Y = \mathrm{TokenMixer}(\mathrm{Norm}(X)) + X,
\end{equation}
where $\mathrm{Norm}(\cdot)$ denotes the normalization such as Layer Normalization \cite{layer_norm} or Batch Normalization \cite{batch_norm}; $\mathrm{TokenMixer}(\cdot)$ means a module mainly working for mixing token information. It is implemented by various attention mechanism in recent vision Transformer models  \cite{vit,refiner,t2t} or spatial MLP in MLP-like models \cite{mlp-mixer, resmlp}. Note that the main function of the token mixer is to propagate token information although some token mixers can also mix channels, like attention. 


The second sub-block primarily consists of a two-layered MLP with non-linear activation, 
\begin{equation}
    Z = \sigma(\mathrm{Norm}(Y)W_1)W_2 + Y,
\end{equation}
where $W_1 \in \mathbb{R}^{C \times rC}$ and $W_2 \in \mathbb{R}^{rC \times C}$ are learnable parameters with MLP expansion ratio $r$; $\sigma(\cdot)$ is a non-linear activation function, such as GELU \cite{gelu} or ReLU \cite{relu}. 

\myPara{Instantiations of MetaFormer} MetaFormer describes a general architecture 
% a general architecture that is powerful at solving computer vision tasks.  
with which different models can be obtained immediately by specifying the concrete design of the token mixers. 
As shown in Figure \ref{fig:first_figure}(a), if the token mixer is specified as attention or spatial MLP, MetaFormer then becomes a Transformer or MLP-like model respectively. 

\subsection{PoolFormer}
From the introduction of Transformers \cite{transformer}, lots of works attach much importance to the attention and focus on designing various attention-based token mixer components. In contrast, these works pay little attention to the general architecture, \ie, the MetaFormer.


In this work, we argue that this MetaFormer general architecture contributes mostly to the success of the recent Transformer and MLP-like models. 
To demonstrate it, we deliberately employ an embarrassingly simple operator, pooling, as the token mixer. This operator has no learnable parameters and it just makes each token averagely aggregate its nearby token features. 


Since this work is targeted at vision tasks,  we assume the input is in channel-first data format, \ie,  $T \in \mathbb{R}^{C \times H \times W}$. The pooling operator can be expressed as
\begin{equation}
\label{eq:pool}
    T'_{:, i, j} =  \frac{1}{K \times K} \sum_{p,q=1}^{K}T_{:, i+p-\frac{K+1}{2}, i+q-\frac{K+1}{2}} - T_{:, i, j},
\end{equation}
where $K$ is the pooling size. Since the MetaFormer block already has a residual connection, subtraction of the input itself is added in Equation (\ref{eq:pool}). The PyTorch-like code of the pooling is shown in Algorithm \ref{alg:code}.


As well known, self-attention and spatial MLP have computational complexity quadratic to the number of tokens to mix. Even worse, spatial MLPs bring much more parameters when handling longer sequences. As a result, self-attention and spatial MLPs usually can only process hundreds of tokens. In contrast, the pooling needs a computational complexity linear to the sequence length without any learnable parameters.  Thus, we take advantage of pooling by adopting a hierarchical structure similar to traditional CNNs \cite{alexnet, vgg, resnet} and recent hierarchical Transformer variants \cite{swin, pvt}. Figure \ref{fig:overall_architecture} shows the overall framework of PoolFormer. Specifically, PoolFormer has 4 stages with $\frac{H}{4} \times \frac{W}{4}$, $\frac{H}{8} \times \frac{W}{8}$, $\frac{H}{16} \times \frac{W}{16}$, and $\frac{H}{32} \times \frac{W}{32}$ tokens respectively, where $H$ and $W$ represent the width and height of the input image. There are two groups of embedding size: 1) small-sized models with embedding dimensions of 64, 128, 320, and 512 responding to the four stages; 2) medium-sized models with embedding dimensions 96, 192, 384, and 768. Assuming there are $L$ PoolFormer blocks in total, stages 1, 2, 3, and 4 will contain $L/6$, $L/6$, $L/2$, and $L/6$ PoolFormer blocks respectively. The MLP expansion ratio is set as 4. According to the above simple model scaling rule, we obtain 5 different model sizes of PoolFormer and their hyper-parameters are shown in Table \ref{tab:model}.


%%%%%%%%% Table: Model Configurations
\begin{table}[t]
\footnotesize
\centering
\setlength{\tabcolsep}{2pt}
% \scalebox{0.65}{\newcommand{\blockc}[4]{
$\begin{bmatrix}
	\begin{array}{l}
	R_1=#1 \\
	N_1=#2 \\
	E_1=#3 \\
	\end{array}
\end{bmatrix} \times #4$
}

\newcommand{\sblock}[3]{
$\begin{matrix}
E_{#1}=#2 \\
L_{#1}=#3 \\
\end{matrix}$
}

\newcommand{\poollayer}{
Pooling Size & \multicolumn{5}{c}{$3 \times 3$, stride 1}\\
\cline{4-9}
}

\newcommand{\stitle}[6]{
\multirow{5}{*}{#1} & \multirow{5}{*}{\scalebox{1}{$\frac{H}{#2}\times \frac{W}{#2}$}} & \multirow{2}{*}{\tabincell{c}{Patch \\ Embedding}} & Patch Size & \multicolumn{5}{c}{$#3 \times #3$, stride $#4$} \\
\cline{4-9}
    &    &    & Embed. Dim. & \multicolumn{3}{c|}{$#5$} & \multicolumn{2}{c}{$#6$} \\
\cline{3-9}
& & \multirow{3}{*}{\tabincell{c}{\modelname{}\\Block}} 
}

\begingroup
\renewcommand{\arraystretch}{1.1}
\begin{tabular}{c|c|c|c|c|c|c|c|c}
\toprule
  \multirow{2}{*}{Stage} & \multirow{2}{*}{\#Tokens} & \multicolumn{2}{c|}{\multirow{2}{*}{Layer Specification}} & \multicolumn{5}{c}{\modelname{}} \\
\cline{5-9}
 & & \multicolumn{2}{c|}{} & S12 & S24 & S36 & M36 & M48 \\
\whline
\stitle{1}{4}{7}{4}{64}{96}    & \poollayer
 & & & MLP Ratio & \multicolumn{5}{c}{4} \\
\cline{4-9}
 & & & \# Block & 2 & 4 & 6 & 6 & 8 \\
\hline
\stitle{2}{8}{3}{2}{128}{192}  & \poollayer
 & & & MLP Ratio & \multicolumn{5}{c}{4} \\
 \cline{4-9}
 & & & \# Block & 2 & 4 & 6 & 6 & 8 \\
\hline
\stitle{3}{16}{3}{2}{320}{384} & \poollayer
 & & & MLP Ratio & \multicolumn{5}{c}{4} \\
\cline{4-9}
 & & & \# Block & 6 & 12 & 18 & 18 & 24 \\
\hline
\stitle{4}{32}{3}{2}{512}{768} & \poollayer
 & & & MLP Ratio & \multicolumn{5}{c}{4} \\
 \cline{4-9}
 & & & \# Block & 2 & 4 & 6 & 6 & 8 \\
\hline
\multicolumn{4}{c|}{Parameters~(M)}& 11.9 & 21.4              &  30.8            &  56.1              &  73.4    \\
\hline
\multicolumn{4}{c|}{MACs~(G)}     & 1.8  &  3.4              &  5.0             &  8.8              &  11.6    \\
\bottomrule
\end{tabular}
\endgroup
}
\newcommand{\blockc}[4]{
$\begin{bmatrix}
	\begin{array}{l}
	R_1=#1 \\
	N_1=#2 \\
	E_1=#3 \\
	\end{array}
\end{bmatrix} \times #4$
}

\newcommand{\sblock}[3]{
$\begin{matrix}
E_{#1}=#2 \\
L_{#1}=#3 \\
\end{matrix}$
}

\newcommand{\poollayer}{
Pooling Size & \multicolumn{5}{c}{$3 \times 3$, stride 1}\\
\cline{4-9}
}

\newcommand{\stitle}[6]{
\multirow{5}{*}{#1} & \multirow{5}{*}{\scalebox{1}{$\frac{H}{#2}\times \frac{W}{#2}$}} & \multirow{2}{*}{\tabincell{c}{Patch \\ Embedding}} & Patch Size & \multicolumn{5}{c}{$#3 \times #3$, stride $#4$} \\
\cline{4-9}
    &    &    & Embed. Dim. & \multicolumn{3}{c|}{$#5$} & \multicolumn{2}{c}{$#6$} \\
\cline{3-9}
& & \multirow{3}{*}{\tabincell{c}{\modelname{}\\Block}} 
}

\begingroup
\renewcommand{\arraystretch}{1.1}
\begin{tabular}{c|c|c|c|c|c|c|c|c}
\toprule
  \multirow{2}{*}{Stage} & \multirow{2}{*}{\#Tokens} & \multicolumn{2}{c|}{\multirow{2}{*}{Layer Specification}} & \multicolumn{5}{c}{\modelname{}} \\
\cline{5-9}
 & & \multicolumn{2}{c|}{} & S12 & S24 & S36 & M36 & M48 \\
\whline
\stitle{1}{4}{7}{4}{64}{96}    & \poollayer
 & & & MLP Ratio & \multicolumn{5}{c}{4} \\
\cline{4-9}
 & & & \# Block & 2 & 4 & 6 & 6 & 8 \\
\hline
\stitle{2}{8}{3}{2}{128}{192}  & \poollayer
 & & & MLP Ratio & \multicolumn{5}{c}{4} \\
 \cline{4-9}
 & & & \# Block & 2 & 4 & 6 & 6 & 8 \\
\hline
\stitle{3}{16}{3}{2}{320}{384} & \poollayer
 & & & MLP Ratio & \multicolumn{5}{c}{4} \\
\cline{4-9}
 & & & \# Block & 6 & 12 & 18 & 18 & 24 \\
\hline
\stitle{4}{32}{3}{2}{512}{768} & \poollayer
 & & & MLP Ratio & \multicolumn{5}{c}{4} \\
 \cline{4-9}
 & & & \# Block & 2 & 4 & 6 & 6 & 8 \\
\hline
\multicolumn{4}{c|}{Parameters~(M)}& 11.9 & 21.4              &  30.8            &  56.1              &  73.4    \\
\hline
\multicolumn{4}{c|}{MACs~(G)}     & 1.8  &  3.4              &  5.0             &  8.8              &  11.6    \\
\bottomrule
\end{tabular}
\endgroup

\vspace{-3mm}
\caption{\textbf{ Configurations of different PoolFormer models.} There are two groups of embedding dimensions, \ie, small size with [64, 128, 320, 512] dimensions and medium size with [96, 196, 384, 768]. Notation ``S24" means the model is in small size of embedding dimensions with 24 PoolFormer blocks in total. The numbers of MACs are counted by \texttt{fvcore}\cite{fvcore} library.
}
\label{tab:model}
\vspace{-4mm}
\end{table}

\label{sec:method}
%-------------------------------------------------------------------------------

%-------------------------------------------------------------------------------
\section{Experimental Setup}
\textbf{Models.} As target model \textit{LM} we use the pretrained 1.3B CroissantLLM~\cite{faysse2024croissantllm}, which we fine-tune on documents containing fuzzy trap sequences. As reference model $\textit{LM}_{\text{ref}}$, we use the pretrained LLaMA-2 7B~\cite{touvron2023llama2} to generate the synthetic reference trap sequences. Finally, as masked language model $\textit{MLM}$, we use RoBERTa~\cite{liu2019roberta} to generate the fuzzy duplicates.

\textbf{Fuzzy trap sequences.} We generate $100$ reference trap sequences using $\textit{LM}_{\text{ref}}$. We specifically control for the sequence length ($L_{\text{ref}}(X_{\text{ref}}) = 100$), and perplexity (between 90 and 100) to minimize variance in memorization, as previous works have shown length and perplexity to correlate with memorization~\cite{meeus2024copyright, carlini2022quantifying}. Unless stated otherwise, we use $k=50$ when selecting top $k$ tokens predicted by masked language model $\textit{MLM}$. For number of replacements we consider $R=\{1, 2, 4, 8, 16, 32\}$.

\textbf{Data.} We inject fuzzy trap sequences in a collection of books available in the public domain. We use the open-source library~\cite{kpullygutenberg} to collect $100$ books made available under a permissive license on Project Gutenberg~\cite{projectgutenberg} but have not been included in the training dataset of CroissantLLM~\cite{faysse2024croissantllm}. The books we selected contain $8.7$M tokens in total (tokenized with CroissantLLM tokenizer). For each of the $100$ reference trap sequences, we inject the $n_{\text{dup}}=10$ fuzzy trap sequences at random into one book and use this collection of modified books as dataset for fine-tuning. 

\textbf{Fine-tuning.} In all of our experiments we fine-tune CroissantLLM~\cite{faysse2024croissantllm} on the $100$ modified books for $1$ epoch. We use its maximum sequence length of $2048$ tokens, a batch size of $6$ and optimizer Adam with constant learning rate \num{3e-6} and weight decay of $0.01$. Unsurprisingly, we find that the extent to which the target model \textit{LM} memorizes the trap sequences at the fixed number of training steps, heavily depends on the learning rate. We elaborate on this in Sec~\ref{section:ablations} and throughout the rest of the experiments consider the learning rate fixed to \num{3e-6}. We argue, however, that the \emph{absolute} extent of memorization does not impact our findings, as we here study how fuzzy trap sequences are memorized \emph{relative} to exact duplication of trap sequences, which have been shown by Meeus et al.~\cite{meeus2024copyright} to be memorized in a real-world scenario. Fine-tuning one model on $100$ trap-injected books took roughly 2 GPU-hours on A100 NVIDIA GPUs.

\textbf{Membership Inference Attack (MIA).} To measure the memorization of the fuzzy trap sequences, we instantiate a sequence-level MIA to infer whether $X_{\text{ref}}$ has been seen by the target model \textit{LM}. For this, we generate $100$ new reference trap sequences, which we do not include in the training dataset and thus consider as \emph{non-members}. We consider the $100$ reference trap sequences that are included in the training dataset as \emph{members}. As MIA methodology, we use the \textit{Ratio} attack~\cite{carlini2021extracting}. For each $X_{\text{ref}}$, either \emph{member} or \emph{non-member}, we compute the target model loss divided by the loss computed using the reference model, which we call \emph{membership score} $\alpha(X_{\text{ref}}) = \mathcal{L}_{\textit{LM}}(X_{\text{ref}}) / \mathcal{L}_{\textit{LM}_{\text{ref}}}(X_{\text{ref}})$. We then use $\alpha(X_{\text{ref}})$ to compute the ROC AUC for the binary membership prediction task and use the AUC to measure to what extent the fully trained target model \textit{LM} memorizes the trap sequences. We compute the AUC on $25$ bootstrapped subsets of members and non-members and report both the mean and standard deviation across all results~\cite{bertail2008bootstrapping}. %\todo{See Q.7 in the checklist: we need to explain more on the error bars}

\textbf{Baselines.} The primary goal of this paper is to quantify how fuzzy trap sequences are memorized compared to exact duplication. For 10 fuzzy trap sequences $\{X_\text{ref}, X_2, X_3, \ldots X_{n_{\text{dup}}}\}$, where each $X_i$ has $R$ tokens replaced compared to $X_{\text{ref}}$, we consider the following baselines. As an upper bound, we consider the exact repetition of the reference trap sequence $X_{\text{ref}}$ for $n_{\text{dup}}=10$ times. As a lower bound we consider a single injection of the reference trap sequence, i.e. $n_{\text{dup}}=1$. 

\textbf{Reproducibility.} We will share the code to reproduce our results in the camera-ready version.
\label{sec:experimental_setup}
%-------------------------------------------------------------------------------


%-------------------------------------------------------------------------------
\section{LLMs have mosaic memory}
\begin{table*}[ht]
    \centering
    \begin{tabular}{ccc|cc}
    \toprule
         & \multicolumn{2}{c}{AUC} & \multicolumn{2}{c}{$C_\text{max}$} \\
        $R$ & Random & Even & Random & Even\\
        \midrule
        $1$ & $0.893 \pm 0.022$ & $0.903 \pm 0.022$ & $91.43 \pm 1.60$ & $91.01 \pm 1.80$ \\ 
        \cmidrule{1-5}
        $2$ & $0.891 \pm 0.026$ & $0.890 \pm 0.025$ & $79.97 \pm 1.35$ & $73.46 \pm 1.86$ \\ 
        \cmidrule{1-5}
        $4$ & $0.869 \pm 0.022$ & $0.861 \pm 0.024$ & $61.40 \pm 1.77$ & $41.26 \pm 0.82$ \\ 
         \cmidrule{1-5}
        $8$ & $0.840 \pm 0.027$ & $0.834 \pm 0.029$ & $42.39 \pm 1.47$ & $21.51 \pm 0.51$ \\  
         \cmidrule{1-5}
        $16$ & $0.783 \pm 0.032$ & $0.754 \pm 0.032$ & $25.90 \pm 0.70$ & $10.90 \pm 0.23$ \\ 
         \cmidrule{1-5}
        $32$ & $0.682 \pm 0.022$ & $0.646 \pm 0.041$ & $13.54 \pm 0.42$ & $5.89 \pm 0.13$ \\ 
         \bottomrule
    \end{tabular}
    \caption{AUC and $C_\text{max}$ (mean and standard deviation) for \emph{Random} and \emph{Even} replacement position distribution.}
    \label{tab:uniform_vs_random_distr}
\end{table*}

Fig.~\ref{fig:auc_vs_r_main} shows how the AUC varies for increasing number of replacements $R$ made to the fuzzy trap sequences. We find that the AUC only drops slightly for smaller values of replacements $R$. For $R=4$, when $n_{\text{dup}}=10$ fuzzy trap sequences are injected, the mean AUC only drops from $0.90$ to $0.87$. Even at $R=32$, when we replace roughly one third of all tokens in the sequence, the mean AUC is $0.68$ and remains significantly higher than a single repetition $n_{\text{dup}}=1$ with a mean AUC of $0.59$. This demonstrates the mosaic memory in the target LLM, i.e. the overlapping fragments of multiple, slightly different sequences contribute to the memorization of the reference trap sequence.

\subsection{Replacement position distribution.} \label{section:spread_replacements}

We now aim to further reduce the chances of fuzzy trap sequences being removed by deduplication, and ensure that token replacements are evenly spread across the sequence. Above, we have uniformly at random sampled tokens to be replaced across fuzzy duplicates. Here, we instantiate the exact same setup, but we split the tokenized reference trap sequence using the MLM tokenizer: $T_{\text{MLM}}(X_{\text{ref}}) = \{t_1,\ldots,t_N\}$ in $R$ equally-sized chunks of size $\lceil\frac{N}{R}\rceil$. We then replace exactly one (selected uniformly at random) token for every chunk. Below we refer to this strategy as \emph{Even}, and to the previously used strategy as \emph{Random}.

We also compute the length of subsequences repeated exactly within clusters of fuzzy duplicates, simulating a sequence-level deduplication algorithm. We define $C_\text{max}$ as the maximum length (in tokens) of a substring shared by at least two fuzzy duplicate sequences within a cluster. We report the mean $C_\text{max}$ across 100 clusters used in our experiments.

Table~\ref{tab:uniform_vs_random_distr} shows that for smaller values of $R$ (up to $R=8$) the AUC for \textit{randomly} and \textit{evenly} distributed token replacements remains highly similar. For larger values of $R$ ($R \geq 16$), the impact of uniform spreading becomes more apparent, leading to a slightly lower AUC. At the same time, $C_\text{max}$ is significantly lower for the evenly distributed token replacements, with the impact more pronounced at higher values of $R$. This demonstrates how a reasonable number of token replacements ($R=4$ or $R=8$) would evade even the sequence-level deduplication with the most aggressive threshold (e.g. $50$ tokens), while retaining significant memorization.

\subsection{MIAs adapted to fuzzy trap sequences.} \label{section:adapted_mia}

So far we have used the unmodified \textit{Ratio} attack~\cite{carlini2021extracting} to infer whether the reference trap sequence has been seen by the target model. Specifically, we compute a single $\alpha(X_{\text{ref}}) = \alpha(X_1)$ for each trap sequence, and do not utilize our knowledge of the fuzzy counterparts $\{X_i \mid i=2 \ldots n_{\text{dup}}\}$. To evaluate whether this could further improve detectability, we now compute the membership score for each of the fuzzy trap sequences, i.e. $\{\alpha(X_i) \mid i=1 \ldots n_{\text{dup}}\}$, and aggregate the membership scores with aggregation function $\mathcal{A}(\cdot)$, i.e. $\alpha_{\mathcal{A}}(X_{\text{ref}}) = \mathcal{A}\left( \{\alpha(X_i) \mid i=1 \ldots n_{\text{dup}}\}\right)$. We then compute $\alpha_{\mathcal{A}}(X_{\text{ref}})$ for each reference trap sequence. We compute AUC on a balanced set of \emph{members} and \emph{non-members}, and so generate fuzzy trap sequences for all non-member sequences too. As aggregation function $\mathcal{A}$ we consider the mean, median, minimum and maximum. We report the MIA AUC for all aggregation functions, and for $R=\{2, 8, 32\}$ for models trained in the main experiment (Fig.~\ref{fig:auc_vs_r_main}). As a baseline, we also provide the previously used MIA based on the reference trap sequence $\alpha(X_{\text{ref}})$ alone.

 Table~\ref{tab:custom_MIA} shows that aggregating the membership score on all fuzzy trap sequences $\alpha_{\mathcal{A}}(X_{\text{ref}})$ does not provide substantial benefits compared to the baseline. We attribute this to the fact that all fuzzy trap sequences $X_i$ only differ from $X_{\text{ref}}$ by $R$ replacements, and therefore the target model loss computed on the reference trap sequence likely captures an aggregation across its fuzzy trap sequences already.

\begin{table*}[ht]
    \centering
    \begin{tabular}{cccc}
    \toprule
         & \multicolumn{3}{c}{AUC} \\
        Aggregation $\mathcal{A}$ & $R=2$ & $R=8$ & $R=32$ \\
        \midrule
        $\alpha(X_{\text{ref}})$ - no aggregation & $0.891 \pm 0.026$ & $0.840 \pm 0.027$ & $0.682 \pm 0.022$ \\ 
         \midrule
         \midrule
        Mean & $0.870 \pm 0.021$ & $0.828 \pm 0.029$ & $0.683 \pm 0.026$ \\ 
         \cmidrule{1-4}
        Median & $0.869 \pm 0.028$ & $0.824 \pm 0.030$ & $0.692 \pm 0.039$ \\  
         \cmidrule{1-4}
        Minimum & $0.881 \pm 0.021$ & $0.823 \pm 0.030$ & $0.624 \pm 0.040$  \\ 
         \cmidrule{1-4}
        Maximum & $0.879 \pm 0.030$ & $0.821 \pm 0.034$ & $0.679 \pm 0.041$  \\ 
         \bottomrule
    \end{tabular}
    \label{tab:custom_MIA}
    \caption{AUC (mean and standard deviation) for MIA methodologies adapted to fuzzy trap sequences.}
\end{table*}



\subsection{Ablation studies.} \label{section:ablations}

\textbf{Token replacement hyperparameters.} We here explore whether the semantic coherence of fuzzy trap sequences $X_i$ affect memorization. For this, we vary the value $k$ when sampling replacements from the top-$k$ tokens predicted by the MLM when a fixed number $R=8$ of replacements are made (see Sec.~\ref{sec:method_gen_fuzzy}). Recall that thus far we only considered $k=50$ and that for $k=|\mathcal{V}_{\text{MLM}}|$ -which is $50,000$ for RoBERTa~\cite{liu2019roberta}-, we effectively randomly sample a token from the entire MLM vocabulary $\mathcal{V}_{\text{MLM}}$. In addition to sampling uniformly from the top $k$ tokens, we also consider sampling directly from the full probability distribution predicted by the MLM (\textit{Sample directly}).

Fig.~\ref{fig:robustness}(a) shows that the AUC decreases as $k$ increases. We find $k$ and AUC to be strongly correlated with a Spearman coefficient of $-0.47$ and a p-value of \num{4e-11}, suggesting that semantic coherence is important for mosaic memorization. Further, Fig.~\ref{fig:robustness}(b) compares the AUC for $k=50$ and $k=|\mathcal{V}_{\text{MLM}}|$ for increasing values of $R$. For a smaller amount of replacements $R$, the AUC for $k=50$ and random token replacement remains very similar. For larger values of $R$ the mean estimations start to diverge, yet with no statistically significant difference.

\textbf{Impact of learning rate.} We here show, that the key outcome of our experiments - the \emph{relative} memorization of fuzzy duplicates compared to exact duplicates holds regardless of the \emph{absolute} level of memorization. At a fixed number of training steps we can control the baseline memorization with varying learning rate. So far in previous experiments, we have considered a fixed learning rate of $\num{3e-6}$. Fig.~\ref{fig:robustness} (b) shows how for increasing learning rate the AUC for fuzzy trap sequences remains consistently slightly lower than the upper bound ($n_{dup} = 10$), while remaining significantly higher than the lower bound ($n_{dup} = 1$). This suggests that fuzzy trap sequences are also memorized in lower and higher memorization regimes. 

\begin{figure*}[ht]
\centering
\subfigure{
\includegraphics[width=0.3\linewidth]{figures/vary_k.pdf}
}
\subfigure{
\includegraphics[width=0.3\linewidth]{figures/AUC_vs_R_both_strategies.pdf}
}
\subfigure{
\includegraphics[width=0.3\linewidth]{figures/vary_lr.pdf}
}

    \caption{\textbf{Ablation.} MIA AUC (mean and standard deviation) for fuzzy trap sequences for (a) varying $k$ for $R=8$, (b) $k=50$ and $k=50,000$ across $R$ and (c) varying learning rate used in fine-tuning.} 
\label{fig:robustness}
\end{figure*} 

\label{sec:experiments}
%-------------------------------------------------------------------------------

%-------------------------------------------------------------------------------
\section{Implications for measuring post-hoc LLM memorization.}
\begin{figure*}[ht]
\centering
\subfigure{
\includegraphics[width=0.4\linewidth]{figures/near_duplicates_proportion.pdf}
}
\subfigure{
\includegraphics[width=0.4\linewidth]{figures/near_duplicates_percentiles.pdf}
}
\caption{\textbf{Fuzzy duplicate sequences in The Pile.} (a) Proportion of sequences with at least one fuzzy duplicate given the maximum threshold $R_\text{max}$ (b) An increase in total number of repetitions if accounting for fuzzy duplicates given the maximum threshold $R_\text{max}$.}
\label{fig:near_dups}
\end{figure*}


Prior work on memorization in LLMs often relies on the natural presence of the duplicated sequences in large text datasets~\cite{carlini2021extracting,carlini2022quantifying,kandpal2022deduplicating, ippolito2022preventing}. For instance, seminal work by Carlini et al.~\cite{carlini2022quantifying} identifies sequences repeated multiple times in The Pile~\cite{pile}, and uses this to study the relationship between the number of repetitions and memorization. Kandpal et al.~\cite{kandpal2022deduplicating} adopts a similar approach with C4~\cite{2019t5} and OpenWebText~\cite{Gokaslan2019OpenWeb}.

We argue that relying on naturally occurring duplicates might overestimate the impact of exact duplication on memorization. Instead, some memorization could be attributed to fuzzy duplicates. Any memorization study using the set of exact duplicates found in a large text dataset is susceptible to the confounding factor from the presence of fuzzy duplicates. As we show below, the number of fuzzy duplicates tend to be distributed in a highly non-uniform fashion, likely introducing bias when studying memorization.

To verify this, we here re-create the set of sequences duplicated in The Pile considered by prior work. We use the code provided by Lee et al.~\cite{lee2022deduplicating}\footnote{Distributed under Apache License 2.0.} to build a suffix array and identify sequences occurring at least twice in the dataset. We consider sequences of 100 BPE (GPT-2) tokens~\cite{radford2019language}. As the original The Pile has been taken down due to copyright violations in some of its subsets, we study the non-copyrighted version of The Pile~\cite{pile_uncopyrighted}, which covers roughly 80\% of the original data\footnote{The rest of The Pile is distributed under a range of permissive licences.}. While this affects the absolute counts of exact and fuzzy duplicates, we effectively report the lower bound on both counts, which allows us to study their relative prevalence in large datasets. Data processing took us $\sim96$ hours on a machine with 96 CPUs, 500GB of RAM and 5TB of available disk space.

Following the approach proposed by Carlini et al.~\cite{carlini2022quantifying}, we group duplicate sequences into buckets by the number of exact repetitions, where the $n$-th bucket contains sequences repeated between $2^{n/4} \leq n_{dup} < 2^{(n+1)/4}$ times in the dataset. We consider $n \in [20, 40)$, thus covering sequences repeated between $32$ and $1024$ times. We then sample $100$ random sequences from each bucket (2000 sequences overall) and search for their fuzzy duplicates. As in the experiments above, we define a fuzzy duplicate as a sequence of the same length with up to a certain number of tokens ($R_\text{max}$) replaced between the two sequences (we consider $R_\text{max} \leq 8$). Note that the notion of the number of tokens replaced between two sequences of the same length is equivalent to the Hamming distance - we here adopt the former for consistency.  
For computational reasons, we only look for fuzzy duplicates among sequences identified as duplicates themselves, i.e. repeated in the original dataset at least twice. Our fuzzy duplicate counts, therefore, only provide a lower bound estimation of the total number of fuzzy duplicates for a given sequence.

Our results show that a significant proportion of sequences duplicated in The Pile have additional fuzzy duplicates in the dataset. Fig.~\ref{fig:near_dups}(a) shows that almost 15\% and 30\% of sequences have at least one fuzzy duplicate with only a $R_\text{max}=1$ and $R_\text{max}=4$ token replacements respectively (out of a 100 tokens in the sequence). Strikingly, Fig.~\ref{fig:near_dups}(b) shows that for 10\% of all duplicate sequences, the number of fuzzy duplicates ($R_\text{max}=4$) is larger than the number of exact repetitions. For instance, 10\% of all sequences repeated $32$ times in The Pile have more than $32$ additional fuzzy duplicates. This data suggests that using naturally occurring duplicates to study memorization is prone to unforeseen confounding factors, such as the presence of fuzzy duplicates illustrated above.
\label{sec:naturally_occuring}
%-------------------------------------------------------------------------------

%-------------------------------------------------------------------------------
\section{Discussion and Future Work}

\textbf{Feasibility of copyright traps.} We here focus on addressing one core limitation of implementing copyright traps in practice: the potential accidental removal during training data deduplication~\cite{lee2022deduplicating,kudugunta2024madlad,penedo2023refinedweb}. We now consider other potential limitations. First, traps might be affected by quality filters. For instance, prior work has implemented filtering based on language ~\cite{penedo2023refinedweb,soldaini2024dolma}, certain heuristics (e.g. removing sequences with a high ratio of special characters)~\cite{kudugunta2024madlad,rae2021scaling,laurenccon2022bigscience} and perplexity~\cite{wenzek2019ccnet}. We here argue that trap sequences are designed to resemble human-generated text, and preprocessing techniques sufficiently aggressive to remove traps are also likely to hurt model utility. While we do not believe any of the current filtering techniques to affect fuzzy copyright traps, we leave for future work to explore if the wide variety of filtering mechanisms~\cite{albalak2024survey} could remove trap sequences. Second, (fuzzy) trap sequences likely impact the readability of the original content. However, as also argued when introduced~\cite{meeus2024copyright,wei2024proving}, traps injected in web-based publications can be made invisible to a human reader yet picked up by a scraper. Lastly, we acknowledge that traps can potentially be removed by an informed and motivated adversary with techniques targeted specifically at trap sequences.

\textbf{Privacy and confidentiality.} Our findings highlight new challenges in addressing privacy issues associated with LLMs. Sec.~\ref{sec:naturally_occuring} elaborates on challenges for post-hoc studies on LLM memorization. Further, we argue that deduplication does not necessarily eliminate privacy risks in LLMs~\cite{kandpal2022deduplicating}, as some duplicate sequences also tend to have many fuzzy duplicates in the same dataset. Thus, even the more aggressive sequence-level deduplication techniques would not be able to remove fuzzy duplicates and leave the data prone to memorization. 

\textbf{Potential for misuse.} With the injection of carefully designed fuzzy sequences and their memorization in a target LLM becoming feasible, we also acknowledge the potential misuse of this technology. For instance, malicious actors could leverage similar techniques to inject targeted misinformation into public content, which could potentially be propagated through LLMs deployed in practice. Future work could study whether our findings on memorization apply to the spread of misinformation and investigate potential mitigation strategies.  


\label{sec:discussion}
%-------------------------------------------------------------------------------

%-------------------------------------------------------------------------------
\section{Conclusion}
\vspace{-2mm}
\section{Conclusion and Broader Impact}
\label{sec:conclusion}

We have introduced \textsc{diamond}, a reinforcement learning agent trained in a diffusion world model. 
We explained the key design choices we made to adapt diffusion for world modeling and to make our world model stable over long time horizons with a low number of denoising steps.
\textsc{diamond} achieves a mean human normalized score of $1.46$ on the well-established Atari 100k benchmark; a new best among agents trained entirely within a world model. 
We analyzed our improved performance in some games and found that it likely follows from better modeling of critical visual details.
We further demonstrated \textsc{diamond}'s diffusion world model can successfully model 3D environments and serve as a real-time neural game engine by training on static \textit{Counter-Strike: Global Offensive} gameplay.

World models constitute a promising direction to address sample efficiency and safety concerns associated with training agents in the real world. However, imperfections in the world model may lead to suboptimal or unexpected agent behaviors. We hope that the development of more faithful and interactive world models will contribute to broader efforts to further reduce these risks.

%The development of more visually faithful world models such as \textsc{diamond} provides a promising avenue to further reduce these risks.% associated with deploying agents trained in simulation. 

%hope that developing more visually faithful world models such as \textsc{diamond} constitutes a promising avenue to further reduce these risks. 

%In general, the deployment of agents in the real world raises safety concerns. Training in a world model reduces these risks by minimizing the time spent in the real environment, but imperfect world models may lead to unexpected agent behaviors. The development of more visually faithful world models such as \textsc{diamond} provides a promising avenue to further reduce the risks associated with deploying agents trained in simulation. 



%Our work considers the training of autonomous agents using a world model. 

%Taking a step back, the deployment of autonomous agents in the real world raises safety concerns regarding the potential harm they may cause, and training in a world model reduces these risks by minimizing the time the agent spends interacting with the environment. However, imperfect world models may lead to unexpected behaviors upon deployment of agents in the real-world, and the development of more realistic world models such as \textsc{diamond} provides a promising avenue to further reduce the risks associated with deploying agents trained in simulation.

%Additionally, as with all advances in the field of Machine Learning, there are many other potential societal consequences of our work, but none of which we feel must be specifically highlighted here.

% We have introduced \textsc{diamond}, a visually faithful diffusion world model for training agents in imagination. We explained key design choices for \textsc{diamond}'s diffusion world model that enables our implementation to remain stable over long time horizons with a low number of denoising steps. We demonstrate that this improved visual quality translates into improved agent performance on the well-established Atari 100k benchmark. As a result, \textsc{diamond} achieves a new state of the art among world model agents on this benchmark, with a mean human normalized score of $1.46$. The performance improvements are particularly evident for visually challenging environments, which is a promising sign that \textsc{diamond} could scale to visually complex real-world environments.
% % which is promising for scaling \textsc{diamond} to more visually complex real-world environments.

% This paper considers the training of autonomous agents using a world model. The deployment of autonomous agents in the real world raises safety concerns regarding the potential harm that may be caused by an agent's actions. Training in simulation reduces these risks by reducing the time the agent spends interacting with the environment. However, imperfect world models may lead to unexpected behaviors in the real world. The development of more realistic world models should therefore reduce the risk associated with deploying agents trained in this manner. Additionally, as with all advances in the field of Machine Learning, there are many other potential societal consequences of our work, but none of which we feel must be specifically highlighted here.
\label{sec:conclusion}
%-------------------------------------------------------------------------------

\bibliographystyle{plain}
\bibliography{bibliography.bib}

\onecolumn
\newpage
\appendix

\section{Characterizing fuzzy trap sequences}
\label{app:fuzzy_trap_sequences}
In Sec.~\ref{sec:method_gen_fuzzy}, we describe the generation of fuzzy trap sequences from a reference trap sequence $(X_{\text{ref}})$. First, we tokenize the reference trap sequence using the tokenizer of a masked language model MLM, resulting in $T_{\text{MLM}}(X_{\text{ref}}) = \{t_1,\ldots,t_N\}$. Due to the different tokenization of $T_{\text{ref}}$ and $T_{\text{MLM}}$, $N$ differs from $L_{\text{ref}}$. Fig~\ref{fig:tokens_ppl} (a) shows how the $N=|T_{\text{MLM}}(X_{\text{ref}})|$ is distributed for $100$ reference trap sequences - which all have $L_{\text{ref}}=100$. 

Next, to generate fuzzy trap sequences, we replace $R$ tokens by sampling from the top $k$ tokens predicted by the MLM. For smaller values of $k$, we expect more meaningful token replacements to be made than for larger values of $k$. Indeed, when $k$ equals the size of the vocabulary of the MLM, or $k=|\mathcal{V}_{\text{MLM}}|$, we would effectively replace the token with a random other token regardless of its context in the reference trap sequence. This will likely impact the difference across fuzzy duplicates. We quantify this difference with the commonly used notion of \emph{perplexity}. For the sequence of textual characters $X$, tokenized as $T(X) = \{t_1,\ldots,t_L\}$, we denote the loss of language model $\textit{LM}$ with tokenizer $T$ as:

\begin{equation}
\label{eq:loss}
\mathcal{L}_{\textit{LM}}(X) = -\frac{1}{L}\sum_{i=1}^{L} \log\left( \textit{LM}_{\theta}(t_i | t_1 \ldots, t_{i-1})\right) 
\end{equation}

Perplexity is then computed as the exponent of the loss, i.e. $\mathcal{P}_{\textit{LM}}(X) = \exp\left(\mathcal{L}_{\textit{LM}}(X)\right)$. The higher the value of perplexity, the more 'surprised' language model $\textit{LM}$ is to observe sequence $X$. 

We now compute how the perplexity of the fuzzy trap sequences differs from the perplexity of the reference trap sequence. We generate synthetic reference trap sequences of length $L_{\text{ref}}(X_{\text{ref}})=100$ and perplexity $90 \leq \mathcal{P}_{\textit{LM}_{\text{ref}}}(X_{\text{ref}}) < 100$. Making $R$ replacements in such a sequence will have undoubtedly altered this perplexity. Fig~\ref{fig:tokens_ppl} (b) shows how the perplexity of fuzzy trap sequences computed using $\textit{LM}_{\text{ref}}$ varies for increasing number of replacements $R$. We find that, when more tokens are replaced, the perplexity indeed increases compared to the perplexity of the reference trap sequence $X_{\text{ref}}$, and more rapidly so when token replacements are made with higher values of $k$..  

\begin{figure*}[ht]
\centering
\subfigure{
\includegraphics[width=0.35\linewidth]{figures/MLM_tokens.pdf}
}
\subfigure{
\includegraphics[width=0.35\linewidth]{figures/llama_ppl_neardupls.pdf}
}
\caption{\textbf{Characterizing fuzzy trap sequences.} (a) The distribution of number of tokens $|T_{\text{MLM}}(X_{\text{ref}})|$ of $100$ reference trap sequences. (b) The perplexity computed using $\textit{LM}_{\text{ref}}$ each for $100$ fuzzy trap sequences for token replacement strategies for different values of $k$ across values for $R$.} 
\label{fig:tokens_ppl}
\end{figure*} 

\end{document}