\begin{tcolorbox}[title = {Prompt Template For Summary},breakable]
\label{summary}
\small
Your objective is to summarize the provided text: [begin] \{instance\} [end], within 100 words, including the relevant information for the use case in the summary as much as possible. \\
    The summary will represent the input data for clustering in the next step. \\
    Be concise and clear. \\
    Do not add phrases like "This is the summary of" or "Summarized text:"... \\
    Do not include any line breaks in the summary. \\
    Provide your answer in English only. \\
    Your comprehensive output should mirror this structure: \{\{"summary": ""\}\}. \\
\end{tcolorbox}

\begin{tcolorbox}[title = {Prompt Template For First-level Tagging},breakable]
\label{first-level tagging}
\small
You are an advanced tagging system designed to identify the most pertinent theme within a given text passage: [begin] \{instance\} [end].\\
    Your role is to analyze the text meticulously and choose the most fitting tag from the predefined list: Natural Sciences, Humanities and Social Sciences, Industrial Manufacturing, Medical and Health, Agriculture and Forestry, Energy and Mining, Finance and Real Estate, Education, Transportation, Technology and Internet, Law, Military, Travel and Tourism, Entertainment, Arts and Culture, Emotional Psychology, Fashion and Beauty, Sports, Home and Lifestyle, Public Administration, and Social Events.\\
    Your task is to determine the single most relevant tag that encapsulates the primary theme of the text. \\
    Your selection should be substantiated with a detailed explanation, elucidating why this tag is the most accurate representation of the text's central subject matter.\\
    Your output should follow this structure: \{\{"tag": "Selected Tag", "explanation": "Provide a detailed explanation in English  on why this is the most fitting tag."\}\}.
\end{tcolorbox}

\begin{tcolorbox}[title = {Prompt Template For Second-level And Third-level Tagging},breakable]
\label{second-level and third level tagging}
\small
You are an advanced tagging system designed to categorize a given text passage related to the first level tag "\{first\_level\_tag\}" into specific second and third-level tags within a predefined hierarchy. \\
    Here is the tag hierarchy for the "\{first\_level\_tag\}" category in json format: \{tag\_tree\} \\
    Here is the given text passage: [begin] \{instance\} [end]. \\
    Your task is to analyze the text snippet above and assign the most fitting second-level and third-level tags, ensuring both tags align within the same hierarchical path. \\
    The output should precisely reflect the main focus of the text, justifying why these tags are the most suitable choices. \\
    Your output should follow this structure: \{\{"second\_level\_tag": "Selected Second Level Tag", "third\_level\_tag": "Selected Third Level Tag", "explanation": "Provide a detailed explanation in English on why these tags accurately represent the text's core content."\}\}.
\end{tcolorbox}

