We first prove the condition that $X^T u = 0$ is equivalent to pseudo-Euclidean
centering.

\begin{proof}[Proof of Lemma~\ref{lmm:pe-centered}]
In the hyperboloid model, the variance term $\Psi$ can be written as
\begin{align*}
  \Psi(z; x_1, x_2, \ldots, x_n)
  &=
  \sum_{i=1}^k \sinh^2(d_H(x_i, z)) \\
  &=
  \sum_{i=1}^k \left( \cosh^2(d_H(x_i, z)) - 1 \right) \\
  &=
  \sum_{i=1}^k \left( (x_i^T Q z)^2 - 1 \right) \\
  &=
  \sum_{i=1}^k \left( (x_{0,i} z_0 - \vec{x}_i^T \vec{z})^2 - 1 \right) \\
  &=
  \sum_{i=1}^k \left( \left(x_{0,i} \sqrt{ 1 + \| \vec{z} \|^2 } - \vec{x}_i^T \vec{z} \right)^2 - 1 \right).
\end{align*}
The derivative of this with respect to $\vec{z}$ is
\begin{align*}
  \nabla_{\vec{z}} \Psi(z; x_1, x_2, \ldots, x_n)
  &=
  2 \sum_{i=1}^k 
  \left(x_{0,i} \sqrt{ 1 + \| \vec{z} \|^2 } - \vec{x}_i^T \vec{z} \right)
  \left(x_{0,i} \frac{\vec{z}}{\sqrt{ 1 + \| \vec{z} \|^2 }} - \vec{x}_i \right).
\end{align*}
At $\vec{z} = 0$ (or equivalently $z = e_0$), this becomes
\begin{align*}
  \left. \nabla_{\vec{z}} \Psi(z; x_1, x_2, \ldots, x_n) \right|_{\vec{z} = 0}
  &=
  2 \sum_{i=1}^k 
  \left(x_{0,i} \sqrt{ 1 + 0 } - 0 \right)
  \left(x_{0,i} \frac{0}{\sqrt{ 1 + 0 }} - \vec{x}_i \right) \\
  &=
  -2 \sum_{i=1}^k x_{0,i} \vec{x}_i.
\end{align*}
If we define the matrix $X \in \R^{n \times k}$ such that $X^T e_i = \vec{x}_i$ and the vector $u \in \R^k$ such that $u_i = x_{0,i}$, then
\begin{align*}
  \left. \nabla_{\vec{z}} \Psi(z; x_1, x_2, \ldots, x_n) \right|_{\vec{z} = 0}
  &=
  -2 \sum_{i=1}^k X^T e_i e_i^T u \\
  &=
  -2 X^T u.
\end{align*}
\end{proof}


\paragraph*{Centering and Geodesic Submanifolds}
A well-known property of the hyperboloid model is that the geodesic submanifolds on $\mathbb{M}_r$ are exactly the linear subspaces of $\R^{r+1}$ intersected with the hyperboloid model (Corollary A.5.5. from~\cite{Benedetti}).
This is analogous to how the affine subspaces of $\R^r$ are the linear subspaces of $\R^{r+1}$ intersected with the homogeneous-coordinates model of $\R^r$.
Notice that this directly implies that any geodesic submanifold can be written as a geodesic submanifold centered on any of the points in that manifold.
To be explicit with the definitions:
\begin{definition}
A geodesic submanifold is a subset $S$ of a manifold such that for any two points $x, y \in S$, the geodesic from $x$ to $y$ is fully contained within $S$.
\end{definition}

\begin{definition}
A geodesic submanifold rooted at a point $x$, given some local subspace of its tangent bundle $T$, is the subset $S$ of the manifold that is the union of all the geodesics through $x$ that are tangent at $x$ in a direction contained in $T$.
\end{definition}

Now we prove that centering with the pseudo-Euclidean mean preserves geodesic submanifolds.

First, we need the following technical lemma showing that projection to a manifold decreases distances.
\begin{lemma}
  \label{lmm:manifold-projection}
  Consider a dimension-$r$ geodesic submanifold $S$ and point $\bar x$ outside of it.
  Let $z$ be the projection of $\bar x$ onto $S$.
  Then for any point $x \in S$, $d_H(x, \bar x) > d_H(x, z)$.
\end{lemma}
\begin{proof}
  As a consequence of the projection, the points $x, z, \bar x$ form a right angle.
  From the hyperbolic Pythagorean theorem, we know that
  \[
    \cosh(d_H(x, \bar x)) = \cosh(d_H(x, z)) \cosh(d_H(z, \bar x)).
  \]
  Since $\cosh$ is increasing and at least $1$ (with equality only at $\cosh(0) = 1$), this implies that
  \[
    d_H(x, \bar x) > d_H(x, z).
  \]
\end{proof}

\begin{lemma}
If some points $x_1, \ldots, x_k$ lie in a dimension-$r$ geodesic submanifold $S$, then both a Karcher mean and a pseudo-Euclidean mean lie in this submanifold.
Equivalently, if the points lie in a submanifold, then this submanifold can be written as centered at the Karcher mean or the pseudo-Euclidean mean. 
\label{lemmaSubmanifoldCentering}
\end{lemma}
\begin{proof}
Suppose by way of contradiction that there is a Karcher mean $\bar x$ that lies outside this submanifold $S$.
Then, consider the projection $z$ of $\bar x$ onto $S$.
From Lemma~\ref{lmm:manifold-projection}, projecting onto $S$ has strictly decreased the distance to all the points on $S$.

As a result, the Frechet variance
\[
  \sum_{i=1}^k d_H^2(x_i, \bar x)
\]
also decreases when $\bar x$ is projected onto $S$.
From this, it follows that there is a minimum value of the Frechet variance (a Karcher mean) that lies on $S$.
An identical argument works for the pseudo-Euclidean distance, since the pseudo-Euclidean distance uses a variance that is just the sum of monotonically increasing functions of the hyperbolic distance.
\end{proof}

\begin{lemma}
Given some pairwise distances $d_{i,j}$, if it is possible to embed the distances in a dimension-$r$ geodesic submanifold rooted and centered at a pseudo-Euclidean mean, then it is possible to embed the distances in a dimension-$r$ geodesic submanifold rooted and centered at a Karcher mean, and vice versa.
\end{lemma}
\begin{proof}
Suppose that it is possible to embed the distances as some points $x_1, \ldots, x_k$ in a dimension-$r$ geodesic submanifold $S$.
Then, by Lemma~\ref{lemmaSubmanifoldCentering}, $S$ contains both a Karcher mean $\bar x$ and a pseudo-Euclidean mean $\bar x_P$ of these points.
If we reflect all the points such that $\bar x$ is reflected to the origin, then the new reflected points will also be an embedding of the distances (since reflection is isometric) and they will also be centered at the origin.
Furthermore, we know that they will still lie in a dimension-$r$ submanifold (now containing the origin) since reflection also preserves the dimension of geodesic submanifolds.
So the reflected points that we have constructed are an embedding of $d_{i,j}$ into a dimension-$r$ geodesic submanifold rooted and centered at a Karcher mean.
The same argument will show that (by reflecting $\bar x_P$ to the origin instead of $\bar x$) we can construct an embedding of $d_{i,j}$ into a dimension-$r$ geodesic submanifold rooted and centered at the pseudo-Euclidean mean.
This proves the lemma.
\end{proof}

%%% Local Variables:
%%% mode: latex
%%% TeX-master: "hyperbolic_arxiv"
%%% End:
