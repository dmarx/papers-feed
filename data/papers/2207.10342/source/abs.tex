\begin{abstract}
Prompted models have demonstrated impressive few-shot learning abilities.
Repeated interactions at test-time with a single model, or the composition of multiple models together, further expands capabilities. These compositions are probabilistic models, and may be expressed in the language of graphical models with random variables whose values are complex data types such as strings. Cases with control flow and dynamic structure require techniques from probabilistic programming, 
which allow implementing disparate model structures and inference strategies in a unified language.
We formalize several existing techniques from this perspective, including scratchpads / chain of thought, verifiers, STaR, selection-inference, and tool use. We refer to the resulting programs as \emph{language model \cascades}.
%Furthermore, the few shot learning abilities of large language models may be used to amortize inference across tasks.
\end{abstract}