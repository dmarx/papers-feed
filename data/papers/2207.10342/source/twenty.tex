\subsection{Twenty questions}
\label{sec:twenty}

In this section, we discuss experimental results using \cascades\ to solve the
 ``Twenty Questions'' task from BigBench \citep{bigbench}.
This task  involves a conversation between two agents, Alice and Bob.
Both agents are presented with the rules of the game, and Alice is additionally presented with a concept (e.g. `apple') to describe.
Bob has to guess the concept by asking a series of  questions
$B_t$ of the form ``Is it X?'', to which Alice answers
$A_t \in \{ \text{`Yes.'}, \text{`No.'}\}$.
We repeat this process until Bob guesses correctly, or we hit the limit of $T$ rounds.
This can be thought of as a pair of interacting Markov chains, which exchange strings, until some final end state is reached,
 as illustrated in \cref{fig:twenty}.


\begin{figure}[h!]
\centering
% \begin{tikzpicture}
%   % Define nodes
%   \node[obs]    at (0,0)       (C) {\tiny{\texttt{CONCEPT}}};
%   \node[latent, right=1.4cm of C]         (A1) {$A_1$};
%   \node[latent, right=0.9cm of A1]         (A2) {$A_2$};
% %   \node[latent, right=0.6cm of A2]         (A3) {$A_3$};
%   \node[latent, above=0.6cm of A1, xshift=-.8cm]         (B1) {$B_1$};
%   \node[latent, right=0.9cm of B1]         (B2) {$B_2$};
% %   \node[latent, right=0.6cm of B2]         (B3) {$B_3$};

%   \node[obs, left=0.75cm of B1]           (R) {\tiny{\texttt{RULES}}};

%   \node[right=0.3cm of B2] (label1) {\Large\textbf{. . .}};
%   \node[right=0.3cm of A2] (label1) {\Large\textbf{. . .}};


%   % Connect the nodes
%   \edge {R} {B1} ; %
%   \edge {C,R,B1} {A1} ; %
%   \edge {R,A1,B1,B2} {A2} ; %
% %   \edge {A2,B2,B3} {A3} ; %
%   \edge {R} {B1} ; %
%   \edge {A1,B1} {B2} ; %
% %   \edge {A2,B2} {B3} ; %
%   \draw [->] (R) to [out=40,in=140] (B2);
%   \draw [->] (C) to [out=-40,in=-140] (A2);
% %   \draw [->] (Q) to [out=40,in=140] (A);
% %   \draw [->] (S) to [out=40,in=140] (A);
% \end{tikzpicture}

\begin{tikzpicture}
  \node[obs]       (R) {\tiny{\texttt{RULES}}};

  \node[obs, below=0.4cm of R]       (C) {\tiny{\texttt{CONCEPT}}};

% stream
  \node[latent, double, minimum size=0.22cm, right=0.6cm of R, yshift=-0.3cm]       (BS1) {};
  \node[latent, double, minimum size=0.22cm, right=0.65cm of BS1]       (BS2) {};
  \node[latent, double, minimum size=0.22cm, right=0.65cm of BS2]       (BS3) {};
  \node[latent, double, minimum size=0.22cm, right=0.65cm of BS3]       (BS4) {};
  \node[latent, double, minimum size=0.22cm, right=0.65cm of BS4]       (BS5) {};

  \node[latent, above=0.4cm of BS1, xshift=0.35cm]         (B1) {$B_1$};
  \node[latent, above=0.4cm of BS3, xshift=0.35cm]         (B2) {$B_2$};
%   \node[latent, right=0.9cm of B1]         (B2) {$B_2$};

  \node[latent, below=0.4cm of BS2, xshift=0.35cm]         (A1) {$A_1$};
  \node[latent, below=0.4cm of BS4, xshift=0.35cm]         (A2) {$A_2$};


  \node[right=0.3cm of BS5] (label1) {\Large\textbf{. . .}};

    \edge {R} {BS1} ; 
    \edge {BS1} {BS2}; 
    \edge {BS2} {BS3}; 
    \edge {BS3} {BS4}; 
    \edge {BS4} {BS5}; 

    \edge {BS1} {B1} ;
    \edge {B1} {BS2} ;
    \edge {BS2} {A1} ;
    \edge {A1} {BS3} ;
    \edge {BS3} {B2} ;
    \edge {B2} {BS4} ;
    \edge {BS4} {A2} ;
    \edge {A2} {BS5} ;

    \edge {C} {A1} ;
      \draw [->] (C) to [out=-25,in=-155] (A2);


\end{tikzpicture}

% \begin{tikzpicture}

%   \node[obs]    at (0,0)       (C) {\tiny{\texttt{CONCEPT}}};
%   \node[obs, above=0.3cm of C]           (R) {\tiny{\texttt{RULES}}};

% %% alice stream
%   \node[latent, double, minimum size=0.22cm, right=0.6cm of C, yshift=0.15cm]       (AS1) {};
%   \node[latent, double, minimum size=0.22cm, right=0.65cm of AS1]       (AS2) {};
%   \node[latent, double, minimum size=0.22cm, right=0.65cm of AS2]       (AS3) {};
%   \node[latent, double, minimum size=0.22cm, right=0.65cm of AS3]       (AS4) {};
%   \node[latent, double, minimum size=0.22cm, right=0.65cm of AS4]       (AS5) {};

% %% bob stream
%   \node[latent, double, minimum size=0.22cm, above=0.6cm of AS1]       (BS1) {};
%   \node[latent, double, minimum size=0.22cm, right=0.65cm of BS1]       (BS2) {};
%   \node[latent, double, minimum size=0.22cm, right=0.65cm of BS2]       (BS3) {};
%   \node[latent, double, minimum size=0.22cm, right=0.65cm of BS3]       (BS4) {};
%   \node[latent, double, minimum size=0.22cm, right=0.65cm of BS4]       (BS5) {};

%   \node[latent, above=0.4cm of BS1, xshift=0.3cm]         (B1) {$B_1$};
%   \node[latent, right=0.9cm of B1]         (B2) {$B_2$};

%   \node[latent, below=.4cm of AS2, xshift=0.3cm]         (A1) {$A_1$};
%   \node[latent, right=0.9cm of A1]         (A2) {$A_2$};

%     \edge {BS1} {B1} ;
%     \edge {B1} {BS2,AS2}

%     \edge {C,R} {AS1} ; 
%     \edge {AS1} {AS2}; 
%     \edge {AS2} {AS3}; 
%     \edge {AS3} {AS4}; 
%     \edge {AS4} {AS5}; 
%     \edge {R} {BS1} ; 
%     \edge {BS1} {BS2}; 
%     \edge {BS2} {BS3}; 
%     \edge {BS3} {BS4}; 
%     \edge {BS4} {BS5}; 


% \end{tikzpicture}



\caption{Twenty questions.}
\label{fig:twenty}
\end{figure}


% \begin{align*}
% \text{question}, \text{facts} \ &\sim \text{Tasks} \\
% \text{selection} &\sim S(\text{question}, \text{facts}) \\
% \text{inference} &\sim S(\text{question}, \text{selection}) \\
% \text{answer} &\sim S(\text{question}, \text{inference}) \\
% \text{reward} &\sim \text{Judge}(\text{answer}, \text{question})
% \end{align*}

The goal is to infer what questions Bob should ask to guess the concept as quickly as possible. This can be cast as a reinforcement learning problem with string-valued actions, or equivalently as an inference problem where we condition on the goal state that $A_T=\text{`yes'}$ for the soonest possible $T$ (c.f., planning as inference \cite{rl_inference}). %\todo{JSD: This is inaccurate -- Alice's answers can be yes or no for any question, regardless of whether Bob correctly guessed the concept.}.
%, goal-conditioned policies such as decision-transformer \cite{decision_transformer} and upside-down RL \cite{upsidedown_rl})

In our current preliminary experiments, we use a forward sampling
approach (aka ancestral sampling), in which 
we sample 50 conversations per concept with temperature $1.0$.
We consider a trial successful if the target concept appears in $B_t$.
(i.e., Bob guesses the right answer).
We reject a sampling chain early if it is ``malformed''
(e.g., Bob generates a response that is not a question).

%at least one of these samples has
% $A_t=\text{`yes'}$ for some $t \leq T$
%[JSD: I think this should rather be that we accept the chain if the concept being communicated appears in $B_t$? $A_t$ can be yes or no for any question.]

% TODO: Add this back in after deanonymized
%We use the pretrained "Lamda" LLM with 137B parameters \cite{lamda}.

Bob's turn starts with `Is the concept' which we complete with the LM. Then we let Alice generate an answer;
we post-process Alice's response
by replacing all mentions of the 
true concept with the generic  word ``concept", to prevent information leakage. 
%\rif{Why not use rejection sampling to constrain Alice to answer yes or no, which we said was the rule?} We repeat this for 10 rounds.
%\ddohan{We should have - no good reason.}
Using the LaMDA 137B large LM \citep{lamda},
we find that the model is able to solve $29\%$ of the tasks. %\rif{Is that good?}
See Appendix~\ref{app:20q-details} for more details.
