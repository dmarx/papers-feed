
\section{Efficiency considerations and improvements}
\label{app:sec:efficiency_improvements}


\begin{wraptable}{r}{0.5\textwidth}
\vspace{-4mm}
\caption{\textbf{3-shot and light variants} Performance with different inference-time adaptation budgets.
}
\vspace{-3.5mm}
\centering
\resizebox{0.45\textwidth}{!}{
\begin{tabular}{lc}
\toprule
\textbf{Method} & \textbf{ARC-Challenge} \\
\midrule
\llama & {\normalsize 80.63 {\footnotesize (\grey{1.00})}} \\
\midrule
\quad + CEM 10-shot adaptation & \textbf{{\normalsize 82.61 {\footnotesize (\green{1.02})}}} \\
\quad + CEM 3-shot (\green{30\% of prompts}) & {\normalsize 82.18 {\footnotesize (\green{1.02})}} \\
\quad + CEM light \textbf{(\green{3\% of prompts})} & {\normalsize 82.08 {\footnotesize (\green{1.02})}} \\
\bottomrule
\end{tabular}}
\label{tab:ablation:light}
\vspace{-2mm}
\end{wraptable}


Our CEM-based adaptation method involves running inference on a small number of samples for each target task (up to 10 in our experiments).
In a typical configuration, this process is relatively efficient: for example, our CEM-light approach (3-shot with 10 generations) completes the ARC-Challenge task in approximately 11 minutes.
As shown in Table~\ref{tab:ablation:light}, this lighter setup reduces the total number of samples to just 3\% of the original setting while still delivering substantial performance improvements over the base model.

We acknowledge that CEM-based adaptation entails a trade-off between one-time overhead it spends on searching the optimal combination weights for the SVF-tune vectors and performance.
Increasing the number of few-shot samples or the number of generations can yield higher performance, but this comes at the cost of additional computational overhead.
However, it is important to note that this adaptation cost is a one-time overhead per task.
The cost-per-prompt diminishes significantly when applied to tasks with a large number of prompts.

Moreover, in practical scenarios, CEM-based adaptation offers better scalability than few-shot prompting methods, which require increasing the length of every prompt, leading to much worse scaling as task sizes grow. In contrast, our method focuses on determining optimal expert vector combinations efficiently and avoids repetitive inference-time costs. However, we note that the overhead might be significant for tasks with very few prompts. Thus, the other adaptations methods might be more appropriate for these particular settings.

We also highlight two immediate directions for improving efficiency:
\begin{enumerate}
\item Reducing the number of few-shot samples: As shown in our ablation study in Appendix~\ref{app:sec:ablation_few_shots}, substantial benefits can be seen even in the 3-shot setting, which requires only evaluation of only 30\% of the number of prompts per generation.
\item Reducing the number of maximum generations: In the explored settings, the CEM parameters tend to converge early on, being very close to the final values after a much lower number of generations than 100.
\end{enumerate}

Finally, in this work we only considered CEM due to its simplicity, there exist several different evolution algorithms empirically showing better efficiency and convergence properties that we hope will be explored in future research.
