% We have presented \system - an important contribution in the development of open-source, lawful, and accessible artificial intelligence. It demonstrates impressive multilingual and coding capabilities, along with a unique emphasis on aligning with safety guidelines outlined in the Biden-Harris US Executive Order on AI. \system posses several advantages: (1) it exemplifies the power of collaboration within the open-source community, promoting transparency and accessibility in AI development; (2) the model's ability to handle diverse languages, coding, and specialized domains showcases a broadening of AI utility for various applications; (3) the focus on red-teaming the model according to the Biden-Harris Executive Order underscores the importance of responsible AI development, ensuring alignment with government standards; and (4) the balance between safety, helpfulness, and state-of-the-art performance makes \system a valuable tool for researchers and developers. 



In this work, we introduced \textsc{\system}, a multilingual model that extends the capabilities of code-focused LLMs while maintaining their original coding proficiency. We demonstrate that continual training from code to multilingual tasks is feasible, allowing the model to perform well across both domains. Adhering to the safety guidelines of the Biden-Harris US Executive Order on AI, \textsc{\system} promotes responsible AI development while pushing the boundaries of performance and utility. Our two-stage continual pretraining approach, combined with insights from cross-lingual red-teaming, highlights the adaptability and versatility of modern language models. \textsc{\system} serves as a valuable resource for both researchers and developers, fostering collaboration and transparency in the open-source AI community. Future work will explore continual pretraining on stronger base models with the same two-stage curriculum, focusing on safety for both LLMs and Multimodal-LLMs. We also aim to develop domain-specific expert models, enhancing task specialization and expanding model versatility.







% We introduce \textsc{\system}, designed to align with legal standards and enhance accessibility. This model showcases proficiency in multilingual understanding and coding tasks, while prioritizing compliance with the safety guidelines outlined in the Biden-Harris US Executive Order on AI. Moreover, \textsc{\system}  exemplifies the collaborative nature of the open-source community, promoting transparency and accessibility in AI development. By red-teaming the model in accordance with the Biden-Harris Executive Order, it underscores the significance of responsible AI development and ensures alignment with government standards. Striking a balance between safety, utility, and cutting-edge performance, \textsc{\system} emerges as a valuable tool for researchers and developers. In addition, we present intriguing insights into cross-lingual red-teaming effects and emphasize the importance of the two-stage curriculum-based continual pretraining approach. 

% \paragraph{Future Work.}
% Leveraging insights from \system's development, we plan to explore continual pretraining of stronger base models using the same two-stage curriculum, prioritizing safety. This applies to both LLMs and Multimodal-LLMs (MLLMs). Furthermore, we aim to train multiple independent domain experts based on \system, potentially merging them to improve task specialization.
