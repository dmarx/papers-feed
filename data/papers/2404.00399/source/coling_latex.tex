% This must be in the first 5 lines to tell arXiv to use pdfLaTeX, which is strongly recommended.
\pdfoutput=1
% In particular, the hyperref package requires pdfLaTeX in order to break URLs across lines.

\documentclass[11pt]{article}

% Change "review" to "final" to generate the final (sometimes called camera-ready) version.
% Change to "preprint" to generate a non-anonymous version with page numbers.
\usepackage[preprint]{coling}

% Standard package includes
\usepackage{times}
\usepackage{latexsym}

% For proper rendering and hyphenation of words containing Latin characters (including in bib files)
\usepackage[T1]{fontenc}
% For Vietnamese characters
% \usepackage[T5]{fontenc}
% See https://www.latex-project.org/help/documentation/encguide.pdf for other character sets

% This assumes your files are encoded as UTF8
\usepackage[utf8]{inputenc}

% This is not strictly necessary, and may be commented out,
% but it will improve the layout of the manuscript,
% and will typically save some space.
\usepackage{microtype}

% This is also not strictly necessary, and may be commented out.
% However, it will improve the aesthetics of text in
% the typewriter font.
\usepackage{inconsolata}

%Including images in your LaTeX document requires adding
%additional package(s)
\usepackage{graphicx}


\usepackage{tablefootnote}

% \usepackage[table,x11names]{xcolor}
\usepackage{color, colortbl}
\definecolor{verylightgray}{rgb}{0.9,0.9,0.9}
\definecolor{gray}{rgb}{0.5,0.5,0.5}
\definecolor{pygreen}{rgb}{0.0, 0.5, 0.0}
\definecolor{pyred}{rgb}{0.7, 0.0, 0.0}
\definecolor{pyblue}{rgb}{0.0, 0.0, 0.7}
\definecolor{pygray}{rgb}{0.5, 0.5, 0.5}
\definecolor{pydarkgray}{rgb}{0.3, 0.3, 0.3}

\definecolor{color1}{HTML}{006EB8}
\definecolor{color2}{HTML}{009B55}
\newcommand{\myblue}[1]{\textcolor{azureblue}{~#1}}
\newcommand{\myorange}[1]{\textcolor{orange}{~#1}}

\usepackage{wrapfig}

\usepackage[symbol]{footmisc}

\usepackage[T1]{fontenc}
\usepackage{multirow}
\usepackage{microtype}
\usepackage{hyperref}
\hypersetup{colorlinks=true,urlcolor=color2,linkcolor=color2,citecolor=color2}
\usepackage{url}
\usepackage{booktabs}
\usepackage{graphicx}
\usepackage{subcaption}
\usepackage[whole]{bxcjkjatype}
\usepackage{amsmath} 
\usepackage[inline]{enumitem}
% For Sonny's comment
\newcommand{\SVX}[1]{\textcolor{red}{[SVX: #1]}}
\newcommand{\Q}[1]{\textcolor{red}{[#1]}}


%% for non-anonymous version
% \newcommand{\system}{\textsc{Aurora-M}} 

% for submission
\newcommand{\system}{\textsc{Aurora-M}}


% If the title and author information does not fit in the area allocated, uncomment the following
%
%\setlength\titlebox{<dim>}
%
% and set <dim> to something 5cm or larger.

\title{\textsc{\system}: Open Source Continual Pre-training for Multilingual Language and Code}

% \title{\textsc{\system}: The First Open Source Multilingual Language Model Red-teamed according to the U.S. Executive Order}

% Author information can be set in various styles:
% For several authors from the same institution:
% \author{Author 1 \and ... \and Author n \\
%         Address line \\ ... \\ Address line}
% if the names do not fit well on one line use
%         Author 1 \\ {\bf Author 2} \\ ... \\ {\bf Author n} \\
% For authors from different institutions:
% \author{Author 1 \\ Address line \\  ... \\ Address line
%         \And  ... \And
%         Author n \\ Address line \\ ... \\ Address line}
% To start a separate ``row'' of authors use \AND, as in
% \author{Author 1 \\ Address line \\  ... \\ Address line
%         \AND
%         Author 2 \\ Address line \\ ... \\ Address line \And
%         Author 3 \\ Address line \\ ... \\ Address line}

% \author{First Author \\
%   Affiliation / Address line 1 \\
%   Affiliation / Address line 2 \\
%   Affiliation / Address line 3 \\
%   \texttt{email@domain} \\\And
%   Second Author \\
%   Affiliation / Address line 1 \\
%   Affiliation / Address line 2 \\
%   Affiliation / Address line 3 \\
%   \texttt{email@domain} \\}

% \author{
%    Aurora Team\\
%    Ontocord.AI
% }
\begin{center}
\textbf{Anton Lozhkov}$^1$\quad
\textbf{Raymond Li}$^2$\quad
\textbf{Loubna Ben Allal}$^1$\quad
\textbf{Federico Cassano}$^4$\quad
\textbf{Joel Lamy-Poirier}$^2$\quad
\textbf{Nouamane Tazi}$^1$\quad
\textbf{Ao Tang}$^3$\quad
\textbf{Dmytro Pykhtar}$^3$\quad
\textbf{Jiawei Liu}$^7$\quad
\textbf{Yuxiang Wei}$^7$\quad
\textbf{Tianyang Liu}$^{25}$\quad
\textbf{Max Tian}$^2$\quad
\textbf{Denis Kocetkov}$^2$\quad
\textbf{Arthur Zucker}$^1$\quad
\textbf{Younes Belkada}$^1$\quad
\textbf{Zijian Wang}$^5$\quad
% Data filtering
\textbf{Qian Liu}$^{12}$\quad
\textbf{Dmitry Abulkhanov}$^5$\quad
\textbf{Indraneil Paul}$^{32}$\quad
\textbf{Zhuang Li}$^{14}$\quad
\textbf{Wen-Ding Li}$^{26}$\quad
\textbf{Megan Risdal}$^{24}$\quad
\textbf{Jia Li}$^5$\quad
\textbf{Jian Zhu}$^{16}$\quad
\textbf{Terry Yue Zhuo}$^{14,15}$\quad
\textbf{Evgenii Zheltonozhskii}$^{13}$\quad
% Data Inspection
\textbf{Nii Osae Osae Dade}$^{28}$\quad
\textbf{Wenhao Yu}$^{20}$\quad
\textbf{Lucas Krauß}$^5$\quad
% \textbf{Zijie W}$^?$\quad
\textbf{Naman Jain}$^{27}$\quad
\textbf{Yixuan Su}$^{30}$\quad
\textbf{Xuanli He}$^{23}$\quad
\textbf{Manan Dey}$^{31}$\quad
\textbf{Edoardo Abati}$^5$\quad
\textbf{Yekun Chai}$^{33}$\quad
\textbf{Niklas Muennighoff}$^{29}$\quad
\textbf{Xiangru Tang}$^34$\quad
\textbf{Muhtasham Oblokulov}$^{18}$\quad
% Evaluation
\textbf{Christopher Akiki}$^{9,10}$\quad
\textbf{Marc Marone}$^8$\quad
\textbf{Chenghao Mou}$^5$\quad
\textbf{Mayank Mishra}$^{19}$\quad
\textbf{Alex Gu}$^{17}$\quad
\textbf{Binyuan Hui}$^{5}$\quad
\textbf{Tri Dao}$^{21}$\quad
\textbf{Armel Zebaze}$^1$\quad
\textbf{Olivier Dehaene}$^1$\quad
\textbf{Nicolas Patry}$^1$\quad
\textbf{Canwen Xu}$^{25}$\quad
\textbf{Julian McAuley}$^{25}$\quad
\textbf{Han Hu}$^{14}$\quad% forgot to add Han Hu, who helped a bit on the Asleep data process
\textbf{Torsten Scholak}$^2$\quad
\textbf{Sebastien Paquet}$^2$\quad
\textbf{Jennifer Robinson}$^6$\quad
\textbf{Carolyn~Jane Anderson}$^{22}$\quad
\textbf{Nicolas Chapados}$^2$\quad
\textbf{Mostofa Patwary}$^3$\quad
\textbf{Nima Tajbakhsh}$^3$\quad
% \textbf{Brendan Dolan-Gavitt}$^{29}$\quad
\textbf{Yacine Jernite}$^1$\quad
\textbf{Carlos Muñoz Ferrandis}$^1$\quad
\textbf{Lingming Zhang}$^7$\quad
\textbf{Sean Hughes}$^6$\quad
\textbf{Thomas Wolf}$^1$\quad
\textbf{Arjun Guha}$^{4,11}$\quad
\textbf{Leandro von Werra}$^{1, \star}$\quad
\textbf{Harm de Vries}$^{2, \star}$\quad

% % Affiliation
$^{1}$Hugging~Face \quad
$^{2}$ServiceNow Research\quad
$^{3}$Nvidia\quad
$^{4}$Northeastern University\quad
$^{5}$Independent\quad
$^{6}$ServiceNow\quad
$^{7}$University of Illinois Urbana-Champaign\quad
$^{8}$Johns Hopkins University\quad
$^{9}$Leipzig University\quad
$^{10}$ScaDS.AI\quad
% $^{11}$Queen Mary University of London\quad
$^{11}$Roblox\quad
$^{12}$Sea AI Lab\quad
$^{13}$Technion -- Israel Institute of Technology\quad
$^{14}$Monash University\quad
$^{15}$CSIRO's Data61\quad
% $^{17}$Alibaba Group\quad
% $^{18}$Saama AI Research Lab\quad
$^{16}$University of British Columbia\quad
$^{17}$MIT\quad
$^{18}$Technical University of Munich\quad
$^{19}$IBM Research\quad
% $^{23}$University of Vermont\quad
% $^{24}$UnfoldML\quad
% $^{25}$SAP\quad
$^{20}$University of Notre Dame\quad
% $^{27}$Columbia University\quad
% $^{28}$Discover Dollar Pvt Ltd\quad
% $^{29}$NYU\quad
% $^{30}$University of Allahabad\quad
% $^{31}$Telefonica I+D\quad
% $^{32}$Toloka\quad
$^{21}$Princeton University\quad
% $^{34}$Weizmann Institute of Science\quad
% $^{35}$The Alan Turing Institute\quad
$^{22}$Wellesley College\quad
% $^{37}$Eleuther AI\quad
% $^{38}$Forschungszentrum J{\"u}lich\quad
$^{23}$University College London\quad
$^{24}$Kaggle\quad
$^{25}$UC San Diego\quad
$^{26}$Cornell University\quad
$^{27}$UC Berkeley\quad
$^{28}$Mazzuma\quad
$^{29}$Contextual AI\quad
$^{30}$Cohere\quad
$^{31}$Salesforce\quad
$^{32}$Technical University of Darmstadt\quad
$^{33}$Baidu\quad
$^{34}$Yale University\quad


Corresponding authors ($\star$) can be contacted at \href{contact@bigcode-project.org}{contact@bigcode-project.org}\\
\vspace{0.5cm}
% {\bf Reviewed on OpenReview:} \openreview
\end{center}


%\author{
%  \textbf{First Author\textsuperscript{1}},
%  \textbf{Second Author\textsuperscript{1,2}},
%  \textbf{Third T. Author\textsuperscript{1}},
%  \textbf{Fourth Author\textsuperscript{1}},
%\\
%  \textbf{Fifth Author\textsuperscript{1,2}},
%  \textbf{Sixth Author\textsuperscript{1}},
%  \textbf{Seventh Author\textsuperscript{1}},
%  \textbf{Eighth Author \textsuperscript{1,2,3,4}},
%\\
%  \textbf{Ninth Author\textsuperscript{1}},
%  \textbf{Tenth Author\textsuperscript{1}},
%  \textbf{Eleventh E. Author\textsuperscript{1,2,3,4,5}},
%  \textbf{Twelfth Author\textsuperscript{1}},
%\\
%  \textbf{Thirteenth Author\textsuperscript{3}},
%  \textbf{Fourteenth F. Author\textsuperscript{2,4}},
%  \textbf{Fifteenth Author\textsuperscript{1}},
%  \textbf{Sixteenth Author\textsuperscript{1}},
%\\
%  \textbf{Seventeenth S. Author\textsuperscript{4,5}},
%  \textbf{Eighteenth Author\textsuperscript{3,4}},
%  \textbf{Nineteenth N. Author\textsuperscript{2,5}},
%  \textbf{Twentieth Author\textsuperscript{1}}
%\\
%\\
%  \textsuperscript{1}Affiliation 1,
%  \textsuperscript{2}Affiliation 2,
%  \textsuperscript{3}Affiliation 3,
%  \textsuperscript{4}Affiliation 4,
%  \textsuperscript{5}Affiliation 5
%\\
%  \small{
%    \textbf{Correspondence:} \href{mailto:email@domain}{email@domain}
%  }
%}

\begin{document}
\setlength\titlebox{35\baselineskip}
\maketitle
\begin{abstract}
Pretrained language models are an integral part of AI applications, but their high computational cost for training limits accessibility. 
Initiatives such as \textsc{Bloom} and \textsc{StarCoder} aim to democratize access to pretrained models for collaborative community development. 
Despite these efforts, such models encounter challenges such as limited multilingual capabilities, risks of catastrophic forgetting during continual pretraining, and the high costs of training models from scratch, alongside the need to align with AI safety standards and regulatory frameworks.
%However, existing models face challenges: limited multilingual capabilities, continual pretraining causing catastrophic forgetting whereas pretraining from scratch is computationally expensive, and compliance with AI safety and development laws. 
%
This paper presents \textbf{\textcolor{violet}{\system}}, a \texttt{15B} parameter multilingual open-source model trained on English, Finnish, Hindi, Japanese, Vietnamese, and code. Continually pretrained from \textsc{StarCoderPlus} on \texttt{435B} additional tokens, \system\ surpasses \texttt{2T} tokens in total training token count. It is the first open-source multilingual model fine-tuned on human-reviewed safety instructions, thus aligning its development not only with conventional red-teaming considerations, but also with the specific concerns articulated in the Biden-Harris Executive Order on the Safe, Secure, and Trustworthy Development and Use of Artificial Intelligence.
%
We evaluate \system\ across a wide range of tasks and languages, showcasing its robustness against catastrophic forgetting and its superior performance in multilingual settings, particularly in safety evaluations. We open-source \system\ and its variants to encourage responsible open-source development of large language models at \url{https://huggingface.co/aurora-m}.

% \begin{center}
%    % \includegraphics[width=0.03\textwidth]{img/huggingface_logo_svg-tex.pdf}
%    \url{https://huggingface.co/aurora-m}
% \end{center}
\end{abstract}

% To encourage responsible open-source development of large language models, \system\ and its variants are available at \url{https://huggingface.co/aurora-m}.
% \system\ is rigorously evaluated across various tasks and languages, demonstrating robustness against catastrophic forgetting and outperforming alternatives in multilingual settings, particularly in safety evaluations. To promote responsible open-source LLM development, \system\ and its variants are released at \url{https://huggingface.co/aurora-m}.
% \href{https://huggingface.co/collections/aurora-m/aurora-m-models-65fdfdff62471e09812f5407}{here}.
% \system\ is rigorously evaluated across various tasks and languages, demonstrating robustness against catastrophic forgetting and outperforming alternatives in multilingual settings, particularly in safety evaluations. To promote responsible open-source LLM development, \system\ and its variants will be made publicly available. 



\section{Introduction}
\section{Introduction}
\label{sec:introduction}

\begin{wrapfigure}{r}{0.5\textwidth}
\vspace{-6mm}
\begin{center}
    \includegraphics[width=0.5\textwidth]{images/cover.pdf}
  \end{center}
  \vspace{-4mm}
  \caption{\textbf{Overview of \implname.} In training, we tune the singular values of the weight matrices to generate a set of ``expert'' vectors specializing in different tasks. In inference, a two-pass process is adopted where the first applies the expert and the second generates the answer.}
  \label{fig:cover}
  \vspace{-4mm}
\end{wrapfigure}

Self-adaptive large language models (LLMs) would represent a significant advancement in artificial intelligence, enabling real-time adaptation to various tasks and contexts.
While compositionality and scalability are crucial for effective adaptation, current LLM training methodologies fall short of achieving both these properties simultaneously.
Our research aims to present a solution to address these gaps.

In principle, the first step toward achieving self-adaptive LLMs can be realized through the development of specialized expert modules, each fine-tuned~\citep{kaplan2020scaling} via techniques such as low-rank adaptation (LoRA)~\citep{hu2021lora}. 
However, several challenges need to be addressed to make this approach both scalable and compositional: (1) multiple expert modules significantly increase the number of parameters; (2) expert modules are often prone to overfitting; and (3) flexible composition of these experts is still an open problem.

To overcome these limitations, we first propose \svdacro, a novel parameter-efficient fine-tuning (PEFT) method to obtain effective building blocks for self-adaptation.
\svdacro works by extracting and selectively tuning only the singular values within the model's weight matrices.
By focusing on this essential and principled parameterization, our approach mitigates the risk of overfitting, drastically reduces computational demands, and allows for inherent compositionality.

We then introduce our full \implname framework, which entails a two-pass inference mechanism to produce dynamically adapted weights targeted for the test-time conditions (Figure~\ref{fig:cover}).
We design three different adaptation strategies that can be used within \implname, which we show provide monotonic performance benefits with increasing access to the test-time conditions.
We evaluate \svdacro and the full \implname framework through extensive experiments across a diverse range of LLMs and tasks.
\svdacro outperforms traditional efficient fine-tuning methods like LoRA on domain-specific datasets with far fewer parameters. 
\implname further improves performance, even for out-of-distribution tasks like visual QA. 
Our analysis even shows that \implname allows the reuse of \svdacro experts across different LLMs. In summary, our key technical contributions are: 
\vspace{-2mm}
\begin{itemize}
\item The development of \implname as a pivotal self-adaptation framework for LLMs, providing a blueprint to adapt the behavior of LLMs from a growing set of pre-trained skills.
\item The introduction of \svdacro, a novel PEFT method trainable with RL on small datasets, producing compact expert vectors with inherent compositionality.
\item The implementation of three adaptation strategies, effectively dispatching \svdacro-trained experts with properties designed to cope with different deployment scenarios.
\end{itemize}

\vspace{-2mm}

\section{Datasets}

\section{Text and Multi-modal Datasets}\label{sec:dataset}


In this section, we introduce the text and multi-modal datasets used to evaluate LLM efficiency.

\subsection{Text Dataset}\label{ssec:text_dataset}

% Text dataset, as the most commonly seen datasets involved in LLM's training, validation and testing, has diverged into many different tasks.
% We summarized some long-context datasets that was adapted and used in state-of-the-art benchmark frameworks, namely \textbf{L-Eval}~\cite{an_l-eval:_2023}, 
% \textbf{M4LE}~\cite{kwan_m4le:_2023}, \textbf{BAMBOO}~\cite{dong2023bamboo}, \textbf{LongBench}~\cite{bai_longbench:_2023}, \textbf{LRA}~\cite{tay_long_lra_2020}, 
% \textbf{SCORLLS}~\cite{shaham_scrolls:_2022}, \textbf{ZEROSCROLLS}~\cite{shaham_zeroscrolls:_2023}  \textbf{LooGLE}~\cite{li_loogle:_2023}, \textbf{LongEval}~\cite{longchat2023}  and \textbf{StreamingEval}~\cite{DBLP:conf/iclr/XiaoTCHL24}.  
% We do our survey at dataset level, and looked into the original dataset for more detailed information.  \ref{tab:text-dataset-qa}, \ref{tab:text-dataset-summarization}, \ref{tab:text-dataset-nli-classification}, \ref{tab:text-dataset-generation}, \ref{tab:text-dataset-retrieval}, \ref{tab:text-dataset-aggregation} describe datasets that are ready-to-use, or have a large text resource to build new datasets with. In all, the tasks text datasets apply to can be categorized into the following types:

We collect a lot of long-context datasets from state-of-the-art benchmark frameworks and various papers, including L-Eval\cite{an_l-eval:_2023}, M4LE\cite{kwan_m4le:_2023}, BAMBOO\cite{dong2023bamboo}, LongBench\cite{bai_longbench:_2023}, LRA\cite{tay_long_lra_2020}, SCROLLS\cite{shaham_scrolls:_2022}, ZEROSCROLLS\cite{shaham_zeroscrolls:_2023}, LooGLE\cite{li_loogle:_2023}, LongEval\cite{longchat2023}, and StreamingEval\cite{DBLP:conf/iclr/XiaoTCHL24}.
Specifically,
we categorize these datasets into different tasks, including question answering, text summarization, text reasoning, text retrieval, and text generation.

%Text-QA
\begin{table*}[t]
    \centering
    \caption{Text question answering (QA) dataset. In the \textbf{Avg. Len: }  average length, \textbf{Tok:}  tokens;  \textbf{W: } words. In the Instances column, \textbf{Doc:} documents, \textbf{Q:} questions, \textbf{Inst:} instructions. Particularly, AltQA, PaperQA and MeetingQA have two datasets  with different length levels, and is separated with $/$. \\
    %In the Metric column, \textbf{EM:} Exact Match. \textbf{Acc:} Accuracy. \textbf{METEOR}~\cite{denkowski-lavie-2011-meteor} is a machine translation evaluation metric that incorporates factors like stemming, synonym matching, and word order to provide a more nuanced assessment of translation quality. 
    %\textbf{BLEU}\cite{papineni-etal-2002-bleu} is also a metric designed for machine translation. 
    %\textbf{Rouge}~\cite{lin-2004-rouge} and its variants measure model's performance by calculating overlap between model output and reference answer with unigram(Rouge-1), bigram(Rouge-2), LCS(Rouge-L), etc.  
    %\textbf{F1}~\cite{rajpurkar-etal-2016-squad-F1} calculates unigram overlap between model output and answers after processing elments like white-spaces and stop-words. 
    Particularly, for datasets from L-Eval, the GPT-4 metric means the win-rate against Turbo-16K, judged by GPT-4. $\Delta L$ is the length difference between answer length and ground truth. For NarrativeQA, \textbf{MRR: }Mean Reciprocal Rank .} 
    \renewcommand{\arraystretch}{1.5} % 调整行间距
    \setlength{\tabcolsep}{1pt} % 减小列间距
    \label{tab:text-dataset-qa}
\begin{tabular}{c|c|c|c|c|c|c}
\hline
\textbf{Task} &
  \textbf{Name} &
  \textbf{Source} &
  \textbf{Instances} &
  \textbf{Avg Len} &
  \textbf{Metric} &
  \textbf{Lang.} \\ \hline
QA &
  \href{https://github.com/abacusai/long-context}{AltQA}~\cite{pal2023giraffeadventuresexpandingcontext} &
  Wikipedia &
  200/200 &
  3243/13,084 Tok&
  Acc &
  EN \\ \hline

QA &
  \href{https://github.com/RUCAIBox/BAMBOO/tree/main/datasets}{PaperQA(BAMBOO)}~\cite{dong2023bamboo} &
  Paper &
  100/100 &
  3101/6838 Tok&
  Acc &
  EN \\ \hline

QA &
  \href{https://github.com/RUCAIBox/BAMBOO/tree/main/datasets}{MeetingQA(BAMBOO}~\cite{dong2023bamboo}  &
  Meeting &
  100/100 &
  2738/9838 Tok&
  Acc &
  EN \\ \hline



QA &
  \href{https://huggingface.co/datasets/mandarjoshi/trivia_qa}{TriviaQA}~\cite{joshi_triviaqa:_2017}  &
  Web Question, Wiki &
  95,956 Q, 662,659 Doc&
  17,370 W&
  EM, F1 &
  EN \\ \hline



QA&
  \href{https://github.com/OpenLMLab/LEval}{TOEFL(L-Eval)}~\cite{an_l-eval:_2023} &
  TOFEL-QA\cite{tseng_towards_tofelqa_2016} &
  15 Doc, 269 Inst&
  3907 Tok&
   Rouge-L, GPT-4, $\Delta L$&
  EN \\ \hline


QA&
   \href{https://github.com/OpenLMLab/LEval}{Coursera(L-Eval)}~\cite{an_l-eval:_2023} &
  Video Subtitles&
  15 Doc, 172 Inst&
  9075 Tok&
   Rouge-L, GPT-4, $\Delta L$&
  EN \\ \hline


QA &
   \href{https://github.com/OpenLMLab/LEval}{SFiction(L-Eval)}~\cite{an_l-eval:_2023} &
  SFGram \cite{sfgram}, fiction&
  7 Doc, 64 Inst&
  16,381 Tok&
   Rouge-L, GPT-4, $\Delta L$&
  EN \\ \hline


QA &
   \href{https://github.com/OpenLMLab/LEval}{LongFQA(L-Eval)}~\cite{an_l-eval:_2023} &
  Financial Transcripts&
  6 Doc, 52 Inst&
  6032 Tok&
   Rouge-L, GPT-4, $\Delta L$&
  EN \\ \hline

QA &
   \href{https://github.com/OpenLMLab/LEval}{CUAD(L-Eval)}~\cite{an_l-eval:_2023} &
  CUAD\cite{hendrycks_cuad:_2021} &
  20 Doc, 130 Inst&
  30,966 Tok&
   Rouge-L, GPT-4, $\Delta L$&
  EN \\ \hline


QA &
  \href{https://github.com/KwanWaiChung/M4LE}DuoRC~\cite{kwan_m4le:_2023} &
  Movie &
   &
  3572 W&
  Acc &
  EN \\ \hline


QA &
  \href{https://ai.google.com/research/NaturalQuestions/download}{NQ~\cite{kwiatkowski_nq_2019}}&
  Wiki &
  307,373 &
  9005 W &
  Rouge &
  EN \\ \hline


  
QA-SG&
  \href{https://github.com/google-deepmind/narrativeqa}{NarrativeQA}~\cite{narrativeqa}  &
  Story &
  1572 Doc&
  62,528 Tok&
  \begin{tabular}[c]{@{}c@{}}BLEU, METEOR, \\ Rouge-L, MRR\end{tabular}&
  EN \\ \hline

QA-SG &
  \href{https://huggingface.co/datasets/THUDM/LongBench}{NarrativeQA(LongBench)}~\cite{bai_longbench:_2023} &
  Story &
  200 &
  18,409 W &
  F1 &
  EN \\ \hline

QA-SG&
  \href{https://huggingface.co/datasets/allenai/qasper}{Qasper}~\cite{dasigi_dataset_2021_qasper} &
  Paper &
  1585 &
  5001 W&
  F1 &
  EN \\ \hline

QA-SG &
  \href{https://huggingface.co/datasets/THUDM/LongBench}{Qasper(LongBench)}~\cite{bai_longbench:_2023}  &
  Paper &
  200 &
  3619 W&
  F1 &
  EN \\ \hline

QA-SG&
  \href{https://huggingface.co/datasets/THUDM/LongBench}{MultifieldQA-en}~\cite{bai_longbench:_2023}  &
  \begin{tabular}[c]{@{}c@{}}Paper, Legal, \\ Gov, Google\end{tabular}&
  200 &
  4459 W&
  F1 &
  EN \\ \hline

QA-SG&
  \href{https://huggingface.co/datasets/THUDM/LongBench}{MultifieldQA-zh}~\cite{yang_hotpotqa:_2018} &
\begin{tabular}[c]{@{}c@{}}Paper, Legal, \\ Gov, Google\end{tabular}&
  200 &
  6701 W&
  F1 &
  ZH \\ \hline


  QA-SG &
  \href{https://github.com/nyu-mll/quality}{QuALITY}~\cite{pang_quality_dataset:_2021} &
  Story, magazine&
  381 Doc, 6737 Q&
  4203 W&
  EM &
  EN \\ \hline


  
QA-MT&
  \href{https://huggingface.co/datasets/hotpotqa/hotpot_qa}{HotpotQA}~\cite{yang_hotpotqa:_2018} &
  Wiki &
  112,779 &
  1138 W&
  EM, F1 &
  EN \\ \hline


QA-MT&
  \href{https://huggingface.co/datasets/THUDM/LongBench}{HotpotQA(LongBench)}~\cite{bai_longbench:_2023} &
  Wiki &
  200 &
  9151 W&
  F1 &
  EN \\ \hline


QA-MT&
  \href{https://github.com/Alab-NII/2wikimultihop}{2WikiMultihopQA}~\cite{ho_multihopqa_2020} &
  Wiki &
  192,606 Q&
  639 W&
  EM, F1 &
  EN \\ \hline


QA-MT&
  \href{https://github.com/stonybrooknlp/musique}{MuSiQue}~\cite{trivedi_musique:_2021} &
  Wiki &
  24,814 &
  1827 W&
  F1 &
  EN \\ \hline

QA-MT&
  \href{https://github.com/baidu/DuReader}{DuReader}~\cite{he_dureader:_2017} &
  Baidu &
  200,000 Q, 1,000,000 Doc&
  396 W&
  BLEU, Rouge-L &
  ZH,EN \\ \hline
  
QA+RET &
  \href{https://github.com/KwanWaiChung/M4LE}{NewsQA(M4LE)}~\cite{kwan_m4le:_2023} &
  News &
  - &
  3679 W&
  Acc &
  EN \\ \hline

QA+RET &
  \href{https://github.com/KwanWaiChung/M4LE}{C3(M4LE)}~\cite{kwan_m4le:_2023} &
  Textbook&
  - &
  3797 W&
  Acc &
  ZH \\




 \hline
\end{tabular}
\end{table*}



\subsubsection{Question Answering (QA) Task}
Dataset for this task usually consist of question-answer pairs, and documents that contains the answer to the question. For a model to run such task, documents and questions are usually used as the model input, while the output can differ greatly. Some datasets' answers are closed-ended, meaning that the model should only output its answer in designated form, typically multiple choice answers, while the open-ended answers take a more free form. According to the number of documents involved in a question-answer pair, we can categorize QA task datasets into single-doc QA(QA-SG) and multiple-doc QA(QA-MT). The detailed statistics of the datasets for question answering are provided in Table~\ref{tab:text-dataset-qa}.


%Text-SUM
\begin{table*}[]
    \centering
    \caption{Text Dataset-Summarization. In the \textbf{Avg. Len: }  average length, \textbf{Tok:}  tokens;  \textbf{W: } words. In the Instances column, \textbf{Doc:} documents, \textbf{Q:} questions, \textbf{Inst:} instructions. Particularly, SPACE has the concept of 'Entity', and R/Ent stands for reviews per entity. Sum stands for summary.  
    In the Metric column, \textbf{EM:} Exact Match. \textbf{PM:} Partial Match. \textbf{Acc:} Accuracy. 
    %\textbf{METEOR}~\cite{denkowski-lavie-2011-meteor} and \textbf{BLEU}\cite{papineni-etal-2002-bleu} are a machine translation evaluation metric.
    %that incorporates factors like stemming, synonym matching, and word order to provide a more nuanced assessment of translation quality. 
    %\textbf{BLEU}\cite{papineni-etal-2002-bleu} is also a metric designed for machine translation. 
    %\textbf{Rouge}~\cite{lin-2004-rouge} measures model's performance with the ngram overlap between model output and reference answer. \textbf{F1}~\cite{rajpurkar-etal-2016-squad-F1} calculates unigram overlap between model output and answers after processing elments like white-spaces and stop-words. 
    %\textbf{BERT}~\cite{zhang2020bertscoreevaluatingtextgeneration} is a text generation task metric using contextual embeddings. 
    For MultiNews, \textbf{Rouge-SU} skip bigrams when having a distance larger than 4 words. Particularly, LooGLE utilizes GPT-4 for its QA and summarization task, using it for answer's semantic judgement.}
    \renewcommand{\arraystretch}{1.3} % 调整行间距
    \setlength{\tabcolsep}{1pt} % 减小列间距
    \label{tab:text-dataset-summarization}

\begin{tabular}{c|c|c|c|c|c|c}
\hline
\textbf{Task} &
  \textbf{Name} &
  \textbf{Source} &
  \textbf{Instances} &
  \textbf{Avg Len} &
  \textbf{Metric} &
  \textbf{Lang.} \\ \hline
 SUM& 
\href{https://huggingface.co/datasets/ccdv/cnn_dailymail}{CNN/Dailymail}~\cite{nallapati_abstractive_dailymail_2016}& News& 
300,000& 
766 W& 
Rouge-1/2/L&
EN\\ \hline
 SUM
 & \href{https://huggingface.co/datasets/EdinburghNLP/xsum}{XSum}~\cite{narayan_dont_xsum_2018}
 & News
 & 400,000
 & 431 W
 & Rouge-1/2/L
 &EN\\ \hline
 SUM
 & \href{https://github.com/Yale-LILY/QMSum}{QMSum}~\cite{zhong_qmsum:_2021}
 & Meeting
 & 232 Meets, 1808 Q
 & 9070 W
 & Rouge-1/2/L
 &EN\\  \hline
 SUM
 & \href{https://github.com/Alex-Fabbri/Multi-News}{MultiNews}~\cite{fabbri_multi-news:_2019}
 & News
 & 51,216
 & 5866 W %2103.49*2.788
 & Rouge-1/2/SU
 &EN\\ \hline
 
\begin{tabular}[c]{@{}c@{}}SUM-QB+\\Reasoning+\\QA\end{tabular}& %Note: also support Cloze task 
  \href{https://github.com/bigai-nlco/LooGLE}{LooGLE}~\cite{li_loogle:_2023} &
  \begin{tabular}[c]{@{}c@{}}Papers, Wiki,\\ Movie, TV \end{tabular}&
  776 Doc, 6448 Q&
  \begin{tabular}[c]{@{}c@{}}19,367 W\\24,005 Tok\end{tabular}&
  \begin{tabular}[c]{@{}c@{}}BLEU, Rouge, METEOR, \\BERT, GPT4, \\EM, PM \end{tabular}&
  EN,ZH \\ \hline

SUM &
  \href{https://huggingface.co/datasets/ccdv/govreport-summarization}{GovReport}~\cite{huang_govreport_2021} &
  Gov &
  19,466 &
  \begin{tabular}[c]{@{}c@{}}9409.4 W\end{tabular}&
  Rouge-1/2/L&
  EN \\ \hline
SUM &
  \href{https://github.com/hahahawu/VCSum}{VCSUM}~\cite{wu_vcsum:_2023} &
  Meeting &
  239 &
  14,107 Tok&
  F1, Gold Rouge-1&
  ZH \\ \hline
%SUM &
%  \href{https://arxiv.org/src/1911.12237v2/anc}{SAMSum}~\cite{gliwa_samsum_2019} &
%  Dialogue &
%  16,369 &
%  109 W &
%  Rouge&
%  EN \\ \hline
SUM &
  \href{https://github.com/mingdachen/SummScreen}{SummScreenFD}~\cite{chen_summscreen:_2021} &
  TV &
  269,000 &
  6613 Tok&
  Rouge&
  EN \\ \hline

SUM &
  \href{https://evasharma.github.io/bigpatent/}{BigPatent}~\cite{sharma_bigpatent:_2019} &
  Patent &
  1,341,362 &
  3573 W&
  Rouge-1/2/L&
  EN \\ \hline
SUM &
  \href{https://github.com/stangelid/qt}{SPACE}~\cite{angelidis_extractive_space_2021} &
  Review&
  \begin{tabular}[c]{@{}c@{}}50 Entities, \\1,140,000 Reviews, \\100R/Ent, \\1050 Sum\end{tabular}&
  15,532 W &
  Rouge-1/2/L&
  EN \\ \hline
SUM &
  \href{https://github.com/nyu-mll/SQuALITY}{SQuALITY} ~\cite{wang2022squality}&
  Story&
  625&
  \begin{tabular}[c]{@{}c@{}}5200 W\end{tabular}&
  \begin{tabular}[c]{@{}c@{}}Rouge-1/2/L, \\METEOR, \\BERT\end{tabular}&
  EN \\ \hline
SUM+RET &
  \href{https://github.com/KwanWaiChung/M4LE}{CNNNews(M4LE)}~\cite{kwan_m4le:_2023} &
  News &
  - &
  3754 W &
  Rouge-L &
  EN \\ \hline
SUM+RET &
  \href{https://github.com/KwanWaiChung/M4LE}{CEPSUM(M4LE)}~\cite{kwan_m4le:_2023} &
  E-Commerce &
  - &
  4003 W &
  Rouge-L & 
  ZH \\  \hline
SUM+RET &
  \href{https://github.com/KwanWaiChung/M4LE}{LCSTS(M4LE)}~\cite{kwan_m4le:_2023} &
  News &
  - &
  4102 W &
  Rouge-L &
  ZH \\ \hline
SUM+RET &
  \href{https://github.com/KwanWaiChung/M4LE}{NCLS(M4LE)}~\cite{kwan_m4le:_2023} &
  NCLS~\cite{zhu_ncls:_2019} &
  - &
  3470 W&
  Rouge-L &
  EN,ZH \\ \hline
SUM+RET &
  \href{https://github.com/KwanWaiChung/M4LE}{WikiHow}~\cite{kwan_m4le:_2023} &
  Wiki &
  - &
  3514 W &
  Rouge-L &
  EN \\ \hline
SUM+RET &
  \href{https://github.com/KwanWaiChung/M4LE}{News2016}~\cite{kwan_m4le:_2023} &
  News &
  - &
  3785 W &
  Rouge-L &
  ZH \\ \hline

SUM &
  \href{https://github.com/KwanWaiChung/M4LE}{Pubmed(M4LE)}~\cite{kwan_m4le:_2023} &
  Medical &
  1267 &
  3678 W &
  Rouge-L &
  EN \\ \hline
SUM &
  \href{https://github.com/KwanWaiChung/M4LE}{BookSum(M4LE)}~\cite{kwan_m4le:_2023} &
  Book &
  - &
  2643 W &
  Rouge-L &
  EN \\ \hline
SUM &
  \href{https://github.com/KwanWaiChung/M4LE}{CNewsum(M4LE)}~\cite{kwan_m4le:_2023} &
  News &
  690 &
  1883 W &
  Rouge-L &
  ZH \\ \hline
SUM &
  \href{https://github.com/KwanWaiChung/M4LE}{CLTS+(M4LE)}~\cite{kwan_m4le:_2023} &
  News &
  - &
  3158 W &
  Rouge-L &
  ZH \\ \hline
SUM &
  \href{https://github.com/KwanWaiChung/M4LE}{Arxiv(M4LE)}~\cite{kwan_m4le:_2023} &
  Paper&
  1550 &
  3748 W &
  Rouge-L &
  EN \\ \hline


 
\end{tabular}
\end{table*}

\begin{itemize}[leftmargin=10pt]
    \item \textbf{Qasper~\cite{dasigi_dataset_2021_qasper}} consists of 5049 questions based on 1585 papers on NLP.  Question is from NLP practitioners that only have read the abstract and title of a paper, then another set of practitioners answer these questions by reading through the whole paper. The supporting evidences is provided correspondingly. Each instance of the dataset consists of a question, an answer, corresponding paper and supporting evidence. Instances built by LongBench~\cite{bai_longbench:_2023} doesn't require evidence. 
    \item  \textbf{HotpotQA~\cite{yang_hotpotqa:_2018} } is a typical for a multi-doc QA dataset. It's built based on Wikipedia, and each instance consists of multiple documents, a question, an answer and supporting facts. Supporting facts is a set of paragraph indexes, annotated manually. 
    \item \textbf{AltQA~\cite{pal2023giraffeadventuresexpandingcontext}} is based on google's NQ~\cite{kwiatkowski_nq_2019} dataset. The answer are all numerical. The original document is "altered" so that each occurrences of the numerical answer is different from the original document, so as to avoid data contamination from pretraining. This dataset is also used in BAMBOO~\cite{dong2023bamboo} benchmark.
    \item \textbf{PaperQA} and \textbf{MeetingQA} from BAMBOO~\cite{dong2023bamboo} benchmark are question answering tasks in the form of multiple-choice.  Each instance of the two datasets consists of question , evidence, answer and corresponding content.
    \item \textbf{NarrativeQA}~\cite{narrativeqa} uses complex narratives that are self-contained as input documents. Both books and movie scripts are used. For question construction, annotators are only given a story summary, and are asked to write questions based on it. For each story(1572 stories in total), about 30 question-answer pairs are constructed from each summary-story pair. Notably, because of the consistency in story context, the task can be simplified to selecting a correct answer from all answers that relates to the story. 
    \item \textbf{MultifieldQA}~\cite{bai_longbench:_2023} is an original dataset from Longbench. Its contents covers scientific papers, legal documents, government reports and google results. The dataset has both Chinese and English version, and each instance consists of context built on documents, and a question-answer pair. 
    \item \textbf{2WikiMultihopQA}~\cite{ho_multihopqa_2020} is a multi-document QA dataset built on Wikipedia and Wikidata. WikiData is a Knowledge Graph database, from which the author was able to extract the (subject entity, property, object entity) triple that corresponds to a Wikipidia document. These triples are used as evidences in each QA pair, as a way for model to show its inference process. The dataset consists of 192,606 questions in total.
    \item \textbf{Musique~\cite{trivedi_musique:_2021}} is also a multi-document dataset(or multi-hop dataset, as the paper refers to). Its data is extracted from existing single-hop QA datasets. These single-hop QAs are then composed into multi-hop QA pairs. In addition, Musique add some unanswerable QA pairs in order to further test model's ability. There are 24,814 answerable questions in Musique, and each answerable question corresponds to an unanswerable question.
    \item \textbf{DuReader}~\cite{he_dureader:_2017} is a multi-document QA dataset, whose data is based on Baidu search results. It consists of 200,000 questions, 1,000,000 documents and 420,000 answers. Each instance contains a question, multiple possible answers(also possible to be empty), and multiple documents. 
    \item \textbf{TriviaQA}~\cite{joshi_triviaqa:_2017 } is a multi-document reading comprehension QA dataset. All QA pairs are from 14 trivia websites, written by trivia enthusiasts.  For each QA pair, 6 supporting documents(evidence) are provided, collected from Bing search API as well as Wikipedia. The total number of QA pairs is 95,956, with a total of 662,659 supporting documents, the average length of each document is 2895 words.
    \item \textbf{TOEFL(L-Eval)}~\cite{an_l-eval:_2023} collect lectures from the TOEFL Practice Online as context . Each instance consists of a long input of lectures, multiple instructions(questions) and corresponding answers. 
    \item \textbf{Coursera(L-Eval)}~\cite{an_l-eval:_2023} is a dataset built on Coursera website. Similar to TOFEL, Each instance consists of a long input of lectures, multiple instructions and corresponding answers.     
    \item \textbf{SFiction(L-Eval)}~\cite{an_l-eval:_2023} is based on scientific fictions, in which context real-world principles don't apply. The questions contained in the documents ask the model to answer it based on either contextual information or real-world knowledge, as a way to test model hallucination. 
    \item \textbf{LongFQA(L-Eval)}~\cite{an_l-eval:_2023} is an open-ended QA dataset on finance based on earnings call transcripts. 
    \item \textbf{CUAD(L-Eval)}\cite{an_l-eval:_2023} is drawn from the CUAD~\cite{hendrycks_cuad:_2021} dataset, which use legal contract as its context. 
    \item \textbf{QuALITY}~\cite{pang_quality_dataset:_2021} is a multiple-choice single-document QA dataset. It uses science fictions, magazine articles and nonfiction articles as input documents. The question is written by those that have read the full document. Each instance contains a document, a multiple-choice questions and corresponding answers. Notably, part of the questions are unanswerable. 
    \item \textbf{NewsQA}~\cite{kwan_m4le:_2023} and \textbf{DuoRC}~\cite{kwan_m4le:_2023} are English QA datasets, constructed from news and movie plots, respectively. 
    \item \textbf{C3}~\cite{kwan_m4le:_2023} is a multiple-choice QA dataset, based on second-language Chinese exams.  
    \item \textbf{NQ}~\cite{kwiatkowski_nq_2019} is a QA dataset based on Wikipedia pages. Each instance(or example, as referred to in original paper) consists of a question, corresponding wikipedia page, a long answer and a short answer. 
\end{itemize}



%Text-NLI

\begin{table*}[]
    \centering
    \caption{Text Reasoning/Classification Datasets. \textbf{CLS:} Classification. In the \textbf{Avg. Len: }  average length, \textbf{Tok:}  tokens;  \textbf{W: } words. In the Instances column, \textbf{Doc:} documents, \textbf{Inst:} instructions. In the Metric column, \textbf{EM:} Exact Match.  \textbf{Acc:} Accuracy. %\textbf{Rouge}~\cite{lin-2004-rouge} and its variants measure model's performance by calculating overlap between model output and reference answer with unigram(Rouge-1), bigram(Rouge-2), LCS(Rouge-L), etc. \textbf{F1}~\cite{rajpurkar-etal-2016-squad-F1} calculates unigram overlap between model output and answers after processing elments like white-spaces and stop-words. Particularly, for datasets from L-Eval, the GPT-4 metric means the win-rate against Turbo-16K, judged by GPT-4. $\Delta L$ is the length difference between answer length and ground truth.
    }
    \label{tab:text-dataset-reasoning}
    
    \renewcommand{\arraystretch}{1.3} % 调整行间距
    \setlength{\tabcolsep}{1pt} % 减小列间距
    \label{tab:text-dataset-nli-classification}
\begin{tabular}{c|c|c|c|c|c|c}
\hline
\textbf{Task} &
  \textbf{Name} &
  \textbf{Source} &
  \textbf{Instances} &
  \textbf{Avg Len} &
  \textbf{Metric} &
  \textbf{Lang.} \\ \hline
CLS/Reasoning &
  \href{https://github.com/google-research/long-range-arena}{Long ListOps} ~\cite{tay_long_lra_2020}&
  Generated&
  100,003 &
  3106 W &
  Acc &
  EN \\ \hline
%CLS &
%  \href{https://github.com/google-research/long-range-arena}{Byte-Level Text Classification} ~\cite{tay_long_lra_2020}&
%  Reviews&
%   &
%  4000 Tok&
%  Acc(CLS) &
%  EN \\ \hline
Reasoning &
  \href{https://stanfordnlp.github.io/contract-nli/}{ContractNLI} ~\cite{koreeda-manning-2021-contractnli-dataset}&
  Legal &
  10,319 & %607*17
  2254 Tok&
  EM &
  EN \\ \hline
CLS &
  \href{https://huggingface.co/datasets/THUDM/LongBench}{LSHT(LongBench)}~\cite{bai_longbench:_2023} &
  News &
  200 &
  22,337 W &
  Acc &
  ZH \\ \hline
Reasoning &
  \href{https://github.com/OpenLMLab/LEval}{GSM(16 shot)}~\cite{an_l-eval:_2023} &
  GSM8K \cite{cobbe_training——gsm8k_2021} &
  100 Doc, 100 Inst&
  5557 Tok&
  Rouge-L, GPT-4, $\Delta L$&
  EN \\ \hline
Reasoning &
  \href{https://github.com/RUCAIBox/BAMBOO/tree/main/datasets}{SenHallu(BAMBOO)} ~\cite{dong2023bamboo}&
  Paper &
  200/200 &
  3170/6357 Tok&
  Precision, Recall, F1&
  EN    \\ \hline
Reasoning &
  \href{https://github.com/RUCAIBox/BAMBOO/tree/main/datasets}{AbsHallu(BAMBOO)} ~\cite{dong2023bamboo}&
  Paper &
  200/200 &
  3314/6445 Tok&
  Precision, Recall, F1&
  EN \\ \hline
CLS &
  \href{https://github.com/alinapetukhova/mn-ds-news-classification}{MNDS News} ~\cite{petukhova_mn-ds:_2023}&
  News &
  10,917 &
  637 W &
  Acc &
  EN \\  \hline
\end{tabular}
\end{table*}


\subsubsection{Text Summarization Task}
A summarization dataset is a curated collection of texts and their corresponding summaries. They typically include diverse content, such as news articles, scientific papers, or conversational data, paired with concise and accurate summaries. 
The detailed statistics of the datasets for text summarization are listed in Table~\ref{tab:text-dataset-summarization}.

\begin{itemize}[leftmargin=10pt]
    \item \textbf{CNN/Dailymail}~\cite{nallapati_abstractive_dailymail_2016}, \textbf{GovReport}~\cite{huang_govreport_2021}, and \textbf{XSum}~\cite{narayan_dont_xsum_2018} include a document and its corresponding summary in each instance. CNN/Dailymail is based on over 300,000 news articles, GovReport is based on 14,466 long government reports, and XSum is based on BBC news. 
 \item \textbf{MultiNews}~\cite{fabbri_multi-news:_2019} is a multi-doc summary dataset, each instance consists of multiple news and a summary. 
 \item \textbf{Loogle}~\cite{li_loogle:_2023} is based on papers, WikiPedia, movie and TV scripts. Each long input text corresponds to mutiple question-answer-summary triad. In total there are 776 documents and 6,448 questions. Average document length is 19.367 words. 
 \item \textbf{VCSUM}~\cite{wu_vcsum:_2023} is based on real-world Chinese meeting transcripts. Each meeting tarnscript corresponds to a headline, segmentation summaries and an overall summary. There're 239 meetings in total. 
 %\item \textbf{SAMSum}~\cite{gliwa_samsum_2019} is based on message-like conversations, written by fluent English users. Each instance contains a dialogue and corresponding summary. 
 \item \textbf{SummScreenFD}~\cite{chen_summscreen:_2021} is based on TV transcripts. Each instance consists of a TV transcript containing conversations, scenes and actor actions, and a summary(recapitulation, as referred to in original paper). 
 \item \textbf{BigPatent}~\cite{sharma_bigpatent:_2019} is based on 1,341,362 patent documents. The highlight of this dataset is that important information is distributed evenly in patent documents, compared to other types of documents. Each instance contains a document and its corresponding summary(human written abstract). 
 \item \textbf{SPACE}~\cite{angelidis_extractive_space_2021} is based on reviews of 50 hotels. The highlight of the dataset is that the summaries are written in 6 different aspects, based on the hotel's review. Each hotel constructs an instance, containing the hotel's name, multiple reviews, summaries of different aspects and an overall summary. 
 \item \textbf{SQuality}~\cite{wang2022squality} is based on the same stories domain as QuALITY~\cite{pang_quality_dataset:_2021} dataset. It's a query-based summarization dataset. Each instance contains a story, multiple summarization questions, and multiple summarizations that corresponds to each questions. There are 625 QA pairs in total. 
 \item \textbf{CNNNews(M4LE)}~\cite{kwan_m4le:_2023} is based on CNN English news. Each instance of the dataset is paired with a multi-sentence summary. %NOTE: not M4LE original
 \item \textbf{CEPSUM(M4LE)}~\cite{kwan_m4le:_2023} is based on product information from Chinese e-commerce platform. Each instance contains a product description and corresponding summary. 
 \item \textbf{LCSTS(M4LE)}~\cite{kwan_m4le:_2023} is a summarization dataset in Chinese. It consists of over 2 million posts from a Chinese micro-blogging website, each post is paired with a summary. M4LE selects instances whose article has over 30 words.
 \item \textbf{NCLS(M4LE)}~\cite{kwan_m4le:_2023} is a summarization dataset with articles and corresponding summaries in different language, which highlights model's cross-lingual ability. Original NCLS is constructed from CNNNews and LCSTS.
 \item \textbf{WikiHow(M4LE)}~\cite{kwan_m4le:_2023} is based on procedural descriptions on Wikipedia. Each article is entitled with a beginning of "How to...". Each paragraph of the article describes one step in the procedure, and corresponds to short summary. These summaries are then put together as the suymmary of the article. 
 \item \textbf{News2016(M4LE)}~\cite{kwan_m4le:_2023} is based on ove 2 million news articles in Chinese. For each article, its title is used as golden summary. M4LE remove instances whose length is less than 200 words or over 800 words. 
 \item \textbf{PubMed(M4LE)}~\cite{kwan_m4le:_2023} is based on medical papers. In M4LE, each paper's abstract is used as the summary of the paper. 
 \item \textbf{BookSum(M4LE)}~\cite{kwan_m4le:_2023} is a dataset containing 405 English books, whose contents covers plays, novels and short stories. Each chapter of the content corresponds to a human-written summary. 
 \item \textbf{CNewsum(M4LE)}~\cite{kwan_m4le:_2023} is based on 304,307 news articles in Chinese. Each article corresponds to a human-written summary.
 \item \textbf{CLTS+(M4LE)}~\cite{kwan_m4le:_2023} is based on CLTS~\cite{zhu_natural_clts_2020}. CLTS contains over 180,000 Chinese articles, and CLTS+ uses back translation to make summaries more abstractive. M4LE selects part of these instances for benchmark.
 \item \textbf{Arxiv(M4LE)}~\cite{kwan_m4le:_2023} is based on papers collected from arXiv.org. For each paper, its abstract is used as golden summary.

\end{itemize}

\subsubsection{Text Reasoning Task}
A reasoning task involves the ability of a model to draw logical conclusions, make inferences, or solve problems based on given information. It requires understanding relationships, patterns, or rules within the data to arrive at accurate and coherent outcomes.Natural Language Inference(NLI) can be considered a subset of reasoning. It highlights model's ability to perform logical inference instructed by natural language.In an NLI task, the typical goal is to determine the relationship between two pieces of text: a premise and a hypothesis.
The detailed statistics of the datasets for text reasoning are listed in Tab.~\ref{tab:text-dataset-reasoning}.

\begin{itemize}[leftmargin=10pt]
    \item \textbf{Long Listops}~\cite{tay_long_lra_2020} is a mathematical reasoning dataset. It inputs an listop expression, instructing the model to perform calculation and output the exact numeric answer. A listop expression has a hierarchical structure that involves a set of simple mathematical operators. The final answer is a number in 0-9, described in original paper as "a ten-way classification task". 
    \item  \textbf{GSM}~\cite{cobbe_training——gsm8k_2021} is a mathematcal reasoning dataset, which describes mathematical problems in natural language and ask the model to solve it.
    \item \textbf{ContractNLI}~\cite{koreeda-manning-2021-contractnli-dataset} uses contracts as context, and provides hypothesis, answer, and added evidence to each instance as well. The task requires model to judge the relationship between the hypothesis and context. Each instance contains 607 contracts, each contract has 17 annotated hypothesis and corresponding answers. 
    \item \textbf{LSHT(LongBench)}~\cite{bai_longbench:_2023} is a Chinese classification dataset. It's based on Xinhua News. The model is asked to classify the input news articles into different categories.
    \item \textbf{SenHallu}~\cite{dong2023bamboo} and \textbf{AbsHallu} ~\cite{dong2023bamboo}use content and a related hypothesis as model's input, and instruct the model to determine whether the hypothesis is true based on the content. The false hypothesis(hallucination, as referred to by original paper) is generated by GPT.
    \item \textbf{MNDS News}~\cite{petukhova_mn-ds:_2023} is a classification dataset consisting of 10.917 news articles. The news articles have 17 first level categories and 109 second-level categories. 
\end{itemize}



%Text-RET

\begin{table*}[]
    \centering
    \caption{Text Dataset-Retrieval. In the \textbf{Avg. Len: }  average length, \textbf{W: } words. Particularly, LongEval, StreamingEval and TopicRet is more of a data generation method, which makes their length and instance number flexible, denoted by '-'. In the Metric column, \textbf{Acc:} Accuracy. \textbf{F1}~\cite{rajpurkar-etal-2016-squad-F1} calculates unigram overlap between model output and answers after processing elments like white-spaces and stop-words. }
    \renewcommand{\arraystretch}{1.3} % 调整行间距
    \setlength{\tabcolsep}{1pt} % 减小列间距
    \label{tab:text-dataset-retrieval}
\begin{tabular}{c|c|c|c|c|c|c}
\hline
\textbf{Task} &
  \textbf{Name} &
  \textbf{Source} &
  \textbf{Instances} &
  \textbf{Avg Len} &
  \textbf{Metric} &
  \textbf{Lang.} \\ \hline
CLS/RET &
  \href{https://huggingface.co/datasets/THUDM/LongBench}{TREC(LongBench)}~\cite{bai_longbench:_2023} &
  Web Question &
  200 &
  5177 W &
  Acc &
  EN \\ \hline
RET &
  \href{https://github.com/DachengLi1/LongChat}{LongEval} ~\cite{longchat2023} &
  Conversations&
  - &
  - &
  Acc&
  EN \\ \hline
RET &
  \href{http://arxiv.org/abs/2309.17453}{StreamingEval} ~\cite{DBLP:conf/iclr/XiaoTCHL24} &
  LongChat\cite{longchat2023} &
  -&
  - &
  Acc&
  EN \\ \hline
RET &
  \href{https://github.com/OpenLMLab/LEval}{TopicRet(L-Eval)} ~\cite{an_l-eval:_2023} &
  LongChat\cite{longchat2023} &
  -&
  -&
  Acc&
  EN \\ \hline
%RET &
%  \href{https://github.com/KwanWaiChung/M4LE}{WoW(M4LE)}~\cite{kwan_m4le:_2023} &
%  Wiki &
%   &
%  3434W &
%  Acc &
%  EN \\ \hline
RET &
  \href{https://github.com/KwanWaiChung/M4LE}{DRCD(M4LE)}~\cite{kwan_m4le:_2023} &
  Wiki &
  - &
  3617 W&
  Acc &
  ZH \\ \hline
%RET &
%  \href{https://github.com/google-research/long-range-arena}{Byte-Level Document Retrieval} ~\cite{tay_long_lra_2020}&
%  Papers&
%   &
%  9386.2 W &
%  Acc&
%  EN \\ \hline
CLS+RET &
  \href{https://github.com/KwanWaiChung/M4LE}MARC~\cite{kwan_m4le:_2023} &
  E-Commerce &
  2200 &
  3543 W&
  F1 &
  EN,ZH \\ \hline
CLS+RET &
  \href{https://github.com/KwanWaiChung/M4LE}{Online Shopping(M4LE)}~\cite{kwan_m4le:_2023} &
  E-Commerce &
  2200 &
  3714 W&
  F1 &
  ZH \\ \hline

CLS+RET &
  \href{https://github.com/KwanWaiChung/M4LE}{MNDS News(M4LE)}~\cite{kwan_m4le:_2023} &
  MNDS News~\cite{petukhova_mn-ds:_2023} &
  - &
  3805 W &
  Acc &
  EN \\ \hline
CLS+RET &
  \href{https://github.com/KwanWaiChung/M4LE}{THUCNews(M4LE)}~\cite{kwan_m4le:_2023} &
  News &
  - &
  3721 W&
  Acc &
  ZH \\ \hline
\end{tabular}
\end{table*}

%Text-Aggregation



\subsubsection{Text Retrieval Task}
A retrieval task in LLM benchmarks evaluates a model's ability to retrieve relevant information from a large collection of data based on a given query. It tests the model's understanding of the query, semantic matching, and efficiency in identifying the most relevant documents or pieces of information. 
The detailed statistics of the datasets for text retrieval are listed in Table~\ref{tab:text-dataset-retrieval}.

\begin{itemize}[leftmargin=10pt]
    \item \textbf{LongChat}~\cite{longchat2023} has two subtask dataset for retrieval. Coarse-grained Topic Retrieval dataset use a long document that talk about a number of different topics, and instrutct the model to retrieve the first topic of the document. Fine-grained Line retrieval, on the other hand, is more challenging, which present the model with multiple lines that contain a diffrernt number and label, with similar line patterns. The model is asked to retrieve the number of a specific labeled line.Notably, such dataset can be easily constructed or generated, so it's easy to create an ultra long dataset of this type. Because the dataset is easily constructed by definition, the length of the dataset and the number of instances is indefinite. 
    \item \textbf{StreamingEval}\cite{DBLP:conf/iclr/XiaoTCHL24} construct a line retrieval task based on LongChat, which makes a query in every 10 lines, with its answer about 20 lines above, so as to evaluate the streaming conversation scenario.
    \item  \textbf{TopicRet}~\cite{an_l-eval:_2023} on the other hand, is based on the coarse-grained topic retrieval task, but ask about the second or third topic instead of the first one, so as to make the task more challenging.
%    \item \textbf{WOW(M4LE)}~\cite{kwan_m4le:_2023}  %NOTE(xzc): Not described in M4LE
    \item \textbf{DRCD(M4LE)}~\cite{kwan_m4le:_2023} is a reading comprehension dataset. In M4LE, DRCD is constructed into two subset, one(DRCD explicit) require model to return the articles' IDs related to a given topic, and another subset(DRCD semantic) requires the model to answer specific questions given multiple paragraphs. 
%    \item \textbf{Byte-Level Document Retrieval}~\cite{tay_long_lra_2020}
    \item \textbf{MARC}~\cite{kwan_m4le:_2023} consists of  bilingual(namely English and Chinese) 
reviews. The model is asked to identify all positive reviews and retrieve them. 
\item \textbf{Online Shopping(M4LE)}~\cite{kwan_m4le:_2023} is based on 60K product reviews on Chinese e-commerce platforms. Reviews are categorized into positive and negative. 
\end{itemize}

%Text-GEN
\begin{table*}[]
    \centering
    \caption{Text Dataset-Generation.In the \textbf{Avg. Len: }  average length, \textbf{Tok:}  tokens;  \textbf{W: } words. In the Instances column, \textbf{Doc:} documents, \textbf{Inst:} instructions. In the Metric column, \textbf{EM:} Exact Match. \textbf{Acc:} Accuracy. 
    %\textbf{BLEU}\cite{papineni-etal-2002-bleu}, denoting Bilingual Evaluation Understudy, is also a metric designed for machine translation. \textbf{Rouge}~\cite{lin-2004-rouge} and its variants measure model's performance by calculating overlap between model output and reference answer with unigram(Rouge-1), bigram(Rouge-2), LCS(Rouge-L), etc. \textbf{F1}~\cite{rajpurkar-etal-2016-squad-F1} calculates unigram overlap between model output and answers after processing elments like white-spaces and stop-words. \textbf{BERT}~\cite{zhang2020bertscoreevaluatingtextgeneration} is a text generation task metric using contextual embeddings. \textbf{Edit Sim} is a metric based on edit diatance. Particularly, in MultiDoc2Dia, \textbf{SacreBLEU}~\cite{post-2018-call-sacrebleu} is a unified reference version of BLEU. For datasets from L-Eval, the GPT-4 metric means the win-rate against Turbo-16K, judged by GPT-4. $\Delta L$ is the length difference between answer length and ground truth.
    }
    \renewcommand{\arraystretch}{1.3} % 调整行间距
    \setlength{\tabcolsep}{1pt} % 减小列间距
    \label{tab:text-dataset-generation}
\begin{tabular}{c|c|c|c|c|c|c}
\hline
\textbf{Task} &
  \textbf{Name} &
  \textbf{Source} &
  \textbf{Instances} &
  \textbf{Avg Len} &
  \textbf{Metric} &
  \textbf{Lang.} \\ \hline
GEN &
  \href{https://github.com/microsoft/CodeBERT/tree/master/LongCoder}{LCC} ~\cite{guo_longcoder_lcc:_2023}&
  Code &
  360000 &
  1337 W &
  EM, Edit Sim &
  Python/CSharp/Java \\ \hline
GEN &
  \href{https://github.com/Leolty/repobench}{RepoBench-P(LongBench)}~\cite{bai_longbench:_2023} &
  Code &
  500 &
  4206 W &
  Edit Sim &
  Python/Java \\ \hline
GEN/RET &
  \href{https://github.com/IBM/multidoc2dial}{MultiDoc2Dial} \cite{feng_multidoc2dial:_2021}&
  Doc2Dial \cite{feng_doc2dial:_2020} &
  \begin{tabular}[c]{@{}c@{}}488 Doc, \\4796 Dialogues\end{tabular} &
  \begin{tabular}[c]{@{}c@{}}4283 T\end{tabular}&
  \begin{tabular}[c]{@{}c@{}}F1, EM, SacreBLEU, Recall\end{tabular}&
  EN \\ \hline
GEN &
   \href{https://github.com/OpenLMLab/LEval}{OpenReview(L-Eval)} ~\cite{an_l-eval:_2023}&
  ASAP-Review\cite{yuan_can_asap_review_2021} &
  20 Doc 60 Inst &
  11,170 Tok&
   Rouge-L, GPT-4, $\Delta L$&
   EN\\ \hline
GEN &
  \href{https://github.com/neulab/ReviewAdvisor}{ASAP-Review} ~\cite{yuan_can_asap_review_2021}&
  Paper&
  \begin{tabular}[c]{@{}c@{}}8877 Papers, \\25,986 Reviews\end{tabular} &
  \begin{tabular}[c]{@{}c@{}}6782 W/Paper\end{tabular}&
  Rouge-1/2/L, BERT&
  EN \\ \hline
GEN &
  \href{https://github.com/RUCAIBox/BAMBOO/tree/main/datasets}{ShowsPred} ~\cite{dong2023bamboo}&
  TV Shows &
  100/100 &
  2389/4860 Tok&
  Acc&
  EN \\ \hline
GEN &
  \href{https://github.com/RUCAIBox/BAMBOO/tree/main/datasets}{MeetingPred} ~\cite{dong2023bamboo}&
  Meeting &
  100/100 &
  3689/11578 Tok&
  Acc&
  EN \\ \hline
GEN-Code &
  \href{https://github.com/RUCAIBox/BAMBOO/tree/main/datasets}{PrivateEval} ~\cite{dong2023bamboo}&
  Code &
  152/152 &
  3149/6230 Tok&
  Pass@1&
  EN, Python \\ \hline
GEN-Code&
   \href{https://github.com/OpenLMLab/LEval}{CodeU(L-Eval)} ~\cite{an_l-eval:_2023}&
  Code&
  90 Doc 10 Inst&
  31,575 Tok&
   Rouge-L, GPT-4, $\Delta L$&
  Python \\ 

 \hline
\end{tabular}
\end{table*}

\begin{table*}[t]
    \centering
    \caption{Text Dataset-Aggregation. In the \textbf{Avg. Len: }  average length, \textbf{Tok:}  tokens;  \textbf{W: } words. In the Instances column, \textbf{Doc:} documents, \textbf{Inst:} instructions. In the Metric column, \textbf{Acc:} Accuracy. \textbf{ES:} Exponential Similarity, \textbf{CI:} Concordance Index}
    \renewcommand{\arraystretch}{1.3} % 调整行间距
    \setlength{\tabcolsep}{1pt} % 减小列间距
    \label{tab:text-dataset-aggregation}
\begin{tabular}{c|c|c|c|c|c|c}
\hline
\textbf{Task} &
  \textbf{Name} &
  \textbf{Source} &
  \textbf{Instances} &
  \textbf{Avg Len} &
  \textbf{Metric} &
  \textbf{Lang.} \\ \hline

AGG &
  \href{https://github.com/tau-nlp/zero_scrolls}{SpaceDigest} ~\cite{shaham_zeroscrolls:_2023}&
  Reviews &
  500 &
  5481 W &
  ES & %zeroscrolls ES -> exponential similarity and Cidx -> concordance index.
  EN \\ \hline
AGG &
  \href{https://github.com/tau-nlp/zero_scrolls}{BookSumSort} ~\cite{shaham_zeroscrolls:_2023}&
  Literature &
  500 &
  6840 W&
  CI & % concordance index
  EN \\ \hline
AGG &
  \href{https://huggingface.co/datasets/THUDM/LongBench}{PassageRetrieval}-en~\cite{bai_longbench:_2023} &
  Wiki &
  200 &
  9289 W &
  Acc &
  EN \\ \hline
AGG &
  \href{https://huggingface.co/datasets/THUDM/LongBench}{PassageRetrieval-zh}~\cite{bai_longbench:_2023} &
  C4 Dataset &
  200 &
  6745 W &
  Acc &
  ZH \\ \hline
AGG &
  \href{https://huggingface.co/datasets/THUDM/LongBench}{PassageCount} ~\cite{bai_longbench:_2023} &
  Wiki &
  200 &
  11,141 W &
  Acc &
  EN \\ \hline
AGG &
  \href{https://github.com/RUCAIBox/BAMBOO/tree/main/datasets}{ShowsReport(BAMBOO)} ~\cite{dong2023bamboo}&
  TV Shows &
  200/200 &
  2992/6411 Tok&
  CI & %Bamboo refered concordance index as CI
  EN \\ \hline
AGG&
  \href{https://github.com/RUCAIBox/BAMBOO/tree/main/datasets}{ReportSumSort(BAMBOO)} ~\cite{dong2023bamboo}&
  Reports &
  150/150 &
  3753/8309 Tok&
  CI &
  EN \\ \hline
\end{tabular}
\end{table*}


\subsubsection{Text Generation Task}
Generation tasks require model to generate contents based on the given instructions and context. 
The detailed statistics of the datasets for text generation are listed in Table~\ref{tab:text-dataset-generation}.

\begin{itemize}[leftmargin=10pt]
    \item \textbf{MultiDoc2Dial}~\cite{feng_multidoc2dial:_2021} gives model a dialogue history and all involved documents, and instruct model to generate the next turn of the dialogue.
    \item \textbf{OpenReview(L-Eval)}~\cite{an_l-eval:_2023}, which is based on \textbf{ASAP-Review}~\cite{yuan_can_asap_review_2021}, provides LLM with a paper and instruct it to generate a review.
    \item \textbf{ShowsPred} and \textbf{MeetingPred}~\cite{dong2023bamboo} use dialogue history as input, and ask model to infer which role said the last turn of the conversation. Apart from natural language context, code generation is also an important implementation for LLMs.
    \item \textbf{LCC}~\cite{guo_longcoder_lcc:_2023} gives model long code snippets as context, and instruct model to generate the following line of code.
    \item \textbf{RepoBench-P}~\cite{liu_repobench:_2023} requires model to retrieve toe most relevant code snippets from a long input, and then generate code according to the instruction.
    \item \textbf{PrivateEval}~\cite{dong2023bamboo} use API documents and a code snippet as input, and instruct the model to generate 
code acccordingly. Notably, to avoid data contamination caused by pre-training, the keywords in API documents are modified, making the document "private".
\item \textbf{CodeU}~\cite{dong2023bamboo} use the same practice of modifying keyword, only that it uses modified source code of public library, rather than API document, as an input. 
\end{itemize}



\subsubsection{Aggregation Task}
Aggregation task involves understanding and aggregating information from the whole input to answer complex instructions, such as calculating the percentage of positive comments given a set of comments of different attitudes. 
The detailed statistics of the datasets for text aggregation are listed in Table~\ref{tab:text-dataset-aggregation}.

\begin{itemize}[leftmargin=10pt]
    \item \textbf{SpaceDigest}~\cite{shaham_zeroscrolls:_2023} give the model a set of hotel reviews, and ask the model to output the percentage of positive reviews in the context.
    \item \textbf{BookSumSort}~\cite{shaham_zeroscrolls:_2023}, \textbf{ReportSumSort}~\cite{dong2023bamboo}, and \textbf{ShowsSort}~\cite{dong2023bamboo} use shuffled paragraphs from book summaries, TV transcripts or government reports as context, and ask the model to sort them in the correct order.
    \item \textbf{PassageCount}~\cite{bai_longbench:_2023} selects multiple passage, duplicates some of the paragraphs, and put all those paragraphs into an instance after shuffling. The model is then asked to determine how many documents are used to construct this instance.
    \item \textbf{PassageRetrieval}~\cite{bai_longbench:_2023}, on the other hand, selects 30 wikipedia passages, and use GPT-3.5-Turbo to write a summary for one of them. Then these passages and the generated summary are used as the model input. The model is then instructed to tell which passage was the summary generated from.
\end{itemize}

\subsubsection{Evaluation Metric for Text Datasets}\label{sssec:text_metric}
General evaluation metrics used by text datasets mentioned above include \textbf{Exact Match}~\cite{DBLP:journals/corr/RajpurkarZLL16}, \textbf{Partial Match}, \textbf{Accuracy}, \textbf{Recall}, \textbf{Precision}, \textbf{F1}, \textbf{BLEU}~\cite{papineni-etal-2002-bleu}, \textbf{SacreBLEU}~\cite{post-2018-call-sacrebleu}, \textbf{Rouge}~\cite{lin-2004-rouge}, \textbf{METEOR}~\cite{denkowski-lavie-2011-meteor}, \textbf{BERT}~\cite{zhang2020bertscoreevaluatingtextgeneration}, \textbf{Edit Similarity}, \textbf{Pass@k}~\cite{chen2021evaluatinglargelanguagemodels} , \textbf{Exponential Similarity}, \textbf{Concordance Index},  \textbf{Mean Reciprocal Rank}. In addition to general evaluation metrics, some more specific metrics are used in particular benchmarks. For datasets from L-Eval~\cite{an_l-eval:_2023}, the \textbf{GPT-4} metric means the win-rate against Turbo-16K, judged by GPT-4. \textbf{$\Delta L$} is the length difference between answer length and ground truth. For LooGLE~\cite{li_loogle:_2023}, it utilizes \textbf{GPT-4} for its QA and summarization task, using it for answer's semantic judgment.
%And particularly...
%\textbf{GPT-4, $\Delta L$} %L-Eval
%\textbf{MRR} %narrativeqa
%\textbf{Partial Match, GPT-4} %Loogle
%\textbf{Gold Rouge-1} %VCSUM
%\textbf{Precision, Recall} %BAMBOO
%REF: https://zhuanlan.zhihu.com/p/405658103 https://zhuanlan.zhihu.com/p/130570024
\begin{itemize}[leftmargin=10pt]
    \item \textbf{Exact Match (EM)}~\cite{DBLP:journals/corr/RajpurkarZLL16} is a metric used to evaluate the accuracy of models in tasks like question answering or text generation. It measures the percentage of predictions that exactly match the ground truth answer, considering both the content and format.% EM is a strict metric, as even minor differences (e.g., punctuation or phrasing) between the prediction and the reference can result in a non-match. %NOTE: Verified
    \item \textbf{Partial Match (PM)} metric evaluates the similarity between a model's output and the reference by allowing partial credit for partially correct answers. Unlike strict metrics like Exact Match (EM), PM accounts for overlaps or shared elements, such as keywords or phrases, making it more flexible in assessing performance.% This metric is particularly useful in tasks where approximate correctness is acceptable or expected. %NOTE: No paper! May be different!
    \item \textbf{Accuracy} is a metric used to evaluate the overall performance of a model by measuring the proportion of correctly predicted instances (both positive and negative) out of the total instances. %It is simple to calculate and widely used, but it may not be reliable for imbalanced datasets where one class dominates. In such cases, alternative metrics like F1 score, Precision, or Recall are often more informative. %NOTE: Verified
    \item \textbf{Recall} is a metric used to evaluate a model's ability to retrieve all relevant instances in a dataset. It is calculated as the ratio of correctly retrieved relevant items to the total number of relevant items, emphasizing completeness. %High recall indicates that the model retrieves most of the relevant items, but it does not account for the precision or quality of the retrieved results. %NOTE: Verified
    \item \textbf{Precision} is a metric used to evaluate the accuracy of a model by measuring the proportion of correctly predicted positive instances out of all predicted positive instances. %NOTE: Verified
    \item  \textbf{F1} is a performance measure that combines Precision and Recall into a single score using their harmonic mean. It provides a balanced evaluation, especially useful in datasets with imbalanced classes, by considering both false positives and false negatives. %Verified
    \item \textbf{BLEU}~\cite{papineni-etal-2002-bleu}, is a widely used metric for evaluating the quality of machine-generated text, especially in machine translation. It works by comparing n-grams in the generated output with reference texts to measure overlap, while applying penalties for overly short outputs to ensure fluency. %Verified
    \item \textbf{SacreBLEU}~\cite{post-2018-call-sacrebleu} is a standardized version of the BLEU metric used to evaluate machine translation quality. It simplifies BLEU's implementation by fixing preprocessing steps like reference handling to ensure consistent and reproducible results across different systems. %NOTE: verified
    \item  \textbf{Rouge}~\cite{lin-2004-rouge} and its variants measure model's performance by calculating overlap between model output and reference answer with unigram(\textbf{Rouge-1}), bigram(\textbf{Rouge-2}), LCS(\textbf{Rouge-L}), etc. \textbf{Gold Rouge-1} in VCSUM dataset refers to using high-quality reference summaries (gold standards) for evaluation, ensuring reliable and meaningful comparisons. %Verified
    \item \textbf{METEOR}~\cite{denkowski-lavie-2011-meteor} (Metric for Evaluation of Translation with Explicit ORdering) is a text evaluation metric designed to assess the quality of machine translation. %It improves upon BLEU by considering exact matches, synonyms, stemming, and word order, providing a more nuanced evaluation of semantic similarity between generated and reference texts. %Verified
    \item \textbf{BERT}~\cite{zhang2020bertscoreevaluatingtextgeneration} metric, often referred to as BERTScore, is a text evaluation metric that uses contextual embeddings from the BERT model to compare similarity between generated and reference texts. %Unlike traditional metrics like BLEU or ROUGE, it captures semantic meaning by aligning tokens based on their contextual representations. This makes BERTScore more effective at evaluating nuanced and meaning-based similarities in natural language tasks.  %Verified
    \item    \textbf{Edit Similarity} is a metric that measures the similarity between two text sequences based on the minimum number of edit operations required to transform one sequence into another. It is derived from the concept of edit distance such as Levenshtein distance. % and is often normalized to produce a similarity score between 0 and 1. %NOTE: Verified
    \item \textbf{Pass@k}~\cite{chen2021evaluatinglargelanguagemodels} evaluates the performance of a model by measuring the percentage that at least one of the top k generated outputs contains a correct solution. In datasets we surveyed, only \textbf{Pass@1} is used. %A higher pass@1 score indicates that the model is more likely to produce the correct result on its first attempt.  %NOTE: Verified, do we need change it to pass@k?
    \item \textbf{Exponential Similarity} is a metric that measures the similarity between two items by exponentially weighting their differences, giving more importance to smaller discrepancies.% This approach ensures that minor differences have a larger impact on the similarity score, making it sensitive to fine-grained variations. It is particularly useful in tasks where small deviations are critical, such as text similarity or pattern recognition. %NOTE: Uncertain
    \item \textbf{Concordance Index} is a metric used to evaluate the predictive accuracy of models, particularly in survival analysis or ranking tasks. % A C-index of 1 indicates perfect prediction, while 0.5 suggests random performance, making it a useful measure for assessing ranking consistency.  %NOTE: Uncertain
    \item \textbf{Mean Reciprocal Rank (MRR)} is an evaluation metric commonly used in information retrieval and recommendation systems to measure the quality of ranked results. It calculates the reciprocal of the rank of the first relevant item in a result list and averages it across all queries.% MRR emphasizes the importance of placing relevant results higher in the ranking, making it particularly useful for tasks where early retrieval of relevant items is critical. 

\end{itemize}

%\subsubsection{Text Translation Task}  %NOTE: Maybe not really necessary?
%Translation tasks involve training models to accurately convert text from one language to another while preserving meaning, context, and tone. These tasks help improve multilingual capabilities of language models, enabling better cross-cultural communication and understanding. Due to the task's straightforward nature, it's easy to create/obtain long datasets from sources like subtitles, transcripts. 

%\begin{itemize}
%    \item \textbf{TedTalks(M4LE)} The TED Talks dataset is a multilingual parallel corpus derived from TED Talk transcripts. It includes translations between English and six other languages, organized into three pairs of linguistically related languages: Galician-Portuguese, Azerbaijani-Turkish, and Belarusian-Russian. These pairs span Romance, Turkic, and Slavic language families, offering a diverse resource for studying translation across languages with varying similarities. 
%    \item \textbf{OpenSubtitles(M4LE)} 
%    \item \textbf{News Commentary(M4LE)}
%\end{itemize}



\begin{table*}[]
    \centering
    \caption{Multimodal Dataset.  Specfically, for data type, \textbf{Img}: Image; \textbf{T}: text; \textbf{V:} Video. For task abbreviation, \textbf{Conv}: conversation task; \textbf{Desc}: description task;  \textbf{Reas: } reasoning task; \textbf{Perc:} perception task;  \textbf{Pred: } prediction; \textbf{NTH: }  needle in the haystack; \textbf{SUMM:}  summary.
    For instance and average column, \textbf{Q}: questions; \textbf{W}: words; \textbf{s}: seconds.
    For example, \textbf{54 Img, 150 Q} denote that there are 54 images with 150 questions.    }
    \label{Multimodal Dataset}
    \renewcommand{\arraystretch}{1.4} % 调整行间距
    \setlength{\tabcolsep}{1.5pt} % 减小列间距
\begin{tabular}{c|c|c|c|c|c|c|c}
\toprule
Tasks &
  Name &
  Data &
  Source &
  Instance &
  Average &
  Metric &
  Language \\ \midrule
  
Conv, Desc, Reas &
  \href{https://github.com/LLaVA-Annonymous/LLaVA}{LLaVA-Bench}~\cite{liu2023llava} &
  Img, T &
  COCO, In-The-Wild &
  54 Img, 150 Q &
  1 Img, 59.9 W &
  Relative Score &
  EN \\ \midrule
Perc, Reas &
  \href{https://github.com/open-compass/MMBench}{MMBench}~\cite{MMBench} &
  Img, T &
  Internet &
  2948 Q &
  1 Img, 114.5 W &
  Acc &
  EN/CN \\\midrule

  \begin{tabular}[c]{@{}c@{}}Pred, Count, \\ NIH, Retrieval\end{tabular}  &
  \href{https://milebench.github.io/}{MileBench}~\cite{song2024milebench} &
  Img, T &
  \begin{tabular}[c]{@{}c@{}}Public, \\ self-building\end{tabular} &
  6440 Q &
  15.2 Img, 422.3 W &
  Acc, ROUGE-L &
  EN \\\midrule
  
\begin{tabular}[c]{@{}c@{}}Reas, NIH, SUMM,\\ Desc, Order, Count\end{tabular} &
  \href{https://github.com/JUNJIE99/MLVU}{MLVU}~\cite{MLVU} &
  V, T &
  \begin{tabular}[c]{@{}c@{}}Public, \\ self-collection\end{tabular} &
  1334 V, 2593 Q &
  704.6s V, 39.7 W &
  M-Avg, G-Avg &
  EN \\\midrule
  
Reas, Retrieval &
  \href{https://longvideobench.github.io/}{LongVideoBench}~\cite{wu2024longvideobench} &
  V, T &
  web-collected &
  3763 V, 6678 Q &
  730.5s V, 49.5 W &
  Acc &
  EN \\\midrule
  
Perc, Recognition, Reas &
  \href{https://video-mme.github.io/home_page.html}{Video-MME}~\cite{fu2024video} &
  V, T &
  YouTube &
  900 V, 2700 Q &
  1017.9s V &
  Acc &
  EN \\\midrule
  
Desc, Reas &
  \href{https://github.com/doc-doc/NExT-QA}{NExT-QA}~\cite{xiao2021next} &
  V, T &
\begin{tabular}[c]{@{}c@{}}YouTube, \\ TV Show, Public\end{tabular} &
  1000 V, 47962 Q &
  44s V, 25.5 W &
  Acc, WUPS &
  EN \\\midrule
  
Perc, Count, Reas &
  \href{https://github.com/OpenGVLab/Ask-Anything/blob/main/video_chat2/MVBENCH.md}{MVBench}~\cite{2023videochat} &
  V, T &
  Public &
  4000 Q &
  16.7s V, 31.3 W &
  Acc &
  EN \\\midrule
  
Decs &
  \href{https://github.com/xudejing/video-question-answering}{MSVD-QA}~\cite{xu2017video} &
  V, T &
  MSVD &
  1970 V, 50505 Q &
  10s V &
  Acc &
  EN \\\midrule
  
Desc &
  \href{https://github.com/xudejing/video-question-answering}{MSRVYY-QA}~\cite{xu2017video} &
  V, T &
  MSRVTT &
  10000 V, 243690 Q &
  15s V &
  Acc &
  EN \\ \bottomrule
\end{tabular}
\end{table*}

\subsection{Multimodal Datasets and Evaluation Metric}\label{ssec:multimodal_dataset}

\subsubsection{Multimodal Datasets}

Multimodal datasets have emerged to address the need for a comprehensive understanding of the complex real world by integrating diverse data types such as text, images, audio, and video. These datasets drive advancements in AI, particularly in machine learning and deep learning, by offering rich and diverse data to train more robust and versatile models.
We analyze the multimodal benchmarks listed in Table \ref{Multimodal Dataset}, highlighting their distinct focuses. Each benchmark is built upon one or more multimodal datasets, involving their collection, processing, and the use of specific validation metrics. Below, we provide a detailed introduction and description of each multimodal benchmark.

\begin{itemize}[leftmargin=10pt]
    \item \textbf{LLaVA-Bench}~\cite{liu2023llava}
The benchmark is structured around image-ground-truth textual description-question-answer triplets, segmented across COCO and In-The-Wild datasets. It assesses a model's proficiency in multimodal instruction adherence and visual reasoning. By employing a suite of tasks and metrics, it quantifies the model's ability to comprehend and act on visual-language directives, articulate comprehensive descriptions, and engage in intricate reasoning processes.

\item \textbf{MMBench}~\cite{MMBench}
This benchmark serves as a bilingual multimodal benchmark, facilitating a comparative analysis of VLM performance across English and Chinese linguistic contexts. It distinctively assesses multimodal models using a hierarchical taxonomy of abilities, stringent quality assurance measures, and a dual-language evaluation framework. Unlike other benchmarks, MMBench~\cite{MMBench} incorporates the CircularEval strategy for comprehensive evaluation and utilizes LLMs for precise extraction of choices, setting it apart from its counterparts.

\item \textbf{MileBench}~\cite{song2024milebench}
evaluates the multi-modal long-context capabilities of LLMs, including both diagnostic and realistic evaluation sets. It emphasizes long-context and multi-image tasks. This unique focus allows it to capture the complexity and diversity of real-world multimodal challenges, setting it apart from existing benchmarks. The dataset in MileBench~\cite{song2024milebench} is characterized by its inclusion of long texts integrated with multiple images, reflecting real-world scenarios where context is key. It contains a diverse range of tasks that require both comprehension and generation. %making it a comprehensive tool for evaluating LLMs.

\item \textbf{MLVU}~\cite{MLVU}
is a holistic benchmark, designed to gauge the capabilities of multi-modal LLMs in comprehending  video content, transcends the constraints of its predecessors by significantly increasing video durations, encompassing diverse video genres, and crafting a spectrum of assessment tasks. This benchmark offers an extensive array of tasks and video genres to evaluate the comprehensive competencies of MLLMs. It highlights the substantial potential for enhancement in current methodologies and emphasizes the critical factors of context length, image comprehension quality, and the selection of LLM architecture for future progress.

\item \textbf{LongVideoBench}~\cite{wu2024longvideobench}
This benchmark offers an extensive benchmarking framework aimed at assessing the capacity of large multimodal models (LMMs) to comprehend lengthy videos with subtitles, extending up to an hour. It places a strong focus on the retrieval and reasoning capabilities over extended, interwoven video and language data streams, tackling the challenge of single-frame bias and underscoring its proficiency in evaluating multimodal comprehension in long contexts.

\item \textbf{Video-MME}~\cite{fu2024video}
A benchmark for comprehensive evaluation, it assesses the proficiency of Multi-modal Large Language Models (MLLMs) in analyzing videos. This dataset comprises a wide array of 900 videos spanning diverse domains and subfields, ensuring extensive scenario coverage. It encompasses videos with lengths ranging from 11 seconds to 1 hour to gauge model flexibility across various time frames. Furthermore, it incorporates various data modalities, including subtitles and audio tracks, to evaluate the comprehensive competencies of MLLMs. The benchmark aims to test the models' capacity for sequential visual data comprehension, with an emphasis on temporal reasoning and the processing of multi-modal inputs.

\item \textbf{NExT-QA}~\cite{xiao2021next}
Advancing video comprehension from mere description to explanation of causal, temporal, and descriptive actions, a video question answering (VideoQA) benchmark has been established. This benchmark boasts a dataset with 5,440 videos and approximately 52K manually annotated question-answer pairs, sorted into causal, temporal, and descriptive categories. It poses a challenge to QA models to engage in reasoning about causal and temporal actions and to decipher complex object interactions within daily activities. Distinguished from other video benchmarks, this benchmark specifically focuses on causal and temporal action reasoning within realistic videos that are rich in object interactions. It stands as one of the largest manually annotated VideoQA datasets, offering support for both multiple-choice and open-ended questions, and includes a variety of videos that mirror real-life scenarios.

\item \textbf{MVBench}~\cite{2023videochat}
Featuring a substantial dataset, the benchmark comprises 200 multiple-choice question-answer (QA) pairs for each of the 20 temporal understanding tasks, amassing a total of 4,000 QA pairs. It draws from a variety of videos across 11 public datasets, spanning diverse domains and scenes, thereby testing models' abilities to comprehend temporal sequences. The benchmark automates the generation of multiple-choice QA pairs from existing video annotations, minimizing human involvement and ensuring a fair evaluation process.

\item \textbf{MSVD-QA}~\cite{xu2017video}
The MSVD dataset is a collection of 1,970 video clips with descriptive captions, initially for video captioning. It features diverse real-world scenarios and assesses multimodal learning models' capabilities in understanding video content and generating natural language descriptions.

\item \textbf{MSRVTT-QA}~\cite{xu2017video}
The MSR-VTT dataset comprises 10,000 video clips with 20 human-transcribed sentences each, focusing on connecting video content with language descriptions. It evaluates multimodal learning models' ability to comprehend video information and translate it into coherent captions, testing their video understanding and language generation skills in a more complex and diverse environment.
\end{itemize}


\subsubsection{Evaluation Metric for Multimodal Datasets}
The evaluation metrics for multimodal datasets include \textbf{Relative Score}, \textbf{Accuracy}, \textbf{ROUGE-L}, \textbf{M-Avg}, \textbf{G-Avg}, \textbf{WUPS}. 
Several common metrics, including \textbf{Accuracy}, \textbf{ROUHE-L}, have been introduced in Sec.~\ref{sssec:text_metric}.
Here, we only introduce the special metrics of multimodal datasets, which include \textbf{Relativa Score}, \textbf{M-Avg}, \textbf{G-Avg}, \textbf{WUPS} as follows:
\begin{itemize}[leftmargin=10pt]
    \item \textbf{Relative Score} This metric is used in LLaVA-Bench to evaluate the performance of multimodal models by comparing their outputs to a reference model, typically text-based GPT-4. It is calculated as the percentage ratio of the candidate model's score to the reference model's score, based on dimensions such as helpfulness, relevance, accuracy, and level of detail.
    
    \item \textbf{M-Avg} Multiple-Choice Average  is calculated as the mean accuracy across all multiple-choice tasks in the MLVU benchmark. The accuracy for each task is determined by the proportion of correctly predicted answers compared to the total number of questions within that task.

    \item \textbf{G-Axg} Generation Average s calculated as the mean score across all generation tasks in the MLVU benchmark. Each task is evaluated on multiple dimensions (e.g., Accuracy, Relevance, Completeness, and Reliability) using GPT-4, with scores ranging from 1 to 5. The overall score for each task is the average of these dimensions, and G-Avg is the mean of these task-level scores.

    \item \textbf{WUPS}~\cite{k2012newsimilaritymeasuretaxonomy} Wu-Palmer Similarity measures the semantic similarity between two words based on their positions in a taxonomy (e.g., WordNet). It calculates how closely related two words are by considering their least common ancestor (LCS).

\end{itemize}


\section{Model Training}
\iffalse
The continued pretraining process followed a carefully designed two-stage curriculum. In the first stage, the model was trained on a randomly shuffled corpus comprising all datasets for a total of approximately 380B tokens. This extensive training aimed to expose the model to a diverse range of data and establish a strong foundation for the subsequent stage.

In the second stage, the training continued for an additional approximately 60B tokens. During this stage, we employed a strategic data-mixing approach to enhance the model's performance in specific areas and align it with our desired objectives. Safety instructions were intermixed to reinforce the model's adherence to ethical guidelines. Wikipedia data was oversampled to strengthen the model's knowledge acquisition and factual understanding. To improve the model's language generation capabilities, English data such as stories were subsampled. Furthermore, Python coding data and markdown text were oversampled to steer the model towards producing well-formatted and code-friendly outputs, ultimately enhancing its performance in tasks related to Python programming.

This two-stage curriculum, with its targeted data mixing strategy, aimed to fine-tune the model's abilities in key areas while maintaining a balanced exposure to diverse data sources. By allocating more tokens to specific data types, such as Python coding and Wikipedia, we sought to improve the model's proficiency in these domains and align its outputs with the desired format and style.

[Discuss learning rate and total flops]

We then subsample certain domains and oversampled other datasets in order to guide the model to produce more factual data and improve it’s abilities on python.

We also noticed that intermediate checkpoints produced answers that had undesirable formatting such as “$>>>$” which was present in one of the datasets (gorilla), so we reformatted that dataset to have more regular formatting with brackets. 

Moreover, we noticed that the original Stack dataset used the tag <NAME> and this caused our model to generate this tag in some instances. Thus we augmented Stack text by adding fake names.

We also noticed that the fill-in-the-middle tags from Starcoder appeared in random places when outputting text so we continued the 20,000 steps without the fill-in-the-middle objective. Instead, we augmented certain text such as Python code with instructions and the fill-in-the-middle tags as control tokens.

We then finish the continued pretraining for a total of 100,000 steps on 400b tokens.

We then adversarially prompted the 100K step model for safety and where the model failed we created additional training examples.

Next we performed a positional encoding removing as described in Section \_. 

We then performed a finetune with the improved safety instructions and along high quality instruction data in multiple domains and languages, including instructions that produced long output, [Enhanced multilingual domain specific caption dataset], and the MEMIT-IT instruction dataset and the CLAP-INST dataset.  In this stage we added an embedding projection to the beginning of the instructions in some cases, along with a special tag <embed>. The projection is a normalized clip tower embedding. This is to align the model to be able to understand multimodal input. The training in this section totaled roughly 10B tokens.

Lastly, we performed 16+256+4096 LoRA finetune to create LoRA for our experts as described in Section \_.  The training in this section totaled roughy 90B tokens. 

After tuning, we have a model that has approximately 20B parameters (with the 16B base model, clip image towers, clap audio towers, and the 4096 LoRa). Using our router at inference, the model can route an instruction to a specific domain.

Collectively the model was trained on approximately 2T tokens and performs comparable to other english models of this size but also able to perform in other languages, domains, and modals. 

Our model is able to see and hear and understand multiple languages and is adapted to 4096 domains of tasks, while being safety aligned. 

We release all of our data except books3 and all of our models open source.


\fi

\section{Training Details}

\paragraph{High-Quality Fine-Tuning.}
Since the filtered pre-training data still contains a certain proportion of dirty data, such as subtitles, watermarks, and low-bitrate videos, we selected a subset of higher quality video data, accounting for 20\% of the total dataset, for fine-tuning in the final stage. This step effectively removed generated subtitles and watermarks and slightly improved the visual quality. However, we also observed a slight degradation in the model's semantic ability.

\paragraph{Visualizing different rope interpolation methods}
When adapting low-resolution position encoding to high-resolution, we consider two different methods: interpolation and extrapolation. We show the effects of two methods in Figure~\ref{fig:ive}. Interpolation tends to preserve global information more effectively, whereas the extrapolation better retains local details. Given that RoPE is a relative position encoding, We chose the extrapolation to maintain the relative position between pixels. 
\begin{figure}[h]
\begin{center}
\includegraphics[width=0.7\linewidth]{images/ive.jpg}
\end{center}
\caption{The comparison between the initial generation states of extrapolation and interpolation when increasing the resolution with RoPE. Extrapolation tends to generate multiple small, clear, and repetitive images, while interpolation generates a blurry large image.}
\label{fig:ive}
\end{figure}

\vspace{-2em}
\paragraph{Model \& Training Hyperparameters}
We present the model and training hyperparameters in \cref{tab:hyper2} and \cref{tab:hyper}.

\begin{table}[htbp]
\centering
\small
\begin{tabular}{ccccc}
\toprule
\textbf{Training Stage} & \textbf{stage1} & \textbf{stage2} & \textbf{stage3} & \textbf{stage4 (FT)} \\ 
\midrule
Max Resolution & 256$\times$384 & 480$\times$720 & 768$\times$1360& 768$\times$1360 \\
Max duration & 6s & 6s & 10s & 10s \\
Batch Size & 2000 & 1000 & 250 & 100 \\
Sequence Length & 25k & 75k & 700k & 700k \\
Training Steps & 400k & 220k & 120k & 10k \\

\bottomrule
\end{tabular}
\caption{Hyperparameters of CogvideoX-2b and CogVideo-5b.}
\label{tab:hyper2}
\end{table}

\begin{table}[htbp]
\centering
\small
\vspace{-2em}
\begin{tabular}{ccc}
\toprule
\textbf{Hyperparameter} & \textbf{CogvideoX-2b} & \textbf{CogVideo-5b} \\ 
\midrule
Number of Layers & 30 & 42 \\ 
Attention heads & 32 & 48 \\ 
Hidden Size & 1920 & 3072 \\
Position Encoding & sinusoidal & RoPE \\
Time Embedding Size & \multicolumn{2}{c}{256} \\
Weight Decay & \multicolumn{2}{c}{1e-4} \\
Adam $\epsilon$ & \multicolumn{2}{c}{1e-8} \\
Adam $\beta_1$ & \multicolumn{2}{c}{0.9} \\
Adam $\beta_2$ & \multicolumn{2}{c}{0.95} \\
Learning Rate Decay & \multicolumn{2}{c}{cosine} \\
Gradient Clipping & \multicolumn{2}{c}{1.0} \\
Text Length & \multicolumn{2}{c}{226} \\ 
Max Sequence Length & \multicolumn{2}{c}{82k} \\
Lowest aesthetic-value & \multicolumn{2}{c}{4.5} \\
Training Precision & \multicolumn{2}{c}{BF16} \\
\bottomrule
\end{tabular}
\caption{Hyperparameters of CogvideoX-2b and CogVideo-5b.}
\label{tab:hyper}
\end{table}



% params:
%   time_embed_dim: 512
%   elementwise_affine: True
%   num_frames: 49
%   time_compressed_rate: 4
%   latent_width: 90
%   latent_height: 60
%   num_layers: 30
%   patch_size: 2
%   in_channels: 16
%   out_channels: 16
%   hidden_size: 1920
%   adm_in_channels: 256
%   num_attention_heads: 30

%      target: dit_video_concat.DiffusionTransformer
% params:
%   time_embed_dim: 512
%   elementwise_affine: True
%   num_frames: 49
%   time_compressed_rate: 4
%   latent_width: 90
%   latent_height: 60
%   num_layers: 42
%   patch_size: 2
%   in_channels: 16
%   out_channels: 16
%   hidden_size: 3072
%   adm_in_channels: 256
%   num_attention_heads: 48

\section{Safety}\label{sec:safety}
LLMs can propagate harmful content, reinforce biases, or amplify misinformation. While users are responsible for assessing the potential risks of generated content, developers must prioritize legal and safety considerations, strengthening models against attacks that may bypass safety protocols. 

In line with the Biden-Harris US Executive Order on AI \citep{whitehouse2023fact}, we curated the Biden-Harris Redteam Dataset, consisting of 5000 instruction-response pairs, addressing key concerns such as harm, cyber-attacks, CNBR risks, illegal acts, and privacy infringement. This dataset was created using a combination of filtering human preference data on harmlessness and template-based methods, with responses reviewed and edited for quality and safety. We used this dataset to instruction-tune \system\ and evaluated its safety levels before and after tuning. Details are provided in Section \ref{sec:experiments}, with further dataset insights in Appendix \ref{ap:safety}.

\section{Evaluation}\label{sec:experiments}
\section{Experiments}
\label{sec:experiments}
\begin{figure}[h]
\centering
\includegraphics[width=\textwidth]{figures/train_nll_softmax_vs_sigmoid_v4.pdf}
\caption{Train losses comparing $\sigmoidattn$ with $\softmaxattn$.}
\label{fig:summary_nll}
\end{figure}
To empirically validate $\sigmoidattn$, we evaluate across several domains: supervised image classification using vision transformers \citep{DBLP:conf/iclr/DosovitskiyB0WZ21}, self-supervised image representation learning with SimCLR \citep{DBLP:conf/icml/ChenK0H20, DBLP:conf/icml/ZhaiLLBR0GS23}, Bootstrap Your Own Latent (BYOL) \citep{DBLP:conf/nips/GrillSATRBDPGAP20, DBLP:conf/nips/BusbridgeRALDCW23} and Masked AutoEncoders (MAE) \citep{DBLP:conf/cvpr/HeCXLDG22} as well as automatic speech recognition (ASR) \citep{synnaeve2019end,conformer} and auto-regressive language modeling (LM) \citep{DBLP:conf/nips/BrownMRSKDNSSAA20}. We also validate sequence length generalization on TED-LIUM v3~\citep{hernandez2018ted} for ASR and in small scale synthetic experiments in \cref{sec:a_se_pair_repeat_prob}.
Across all these domains and algorithms, we demonstrate that $\sigmoidattn$ matches the performance of $\softmaxattn$ (\cref{fig:summary_nll,fig:test_top1_results}), while offering training and inference speed-ups as highlighted in \cref{sec:FlashSigmoidHardwareAwareImplementation}. Empirically we make the following observations:
\begin{enumerate}[itemsep=0pt,leftmargin=*]
    \item $\sigmoidattn$ is effective for vision tasks without a bias (except MAE), but relies on LayerScale to match the performance of the baseline $\softmaxattn$ (\cref{fig:imagenet_top_1_ablations}-a) in a hyper-parameter free manner.\footnote{\Cref{sec:layerscale_free_sigmoid} demonstrates that supervised vision tasks using $\sigmoidattn$ without LayerScale can match baseline $\softmaxattn$ performance by relying on \emph{learnable} scalar bias and temperature: $\{b, t\} \in \mathbb{R}$.} All results presented for $\softmaxattn$ also fairly add LayerScale unless specified.
    \item LM and ASR are sensitive to the initial norm $|| \sigma(\mQ \mK^T/\sqrt{d_{qk}}) \mV ||$. Modulation is required via (a) relative positional embeddings like ALiBi \citep{DBLP:conf/iclr/PressSL22}, which reduces the initial attention norm by shifting logit mass to the zero regime under $\sigmoidattn$, (b) appropriate initialization of $b$ to achieve the same effect -- enabling usage of any positional embedding.
\end{enumerate}

\begin{figure}[htbp]
    \centering
    \begin{minipage}{0.48\textwidth}
        \centering
        \includegraphics[width=\textwidth]{figures/attn_norm_seed1000001_softmax_rope_vs_softmax_alibi_vs_sigmoid_sincos.png}    
        \captionsetup{justification=centering}
        \caption{$\sigmoidattn$ with SinCos.}
        \label{fig:rope_vs_sincos}
    \end{minipage}\hfill
    \begin{minipage}{0.48\textwidth}
        \centering        
        \includegraphics[width=\textwidth]{figures/attn_norm_seed1000001_softmax_rope_vs_softmax_alibi_vs_sigmoid_rope.png}
        \captionsetup{justification=centering}
        \caption{$\sigmoidattn$ with RoPE.}
        \label{fig:rope_vs_rope}
    \end{minipage}
    \hfill
    \begin{minipage}{0.48\textwidth}
        \centering
        \includegraphics[width=\textwidth]{figures/attn_norm_seed1000001_softmax_rope_vs_softmax_alibi_vs_sigmoid_alibi.png}
        \captionsetup{justification=centering}
        \caption{$\sigmoidattn$ with ALiBi.}
        \label{fig:rope_vs_alibi}
    \end{minipage}\hfill
    \begin{minipage}{0.48\textwidth}
        \centering        
        \includegraphics[width=\textwidth]{figures/attn_norm_seed1000001_softmax_rope_vs_softmax_alibi_vs_sigmoid_rope_b=-10.png}
        \captionsetup{justification=centering}
        \caption{$\sigmoidattn$ with RoPE, $b=-10$.}
        \label{fig:rope_vs_rope_b-10}
    \end{minipage}  
    \vspace{-0.4cm}
\end{figure}

\subsection{Ablations}
\label{sec:ablations}
We begin with ablations to dissect the benefits of each of our introduced components. To gain intuition about $\sigmoidattn$, we developed a research-friendly auto-regressive (AR) LM training framework to measure all components of attention and validate the effects of LayerScale, LayerNorm applied to Q and K (QK norm), different positional embedding techniques, and initialization values for $b$.
\begin{figure}[h]
    \centering
    \begin{minipage}[t]{0.48\textwidth}
        \centering
        \includegraphics[width=\textwidth]{figures/lines=activation-cols=layerscale_with_log_n_or_max3std.pdf} 
        \caption{LR sensitivity LayerScale ablation.}
        \label{fig:layerscale_ablation}
    \end{minipage}%
    \hfill
    \begin{minipage}[t]{0.48\textwidth}
        \centering
        \includegraphics[width=\textwidth]{figures/lines=activation-cols=qknorm_with_log_n_or_max3std.pdf}
        \caption{LR sensitivity QK norm ablation.}
        \label{fig:qk_norm_ablation}
    \end{minipage}
\end{figure}
\begin{figure}[h]
    \centering
    \vspace{-0.2cm}
    \includegraphics[width=\textwidth]{figures/imagenet_ablations_top1.pdf}
    \caption{ImageNet1k ViT-B/16 classification. (a) $\sigmoidattn$ is robust without QK norm (+LayerScale, -QKNorm). Removing LayerScale reduces accuracy by 1.0\% (-LayerScale, +/-QKNorm). $n^{-\alpha}$ normalization \citep{wortsman2023replacing} underperforms without LayerScale. (b) $\sigmoidattn$ multi-query attention (MQA) \citep{DBLP:journals/corr/abs-1911-02150} with one head matches multi-head attention (MHA). (c) Sigmoid with LayerScale and QK norm performs comparably to other activations, except TanH. ReLU$^2$ \citep{DBLP:conf/icml/HuaDLL22} underperforms without LayerScale and QK norm.}
    \label{fig:imagenet_top_1_ablations}
    \vspace{-0.4cm}
\end{figure}
\paragraph{Mitigating Large Attention Norms} We train a single layer AR transformer block (E=3072, D\_FF=12288) on the realnews split of C4 \citep{DBLP:journals/jmlr/RaffelSRLNMZLL20}. We train for $2^{16}$ steps using a batch size of 6 and max sequence length of 4096 using a single cycle cosine learning rate (LR) schedule without weight decay. $\sigmoidattn$ initially underperformed $\softmaxattn$ when using absolute sinusoidal (SinCos) (\cref{fig:rope_vs_sincos}) or relative (\cref{fig:rope_vs_rope}) positional embeddings (PE), which we attribute to high initial attention Frobenius norms, $\lVert \sigma(\mQ \mK^T / \sqrt{d}) \mV \rVert$. A corresponding evolution of the attention distribution and sparsity can be seen in Appendix \cref{fig:attn_evolve} and \cref{fig:attn_metric_evolve} on a synthetic task.
To address these larger attention norms, we propose: (a) using ALiBi \citep{DBLP:conf/iclr/PressSL22} whose relative bias moves initial attention logit mass to the zero region under the sigmoid activation, producing equivalent train negative log-likelihoods (\cref{fig:rope_vs_alibi}); or (b) set the attention logit bias $b$ to a negative offset proportional to the sequence length, $b \propto -\ln n$ (see \cref{sec:attn_bias_ablation} for an ablation on $b$). This enables the usage of other PE techniques like RoPE~\citep{DBLP:journals/ijon/SuALPBL24} (\cref{fig:rope_vs_rope_b-10}). 
\paragraph{LayerScale} To validate the need for LayerScale, we follow \citet{DBLP:journals/corr/abs-2309-14322} to quantify the impact on stability.
All models are trained with RoPE with $b \propto -\ln n$, using AdamW  \citep{loshchilov2017decoupled} on the 
realnews split of C4 
with $(\beta_1,\beta_2)=(0.9, 0.95)$, $\eps=10^{-8}$,  $wd=0$, 
batch size 24, maximum token sequence length of 512 from the T5 tokenizer \citep{DBLP:journals/jmlr/RaffelSRLNMZLL20}, cosine LR schedule of $2^{14}$ steps including a linear warmup of $2^{10}$ steps. 
Models have 
$n_{\text{heads}}=\kappa$,
$n_{\text{layers}}=2\times \kappa$,
$d_{\text{model}}=64\times \kappa$ and
$d_{\text{feed-forward}}=256\times\kappa$
for a scaling value $\kappa\in\{1,2,4,8,16\}$
leading to models with $\{2.2, 4.9,15.0,67.0,440.0\}M$ trainable non-embedding parameters.
Following \citet{DBLP:journals/corr/abs-2309-14322},
we sweep learning rates
$\eta\in \{3\times 10^{-4}, 1\times 10^{-3}, 3\times 10^{-3}, 1\times 10^{-2}, 3\times 10^{-2}, 1\times 10^{-1}, 3\times 10^{-1}\}$.
LR sensitivity is defined as 
$\mathbb E_{\eta\in[a,b]}\left[\min(\ell(\mathcal A(\eta)),\ell_0)-\ell^*\right]$
where $\ell(\mathcal A(\eta))$ is the loss achieved by the learning algorithm $\mathcal A$ with LR $\eta$,
$\ell_0$ is the loss at initialization, and
$\ell^*$ is the loss achieved by the best LR.
LayerScale is initialized at $10^{-4}$. 
Unlike vision tasks, where LayerScale \emph{improves performance} (\cref{fig:imagenet_top_1_ablations}-a), in LM, we observe that $\softmaxattn$ slightly benefits from LayerScale, while the performance of $\sigmoidattn$ remains largely unaffected.
\paragraph{Stability with QK Norm} \Cref{thm:regularity} indicates that the Jacobian of $\sigmoidattn$ has favorable properties compared to $\softmaxattn$. We explore this by repeating the analysis of \citet{DBLP:journals/corr/abs-2309-14322}, as described in the LayerScale analysis, to investigate the impact of QK norm \citep{DBLP:conf/icml/0001DMPHGSCGAJB23}. For language modeling, both $\sigmoidattn$ and $\softmaxattn$ exhibit sensitivity to learning rate changes without QK norm. However, incorporating QK norm significantly stabilizes performance (\cref{fig:qk_norm_ablation}). In vision tasks, $\sigmoidattn$ demonstrates robustness with and without QK norm (\cref{fig:imagenet_top_1_ablations}-a) and without the need for $n^{-\alpha}$ normalization from \citet{wortsman2023replacing}.\footnote{We ablate multiplicative sequence length scaling in more detail in \cref{sec:appendix_normalization}.}
\paragraph{Multi-query attention (MQA)} In \cref{fig:imagenet_top_1_ablations}-b we explore MQA \citep{DBLP:journals/corr/abs-1911-02150} for vision using only one head for $\{ \mK, \mV \}$. We find that both $\sigmoidattn$ and $\softmaxattn$ perform equally well with or without multiple heads even at the small scale of ViT-B/16.
\paragraph{Activation Function Ablations} As in \citet{wortsman2023replacing}, various activation functions, when combined with LayerScale and QK norm, perform equally well for vision tasks (\cref{fig:imagenet_top_1_ablations}-c). However, for sequence-critical tasks like ASR, activation functions such as ReLU pose instabilities and underperform. In the same figure, we also compare to the ReLU$^2$ proposal from \citet{DBLP:conf/icml/HuaDLL22} and find that it underperforms without LayerScale and QK norm.
\subsection{Supervised Image Classification}
\label{sec:supervised_image_classification}
Vision transformers \citep{DBLP:conf/iclr/DosovitskiyB0WZ21} extend transformers  \citep{DBLP:conf/nips/VaswaniSPUJGKP17} to treat $K \times K$ image grids as disparate tokens. All tokens are refined through sequential layers of self-attention, pooled using a CLS token or global average pooling layer, and optimized using the negative log likelihood, $\ln p(\vy|\vx)$. We train ViT-B/16 models using $\mathbb{R}^{224 \times 224 \times 3}$ images for 300 epochs using the recipe provided in \cref{sec:appendix_vision_hyperparams}. We use the same set of training hyper-parameters for both $\softmaxattn$ and $\sigmoidattn$, changing only the activation function between trials. The train negative log-likelihood is reported in \cref{fig:summary_nll} and the test top-1\% is reported in \cref{fig:test_top1_results}. We find that $\sigmoidattn$ matches both the training dynamics and the evaluation performance of $\softmaxattn$.
\subsection{Self-Supervised Image Representation Learning}
\label{sec:ssl}
Self-supervised representation learning (SSL) exploits vast quantities of unlabeled data to learn semantic representations based on inductive biases such as augmentation invariance (SimCLR \cite{DBLP:conf/icml/ChenK0H20}, BYOL \citep{DBLP:conf/nips/GrillSATRBDPGAP20}) or reconstruction from compressed representations (MAE \citep{DBLP:conf/cvpr/HeCXLDG22}). We employ vision transformer training recipes from \cite{DBLP:conf/icml/ZhaiLLBR0GS23} and \cite{DBLP:conf/nips/BusbridgeRALDCW23} (\cref{sec:appendix_vision_hyperparams}) for SimCLR and BYOL. As with supervised learning, we use the same set of training hyper-parameters for both $\softmaxattn$ and $\sigmoidattn$, changing only the activation function between trials. \Cref{fig:summary_nll} reports the train losses, and \cref{fig:test_top1_results} highlights the linear probe and finetuned test top-1\%. Despite the diverse training objectives in SSL, $\sigmoidattn$ matches $\softmaxattn$ while improving training and inference throughput (\cref{sec:FlashSigmoidHardwareAwareImplementation}).
\subsection{Automatic Speech Recognition (ASR)}
\label{sec:asr}
\begin{table}[t!]
\centering
\caption{Word error rate (\%) on LibriSpeech test sets and TED-LIUM v3~\citep{hernandez2018ted} (``TED'', joint validation and test sets split according to  duration) for transformer (255M params) with either $\softmaxattn$ or $\sigmoidattn$ (LayerScale and QK norm are used with $b=-\log n$) trained on LibriSpeech 960h data (mean duration is 10-15s). Hyper-parameters are in~\cref{sec:asr_hps}.}
\label{tab:asr-results}
\begin{center}
\begin{scriptsize}
\begin{sc}
\resizebox{\columnwidth}{!}{%
\begin{tabular}{lc|rr|rrrr}
\toprule
 attn & PE & test-clean & test-other & ted 0-10s & ted 10-20s & ted 20-30s & ted 30s+  \\
\midrule 
softmax & \multirow{7}{*}{CAPE} & 2.3 & 5.7 & 12.4 & 10.5 & 11.9 & 9.1 \\
 sigmoid &  & 2.4 & 5.5 & 12.4 & 10.3 & 12.3 & 9.7 \\
 \,\,\,\, - QK norm &  & \multicolumn{6}{c}{unstable, gradient norm and loss spikes} \\
 \,\,\,\, - LayerScale &  & 2.5 & 6.1 & 13.6 & 11.5 & 13.4 & 8.9 \\
 sigmoid ($b=-10$, learnable) &  & 2.3 & 5.5 & 12.1 & 10.5 & 13.0 & 9.3 \\
 sigmoid ($b=-5$ in $Q$, learnable) &  & 2.3 & 5.4 & 12.2 & 10.8 & 12.4 & 9.9 \\
 \,\,\,\, - QK norm &  & \multicolumn{6}{c}{unstable, gradient norm and loss spikes} \\

\midrule
softmax & \multirow{5}{*}{RoPE} & 2.2 & 5.5 & 12.7 & 10.6 & 12.8 & 9.5 \\
 sigmoid &  & 2.3 & 5.4 & 12.3 & 10.1 & 12.3 & 8.6 \\
 sigmoid ($b=-10$, learnable) &  & 2.2 & 5.2 & 12.4 & 10.5 & 12.3 & 21.8 \\
 \,\,\,\, + $\alpha=1$ &  & 2.7 & 6.6 & 14.1 & 12.0 & 14.5 & 14.9 \\
 sigmoid ($b=-5$ in $Q$, learnable) &  & \multicolumn{6}{c}{unstable, gradient norm and loss spikes} \\
\midrule
 softmax & \multirow{5}{*}{ALiBi} & 2.2 & 5.4 & 12.3 & 10.7 & 12.1 & 8.6 \\
 sigmoid &  & 2.3 & 5.1 & 12.3 & 10.5 & 12.6 & 9.1 \\
 sigmoid ($b=-10$, learnable) &  & 2.2 & 5.2 & 12.4 & 10.4 & 11.7 & 9.1 \\
 \,\, + $\alpha=1$ &  & 2.6 & 6.6 & 13.9 & 11.9 & 14.2 & 8.6 \\
 sigmoid ($b=-5$ in $Q$, learnable) &  & 2.2 & 5.2 & 12.1 & 10.4 & 12.0 & 8.2 \\
\bottomrule
\vspace{-0.4cm}
\end{tabular}
}
\end{sc}
\end{scriptsize}
\end{center}
\end{table}
We benchmark ASR using LibriSpeech data \citep{DBLP:conf/icassp/PanayotovCPK15} on 100h and 960h settings of paired speech and text transcriptions. Our PyTorch implementations of encoder-based vanilla transformer~\citep{synnaeve2019end} and conformer \citep{DBLP:conf/interspeech/GulatiQCPZYHWZW20} are trained with Connectionist Temporal Classification (CTC) \citep{DBLP:conf/icml/GravesFGS06} w/ BF16 mixed precision, w/o QK norm and w/o LayerScale. After extensively tuning $\softmaxattn$ baselines, we switch to $\sigmoidattn$ per \cref{eq:sigmoid_attn} without any other changes. We investigate the effects of post/pre-LayerNorm, model depth, optimizer type, small data regime, and connection to local attention, with details in~\cref{sec:asr_hps}.

Our main findings are: i) CAPE~\citep{DBLP:conf/nips/LikhomanenkoXSC21} PE is the most unstable for $\sigmoidattn$; ii) post-LayerNorm models with $\softmaxattn$ are hard to match with stable $\sigmoidattn$; iii) w/o QK norm $\sigmoidattn$ is unstable and significant spikes happen in both gradient norms and training loss; iv) LayerScale is needed for generalization; v) learnable bias $b=-10$ gives no loss and gradient norms spikes while matching the $\softmaxattn$ (which does not benefit from the improved throughput of \textsc{FlashSigmoid}); vi) adding a learnable bias, $b=-5$, to $Q$ instead of the attention logits also solves the initial large attention norms for CAPE and ALiBi but not for RoPE; vii) $b=-\log n$ gives rare (2-5 times) marginal gradient norms spikes with smooth loss while matching $\softmaxattn$.


\Cref{tab:asr-results} shows the main result for pre-LayerNorm  transformers with CAPE, RoPE, and ALiBi, where $\sigmoidattn$ uses LayerScale, QK norm, $b=-\log n$, and no sequence normalization. The bias is ablated with learnable bias (one per layer) in attention or $Q$ with or without sequence normalization. $\sigmoidattn$ is stabilized with bias while matching $\softmaxattn$, and $b=-\log n$ works well. In most cases, bias allows generalization to longer sequences without sequence normalization, except for RoPE where it helps for longer sequences but hurts overall performance.









\subsection{Autoregressive Large Language Modeling}
\label{sec:llm}

\newcolumntype{R}[2]{%
    >{\adjustbox{angle=#1,lap=\width-(#2)}\bgroup}%
    l%
    <{\egroup}%
}
\newcommand*\rotdiag{\multicolumn{1}{R{30}{1em}}}%

\begin{table}[t]
\centering
\caption{1B LLM English evaluation.}
\label{tab:lm_results}
\begin{sc}
\begin{scriptsize}
\bgroup
\setlength{\tabcolsep}{.35em}
\begin{tabular}{@{}lllllllllllllll@{}}
\toprule
Model   & \makecell{Seq.\\Len.} & \makecell{ARC\\Easy} & \makecell{ARC\\Challenge} & \makecell{Hella-\\swag} & Piqa & Sciq & \makecell{Wino-\\grande} & \makecell{Lambada\\OpenAI} & \makecell{TriviaQA\\(1-shot)} & \makecell{WebQS\\(1-shot)} & AVG & \makecell{Step\\time (s)} \\ \midrule
Softmax (ALiBi) & 2k & 62.2       &     26.8           &    42.4       &  59.0    &   72.3   &     88.1       &     58.4           &      19.9             &    15.4            &    49.4   & 0.38   \\
Sigmoid (ALiBi) & 2k &  62.8       &      28.8         &    42.5       &  59.7    &   70.3   &     88.6       &      59.7          &       19.1            &   13.8             &       49.5  & 0.34   \\
\midrule
Softmax (RoPE) & 4k & 63.3       &     29.3           &    43.3       &  58.1    &   71.3   &     86.9       &     58.8           &  20.4             &    15.6            &    49.7   & 0.84   \\
Softmax (ALiBi) & 4k & 62.6       &     27.7           &    42.4       &  58.6    &   71.1   &     88.2       &     58.6           &      18.9             &    14.7            &    49.2   & 0.84   \\
Sigmoid (ALiBi) & 4k &  60.5       &      27.3         &    41.3       &  57.8    &   70.5   &     87.0       &      57.6          &       18.9            &   12.6             &       48.2  & 0.67   \\ \bottomrule
\end{tabular}
\egroup
\end{scriptsize}
\end{sc}
\vspace{-0.4cm}
\end{table}

We initially iterated at the 85M scale, as it served as a proxy for larger scale training. Our findings show that: i) attention bias is required for stability, which can be learnable, but setting it to $-\log(n)$, where $n$ is the maximum training sequence length of 4096, works well and is faster; ii) RoPE is more challenging to stabilize; iii) the final setting exhibits smooth loss curves, but still shows gradient norm fluctuations. We then turn our attention to validating $\sigmoidattn$ at scale.

We train a 1B language model using the Llama2 \citep{touvron2023llama} recipe with ALiBi instead of RoPE positional embedding, and the RedPajama \citep{together2023redpajama} dataset (see \cref{sec:llm_appendix}). At sequence length 4096, $\sigmoidattn$ achieves a \textbf{1.23}$\mathbf{\times}$ step-time improvement over $\softmaxattn$ in JAX without \textsc{FlashAttention} (\cref{tab:lm_results}). All LLMs are trained using the AXLearn framework, which include the recipe and $\sigmoidattn$ implementation.\footnote{https://github.com/apple/axlearn}

$\softmaxattn$ and $\sigmoidattn$ have matching train and validation NLL at 85M (\cref{fig:85m_4k_nll}) and at 1B scale when using 2048 sequence length (\cref{fig:summary_nll}). However, a slight disparity is observed at 1B scale when using 4096 sequence length, which we leave for future investigation (more details in \cref{sec:llm_appendix}).




\section{Related Work}


% \paragraph{Multilingual LLMs} Early work on large language models has mostly focused on English~\cite{brown2020language}. However, there has been a surge in multilingual models and accompanying datasets to address the need to use these models in non-English languages.

\textbf{Expanding Multilingual Language Models.} Initially, the development of LLMs has predominantly targeted the English language~\citep{brown2020language}, leveraging the extensive corpus of English data available on the Web and the broad applicability of models trained on English text. However, this emphasis has often come at the cost of accommodating the linguistic diversity found across various language demographics~\citep{zhu2023extrapolating,bang2023multitask,zhang2024m3exam}. Recognizing this significant limitation~\citep{robinson2023chatgpt,peng-etal-2024-humaneval}, recent research has proposed foundational LLMs equipped with multilingual capabilities~\citep{chai2023ernie, scao2022bloom,wei2023polylm,shliazhko2022mgpt}, or has explicitly concentrated on addressing the challenges posed by low-resource languages~\citep{ustun2024aya,singh2024aya,gala2023indictrans2}. To integrate multilingual capabilities into existing LLMs, researchers have proposed a variety of methods to enhance multilingual adaptation. These approaches range from continual pretraining techniques~\citep{ibrahim2024simple,gupta2023continual} to initial training on extensive multilingual datasets~\citep{scao2022bloom,chai2023ernie} and then subsequent specialized fine-tuning on a target language~\citep{yang2023bigtranslate, han-etal-2022-x}, and even adaptation through instruction tuning~\citep{shaham2024multilingual,kew2023turning,gala2024airavata}. Critical aspects in multilingual adaptation remain on the availability of high-quality diverse multilingual corpus~\citep{correa2024teenytinyllama} and further the scope of vocabulary of the specific language.

% \SVX{need to clearly state the difference from BLOOM and ERNIE-Code, LLMs on both multilingual NLs and multilingual code}

\textbf{Continual Pretraining.}  Static datasets are impractical for adapting to evolving real-world data, making continual learning essential~\citep{ring1998child,thrun1998lifelong}. Continual pretraining~\citep{gururangan2020dont} allows models to incorporate new knowledge without retraining from scratch, a costly endeavor. As curated datasets like RedPajama~\citep{together2023redpajama} and Dolma~\citep{soldaini2024dolma} become available, integrating them efficiently is crucial. This also enables the extension of models to new modalities, such as code (e.g., StableCode). Previous approaches focus on replay techniques, optimizing learning schedules~\citep{ibrahim2024simple}, soft masking~\citep{ke2023adapting}, and forward/backward transfer~\citep{yildiz2024investigating}.

% The notion of large static datasets becomes impractical when confronted with the dynamic nature of real-time events, evolving facts, and the introduction of new data or concepts within a domain. In such scenarios, continual learning~\citep{ring1998child,thrun1998lifelong,Kirkpatrick2018overcoming,zenke2017continual,rebuffi2017icarl,lopez2017gradient} becomes imperative for large pretrained models to swiftly adapt to these shifting environments. The drive for continual pretraining~\citep{gururangan2020dont} extends beyond the dynamic nature of real-world data; it is also fueled by the prohibitive expense associated with training current foundation models from scratch. As more curated datasets become accessible (e.g. RedPajama~\citep{together2023redpajama}, Dolma~\citep{soldaini2024dolma}, CommonCorpus\footnote{\url{https://huggingface.co/blog/Pclanglais/common-corpus}}, ToolQA~\citep{zhuang2023toolqa}) the idea of incorporating knowledge from these datasets through retraining on the union of all available data sets becomes inherently unfeasible. This also encompasses the integration of new capabilities into foundational models trained on specific data distributions. For example, it involves expanding natural language-based models to include the structured modality of code (e.g. StableCode\footnote{\url{https://stability.ai/news/stable-code-2024-llm-code-completion-release}}). Previous studies on continual pretraining have primarily concentrated on replay, optimizing the learning rate schedule~\citep{ibrahim2024simple}, preserving general knowledge through soft masking attention heads~\citep{ke2023adapting}, exploring the effects of domain similarity and model capacity on forward and backward transfer~\citep{yildiz2024investigating}, and continual post-training for few-shot adaptation~\citep{ke2022continual}.

% \citep{scao2022bloom}
% \citep{chai2023ernie}
% \citep{ustun2024aya}
% \citep{lin2021few}
% \citep{muennighoff2022crosslingual}
% \citep{shliazhko2022mgpt}
% \citep{scao2022language}

% \citep{singh2024aya}
% \citep{kudugunta2024madlad}
% \citep{longpre2023data}
% \citep{laurenccon2022bigscience}
% \citep{albalak2024survey}
% \citep{nguyen2023culturax}

% \textbf{LLM Compliance}
% The extensive utilization of Large Language Models (LLMs) across various applications underscores the necessity for their operation to uphold user privacy, mitigate risks such as misinformation or biased outputs, and ensure transparency regarding their functionality and utilization, all while adhering to local regulations. Independent evaluations and red teaming play crucial roles in assessing these risks. However, conflicts of interest within major AI companies may impede such safety evaluations, underscoring the necessity for a safe harbor for safety research~\citep{longpre2024safe}. Benchmark datasets facilitating such research efforts include those proposed by \cite{zhang2023safetybench} and \cite{sun2023safety}. Studies aimed at enhancing the security, safety, and legal compliance of Large Language Models (LLMs) have encompassed various approaches. These include the creation of datasets based on FAIR data principles~\citep{raza2024fair}, the development of structured LLM auditing mechanisms~\citep{M_kander_2023}, risk assessment of LLM alignment through personalized feedback~\citep{kirk2023personalisation}, structured evaluation of risks associated with LLM deployment~\citep{derczynski2023assessing}, and profiling foundation model transparency~\citep{bommasani2023foundation,bommasani2024foundation}.


\section{Conclusion}
\section{Conclusion}
\label{sec:conclusion}
\vspace{-2mm}

We introduced \implname, a novel blueprint toward realizing self-adaptive LLMs. 
We first proposed \svdacro, offering superior performance, reduced costs, high compositionality, and overfitting regularization. 
Building on \svdacro experts, we developed three effective self-adaptation strategies, each providing unique benefits and scalable performance improvements. 
Future work could explore cross-model compatibility and expert skill recycling across model generations, drawing from recent model merging techniques~\citep{yu2024language,goddard2024arcee,akiba2024evolutionary} to overcome individual LLM limitations.

% \newpage

\section*{Ethical Consideration}
We believe that transparency and accessibility are fundamental principles in the development and deployment of artificial intelligence technologies. Closed-source LLMs limit public scrutiny, hinder collaboration, and potentially reinforce biases inherent in their development process. 
In contrast, our commitment to open source models fosters a culture of accountability, collaboration, and inclusivity. By making \system\ accessible to all, we promote innovation, empower diverse voices, and strive for equitable outcomes in AI applications. We firmly believe that openness in AI development is essential for creating solutions that truly serve the needs and values of society. To this end, we prioritized safety guardrails in alignment with the Biden-Harris Executive Order on AI. Furthermore, the multilingual capability of \system\ enhances its usability for users across the world.

On the other hand, each promise comes with peril, and improved technological access through \system\ might also increase the potential number of malicious actors. We overall believe that the general benefit far outweighs the potential misuse and want to emphasize the importance of a considered and ethical use of this technology and thus also of \system.

Lastly, we recognize that safety and lawfulness can be contextual to different cultures and laws. We recognize that in our work we focused on a U.S. centric standard, and we believe future work should also explore multi-jurisdictional redteaming.


\section*{Acknowledgments}
This work was supported by the ``R\&D Hub Aimed at Ensuring Transparency and Reliability of Generative AI Models'' project of the Ministry of Education, Culture, Sports, Science and Technology, and used resources of LUMI supercomputer under project\_462000316.

% This document has been adapted by Emily Allaway from the instructions for earlier ACL and NAACL proceedings, including those for NAACL 2024 by Steven Bethard, Ryan Cotterell and Rui Yan,
% ACL 2019 by Douwe Kiela and Ivan Vuli\'{c},
% NAACL 2019 by Stephanie Lukin and Alla Roskovskaya,
% ACL 2018 by Shay Cohen, Kevin Gimpel, and Wei Lu,
% NAACL 2018 by Margaret Mitchell and Stephanie Lukin,
% Bib\TeX{} suggestions for (NA)ACL 2017/2018 from Jason Eisner,
% ACL 2017 by Dan Gildea and Min-Yen Kan,
% NAACL 2017 by Margaret Mitchell,
% ACL 2012 by Maggie Li and Michael White,
% ACL 2010 by Jing-Shin Chang and Philipp Koehn,
% ACL 2008 by Johanna D. Moore, Simone Teufel, James Allan, and Sadaoki Furui,
% ACL 2005 by Hwee Tou Ng and Kemal Oflazer,
% ACL 2002 by Eugene Charniak and Dekang Lin,
% and earlier ACL and EACL formats written by several people, including
% John Chen, Henry S. Thompson and Donald Walker.
% Additional elements were taken from the formatting instructions of the \emph{International Joint Conference on Artificial Intelligence} and the \emph{Conference on Computer Vision and Pattern Recognition}.

% % Bibliography entries for the entire Anthology, followed by custom entries
% \bibliography{anthology,custom}
% % Custom bibliography entries only
\bibliography{custom}

\appendix

\section{Appendix} \label{appendix}


\subsection{NewYorker Data for evaluation}

\begin{figure}[!ht]
\small
\centering
\includegraphics[width=0.4\textwidth]{figures/length.png}
\caption{\label{lengthdist} Distribution of word count of stories in our test set}
\end{figure}

Table \ref{teststories} shows the data used for conducting our evaluation. The 12 stories shown are taken from The New Yorker and summarized into single-sentence plots. These stories come from highly established literary experts acting as an upper bound for what it means to be creative. These stories span complex themes.

\begin{table*}[!ht]
\centering
\small
\def\arraystretch{1.35}
\begin{tabular}{|l|}
\hline
\begin{tabular}[c]{@{}l@{}}Write a New Yorker-style story given the plot below. Make sure it is atleast \textbf{\color{blue}\{\{word\_count\}\}} words. Directly start with the\\ story, do not say things like `Here's the story {[}...{]}:\end{tabular}                                                                                                                                                                                            \\ \hline\hline
\begin{tabular}[c]{@{}l@{}}You wrote the story I gave you below. I requested a story with \textbf{\color{blue}\{\{word\_count\}\}} words, but the story only has\\ \textbf{\color{blue}\{\{current\_word\_count\}\}} words. Can you rewrite the story to make it longer, and closer to the \textbf{\color{blue}\{\{word\_count\}\}} word target\\ I gave you. Directly start with the story, do not say things like `Here's the story {[}...{]}:`\\ \\ Current story: \{\{story\}\}\end{tabular} \\ \hline
\end{tabular}
\vspace{2ex}
\caption{\label{promptstory}Prompt to write the initial story (Row1) vs Prompt to rewrite the initial story to be longer. word\_count represents the number of words in the human written story on a given plot (P) while current\_word\_count represents the number of words in the LLM generated story on the same plot (P)}
\end{table*}

\begin{table*}[!ht]
\def\arraystretch{1.15}
\small
\begin{tabular}{|l|l|}
\hline
Story                                    & Plot                                                                                                                                                                                                                                                                                                                                                                                                                                                                                                                                   \\ \hline
\href{https://www.newyorker.com/books/flash-fiction/a-triangle}{A Triangle}                               & \begin{tabular}[c]{@{}l@{}}An observer becomes entranced by a seemingly ordinary couple on the street, follows them home, and then \\watches them from outside in the rising floodwaters, drawing an eerie connection between the woman and\\ a discarded, burned chair they'd noticed earlier.\end{tabular}                                                                                                                                                                    \\ \hline\hline
\href{https://www.newyorker.com/books/flash-fiction/barbara-detroit-1966}{\begin{tabular}[c]{@{}l@{}}Barbara\\ Detroit,1966\end{tabular}}                    & \begin{tabular}[c]{@{}l@{}}On Feb 12, 1966, a heavily pregnant woman named Barbara experienced a shocking incident in her synagogue\\in Southfield, Detroit, where a young man shot and killed the renowned Rabbi Adler before turning the gun\\ on himself, and though Barbara tried to reach the shooter, she was swept away by the fleeing crowd.\end{tabular}                                                                              \\ \hline\hline
\href{https://www.newyorker.com/books/flash-fiction/beyond-nature}{Beyond Nature}                           & \begin{tabular}[c]{@{}l@{}}A solitary man walking in a remote mountainous region comes across a car crash, and stays by the side\\ of the lifeless female victim, narrating stories of his past and reflecting on the impermanence of \\events and life itself, while awaiting emergency services amidst the looming presence of wilderness.\end{tabular}                                                                                                                \\ \hline\hline
\href{https://www.newyorker.com/books/flash-fiction/certain-european-movies}{\begin{tabular}[c]{@{}l@{}}Certain European\\ Movies\end{tabular}}                  & \begin{tabular}[c]{@{}l@{}}Two individuals, at a residency together, navigate the complexity of their ephemeral relationship during\\ their final beach trip, framed by misadventures, subtle tensions, unspoken desires, and looming departures.\end{tabular}                                                                                                                                                                                   \\ \hline\hline
\href{https://www.newyorker.com/books/flash-fiction/keys}{Keys}                                     & \begin{tabular}[c]{@{}l@{}}Daniel, struggling with recurring dreams of his ex-wife Rachel and a mysterious unused flat, eventually \\discusses them with his current partner Isabel, sparking various reflections and conversations about their\\ past relationships, until a real-life discovery of old keys triggers a nostalgic memory and helps him find a\\ way to reconnect with his present relationship through canoeing.\end{tabular}                                     \\ \hline\hline
\href{https://www.newyorker.com/books/flash-fiction/listening-for-the-click}{\begin{tabular}[c]{@{}l@{}}Listening For\\ the Click\end{tabular}}                  & \begin{tabular}[c]{@{}l@{}}Navigating a complex social landscape, the protagonist experiences a series of complex relationships \\and emotional turmoil in a student environment, and engages in self-discovery and self-reflection as she\\ interacts with the characters Carl, Martin, Lizzy, and Johan, resulting in a journey of introspection,\\ betrayal, love, and personal growth.\end{tabular}                                                          \\ \hline\hline
\href{https://www.newyorker.com/magazine/2023/05/15/maintenance-hvidovre-fiction-olga-ravn}{\begin{tabular}[c]{@{}l@{}}Maintenance,\\ Hvidovre\end{tabular}}                   & \begin{tabular}[c]{@{}l@{}}A woman experiences a disorienting night in a maternity ward where she encounters other similarly \\disoriented new mothers, leading to an uncanny mix-up where she leaves the hospital with a baby \\that she realizes is not her own, yet accepts the situation with an inexplicable sense of happiness.\end{tabular}                                                                                                  \\ \hline\hline
\href{https://www.newyorker.com/magazine/2022/11/14/returns}{Returns}                                  & \begin{tabular}[c]{@{}l@{}}The narrator visits their elderly mother in her small town, spending a day with her that is filled with \\nostalgia, conversation, and old habits, only to return a month later after her hospitalization due to\\ a sunstroke, finding remnants of their last visit.\end{tabular}                                                                                                                                                                      \\ \hline\hline
\href{https://www.newyorker.com/books/flash-fiction/the-facade-renovation-thats-going-well}{\begin{tabular}[c]{@{}l@{}}The Facade \\Renovation\\ That’s Going Well\end{tabular}} & \begin{tabular}[c]{@{}l@{}}An academic faculty housed in a building with a critical waterproofing layer missing experiences a series\\ of disruptive and problematic construction repairs, causing tension, inconvenience, and health concerns\\ among the tenants, ultimately leading to resignation and endurance in hopes of better future circumstances.\end{tabular}                                                        \\ \hline\hline
\href{https://www.newyorker.com/books/flash-fiction/the-kingdom-that-failed}{\begin{tabular}[c]{@{}l@{}}The Kingdom\\ That Failed\end{tabular}}                  & \begin{tabular}[c]{@{}l@{}}The narrator recounts their college friendship with the seemingly flawless Q, and after a decade apart, \\they accidentally cross paths at a pool, where the narrator anonymously observes Q's failed attempt to \\let down a woman about a work-related issue, demonstrating that Q, too, has his share of difficulties.\end{tabular}                                                                                                \\ \hline\hline
\href{https://www.newyorker.com/magazine/2022/06/13/trash }{Trash}                                    & \begin{tabular}[c]{@{}l@{}}A woman unexpectedly marries the son of a successful, ambitious woman named Miss Emily, finding both \\acceptance and critique from her mother-in-law as she navigates this new relationship and confronts the \\stark contrasts between her former life as a supermarket cashier and her new life as part of a well-off family.\end{tabular}                                                                                                            \\ \hline\hline
\href{https://www.newyorker.com/culture/personal-history/the-last-dance-with-my-dad}{\begin{tabular}[c]{@{}l@{}}The Last Dance\\ with my Dad \end{tabular}}               & \begin{tabular}[c]{@{}l@{}}A young teenager recounts her experiences of fitting into her father's gay lifestyle, highlighted by a\\ seven-day cruise with hundreds of gay men, where she experienced acceptance and connection, had her\\ first genuine interaction with a  boy, and shared a last dance with her terminally ill father.\end{tabular}                                                                                                       \\ \hline
\end{tabular}
\vspace{2ex}
\caption{\label{teststories} Expert-written short stories from the New Yorker along with their human-verified GPT4 generated summary as plots that are included as part of our test data for Creativity Evaluation}
\end{table*}


\subsection{Expert Perception on the TTCW tests}

\begin{figure*}[!ht]
    \centering
     \includegraphics[width=\textwidth]{figures/rel.pdf}
    \caption{\label{relev} Relative Evaluation by Creative Writing Experts within a given group of four stories}
\end{figure*}

\begin{table*}[!ht]
\small
\centering
\begin{tabular}{|l|l|}
\hline
E5 & \begin{tabular}[c]{@{}l@{}}It was a pretty effective rubric! I'm used to being more subjective in my work -- did you like a story? Did it connect with \\you? Did it make sense? Why or why not? It was often challenging to break it down into more regimented segments \\like the rubric asked for -- but I do think that it allowed me to express the subjective feelings in a pretty thorough and\\ structured way!\end{tabular}                                                                                                                                                                 \\ \hline
E3 & \begin{tabular}[c]{@{}l@{}}As for the rubric, I thought it was quite thorough. There were some categories where I would say the story didn’t ``pass,"\\ but which were excellent. This happened often with the categories about multiple points of view, and innovative\\ structure and form. Overall, I think the rubric was helpful in helping me think about the different aspects of storytelling.\end{tabular}                                                                                                                                                                                 \\ \hline
E4 & \begin{tabular}[c]{@{}l@{}}I thought the rubric felt pretty thorough; the only part I felt could be added was that suggestion about consistency in\\ voice \& diction!\end{tabular}                                                                                                                                                         \\ \hline
E2 & \begin{tabular}[c]{@{}l@{}}The rubric seemed great to me! It’s however hard to talk about something like pacing without talking about scene and \\summary, for instance. Or the difference between originality of thought and originality in theme/content—wouldn’t the \\latter make up the former and vice/versa? But it is also comprehensive and I can see the merits of this sort of repetition in\\ teasing out a fuller picture of things\end{tabular} \\ \hline
E1 & \begin{tabular}[c]{@{}l@{}}I thought the rubric was pretty good tbh. I think there is overlap in some of the different elements, like "language \\proficiency \& literary devices" and "originality in thought." it's tricky to use words like "satisfying" and "sophisticated" \\when assessing art, but there's always going to be a great deal of subjectivity in these matters.I think that voice is a crucial \\aspect of high-quality writing that is being overlooked by the rubric, and one that greatly informs how I as a reader\\ evaluate 
and appreciate literary writing.\end{tabular} \\ \hline
\end{tabular}
\vspace{2ex}
\caption{\label{expertfeedbackrubric}Expert perception and feedback on the TTCW tests they conducted as part of our data collection.}
\end{table*}

Since the experts listed in Table \ref{creativeexperts} were not involved in designing the rubric but evaluated several stories based on the rubric we asked them their \textit{overall thought about the rubric and any potentially crucial test we missed out on that they use to discriminate between good and bad writing}.As can be seen in Table \ref{expertfeedbackrubric} in Appendix overall almost every expert agreed on the thorough and effective nature of our rubric. Many of them agreed on the fact that our rubric helped them to think about different aspects of storytelling in a more structured way. One of the difficult things about coming up with a rubric for creativity is ensuring coverage. Even though our rubric covers most aspects of creative writing, some experts such as E1 and E4 emphasized on the utility of \textbf{Consistency of Voice and Diction} as a measurable test. In E4's words \textit{``Inconsistent voice and diction are sometimes/often notable in stories that aren't very good, and when voice \& diction are used beautifully, it enhances a story considerably"}. E1 similarly exclaimed \textit{``One of the most meaningful aspects of high-quality literary writing is voice, which conveys qualities of proficiency, artistry, personality, and identity."}. We hope future work can adapt this as a meaningful test in addition to the tests covered in our rubric. Finally, some of the tests from our rubric can have potential overlaps as pointed out by E2. This is further corroborated by the similar numbers for \textit{Narrative Pacing} and \textit{Scenes vs Exposition} suggesting a strong correlation between the two.
\begin{table*}[!ht]
\small
\centering
% \def\arraystretch{1.3}
\begin{tabular}{|l|l|l|}
\hline
Test & Passing Stories & Failing Stories \\ \hline
\begin{tabular}[c]{@{}l@{}}Originality in\\ Form\end{tabular} & \begin{tabular}[c]{@{}l@{}}Inventive techniques like time jumping, varied \\ perspectives, unconventional punctuation, and\\ delayed revelation of key information\end{tabular} & \begin{tabular}[c]{@{}l@{}}Conventional and linear in its form, language, \\ and narrative, with occasional attempts at \\ innovation that do not significantly contribute to \\ its overall originality or creativity\end{tabular} \\ \hline
\begin{tabular}[c]{@{}l@{}}Originality in\\ Thought\end{tabular} & \begin{tabular}[c]{@{}l@{}}Fresh language, unique plot and characters, subtle\\ emotional resonance, and inventive metaphors. Minor \\ familiar elements, but do not undermine the overall \\ sense of imagination and distinctiveness\end{tabular} & \begin{tabular}[c]{@{}l@{}}Stories relies heavily on cliches \& tired tropes.\\ Language does not feel fresh or original with \\ narrative arc following a predictable trajectory.\\ Metaphors, descriptions, and overall premise \\ cover familiar ground that lacks novelty or nuance\end{tabular} \\ \hline
\begin{tabular}[c]{@{}l@{}}Originality in\\ Theme/Content\end{tabular} & \begin{tabular}[c]{@{}l@{}}Unconventional, dreamlike exploration of emotions\\ such as love and loss, evoking empathy and reflection\\ through its distinct main character perspective, \\ eschewing simplistic meanings for ambiguity, and \\ valuing open-ended questions over singular messages,\\ thus providing a unique reading experience compared\\ to conventional stories.\end{tabular} & \begin{tabular}[c]{@{}l@{}}Disjointed narrative, underdeveloped themes, \\ inconsistent tone, vaguely defined characters, and\\ abrupt context shifts, lack depth and fail to provide \\ substantive insight or originality to the reader.\end{tabular} \\ \hline\hline
\begin{tabular}[c]{@{}l@{}}World Building\\ and Setting\end{tabular} & \begin{tabular}[c]{@{}l@{}}Strategic use of concrete, specific sensory details from\\ a particular character’s perspective balances narrative\\ momentum, making a fictional world feel real, vivid\\ and immersive for readers. Thoughtful depiction of\\ everyday objects, and idiosyncratic elements within\\ narrative and dialogue to balance exposition with \\ vivid scene-setting, creating authenticity and realism \\ that serves the plot and characters\end{tabular} & \begin{tabular}[c]{@{}l@{}}Fictional world is not always convincingly \\established through sensory details and language. \\Stories rely too heavily on overwrought imagery\\ and figurative language without grounding \\the reader in a tangible reality.\end{tabular} \\ \hline
\begin{tabular}[c]{@{}l@{}}Character\\ Development\end{tabular} & \begin{tabular}[c]{@{}l@{}}Fully realized characters with contradictions, \\ motivations, and backstories that make them\\ feel lifelike. Flatter, less developed characters\\ that feel appropriate for the narrative goals \\ and style is not necessarily a weakness\end{tabular} & \begin{tabular}[c]{@{}l@{}}Characters not well rounded. easily resorting to \\stereotypes. Predictable arcs not making them\\memorable. Actions or motivations unclear leading \\to disconnect\end{tabular} \\ \hline
\begin{tabular}[c]{@{}l@{}}Rhetorical\\ Complexity\end{tabular} & \begin{tabular}[c]{@{}l@{}}Rich subtext that emerges through contrasts between\\ characters and settings. Omissions that let readers \\ fill in meaning, metaphors with layered significance, \\ implicit characterization, ambiguous endings, \\ intertextual references, and rich language that goes\\ beyond straightforward storytelling\end{tabular} & \begin{tabular}[c]{@{}l@{}}Straightforward storytelling, lack of complexity,\\ nuance, and ambiguity.The absence of context and \\ ironic tensions, overtelling instead of showing, \\ Use of expository dialogue, blunt symbolism, \\ predictable characters, and definitive endings, all \\of which limit interpretation and undermine\\ their potential resonance\end{tabular} \\ \hline
\end{tabular}
\vspace{2ex}
\caption{\label{expertexpl2}Common themes and issues found in expert explanations for tests focusing on TTCW-Originality and TTCW-Elaboration}
\vspace{-8ex}
\end{table*}

\subsection{Common themes in TTCW of Originality and Elaboration}


Table~\ref{expertexpl2} provides common themes and issues found in the expert explanations for each TTCW in the Originality and Elaboration dimensions.


\subsection{Example LLM-generated and expert-written explanations for a TTCW assessment} 

\begin{table*}[!ht]
\centering
\small
\def\arraystretch{1.35}
\begin{tabular}{|l|l|}
\hline
GPT4 & \begin{tabular}[c]{@{}l@{}}The story incorporates numerous instances of subtext that serve to further develop the characters' internal emotions \\ and relationships, as well as accentuate the atmospheric tension and undertones of longing. The subtext, which is\\ subtly weaved into surface-level interactions, actions, and observations, is eloquently crafted and does not feel \\ forced, enriching both the narrative depth and setting.\\ One key passage with evident subtext is when he is in the ocean, swimming and jumping off the rock, while she\\ observes from the towel. The surface-level interpretation suggests a casual, relaxed beach day. However, the \\ subtext is rife with allusions of emotional undercurrents in their interaction. His unawareness of her going \\ into the water parallels the emotional disconnect between them, insinuating the vast, metaphorical space that \\ separates their respective experiences and understanding of their relationship.\\ Another instance of subtext occurs when he takes a photo of her with her swimsuit top off, indicating his attraction\\ to her. But the meaning deepens when we understand that he will delete the photo before going home, implicitly\\ revealing his guilt or fear of being discovered, and again highlighting the transient, secret nature of their relationship.\\ The line, "As in certain European movies" suggests the fleeting, impermanent element of their relationship and the\\ eventual abruptness and intensity of feelings that often feature in said films. The 'European movies' serve as a \\ metaphor for the unexpected and sudden end of their encounter. \\ In these examples, the subtext does not feel forced but instead flows naturally from the characters and \\ circumstances, subtly conveying deeper meanings that heighten both the narrative tension and emotional depth.\\ \\ So Yes.\end{tabular} \\ \hline
E3   & \begin{tabular}[c]{@{}l@{}}There is rich subtext, as the main character seems continually conflicted about whether she wants to be where she is, \\ doing what she is doing. On the surface, she is carefree, riding to the beach with the guy she met, skipping the ceramics\\ and the museum, and whatever else. And yet, she is unhappy and unsatisfied, longing for a beer, imagining that if their\\ relationship continued they would only hate each other. This tension is maintained throughout the story.\end{tabular}                                                                                                                                                                                                                                                                                                               \\ \hline
E1   & \begin{tabular}[c]{@{}l@{}}This piece has an iceberg of subtext floating underneath it. The entire story is conveyed through the successful \\ integration of subtext and text. The interactions between the protagonist and the man (Did you see me jump of the \\ rock? No, she hadn't.Did he notice she had gone in the water too, that her hair was dripping? No, he hadn't.)convey\\ a profound disconnect that causes the reader to wonder why the protagonist continues to suffer the presence of this\\ man she clearly disdains and seems to view as an incompetent man-child.\end{tabular}
               \\ \hline
E7   & \begin{tabular}[c]{@{}l@{}}Yes!!!!! Again, the idea of the story was fairly simple (the inevitability of age, parting, change), but it was illustrated\\ in a way that felt inspiring re: questioning how these ideas relate and resonate throughout our own lives. It was really \\ beautiful and I was left feeling changed at the end of it :)\end{tabular}                                                                                                                                                                                                                                                            \\ \hline
\end{tabular}
\vspace{2ex}
\caption{\label{llmvsexpertexpl}LLM explanation vs expert explanation for Rhetorical Complexity}
\end{table*}

In Table~\ref{llmvsexpertexpl}, we show examples of explanations that experts wrote in conjunction with a binary TTCW assessment they made on a story, as well as the corresponding LLM-generated explanations.

\subsection{Can non-experts administer TTCW tests?}

Recruiting experts for data annotation purposes is challenging, and costly, and must consider the time constraint put on the experts. Prior work has shown the potential of crowd-sourcing (through platforms such as Amazon Mechanical Turk) and the ability of non-experts to accomplish complex tasks as a crowd \cite{kittur2013future}, when following an appropriate workflow that iterates and validates the work on individual non-experts. Some prior work has even shown the validity of crowd-based feedback for writing tasks \cite{bernstein2010soylent,nebeling2016wearwrite}. 

In this work, we chose to rely on experts for annotation, to maximize the validity of our experiments, and confirm whether experts with domain knowledge would reach satisfying agreement levels when evaluating stories with TTCW. Future work can leverage our open-sourced annotations to explore whether non-experts correlate with experts when performing TTCW evaluation, which could lead to more cost-effective TTCW evaluation.

\subsection{Prompts for TTCW} \label{allprompts}

All the instructions shown to creative writing experts and LLMs are given in the tables below.


\begin{table*}[!ht]
\centering
\small
\begin{tabular}{|l|l|}
\hline
\begin{tabular}[c]{@{}l@{}}Expert \\ Measure\end{tabular}               & Does the manipulation of time in terms of compression or stretching feel appropriate and balanced?                                                                                                                                    \\ \hline
\begin{tabular}[c]{@{}l@{}}Expanded\\ Expert\\ Measure (M)\end{tabular} & \begin{tabular}[c]{@{}l@{}}`Compression/stretching of time' in fiction writing, also known as pacing, refers to the manipulation of time in \\storytelling for dramatic effect, pacing, or other narrative purposes. Essentially, it's about controlling the perceived \\speed and rhythm at which a story unfolds.\\ \\

Compression of time refers to when events that take a long time (hours, days, weeks, or even years) are summarized \\or condensed into a brief narrative span. For example, a writer might compress several years of a character's life \\into a few paragraphs to quickly convey important changes or developments.\\ \\

On the other hand, stretching of time is when a brief moment or event is drawn out over pages or chapters. It's often \\used to create suspense, emphasize details, or delve deeper into a character's thoughts and feelings. For example, \\the few seconds it takes for a dropped glass to hit the floor might be stretched out with detailed descriptions of the\\ action, reactions, and thoughts of characters involved.\\ \\

Storytime refers to the time within the world of the story, while real-world time refers to the time it takes for the \\reader to read the story. A skilled writer can manipulate the relationship between these two to affect the pacing of \\the narrative, either speeding it up (compression) or slowing it down (stretching). This technique plays a crucial role \\in shaping the reader's experience and engagement with the story.\end{tabular} \\ \hline
\begin{tabular}[c]{@{}l@{}}Human\\ Instruction\end{tabular}             & \begin{tabular}[c]{@{}l@{}}\{\{M\}\}\\ \\ Based on the story that you just read, answer the following question.\\ \textit{\color{blue}Does the manipulation of time in terms of compression or stretching feel appropriate and balanced?}\\ -Yes \\ -No \\\\ Reasoning : \end{tabular}                                                                       \\ \hline
\begin{tabular}[c]{@{}l@{}}LLM\\ Instruction\end{tabular}               & \begin{tabular}[c]{@{}l@{}}\{\{M\}\}\\ \\ Given the story above, list out the scenes in the story in which time compression or time stretching is used, and \\argue for each whether it is successfully implemented.  Then overall, give your reasoning about the question below \\and give an answer to it between 'Yes' or 'No' only \\ \\ \textit{\color{blue} Q) Does the manipulation of time in terms of compression or stretching feel appropriate and balanced?}\end{tabular}                                                                                                                                                                                                                    \\ \hline
\end{tabular}
\vspace{2ex}
\caption{\label{prompting}TTCW Fluency1 (Narrative Pacing) }
\vspace{-5ex}
\end{table*}


% ==================================================





\begin{table*}[!ht]
\centering
\small
% \def\arraystretch{1.15}
\begin{tabular}{|l|l|}
\hline
\begin{tabular}[c]{@{}l@{}}Expert \\ Measure\end{tabular}               & \begin{tabular}[c]{@{}l@{}}Does the story have an appropriate balance between scene and summary/exposition or it relies on one\\ of the elements heavily compared to the other?  \end{tabular}                                                                                                                                  \\ \hline
\begin{tabular}[c]{@{}l@{}}Expanded\\ Expert\\ Measure (M)\end{tabular} & \begin{tabular}[c]{@{}l@{}}'Scene' and 'summary/exposition' are two crucial elements of narrative storytelling, and balancing them \\appropriately is an important skill in fiction writing.\\ \\ 

A 'scene' is a moment in the story that is dramatized in real-time. Scenes are usually vivid and engaging, often \\featuring character interaction, dialogue, and action. They are the building blocks of the plot, and through them, \\the story unfolds.\\ \\ 

'Summary' or 'exposition', on the other hand, involves summarizing events or providing information. Instead of \\unfolding in real time, \\summaries compress time and tell the reader what happened. Exposition provides \\necessary background information, like character history, setting details, or prior events. \\ \\ 

A good writer knows when to use scenes to make the story come alive, show character development, or increase \\tension. They also know when to use summary or exposition to move the story forward, fill in background \\information, or bridge gaps between important scenes. \\ \\ 

The right balance between scene and summary/exposition can vary depending on the story, but in general, it's \\essential for maintaining a good pace, keeping the reader engaged, and delivering necessary information. \\A story with too many scenes and not enough summary might feel overwhelming or slow, while a story with \\too much exposition and not enough scenes could feel dry and unengaging.\end{tabular} \\ \hline
\begin{tabular}[c]{@{}l@{}}Human\\ Instruction\end{tabular}             & \begin{tabular}[c]{@{}l@{}}\{\{M\}\}\\ \\ Based on the story that you just read, answer the following question.\\ \textit{\color{blue} Does the story have an appropriate balance between scene and summary/exposition or it relies on one of the elements} \\\textit{\color{blue}heavily compared to the other?} \\ -Yes \\ -No \\\\ Reasoning : \end{tabular}    
\\ \hline
\begin{tabular}[c]{@{}l@{}}LLM\\ Instruction\end{tabular}               & \begin{tabular}[c]{@{}l@{}}\{\{M\}\}\\ \\ Given the story above, answer the following question. Please first explain your reasoning step by step \\and then given an answer between 'Yes' or 'No' only \\ \\ \textit{\color{blue} Does the story have an appropriate balance between scene and summary/exposition or it relies on one of the elements} \\\textit{\color{blue}heavily compared to the other?}\end{tabular}                                                                                                                                                                                                                    \\ \hline
\end{tabular}
\vspace{2ex}
\caption{\label{prompting}TTCW Fluency2 (Scene vs Exposition) }
\vspace{-5ex}
\end{table*}


% ==================================================


\begin{table*}[!ht]
\centering
\small
% \def\arraystretch{1.15}
\begin{tabular}{|l|l|}
\hline
\begin{tabular}[c]{@{}l@{}}Expert \\ Measure\end{tabular}               & Does the story make sophisticated use of idiom or metaphor or literary allusion?                                                                                                                                     \\ \hline
\begin{tabular}[c]{@{}l@{}}Expanded\\ Expert\\ Measure (M)\end{tabular} & \begin{tabular}[c]{@{}l@{}}`Idiom' refers to phrases or expressions that have a figurative, or sometimes literal, meaning that is \\comprehensible to a particular group of people. These can be cultural, regional, or specific to a certain group or \\profession.Sophisticated use of idiom suggests that the writer is skillfully using these expressions to lend \\authenticity to character voices or to convey specific meanings in a concise way.\\\\

`Metaphor' is a figure of speech that describes an object or action in a way that isn't literally true, but helps explain\\ an idea or make a comparison. Sophisticated use of metaphor suggests the
writer could create impactful, original \\comparisons that reveal deeper insights about themes,
characters, or situations in the story.\\\\

`Literary allusion' refers to a brief and indirect reference to a person, place, thing or idea of
historical, cultural,\\ literary, or political significance. It does not describe in detail the person or thing to which it refers. A sophisticated\\ use of literary allusion implies the writer can effectively incorporate these references to enhance the depth\\ and resonance of the story. They can provide contextual richness, evoke a specific tone, or draw parallels between\\ the narrative and the work alluded to.\\\\

Overall, when a writer uses these techniques well, they add depth, interest, and nuanced \\meaning
to their work. It allows for a richer reading experience, where the literal events are \\imbued with deeper symbolic or thematic significance.\end{tabular} \\ \hline
\begin{tabular}[c]{@{}l@{}}Human\\ Instruction\end{tabular}             & \begin{tabular}[c]{@{}l@{}}\{\{M\}\}\\ \\ Based on the story that you just read, answer the following question.\\ \textit{\color{blue}Does the story make sophisticated use of idiom or metaphor or literary allusion?}\\ -Yes \\ -No \\\\ Reasoning: \end{tabular}                                                                       \\ \hline
\begin{tabular}[c]{@{}l@{}}LLM\\ Instruction\end{tabular}               & \begin{tabular}[c]{@{}l@{}}\{\{M\}\}\\ \\ Given the story above, please list out all the metaphors, idioms and literary allusions, and for each decide \\whether it is successful vs it feels forced or too easy.  Then overall, give your reasoning about the question \\below and give an answer to it between 'Yes' or 'No' only\\ \\ \textit{\color{blue} Q) Does the story make sophisticated use of idiom or metaphor or literary allusion?}\end{tabular}                                                                                                                                                                                                                    \\ \hline
\end{tabular}
\vspace{2ex}
\caption{\label{prompting}TTCW Fluency3 (Language Proficiency \& Literary Devices) }
\vspace{-5ex}
\end{table*}


% ==================================================



\begin{table*}[!ht]
\centering
\small
% \def\arraystretch{1.15}
\begin{tabular}{|l|l|}
\hline
\begin{tabular}[c]{@{}l@{}}Expert \\ Measure\end{tabular}               & Does the end of the story feel natural and earned, as opposed to arbitrary or abrupt?                                                                                                                                    \\ \hline
\begin{tabular}[c]{@{}l@{}}Expanded\\ Expert\\ Measure (M)\end{tabular} & \begin{tabular}[c]{@{}l@{}}If the writer ends the piece simply because they are 'tired of writing', the conclusion might feel abrupt, disjointed, \\or unfulfilling to the reader. It suggests a rushed ending, where plot threads might be left unresolved and character \\arcs incomplete.\\ \\ 

Conversely, if the writer concludes because they've reached `the moment the entire piece has been leading readers \\towards', it implies a well-considered and purposeful ending. The events, character development, and themes \\throughout the story have built towards this climactic moment, providing a satisfying resolution to the reader.\\ \\ 

A strong ending offers a sense of closure, ties up the central conflicts or questions of the story, and generally \\leaves the reader feeling that the narrative journey was worthwhile and complete.\end{tabular} \\ \hline
\begin{tabular}[c]{@{}l@{}}Human\\ Instruction\end{tabular}             & \begin{tabular}[c]{@{}l@{}}\{\{M\}\}\\ \\ Based on the story that you just read, answer the following question.\\ \textit{\color{blue}Does the end of the story feel natural and earned, as opposed to arbitrary or abrupt?}\\ -Yes \\ -No \\\\ Reasoning : \end{tabular}                                                                       \\ \hline
\begin{tabular}[c]{@{}l@{}}LLM\\ Instruction\end{tabular}               & \begin{tabular}[c]{@{}l@{}}\{\{M\}\}\\ \\ Given the story above, answer the following question. Please first explain your reasoning step by step \\ and then given an answer between 'Yes' or 'No' only\\ \\ \textit{\color{blue} Q) Does the end of the story feel natural and earned, as opposed to arbitrary or abrupt?}\end{tabular}                                                                                                                                                                                                                    \\ \hline
\end{tabular}
\vspace{2ex}
\caption{\label{prompting}TTCW Fluency4 (Narrative Ending) }
\vspace{-5ex}
\end{table*}



% ==================================================



\begin{table*}[!ht]
\centering
\small
% \def\arraystretch{1.15}
\begin{tabular}{|l|l|}
\hline
\begin{tabular}[c]{@{}l@{}}Expert \\ Measure\end{tabular}               & Do the different elements of the story work together to form a unified, engaging, and satisfying whole?                                                                                                                                     \\ \hline
\begin{tabular}[c]{@{}l@{}}Expanded\\ Expert\\ Measure (M)\end{tabular} & \begin{tabular}[c]{@{}l@{}}A well-crafted story usually follows a logical path, where the events in the beginning set up the middle, which then\\ logically leads to the end. Every scene, character action, and piece of dialogue should serve the story and propel it \\forward. Well-written stories have an underlying the unity that binds the elements together. The themes, plotlines, \\character arcs, and other elements of the story interweave to create a harmonious whole. A story with 'disorder'\\ might feel disjointed, with characters, scenes, etc that don't connect or contribute to the overall narrative.\end{tabular} \\ \hline
\begin{tabular}[c]{@{}l@{}}Human\\ Instruction\end{tabular}             & \begin{tabular}[c]{@{}l@{}}\{\{M\}\}\\ \\ Based on the story that you just read, answer the following question.\\ \textit{\color{blue}Do the different elements of the story work together to form a unified, engaging, and satisfying whole?}\\ -Yes \\ -No \\\\ Reasoning : \end{tabular}                                                                       \\ \hline
\begin{tabular}[c]{@{}l@{}}LLM\\ Instruction\end{tabular}               & \begin{tabular}[c]{@{}l@{}}\{\{M\}\}\\ \\ Given the story above, answer the following question. Please first explain your reasoning step by step and then \\give an answer between 'Yes' or 'No' only\\ \\ \textit{\color{blue} Q) Do the different elements of the story work together to form a unified, engaging, and satisfying whole?}\end{tabular}                                                                                                                                                                                                                                 \\ \hline
\end{tabular}
\vspace{2ex}
\caption{\label{prompting}TTCW Fluency5 (Understandability \& Coherence) }
\vspace{-5ex}
\end{table*}


% ==================================================



\begin{table*}[!ht]
\centering
\small
% \def\arraystretch{1.15}
\begin{tabular}{|l|l|}
\hline
\begin{tabular}[c]{@{}l@{}}Expert \\ Measure\end{tabular}               & \begin{tabular}[c]{@{}l@{}}Does the story provide diverse perspectives, and if there are unlikeable characters, are their perspectives \\presented convincingly and accurately? \end{tabular}                                                                                                                                     \\ \hline
\begin{tabular}[c]{@{}l@{}}Expanded\\ Expert\\ Measure (M)\end{tabular} & \begin{tabular}[c]{@{}l@{}}A good writer can convincingly and accurately depict a wide range of character viewpoints, including those of\\ characters who may be morally ambiguous, difficult, or otherwise unappealing.\\ \\ 

This can involve diving into the mindset of characters who may act or think in ways that the reader, or even \\the writer, finds objectionable or repugnant. It involves understanding their motivations, their beliefs, and the \\reasons behind their actions, and then conveying these elements in a way that is believable and consistent.\\ \\ 

The purpose of doing so is not to justify or endorse these perspectives, but rather to create complex, three-\\dimensional characters who contribute to the richness and depth of the story. This can also serve to \\challenge the reader, provoke thought, and provide insights into different aspects of the human experience.\end{tabular} \\ \hline
\begin{tabular}[c]{@{}l@{}}Human\\ Instruction\end{tabular}             & \begin{tabular}[c]{@{}l@{}}\{\{M\}\}\\ \\ Based on the story that you just read, answer the following question.\\ \textit{\color{blue}Does the story provide diverse perspectives, and if there are unlikeable characters, are their perspectives presented} \\ \textit{\color{blue}convincingly and accurately?}\\ -Yes \\ -No \\\\ Reasoning : \end{tabular}                                                                       \\ \hline
\begin{tabular}[c]{@{}l@{}}LLM\\ Instruction\end{tabular}               & \begin{tabular}[c]{@{}l@{}}\{\{M\}\}\\ \\ Given the story above, answer the following question. Please first explain your reasoning step by step and then \\give an answer between 'Yes' or 'No' only\\ \\ \textit{\color{blue} Q) Does the story provide diverse perspectives, and if there are unlikeable characters, are their perspectives presented}\\\textit{\color{blue} convincingly and accurately?}\end{tabular}                                                                                                                                                                                                                                 \\ \hline
\end{tabular}
\vspace{2ex}
\caption{\label{prompting}TTCW Flexibility1 (Perspective \& Voice Flexibility) }
\vspace{-5ex}
\end{table*}


% ==================================================




\begin{table*}[!ht]
\centering
\small
% \def\arraystretch{1.15}
\begin{tabular}{|l|l|}
\hline
\begin{tabular}[c]{@{}l@{}}Expert \\ Measure\end{tabular}               & \begin{tabular}[c]{@{}l@{}}Does the story achieve a good balance between interiority and exteriority, in a way that feels \\emotionally flexible? \end{tabular}                                                                                                                                     \\ \hline
\begin{tabular}[c]{@{}l@{}}Expanded\\ Expert\\ Measure (M)\end{tabular} & \begin{tabular}[c]{@{}l@{}}`Emotional flexibility' is asking whether the piece of writing effectively balances action and introspection, and \\if it portrays a broad and realistic spectrum of emotions.\\ \\

`Exteriority' refers to the observable actions, behaviors, or dialogue of a character, and the physical or visible \\aspects of the setting, plot, and conflicts.\\ \\

`Interiority', on the other hand, pertains to the inner life of a character — their thoughts, feelings, memories, \\and subjective experiences.\\ \\

A balance between these two aspects is crucial in creating well-rounded characters and compelling narratives. \\If a piece is too heavy on exteriority, it may feel shallow or lack emotional depth. If it leans too much on\\ interiority, it could become overly introspective and potentially lose the momentum of the plot.
\end{tabular} \\ \hline
\begin{tabular}[c]{@{}l@{}}Human\\ Instruction\end{tabular}             & \begin{tabular}[c]{@{}l@{}}\{\{M\}\}\\ \\ Based on the story that you just read, answer the following question.\\ \textit{\color{blue}Does the story achieve a good balance between interiority and exteriority, in a way that feels emotionally flexible?}\\ -Yes \\ -No \\\\ Reasoning : \end{tabular}                                                                       \\ \hline
\begin{tabular}[c]{@{}l@{}}LLM\\ Instruction\end{tabular}               & \begin{tabular}[c]{@{}l@{}}\{\{M\}\}\\ \\ Given the story above, answer the following question. Please first explain your reasoning step by step and \\then give an answer between 'Yes' or 'No' only\\ \\ \textit{\color{blue}Q) Does the story achieve a good balance between interiority and exteriority, in a way that feels} \\\textit{\color{blue}emotionally flexible?}\end{tabular}                                                                                                                                                                                                                                 \\ \hline
\end{tabular}
\vspace{2ex}
\caption{\label{prompting}TTCW Flexibility2 (Emotional Flexibility) }
\vspace{-5ex}
\end{table*}


% ==================================================




\begin{table*}[!ht]
\centering
\small
% \def\arraystretch{1.15}
\begin{tabular}{|l|l|}
\hline
\begin{tabular}[c]{@{}l@{}}Expert \\ Measure\end{tabular}               & \begin{tabular}[c]{@{}l@{}}Does the story contain turns that are both surprising and appropriate? \end{tabular}                                                                                                                                     \\ \hline
\begin{tabular}[c]{@{}l@{}}Expanded\\ Expert\\ Measure (M)\end{tabular} & \begin{tabular}[c]{@{}l@{}}`Surprising': This refers to the element of unpredictability in a narrative. A good story often has plot twists, \\character developments, or thematic revelations that surprise the reader, subverting their expectations in a \\thrilling way.It's about keeping readers engaged and curious, never fully knowing what's going to happen next.\\ \\ 

`Appropriate': Despite the surprises and twists, the turns in the story must also make sense within the established \\context of the story's universe, its characters, and its themes. This means that even though an event might be \\surprising, it should feel appropriate or fitting in hindsight. It shouldn't feel like the writer has broken the rules \\they've set up, or made a character behave inconsistently without reason, simply for the sake of shock value.\\ \\ 

So when someone wonders if a writer can make turns that are 'both surprising and appropriate', they're asking \\if the writer can strike this balance between unexpectedness and coherence, keeping the reader on their toes \\while maintaining a believable, satisfying narrative. \end{tabular} \\ \hline
\begin{tabular}[c]{@{}l@{}}Human\\ Instruction\end{tabular}             & \begin{tabular}[c]{@{}l@{}}\{\{M\}\}\\ \\ Based on the story that you just read, answer the following question.\\ \textit{\color{blue}Does the story contain turns that are both surprising and appropriate?}\\ -Yes \\ -No \\\\ Reasoning: \end{tabular}                                                                       \\ \hline
\begin{tabular}[c]{@{}l@{}}LLM\\ Instruction\end{tabular}               & \begin{tabular}[c]{@{}l@{}}\{\{M\}\}\\ \\ Given the story above, list each element in the story that is intended to be surprising. For each, decide whether the\\ surprising element remains appropriate with respect to the entire story. Then overall, give your reasoning \\about the question below and give an answer to it between 'Yes' or 'No' only\\ \\ \textit{\color{blue} Q) Does the story contain turns that are both surprising and appropriate?}\end{tabular}                                                                                                                                                                                                                                 \\ \hline
\end{tabular}
\vspace{2ex}
\caption{\label{prompting}TTCW Flexibility3 (Structural Flexibility) }
\vspace{-5ex}
\end{table*}


% ==================================================






\begin{table*}[!ht]
\centering
\small
% \def\arraystretch{1.15}
\begin{tabular}{|l|l|}
\hline
\begin{tabular}[c]{@{}l@{}}Expert \\ Measure\end{tabular}               & \begin{tabular}[c]{@{}l@{}}Will an average reader of this story obtain a unique and original idea from reading it? \end{tabular}                                                                                                                                     \\ \hline
\begin{tabular}[c]{@{}l@{}}Expanded\\ Expert\\ Measure (M)\end{tabular} & \begin{tabular}[c]{@{}l@{}}If a story is good, the reader gains new insights, perspectives, or knowledge from it. This doesn't necessarily\\ mean factual information but could relate to a deeper understanding of human nature, cultural insights,\\ unique viewpoints, or even the exploration of new ideas and themes. Essentially, it's about what\\ the reader takes away from the story beyond just the plot.\\ \\ 

A good story has lasting impacts on its readers and the society. It is meant to entertain, inform, provoke \\thought, challenge beliefs, provide comfort, or raise awareness on specific issues.
 \end{tabular} \\ \hline
\begin{tabular}[c]{@{}l@{}}Human\\ Instruction\end{tabular}             & \begin{tabular}[c]{@{}l@{}}\{\{M\}\}\\ \\ Based on the story that you just read, answer the following question.\\ \textit{\color{blue}Will an average reader of this story obtain a unique and original idea from reading it?}\\ -Yes \\ -No \\\\ Reasoning : \end{tabular}                                                                       \\ \hline
\begin{tabular}[c]{@{}l@{}}LLM\\ Instruction\end{tabular}               & \begin{tabular}[c]{@{}l@{}}\{\{M\}\}\\ \\ Given the story above, list out elements that are unique takeaways of this story for the reader. Then overall, \\give your reasoning about the question below and give an answer to it between 'Yes' or 'No' only\\ \\ \textit{\color{blue} Q) Will an average reader of this story obtain a unique and original idea from reading it?}\end{tabular}                                                                                                                                                                                                                                 \\ \hline
\end{tabular}
\vspace{2ex}
\caption{\label{prompting}TTCW Originality1 (Originality in Theme and Content) }
\vspace{-3ex}
\end{table*}


% ==================================================








\begin{table*}[!ht]
\centering
\small
% \def\arraystretch{1.15}
\begin{tabular}{|l|l|}
\hline
\begin{tabular}[c]{@{}l@{}}Expert \\ Measure\end{tabular}               & \begin{tabular}[c]{@{}l@{}}Is the story an original piece of writing without any cliches?\end{tabular}                                                                                                                                     \\ \hline
\begin{tabular}[c]{@{}l@{}}Expanded\\ Expert\\ Measure (M)\end{tabular} & \begin{tabular}[c]{@{}l@{}}A cliche is an idea, expression, character, or plot that has been overused to the point of losing its original \\meaning or impact. They often become predictable and uninteresting for the reader. Originality suggests\\ that the piece isn't cliche.

 \end{tabular} \\ \hline
\begin{tabular}[c]{@{}l@{}}Human\\ Instruction\end{tabular}             & \begin{tabular}[c]{@{}l@{}}\{\{M\}\}\\ \\ Based on the story that you just read, answer the following question.\\ \textit{\color{blue}Is the story an original piece of writing without any cliches?}\\ -Yes \\ -No \\\\ Reasoning: \end{tabular}                                                                       \\ \hline
\begin{tabular}[c]{@{}l@{}}LLM\\ Instruction\end{tabular}               & \begin{tabular}[c]{@{}l@{}}\{\{M\}\}\\ \\ Given the story above, are there any cliches in the story? If so, list out all the elements in this story that \\are cliche. Then overall, give your reasoning if the piece is negatively impacted by the cliches and give \\an answer to the question below between 'Yes' or 'No' only\\ \\ \textit{\color{blue} Q) Is the story an original piece of writing without any cliches?}\end{tabular}                                                                                                                                                                                                                                 \\ \hline
\end{tabular}
\vspace{2ex}
\caption{\label{prompting}TTCW Originality2 (Originality in Thought) }
\vspace{-5ex}
\end{table*}


% ==================================================




\begin{table*}[!ht]
\centering
\small
% \def\arraystretch{1.15}
\begin{tabular}{|l|l|}
\hline
\begin{tabular}[c]{@{}l@{}}Expert \\ Measure\end{tabular}               & \begin{tabular}[c]{@{}l@{}}Does the story show originality in its form?\end{tabular}                                                                                                                                     \\ \hline
\begin{tabular}[c]{@{}l@{}}Expanded\\ Expert\\ Measure (M)\end{tabular} & \begin{tabular}[c]{@{}l@{}}When someone says that a piece of fiction 'shows an innovative use of form/structure', they're referring to\\ how the writer has chosen to shape and organize the story in an unusual, original, or inventive way. This could \\involve a variety of different elements, including:\\ \\ 

Narrative Structure: This could include unconventional timelines (e.g. a non-linear story, a story told in reverse)\\, multiple perspectives or narrators, or unusual narrative voices (e.g. a story told from the perspective of an \\inanimate object).\\ \\ 

Format: This could be something as simple as using unconventional punctuation or capitalization, or as complex \\as telling a story through a series of letters, diary entries, newspaper clippings, or other documents. In recent years,\\ some authors have even experimented with using social media posts or text messages as a form of narrative structure.\\ \\ 

Genre Hybridity: Combining elements from different genres or sub-genres in unexpected ways can also be seen\\ as an innovative use of form such as Horror-Mystery or Comic Fantasy.\\ \\ 

Plot Structure: Deviating from traditional plot structures such as three-act structure, or following them in unexpected\\ ways.For example, telling a story without a clear resolution, incorporating multiple climaxes or using revelation as a \\device where a surprising, and often shocking, development occurs that was previously kept hidden from the \\characters and/or the audience. It's typically designed to provide new context for interpreting what has previously \\occurred in the story. \\ \\ 

Language and Style: Innovative use of form can also come in the form of unique use of language, style, or \\even typography, such as concrete poetry or writing that visually represents its subject matter on the page.\\ \\ 

The goal of this innovation is often to provide a fresh reader experience, challenge conventional reading\\ expectations, or to create a deeper or more complex exploration of the story's themes.

 \end{tabular} \\ \hline
\begin{tabular}[c]{@{}l@{}}Human\\ Instruction\end{tabular}             & \begin{tabular}[c]{@{}l@{}}\{\{M\}\}\\ \\ Based on the story that you just read, answer the following question.\\ \textit{\color{blue}Does the story show originality in its form?}\\ -Yes \\ -No \\\\ Reasoning: \end{tabular}                                                                       \\ \hline
\begin{tabular}[c]{@{}l@{}}LLM\\ Instruction\end{tabular}               & \begin{tabular}[c]{@{}l@{}}\{\{M\}\}\\ \\ Given the story and the devices mentioned above, list each device used with a short explanation of whether it is \\successful or not. Then overall, give your reasoning about the question below and give an answer to it\\ between 'Yes' or 'No' only\\ \\ \textit{\color{blue} Q) Does the story show originality in its form?}\end{tabular}                                                                                                                                                                                                                                 \\ \hline
\end{tabular}
\vspace{2ex}
\caption{\label{prompting}TTCW Originality3 (Originality in Form) }
\vspace{-5ex}
\end{table*}


% ==================================================




\begin{table*}[!ht]
\centering
\small
% \def\arraystretch{1.15}
\begin{tabular}{|l|l|}
\hline
\begin{tabular}[c]{@{}l@{}}Expert \\ Measure\end{tabular}               & \begin{tabular}[c]{@{}l@{}}Does each character in the story feel developed at the appropriate complexity level, ensuring that no character \\feels like they are present simply to satisfy a plot requirement?\end{tabular}                                                                                                                                     \\ \hline
\begin{tabular}[c]{@{}l@{}}Expanded\\ Expert\\ Measure (M)\end{tabular} & \begin{tabular}[c]{@{}l@{}} A `flat character' is typically a minor character who is not thoroughly developed or who does not undergo \\significant change or growth throughout the story. They often embody or represent a single trait or idea, \\and they're used to advance the plot or highlight certain qualities in other characters.\\ \\ 

A `complex character', also known as a round character, has depth in feelings and passions, has a variety \\of traits of a real human being, and evolves over time. They have their strengths, weaknesses, \\and they learn from their experiences. They tend to be more engaging to the reader, as they mirror \\the complexity of real people.\\ \\ 

In good stories, authors take a character who initially appears to be one-dimensional or stereotypical (flat) and \\add depth to them. This could be done by revealing more about their backstory, introducing unexpected traits \\or motivations, or having them grow and change in response to the events of the story. \\This transformation from a flat to a complex character can make the narrative more engaging and believable.
 \end{tabular} \\ \hline
\begin{tabular}[c]{@{}l@{}}Human\\ Instruction\end{tabular}             & \begin{tabular}[c]{@{}l@{}}\{\{M\}\}\\ \\ Based on the story that you just read, answer the following question.\\  \textit{\color{blue} Q) Does each character in the story feel developed at the appropriate complexity level, ensuring that no character} \\ \textit{\color{blue}feels like they are present simply to satisfy a plot requirement?}\\ -Yes \\ -No \\\\ Reasoning: \end{tabular}                                                                       \\ \hline
\begin{tabular}[c]{@{}l@{}}LLM\\ Instruction\end{tabular}               & \begin{tabular}[c]{@{}l@{}}\{\{M\}\}\\ \\ Given the story above, list each character and the level of development. Then overall, give your reasoning \\about the question below and give an answer to it between 'Yes' or 'No' only\\ \\ 
 \textit{\color{blue} Q) Does each character in the story feel developed at the appropriate complexity level, ensuring that no character} \\ \textit{\color{blue}feels like they are present simply to satisfy a plot requirement?}\end{tabular}                                                                                                                                                                                                                                 \\ \hline
\end{tabular}
\vspace{2ex}
\caption{\label{prompting}TTCW Elaboration2 (Character Development) }
\vspace{-5ex}
\end{table*}


% ==================================================



\begin{table*}[!ht]
\centering
\small
% \def\arraystretch{1.15}
\begin{tabular}{|l|l|}
\hline
\begin{tabular}[c]{@{}l@{}}Expert \\ Measure\end{tabular}               & \begin{tabular}[c]{@{}l@{}}Are there passages in the story that involve subtext and when there is subtext, does it enrich the story's setting \\or does it feel forced?\end{tabular}                                                                                                                                     \\ \hline
\begin{tabular}[c]{@{}l@{}}Expanded\\ Expert\\ Measure (M)\end{tabular} & \begin{tabular}[c]{@{}l@{}} `Surface' level: This is the most apparent and straightforward level of a story. It includes the visible actions, \\explicit dialogue, and clear descriptions. This is what literally happens in the plot: the characters' actions, events, \\and the apparent consequences.\\ \\ 

`Subtext' level: This is the underlying or implicit meaning that isn't directly stated but can be inferred from \\the characters'  actions, dialogue, and other elements of the story. Subtext often reveals deeper truths about \\characters, themes, or the overall message of the piece. It could be a hidden motive, an unstated\\ emotion, a cultural commentary, or a symbolic meaning.\\ \\ 

For example, in a conversation between two characters, the surface text might be polite and cordial, but the \\subtext \\discerned from the characters' nonverbal cues, previous interactions, or the context of their conversation\\ — could suggest tension or hostility.\\ \\ 

Effective fiction often operates on both levels. The surface text keeps the reader engaged with the plot and \\characters, while the subtext provides depth, complexity, and additional layers of interpretation, \\contributing to a richer and more rewarding reading experience.
 \end{tabular} \\ \hline
\begin{tabular}[c]{@{}l@{}}Human\\ Instruction\end{tabular}             & \begin{tabular}[c]{@{}l@{}}\{\{M\}\}\\ \\ Based on the story that you just read, answer the following question.\\  \textit{\color{blue} Q) Are there passages in the story that involve subtext and when there is subtext, does it enrich the story's setting} \\ \textit{\color{blue} or does it feel forced?}\\ -Yes \\ -No \\\\ Reasoning: \end{tabular}                                                                       \\ \hline
\begin{tabular}[c]{@{}l@{}}LLM\\ Instruction\end{tabular}               & \begin{tabular}[c]{@{}l@{}}\{\{M\}\}\\ \\ Given the story above, answer the following question. Please first explain your reasoning step by step \\and then give an answer between 'Yes' or 'No' only\\ \\ 
 \textit{\color{blue} Q)Are there passages in the story that involve subtext and when there is subtext, does it enrich the story's setting} \\ \textit{\color{blue} or does it feel forced?}\end{tabular}                                                                                                                                                                                                                                 \\ \hline
\end{tabular}
\vspace{2ex}
\caption{\label{prompting}TTCW Elaboration3 (Rhetorical Complexity) }
\vspace{-5ex}
\end{table*}


% ==================================================




\end{document}
