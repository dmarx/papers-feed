
\newcommand{\baseb}[3]{\parens{{#1},{#2}}_{{#3}}}
\newcommand{\mx}[1]{\mathbf{#1}}
\newcommand{\floors}[1]{\left \lfloor #1 \right \rfloor}
\newcommand{\parens}[1]{\left( {#1}\right)}

\section{General Monarch Matrix Parametrization}
\label{sec:permutation}

In Section \ref{sec:Monarch_square}, we define a parametrization for square Monarch matrices of different ``block sizes'' (i.e., not necessarily $\sqrt{n}$), and prove some basic properties about them. In Section \ref{sec:Monarch_rect}, we further extend this to define rectangular Monarch matrices, and prove some basic properties about them.

Note: In this section, we use 0-indexing rather than 1-indexing, for notational convenience.

\subsection{General square matrices}
\label{sec:Monarch_square}
\subsubsection{Parametrization}
\label{sec:Monarch_square_param}
In this section, we define a more general Monarch parametrization for square matrices, allowing for different ``block sizes.'' Like \cref{def:Monarch}, the parametrization involves the product of a permuted block-diagonal matrix with another block-diagonal matrix; the difference is that we now allow the matrices $\vL$ and $\vR$ to have diagonal blocks of different sizes. Thus, the permutations applied to $\vL$ (to turn it into a block matrix where each block matrix is diagonal) will correspondingly also be different.

First, in \cref{def:square_r}, we define notation for a class of block-diagonal matrices.

\begin{definition}[Class $\BD\ind{b, n}$]
\label{def:square_r}
Let $b \in (1, n)$ be an integer that divides $n$. For $0\le i< \frac {n}{b}$, let $\mx{R}_{i}\in\F^{b \times b }$ be a $b \times b $ ``block" matrix. Then define the matrix $\vR$ with {\em block size} $b$ as follows:
\begin{equation}
 \label{eq:def-R}
  \vR = \diag\left(\vR_0, \dots, \vR_{\frac {n}{b}-1}\right).
\end{equation}
\end{definition}
(Note that the number of possible nonzero values in $\vR$ is $\frac {n}{b}\cdot b^2=nb$.)
We denote the class of all matrices $\vR$ expressible in this form by $\BD\ind{b, n}$. Note that this class is closed under (conjugate) transposition and contains the identity matrix.

Next, in \cref{def:Matrix L}, we define notation for a class of block matrices whose \emph{blocks} are diagonal.

\begin{definition}[Class $\DB\ind{b,n}$]
\label{def:Matrix L}
Let $b \in (1, n)$ be an integer that divides $n$. For $0 \le i, j < b$, let $\mx{D}_{i,j}\in\F^{b\times b}$ be a $b \times b$ diagonal matrix.
Then let $\vL$ be an $n \times n$ matrix with the following form: 
    \begin{equation}
        	\label{eq:def-L}
    \vL=
    	\begin{bmatrix}
    		\mx{D}_{0,0} & \dots & \mx{D}_{0,\frac{n}{b} -1} \\
    		\vdots & \ddots & \vdots \\
    		\mx{D}_{\frac{n}{b} -1,0} & \dots & \mx{D}_{\frac{n}{b} -1,\frac{n}{b} -1}
    	\end{bmatrix}
    \end{equation}
\end{definition}
(Note that the number of possible nonzero values in $\vL$ is $\parens{\frac nb}^2\cdot b=\frac{n^2}b$.)
We denote the class of all matrices $\vL$ expressible in this form by $\DB\ind{b, n}$. Note that this class is closed under (conjugate) transposition and contains the identity matrix. As we show in \cref{sec:sq-Monarch-properties}, $\vL$ can be written as a block-diagonal matrix with $b$ blocks of size $\ff{n}{b} \times \ff{n}{b}$ (i.e., a matrix in $\BD\ind{\frac{n}{b}, \, n}$), multiplied on the left and right with appropriate permutation matrices.
We denote the class of all matrices $\vL$ expressible in this form by $\DB\ind{b, n}$. Note that this class is closed under (conjugate) transposition. As we show in \cref{sec:sq-Monarch-properties}, $\vL$ can be written as a block-diagonal matrix with $b$ blocks of size $\ff{n}{b} \times \ff{n}{b}$ (i.e., a matrix in $\BD\ind{\frac{n}{b}, \, n}$), multiplied on the left and right with appropriate permutation matrices.

Using these two definitions, we define the class of Monarch matrices with a given block size.
\begin{definition}[Class $\M\ind{b,n}$]
\label{def:block_Monarch}
Let $b \in (1, n)$ be an integer that divides $n$. A \emph{Monarch matrix} of size $n \times n$ and ``block size $b$'' is a matrix of the form: 
    \begin{equation}
        	\label{eq:Monarch-general}
    \vM= \vL \vR
    \end{equation}
    where $\vL \in \DB\ind{b,n}$ and $\vR \in \BD\ind{b,n}$.
\end{definition}
We denote the class of all matrices $\vM$ expressible in this form by $\M\ind{b, n}$. Observe that when $b = \sqrt{n}$, this is exactly the matrix class $\M\ind{n}$ in \cref{def:Monarch}. (In other words, $\M\ind{n}$ is shorthand for $\M\ind{\sqrt{n}, n}$.) Note that a matrix in $\M\ind{b,n}$ is represented by $\frac{n^2}{b} + nb$ parameters.

We remark that $\M\ind{b,n} \supset \B\ind{n}$ for all block sizes $b \in (1, n)$ that divide $n$.

Based on \cref{def:block_Monarch}, we define the classes $\M\M^{*(b,n)}$ and $\M^*\M^{(b,n)}$::
\begin{definition}[Class $\M\M^{*(b,n)}$, $\M^*\M^{(b,n)}$]
\label{def:block_MM}
Let $b \in (1, n)$ be an integer that divides $n$ and suppose $\vM_1, \vM_2 \in \M^{(b,n)}$. We define $\M\M^{*(b,n)}$ to be the the class of all matrices $\vM$ expressible in the form $\vM= \vM_1 \vM_2^*$. \newline
We define $\M^*\M^{(b,n)}$ to be the the class of all matrices $\vM$ expressible in the form $\vM= \vM_1^* \vM_2$.
\end{definition}
Observe that when $b = \sqrt{n}$, $\M\M^{*(b,n)}$ is exactly the matrix class $\M\M^{*(n)}$ defined in \cref{sec:theory}. Note that a matrix in $\M\M^{*(b,n)}$ or $\M^*\M\ind{b,n}$. is represented by $2\frac{n^2}{b} + 2nb$ parameters.

Finally, we define the following ``Monarch hierarchy'' based on the kaleidoscope hierarchy of \cite{dao2020kaleidoscope}:
\begin{definition}[Class $(\M\M^{*(b,n)})^w_e$]
\label{def:block_MM}
Let $b \in (1, n)$ be an integer that divides $n$. We define the matrix class $(\M\M^{*(b,n)})^w_e$ as the set of all matrices $\vM$ that can be expressed as
    \begin{equation}
        	\label{eq:mm-hierarchy}
    \vM= \lt \pd{i=1}{w} \vM_i\rt [1:n, 1:n]
    \end{equation}
    where each $\vM_i \in \M\M^{*(b,e\cdot n)}$.
\end{definition}
Note that a matrix in $(\M\M^{*(b,n)})^w_e$ is represented by $2w\frac{e^2n^2}{b} + 2wenb$ parameters.

\subsubsection{Properties}
\label{sec:sq-Monarch-properties}
Here we show some properties of the matrix classes defined above. We first show some basic equivalent ways to define these classes. We then show (\cref{thm:lr_permutation}) that the matrices in $\DB\ind{b, n}$ are permuted block-diagonal matrices; specifically, that they can be converted to matrices in $\BD\ind{\frac{n}{b}, n}$ by applying the appropriate permutation. Finally, we state an expressivity result for the general ``Monarch hierarchy''  which follows from Theorem 1 of \cite{dao2020kaleidoscope}.

First, we define a class of permutations.
Let $1\le b\le n$ be integers such that $b$ divides $n$.
We will need to express each index $0\le i<n$ in ``block form.'' More specifically:

\begin{definition}\label{def:$i$}
Let $i \ge 0$, $b \ge 1$ be integers. Then define
\[i_0=i\mod{b},\]
and
\[i_1=\floors{\frac ib}.\] 
We use the notation $i\equiv\baseb{i_1}{i_0}{b}$ to denote the representation above. In particular, if $i\equiv(i_1,i_0)_{b}$,
then we have
\[
        i = i_1 \cdot b + i_0
\]
\end{definition}

Using this notation, we define the following class of permutations:
\begin{definition}
\label{def:sigma-b}
Let $b \in [1, n]$ be an integer that divides $n$.  Let $i\equiv\baseb{i_1}{i_0}{b}$. Define
    \begin{equation}
            \label{eq:sigma_b-def}
        \sigma_{(b,n)}(i) = i_0\cdot\frac{n}{b} + i_1.
    \end{equation}
That is, $\sigma_{(b,n)}(i)\equiv \baseb{i_0}{i_1}{\frac {n}{b}}$.
Let $\vP_{(b,n)}$ denote the $n \times n$ permutation matrix defined by the permutation $\sigma_{(b,n)}$.
\end{definition}
Intuitively, $\vP_{(b,n)}$ can be interpreted as reshaping a length-$n$ vector into an $b \times \ff{n}{b}$ matrix in row-major order, transposing the result, and then flattening this back into a vector (again in row-major order).


Now, we restate the formulation in \cref{def:square_r} equivalently as:
\begin{proposition}

\label{prop:R-eqv-def}
A matrix $\vR$ satisfies~\Cref{eq:def-R} (i.e., $\vR \in \BD\ind{b,n}$) if and only if the following holds for any
$0\le i,j< n$. Let $i\equiv\baseb{i_1}{i_0}{b}$ and $j\equiv\baseb{j_1}{j_0}{b}$.  Then
\begin{enumerate}
    \item\label{item:zero-loc-R} if $i_1\ne j_1$, then $\vR[i,j]=0$. 
    \item \label{item:nonzero-loc-R} Else (i.e., when $i_1=j_1$), then $\vR[i,j]=\vR_{i_1}[i_0,j_0]$.
\end{enumerate}

\end{proposition}



We restate the formulation in \cref{def:Matrix L} equivalently as:
\begin{proposition}
\label{prop:L-eqv-def}
A matrix $\vL$ satisfies~\Cref{eq:def-L} (i.e., $\vL \in \DB\ind{b,n}$) if and only if the following holds for any
$0\le i,j< n$. Let $i\equiv\baseb{i_1}{i_0}{b}$ and $j\equiv\baseb{j_1}{j_0}{b}$. Then 
\begin{enumerate}
    \item\label{item:zero-loc-L} if $i_0\ne j_0$, then $\vL[i,j]=0$. 
    \item \label{item:nonzero-loc-L} Else, (i.e., when $i_0=j_0$), then $\vL[i,j]=\vD_{i_1,j_1}[i_0,i_0]$.
\end{enumerate}
\end{proposition}

We will argue the following:
\begin{theorem}\label{thm:lr_permutation} Let $1\le b\le n$ such that $b$ divides $n$.
Recall that $\vP_{(b,n)}$ is the permutation matrix defined by the permutation $\sigma_{(b,n)}$. Let $\vL$ be a matrix in $\DB\ind{b, n}$. Then we have
\[\vR'=\vP_{(b,n)}\cdot\vL\cdot\vP_{(b,n)}^\top,\]
where $\vR' \in \BD\ind{\frac{n}{b},\, n}$.
\end{theorem}

\begin{proof}
We first note that multiplying an $n\times n$ matrix on the right (and left resp.) by $\vP_{(b,n)}^\top = \vP_{(\frac nb,n)}$ (and $\vP_{(b,n)}$ resp.) permutes the columns (and rows resp.) of the matrix according to $\sigma_{(b,n)}$.\footnote{This uses the fact that $\parens{\sigma_{(b,n)}}^{-1}=\sigma_{(\frac nb,n)}$ (which means $P_{(\frac{n}{b}, n)} = P_{(b, n)}^\top$ since the inverse of a permutation matrix is its transpose).} This implies that for any $0\le i,j<n$:
\begin{equation}
\label{eq:L-permuted}
\vR'[\sigma_{(b,n)}(i),\sigma_{(b,n)}(j)]=\vL[i,j].
\end{equation}
To complete the proof, we will argue that $\vR'$ satisfies the two conditions in~\Cref{prop:R-eqv-def}.

Towards this end, let $0\le i,j<n$ be arbitrary indices and further, define $i=\baseb{i_1}{i_0}{b}$ and $j=\baseb{j_1}{j_0}{b}$. Then note that $\sigma_{(b,n)}(i)=\baseb{i_0}{i_1}{\frac nb}$ and $\sigma_{(b,n)}(j)=\baseb{j_0}{j_1}{\frac nb}$.

By~\Cref{prop:L-eqv-def}, we have that if $i_0\ne j_0$, then $\vL[i,j]=0$. Note that $i_0\ne j_0$ satisfies the pre-condition for base size $\frac nb$ for indices $(\sigma_{(b,n)}(i),\sigma_{(b,n)}(j))$ in item~\ref{item:zero-loc-R} in~\Cref{prop:R-eqv-def}.  Then   by~\cref{eq:L-permuted}, we have that $\vR'[\sigma_{(b,n)}(i),\sigma_{(b,n)}(j)]=0$, which satisfies item~\ref{item:zero-loc-R} in~\Cref{prop:R-eqv-def}.

Now consider the case that $i_0=j_0$; then by item~\ref{item:nonzero-loc-L} in~\Cref{prop:L-eqv-def}, we have that $\vL[i,j]=\vD_{i_1,j_1}[i_0,i_0]$.  Note that $i_0= j_0$ satisfies the pre-condition for base size $\frac nb$ for indices $(\sigma_{(b,n)}(i),\sigma_{(b,n)}(j))$ in item~\ref{item:nonzero-loc-R} in~\Cref{prop:R-eqv-def} if we define $\vR'_{i_0}\in\F^{\frac nb\times\frac nb}$ as follows:
\[\vR'_{i_0}[i_1,j_1]=\vD_{i_1,j_1}[i_0,i_0].\] 
Note that the above implies that 
\[\vR'=\diag\parens{\vR'_0,\dots,\vR'_{b-1}},\]
where $\vR'_{\cdot}$ is as defined in the above paragraph. This means $\vR' \in \BD\ind{\frac{n}{b}, n}$, since each block $\vR_{i_0}'$ is a matrix of size $\frac{n}{b} \times \frac{n}{b}$.
\end{proof}

We now briefly note some alternate ways to express matrices in $\M\M^{*(b,n)}$.
\begin{proposition}
\label{prop:mm-eqv-def}
For any $\vM \in \M\M^{*(b,n)}$, we can write $\vM = (\vP_{(b,n)}^\top \vL_1 \vP_{(b,n)}) \vR (\vP_{(b,n)}^\top\vL_2\vP_{(b,n)})$, where $\vL_1,\vL_2 \in \BD\ind{\frac{n}{b},n}$ and $\vR \in \BD\ind{b,n}$.
\end{proposition}
\begin{proof}
By definition (see \cref{def:square_r} and \cref{def:Matrix L}), if $\vM \in \M\M^{*(b,n)}$, we can write
$\vM = (\vL_1' \vR_1) (\vL_2' \vR_2)^* = \vL_1' (\vR_1^* \vR_2) \vL_2'^*$,
where $\vL_1',\vL_2' \in \DB\ind{b,n},\vR_1,\vR_2 \in \BD\ind{b,n}$.

Notice that since $\vR_1^*, \vR_2$ are both block-diagonal with the same structure (i.e., both have blocks of size $b \times b$), their product $\vR$ is also in $\BD\ind{b,n}$.
Also, by \cref{thm:lr_permutation} we can write $\vL_1 = \vP_{(b,n)} \vL_1' \vP_{(b,n)}^\top$, $\vL_2 = \vP_{(b,n)} \vL_2' \vP_{(b,n)}^\top$, where $\vL_1,\vL_2$ are both in $\BD\ind{\frac{n}{b},n}$ (i.e., block diagonal with blocks of size $\frac{n}{b} \times \frac{n}{b}$).


Thus, we can write $\vM = (\vP_{(b,n)}^\top \vL_1 \vP_{(b,n)}) \vR (\vP_{(b,n)}^\top\vL_2\vP_{(b,n)})$, where $\vL_1,\vL_2 \in \BD\ind{\frac{n}{b},n}$ and $\vR \in \BD\ind{b,n}$.
\end{proof}

We use the above to show a simple relationship between $\M\M^{*(b,n)}$ and $\M^*\M^{(b,n)}$.
\begin{proposition}
\label{prop:mstarm}
If $\vM \in \M\M^{*(b,n)}$, then $\vP_{(b,n)} \vM \vP_{(b,n)}^\top \in \M^*\M\ind{\frac{n}{b},n}$. Conversely, if $\vM \in \M^*\M^{(b,n)}$, then $\vP_{(b,n)}^\top \vM \vP_{(b,n)} \in \M^*\M\ind{\frac{n}{b},n}$.
\end{proposition}
\begin{proof}
Suppose $\vM \in \M\M^{*(b,n)}$. By \cref{prop:mm-eqv-def} we can write $\vM = (\vP_{(b,n)}^\top \vL_1 \vP_{(b,n)}) \vR (\vP_{(b,n)}^\top\vL_2\vP_{(b,n)})$, where $\vL_1,\vL_2 \in \BD\ind{\frac{n}{b},n}$ and $\vR \in \BD\ind{b,n}$.
Thus $\vP_{(b,n)} \vM \vP_{(b,n)}^\top =
\vL_1 (\vP_{(b,n)} \vR \vP_{(b,n)}^\top) \vL_2$.

Letting $\vL_1' = \vL_1, \vL_2' = \vL_2^*, \vR_1' = \vP_{(b,n)} \vR \vP_{(b,n)}^\top$, and $\vR_2' = \vI$, we have $\vL_1', \vL_2' \in \BD\ind{\frac{n}{b}, n}$, $\vR_1', \vR_2' \in \DB\ind{\frac{n}{b}, n}$, and
$\vL_1 (\vP_{(b,n)} \vR \vP_{(b,n)}^\top) \vL_2 = 
\vL_1' \vR_1' \vR_2'^* \vL_2'^* = (\vL_1' \vR_1')(\vL_2' \vR_2')^* = \vM_1'\vM_2'^*$, where $\vM_1' = \vL_1' \vR_1', \vM_2' = \vL_2' \vR_2'$, so $\vM_1', \vM_2' \in \M^*\M\ind{\frac{n}{b},n}$.

Now instead suppose $\vM \in \M^*\M^{(b,n)}$. So $\vM = \vM_1^* \vM_2 = \vR_1^* \vL_1^* \vL_2 \vR_2$ for some $\vR_1, \vR_2 \in \BD\ind{b,n}$ and $\vL_1, \vL_2 \in \DB\ind{b,n}$. Thus by \cref{thm:lr_permutation} (and the fact that $\BD\ind{b,n}$ is closed under conjugate transposition) we can write $\vR_1^* = \vP_{(\frac{n}{b},n)}^\top \vR_1' \vP_{(\frac{n}{b}, n)} = \vP_{(b,n)} \vR_1' \vP_{(b, n)}^\top$ for some $\vR_1' \in \DB\ind{\frac{n}{b}, n}$, and similarly, can write $\vR_2 = \vP_{(b,n)} \vR_2' \vP_{(b,n)}^\top$ for some $\vR_2' \in \DB\ind{\frac{n}{b}, n}$. 

So $\vP_{(b,n)}^\top \vM \vP_{(b,n)} = \vR_1' (\vP_{(b, n)})^\top \vL_1^*)(\vL_2 \vP_{(b, n)})) \vR_2' =
 \vR_1' (\vP_{(b, n)}^\top \vL_1^* \vP_{(b, n)})(\vP_{(b, n)}^\top \vL_2 \vP_{(b, n)}) \vR_2' = (\vR_1' \vL_1')(\vL_2' \vR_2')$, where $\vL_1' = \vP_{(b, n)}^\top \vL_1^* \vP_{(b, n)}$, $\vL_2' = \vP_{(b, n)}^\top \vL_2 \vP_{(b, n)}$ are in $\BD\ind{\frac{n}{b}, n}$ by \cref{thm:lr_permutation}. Thus letting $\vM_1' = \vR_1'\vL_1'$, $\vM_2' = \vR_2^*\vL_2'^*$, we have $\vM = \vM_1' \vM_2'^*$ with $\vM_1', \vM_2' \in \M^{*(\frac{n}{b},n)}$.
\end{proof}

We now show that the class $\M\ind{b,n}$ strictly contains the class $\B\ind{n}$ of $n \times n$ butterfly matrices (as defined in \citet{dao2020kaleidoscope}). We first show two elementary ``helper'' results.

\begin{proposition}
\label{prop:bd-contain}
If $b,\, c \in (1, n)$ are such that $b$ divides $c$ and $c$ divides $n$, then $\BD\ind{b, n} \subseteq \BD\ind{c, n}$.
\end{proposition}
\begin{proof}
Suppose $\vR \in \BD\ind{b, n}$. Then by \cref{prop:R-eqv-def}, $\vR[i, j] = 0$ whenever $\floor{\frac{i}{b}} \ne \floor{\frac{j}{b}}$. Thus, whenever $\floor{\frac{i}{c}} \ne \floor{\frac{j}{c}}$, $\vR[i, j] = 0$, since $\floor{\frac{i}{c}} \ne \floor{\frac{j}{c}}$ implies $\floor{\frac{i}{b}} \ne \floor{\frac{j}{b}}$ by the assumption that $b$ divides $c$.
Applying \cref{prop:R-eqv-def} again, this means $\vR \in \BD\ind{c,n}$ as well.
\end{proof}

\begin{proposition}
\label{prop:db-contain}
If $b,\, c \in (1, n)$ are such that $b$ divides $c$ and $c$ divides $n$, then $\DB\ind{c, n} \subseteq \DB\ind{b, n}$.
\end{proposition}
\begin{proof}
Suppose $\vL \in \DB\ind{c, n}$. Then by \cref{prop:L-eqv-def}, $\vL[i, j] = 0$ whenever $(i \mod c) \ne (j \mod c)$. Thus, whenever $(i \mod b) \ne (j \mod b)$, $\vL[i, j] = 0$, since $(i \mod b) \ne (j \mod b)$ implies $(i \mod c) \ne (j \mod c)$ by the assumption that $b$ divides $c$.
Applying \cref{prop:L-eqv-def} again, this means $\vL \in \DB\ind{b,n}$ as well.
\end{proof}


\begin{theorem}
\label{thm:b_contained}
Let $n \ge 4$ be a power of 2. The class of matrices $\B\ind{n}$ is a subset of the class $\M\ind{b, n}$, for all $b \in (1, n)$ that divide $n$. When $n \ge 512$ it is a strict subset.
\end{theorem}
\begin{proof}
Recall from \cref{sec:butterfly} that if $\vB \in \B\ind{n}$, it has a \emph{butterfly factorization} 
$\vB = \vB_n \vB_{n/2} \hdots \vB_2$, where each $\vB_i \in \BF\ind{n, i}$.

Consider multiplying together the factors $\vB_b \vB_{b/2} \dots \vB_2$ (where $b \in (1, n)$ divides $n$). Since $\vB_i \in \BF\ind{n,i}$, by definition it is block diagonal with diagonal blocks of size $i \times i$; in other words, $\vB_i \in \BD\ind{i, n}$. Thus, each of the matrices $\vB_b, \vB_{b/2}, \dots, \vB_2$ is in $\BD\ind{b, n}$ (by \cref{prop:bd-contain}), i.e. block-diagonal with block size $b \times b$. This means their product $\vB_b \vB_{b/2} \dots \vB_2$ is also block diagonal with block size $b \times b$, i.e., it is in $\BD\ind{b, n}$.

Now, note that since $\vB_i \in \BF\ind{n,i}$, by definition it is a block matrix with blocks of size $i/2 \times i/2$, where each block is a diagonal matrix (note that some of these blocks are zero, except for the case of $\vB_n$). In other words, $\vB_i \in \DB\ind{i/2, n}$. Thus, for all $i \in \{n, n/2, \dots, 2b\}$, $\vB_i \in \DB\ind{(2b)/2, n} = \DB\ind{b, n}$ (by \cref{prop:db-contain}). So, their product $\vB_n \vB_{n/2} \dots \vB_{2b}$ is in $\DB\ind{b, n}$ as well, as by \cref{thm:lr_permutation} we can write $\vB_n \vB_{n/2} \dots \vB_{2b} = \vP_{(b,n)}^\top (\vP_{(b,n)} \vB_n \vP_{(b,n)}^\top) (\vP_{(b,n)} \vB_{n/2} \vP_{(b,n)}^\top) \dots (\vP_{(b,n)} \vB_{2b} \vP_{(b,n)}^\top) \vP_{(b,n)}$ and each of the $\vP_{(b,n)} \vB_i \vP_{(b,n)}^\top$'s in the preceding expression is in $\BD\ind{\frac{n}{b}, n}$.

Thus, if we let $\vL = \vB_n \vB_{n/2} \dots \vB_{2b}$ and $\vR = \vB_b \vB_{b/2} \dots \vB_2$,  we have $\vB = \vL\vR$ and $\vL \in \DB\ind{b, n}$, $\vR \in \BD\ind{b, n}$, which means that $\vB \in \M\ind{b,n}$ (\cref{def:block_Monarch}).

To show that the inclusion is strict, notice that any $\vM \in \M\ind{b,n}$ is the product of $\vL$ and $\vR$, where $\vR \in \BD\ind{b, n}$ and $\vP_{(b,n)}^\top \vL \vP_{(b,n)} \in \BD\ind{\frac{n}{b}, n}$ (by \cref{thm:lr_permutation}). Notice that the identity matrix is contained in both $\BD\ind{b,n}$ and $\DB\ind{b,n}$. Suppose first that $b \le \sqrt{n}$. Then even if we set $\vR$ to the identity, $\vM$ has at least $\frac{n^2}{b} \ge n^{3/2}$ free parameters (the entries in the blocks of the block-diagonal matrix $\vP_{(b,n)}^\top \vL \vP_{(b,n)}$ can be arbitrary, and there are $b$ such blocks each of size $\frac{n}{b}$). Similarly, in the case $b > \sqrt{n}$, we can set $\vL$ to the identity, and $\vM$ has at least $nb \ge n^{3/2}$ free parameters (the entries of the block-diagonal matrix $\vR$ can be arbitrary, and there are $nb$ total of these). Thus, at least $n^{3/2}$ parameters are required to uniquely describe any matrix in $\M\ind{b,n}$. However, a butterfly matrix in $\B\ind{n}$ has only $2n \log_2 n$ parameters. For $n > 256$, $2n \log_2 n < n^{3/2}$. (Note that this analysis is not tight: a more careful analysis can show the inclusion is strict even for smaller values of $n$.)

\end{proof}

We end this section with a theorem on the expressivity of the ``monarch hierarchy'' (products of monarch matrices), which follows from Theorem 1 of \cite{dao2020kaleidoscope}.
\begin{theorem}[Monarch hierarchy expressivity]
\label{thm:monarch_hierarchy}
Let $\vM$ be an $n \times n$ matrix such that matrix-vector multiplication of $\vM$ and an arbitrary vector $\vv$ (i.e., computation of $\vM \vv$) can
be represented as a linear arithmetic circuit with depth $d$ and $s$ total gates. Let $b \in (1, n)$ be a power of 2 that divides $n$.
Then, $\vM \in (\M\M^{*(b, n)})^{O(d)}_{O(s/n)}$.
\end{theorem}
\begin{proof}
Theorem 1 of \citet{dao2020kaleidoscope} says that if $n$ is a power of 2 and $\vA$ is an $n \times n$ matrix such that multiplying any vector $v$ by $\vA$ can be represented as a linear arithmetic circuit with depth $\le d$ and $\le s$ total gates, then $\vA \in (\B\B^{*(n)})^{O(d)}_{O(s/n)}$ (this is the ``kaleidoscope representation'' of $\vA$).

Recall from \cref{thm:b_contained} that for any $b \in (1, n)$ that is a power of 2 and divides $n$, $\M\ind{b, n} \supset \B\ind{n}$; thus, this implies $\M\M^{*(b,e\cdot n)} \supset \B\B^{*(e\cdot n)}$, and in turn $(\M\M^{*(b,n)})^w_e \supset (\B\B^{*(n)})^w_e$.

As $\vA \in  (\B\B^{*(n)})^{O(d)}_{O(s/n)}$, we thus have $\vA \in  (\M\M^{*(b,n)})^{O(d)}_{O(s/n)}$.
\end{proof}

As per \cite{dao2020kaleidoscope}, the class of kaleidoscope matrices $(\B\B^{*(n)})^{O(d)}_{O(s/n)}$ has $O(ds \log s)$ parameters and runtime, compared to the $O(s)$ parameters and runtime of the circuit. Note that at worst, $s$ is $O(n^2)$.

Define $f(n,s)$ to be the largest power of 2 that is $\le \min\left\{\ff{n}{2}, \sqrt{s}\right\}$. Note that $f(n,s) = O(\sqrt{s})$, and since $s = O(n^2)$, $f(n,s) = \Omega(\sqrt{s})$, so $f(n,s) = \Theta(\sqrt{s})$. %
We thus have $\vA \in (\M\M^{*(f(n,s), n)})^{O(d)}_{O(s/n)}$. The class $(\M\M^{*(f(n,s), n)})^{O(d)}_{O(s/n)}$ has $O(d\frac{s^2}{f(n,s)} + dsf(n,s)) = O(ds^{3/2})$ parameters. Thus, the monarch representation of $\vA$ is suboptimal by at most an $O(d\sqrt{s})$ factor compared to the $O(d{}\,\log s)$ of kaleidoscope.

\subsection{General rectangular matrices}
\label{sec:Monarch_rect}
In this section, we extend the Monarch parametrization to apply to \emph{rectangular} matrices, and prove some basic properties of the relevant matrix classes. (Note that our subsequent theoretical results (\cref{sec:proofs}) do not depend on this section, as they focus on the square parametrization.)

For the rest of the section, we will assume that $n_1, n_2, n_3, b_1, b_2 , b_3 \ge 1$ are integers such that:
\begin{itemize}
\item $b_i$ divides $n_i$ for all $1\le i\le 3$, and 
\item $\frac{n_1}{b_1} = \frac{n_2}{b_2}$.
\end{itemize}

We begin with the definition of the following class of rectangular block-diagonal matrices:
\begin{definition}
\label{def:monarch_rectangular}
For $0\le i< \frac{n}{b_1}$, let $\vR_{i}\in\F^{b_2 \times b_1}$ be a $b_2 \times b_1$ matrix. Then define the matrix $\vR \in \F^{n_2\times n_1}$ as follows:
\begin{equation}
 \label{eq:def-rect-R}
  \vR = \diag\left(\vR_0, \dots, \vR_{\frac {n_1}{b_1}-1}\right).
\end{equation}
\end{definition}
We say that $\vR$ has {\em block size} $b_2 \times b_1$. Recall that we have assumed $\frac {n_1}{b_1}=\frac{n_2}{b_2}$, so~\cref{eq:def-rect-R} is well-defined.
(Note that the number of possible nonzero values in $\vR$ is $\frac {n_1}{b_1}\cdot b_1 \times b_2 =n_1b_2$.)
We denote the class of all matrices $\vR$ expressible in this form by $\BD\ind{b_2 \times b_1, n_2 \times n_1}$.
Note that this class is only defined when $\frac {n_1}{b_1}=\frac{n_2}{n_2}$.



We restate the above definition equivalently as:
\begin{proposition} \label{prop:rect-R-eqv-def} 
$\vR\in\F^{n_2\times n_1}$ is in $\BD\ind{b_2 \times b_1, n_2 \times n_1}$ (with $\frac {n_1}{b_1}=\frac{n_2}{n_2}$) if and only if the following holds for any
$0\le i < n_2$ and $0\le j< n_1$. Let $i\equiv\baseb{i_1}{i_0}{b_2}$ and $j\equiv\baseb{j_1}{j_0}{b_1}$ (recalling this notation from \cref{def:$i$}.  Then
\begin{enumerate}
    \item\label{item:rect-zero-loc-R} if $i_1\ne j_1$, then $\vR[i,j]=0$. 
    \item \label{item:rect-nonzero-loc-R} Else (i.e., when $i_1=j_1$), then $\vR[i,j]=\vR_{i_1}[i_0,j_0]$.
\end{enumerate}

\end{proposition}

Before we define the rectangular $\vL$, we first need to define the notion of a `wrapped diagonal' matrix:
\begin{definition} 
\label{def:wrapped-diag}
A {\em wrapped diagonal} matrix $\mx{S} \in\F^{b_3\times b_2}$ is defined as follows. First assume $b_2\le b_3$. Then for any $0\le i<b_3$ and $0\le j<b_2$, we have the following. If $i\mod{b_2}\ne j$, then $\vS[i,j]=0$. (If $b_2>b_3$, then instead apply the previous definition to $\vS^{\top}$.)

\end{definition}


We now define the following class of block matrices with each block a \emph{wrapped diagonal} matrix.
\begin{definition}\label{def:rect-Matrix L}
Let $\vL\in\F^{n_3\times n_2}$ have the form: 
    \begin{equation}
        	\label{eq:rect-def-L}
    \vL=
    	\begin{bmatrix}
    		\mx{S}_{0,0} & \dots & \mx{S}_{0,\frac{n_2}{b_2} -1} \\
    		\vdots & \ddots & \vdots \\
    		\mx{S}_{\frac{n_3}{b_3} -1,0} & \dots & \mx{S}_{\frac{n_3}{b_3} -1,\frac{n_2}{b_2} -1}
    	\end{bmatrix},
    \end{equation}
where each $\vS_{\cdot,\cdot}$ is a wrapped diagonal matrix in $\F^{b_3 \times b_2}$.
\end{definition}
We say that $\vL$ has {\em block size} $b_3 \times b_2$.
(Note that the number of possible nonzero values in $\vL$ is $\parens{\frac {n_2}{b_2}\cdot \frac{n_3}{b_3}} \max(b_2,b_3)=\frac{n_2 \cdot n_3}{\min(b_2,b_3)}$.)
We denote the class of all matrices $\vL$ expressible in this form by $\DB\ind{b_3 \times b_2, n_3 \times n_2}$.

We restate the above definition equivalently as:


\begin{proposition}
\label{prop:rect-L-eqv-def}
$\vL\in\F^{n_3\times n_2}$ is in $\DB\ind{b_3 \times b_2, n_3 \times n_2}$ if and only if the following holds for any
$0\le i < n_3$ and $0 \le j< n_2$. Let $i\equiv\baseb{i_1}{i_0}{b_3}$ and $j\equiv\baseb{j_1}{j_0}{b_2}$.  Assuming $b_2 \le b_3$, we have:
\begin{enumerate}
    \item\label{item:rect-zero-loc-L} if $i_0\mod{b_2}\ne j_0$, then $\vL[i,j]=0$. 
    \item \label{item:rect-nonzero-loc-L} Else, (i.e., when $i_0\mod{b_2}=j_0$), then $\vL[i,j]=\vS_{i_1,j_1}[i_0,j_0]$.
\end{enumerate}
If $b_2>b_3$, then in the above, the condition ``$i_0\mod{b_2}\ne j_0$'' gets replaced by ``$j_0\mod{b_2}\ne i_0$.''
\end{proposition}

Using the above definitions, we now define the class of rectangular Monarch matrices.
\begin{definition}[Rectangular Monarch Matrix]
\label{def:block_Monarch}
Let $\vM \in \F^{n_3 \times n_1}$ be a matrix of the form: 
    \begin{equation}
        	\label{eq:Monarch-general}
    \vM= \vL \vR
    \end{equation}
    where $\vL \in \DB\ind{b_3 \times b_2, n_3 \times n_2}$ and $\vR \in \BD\ind{b_2 \times b_1, n_2 \times n_1}$.
\end{definition}
(As mentioned before, we assume $b_i$ divides $n_i$ for $i = 1,2,3$ and that $n_1/b_1 = n_2/b_2$.)
We denote the class of all matrices $\vM$ expressible in this form by $\M\ind{(b_1,b_2,b_3), (n_1,n_2,n_3)}$. Observe that when $b_1 = b_2 = b_3 = b$ and $n_1 = n_2 = n_3 = n$, this is exactly the matrix class $\M\ind{b, n}$ in \cref{def:block_Monarch}.

We are now ready to prove our main result in this section, which essentially follows from the observation that if we permute the rows and columns of $\vL$ such that the row/column block size in $\vL$ becomes the number of row/columns blocks in the permuted matrix (and vice-versa) then the permuted matrix has the form of $\vR$.


\begin{theorem} Let $1\le b,n_2,n_3$ be such that $b$ divides $n_2$ and $n_3$.
Suppose $\vL\in\F^{n_3\times n_2} \in \DB\ind{b \times b, n_3 \times n_2}$.
Then if we define
\[\vR'=\vP_{(b,n_3)}\cdot\vL\cdot\vP_{(b,n_2)}^\top,\]
we have that $\vR' \in \BD\ind{\frac{n_3}{b_3} \times \frac{n_2}{b_2}, n_3 \times n_2}$.
\end{theorem}


\begin{proof}
We recall  that multiplying an $m\times n$ matrix on the right (and left resp.) by $\vP_{(b,n)}^\top = \vP_{(\frac nb,n)}$ (and $\vP_{(b,m)}$ resp.) permutes the columns (and rows resp.) of the matrix according to $\sigma_{(b,n)}$ (and $\sigma_{(b,m)}$) respectively.\footnote{This uses the fact that $\parens{\sigma_{(b,n)}}^{-1}=\sigma_{(\frac nb,n)}$.} This implies that for any $0\le i,j<n$:
\begin{equation}
\label{eq:rect-L-permuted}
\vR'[\sigma_{(b,n_3)}(i),\sigma_{(b,n_2)}(j)]=\vL[i,j].
\end{equation}


Recall that in the notation of \cref{def:rect-Matrix L} we have $b_2=b_3=b$, so we are in the $b_2 \le b_3$ case.
To complete the proof, we will argue that $\vR'$ satisfies the two conditions in~\Cref{prop:rect-R-eqv-def}.\footnote{Note that we also need that the ratios of the row/column length to the row/column block sizes are the same; i.e., in our case we need that $\frac{n_3}{n_3 / b_3}=\frac{n_2}{n_2 / b_2}$, which is true because $b_2=b_3=b$.}

Towards this end, let $0\le i,j<n$ be arbitrary indices and further, define $i=\baseb{i_1}{i_0}{b}$ and $j=\baseb{j_1}{j_0}{b}$. Then note that $\sigma_{(b,n_3)}(i)=\baseb{i_0}{i_1}{\frac {n_3}{b}}$ and $\sigma_{(b,n_2)}(j)=\baseb{j_0}{j_1}{\frac {n_2}{b}}$.

By~\Cref{prop:rect-L-eqv-def}, we have that if $i_0\mod{b}\ne j_0$, then $\vL[i,j]=0$. Note  that since $i_0,j_0<b$ by definition, the condition $i_0\mod{b}\ne j_0$ is equivalent to saying $i_0\ne j_0$. Note that $i_0\ne j_0$ satisfies the pre-condition for base size $\frac {n_3}{b}\times \frac{n_2}{b}$ for indices $(\sigma_{(b,n_3)}(i),\sigma_{(b,n_2)}(j))$ in item~\ref{item:rect-zero-loc-R} in~\Cref{prop:rect-R-eqv-def}.  Then   by~\cref{eq:rect-L-permuted}, we have that $\vR'[\sigma_{(b,n_3)}(i),\sigma_{(b,n_2)}(j)]=0$, which satisfies item~\ref{item:rect-zero-loc-R} in~\Cref{prop:rect-R-eqv-def}.

Now consider the case that $i_0=j\mod b$, which by the observation in the above paragraph is the same as $i_0=j_0$. Then by item~\ref{item:rect-nonzero-loc-L} in~\Cref{prop:rect-L-eqv-def}, we have that $\vL[i,j]=\vS_{i_1,j_1}[i_0,j_0]$.  Note that $i_0= j_0$ satisfies the pre-condition for base size $\frac{n_3}{b}\times \frac{n_2}{b}$ for indices $(\sigma_{(b,n_3)}(i),\sigma_{(b,n_2)}(j))$ in item~\ref{item:rect-nonzero-loc-R} in~\Cref{prop:rect-R-eqv-def} if we define $\vR'_{i_0}\in\F^{\frac{n_3}{b}\times\frac{n_2}{b}}$ as follows:
\[\vR'_{i_0}[i_1,j_1]=\vS_{i_1,j_1}[i_0,j_0].\] 

Note that the above implies that 
\[\vR'=\diag\parens{\vR'_0,\dots,\vR'_{b-1}},\]
where $\vR'_{\cdot}$ is as defined in the above paragraph.
This means $\vR' \in \BD\ind{\frac{n_3}{b} \times \frac{n_2}{b}, n_3 \times n_2}$, since $\vR'$ has size $n_3 \times n_2$ and each block $\vR_{i_0}'$ is a matrix of size $\frac{n_3}{b} \times \frac{n_2}{b}$.
\end{proof}