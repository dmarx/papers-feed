\documentclass[nohyperref]{article}

\usepackage{etoolbox}
\newtoggle{arxiv}
\toggletrue{arxiv}

\usepackage{microtype}
\usepackage{graphicx}
\usepackage{subfigure}
\usepackage{booktabs} %
\usepackage{tabularx}
\usepackage{hyperref}
\usepackage{multirow}
\usepackage{setspace}
\usepackage{xfrac}
\newcommand{\theHalgorithm}{\arabic{algorithm}}

\iftoggle{arxiv}{
  \usepackage[numbers]{natbib}
}
{
\usepackage{icml2022}
}


\usepackage{amsmath}
\usepackage{amssymb}
\usepackage{mathtools}
\usepackage{amsthm}
\usepackage{relsize}
\usepackage{comment}
\usepackage{algorithm}
\usepackage{algorithmic}

\usepackage{subfigure}

\usepackage[capitalize]{cleveref}

\usepackage[inline]{enumitem}


\usepackage[textsize=tiny]{todonotes}


\theoremstyle{plain}
\newtheorem{theorem}{Theorem}
\newtheorem{lemma}{Lemma}[section]
\newtheorem{proposition}[lemma]{Proposition}
\newtheorem{corollary}[lemma]{Corollary}
\theoremstyle{definition}
\newtheorem{definition}[lemma]{Definition}
\newtheorem{assumption}[lemma]{Assumption}
\theoremstyle{remark}
\newtheorem{remark}[lemma]{Remark}


\newcommand{\diag}{\mathrm{diag}}
\newcommand{\norm}[1]{\left\|{#1}\right\|} %
\newcommand{\rank}{\mathrm{rank}}
\newcommand{\B}{\mathcal{B}}
\newcommand{\M}{\mathcal{M}}
\newcommand{\BS}{\B^*}
\newcommand{\BBS}{\B\B^*}
\newcommand{\BSB}{\B^*\B}
\newcommand{\MMS}{\M\M^*}
\newcommand{\MSM}{\M^*\M}
\newcommand{\BF}{\mathcal{B}\mathcal{F}}
\newcommand{\BD}{\mathcal{B}\mathcal{D}}
\newcommand{\DB}{\mathcal{D}\mathcal{B}}
\newcommand{\ind}[1]{^{(#1)}}

\newcommand{\vzero}{\mathbf{0}}
\newcommand{\vone}{\mathbf{1}}
\newcommand{\va}{\mathbf{a}}
\newcommand{\vb}{\mathbf{b}}
\newcommand{\vc}{\mathbf{c}}
\newcommand{\vd}{\mathbf{d}}
\newcommand{\ve}{\mathbf{e}}
\newcommand{\vf}{\mathbf{f}}
\newcommand{\vg}{\mathbf{g}}
\newcommand{\vh}{\mathbf{h}}
\newcommand{\vi}{\mathbf{i}}
\newcommand{\vj}{\mathbf{j}}
\newcommand{\vk}{\mathbf{k}}
\newcommand{\vl}{\mathbf{l}}
\newcommand{\vm}{\mathbf{m}}
\newcommand{\vn}{\mathbf{n}}
\newcommand{\vo}{\mathbf{o}}
\newcommand{\vp}{\mathbf{p}}
\newcommand{\vq}{\mathbf{q}}
\newcommand{\vr}{\mathbf{r}}
\newcommand{\vs}{\mathbf{s}}
\newcommand{\vt}{\mathbf{t}}
\newcommand{\vu}{\mathbf{u}}
\newcommand{\vv}{\mathbf{v}}
\newcommand{\vw}{\mathbf{w}}
\newcommand{\vx}{\mathbf{x}}
\newcommand{\vy}{\mathbf{y}}
\newcommand{\vz}{\mathbf{z}}
\newcommand{\vA}{\mathbf{A}}
\newcommand{\vB}{\mathbf{B}}
\newcommand{\vC}{\mathbf{C}}
\newcommand{\vD}{\mathbf{D}}
\newcommand{\vE}{\mathbf{E}}
\newcommand{\vF}{\mathbf{F}}
\newcommand{\vG}{\mathbf{G}}
\newcommand{\vH}{\mathbf{H}}
\newcommand{\vI}{\mathbf{I}}
\newcommand{\vJ}{\mathbf{J}}
\newcommand{\vK}{\mathbf{K}}
\newcommand{\vL}{\mathbf{L}}
\newcommand{\vM}{\mathbf{M}}
\newcommand{\vN}{\mathbf{N}}
\newcommand{\vO}{\mathbf{O}}
\newcommand{\vP}{\mathbf{P}}
\newcommand{\vQ}{\mathbf{Q}}
\newcommand{\vR}{\mathbf{R}}
\newcommand{\vS}{\mathbf{S}}
\newcommand{\vT}{\mathbf{T}}
\newcommand{\vU}{\mathbf{U}}
\newcommand{\vV}{\mathbf{V}}
\newcommand{\vW}{\mathbf{W}}
\newcommand{\vX}{\mathbf{X}}
\newcommand{\vY}{\mathbf{Y}}
\newcommand{\vZ}{\mathbf{Z}}

\newcommand{\cA}{\mathcal{A}}
\newcommand{\cB}{\mathcal{B}}
\newcommand{\cC}{\mathcal{C}}
\newcommand{\cD}{\mathcal{D}}
\newcommand{\cF}{\mathcal{F}}
\newcommand{\cG}{\mathcal{G}}
\newcommand{\cH}{\mathcal{H}}
\newcommand{\cI}{\mathcal{I}}
\newcommand{\cK}{\mathcal{K}}
\newcommand{\cL}{\mathcal{L}}
\newcommand{\cM}{\mathcal{M}}
\newcommand{\cN}{\mathcal{N}}
\newcommand{\cP}{\mathcal{P}}
\newcommand{\cS}{\mathcal{S}}
\newcommand{\cT}{\mathcal{T}}
\newcommand{\cX}{\mathcal{X}}

\newcommand{\R}{\mathbb{R}}
\newcommand{\F}{\mathbb{F}}

\newcommand{\Z}{\mathbb{Z}}
\newcommand{\ep}{\epsilon}
\newcommand{\g}{\gamma}
\newcommand{\Y}{\infty}
\newcommand{\f}[2]{\dfrac{#1}{#2}}
\newcommand{\ff}[2]{\tfrac{#1}{#2}}
\newcommand{\lm}[2]{\lim_{#1\rightarrow #2}}
\newcommand{\de}{\delta}
\newcommand{\T}{\theta}
\newcommand{\tm}{\times}
\newcommand{\su}[2]{\mathlarger{\sum\limits_{#1}^{#2}}}
\newcommand{\pd}[2]{\mathlarger{\prod\limits_{#1}^{#2}}}
\renewcommand{\sec}[1]{\section*{#1}}
\newcommand{\st}[1]{\subsection*{#1}}
\newcommand{\sst}[1]{\subsubsection*{#1}}
\renewcommand{\b}{\textbf}
\newcommand{\lessim}{\lesssim}
\newcommand{\E}{\mathbb{E}}
\newcommand{\p}{\partial}
\newcommand{\lt}{\left(}
\newcommand{\rt}{\right)}
\newcommand{\Lt}{\left[}
\newcommand{\Rt}{\right]}
\newcommand{\A}{\alpha}
\renewcommand{\b}{\beta}
\newcommand{\I}[2]{\mathlarger{\int_{#1}^{#2}}}
\newcommand{\G}{\nabla}
\newcommand{\Om}{\Omega}
\newcommand{\y}{\tau}
\newcommand{\K}{\mathcal{K}}
\newcommand{\C}{\mathbb{C}}
\newcommand{\om}{\omega}
\newcommand{\D}{\Delta}
\newcommand{\N}{\mathcal{N}}
\newcommand{\ra}{\rightarrow}
\newcommand{\Ra}{\Rightarrow}
\newcommand{\floor}[1]{\left\lfloor#1\right\rfloor}
\newcommand{\ceil}[1]{\left\lceil#1\right\rceil}
\newcommand{\ip}[1]{\left\langle#1\right\rangle}
\renewcommand{\mod}{\text{ mod }}
\newcommand{\sign}{\text{sign}}
\newcommand{\defeq}{:=}
\renewcommand{\a}{\bar{a}}
\newcommand{\MM}{\widetilde{\vM}}

\DeclareMathOperator*{\argmin}{argmin}


\newtoggle{comment}
\toggletrue{comment}
\togglefalse{comment} %

\iftoggle{comment}{
\newcommand{\Beidi}[1]{{\color{orange} [Beidi: {#1}]}}
\newcommand{\Tri}[1]{{\color{cyan} [Tri: {#1}]}}
\newcommand{\nimit}[1]{{\color{red} [Nimit: {#1}]}}
\newcommand{\micp}[1]{{\color{blue!70} [Michael: {#1}]}}
\newcommand{\arjun}[1]{{\color{green} [Arjun: {#1}]}}
}{
\newcommand{\Beidi}[1]{}
\newcommand{\Tri}[1]{}
\newcommand{\nimit}[1]{}
\newcommand{\micp}[1]{}
\newcommand{\arjun}[1]{}
}

\iftoggle{arxiv}{
  \setlength{\textwidth}{6.5in}
  \setlength{\textheight}{9in}
  \setlength{\oddsidemargin}{0in}
  \setlength{\evensidemargin}{0in}
  \setlength{\topmargin}{-0.5in}
  \newlength{\defbaselineskip}
  \setlength{\defbaselineskip}{\baselineskip}
  \setlength{\marginparwidth}{0.8in}
}{
\usepackage[compact]{titlesec}
\titlespacing{\section}{0pt}{*1.0}{*0}
\titlespacing{\subsection}{0pt}{*0}{*0}
\titlespacing{\subsubsection}{0pt}{*0}{*0}

\usepackage[subtle, mathdisplays=tight, charwidths=tight, leading=normal]{savetrees}


\def\setstretch#1{\renewcommand{\baselinestretch}{#1}}
\setstretch{0.99}
\addtolength{\parskip}{-0.3pt}
}


\iftoggle{arxiv}{
\title{Monarch: Expressive Structured Matrices for Efficient and Accurate Training}
\usepackage{authblk}
\author[1]{Tri Dao}
\author[1]{Beidi Chen}
\author[1]{Nimit Sohoni}
\author[1]{Arjun Desai}
\author[1]{Michael Poli}
\author[2]{Jessica Grogan}
\author[3]{Alexander Liu}
\author[3]{Aniruddh Rao}
\author[2]{Atri Rudra}
\author[1]{Christopher R{\'e}}
\affil[1]{Stanford University}
\affil[2]{University at Buffalo, SUNY}
\affil[2]{University of Michigan}
\affil[ ]{\texttt{\{trid,beidic,nims,arjundd,poli\}@stanford.edu}, \texttt{\{jrgrogan,atri\}@buffalo.edu}, \texttt{\{avliu,anrao\}@umich.edu}, \texttt{chrismre@cs.stanford.edu}}
}{
\icmltitlerunning{Monarch}
}

\begin{document}


\iftoggle{arxiv}{
  \maketitle
}{
\twocolumn[
\icmltitle{Monarch: Expressive Structured Matrices for Efficient and Accurate Training}



\icmlsetsymbol{equal}{*}

\begin{icmlauthorlist}
\icmlauthor{Firstname1 Lastname1}{equal,yyy}
\icmlauthor{Firstname2 Lastname2}{equal,yyy,comp}
\icmlauthor{Firstname3 Lastname3}{comp}
\icmlauthor{Firstname4 Lastname4}{sch}
\icmlauthor{Firstname5 Lastname5}{yyy}
\icmlauthor{Firstname6 Lastname6}{sch,yyy,comp}
\icmlauthor{Firstname7 Lastname7}{comp}
\icmlauthor{Firstname8 Lastname8}{sch}
\icmlauthor{Firstname8 Lastname8}{yyy,comp}
\end{icmlauthorlist}

\icmlaffiliation{yyy}{Department of XXX, University of YYY, Location, Country}
\icmlaffiliation{comp}{Company Name, Location, Country}
\icmlaffiliation{sch}{School of ZZZ, Institute of WWW, Location, Country}

\icmlcorrespondingauthor{Firstname1 Lastname1}{first1.last1@xxx.edu}
\icmlcorrespondingauthor{Firstname2 Lastname2}{first2.last2@www.uk}

\icmlkeywords{Machine Learning, ICML}

\vskip 0.3in
]



\printAffiliationsAndNotice{\icmlEqualContribution} %
}

\begin{abstract}
  Large neural networks excel in many domains, but they are expensive to train and fine-tune.
  A popular approach to reduce their compute/memory requirements is to replace dense weight matrices with structured ones (e.g., sparse, low-rank, Fourier transform).
  These methods have not seen widespread adoption (1) in end-to-end training due to
  unfavorable efficiency--quality tradeoffs, and
  (2) in dense-to-sparse fine-tuning due to lack of tractable algorithms to
  approximate a given dense weight matrix.
  To address these issues, we propose a class of matrices (Monarch) that is \emph{hardware-efficient} (they are parameterized as products of two block-diagonal matrices for better hardware utilization) and \emph{expressive} (they can represent many commonly used transforms).
  Surprisingly, the problem of approximating a dense weight matrix with a Monarch matrix, though nonconvex, has an analytical optimal solution.
  These properties of Monarch matrices unlock new ways to train and fine-tune sparse and dense models.
  We empirically validate that Monarch can achieve favorable accuracy–efficiency tradeoffs in several end-to-end sparse training applications: speeding up ViT and GPT-2 training on ImageNet classification and Wikitext-103 language modeling by 2$\times$ with comparable model quality, and reducing the error on PDE solving and MRI reconstruction tasks by 40\%.
  In sparse-to-dense training, with a simple technique called ``reverse sparsification,'' Monarch matrices serve as a useful intermediate representation to speed up GPT-2 pretraining on OpenWebText by 2$\times$ without quality drop.
  The same technique brings 23\% faster BERT pretraining than even the very optimized implementation from Nvidia that set the MLPerf 1.1 record.
  In dense-to-sparse fine-tuning, as a proof-of-concept, our Monarch approximation algorithm speeds up BERT fine-tuning on GLUE by 1.7$\times$ with comparable accuracy.
\end{abstract}

\section{Introduction}
%
Neural network (NN) learning has underpinned state of the art empirical
results in numerous applied machine learning tasks (see for
instance~\cite{krizhevsky2012imagenet,lecun2015deep}). Nonetheless, neural
network learning remains rather poorly understood in several regards.
Notably, it remains unclear why training algorithms find good weights, how
learning is impacted by the network architecture and activations, what is
the role of random weight initialization, and how to choose a concrete
optimization procedure for a given architecture.

We start by analyzing the expressive power of NNs subsequent to the random
weight initialization. The motivation is the empirical success of training
algorithms despite inherent computational intractability, and the fact that
they optimize highly non-convex objectives with potentially many local minima.
Our key result shows that random initialization already positions learning
algorithms at a good starting point. We define an object termed a {\em
computation skeleton} that describes a distilled structure of feed-forward
networks. A skeleton induces a family of network architectures along with a
hypothesis class $\ch$ of functions obtained by certain non-linear
compositions according to the skeleton's structure.  We show that the
representation generated by random initialization is sufficiently rich to
approximately express the functions in $\ch$. Concretely, all functions in
$\ch$ can be approximated by tuning the weights of the last layer, which is
a convex optimization task.

In addition to explaining in part the success in finding good weights, our
study provides an appealing perspective on neural network learning.  We
establish a tight connection between network architectures and their dual
kernel spaces. This connection generalizes several previous constructions
(see Sec~\ref{sec:related}). As we demonstrate, our dual view gives rise to
design principles for NNs, supporting current practice and suggesting
new ideas. We outline below a few points.

\begin{itemize}

\item Duals of convolutional networks appear a more suitable fit for
	vision and acoustic tasks than those of fully connected networks.

\item Our framework surfaces a principled initialization scheme. It is
	very similar to common practice, but incorporates a small correction.

\item By modifying the activation functions, two consecutive fully connected
	layers can be replaced with one while preserving the network's dual kernel.

\item The ReLU activation, i.e. $x \mapsto \max(x,0)$, possesses favorable
	properties. Its dual kernel is expressive, and it can be well approximated by
	random initialization, even when the initialization's scale is moderately
	changed.

\item As the number of layers in a fully connected network becomes very
	large, its dual kernel converges to a degenerate form for any non-linear
	activation.

\item Our result suggests that optimizing the weights of the last layer can
	serve as a convex proxy for choosing among different architectures prior
	to training. This idea was advocated and tested empirically
	in~\cite{saxe2011random}.

\end{itemize}


\section{Related Work}
\label{sec:related_work}

\paragraph{Attention variants and distributed attention}
Ever since attention became popular with the Transformer
architecture~\citep{vaswani2017attention}, there has been a large body of work
on approximating attention to scale it to longer sequences.
These approximation methods can generally be categorized into two classes:
sparse and low-rank.
Sparse attention only computes some entries of the attention matrix ($\mathrm{softmax}(\vQ
\vK^T)$) and assumes that other entries are zero.
Different methods have different ways of choosing which entries should be zero,
either with a fixed pattern~\citep{child2019generating}, with a sliding
window~\citep{beltagy2020longformer}, or with a dynamic pattern through
hashing~\citep{kitaev2020reformer} or routing~\citep{roy2020efficient}.
The low-rank approach instead assumes that the attention matrix has a low-rank
structure, and apply a pointwise nonlinearity to the query and
key~\citep{katharopoulos2020transformers} with random
projection~\citep{choromanski2021rethinking, peng2021random, xiong2021nystromformer}.
One can also combine the sparse and low-rank approximation for better
quality~\citep{zaheer2020bigbird,scatterbrain}.
However, these approximation methods typically do not offer the same model
quality as standard attention~\citep{tay2020efficient}, and so most large-scale
models do not employ these techniques.

There are other variants of attention aimed at reducing the size of the KV cache
to improve inference efficiency. Multi-query attention~\citep{shazeer2019fast} and grouped query
attention~\citep{ainslie2023gqa} tie different heads of $\vK$ and $\vV$, and
multiple query heads interact with the same key and value head.
Multi-head latent attention~\citep{deepseekv2} parameterizes the $\vK$ and $\vV$
as low-rank projections of a shared matrix to further reduce the KV cache size.
However, all of these approaches do not change the core computation
$\mathrm{softmax}(\vQ \vK^T) \vV$ during training and simply change how $\vQ, \vK, \vV$ are
obtained.
As a result, any efficiency or accuracy improvement to the standard attention
computation benefits these methods.

To extend to even longer context, attention computation can be distributed
across multiple GPUs.
Methods such as Ring attention~\citep{liu2023ring,liu2024world} and
variants~\citep{brandon2023striped} can reach a context length of up to 1
million.
They use \fa (or \faa) as a primitive, and so the improvement from \fat would
benefit these distributed attention methods as well.

\paragraph{Alternative architectures}
Motivated by the limitations of attention, a variety of alternative
architectures have been proposed.
They build on the connection between linear
attention~\citep{katharopoulos2020transformers} and recurrent neural networks
(RNNs).
RWKV~\citep{peng2023rwkv}, H3~\citep{dao2023hungry}, MEGA~\citep{ma2023mega},
Retnet~\citep{sun2023retentive}  enhance the expressivity of the simple
cumulative sum in linear attention with more sophisticated recurrences.
Mamba~\citep{gu2023mamba} and xLSTM~\citep{beck2024xlstm} use learnable
weighting for the recurrence and can match the quality of Transformers in
language modeling at small or medium scale.
These approaches can be connected to generalizations of linear attention through
the lens of the structure of the token-mixing matrix~\citep{dao2024transformers}.
These models have started to see some traction, seeing usage in some medium to
large-scale models such as Jamba~\citep{jamba}, Zamba~\citep{zamba},
Megalodon~\citep{ma2024megalodon}, and Mamba2-hybrid~\citep{waleffe2024empirical}.
For the highest quality, these SSM- and RNN-based models still employ
many layers of attention.
We expect that techniques to speed up attention presented in this work will be
useful to speedup these alternative architectures.

\paragraph{Low-precision attention}
Quantization is a promising approach to speed up attention, but they have mostly
focused on reducing the space for KV cache for inference efficiency.
QuIP~\citep{chee2024quip} and QuIP\#\citep{tseng2024quip} use incoherent processing to reduce the quantization,
and we adapted this technique for FP8 \fat.
Recent work suggests that for inference the KV cache is highly compressible down to 4-, 3-, or
even 2-bits~\citep{hooper2024kvquant, liu2024kivi}.
However, quantization during training is still challenging as higher precision
is typically required for stable training.

\paragraph{Hardware-aware Algorithms}
Our work presented in this paper focuses on the micro-architecture
specific tuning to leverage new instruction sets and adopt a natively
asynchronous programming model. There are other orthogonal axes for
hardware-aware algorithm co-design being explored.
A recent example of this is LeanAttention~\citep{sanovar2024-leanattention},
which recognizes the poor GPU occupancy and high memory bandwidth requirements
of the sequential token generation phase as primary bottlenecks for inference
and optimizes it via a smarter load balancing strategy similar to Stream-K
load balancing~\citep{streamk} to achieve nearly peak occupancy.
There is a large literature on optimizing GEMM for specific hardware that employs
many of the same techniques.
As an example, \citet{abdel2016batched} presents a high performance batched GEMM kernel on
K40c Graphics Processing Units (GPU) for both fixed and variable sizes,
proposing specialized GEMM designs
and a comprehensive autotuning process to deliver state-of-the-art 
performance.



\section{Theoretical Properties of Sigmoid Attention}
\label{sec:theory}
We analyze $\sigmoidattn$, with two objectives: (1) showing that a transformer architecture remains a universal function approximator when $\sigmoidattn$ replaces $\softmaxattn$, and (2) recovering a measure of regularity of $\sigmoidattn$ by computing its Lipschitz constant.

\subsection{Are Transformers with Sigmoid Attention Universal Approximators?}
\label{sec:ufa}
\cite{Yun_UAP} demonstrate that classical transformers can approximate continuous sequence-to-sequence functions to arbitrary precision, a property known as the \emph{Universal Approximation Property} (UAP). UAP is highly desirable as it provides proof of an architecture's generalizability and representation capability.
As $\sigmoidattn$ modifies the transformer architecture, it is crucial to theoretically guarantee that this modification does not impact the representation capability and that UAP is retained. We provide this guarantee with the following theorem.
\begin{theorem}[UAP for $\sigmoidattn$]
    \label{thm::UAP}
    We denote with $\mathcal{T}^{h,d_v,r}_{\sigma}$ the class of transformer networks obtainable by combining an arbitrary number of $\sigmoidattn$ layers (each of $h$ heads of dimension $d_v$) followed by FFN layers of hidden dimension $r$.
    For any given continuous, permutation-equivariant function $f:\Omega\subset\mathbb{R}^{n\times d}\to\mathbb{R}^{n\times d}$ with compact support $\Omega$, and for any arbitrarily small error $\varepsilon$, there exists a transformer network $g\in\mathcal{T}_\sigma^{4,1,4}$ such that
    \begin{equation}
        \left(\int_{\Omega}\|f(\bb{X})-g(\bb{X})\|^p_p d\bb{X}\right)\leq\varepsilon,\qquad\text{for}\quad 1\leq p<\infty.
    \end{equation}
\end{theorem}
\Cref{thm::UAP} is the exact counterpart of \cite[Thm.~2]{Yun_UAP}, which shows UAP for classical transformers. Our proof largely follows the same path, an outline of the original proof provided in \cref{app:UAP_proof}. Here, we present an overview of the main adaptations required to prove \cref{thm::UAP} for $\sigmoidattn$, with further details in \cref{sec::proof_modified_sigmoid,sec::proof_contextual_mapping_top}.

\paragraph{Sigmoid Attention layers can implement contextual mappings:} A key step in proving \cref{thm::UAP} is showing that, even with $\sigmoidattn$, a sequence of transformer blocks can implement a \emph{Contextual Mapping} \cite[Def.~3.1]{Yun_UAP}. A contextual mapping characterizes a function that maps each input sequence element to an output \emph{uniquely} dependent on the \emph{whole} sequence. This property allows a transformer to capture and store global context within each token, even if each layer only performs pairwise comparisons. Subsequent layers can then use this global information to map individual tokens to the correct output, ultimately approximating any arbitrary sequence-to-sequence function.

In \cite{Yun_UAP}, the contextual mapping is assembled by modifying individual transformer blocks: each block is tuned to react to a specific input token. By stacking a sequence of these blocks, a transformer can be turned into an accumulator, mapping a given input token sequence to a unique global index. This outcome is achieved via a \emph{selective shift layer} \cite[App.~B.5]{Yun_UAP}:
\begin{equation}
    \Psi(\bb{X};b,b')_{i,1}\coloneqq \begin{cases}
        \max_k \bb{X}_{k,1}-\min_k\bb{X}_{k,1}&\text{if}\quad b<\bb{X}_{i,1}<b'\\
        0&\text{otherwise},
    \end{cases}
    \label{eqn::shift_operation_original}
\end{equation}
and can be approximated using classic attention.
Although $\sigmoidattn$ cannot directly approximate~\cref{eqn::shift_operation_original}, our accumulator definition relies on an equivalent selective shift operation:
\begin{equation}
    \Psi_\sigma(\bb{X};b,b')_{i,1}\coloneqq\begin{cases}
        \sum_{k:\bb{X}_{k,1}> b'} \bb{X}_{k,1} &\text{if}\quad b<\bb{X}_{i,1}<b' \\
        0 &\text{otherwise},
    \end{cases}
    \label{eqn::shift_operation_ours}
\end{equation}
which can be approximated by $\sigmoidattn$ (described in \cref{sec::proof_modified_sigmoid}). In~\cref{sec::proof_contextual_mapping}, we show that~\cref{eqn::shift_operation_ours} shares similar properties with~\cref{eqn::shift_operation_original}, allowing us to use the original proof framework in \cite{Yun_UAP} and demonstrate that UAP holds in our case as well.

Our proof is largely equivalent to that in \cite{Yun_UAP}, with two relevant differences: to approximate \cref{eqn::shift_operation_ours}, we require $\sigmoidattn$ with \textit{at least four heads} and shifts included in both query and key definitions. In contrast, $\softmaxattn$ requires \textit{at least two heads} to approximate~\cref{eqn::shift_operation_original}, with shifts only in the query definition. However, this is primarily a theoretical requirement for the proof and does not affect performance. Notably, the total number of parameters required by both architectures for the approximation follows the same tight scaling of \cite{Yun_UAP}.






\subsection{Regularity of Sigmoid Attention}
\label{sec:regularity}
As with any layer in a neural network, the regularity of $\sigmoidattn$ is important to study, as it gives insights into the robustness of the corresponding network and the ease of optimizing it.
The most standard way to quantify the regularity of a layer function $\phi$ is to compute its \emph{Lipschitz constant} over a set $\mathcal{X}$, that is a constant $C>0$ such that for all $\mX, \mY\in \mathcal{X}$, it holds $\|\phi(\mX) - \phi(\mY)\|\leq C \|\mX - \mY\|$, where $\|\cdot\|$ is the standard Frobenius norm.
The \emph{local} Lipschitz constant is the spectral norm of the Jacobian of $\phi$ at $\mX$.
The two are related: the Lipschitz constant of $\phi$ over $\mathcal{X}$ is the greatest local Lipschitz constant for all $\mX\in \mathcal{X}$.
We turn to the theorem giving the regularity of $\sigmoidattn$:
\begin{theorem}
\label{thm:regularity}
    Define $A = \{\langle \mW_q \vx_i \mW_k \vx_j\rangle|,\enspace i, j\in \{1,\dots,n\}\}\subset\mathbb{R}$ the set of attention weights,  and the scaled activation norms $\sigma_{\infty} = n\times\sup_{u\in A} |\sigma(u)|$ and $\sigma'_{\infty} = n\times \sup_{u\in A} |\sigma'(u)|$.
    Then, the Jacobian of $\sigmoidattn$ at $\mX = (\vx_1, \dots, \vx_n)$ has a spectral norm of at most:
    \begin{equation}
        \|\mW_v\|_2\left(\sigma_{\infty} + 2\sigma'_{\infty} \|\mW_q^T \mW_k\|_2\left(\frac1n\sum_{i=1}^n\|\vx_i\|_2^2\right)\right).
    \end{equation}
\end{theorem}
The proof is found in \cref{app:lipschitz_proof}.
In $\sigmoidattn$, if we assume that the attention weights $\langle \mW_q \vx_i, \mW_k \vx_j\rangle$ are all bounded by a constant $\mu$ --- this is true, e.g., if the activations are bounded --- we get $\sigma_{\infty}\leq \exp(\mu)$ and $\sigma'_{\infty}\leq\exp(\mu)$ thanks to the choice of $b = -\log(n)$.
The bound in \cref{thm:regularity} depends only on the \emph{average} squared-norm of the input sequence $\vx_i$, while classical results for the study of attention all rely on the largest value of $\|\vx_i\|^2_2$~\citep{kim2021lipschitz,castin2023understanding}. 
This is another consequence of the simplicity of sigmoid attention and is due to the removal of the normalizing constant in $\softmaxattn$.
Our result implies that if all $\vx_i$ are within a ball of radius $R$ then the Lipschitz constant of $\sigmoidattn$ grows at most like $R^2$, but it is stronger since we can apply this to unbounded distributions $\vx_i$; it matters only that the second moment is bounded.
This result contrasts sharply with the bounds obtained for $\softmaxattn$: \citet[Thm.~3.4.]{castin2023understanding} show that there exists a sequence $\mX = (\vx_1, \dots, \vx_n)$ with $\|\vx_i\|_2\leq R$ for all $i$ such that the spectral norm of the Jacobian of $\attn$ at $\mX$ is at least $cR^2\exp(cR^2)$ for some constant $c>0$.
On the other hand, our bound scales in $R^2$: this means that the local Lipschitz constant of $\sigmoidattn$ is much lower than the worst local Lipschitz constant of $\softmaxattn$.


\section{Retrieval with Synchronised Graph Expansion}
\label{sec:graph_retrieval}

\def\Tqinit{\mathbf{T}_\mathbf{q}}


\begin{figure}[thbp]
  \includegraphics[width=\columnwidth]{figures/gear-sys-fig.pdf}
  \caption{\label{fig:system_diagram}System Architecture}
\end{figure}

% Start: Zhili --------------------------


Given an input query $\mathbf{q}$, let $\mathbf{C}_\mathbf{q}' = h^k_{\text{base}}\left( \mathbf{q}, {\mathbf{C}}\right )$  be a list of passages returned by the base retriever\footnote{The choice of a base retriever within our framework is flexible, without requiring any multi-hop capabilities.}.
Given this initially retrieved list of passages, $\mathbf{C}_\mathbf{q}'$, our goal is to derive relevant multi-hop contexts (passages) by retrieving a sub-graph of triples that interconnect their source passages. There are two challenges for materialising such sub-graph retrieval: \begin{inparaenum}[(i)]\item how to locate initial triples (i.e. starting nodes) $\Tqinit$, and \item how to expand the graph based on initial triples while reducing the search space\end{inparaenum}. The following sections address these challenges respectively, within \gear.



\subsection{Knowledge Synchronisation}
\label{subsection:knowledge_syncro}
\def\linkTriple{\texttt{tripleLink}}

We describe a knowledge \textbf{Sync}hronisation (\textbf{Sync}) process for locating initial nodes for graph expansion. We first employ an LLM to \texttt{read} $\mathbf{C}_\mathbf{q}'$ (see Appendix~\ref{subsec:online_retrieval_prompts}) and summarise knowledge triples that can support answering the current query $\mathbf{q}$, as defined:
\begin{align}
    \mathbf{T}_\mathbf{q}' = \texttt{read}\left (\mathbf{C}_\mathbf{q}', \mathbf{q}\right ).
    \label{eq:proximal_read}
\end{align}
$\mathbf{T}_\mathbf{q}'$ is a collection of triples to which we refer as \textit{proximal triples}. Initial nodes $\Tqinit$ for graph expansion can then be identified by linking each triple in $\mathbf{T}_\mathbf{q}'$ to a triple in $\mathbf{T}$, using the \linkTriple{} function:
\begin{align}
    \Tqinit =\left \{t_i | t_i = \linkTriple(t_i') ~ \forall t_i' \in \mathbf{T}_\mathbf{q}'\right \}.
\end{align}
The implementation of \linkTriple{} can vary. However, in this paper we consider it to be simply retrieving the most similar triple from $\mathbf{T}$.



\begin{algorithm}[ht]
\textbf{Input:} $\mathbf{q}$: query \\
\hspace*{3em} $b$: beam size \\
\hspace*{3em} $l$: maximum length \\
\hspace*{3em} $\mathrm{score}(\cdot, \cdot)$: scoring function \\
\hspace*{3em} $\{t_1, t_2, \ldots, t_n\}$: initial triples \\
\hspace*{3em} $\gamma$: hyperparameter for diversity


\begin{algorithmic}[1]
\State $B_0 \gets [\;]$
\For{$t \in \{t_1, t_2, ..., t_n\}$}
    \State $s \gets \mathrm{score}(\mathbf{q}, [t])$
    \State $B_0.\mathrm{add}(\langle s, [t] \rangle)$
\EndFor

\State $B_0 \gets \mathrm{top}(B_0, b)$


\For{$i \in \{1, \dots, l - 1\}$}
    \State $B \gets [\;]$
    
    \For{$\langle s, T \rangle \in B_{i-1}$}
        \State $V \gets [\;]$

        \For{$t \in \mathrm{get\_neighbours}(T.\mathrm{last}())$}
            \If{$\mathrm{exists}(t, B_{i-1})$}
                \State \textbf{continue}
            \EndIf
            
            \State $s' \gets s + \mathrm{score}(\mathbf{q}, T \circ t)$ ~ \texttt{\# concat} 
            \State $V.\mathrm{add}(\langle s', T \circ t \rangle)$
        \EndFor

        \State $\mathrm{sort}(V, \mathrm{descending})$

        \For{$n \in \{0, \dots, V.\mathrm{length()} - 1\}$}
            \State $\langle s', T \circ t \rangle \gets V[n]$
            \State $s' \gets s' \times e^{- \frac{\mathrm{min}(n, \gamma)}{\gamma}}$
            \State $B.\mathrm{add}(\langle s', T \circ t \rangle)$
        \EndFor
        
    \EndFor
    \State $B_i \gets \mathrm{top}(B, b)$
    
\EndFor

\State \Return $B_i$
\end{algorithmic}

\caption{Diverse Triple Beam Search}
\label{alg:beam_search}
\end{algorithm}

\subsection{Diverse Triple Beam Search}

We borrow the idea of constructing reasoning triple chains \cite{Fang2024} for expanding the graph, and present a retrieval algorithm: \textit{Diverse Triple Beam Search} (see Alg.~\ref{alg:beam_search}). 

We maintain top-$b$ sequences (beams) of triples and the scores at each step are determined by a scoring function. In this paper, we focus on leveraging a dense embedding model to compute the cosine similarity between embeddings of the query and a candidate sequence of triples, leaving other implementations of the scoring function for future work (see Section~\ref{sec:limitations}).

Considering all possible triple extensions at each step, in a Viterbi decoding fashion, would be intractable due to the size of $\mathbf{T}$. Consequently, we define the neighbourhood of a triple as the set of triples with shared head or tail entities (i.e. $\mathrm{get\_neighbours}$ in Alg.~\ref{alg:beam_search}). During each expansion step, we only consider neighbours of the last triple in the sequence, and avoid selecting previously visited triples (i.e. $\mathrm{exists}$ in Alg.~\ref{alg:beam_search}) to further reduce the search space.

While regular beam search can reduce the search space, it is prone to producing high-likelihood sequences that differ only slightly from one another \cite{Ippolito2019, Vijayakumar2018}. Our algorithm increases the diversity across beams to improve the recall for retrieval. In detail, for each beam, we sort candidate sequences extended from that beam in descending order, and weight their scores based on their relative positions. Candidate sequences that are ranked lower, within a beam, will receive smaller weights. Consequently, the resulting top-$b$ beams at each step are less likely to share the same starting sequence. 

The top-$b$ returned sequences are flattened in a breadth-first order. Each triple in the resulting list is then mapped to its source passage. This alignment between triples and passages is described in more detail in Section~\ref{sec:preliminaries}. Let $\widetilde{\mathbf{C}}_\mathbf{q}$ be the list of unique passages after alignment. The output of our graph expansion is then given by the Reciprocal Rank Fusion (RRF) \cite{Cormack2009} of $\widetilde{\mathbf{C}}_\mathbf{q}$ and the initial $\mathbf{C}_\mathbf{q}'$ list of passages :
\begin{align}
    \mathbf{C}_{\mathbf{q}} = \mathrm{RRF}\left(\widetilde{\mathbf{C}}_\mathbf{q}, \mathbf{C}_\mathbf{q}'\right ).
\end{align}
We refer to this graph-based method of retrieving relevant passages as \textbf{Sync}ronised \textbf{G}raph \textbf{E}xpansion (\textbf{SyncGE}).


\section{Multi-step Extension}


While SyncGE can enhance a base retriever with multi-hop context, some queries inherently require multiple steps to gather all necessary evidence. We materialise \gear by incorporating an agent with multi-turn capabilities, capable of interacting with the graph-retriever described above. We focus on:
\begin{itemize}
\item maintaining a gist memory of proximal knowledge obtained throughout the different steps 
\item incorporating a similar synchronisation process 
that summarises retrieved passages in proximal triples to be stored in this multi-turn gist memory
\item determining if additional steps are needed for answering the original input question
\end{itemize}
%
Within this multi-turn setting, the original input question $\mathbf{q}$ is iteratively decomposed into simpler queries: $\mathbf{q}^{(1)}, \ldots, \mathbf{q}^{(n)}$, where $\mathbf{q}^{(1)} = \mathbf{q}$ and $n \in \mathbb{N}$ represents the number of the current step.
For each query $\mathbf{q}^{(n)}$, we use the graph retrieval method introduced in Section~\ref{sec:graph_retrieval} in order to retrieve relevant passages $\mathbf{C}_{\mathbf{q}^{(n)}}$.



\subsection{Gist Memory Constructor}
To facilitate the multi-step capabilities of our agent, we introduce a \textit{gist memory}, $\mathcal{G}^{(n)}$, which is used for storing knowledge as an array of proximal triples. At the beginning of the first iteration, the gist memory is empty. During the $n$-th iteration, similar to the knowledge synchronisation module explained in Section~\ref{subsection:knowledge_syncro}, we employ an LLM to read a collection of retrieved paragraphs $\mathbf{C}_{\mathbf{q}^{(n)}}$ and summarise their content with proximal triples:

\begin{align}
\mathbf{T}_{\mathbf{q}^{(n)}}^{\mathcal{G}} = 
\begin{cases} 
    \texttt{read}\left(\mathbf{C}_{\mathbf{q}^{(n)}}, \mathbf{q} \right), & \text{if } n = 1 \\
    \texttt{read}\left(\mathbf{C}_{\mathbf{q}^{(n)}}, \mathbf{q}\textcolor{blue}{, \mathcal{G}^{(n-1)}} \right), & \text{if } n \geq 2
\end{cases}
\label{eq:proximal_read_agent}
\end{align}


Apart from the first iteration where Eq.~\ref{eq:proximal_read} and ~\ref{eq:proximal_read_agent} are identical, the inclusion of the memory in the \texttt{read} operation differentiates the construction of proximal triples produced at the subsequent steps compared to the ones from Eq.~\ref{eq:proximal_read}. $\mathcal{G}^{(n)}$ maintains the aggregated content of proximal triples s.t. 
\begin{align}
\mathcal{G}^{(n)} = \left[ \mathbf{T}_{\mathbf{q}^{(1)}}^{\mathcal{G}}  \circ \cdots \circ \mathbf{T}_{\mathbf{q}^{(n)}}^{\mathcal{G}} \right],
\end{align}where $\circ$ defines the concatenation operation. The triple memory serves as a concise representation of all the accumulated evidence, up to the $n$-th step. 

We believe the process introduced by the \texttt{read} step along with the information storage paradigm served by the gist memory, aligns well with the communication between the hippocampus and neocortex. The combination of the two establishes the synergetic behaviour between our graph retriever and the LLM that we seek to achieve within \gear.



\subsection{Reasoning for Termination}
After $\mathcal{G}^{(n)}$ is updated, we check the sufficiency of the accumulated evidence, within it, for answering the original question. This is achieved with the following LLM reasoning step:
\begin{align}
\mathbf{a}^{(n)}, \mathbf{r}^{(n)}   = \texttt{reason}(\mathcal{G}^{(n)}, \mathbf{q}),
\end{align}
% We can also call it 'sufficiency' instead of 'answerability'. I do not really have a preference.
where $\mathbf{a}^{(n)}$ denotes the query's answerability given the available evidence in $\mathcal{G}^{(n)}$, and $\mathbf{r}^{(n)}$ represents the reasoning behind this determination. When the query is deemed answerable, the system concludes its iterative process.



\subsection{Query Re-writing}
The query re-writing process leverages an LLM that incorporates three key inputs: the original query $\mathbf{q}$, the accumulated memory, and crucially, the reasoning output $\mathbf{r}^{(n)}$ from the previous step. This process can be formally expressed as:
\begin{align}
\mathbf{q}^{(n+1)} = \texttt{rewrite}\left (\mathcal{G}^{(n)}, \mathbf{q}, \mathbf{r}^{(n)} \right),
\end{align}
where $\mathbf{q}^{(n+1)}$ represents the updated query, which serves as input for the retriever in the next iteration.\\
\subsection{After Termination}
\gear aims to return a single ranked list of passages. Given the final gist memory $\mathcal{G}^{(n)}$ upon termination, we link each proximal triple in $\mathcal{G}^{(n)}$ to a list of passages as follows:
\begin{align}
    \mathbf{C}_{t_j} = \texttt{passageLink}\left(t_j, k\right),
\end{align}
where $j \in \left \{1, \dots, \vert\mathcal{G}^{(n)}\vert \right \}$. Similar to \texttt{tripleLink}, \texttt{passageLink} is implemented by retrieving passages with a triple as the query (see Appendix~\ref{appendixpara:passage_link}). The final list of passages returned by \gear is the RRF of the resulting linked passages and passages retrieved across iterations:
\begin{align}
\mathbf{C}_\mathbf{q}^{(n)} = \mathrm{RRF}\big(&\mathbf{C}_{t_1}, \ldots,\mathbf{C}_{t_{\vert\mathcal{G}^{(n)}\vert}}, \nonumber\\
    &\mathbf{C}_{\mathbf{q}^{(1)}}, \ldots, \mathbf{C}_{\mathbf{q}^{(n)}} \big).
\end{align}

All relevant prompts for the \texttt{read}, \texttt{reason} and \texttt{rewrite} steps are provided in Appendix~\ref{subsec:online_retrieval_prompts}.


\section{Experiments}
\label{sec:experiments}
\begin{figure}[h]
\centering
\includegraphics[width=\textwidth]{figures/train_nll_softmax_vs_sigmoid_v4.pdf}
\caption{Train losses comparing $\sigmoidattn$ with $\softmaxattn$.}
\label{fig:summary_nll}
\end{figure}
To empirically validate $\sigmoidattn$, we evaluate across several domains: supervised image classification using vision transformers \citep{DBLP:conf/iclr/DosovitskiyB0WZ21}, self-supervised image representation learning with SimCLR \citep{DBLP:conf/icml/ChenK0H20, DBLP:conf/icml/ZhaiLLBR0GS23}, Bootstrap Your Own Latent (BYOL) \citep{DBLP:conf/nips/GrillSATRBDPGAP20, DBLP:conf/nips/BusbridgeRALDCW23} and Masked AutoEncoders (MAE) \citep{DBLP:conf/cvpr/HeCXLDG22} as well as automatic speech recognition (ASR) \citep{synnaeve2019end,conformer} and auto-regressive language modeling (LM) \citep{DBLP:conf/nips/BrownMRSKDNSSAA20}. We also validate sequence length generalization on TED-LIUM v3~\citep{hernandez2018ted} for ASR and in small scale synthetic experiments in \cref{sec:a_se_pair_repeat_prob}.
Across all these domains and algorithms, we demonstrate that $\sigmoidattn$ matches the performance of $\softmaxattn$ (\cref{fig:summary_nll,fig:test_top1_results}), while offering training and inference speed-ups as highlighted in \cref{sec:FlashSigmoidHardwareAwareImplementation}. Empirically we make the following observations:
\begin{enumerate}[itemsep=0pt,leftmargin=*]
    \item $\sigmoidattn$ is effective for vision tasks without a bias (except MAE), but relies on LayerScale to match the performance of the baseline $\softmaxattn$ (\cref{fig:imagenet_top_1_ablations}-a) in a hyper-parameter free manner.\footnote{\Cref{sec:layerscale_free_sigmoid} demonstrates that supervised vision tasks using $\sigmoidattn$ without LayerScale can match baseline $\softmaxattn$ performance by relying on \emph{learnable} scalar bias and temperature: $\{b, t\} \in \mathbb{R}$.} All results presented for $\softmaxattn$ also fairly add LayerScale unless specified.
    \item LM and ASR are sensitive to the initial norm $|| \sigma(\mQ \mK^T/\sqrt{d_{qk}}) \mV ||$. Modulation is required via (a) relative positional embeddings like ALiBi \citep{DBLP:conf/iclr/PressSL22}, which reduces the initial attention norm by shifting logit mass to the zero regime under $\sigmoidattn$, (b) appropriate initialization of $b$ to achieve the same effect -- enabling usage of any positional embedding.
\end{enumerate}

\begin{figure}[htbp]
    \centering
    \begin{minipage}{0.48\textwidth}
        \centering
        \includegraphics[width=\textwidth]{figures/attn_norm_seed1000001_softmax_rope_vs_softmax_alibi_vs_sigmoid_sincos.png}    
        \captionsetup{justification=centering}
        \caption{$\sigmoidattn$ with SinCos.}
        \label{fig:rope_vs_sincos}
    \end{minipage}\hfill
    \begin{minipage}{0.48\textwidth}
        \centering        
        \includegraphics[width=\textwidth]{figures/attn_norm_seed1000001_softmax_rope_vs_softmax_alibi_vs_sigmoid_rope.png}
        \captionsetup{justification=centering}
        \caption{$\sigmoidattn$ with RoPE.}
        \label{fig:rope_vs_rope}
    \end{minipage}
    \hfill
    \begin{minipage}{0.48\textwidth}
        \centering
        \includegraphics[width=\textwidth]{figures/attn_norm_seed1000001_softmax_rope_vs_softmax_alibi_vs_sigmoid_alibi.png}
        \captionsetup{justification=centering}
        \caption{$\sigmoidattn$ with ALiBi.}
        \label{fig:rope_vs_alibi}
    \end{minipage}\hfill
    \begin{minipage}{0.48\textwidth}
        \centering        
        \includegraphics[width=\textwidth]{figures/attn_norm_seed1000001_softmax_rope_vs_softmax_alibi_vs_sigmoid_rope_b=-10.png}
        \captionsetup{justification=centering}
        \caption{$\sigmoidattn$ with RoPE, $b=-10$.}
        \label{fig:rope_vs_rope_b-10}
    \end{minipage}  
    \vspace{-0.4cm}
\end{figure}

\subsection{Ablations}
\label{sec:ablations}
We begin with ablations to dissect the benefits of each of our introduced components. To gain intuition about $\sigmoidattn$, we developed a research-friendly auto-regressive (AR) LM training framework to measure all components of attention and validate the effects of LayerScale, LayerNorm applied to Q and K (QK norm), different positional embedding techniques, and initialization values for $b$.
\begin{figure}[h]
    \centering
    \begin{minipage}[t]{0.48\textwidth}
        \centering
        \includegraphics[width=\textwidth]{figures/lines=activation-cols=layerscale_with_log_n_or_max3std.pdf} 
        \caption{LR sensitivity LayerScale ablation.}
        \label{fig:layerscale_ablation}
    \end{minipage}%
    \hfill
    \begin{minipage}[t]{0.48\textwidth}
        \centering
        \includegraphics[width=\textwidth]{figures/lines=activation-cols=qknorm_with_log_n_or_max3std.pdf}
        \caption{LR sensitivity QK norm ablation.}
        \label{fig:qk_norm_ablation}
    \end{minipage}
\end{figure}
\begin{figure}[h]
    \centering
    \vspace{-0.2cm}
    \includegraphics[width=\textwidth]{figures/imagenet_ablations_top1.pdf}
    \caption{ImageNet1k ViT-B/16 classification. (a) $\sigmoidattn$ is robust without QK norm (+LayerScale, -QKNorm). Removing LayerScale reduces accuracy by 1.0\% (-LayerScale, +/-QKNorm). $n^{-\alpha}$ normalization \citep{wortsman2023replacing} underperforms without LayerScale. (b) $\sigmoidattn$ multi-query attention (MQA) \citep{DBLP:journals/corr/abs-1911-02150} with one head matches multi-head attention (MHA). (c) Sigmoid with LayerScale and QK norm performs comparably to other activations, except TanH. ReLU$^2$ \citep{DBLP:conf/icml/HuaDLL22} underperforms without LayerScale and QK norm.}
    \label{fig:imagenet_top_1_ablations}
    \vspace{-0.4cm}
\end{figure}
\paragraph{Mitigating Large Attention Norms} We train a single layer AR transformer block (E=3072, D\_FF=12288) on the realnews split of C4 \citep{DBLP:journals/jmlr/RaffelSRLNMZLL20}. We train for $2^{16}$ steps using a batch size of 6 and max sequence length of 4096 using a single cycle cosine learning rate (LR) schedule without weight decay. $\sigmoidattn$ initially underperformed $\softmaxattn$ when using absolute sinusoidal (SinCos) (\cref{fig:rope_vs_sincos}) or relative (\cref{fig:rope_vs_rope}) positional embeddings (PE), which we attribute to high initial attention Frobenius norms, $\lVert \sigma(\mQ \mK^T / \sqrt{d}) \mV \rVert$. A corresponding evolution of the attention distribution and sparsity can be seen in Appendix \cref{fig:attn_evolve} and \cref{fig:attn_metric_evolve} on a synthetic task.
To address these larger attention norms, we propose: (a) using ALiBi \citep{DBLP:conf/iclr/PressSL22} whose relative bias moves initial attention logit mass to the zero region under the sigmoid activation, producing equivalent train negative log-likelihoods (\cref{fig:rope_vs_alibi}); or (b) set the attention logit bias $b$ to a negative offset proportional to the sequence length, $b \propto -\ln n$ (see \cref{sec:attn_bias_ablation} for an ablation on $b$). This enables the usage of other PE techniques like RoPE~\citep{DBLP:journals/ijon/SuALPBL24} (\cref{fig:rope_vs_rope_b-10}). 
\paragraph{LayerScale} To validate the need for LayerScale, we follow \citet{DBLP:journals/corr/abs-2309-14322} to quantify the impact on stability.
All models are trained with RoPE with $b \propto -\ln n$, using AdamW  \citep{loshchilov2017decoupled} on the 
realnews split of C4 
with $(\beta_1,\beta_2)=(0.9, 0.95)$, $\eps=10^{-8}$,  $wd=0$, 
batch size 24, maximum token sequence length of 512 from the T5 tokenizer \citep{DBLP:journals/jmlr/RaffelSRLNMZLL20}, cosine LR schedule of $2^{14}$ steps including a linear warmup of $2^{10}$ steps. 
Models have 
$n_{\text{heads}}=\kappa$,
$n_{\text{layers}}=2\times \kappa$,
$d_{\text{model}}=64\times \kappa$ and
$d_{\text{feed-forward}}=256\times\kappa$
for a scaling value $\kappa\in\{1,2,4,8,16\}$
leading to models with $\{2.2, 4.9,15.0,67.0,440.0\}M$ trainable non-embedding parameters.
Following \citet{DBLP:journals/corr/abs-2309-14322},
we sweep learning rates
$\eta\in \{3\times 10^{-4}, 1\times 10^{-3}, 3\times 10^{-3}, 1\times 10^{-2}, 3\times 10^{-2}, 1\times 10^{-1}, 3\times 10^{-1}\}$.
LR sensitivity is defined as 
$\mathbb E_{\eta\in[a,b]}\left[\min(\ell(\mathcal A(\eta)),\ell_0)-\ell^*\right]$
where $\ell(\mathcal A(\eta))$ is the loss achieved by the learning algorithm $\mathcal A$ with LR $\eta$,
$\ell_0$ is the loss at initialization, and
$\ell^*$ is the loss achieved by the best LR.
LayerScale is initialized at $10^{-4}$. 
Unlike vision tasks, where LayerScale \emph{improves performance} (\cref{fig:imagenet_top_1_ablations}-a), in LM, we observe that $\softmaxattn$ slightly benefits from LayerScale, while the performance of $\sigmoidattn$ remains largely unaffected.
\paragraph{Stability with QK Norm} \Cref{thm:regularity} indicates that the Jacobian of $\sigmoidattn$ has favorable properties compared to $\softmaxattn$. We explore this by repeating the analysis of \citet{DBLP:journals/corr/abs-2309-14322}, as described in the LayerScale analysis, to investigate the impact of QK norm \citep{DBLP:conf/icml/0001DMPHGSCGAJB23}. For language modeling, both $\sigmoidattn$ and $\softmaxattn$ exhibit sensitivity to learning rate changes without QK norm. However, incorporating QK norm significantly stabilizes performance (\cref{fig:qk_norm_ablation}). In vision tasks, $\sigmoidattn$ demonstrates robustness with and without QK norm (\cref{fig:imagenet_top_1_ablations}-a) and without the need for $n^{-\alpha}$ normalization from \citet{wortsman2023replacing}.\footnote{We ablate multiplicative sequence length scaling in more detail in \cref{sec:appendix_normalization}.}
\paragraph{Multi-query attention (MQA)} In \cref{fig:imagenet_top_1_ablations}-b we explore MQA \citep{DBLP:journals/corr/abs-1911-02150} for vision using only one head for $\{ \mK, \mV \}$. We find that both $\sigmoidattn$ and $\softmaxattn$ perform equally well with or without multiple heads even at the small scale of ViT-B/16.
\paragraph{Activation Function Ablations} As in \citet{wortsman2023replacing}, various activation functions, when combined with LayerScale and QK norm, perform equally well for vision tasks (\cref{fig:imagenet_top_1_ablations}-c). However, for sequence-critical tasks like ASR, activation functions such as ReLU pose instabilities and underperform. In the same figure, we also compare to the ReLU$^2$ proposal from \citet{DBLP:conf/icml/HuaDLL22} and find that it underperforms without LayerScale and QK norm.
\subsection{Supervised Image Classification}
\label{sec:supervised_image_classification}
Vision transformers \citep{DBLP:conf/iclr/DosovitskiyB0WZ21} extend transformers  \citep{DBLP:conf/nips/VaswaniSPUJGKP17} to treat $K \times K$ image grids as disparate tokens. All tokens are refined through sequential layers of self-attention, pooled using a CLS token or global average pooling layer, and optimized using the negative log likelihood, $\ln p(\vy|\vx)$. We train ViT-B/16 models using $\mathbb{R}^{224 \times 224 \times 3}$ images for 300 epochs using the recipe provided in \cref{sec:appendix_vision_hyperparams}. We use the same set of training hyper-parameters for both $\softmaxattn$ and $\sigmoidattn$, changing only the activation function between trials. The train negative log-likelihood is reported in \cref{fig:summary_nll} and the test top-1\% is reported in \cref{fig:test_top1_results}. We find that $\sigmoidattn$ matches both the training dynamics and the evaluation performance of $\softmaxattn$.
\subsection{Self-Supervised Image Representation Learning}
\label{sec:ssl}
Self-supervised representation learning (SSL) exploits vast quantities of unlabeled data to learn semantic representations based on inductive biases such as augmentation invariance (SimCLR \cite{DBLP:conf/icml/ChenK0H20}, BYOL \citep{DBLP:conf/nips/GrillSATRBDPGAP20}) or reconstruction from compressed representations (MAE \citep{DBLP:conf/cvpr/HeCXLDG22}). We employ vision transformer training recipes from \cite{DBLP:conf/icml/ZhaiLLBR0GS23} and \cite{DBLP:conf/nips/BusbridgeRALDCW23} (\cref{sec:appendix_vision_hyperparams}) for SimCLR and BYOL. As with supervised learning, we use the same set of training hyper-parameters for both $\softmaxattn$ and $\sigmoidattn$, changing only the activation function between trials. \Cref{fig:summary_nll} reports the train losses, and \cref{fig:test_top1_results} highlights the linear probe and finetuned test top-1\%. Despite the diverse training objectives in SSL, $\sigmoidattn$ matches $\softmaxattn$ while improving training and inference throughput (\cref{sec:FlashSigmoidHardwareAwareImplementation}).
\subsection{Automatic Speech Recognition (ASR)}
\label{sec:asr}
\begin{table}[t!]
\centering
\caption{Word error rate (\%) on LibriSpeech test sets and TED-LIUM v3~\citep{hernandez2018ted} (``TED'', joint validation and test sets split according to  duration) for transformer (255M params) with either $\softmaxattn$ or $\sigmoidattn$ (LayerScale and QK norm are used with $b=-\log n$) trained on LibriSpeech 960h data (mean duration is 10-15s). Hyper-parameters are in~\cref{sec:asr_hps}.}
\label{tab:asr-results}
\begin{center}
\begin{scriptsize}
\begin{sc}
\resizebox{\columnwidth}{!}{%
\begin{tabular}{lc|rr|rrrr}
\toprule
 attn & PE & test-clean & test-other & ted 0-10s & ted 10-20s & ted 20-30s & ted 30s+  \\
\midrule 
softmax & \multirow{7}{*}{CAPE} & 2.3 & 5.7 & 12.4 & 10.5 & 11.9 & 9.1 \\
 sigmoid &  & 2.4 & 5.5 & 12.4 & 10.3 & 12.3 & 9.7 \\
 \,\,\,\, - QK norm &  & \multicolumn{6}{c}{unstable, gradient norm and loss spikes} \\
 \,\,\,\, - LayerScale &  & 2.5 & 6.1 & 13.6 & 11.5 & 13.4 & 8.9 \\
 sigmoid ($b=-10$, learnable) &  & 2.3 & 5.5 & 12.1 & 10.5 & 13.0 & 9.3 \\
 sigmoid ($b=-5$ in $Q$, learnable) &  & 2.3 & 5.4 & 12.2 & 10.8 & 12.4 & 9.9 \\
 \,\,\,\, - QK norm &  & \multicolumn{6}{c}{unstable, gradient norm and loss spikes} \\

\midrule
softmax & \multirow{5}{*}{RoPE} & 2.2 & 5.5 & 12.7 & 10.6 & 12.8 & 9.5 \\
 sigmoid &  & 2.3 & 5.4 & 12.3 & 10.1 & 12.3 & 8.6 \\
 sigmoid ($b=-10$, learnable) &  & 2.2 & 5.2 & 12.4 & 10.5 & 12.3 & 21.8 \\
 \,\,\,\, + $\alpha=1$ &  & 2.7 & 6.6 & 14.1 & 12.0 & 14.5 & 14.9 \\
 sigmoid ($b=-5$ in $Q$, learnable) &  & \multicolumn{6}{c}{unstable, gradient norm and loss spikes} \\
\midrule
 softmax & \multirow{5}{*}{ALiBi} & 2.2 & 5.4 & 12.3 & 10.7 & 12.1 & 8.6 \\
 sigmoid &  & 2.3 & 5.1 & 12.3 & 10.5 & 12.6 & 9.1 \\
 sigmoid ($b=-10$, learnable) &  & 2.2 & 5.2 & 12.4 & 10.4 & 11.7 & 9.1 \\
 \,\, + $\alpha=1$ &  & 2.6 & 6.6 & 13.9 & 11.9 & 14.2 & 8.6 \\
 sigmoid ($b=-5$ in $Q$, learnable) &  & 2.2 & 5.2 & 12.1 & 10.4 & 12.0 & 8.2 \\
\bottomrule
\vspace{-0.4cm}
\end{tabular}
}
\end{sc}
\end{scriptsize}
\end{center}
\end{table}
We benchmark ASR using LibriSpeech data \citep{DBLP:conf/icassp/PanayotovCPK15} on 100h and 960h settings of paired speech and text transcriptions. Our PyTorch implementations of encoder-based vanilla transformer~\citep{synnaeve2019end} and conformer \citep{DBLP:conf/interspeech/GulatiQCPZYHWZW20} are trained with Connectionist Temporal Classification (CTC) \citep{DBLP:conf/icml/GravesFGS06} w/ BF16 mixed precision, w/o QK norm and w/o LayerScale. After extensively tuning $\softmaxattn$ baselines, we switch to $\sigmoidattn$ per \cref{eq:sigmoid_attn} without any other changes. We investigate the effects of post/pre-LayerNorm, model depth, optimizer type, small data regime, and connection to local attention, with details in~\cref{sec:asr_hps}.

Our main findings are: i) CAPE~\citep{DBLP:conf/nips/LikhomanenkoXSC21} PE is the most unstable for $\sigmoidattn$; ii) post-LayerNorm models with $\softmaxattn$ are hard to match with stable $\sigmoidattn$; iii) w/o QK norm $\sigmoidattn$ is unstable and significant spikes happen in both gradient norms and training loss; iv) LayerScale is needed for generalization; v) learnable bias $b=-10$ gives no loss and gradient norms spikes while matching the $\softmaxattn$ (which does not benefit from the improved throughput of \textsc{FlashSigmoid}); vi) adding a learnable bias, $b=-5$, to $Q$ instead of the attention logits also solves the initial large attention norms for CAPE and ALiBi but not for RoPE; vii) $b=-\log n$ gives rare (2-5 times) marginal gradient norms spikes with smooth loss while matching $\softmaxattn$.


\Cref{tab:asr-results} shows the main result for pre-LayerNorm  transformers with CAPE, RoPE, and ALiBi, where $\sigmoidattn$ uses LayerScale, QK norm, $b=-\log n$, and no sequence normalization. The bias is ablated with learnable bias (one per layer) in attention or $Q$ with or without sequence normalization. $\sigmoidattn$ is stabilized with bias while matching $\softmaxattn$, and $b=-\log n$ works well. In most cases, bias allows generalization to longer sequences without sequence normalization, except for RoPE where it helps for longer sequences but hurts overall performance.









\subsection{Autoregressive Large Language Modeling}
\label{sec:llm}

\newcolumntype{R}[2]{%
    >{\adjustbox{angle=#1,lap=\width-(#2)}\bgroup}%
    l%
    <{\egroup}%
}
\newcommand*\rotdiag{\multicolumn{1}{R{30}{1em}}}%

\begin{table}[t]
\centering
\caption{1B LLM English evaluation.}
\label{tab:lm_results}
\begin{sc}
\begin{scriptsize}
\bgroup
\setlength{\tabcolsep}{.35em}
\begin{tabular}{@{}lllllllllllllll@{}}
\toprule
Model   & \makecell{Seq.\\Len.} & \makecell{ARC\\Easy} & \makecell{ARC\\Challenge} & \makecell{Hella-\\swag} & Piqa & Sciq & \makecell{Wino-\\grande} & \makecell{Lambada\\OpenAI} & \makecell{TriviaQA\\(1-shot)} & \makecell{WebQS\\(1-shot)} & AVG & \makecell{Step\\time (s)} \\ \midrule
Softmax (ALiBi) & 2k & 62.2       &     26.8           &    42.4       &  59.0    &   72.3   &     88.1       &     58.4           &      19.9             &    15.4            &    49.4   & 0.38   \\
Sigmoid (ALiBi) & 2k &  62.8       &      28.8         &    42.5       &  59.7    &   70.3   &     88.6       &      59.7          &       19.1            &   13.8             &       49.5  & 0.34   \\
\midrule
Softmax (RoPE) & 4k & 63.3       &     29.3           &    43.3       &  58.1    &   71.3   &     86.9       &     58.8           &  20.4             &    15.6            &    49.7   & 0.84   \\
Softmax (ALiBi) & 4k & 62.6       &     27.7           &    42.4       &  58.6    &   71.1   &     88.2       &     58.6           &      18.9             &    14.7            &    49.2   & 0.84   \\
Sigmoid (ALiBi) & 4k &  60.5       &      27.3         &    41.3       &  57.8    &   70.5   &     87.0       &      57.6          &       18.9            &   12.6             &       48.2  & 0.67   \\ \bottomrule
\end{tabular}
\egroup
\end{scriptsize}
\end{sc}
\vspace{-0.4cm}
\end{table}

We initially iterated at the 85M scale, as it served as a proxy for larger scale training. Our findings show that: i) attention bias is required for stability, which can be learnable, but setting it to $-\log(n)$, where $n$ is the maximum training sequence length of 4096, works well and is faster; ii) RoPE is more challenging to stabilize; iii) the final setting exhibits smooth loss curves, but still shows gradient norm fluctuations. We then turn our attention to validating $\sigmoidattn$ at scale.

We train a 1B language model using the Llama2 \citep{touvron2023llama} recipe with ALiBi instead of RoPE positional embedding, and the RedPajama \citep{together2023redpajama} dataset (see \cref{sec:llm_appendix}). At sequence length 4096, $\sigmoidattn$ achieves a \textbf{1.23}$\mathbf{\times}$ step-time improvement over $\softmaxattn$ in JAX without \textsc{FlashAttention} (\cref{tab:lm_results}). All LLMs are trained using the AXLearn framework, which include the recipe and $\sigmoidattn$ implementation.\footnote{https://github.com/apple/axlearn}

$\softmaxattn$ and $\sigmoidattn$ have matching train and validation NLL at 85M (\cref{fig:85m_4k_nll}) and at 1B scale when using 2048 sequence length (\cref{fig:summary_nll}). However, a slight disparity is observed at 1B scale when using 4096 sequence length, which we leave for future investigation (more details in \cref{sec:llm_appendix}).



Hyperbolic embeddings embed hierarchical information with high
fidelity and few dimensions. We explored the limits of this approach
by describing scalable, high quality algorithms. We hope the
techniques here encourage more follow-on work on the exciting
techniques of \citet{fb, ucl}. As future work, we hope to explore how
hyperbolic embeddings can be most effectively incorporated into downstream
tasks and applications.


\section*{Acknowledgments}

We thank Laurel Orr, Xun Huang, Trevor Gale, Jian Zhang, Victor Bittorf, Sarah Hooper, Neel Guha, and Michael Zhang for their helpful discussions and feedback on early drafts of the paper.

We gratefully acknowledge the support of NIH under No.\ U54EB020405 (Mobilize), NSF under Nos.\ CCF1763315 (Beyond Sparsity), CCF1563078 (Volume to Velocity), and 1937301 (RTML); ARL under No.\ W911NF-21-2-0251 (Interactive Human-AI Teaming); ONR under No.\ N000141712266 (Unifying Weak Supervision); ONR N00014-20-1-2480: Understanding and Applying Non-Euclidean Geometry in Machine Learning; N000142012275 (NEPTUNE); NXP, Xilinx, LETI-CEA, Intel, IBM, Microsoft, NEC, Toshiba, TSMC, ARM, Hitachi, BASF, Accenture, Ericsson, Qualcomm, Analog Devices, Google Cloud, Salesforce, Total, the HAI-GCP Cloud Credits for Research program,  the Stanford Data Science Initiative (SDSI), and members of the Stanford DAWN project: Facebook, Google, and VMWare. The U.S.\ Government is authorized to reproduce and distribute reprints for Governmental purposes notwithstanding any copyright notation thereon. Any opinions, findings, and conclusions or recommendations expressed in this material are those of the authors and do not necessarily reflect the views, policies, or endorsements, either expressed or implied, of NIH, ONR, or the U.S.\ Government.

\bibliography{ref}
\bibliographystyle{icml2022}


\newpage
\appendix
\onecolumn

\section{Extended Related Work}
\label{app:related}
In this section, we extend the related works referenced in the main paper and discuss them in detail.
\paragraph{Sparse Training.} Our work is loosely related to neural network pruning. By iteratively eliminating neurons and connections, pruning has seen great success in compressing complex models. \citet{han2015deep,han2015learning} put forth two naive but effective algorithms to compress models up to 49x and maintain comparable accuracy. \citet{li2016pruning} employ filter pruning to reduce the cost of running convolution models up to 38 $\%$, \citet{NIPS2017_a51fb975} prunes the network at runtime, hence retaining the flexibility of the full model. \citet{dong2017learning} prunes the network locally in a layer by layer manner.  \citet{sanh2020movement} prunes with deterministic first-order information, which is more adaptive to pretrained model weights. \citet{lagunas2021block} prunes transformers models with block sparsity pattern during fine-tuning, which leads to real hardware speed up while maintaining the accuracy. \citet{zhu2017prune} finds large pruned sparse network consistently outperform the small dense networks with the same compute and memory footprints. Although both our and all the pruning methods are aiming to produce sparse models, we differ in our emphasis on the overall efficiency, whereas pruning mostly focuses on inference efficiency and disregards the cost in finding the smaller model.

There has been more recent work on sparse methods that focuses on speeding up
training and not just inference, such as SNFS~\citep{dettmers2019sparse},
RigL~\citep{dettmers2019sparse}, Top-KAST~\citep{jayakumar2021top}.
These methods often focus on FLOP counts, which may not correlate well with
wall-clock time on modern hardware (e.g., GPUs).
Block-sparsity is another approach that exploits the block-oriented nature of
GPUs~\citep{gray2017gpu, child2019generating, guo2020accelerating}.
Sparse models have also been found useful to improve the training process of
dense models.
For example, sparsity can be used to regularize dense models to improve
accuracy~\citep{han2016dsd}, or to alternate between sparse and dense training
to ease deployment~\citep{peste2021ac}.
Our sparse-to-dense reverse sparsification instead focuses on speeding up dense
training, where the sparse model is used for efficiency and not regularization.


In addition, models proposed in our work can be roughly seen as a class of manually constructed lottery tickets. Lottery tickets \citet{frankle2018lottery} are a set of small sub-networks derived from a larger dense network, which outperforms their parent networks in convergence speed and potentially in generalization. A huge number of studies are carried out to analyze these tickets both empirically and theoretically: \citet{morcos2019one} proposed to use one generalized lottery tickets for all vision benchmarks and got comparable results with the specialized lottery tickets; \citet{frankle2019stabilizing} improves the stability of the lottery tickets by iterative pruning; \citet{frankle2020linear} found that subnetworks reach full accuracy only if they are stable against SGD noise during training; \citet{orseau2020logarithmic} provides a logarithmic upper bound for the number of parameters it takes for the optimal sub-networks to exist; \citet{pensia2020optimal} suggests a way to construct the lottery ticket by solving the subset sum problem and it's a proof by construction for the strong lottery ticket hypothesis. Furthermore, follow-up works \citep{liu2020finding, wang2020picking, tanaka2020pruning} show that we can find tickets without any training labels.


\paragraph{Structured matrices and butterfly matrices.}
Structured matrices are those with asymptotically fast matrix-vector
multiplication algorithm ($o(n^2)$ time complexity) and few parameters ($o(n^2)$
space complexity).
Common examples include sparse \& low-rank matrices, and fast transforms such as
Fourier transform, Chebyshev transform, Legendre transform, and more generally
orthogonal polynomial transforms.
These transforms have been widely used in data preprocessing (e.g., DFT in
speech processing~\citep{jurafsky2014speech}) and kernel
approximation~\citep{le2013fastfood,yu2016orthogonal}.
Many generalizations of these transforms have been used in machine learning to
replace dense weight
matrices~\citep{sindhwani2015structured,thomas2018learning,gu2020hippo}.
\citet{desa2018two} shows that any structured matrix (in the form of arithmetic
circuits) can be written as product of sparse matrices,
and~\citet{dao2020kaleidoscope} shows that products of butterfly matrices can
represent these structured matrices almost optimally in terms of runtime and
memory.
The class of butterfly matrices~\citep{parker1995random} have also been used in
kernel models~\citep{munkhoeva2018quadrature, choromanski2019unifying} and deep
learning models~\citep{vahid2020butterfly,lin2021deformable,
  ailon2021sparse}.

\paragraph{Neural Operators for PDEs.}

Deep learning has found application in the domain of differential equations and scientific computing \cite{rackauckas2020universal}, with methods developed for prediction and control problems \cite{kidger2020neural,massaroli2021differentiable}, as well as acceleration of numerical schemes \cite{poli2020hypersolvers,jolicoeur2021gotta}. Specific to the \textit{partial differential equations} (PDEs) are approaches designed to learn solution operators \cite{raissi2019physics,fan2020solving,li2020fourier}, and hybridized solvers \cite{kochkov2021machine}, evaluated primarily on classical fluid dynamics.

The promise of these approaches is to offer, at the cost of an initial training procedure, accurate yet faster solutions than an appropriate numerical method tuned for a specific problem, which can then be leveraged for real-time forecasting or within larger feedback loops. Nonetheless, optimal design of neural operators remains an open problem, with most relying on fast Fourier transforms (FFT) or standard dense neural architectures. Instead, neural operators based on Monarch are capable of approximating all fast transforms, thus allowing automated optimization towards a suitable transform on a given PDE problem.

\paragraph{MRI.} Accelerated multi-coil MRI is an essential mechanism for reducing long scan times and making certain scan types feasible. In multi-coil MRI, data is acquired in the spatial Fourier domain (a.k.a \textit{k-space}) across multiple coils (sensors). To reduce scan time, this data is sampled below the required rate for recovering the underlying signal (i.e. Nyquist rate), which results in signal aliasing (see Appendix \ref{sec:experiment_details_mri}). In these settings, direct application of the inverse fast Fourier transform (FFT) cannot suppress aliasing artifacts.

Classical MRI reconstruction approaches supplement the FFT by leveraging shared information across multiple coils and strong analytical priors to regularize image recovery objectives. SENSE-based methods jointly dealias images across multiple coils and reweight the final image based on the spatial sensitivity profile of each coil \citep{pruessmann1999sense}. Compressed sensing promotes image sparsity in transformation domains (e.g. Fourier, wavelet) while enforcing data consistency between the Fourier transform of the reconstructed image and the observed measurements \citep{lustig2007sparse}. Low-rank methods enforce low rank structure across slowly-varying dimensions or local patches in the data \citep{ong2016beyond,ravishankar2017low,haldar2013low}. Additionally, GRAPPA-based techniques optimize kernels to directly interpolate missing k-space samples to promote smoothness in the Fourier domain \cite{griswold2002generalized}. Despite their efficacy, these methods have long reconstruction times, require explicit analytical priors, and require careful hyperparameter fine-tuning.

CNNs have shown promise as a fast-at-inference, learnable alternative to classical MRI reconstruction methods \cite{knoll2020deep}. In supervised learning, fully convolutional networks (e.g. U-Net \citep{ronneberger2015u} or unrolled networks \citep{sandino2020compressed,hammernik2018learning}) learn a mapping between paired zero-filled and fully-sampled, ground truth images. However, supervised methods require a large fully-sampled (labeled) data corpus and are sensitive to distribution drifts due to patient, hardware, and sequence heterogeneity \cite{darestani2021measuring}. To reduce dependence on labeled data, unsupervised methods have used generative adversarial networks \citep{cole2020unsupervised, mardani2018deep}, self-supervised learning \cite{yaman2020self}, dictionary learning \cite{lahiri2021blind}, and untrained networks \cite{darestani2021accelerated}. Despite their 
label efficiency, these techniques still underperform supervised methods and are also sensitive to distribution shift. Recently, a family of semi-supervised reconstruction methods demonstrated label efficiency and robustness to physics-driven perturbations, such as changes in signal-to-noise ratio or patient motion \citep{desai2021noise2recon, desai2021vortex}. However, these methods require large amounts of unlabeled data, which can be difficult to curate in few-shot settings. Thus, despite their success in controlled environments, prospective clinical deployment of these models has been stifled \citep{chaudhari2020prospective}.

In our work, we propose a model with a single FFT-initialized factorized Monarch matrix. Such a matrix can provide the benefits of both a simple linearized transformation like FFT and a learnable mechanism to remove aliasing artifacts resulting from the undersampled k-space. The smaller learnable parameter set may reduce overfitting in data-limited settings while preserving the transformation structure of Fourier matrices. Thus, our approach can be interpreted as a hybrid between analytically-constrained classical methods and data-dependent CNNs.


\section{Notation Review}
Throughout this paper, we use lowercase to denote scalars (e.g., $k$), lowercase boldface to denote vectors (e.g., $\vv$), and uppercase boldface to denote matrices (e.g., $\vA$).

$\vI$ denotes the identity matrix. We use $\vA^\top$ to denote the transpose of a matrix and $\vA^*$ to denote the conjugate transpose of a matrix. All results in this paper apply to matrices over the either the reals $\mathbb{R}$ or the complex numbers $\mathbb{C}$; when the field under consideration can be either one of these, we denote it by $\mathbb{F}$.

We use 1-indexing throughout this paper except where explicitly stated.


\newcommand{\baseb}[3]{\parens{{#1},{#2}}_{{#3}}}
\newcommand{\mx}[1]{\mathbf{#1}}
\newcommand{\floors}[1]{\left \lfloor #1 \right \rfloor}
\newcommand{\parens}[1]{\left( {#1}\right)}

\section{General Monarch Matrix Parametrization}
\label{sec:permutation}

In Section \ref{sec:Monarch_square}, we define a parametrization for square Monarch matrices of different ``block sizes'' (i.e., not necessarily $\sqrt{n}$), and prove some basic properties about them. In Section \ref{sec:Monarch_rect}, we further extend this to define rectangular Monarch matrices, and prove some basic properties about them.

Note: In this section, we use 0-indexing rather than 1-indexing, for notational convenience.

\subsection{General square matrices}
\label{sec:Monarch_square}
\subsubsection{Parametrization}
\label{sec:Monarch_square_param}
In this section, we define a more general Monarch parametrization for square matrices, allowing for different ``block sizes.'' Like \cref{def:Monarch}, the parametrization involves the product of a permuted block-diagonal matrix with another block-diagonal matrix; the difference is that we now allow the matrices $\vL$ and $\vR$ to have diagonal blocks of different sizes. Thus, the permutations applied to $\vL$ (to turn it into a block matrix where each block matrix is diagonal) will correspondingly also be different.

First, in \cref{def:square_r}, we define notation for a class of block-diagonal matrices.

\begin{definition}[Class $\BD\ind{b, n}$]
\label{def:square_r}
Let $b \in (1, n)$ be an integer that divides $n$. For $0\le i< \frac {n}{b}$, let $\mx{R}_{i}\in\F^{b \times b }$ be a $b \times b $ ``block" matrix. Then define the matrix $\vR$ with {\em block size} $b$ as follows:
\begin{equation}
 \label{eq:def-R}
  \vR = \diag\left(\vR_0, \dots, \vR_{\frac {n}{b}-1}\right).
\end{equation}
\end{definition}
(Note that the number of possible nonzero values in $\vR$ is $\frac {n}{b}\cdot b^2=nb$.)
We denote the class of all matrices $\vR$ expressible in this form by $\BD\ind{b, n}$. Note that this class is closed under (conjugate) transposition and contains the identity matrix.

Next, in \cref{def:Matrix L}, we define notation for a class of block matrices whose \emph{blocks} are diagonal.

\begin{definition}[Class $\DB\ind{b,n}$]
\label{def:Matrix L}
Let $b \in (1, n)$ be an integer that divides $n$. For $0 \le i, j < b$, let $\mx{D}_{i,j}\in\F^{b\times b}$ be a $b \times b$ diagonal matrix.
Then let $\vL$ be an $n \times n$ matrix with the following form: 
    \begin{equation}
        	\label{eq:def-L}
    \vL=
    	\begin{bmatrix}
    		\mx{D}_{0,0} & \dots & \mx{D}_{0,\frac{n}{b} -1} \\
    		\vdots & \ddots & \vdots \\
    		\mx{D}_{\frac{n}{b} -1,0} & \dots & \mx{D}_{\frac{n}{b} -1,\frac{n}{b} -1}
    	\end{bmatrix}
    \end{equation}
\end{definition}
(Note that the number of possible nonzero values in $\vL$ is $\parens{\frac nb}^2\cdot b=\frac{n^2}b$.)
We denote the class of all matrices $\vL$ expressible in this form by $\DB\ind{b, n}$. Note that this class is closed under (conjugate) transposition and contains the identity matrix. As we show in \cref{sec:sq-Monarch-properties}, $\vL$ can be written as a block-diagonal matrix with $b$ blocks of size $\ff{n}{b} \times \ff{n}{b}$ (i.e., a matrix in $\BD\ind{\frac{n}{b}, \, n}$), multiplied on the left and right with appropriate permutation matrices.
We denote the class of all matrices $\vL$ expressible in this form by $\DB\ind{b, n}$. Note that this class is closed under (conjugate) transposition. As we show in \cref{sec:sq-Monarch-properties}, $\vL$ can be written as a block-diagonal matrix with $b$ blocks of size $\ff{n}{b} \times \ff{n}{b}$ (i.e., a matrix in $\BD\ind{\frac{n}{b}, \, n}$), multiplied on the left and right with appropriate permutation matrices.

Using these two definitions, we define the class of Monarch matrices with a given block size.
\begin{definition}[Class $\M\ind{b,n}$]
\label{def:block_Monarch}
Let $b \in (1, n)$ be an integer that divides $n$. A \emph{Monarch matrix} of size $n \times n$ and ``block size $b$'' is a matrix of the form: 
    \begin{equation}
        	\label{eq:Monarch-general}
    \vM= \vL \vR
    \end{equation}
    where $\vL \in \DB\ind{b,n}$ and $\vR \in \BD\ind{b,n}$.
\end{definition}
We denote the class of all matrices $\vM$ expressible in this form by $\M\ind{b, n}$. Observe that when $b = \sqrt{n}$, this is exactly the matrix class $\M\ind{n}$ in \cref{def:Monarch}. (In other words, $\M\ind{n}$ is shorthand for $\M\ind{\sqrt{n}, n}$.) Note that a matrix in $\M\ind{b,n}$ is represented by $\frac{n^2}{b} + nb$ parameters.

We remark that $\M\ind{b,n} \supset \B\ind{n}$ for all block sizes $b \in (1, n)$ that divide $n$.

Based on \cref{def:block_Monarch}, we define the classes $\M\M^{*(b,n)}$ and $\M^*\M^{(b,n)}$::
\begin{definition}[Class $\M\M^{*(b,n)}$, $\M^*\M^{(b,n)}$]
\label{def:block_MM}
Let $b \in (1, n)$ be an integer that divides $n$ and suppose $\vM_1, \vM_2 \in \M^{(b,n)}$. We define $\M\M^{*(b,n)}$ to be the the class of all matrices $\vM$ expressible in the form $\vM= \vM_1 \vM_2^*$. \newline
We define $\M^*\M^{(b,n)}$ to be the the class of all matrices $\vM$ expressible in the form $\vM= \vM_1^* \vM_2$.
\end{definition}
Observe that when $b = \sqrt{n}$, $\M\M^{*(b,n)}$ is exactly the matrix class $\M\M^{*(n)}$ defined in \cref{sec:theory}. Note that a matrix in $\M\M^{*(b,n)}$ or $\M^*\M\ind{b,n}$. is represented by $2\frac{n^2}{b} + 2nb$ parameters.

Finally, we define the following ``Monarch hierarchy'' based on the kaleidoscope hierarchy of \cite{dao2020kaleidoscope}:
\begin{definition}[Class $(\M\M^{*(b,n)})^w_e$]
\label{def:block_MM}
Let $b \in (1, n)$ be an integer that divides $n$. We define the matrix class $(\M\M^{*(b,n)})^w_e$ as the set of all matrices $\vM$ that can be expressed as
    \begin{equation}
        	\label{eq:mm-hierarchy}
    \vM= \lt \pd{i=1}{w} \vM_i\rt [1:n, 1:n]
    \end{equation}
    where each $\vM_i \in \M\M^{*(b,e\cdot n)}$.
\end{definition}
Note that a matrix in $(\M\M^{*(b,n)})^w_e$ is represented by $2w\frac{e^2n^2}{b} + 2wenb$ parameters.

\subsubsection{Properties}
\label{sec:sq-Monarch-properties}
Here we show some properties of the matrix classes defined above. We first show some basic equivalent ways to define these classes. We then show (\cref{thm:lr_permutation}) that the matrices in $\DB\ind{b, n}$ are permuted block-diagonal matrices; specifically, that they can be converted to matrices in $\BD\ind{\frac{n}{b}, n}$ by applying the appropriate permutation. Finally, we state an expressivity result for the general ``Monarch hierarchy''  which follows from Theorem 1 of \cite{dao2020kaleidoscope}.

First, we define a class of permutations.
Let $1\le b\le n$ be integers such that $b$ divides $n$.
We will need to express each index $0\le i<n$ in ``block form.'' More specifically:

\begin{definition}\label{def:$i$}
Let $i \ge 0$, $b \ge 1$ be integers. Then define
\[i_0=i\mod{b},\]
and
\[i_1=\floors{\frac ib}.\] 
We use the notation $i\equiv\baseb{i_1}{i_0}{b}$ to denote the representation above. In particular, if $i\equiv(i_1,i_0)_{b}$,
then we have
\[
        i = i_1 \cdot b + i_0
\]
\end{definition}

Using this notation, we define the following class of permutations:
\begin{definition}
\label{def:sigma-b}
Let $b \in [1, n]$ be an integer that divides $n$.  Let $i\equiv\baseb{i_1}{i_0}{b}$. Define
    \begin{equation}
            \label{eq:sigma_b-def}
        \sigma_{(b,n)}(i) = i_0\cdot\frac{n}{b} + i_1.
    \end{equation}
That is, $\sigma_{(b,n)}(i)\equiv \baseb{i_0}{i_1}{\frac {n}{b}}$.
Let $\vP_{(b,n)}$ denote the $n \times n$ permutation matrix defined by the permutation $\sigma_{(b,n)}$.
\end{definition}
Intuitively, $\vP_{(b,n)}$ can be interpreted as reshaping a length-$n$ vector into an $b \times \ff{n}{b}$ matrix in row-major order, transposing the result, and then flattening this back into a vector (again in row-major order).


Now, we restate the formulation in \cref{def:square_r} equivalently as:
\begin{proposition}

\label{prop:R-eqv-def}
A matrix $\vR$ satisfies~\Cref{eq:def-R} (i.e., $\vR \in \BD\ind{b,n}$) if and only if the following holds for any
$0\le i,j< n$. Let $i\equiv\baseb{i_1}{i_0}{b}$ and $j\equiv\baseb{j_1}{j_0}{b}$.  Then
\begin{enumerate}
    \item\label{item:zero-loc-R} if $i_1\ne j_1$, then $\vR[i,j]=0$. 
    \item \label{item:nonzero-loc-R} Else (i.e., when $i_1=j_1$), then $\vR[i,j]=\vR_{i_1}[i_0,j_0]$.
\end{enumerate}

\end{proposition}



We restate the formulation in \cref{def:Matrix L} equivalently as:
\begin{proposition}
\label{prop:L-eqv-def}
A matrix $\vL$ satisfies~\Cref{eq:def-L} (i.e., $\vL \in \DB\ind{b,n}$) if and only if the following holds for any
$0\le i,j< n$. Let $i\equiv\baseb{i_1}{i_0}{b}$ and $j\equiv\baseb{j_1}{j_0}{b}$. Then 
\begin{enumerate}
    \item\label{item:zero-loc-L} if $i_0\ne j_0$, then $\vL[i,j]=0$. 
    \item \label{item:nonzero-loc-L} Else, (i.e., when $i_0=j_0$), then $\vL[i,j]=\vD_{i_1,j_1}[i_0,i_0]$.
\end{enumerate}
\end{proposition}

We will argue the following:
\begin{theorem}\label{thm:lr_permutation} Let $1\le b\le n$ such that $b$ divides $n$.
Recall that $\vP_{(b,n)}$ is the permutation matrix defined by the permutation $\sigma_{(b,n)}$. Let $\vL$ be a matrix in $\DB\ind{b, n}$. Then we have
\[\vR'=\vP_{(b,n)}\cdot\vL\cdot\vP_{(b,n)}^\top,\]
where $\vR' \in \BD\ind{\frac{n}{b},\, n}$.
\end{theorem}

\begin{proof}
We first note that multiplying an $n\times n$ matrix on the right (and left resp.) by $\vP_{(b,n)}^\top = \vP_{(\frac nb,n)}$ (and $\vP_{(b,n)}$ resp.) permutes the columns (and rows resp.) of the matrix according to $\sigma_{(b,n)}$.\footnote{This uses the fact that $\parens{\sigma_{(b,n)}}^{-1}=\sigma_{(\frac nb,n)}$ (which means $P_{(\frac{n}{b}, n)} = P_{(b, n)}^\top$ since the inverse of a permutation matrix is its transpose).} This implies that for any $0\le i,j<n$:
\begin{equation}
\label{eq:L-permuted}
\vR'[\sigma_{(b,n)}(i),\sigma_{(b,n)}(j)]=\vL[i,j].
\end{equation}
To complete the proof, we will argue that $\vR'$ satisfies the two conditions in~\Cref{prop:R-eqv-def}.

Towards this end, let $0\le i,j<n$ be arbitrary indices and further, define $i=\baseb{i_1}{i_0}{b}$ and $j=\baseb{j_1}{j_0}{b}$. Then note that $\sigma_{(b,n)}(i)=\baseb{i_0}{i_1}{\frac nb}$ and $\sigma_{(b,n)}(j)=\baseb{j_0}{j_1}{\frac nb}$.

By~\Cref{prop:L-eqv-def}, we have that if $i_0\ne j_0$, then $\vL[i,j]=0$. Note that $i_0\ne j_0$ satisfies the pre-condition for base size $\frac nb$ for indices $(\sigma_{(b,n)}(i),\sigma_{(b,n)}(j))$ in item~\ref{item:zero-loc-R} in~\Cref{prop:R-eqv-def}.  Then   by~\cref{eq:L-permuted}, we have that $\vR'[\sigma_{(b,n)}(i),\sigma_{(b,n)}(j)]=0$, which satisfies item~\ref{item:zero-loc-R} in~\Cref{prop:R-eqv-def}.

Now consider the case that $i_0=j_0$; then by item~\ref{item:nonzero-loc-L} in~\Cref{prop:L-eqv-def}, we have that $\vL[i,j]=\vD_{i_1,j_1}[i_0,i_0]$.  Note that $i_0= j_0$ satisfies the pre-condition for base size $\frac nb$ for indices $(\sigma_{(b,n)}(i),\sigma_{(b,n)}(j))$ in item~\ref{item:nonzero-loc-R} in~\Cref{prop:R-eqv-def} if we define $\vR'_{i_0}\in\F^{\frac nb\times\frac nb}$ as follows:
\[\vR'_{i_0}[i_1,j_1]=\vD_{i_1,j_1}[i_0,i_0].\] 
Note that the above implies that 
\[\vR'=\diag\parens{\vR'_0,\dots,\vR'_{b-1}},\]
where $\vR'_{\cdot}$ is as defined in the above paragraph. This means $\vR' \in \BD\ind{\frac{n}{b}, n}$, since each block $\vR_{i_0}'$ is a matrix of size $\frac{n}{b} \times \frac{n}{b}$.
\end{proof}

We now briefly note some alternate ways to express matrices in $\M\M^{*(b,n)}$.
\begin{proposition}
\label{prop:mm-eqv-def}
For any $\vM \in \M\M^{*(b,n)}$, we can write $\vM = (\vP_{(b,n)}^\top \vL_1 \vP_{(b,n)}) \vR (\vP_{(b,n)}^\top\vL_2\vP_{(b,n)})$, where $\vL_1,\vL_2 \in \BD\ind{\frac{n}{b},n}$ and $\vR \in \BD\ind{b,n}$.
\end{proposition}
\begin{proof}
By definition (see \cref{def:square_r} and \cref{def:Matrix L}), if $\vM \in \M\M^{*(b,n)}$, we can write
$\vM = (\vL_1' \vR_1) (\vL_2' \vR_2)^* = \vL_1' (\vR_1^* \vR_2) \vL_2'^*$,
where $\vL_1',\vL_2' \in \DB\ind{b,n},\vR_1,\vR_2 \in \BD\ind{b,n}$.

Notice that since $\vR_1^*, \vR_2$ are both block-diagonal with the same structure (i.e., both have blocks of size $b \times b$), their product $\vR$ is also in $\BD\ind{b,n}$.
Also, by \cref{thm:lr_permutation} we can write $\vL_1 = \vP_{(b,n)} \vL_1' \vP_{(b,n)}^\top$, $\vL_2 = \vP_{(b,n)} \vL_2' \vP_{(b,n)}^\top$, where $\vL_1,\vL_2$ are both in $\BD\ind{\frac{n}{b},n}$ (i.e., block diagonal with blocks of size $\frac{n}{b} \times \frac{n}{b}$).


Thus, we can write $\vM = (\vP_{(b,n)}^\top \vL_1 \vP_{(b,n)}) \vR (\vP_{(b,n)}^\top\vL_2\vP_{(b,n)})$, where $\vL_1,\vL_2 \in \BD\ind{\frac{n}{b},n}$ and $\vR \in \BD\ind{b,n}$.
\end{proof}

We use the above to show a simple relationship between $\M\M^{*(b,n)}$ and $\M^*\M^{(b,n)}$.
\begin{proposition}
\label{prop:mstarm}
If $\vM \in \M\M^{*(b,n)}$, then $\vP_{(b,n)} \vM \vP_{(b,n)}^\top \in \M^*\M\ind{\frac{n}{b},n}$. Conversely, if $\vM \in \M^*\M^{(b,n)}$, then $\vP_{(b,n)}^\top \vM \vP_{(b,n)} \in \M^*\M\ind{\frac{n}{b},n}$.
\end{proposition}
\begin{proof}
Suppose $\vM \in \M\M^{*(b,n)}$. By \cref{prop:mm-eqv-def} we can write $\vM = (\vP_{(b,n)}^\top \vL_1 \vP_{(b,n)}) \vR (\vP_{(b,n)}^\top\vL_2\vP_{(b,n)})$, where $\vL_1,\vL_2 \in \BD\ind{\frac{n}{b},n}$ and $\vR \in \BD\ind{b,n}$.
Thus $\vP_{(b,n)} \vM \vP_{(b,n)}^\top =
\vL_1 (\vP_{(b,n)} \vR \vP_{(b,n)}^\top) \vL_2$.

Letting $\vL_1' = \vL_1, \vL_2' = \vL_2^*, \vR_1' = \vP_{(b,n)} \vR \vP_{(b,n)}^\top$, and $\vR_2' = \vI$, we have $\vL_1', \vL_2' \in \BD\ind{\frac{n}{b}, n}$, $\vR_1', \vR_2' \in \DB\ind{\frac{n}{b}, n}$, and
$\vL_1 (\vP_{(b,n)} \vR \vP_{(b,n)}^\top) \vL_2 = 
\vL_1' \vR_1' \vR_2'^* \vL_2'^* = (\vL_1' \vR_1')(\vL_2' \vR_2')^* = \vM_1'\vM_2'^*$, where $\vM_1' = \vL_1' \vR_1', \vM_2' = \vL_2' \vR_2'$, so $\vM_1', \vM_2' \in \M^*\M\ind{\frac{n}{b},n}$.

Now instead suppose $\vM \in \M^*\M^{(b,n)}$. So $\vM = \vM_1^* \vM_2 = \vR_1^* \vL_1^* \vL_2 \vR_2$ for some $\vR_1, \vR_2 \in \BD\ind{b,n}$ and $\vL_1, \vL_2 \in \DB\ind{b,n}$. Thus by \cref{thm:lr_permutation} (and the fact that $\BD\ind{b,n}$ is closed under conjugate transposition) we can write $\vR_1^* = \vP_{(\frac{n}{b},n)}^\top \vR_1' \vP_{(\frac{n}{b}, n)} = \vP_{(b,n)} \vR_1' \vP_{(b, n)}^\top$ for some $\vR_1' \in \DB\ind{\frac{n}{b}, n}$, and similarly, can write $\vR_2 = \vP_{(b,n)} \vR_2' \vP_{(b,n)}^\top$ for some $\vR_2' \in \DB\ind{\frac{n}{b}, n}$. 

So $\vP_{(b,n)}^\top \vM \vP_{(b,n)} = \vR_1' (\vP_{(b, n)})^\top \vL_1^*)(\vL_2 \vP_{(b, n)})) \vR_2' =
 \vR_1' (\vP_{(b, n)}^\top \vL_1^* \vP_{(b, n)})(\vP_{(b, n)}^\top \vL_2 \vP_{(b, n)}) \vR_2' = (\vR_1' \vL_1')(\vL_2' \vR_2')$, where $\vL_1' = \vP_{(b, n)}^\top \vL_1^* \vP_{(b, n)}$, $\vL_2' = \vP_{(b, n)}^\top \vL_2 \vP_{(b, n)}$ are in $\BD\ind{\frac{n}{b}, n}$ by \cref{thm:lr_permutation}. Thus letting $\vM_1' = \vR_1'\vL_1'$, $\vM_2' = \vR_2^*\vL_2'^*$, we have $\vM = \vM_1' \vM_2'^*$ with $\vM_1', \vM_2' \in \M^{*(\frac{n}{b},n)}$.
\end{proof}

We now show that the class $\M\ind{b,n}$ strictly contains the class $\B\ind{n}$ of $n \times n$ butterfly matrices (as defined in \citet{dao2020kaleidoscope}). We first show two elementary ``helper'' results.

\begin{proposition}
\label{prop:bd-contain}
If $b,\, c \in (1, n)$ are such that $b$ divides $c$ and $c$ divides $n$, then $\BD\ind{b, n} \subseteq \BD\ind{c, n}$.
\end{proposition}
\begin{proof}
Suppose $\vR \in \BD\ind{b, n}$. Then by \cref{prop:R-eqv-def}, $\vR[i, j] = 0$ whenever $\floor{\frac{i}{b}} \ne \floor{\frac{j}{b}}$. Thus, whenever $\floor{\frac{i}{c}} \ne \floor{\frac{j}{c}}$, $\vR[i, j] = 0$, since $\floor{\frac{i}{c}} \ne \floor{\frac{j}{c}}$ implies $\floor{\frac{i}{b}} \ne \floor{\frac{j}{b}}$ by the assumption that $b$ divides $c$.
Applying \cref{prop:R-eqv-def} again, this means $\vR \in \BD\ind{c,n}$ as well.
\end{proof}

\begin{proposition}
\label{prop:db-contain}
If $b,\, c \in (1, n)$ are such that $b$ divides $c$ and $c$ divides $n$, then $\DB\ind{c, n} \subseteq \DB\ind{b, n}$.
\end{proposition}
\begin{proof}
Suppose $\vL \in \DB\ind{c, n}$. Then by \cref{prop:L-eqv-def}, $\vL[i, j] = 0$ whenever $(i \mod c) \ne (j \mod c)$. Thus, whenever $(i \mod b) \ne (j \mod b)$, $\vL[i, j] = 0$, since $(i \mod b) \ne (j \mod b)$ implies $(i \mod c) \ne (j \mod c)$ by the assumption that $b$ divides $c$.
Applying \cref{prop:L-eqv-def} again, this means $\vL \in \DB\ind{b,n}$ as well.
\end{proof}


\begin{theorem}
\label{thm:b_contained}
Let $n \ge 4$ be a power of 2. The class of matrices $\B\ind{n}$ is a subset of the class $\M\ind{b, n}$, for all $b \in (1, n)$ that divide $n$. When $n \ge 512$ it is a strict subset.
\end{theorem}
\begin{proof}
Recall from \cref{sec:butterfly} that if $\vB \in \B\ind{n}$, it has a \emph{butterfly factorization} 
$\vB = \vB_n \vB_{n/2} \hdots \vB_2$, where each $\vB_i \in \BF\ind{n, i}$.

Consider multiplying together the factors $\vB_b \vB_{b/2} \dots \vB_2$ (where $b \in (1, n)$ divides $n$). Since $\vB_i \in \BF\ind{n,i}$, by definition it is block diagonal with diagonal blocks of size $i \times i$; in other words, $\vB_i \in \BD\ind{i, n}$. Thus, each of the matrices $\vB_b, \vB_{b/2}, \dots, \vB_2$ is in $\BD\ind{b, n}$ (by \cref{prop:bd-contain}), i.e. block-diagonal with block size $b \times b$. This means their product $\vB_b \vB_{b/2} \dots \vB_2$ is also block diagonal with block size $b \times b$, i.e., it is in $\BD\ind{b, n}$.

Now, note that since $\vB_i \in \BF\ind{n,i}$, by definition it is a block matrix with blocks of size $i/2 \times i/2$, where each block is a diagonal matrix (note that some of these blocks are zero, except for the case of $\vB_n$). In other words, $\vB_i \in \DB\ind{i/2, n}$. Thus, for all $i \in \{n, n/2, \dots, 2b\}$, $\vB_i \in \DB\ind{(2b)/2, n} = \DB\ind{b, n}$ (by \cref{prop:db-contain}). So, their product $\vB_n \vB_{n/2} \dots \vB_{2b}$ is in $\DB\ind{b, n}$ as well, as by \cref{thm:lr_permutation} we can write $\vB_n \vB_{n/2} \dots \vB_{2b} = \vP_{(b,n)}^\top (\vP_{(b,n)} \vB_n \vP_{(b,n)}^\top) (\vP_{(b,n)} \vB_{n/2} \vP_{(b,n)}^\top) \dots (\vP_{(b,n)} \vB_{2b} \vP_{(b,n)}^\top) \vP_{(b,n)}$ and each of the $\vP_{(b,n)} \vB_i \vP_{(b,n)}^\top$'s in the preceding expression is in $\BD\ind{\frac{n}{b}, n}$.

Thus, if we let $\vL = \vB_n \vB_{n/2} \dots \vB_{2b}$ and $\vR = \vB_b \vB_{b/2} \dots \vB_2$,  we have $\vB = \vL\vR$ and $\vL \in \DB\ind{b, n}$, $\vR \in \BD\ind{b, n}$, which means that $\vB \in \M\ind{b,n}$ (\cref{def:block_Monarch}).

To show that the inclusion is strict, notice that any $\vM \in \M\ind{b,n}$ is the product of $\vL$ and $\vR$, where $\vR \in \BD\ind{b, n}$ and $\vP_{(b,n)}^\top \vL \vP_{(b,n)} \in \BD\ind{\frac{n}{b}, n}$ (by \cref{thm:lr_permutation}). Notice that the identity matrix is contained in both $\BD\ind{b,n}$ and $\DB\ind{b,n}$. Suppose first that $b \le \sqrt{n}$. Then even if we set $\vR$ to the identity, $\vM$ has at least $\frac{n^2}{b} \ge n^{3/2}$ free parameters (the entries in the blocks of the block-diagonal matrix $\vP_{(b,n)}^\top \vL \vP_{(b,n)}$ can be arbitrary, and there are $b$ such blocks each of size $\frac{n}{b}$). Similarly, in the case $b > \sqrt{n}$, we can set $\vL$ to the identity, and $\vM$ has at least $nb \ge n^{3/2}$ free parameters (the entries of the block-diagonal matrix $\vR$ can be arbitrary, and there are $nb$ total of these). Thus, at least $n^{3/2}$ parameters are required to uniquely describe any matrix in $\M\ind{b,n}$. However, a butterfly matrix in $\B\ind{n}$ has only $2n \log_2 n$ parameters. For $n > 256$, $2n \log_2 n < n^{3/2}$. (Note that this analysis is not tight: a more careful analysis can show the inclusion is strict even for smaller values of $n$.)

\end{proof}

We end this section with a theorem on the expressivity of the ``monarch hierarchy'' (products of monarch matrices), which follows from Theorem 1 of \cite{dao2020kaleidoscope}.
\begin{theorem}[Monarch hierarchy expressivity]
\label{thm:monarch_hierarchy}
Let $\vM$ be an $n \times n$ matrix such that matrix-vector multiplication of $\vM$ and an arbitrary vector $\vv$ (i.e., computation of $\vM \vv$) can
be represented as a linear arithmetic circuit with depth $d$ and $s$ total gates. Let $b \in (1, n)$ be a power of 2 that divides $n$.
Then, $\vM \in (\M\M^{*(b, n)})^{O(d)}_{O(s/n)}$.
\end{theorem}
\begin{proof}
Theorem 1 of \citet{dao2020kaleidoscope} says that if $n$ is a power of 2 and $\vA$ is an $n \times n$ matrix such that multiplying any vector $v$ by $\vA$ can be represented as a linear arithmetic circuit with depth $\le d$ and $\le s$ total gates, then $\vA \in (\B\B^{*(n)})^{O(d)}_{O(s/n)}$ (this is the ``kaleidoscope representation'' of $\vA$).

Recall from \cref{thm:b_contained} that for any $b \in (1, n)$ that is a power of 2 and divides $n$, $\M\ind{b, n} \supset \B\ind{n}$; thus, this implies $\M\M^{*(b,e\cdot n)} \supset \B\B^{*(e\cdot n)}$, and in turn $(\M\M^{*(b,n)})^w_e \supset (\B\B^{*(n)})^w_e$.

As $\vA \in  (\B\B^{*(n)})^{O(d)}_{O(s/n)}$, we thus have $\vA \in  (\M\M^{*(b,n)})^{O(d)}_{O(s/n)}$.
\end{proof}

As per \cite{dao2020kaleidoscope}, the class of kaleidoscope matrices $(\B\B^{*(n)})^{O(d)}_{O(s/n)}$ has $O(ds \log s)$ parameters and runtime, compared to the $O(s)$ parameters and runtime of the circuit. Note that at worst, $s$ is $O(n^2)$.

Define $f(n,s)$ to be the largest power of 2 that is $\le \min\left\{\ff{n}{2}, \sqrt{s}\right\}$. Note that $f(n,s) = O(\sqrt{s})$, and since $s = O(n^2)$, $f(n,s) = \Omega(\sqrt{s})$, so $f(n,s) = \Theta(\sqrt{s})$. %
We thus have $\vA \in (\M\M^{*(f(n,s), n)})^{O(d)}_{O(s/n)}$. The class $(\M\M^{*(f(n,s), n)})^{O(d)}_{O(s/n)}$ has $O(d\frac{s^2}{f(n,s)} + dsf(n,s)) = O(ds^{3/2})$ parameters. Thus, the monarch representation of $\vA$ is suboptimal by at most an $O(d\sqrt{s})$ factor compared to the $O(d{}\,\log s)$ of kaleidoscope.

\subsection{General rectangular matrices}
\label{sec:Monarch_rect}
In this section, we extend the Monarch parametrization to apply to \emph{rectangular} matrices, and prove some basic properties of the relevant matrix classes. (Note that our subsequent theoretical results (\cref{sec:proofs}) do not depend on this section, as they focus on the square parametrization.)

For the rest of the section, we will assume that $n_1, n_2, n_3, b_1, b_2 , b_3 \ge 1$ are integers such that:
\begin{itemize}
\item $b_i$ divides $n_i$ for all $1\le i\le 3$, and 
\item $\frac{n_1}{b_1} = \frac{n_2}{b_2}$.
\end{itemize}

We begin with the definition of the following class of rectangular block-diagonal matrices:
\begin{definition}
\label{def:monarch_rectangular}
For $0\le i< \frac{n}{b_1}$, let $\vR_{i}\in\F^{b_2 \times b_1}$ be a $b_2 \times b_1$ matrix. Then define the matrix $\vR \in \F^{n_2\times n_1}$ as follows:
\begin{equation}
 \label{eq:def-rect-R}
  \vR = \diag\left(\vR_0, \dots, \vR_{\frac {n_1}{b_1}-1}\right).
\end{equation}
\end{definition}
We say that $\vR$ has {\em block size} $b_2 \times b_1$. Recall that we have assumed $\frac {n_1}{b_1}=\frac{n_2}{b_2}$, so~\cref{eq:def-rect-R} is well-defined.
(Note that the number of possible nonzero values in $\vR$ is $\frac {n_1}{b_1}\cdot b_1 \times b_2 =n_1b_2$.)
We denote the class of all matrices $\vR$ expressible in this form by $\BD\ind{b_2 \times b_1, n_2 \times n_1}$.
Note that this class is only defined when $\frac {n_1}{b_1}=\frac{n_2}{n_2}$.



We restate the above definition equivalently as:
\begin{proposition} \label{prop:rect-R-eqv-def} 
$\vR\in\F^{n_2\times n_1}$ is in $\BD\ind{b_2 \times b_1, n_2 \times n_1}$ (with $\frac {n_1}{b_1}=\frac{n_2}{n_2}$) if and only if the following holds for any
$0\le i < n_2$ and $0\le j< n_1$. Let $i\equiv\baseb{i_1}{i_0}{b_2}$ and $j\equiv\baseb{j_1}{j_0}{b_1}$ (recalling this notation from \cref{def:$i$}.  Then
\begin{enumerate}
    \item\label{item:rect-zero-loc-R} if $i_1\ne j_1$, then $\vR[i,j]=0$. 
    \item \label{item:rect-nonzero-loc-R} Else (i.e., when $i_1=j_1$), then $\vR[i,j]=\vR_{i_1}[i_0,j_0]$.
\end{enumerate}

\end{proposition}

Before we define the rectangular $\vL$, we first need to define the notion of a `wrapped diagonal' matrix:
\begin{definition} 
\label{def:wrapped-diag}
A {\em wrapped diagonal} matrix $\mx{S} \in\F^{b_3\times b_2}$ is defined as follows. First assume $b_2\le b_3$. Then for any $0\le i<b_3$ and $0\le j<b_2$, we have the following. If $i\mod{b_2}\ne j$, then $\vS[i,j]=0$. (If $b_2>b_3$, then instead apply the previous definition to $\vS^{\top}$.)

\end{definition}


We now define the following class of block matrices with each block a \emph{wrapped diagonal} matrix.
\begin{definition}\label{def:rect-Matrix L}
Let $\vL\in\F^{n_3\times n_2}$ have the form: 
    \begin{equation}
        	\label{eq:rect-def-L}
    \vL=
    	\begin{bmatrix}
    		\mx{S}_{0,0} & \dots & \mx{S}_{0,\frac{n_2}{b_2} -1} \\
    		\vdots & \ddots & \vdots \\
    		\mx{S}_{\frac{n_3}{b_3} -1,0} & \dots & \mx{S}_{\frac{n_3}{b_3} -1,\frac{n_2}{b_2} -1}
    	\end{bmatrix},
    \end{equation}
where each $\vS_{\cdot,\cdot}$ is a wrapped diagonal matrix in $\F^{b_3 \times b_2}$.
\end{definition}
We say that $\vL$ has {\em block size} $b_3 \times b_2$.
(Note that the number of possible nonzero values in $\vL$ is $\parens{\frac {n_2}{b_2}\cdot \frac{n_3}{b_3}} \max(b_2,b_3)=\frac{n_2 \cdot n_3}{\min(b_2,b_3)}$.)
We denote the class of all matrices $\vL$ expressible in this form by $\DB\ind{b_3 \times b_2, n_3 \times n_2}$.

We restate the above definition equivalently as:


\begin{proposition}
\label{prop:rect-L-eqv-def}
$\vL\in\F^{n_3\times n_2}$ is in $\DB\ind{b_3 \times b_2, n_3 \times n_2}$ if and only if the following holds for any
$0\le i < n_3$ and $0 \le j< n_2$. Let $i\equiv\baseb{i_1}{i_0}{b_3}$ and $j\equiv\baseb{j_1}{j_0}{b_2}$.  Assuming $b_2 \le b_3$, we have:
\begin{enumerate}
    \item\label{item:rect-zero-loc-L} if $i_0\mod{b_2}\ne j_0$, then $\vL[i,j]=0$. 
    \item \label{item:rect-nonzero-loc-L} Else, (i.e., when $i_0\mod{b_2}=j_0$), then $\vL[i,j]=\vS_{i_1,j_1}[i_0,j_0]$.
\end{enumerate}
If $b_2>b_3$, then in the above, the condition ``$i_0\mod{b_2}\ne j_0$'' gets replaced by ``$j_0\mod{b_2}\ne i_0$.''
\end{proposition}

Using the above definitions, we now define the class of rectangular Monarch matrices.
\begin{definition}[Rectangular Monarch Matrix]
\label{def:block_Monarch}
Let $\vM \in \F^{n_3 \times n_1}$ be a matrix of the form: 
    \begin{equation}
        	\label{eq:Monarch-general}
    \vM= \vL \vR
    \end{equation}
    where $\vL \in \DB\ind{b_3 \times b_2, n_3 \times n_2}$ and $\vR \in \BD\ind{b_2 \times b_1, n_2 \times n_1}$.
\end{definition}
(As mentioned before, we assume $b_i$ divides $n_i$ for $i = 1,2,3$ and that $n_1/b_1 = n_2/b_2$.)
We denote the class of all matrices $\vM$ expressible in this form by $\M\ind{(b_1,b_2,b_3), (n_1,n_2,n_3)}$. Observe that when $b_1 = b_2 = b_3 = b$ and $n_1 = n_2 = n_3 = n$, this is exactly the matrix class $\M\ind{b, n}$ in \cref{def:block_Monarch}.

We are now ready to prove our main result in this section, which essentially follows from the observation that if we permute the rows and columns of $\vL$ such that the row/column block size in $\vL$ becomes the number of row/columns blocks in the permuted matrix (and vice-versa) then the permuted matrix has the form of $\vR$.


\begin{theorem} Let $1\le b,n_2,n_3$ be such that $b$ divides $n_2$ and $n_3$.
Suppose $\vL\in\F^{n_3\times n_2} \in \DB\ind{b \times b, n_3 \times n_2}$.
Then if we define
\[\vR'=\vP_{(b,n_3)}\cdot\vL\cdot\vP_{(b,n_2)}^\top,\]
we have that $\vR' \in \BD\ind{\frac{n_3}{b_3} \times \frac{n_2}{b_2}, n_3 \times n_2}$.
\end{theorem}


\begin{proof}
We recall  that multiplying an $m\times n$ matrix on the right (and left resp.) by $\vP_{(b,n)}^\top = \vP_{(\frac nb,n)}$ (and $\vP_{(b,m)}$ resp.) permutes the columns (and rows resp.) of the matrix according to $\sigma_{(b,n)}$ (and $\sigma_{(b,m)}$) respectively.\footnote{This uses the fact that $\parens{\sigma_{(b,n)}}^{-1}=\sigma_{(\frac nb,n)}$.} This implies that for any $0\le i,j<n$:
\begin{equation}
\label{eq:rect-L-permuted}
\vR'[\sigma_{(b,n_3)}(i),\sigma_{(b,n_2)}(j)]=\vL[i,j].
\end{equation}


Recall that in the notation of \cref{def:rect-Matrix L} we have $b_2=b_3=b$, so we are in the $b_2 \le b_3$ case.
To complete the proof, we will argue that $\vR'$ satisfies the two conditions in~\Cref{prop:rect-R-eqv-def}.\footnote{Note that we also need that the ratios of the row/column length to the row/column block sizes are the same; i.e., in our case we need that $\frac{n_3}{n_3 / b_3}=\frac{n_2}{n_2 / b_2}$, which is true because $b_2=b_3=b$.}

Towards this end, let $0\le i,j<n$ be arbitrary indices and further, define $i=\baseb{i_1}{i_0}{b}$ and $j=\baseb{j_1}{j_0}{b}$. Then note that $\sigma_{(b,n_3)}(i)=\baseb{i_0}{i_1}{\frac {n_3}{b}}$ and $\sigma_{(b,n_2)}(j)=\baseb{j_0}{j_1}{\frac {n_2}{b}}$.

By~\Cref{prop:rect-L-eqv-def}, we have that if $i_0\mod{b}\ne j_0$, then $\vL[i,j]=0$. Note  that since $i_0,j_0<b$ by definition, the condition $i_0\mod{b}\ne j_0$ is equivalent to saying $i_0\ne j_0$. Note that $i_0\ne j_0$ satisfies the pre-condition for base size $\frac {n_3}{b}\times \frac{n_2}{b}$ for indices $(\sigma_{(b,n_3)}(i),\sigma_{(b,n_2)}(j))$ in item~\ref{item:rect-zero-loc-R} in~\Cref{prop:rect-R-eqv-def}.  Then   by~\cref{eq:rect-L-permuted}, we have that $\vR'[\sigma_{(b,n_3)}(i),\sigma_{(b,n_2)}(j)]=0$, which satisfies item~\ref{item:rect-zero-loc-R} in~\Cref{prop:rect-R-eqv-def}.

Now consider the case that $i_0=j\mod b$, which by the observation in the above paragraph is the same as $i_0=j_0$. Then by item~\ref{item:rect-nonzero-loc-L} in~\Cref{prop:rect-L-eqv-def}, we have that $\vL[i,j]=\vS_{i_1,j_1}[i_0,j_0]$.  Note that $i_0= j_0$ satisfies the pre-condition for base size $\frac{n_3}{b}\times \frac{n_2}{b}$ for indices $(\sigma_{(b,n_3)}(i),\sigma_{(b,n_2)}(j))$ in item~\ref{item:rect-nonzero-loc-R} in~\Cref{prop:rect-R-eqv-def} if we define $\vR'_{i_0}\in\F^{\frac{n_3}{b}\times\frac{n_2}{b}}$ as follows:
\[\vR'_{i_0}[i_1,j_1]=\vS_{i_1,j_1}[i_0,j_0].\] 

Note that the above implies that 
\[\vR'=\diag\parens{\vR'_0,\dots,\vR'_{b-1}},\]
where $\vR'_{\cdot}$ is as defined in the above paragraph.
This means $\vR' \in \BD\ind{\frac{n_3}{b} \times \frac{n_2}{b}, n_3 \times n_2}$, since $\vR'$ has size $n_3 \times n_2$ and each block $\vR_{i_0}'$ is a matrix of size $\frac{n_3}{b} \times \frac{n_2}{b}$.
\end{proof}

\section{Proofs}

\subsection{Well-behaved activations}
%
The proof of our main results applies to activations that are decent,
i.e.\ well-behaved, in a sense defined in the sequel.  We then show that
$C$-bounded activations as well as the ReLU activation are decent. We
first need to extend the definition of the dual activation and kernel to apply
to vectors in $\reals^d$, rather than just $\mathbb{S}^d$. We denote by
$\cm_+$  the collection of $2\times 2$ positive semi-define matrices and by
$\cm_{++}$ the collection of positive definite matrices.
\begin{definition}
Let $\sigma$ be an activation. Define the following,
\[
\bar\sigma:\cm_{+}^2\to\reals ~~ , ~~
\bar\sigma(\Sigma)=\E_{(X,Y)\sim\gaussian(0,\Sigma)}\sigma(X)\sigma(Y) ~~ , ~~
k_\sigma(\x,\y)=\bar\sigma\begin{pmatrix}
\|\x\|^2 & \inner{\x,\y}
\\
\inner{\x,\y} & \|\y\|^2
\end{pmatrix} \,.
\]
\end{definition}
\noindent
We underscore the following properties of the extension of a
dual activation.
\begin{enumerate}[label=(\alph*)]
\item The following equality holds,
	$$\hat\sigma(\rho)=\bar\sigma\begin{pmatrix}
		1 & \rho \\
		\rho & 1
	\end{pmatrix}$$

\item The restriction of the extended $k_\sigma$ to the sphere agrees
with the restricted definition.

\item The extended dual activation and kernel are defined for every
	activation $\sigma$ such that for all $a\ge 0$, $x\mapsto \sigma(ax)$ is
	square integrable with respect to the Gaussian measure.

\item For $\x,\y\in\reals^d$, if $\w\in\reals^d$ is a multivariate normal
	distribution with zero mean vector and identity covariance matrix,
	then
	$$k_\sigma(\x,\y)=\E_{\w}\sigma(\inner{\w,\x})\sigma(\inner{\w,\y}) \,.$$
\end{enumerate}
Denote
$$\cm^\gamma_+:=\left\{\begin{pmatrix}
\Sigma_{11} & \Sigma_{12}\\
\Sigma_{12} & \Sigma_{22}
\end{pmatrix}\in \cm_+\mid 1-\gamma\le \Sigma_{11},\Sigma_{22}
	\le 1+\gamma\right\} \,. $$
\begin{definition}
A normalized activation $\sigma$ is {\em
$(\alpha,\beta,\gamma)$-decent} for $\alpha,\beta,\gamma\ge 0$ if the
following conditions hold.
\begin{enumerate}[label=(\roman*)]
\item The dual activation $\bar\sigma$ is $\beta$-Lipschitz in
	$\cm_+^\gamma$ with respect to the $\infty$-norm.

\item If $(X_1,Y_1),\ldots,(X_r,Y_r)$ are independent samples from
	$\gaussian\left(0,\Sigma\right)$ for $\Sigma\in \cm_+^\gamma$ then
\[
\Pr\left(\left|\frac{\sum_{i=1}^r\sigma(X_i)\sigma(Y_i)}{r} -
	\bar\sigma(\Sigma)\right|\ge\epsilon\right)
	\le 2\exp\left(-\frac{r\epsilon^2}{2\alpha^2}\right) \,.
\]
\end{enumerate}
\end{definition}

\begin{lemma}[Bounded activations are decent]
	\label{lem:bounded_are_decent}
Let $\sigma:\reals\to\reals$ be a $C$-bounded normalized activation. Then,
$\sigma$ is $(C^2,2C^2,\gamma)$-decent for all $\gamma \ge 0$.
\end{lemma}
\proof
It is enough to show that the following properties hold.
\begin{enumerate}
\item The (extended) dual activation $\bar\sigma$ is $2C^2$-Lipschitz in
	$\cm_{++}$ w.r.t.\ the $\infty$-norm.
\item If $(X_1,Y_1),\ldots,(X_r,Y_r)$ are
 independent samples from $\gaussian\left(0,\Sigma\right)$ then
\[
\Pr\left(\left|\frac{\sum_{i=1}^r\sigma(X_i)\sigma(Y_i)}{r}-\bar\sigma(\Sigma)\right|\ge\epsilon\right) \le 2\exp\left(-\frac{r\epsilon^2}{2C^4}\right)
\]
\end{enumerate}
\noindent
From the boundedness of $\sigma$ it holds that $|\sigma(X)\sigma(Y)| \leq
C^2$. Hence, the second property follows directly from Hoeffding's bound.
We next prove the first part. Let $\z=(x,y)$ and
$\phi(\z) = \sigma(x)\sigma(y)$. Note that for
$\Sigma\in\cm_{++}$ we have
$$\bar\sigma(\Sigma) =
	\frac{1}{2\pi\sqrt{\det(\Sigma)}}
	\int_{\reals^2}\phi(\z)e^{-\frac{\z^\top\Sigma^{-1}\z}{2}}d\z \,.$$
Thus we get that,
\begin{eqnarray*}
\frac{\partial \bar\sigma}{\partial \Sigma} &=&
	\frac{1}{2\pi}\int_{\reals^2}
		\phi(\z)\left[
			\frac{\frac{1}{2}\sqrt{\det(\Sigma)}\Sigma^{-1} -
				\frac{1}{2}\sqrt{\det(\Sigma)}(\Sigma^{-1}\z\z^\top\Sigma^{-1})}
				{\det(\Sigma)}
						\right]
		e^{-\frac{\z^\top\Sigma^{-1}\z}{2}}d\z
\\
&=& \frac{1}{2\pi\sqrt{\det(\Sigma)}}\int_{\reals^2}\phi(\z)\frac{1}{2}\left[
\Sigma^{-1}-\Sigma^{-1}\z\z^\top\Sigma^{-1}
\right]e^{-\frac{\z^\top\Sigma^{-1}\z}{2}}d\z
\end{eqnarray*}
Let $g(\z)=e^{-\frac{\z^\top\Sigma^{-1}\z}{2}}$. Then, the first and second
order partial derivatives of $g$ are
\begin{eqnarray*}
\frac{\partial g}{\partial \z} & = &
	-\Sigma^{-1}\z e^{-\frac{\z^\top\Sigma^{-1}\z}{2}} \\
\frac{\partial^2 g}{\partial^2 \z} & = &
	\left[-\Sigma^{-1} +
		\Sigma^{-1}\z\z^\top\Sigma^{-1}\right]e^{-\frac{\z^\top\Sigma^{-1}\z}{2}} \,.
\end{eqnarray*}
We therefore obtain that,
\[
\frac{\partial \bar\sigma}{\partial \Sigma} =
	-\frac{1}{4\pi\sqrt{\det(\Sigma)}} \int_{\reals^2}
		\phi\frac{\partial^2 g}{\partial^2 \z} d\z \,.
\]
By the product rule we have
\[
\frac{\partial \bar\sigma}{\partial \Sigma} =
-\frac{1}{2\pi\sqrt{\det(\Sigma)}}\frac{1}{2}\int_{\reals^2}\frac{\partial^2
\phi}{\partial^2 \z} gd\z = -\frac{1}{2}\E_{(X,Y)\sim\gaussian(0,\Sigma)}\left[\frac{\partial^2 \phi}{\partial^2 \z}(X,Y)\right]
\]
We conclude that $\bar\sigma$ is differentiable in $\cm_{++}$ with
partial derivatives that are point-wise bounded by $\frac{C^2}{2}$. Thus,
$\bar\sigma$ is $2C^2$-Lipschitz in $\cm_+$ w.r.t.\ the $\infty$-norm. \qed

\medskip
We next show that the ReLU activation is decent.
\begin{lemma}[ReLU is decent]\label{lem:relu_is_decent}
%
There exists a constant $\alpha_\mathrm{ReLU}\ge 1$ such that for
$0\le \gamma\le 1$, the normalized ReLU activation
$\sigma(x)=\sqrt{2}\max(0,x)$ is
$(\alpha_\mathrm{ReLU},1+o(\gamma),\gamma)$-decent.
\end{lemma}
\proof
The measure concentration property follows from standard concentration
bounds for sub-exponential random variables (e.g.\ ~\cite{shalev2014understanding}).  It
remains to show that $\bar\sigma$ is $(1+o(\gamma))$-Lipschitz in
$\cm^\gamma_+$. We first calculate an exact expression for $\bar\sigma$.
The expression was already calculated in~\cite{cho2009kernel}, yet we give
here a derivation for completeness.
%
\begin{claim}\label{claim:relu_dual_ext}
The following equality holds for all $\Sigma\in\cm_{+}^2$,
$$\bar\sigma(\Sigma) = \sqrt{\Sigma_{11}\Sigma_{22}} \,
	\hat\sigma\!\left(\frac{\Sigma_{12}}{\sqrt{\Sigma_{11}\Sigma_{22}}}\right)
\,. $$
\end{claim}
\proof Let us denote
$$\tilde{\Sigma} = \begin{pmatrix}
1 & \frac{\Sigma_{12}}{\sqrt{\Sigma_{11}\Sigma_{12}}} \\
\frac{\Sigma_{12}}{\sqrt{\Sigma_{11}\Sigma_{12}}} & 1
\end{pmatrix} \,. $$
By the positive homogeneity of the ReLU activation we have
\begin{eqnarray*}
\bar\sigma\left(\Sigma\right) &=&
	\E_{(X,Y)\sim\gaussian(0,\Sigma)}\sigma(X)\sigma(Y) \\
&=& \sqrt{\Sigma_{11}\Sigma_{22}}
		\E_{(X,Y)\sim\gaussian(0,\Sigma)}
			\sigma\!\left(\frac{X}{\sqrt{\Sigma_{11}}}\right)
			\sigma\!\left(\frac{Y}{\sqrt{\Sigma_{22}}}\right) \\
&=& \sqrt{\Sigma_{11}\Sigma_{22}}
			\E_{(\tilde{X},\tilde{Y})\sim\gaussian\left(0,\tilde{\Sigma}\right)}
			\sigma\!\left(\tilde{X}\right)\sigma\!\left(\tilde{Y}\right) \\
&=& \sqrt{\Sigma_{11}\Sigma_{22}}\,
	\hat{\sigma}\!\left(\frac{\Sigma_{12}}{\sqrt{\Sigma_{11}\Sigma_{22}}}\right)
	\, .
\end{eqnarray*}
which concludes the proof. \qed

\medskip

For brevity, we henceforth drop the argument from $\bar{\sigma}(\Sigma)$ and
use the abbreviation $\bar{\sigma}$. In order to show that $\bar\sigma$ is
$(1+o(\gamma))$-Lipschitz w.r.t.\ the $\infty$-norm it is enough to show that
for every $\Sigma\in\cm_+^\gamma$ we have,
\begin{equation}\label{eq:grad_l1_bound}
\|\nabla \bar \sigma\|_1 =
	\left|\frac{\partial \bar\sigma}{\partial \Sigma_{12}}\right| +
	\left|\frac{\partial \bar\sigma}{\partial \Sigma_{11}}\right| +
	\left|\frac{\partial \bar\sigma}{\partial \Sigma_{22}}\right|\le
		1+o(\gamma) \,.
\end{equation}
First, Note that ${\partial \bar\sigma}/{\partial \Sigma_{11}}$ and
${\partial \bar\sigma}/{\partial \Sigma_{22}}$ have the same sign,
hence,
$$\|\nabla \bar \sigma\|_1 =
	\left|\frac{\partial \bar\sigma} {\partial \Sigma_{12}}\right| +
	\left|\frac{\partial \bar\sigma}{\partial \Sigma_{11}} +
				\frac{\partial \bar\sigma}{\partial \Sigma_{22}}\right| \,.$$
Next we get that,
\begin{eqnarray*}
\frac{\partial \bar\sigma}{\partial \Sigma_{11}} & = &
	\frac{1}{2}\sqrt{\frac{\Sigma_{22}}{\Sigma_{11}}}\,
	\hat\sigma\!\left(\frac{\Sigma_{12}}{\sqrt{\Sigma_{11}\Sigma_{22}}}\right) -
	\frac{1}{2}\sqrt{\frac{\Sigma_{22}}{\Sigma_{11}}}
		\frac{\Sigma_{12}}{\sqrt{\Sigma_{11}\Sigma_{22}}}\,
		\hat\sigma'\!\left(\frac{\Sigma_{12}}{\sqrt{\Sigma_{11}\Sigma_{22}}}\right)
\\
\frac{\partial \bar\sigma}{\partial \Sigma_{22}} & = &
	\frac{1}{2}\sqrt{\frac{\Sigma_{11}}{\Sigma_{22}}}\,
	\hat\sigma\!\left(\frac{\Sigma_{12}}{\sqrt{\Sigma_{11}\Sigma_{22}}}\right) -
	\frac{1}{2}\sqrt{\frac{\Sigma_{11}}{\Sigma_{22}}}
	\frac{\Sigma_{12}}{\sqrt{\Sigma_{11}\Sigma_{22}}}\,
	\hat\sigma'\!\left(\frac{\Sigma_{12}}{\sqrt{\Sigma_{11}\Sigma_{22}}}\right)
\\
\frac{\partial \bar\sigma}{\partial \Sigma_{12}} & = &
	\hat\sigma'\!\left(\frac{\Sigma_{12}}{\sqrt{\Sigma_{11}\Sigma_{22}}}\right)
	\, .
\end{eqnarray*}
We therefore get that the $1$-norm of $\nabla\bar\sigma$ is,
\[
\|\nabla \bar \sigma\|_1 =
\frac{1}{2}\frac{\Sigma_{11}+\Sigma_{22}}{\sqrt{\Sigma_{11}\Sigma_{22}}}
\left|
	\hat\sigma\!\left(\frac{\Sigma_{12}}{\sqrt{\Sigma_{11}\Sigma_{22}}}\right) -
	\frac{\Sigma_{12}}{\sqrt{\Sigma_{11}\Sigma_{22}}}\,
	\hat\sigma'\!\left(\frac{\Sigma_{12}}{\sqrt{\Sigma_{11}\Sigma_{22}}}\right)
\right| +
	\hat\sigma'\!\left(\frac{\Sigma_{12}}{\sqrt{\Sigma_{11}\Sigma_{22}}}\right)
	\,.
\]
The gradient of
$\frac{1}{2}\frac{\Sigma_{11}+\Sigma_{22}}{\sqrt{\Sigma_{11}\Sigma_{22}}}$
at $(\Sigma_{11},\Sigma_{22})=(1,1)$ is $(0,0)$. Therefore, from the mean
value theorem we get,
$\frac{1}{2}\frac{\Sigma_{11}+\Sigma_{22}}{\sqrt{\Sigma_{11}\Sigma_{22}}} =
	1+o(\gamma)$.
Furthermore, $\hat\sigma$, $\hat\sigma'$ and
$\frac{\Sigma_{12}}{\sqrt{\Sigma_{11}\Sigma_{22}}}$ are bounded by $1$ in
absolute value. Hence, we can write,
\[
\|\nabla \bar \sigma\|_1 =
\left|
\hat\sigma\!\left(\frac{\Sigma_{12}}{\sqrt{\Sigma_{11}\Sigma_{22}}}\right) -
\frac{\Sigma_{12}}{\sqrt{\Sigma_{11}\Sigma_{22}}}
\hat\sigma'\!\left(\frac{\Sigma_{12}}{\sqrt{\Sigma_{11}\Sigma_{22}}}\right)
\right| +
\hat\sigma'\!\left(\frac{\Sigma_{12}}{\sqrt{\Sigma_{11}\Sigma_{22}}}\right)
+ o(\gamma) \,.
\]
Finally, if we let $t=\frac{\Sigma_{12}}{\sqrt{\Sigma_{11}\Sigma_{22}}}$,
we can further simply the expression for $\nabla\bar\sigma$,
\begin{eqnarray*}
\|\nabla \bar \sigma(\Sigma)\|_1 &=& |\hat\sigma(t)-t\hat\sigma'(t)| + |\hat\sigma'(t)|  + o(\gamma)
\\
&=& \frac{\sqrt{1-t^2}}{\pi} + 1 - \frac{\cos^{-1}(t)}{\pi}  + o(\gamma) \,.
\end{eqnarray*}
Finally, the proof is obtained from the fact that the function $f(t)=\frac{\sqrt{1-t^2}}{\pi} + 1 - \frac{\cos^{-1}(t)}{\pi}$ satisfies $0\le f(t)\le 1$ for every $t\in [-1,1]$.
Indeed, it is simple to verify that $f(-1)=0$ and $f(1)=1$. Hence, it suffices to
show that $f'$ is non-negative in $[-1,1]$ which is indeed the case since,
\[
f'(t) = \frac{1}{\pi}\frac{1-t}{\sqrt{1-t^2}} =
	\frac{1}{\pi}\sqrt{\frac{1-t}{1+t}} \ge 0 \,. \qedhere
\]

\subsection{Proofs of Thms.~\ref{thm:main_ker}~and~\ref{thm:main_ker_ReLU}}
We start by an additional theorem which serves as a simple stepping stone
for proving the aforementioned main theorems.
\begin{theorem}\label{thm:ker_appr}
%
Let $\cs$ be a skeleton with $(\alpha,\beta,\gamma)$-decent
activations, $0<\epsilon \le \gamma$, and
$B_d = \sum_{i=0}^{d-1}\beta^i$. Let $\w$ be a random initialization of the
network $\cn=\cn(\cs,r)$ with
$$r \ge \frac{2\alpha^2B_{\depth(\cs)}^2
	\log\left(\frac{8|\cs|}{\delta}\right)} {\epsilon^2} \,. $$
Then, for every $\x,\y$ with probability of at least $1-\delta$, it holds that
\[
|\kappa_\w(\x,\y)-\kappa_\cs(\x,\y)|\le \epsilon \,.
\]
\end{theorem}
\noindent
Before proving the theorem we show that together with
Lemmas~\ref{lem:bounded_are_decent}~and~\ref{lem:relu_is_decent},
Theorems~\ref{thm:main_ker}~and~\ref{thm:main_ker_ReLU} follow from
Theorem~\ref{thm:ker_appr}. We restate them as corollaries, prove them,
and then proceed to the proof of Theorem \ref{thm:ker_appr}.
\begin{corollary}
Let $\cs$ be a skeleton with $C$-bounded activations. Let $\w$ be a random
initialization of $\cn=\cn(\cs,r)$ with
$$r \ge \frac{(4C^4)^{\depth(\cs)+1}
\log\left(\frac{8|\cs|}{\delta}\right)}{\epsilon^2} \,.$$
Then, for every $\x,\y$, w.p.\ $\ge 1-\delta$,
\[
|\kappa_\w(\x,\y)-\kappa_\cs(\x,\y)|\le \epsilon\,.
\]
\end{corollary}
\proof
From Lemma~\ref{lem:bounded_are_decent}, for all $\gamma>0$, each
activation is $(C^2,2C^2,\gamma)$-decent. By Theorem
\ref{thm:ker_appr}, it suffices to show that
$$2\left(C^2\right)^2\left(\sum_{i=0}^{\depth(\cs)-1}(2C^2)^{i}\right)^2
	\le (4C^4)^{\depth(\cs)+1} \,. $$
The sum of can be bounded above by,
\[
\sum_{i=0}^{\depth(\cs)-1}\!\!\!(2C^2)^{i} =
	\frac{(2C^2)^{\depth(\cs)}-1}{2C^2-1} \le
	\frac{(2C^2)^{\depth(\cs)}}{C^2} \,.
\]
Therefore, we get that,
\[
2\left(C^2\right)^2\left(\sum_{i=0}^{\depth(\cs)-1}\!\!\!(2C^2)^{i}\right)^2
	\le \frac{2C^4(4C^4)^{\depth(\cs)}}{C^4} \le (4C^4)^{\depth(\cs)+1} \,,
\]
which concludes the proof. \qed

\begin{corollary}
Let $\cs$ be a skeleton with ReLU activations, and $\w$ a random
initialization of $\cn(\cs,r)$ with $r \ge c_1 \frac{\depth^2(\cs)
\log\left(\frac{8|\cs|}{\delta}\right)}{\epsilon^2}$. For all $\x,\y$ and
$\epsilon\le \min(c_2,\frac{1}{\depth(\cs)})$, w.p.\ $\ge 1-\delta$,
	\[
	|\kappa_\w(\x,\y)-\kappa_\cs(\x,\y)|\le \epsilon
	\]
	Here, $c_1,c_2>0$ are universal constants.
\end{corollary}
\proof
From Lemma \ref{lem:relu_is_decent}, each activation is
$(\alpha_{\mathrm{ReLU}},1+o(\epsilon),\epsilon)$-decent. By Theorem
\ref{thm:ker_appr}, it is enough to show that
$$\sum_{i=0}^{\depth(\cs)-1}\!\!\!(1+o(\epsilon))^{i}=O(\depth(\cs)) \,.$$
This claim follows from the fact that
$(1+o(\epsilon))^{i}\le e^{o(\epsilon)\depth(\cs)}$ as long as
$i\le \depth(\cs)$. Since we assume that
$\epsilon\le{1}/{\depth(\cs)}$, the expression is bounded by $e$
for sufficiently small $\epsilon$.
\proofbox

\medskip

\noindent We next prove Theorem \ref{thm:ker_appr}.
\proof (Theorem \ref{thm:ker_appr})
For a node $u\in\cs$ we denote by $\Psi_{u,\w}:\cx\to\reals^r$ the normalized
representation of $\cs$'s sub-skeleton rooted at $u$.
Analogously, $\kappa_{u,\w}$ denotes the empirical kernel of that network.
When $u$ is the output node of $\cs$ we still use $\Psi_{\w}$ and $\kappa_\w$
for $\Psi_{u,\w}$ and $\kappa_{u,\w}$. Given two fixed $\x,\y\in\cx$ and a node
$u\in\cs$, we denote
\[
\mathcal{K}_\w^u=
\begin{pmatrix}
\kappa_{u,\w}(\x,\x) &
\kappa_{u,\w}(\x,\y)
\\
\kappa_{u,\w}(\x,\y) &
\kappa_{u,\w}(\y,\y)
\end{pmatrix},\;\; \mathcal{K}^u = \begin{pmatrix}
\kappa_u(\x,\x) & \kappa_u(\x,\y)
\\
\kappa_u(\x,\y) & \kappa_u(\y,\y)
\end{pmatrix}
\]
\[
\mathcal{K}_\w^{\leftarrow u}=\frac{\sum_{v\in\IN(u)}\mathcal{K}^{v}_\w}{|\IN(u)|}
,\quad \mathcal{K}^{\leftarrow u}=\frac{\sum_{v\in\IN(u)}\mathcal{K}^{v}}{|\IN(u)|} \,.
\]
For a matrix $\mathcal{K}\in\cm_+$ and a function $f:\cm_+\to\reals$, we denote
\[
f^p(\mathcal{K})=\begin{pmatrix}
f\!\begin{pmatrix}
\mathcal{K}_{11} & \mathcal{K}_{11}
\\
\mathcal{K}_{11} & \mathcal{K}_{11}
\end{pmatrix} & f(\mathcal{K})
\\
f(\mathcal{K}) & f\!\begin{pmatrix}
\mathcal{K}_{22} & \mathcal{K}_{22}
\\
\mathcal{K}_{22} & \mathcal{K}_{22}
\end{pmatrix}
\end{pmatrix}
\]
Note that $\mathcal{K}^u=\bar\sigma_u^p(\mathcal{K}^{\leftarrow u})$.
We say that a node $u\in \cs$, is {\em well-initialized} if
\begin{equation}\label{eq:1}
\|\mathcal{K}_\w^u- \mathcal{K}^u\|_\infty
	\le \epsilon\frac{B_{\depth(u)}}{B_{\depth(\cs)}} \,.
\end{equation}
Here, we use the convention that $B_{0}=0$. It is enough to show that with
probability of at least $\ge 1-\delta$ all nodes are well-initialized. We first note that input nodes are well-initialized by construction since
$\mathcal{K}^u_\w=\mathcal{K}^u$. Next, we show that given that all incoming
nodes for a certain node are well-initialized, then w.h.p.\ the node is
well-initialized as well.
\begin{claim}\label{claim1}
Assume that all the nodes in $\IN(u)$ are well-initialized. Then, the node
$u$ is well-initialized with probability of at least $1-\frac{\delta}{|\cs|}$.
\end{claim}
\proof
It is easy to verify that $\mathcal{K}_\w^u$ is the empirical
covariance matrix of $r$ independent variables distributed according to
$\left(\sigma(X),\sigma(Y)\right)$ where
$(X,Y)\sim\gaussian\left(0,\mathcal{K}_\w^{\leftarrow u}\right)$.
Given the assumption that all nodes incoming to $u$ are well-initialized,
we have,
\begin{eqnarray}\label{eq:2}
\left\|\mathcal{K}_\w^{\leftarrow u}-\mathcal{K}^{\leftarrow u}\right\|_\infty
&=&
\left\|
	\frac{\sum_{v\in\IN(v)}\mathcal{K}_\w^{v}}{|\IN(v)|} -
	\frac{\sum_{v\in\IN(v)}\mathcal{K}^{v}}{|\IN(v)|}\right\|_\infty\nonumber
\\
&\le& \frac{1}{|\IN(v)|}\sum_{v\in\IN(v)}
	\left\|\mathcal{K}_\w^{v}-\mathcal{K}^{v}\right\|_\infty
\\
&\le&  \epsilon\frac{B_{\depth(u)-1}}{B_{\depth(\cs)}}\nonumber \,.
\end{eqnarray}
Further, since $\epsilon\le \gamma$ then
$\mathcal{K}^{\leftarrow u}_\w\in \cm_+^\gamma$. Using the fact
that $\sigma_u$ is
$(\alpha,\beta,\gamma)$-decent and that
$r\ge \frac{2\alpha^2 B^2_{\depth(\cs)}
	\log\left(\frac{8|\cs|}{\delta}\right)}{\epsilon^2}$,
we get that w.p.\ of at least $1- \frac{\delta}{|\cs|}$,
\begin{equation}\label{eq:3}
\left\|\mathcal{K}^u_\w -
	\bar\sigma^p_u\left(\mathcal{K}_\w^{\leftarrow u}\right)\right\|_\infty
	\le \frac{\epsilon}{B_{\depth(\cs)}} \,.
\end{equation}
Finally, using \eqref{eq:2}~and~\eqref{eq:3} along with the fact that
$\bar\sigma$ is $\beta$-Lipschitz, we have
\begin{eqnarray*}
\|\mathcal{K}_\w^{u}-\mathcal{K}^{u}\|_\infty &=&
\left\|\mathcal{K}_\w^{u} -
	\bar\sigma^p_u\left(\mathcal{K}^{\leftarrow u}\right)\right\|_\infty
\\
&\le & \left\|\mathcal{K}^u_\w -
	\bar\sigma^p_u\left(\mathcal{K}_\w^{\leftarrow u}\right)\right\|_\infty +
	\left\| \bar\sigma^p_u\left(\mathcal{K}_\w^{\leftarrow u}\right) -
	\bar\sigma^p_u\left(\mathcal{K}^{\leftarrow u}\right)\right\|_\infty
\\
&\le &   \frac{\epsilon}{B_{\depth(\cs)}} +
\beta\left\| \mathcal{K}_\w^{\leftarrow u} -
	\mathcal{K}^{\leftarrow u}\right\|_\infty
\\
&\le &  \frac{\epsilon}{B_{\depth(\cs)}} +
\beta \epsilon \frac{B_{\depth(u)-1}}{B_{\depth(\cs)}}
\;=\;  \epsilon \frac{B_{\depth(u)}}{B_{\depth(\cs)}} \,. \hspace{2cm} \qed
\end{eqnarray*}

We are now ready to conclude the proof. Let $u_1,\ldots,u_{|\cs|}$ be an ordered
list of the nodes in $\cs$ in accordance to their depth, starting with the
shallowest nodes, and ending with the output node. Denote by $A_q$ the event
that $u_1,\ldots, u_q$ are well-initialized. We need to show that
$\Pr(A_{|\cs|})\ge 1-\delta$. We do so using an induction on $q$ for the
inequality $\Pr(A_q)\ge 1-\frac{q\delta}{|\cs|}$. Indeed, for $q=1,\ldots,n$,
$u_q$ is an input node and $\Pr(A_q)=1$. Thus, the base of the induction
hypothesis holds. Assume that $q>n$. By Claim (\ref{claim1}) we have that
$\Pr(A_q|A_{q-1})\ge 1-\frac{\delta}{|\cs|}$. Finally, from the induction
hypothesis we have,
\[
\Pr(A_q) \geq \Pr(A_q|A_{q-1})\Pr(A_{q-1}) \ge
	\left(1-\frac{\delta}{|\cs|}\right)
	\left(1-\frac{(q-1)\delta}{|\cs|}\right) \ge
	1-\frac{q\delta}{|\cs|} \,. \qed
\]

\subsection{Proofs of Thms.~\ref{thm:main_dist}~and~\ref{thm:main_dist_ReLU}}
Theorems~\ref{thm:main_dist}~and~\ref{thm:main_dist_ReLU} follow from using
the following lemma combined with
Theorems~\ref{thm:main_ker}~and~\ref{thm:main_ker_ReLU}. When we apply
the lemma, we always focus on the special case where one of the kernels is
constant w.p.\ $1$.
\begin{lemma} 
	\label{lem:app_ker_app_act}
Let $\cd$ be a distribution on $\cx\times\cy$, $\ell:\reals\times\cy\to\reals$
be an $L$-Lipschitz loss, $\delta>0$, and $\kappa_1,\kappa_2:\cx\times\cx\to\reals$ be two
independent random kernels sample from arbitrary distributions.
Assume that the following properties hold.
	\begin{itemize}
		\item For some $C>0$, $\forall \x\in \cx,\;
			\kappa_1(\x,\x),\kappa_2(\x,\x)\le C$.
		\item $\forall \x,\y\in
			\cx,\;\Pr_{\kappa_1,\kappa_2}
				\left(|\kappa_1(\x,\y)-\kappa_2(\x,\y)|\ge \epsilon\right)\le \tilde{\delta}$
				for $\tilde{\delta} < c_2 \frac{\epsilon^2\delta}{C^2\log^2\left(\frac{1}{\delta}\right)}$ where $c_2>0$ is a
				universal constant.
	\end{itemize}
	Then, w.p.\ $\ge 1-\delta$ over the choices of $\kappa_1,\kappa_2$, for every
	$f_1\in \ch^{M}_{\kappa_1}$ there is $f_2\in\ch^{\sqrt{2}M}_{\kappa_2}$ such that $\cl_{\cd}(f_2)\le \cl_{\cd}(f_1) + \sqrt{\epsilon}4LM$.
\end{lemma}
\noindent
To prove the above lemma, we state another lemma below followed by a basic
measure concentration result.
\begin{lemma}\label{lem:small_norm_small_l1}
Let $\x_1,\ldots,\x_m\in \reals^d$, $\w^*\in\reals^d$ and
$\epsilon>0$. There are weights $\alpha_1,\ldots,\alpha_m$
such that for $\w:=\sum_{i=1}^m\alpha_i\x_i$ we have,
\begin{itemize}
	\item $\cl(\w):=\frac{1}{m}\sum_{i=1}^m|\inner{\w,\x_i}-\inner{\w^*,\x_i}|\le\epsilon$
	\item $\sum_i |\alpha_i| \le \frac{\|\w^*\|^2}{\epsilon}$
	\item $\|\w\| \le \|\w^*\|$
\end{itemize}
\end{lemma}
\proof
Denote $M=\|\w^*\|$, $C = \max_i \|\x_i\|$, and
$y_i=\inner{\w^*,\x_i}$. Suppose that we run stochastic gradient decent on
the sample $\{(\x_1,y_1),\ldots,(\x_m,y_m)\}$ w.r.t.\ the loss $\cl(\w)$, with
learning rate $\eta = \frac{\epsilon}{C^2}$, and with projections onto the
ball of radius $M$. Namely, we start with $\w_0=0$ and at each iteration
$t\ge 1$, we choose at random $i_t\in [m]$ and perform the update,
\[
\tilde{\w}_t = \begin{cases}
\w_{t-1}-\eta\x_{i_t} & \inner{\w_{t-1},\x_{i_t}} \ge y_{i_t}
\\
\w_{t-1}+\eta\x_{i_t} & \inner{\w_{t-1},\x_{i_t}} < y_{i_t}
\end{cases}
\]
\[
\w_t = \begin{cases}
\tilde{\w}_{t} & \|\tilde{\w}_{t}\|\le M
\\
\frac{M \tilde{\w}_{t}}{\|\tilde{\w}_{t}\|} & \|\tilde{\w}_{t}\| > M
\end{cases}
\]
After $T=\frac{M^2C^2}{\epsilon^2}$ iterations the loss in expectation would
be at most $\epsilon$ (see for instance Chapter 14 in
\cite{shalev2014understanding}). In particular, there exists a sequence of
at most $\frac{M^2C^2}{\epsilon^2}$ gradient steps that attains a solution
$\w$ with $\cl(\w)\le \epsilon$.  Each update adds or subtracts
$\frac{\epsilon}{C^2}\x_i$ from the current solution. Hence $\w$ can be
written as a weighted sum of $\x_i$'s where the sum of each coefficient
is at most $T\frac{\epsilon}{C^2}=\frac{M^2}{\epsilon}$.
\proofbox

\begin{theorem}[\citet{BartlettMe02}] \label{thm:radamacher}
Let $\cd$ be a distribution over $\cx\times\cy$, $\ell:\reals\times\cy\to\reals$
a $1$-Lipschitz loss, $\kappa:\cx\times\cx\to \reals$ a kernel, and
$\epsilon,\delta>0$. Let $S=\{(\x_1,y_1),\ldots,(\x_m,y_m)\}$ be i.i.d.\
samples from $\cd$ such that
$m \ge c
	\frac{M^2 \max_{\x\in\cx}\kappa(\x,\x)+\log\left(\frac{1}{\delta}\right)}
	{\epsilon^2}$ where $c$ is a constant.
Then, with probability of at least $1-\delta$ we have,
\[
\forall f\in \ch^M_{\kappa},\; |\cl_\cd(f) - \cl_S(f)| \le \epsilon \,.
\]
\end{theorem}
%%
\proof (of Lemma \ref{lem:app_ker_app_act})
By rescaling $\ell$, we can assume w.l.o.g that $L=1$.  Let
$\epsilon_1=\sqrt{\epsilon}M$ and $S=\{(\x_1,y_1),\ldots,(\x_m,y_m)\}\sim\cd$
be i.i.d.\ samples which are independent of the choice of
$\kappa_1,\kappa_2$. By Theorem \ref{thm:radamacher}, for a large enough
constant $c$, if $m=c \frac{C  M^2 \log\left(\frac{1}{\delta}\right)}{\epsilon_1^2}=c \frac{C\log\left(\frac{1}{\delta}\right)}{\epsilon}$,
then w.p.\ $\ge 1-\frac{\delta}{2}$ over the choice of the samples we have,
\begin{equation}\label{eq:4}
\forall f\in \ch_{\kappa_1}^{M}\cup\ch_{\kappa_2}^{\sqrt{2}M} ,\;|\cl_\cd(f) - \cl_{S}(f)|\le \epsilon_1
\end{equation}
Now, if we choose $c_2=\frac{1}{2c^2}$ then w.p.\ $\ge 1-m^2\tilde{\delta} \ge 1-\frac{\delta}{2}$
(over the choice of the examples and the kernel), we have that
\begin{equation}\label{eq:5}
\forall i,j\in [m], |\kappa_1(\x_i,\x_j)-\kappa_2(\x_i,\x_j)|< \epsilon \,.
\end{equation}
In particular, w.p.\ $\ge 1-\delta$ \eqref{eq:4} and \eqref{eq:5} hold and
therefore it suffices to prove the conclusion of the theorem under these
conditions. Indeed, let $\Psi_1,\Psi_2:\cx\to \ch$ be two mapping from $\cx$ to
a Hilbert space $\ch$ so that $\kappa_i(\x,\y)=\inner{\Psi_i(\x),\Psi_i(\y)}$.
Let $f_1\in\ch^M_{\kappa_1}$. By lemma \ref{lem:small_norm_small_l1} there are
$\alpha_1,\ldots,\alpha_m$ so that for the vector
$\w=\sum_{i=1}^m\alpha_1\Psi_1(\x_i)$ we have
\begin{equation}\label{eq:6}
\frac{1}{m}\sum_{i=1}^m|\inner{\w,\Psi_1(\x_i)}-f_1(\x_i)|\le
	\epsilon_1,\;\;\|\w\|\le M \,,
\end{equation}
and
\begin{equation}\label{eq:7}
\sum_{i=1}^m|\alpha_i|\le \frac{M^2}{\epsilon_1} \,.
\end{equation}
Consider the function $f_2\in \ch_2$ defined by $f_2(\x)=\sum_{i=1}^m\alpha_1\inner{\Psi_2(\x_i),\Psi_2(\x)}$. We note that
\begin{eqnarray*}
\|f_2\|^2_{\ch_{k_2}} &\le & \left\|\sum_{i=1}^m\alpha_i\Psi_2(\x_i)\right\|^2
\\
&=& \sum_{i,j=1}^m \alpha_i\alpha_j\kappa_2(\x_i,\x_j)
\\
&\le& \sum_{i,j=1}^m \alpha_i\alpha_j\kappa_1(\x_i,\x_j)+\epsilon\sum_{i,j=1}^m |\alpha_i\alpha_j|
\\
&=& \|\w\|^2+\epsilon \left(\sum_{i=1}^m |\alpha_i|\right)^2
\\
&\le& M^2+\epsilon \frac{M^4}{\epsilon^2_1} = 2M^2 \,.
\end{eqnarray*}
Denote by $\tilde f_1(\x) = \inner{\w,\Psi_1(\x)}$ and note that for every
$i\in [m]$ we have,
\begin{eqnarray*}
|\tilde f_1(\x_i)-f_2(\x_i)| &=& \left|\sum_{j=1}^m\alpha_j\left(\kappa_1(\x_i,\x_j)-\kappa_2(\x_i,\x_j)\right)\right|
\\
&\le &\epsilon\sum_{i=1}^m|\alpha_i|\le
	\epsilon \frac{M^2}{\epsilon_1} = \epsilon_1 \,.
\end{eqnarray*}
Finally, we get that,
\begin{eqnarray*}
\cl_{\cd}(f_2) &\le& \cl_{S}(f_2) + \epsilon_1
\\
&=& \frac{1}{m}\sum_{i=1}^m \ell\left(f_2(\x_i),y_i\right) + \epsilon_1
\\
&\le& \frac{1}{m}\sum_{i=1}^m \ell\left(\tilde{f}_1(\x_i),y_i\right) + \epsilon_1 + \epsilon_1
\\
&\le& \frac{1}{m}\sum_{i=1}^m \ell\left(f_1(\x_i),y_i\right) + |\tilde f_1(\x_i)-f_1(\x_i)| + 2\epsilon_1
\\
&\le& \frac{1}{m}\sum_{i=1}^m \ell\left(f_1(\x_i),y_i\right) + 3\epsilon_1
\\
&\le& \cl_S(f_1) + 3\epsilon_1 \le \cl_{\cd}(f_1) + 4\epsilon_1 \,,
\end{eqnarray*}
which concludes the proof.\proofbox


\section{Experiment Details}
\label{sec:experiment_details}

\subsection{Model Configurations and Hyperparameters}

We summarize the details required to replicate our experiments below.

\subsubsection{Image Classification}

\textbf{Baseline Model:} For dense models, we use standard implementations of
ViT~\citep{dosovitskiy2020image}, MLP-Mixer{tolstikhin2021mlp} from the
\texttt{timm} library and from the T2T-ViT codebase~\citep{yuan2021tokens}.

The Monarch version of these models simply swap out the dense weight matrices in the attention blocks (projection matrices) and in the FFN block (linear layers) with Monarch matrices.
We set the number of blocks in the block-diagonal matrices to 4.
We also reduce the amount of regularization (stochastic depth) as our Monarch models are smaller than the dense models.

We adopt the hyperparameters (optimizer, learning rate, learning rate
scheduler) from~\citet{yuan2021tokens}.
Details are in~\cref{table:imagenet_hparams}.

We measure the wall-clock training time on V100 GPUs.

\begin{table}[!htbp]
 \caption{Configuration of the ImageNet experiment}   
\centering
\resizebox{0.8\linewidth}{!}{
\noindent\begin{tabular}{@{}c||ccccccc@{}}
  \specialrule{.15em}{.05em}{.05em}
Model&\multicolumn{1}{c}{Optimizer}&\multicolumn{1}{c}{Weight Decay}&\multicolumn{1}{c}{Learning Rate}&\multicolumn{1}{c}{Drop Path}&\multicolumn{1}{c}{Warmup/Epoch}\\
  \specialrule{.15em}{.05em}{.05em}
ViT-Small& AdamW & 0.05 & 0.001 & 0.1& 5/300 \\
Monarch-ViT-Small& AdamW & 0.05 & 0.001 &0& 5/300 \\
ViT-Base& AdamW & 0.05 & 0.001 &0.1& 5/300 \\
Monarch-ViT-Base& AdamW & 0.05 & 0.001 &0& 5/300 \\
  \specialrule{.15em}{.05em}{.05em}
Mixer-Small &AdamW& 0.1 &0.001&0.1& 5/300 \\
Monarch-Mixer-Small &AdamW&0.1 &0.001& 0 & 5/300 \\
Mixer-Base &AdamW& 0.1 &0.001&0.1& 5/300 \\
Monarch-Mixer-Base &AdamW &0.1 &0.001& 0 & 5/300 \\
  \specialrule{.15em}{.05em}{.05em}
\end{tabular}
}
\label{table:imagenet_hparams}
\end{table}

We follow the naming convention in the Vision Transformer paper and MLP-Mixer paper. In particular, ViT-S and ViT-B refers to the small and base ViT models respectively, and 16 refers to the patch size of 16x16. The MLP-Mixer models follow the same convention.

\subsubsection{Language Modeling}
For dense models, we use standard implementations of
GPT-2~\citep{radford2019language} from Huggingface \texttt{transformers} library and from Nvidia's Megatron-LM repo. 
We follow the training recipe of the Megatron-LM repo.

The Monarch version of these models simply swap out the dense weight matrices in the attention blocks (projection matrices) and in the FFN block (linear layers) with Monarch matrices.
We set the number of blocks in the block-diagonal matrices to 4.
We also reduce the regularization strength (dropout) as our model is smaller.

We report the hyperparameters used in~\cref{table:wt103} and~\cref{table:owt}.
We use an effective batch size of 512, and use gradient accumulation to fit into available GPU memory.

We measure the wall-clock training time on V100 GPUs.
\begin{table}[!h]
    \vspace{-0.5cm}
\centering
\caption{Configuration of the WikiText-103 experiments}
\resizebox{0.8\linewidth}{!}{
\noindent\begin{tabular}{@{}c||ccccccc@{}}
  \specialrule{.15em}{.05em}{.05em}
Model&\multicolumn{1}{c}{Optimizer}&\multicolumn{1}{c}{Weight Decay}&\multicolumn{1}{c}{Learning Rate}&\multicolumn{1}{c}{Dropout}&\multicolumn{1}{c}{Warmup/Epoch}\\
  \specialrule{.15em}{.05em}{.05em}
GPT-2-small& AdamW & 0.1 & 6e-4 & 0.1& 10/100 \\
Monarch-GPT-2-small& AdamW & 0.1 & 6e-4 & 0.0 & 10/100 \\
GPT-2-medium& AdamW & 0.1 & 1.5e-4 & 0.1& 10/100 \\
Monarch-GPT-2-medium & AdamW & 0.1 & 1.5e-4 & 0.0 & 10/100 \\
  \specialrule{.15em}{.05em}{.05em}
\end{tabular}
}
\label{table:wt103}
\end{table}

\begin{table}[!h]
\vspace{-0.5cm}
\centering
\caption{Configuration of the OpenWebText experiments}
\resizebox{0.8\linewidth}{!}{
\noindent\begin{tabular}{@{}c||ccccccc@{}}
  \specialrule{.15em}{.05em}{.05em}
Model&\multicolumn{1}{c}{Optimizer}&\multicolumn{1}{c}{Weight Decay}&\multicolumn{1}{c}{Learning Rate}&\multicolumn{1}{c}{Dropout}&\multicolumn{1}{c}{Warmup/Total iterations}\\
  \specialrule{.15em}{.05em}{.05em}
GPT-2-Small& AdamW & 0.1 & 6e-4 & 0.1& 4k/400k \\
Monarch-GPT-2-Small & AdamW & 0.1 & 6e-4 & 0.0 & 4k/400k \\
GPT-2-Medium& AdamW & 0.1 & 1.5e-4 & 0.1& 4k/400k \\
Monarch-GPT-2-Medium & AdamW & 0.1 & 1.5e-4 & 0.0 & 4k/400k \\
  \specialrule{.15em}{.05em}{.05em}
\end{tabular}
}
\label{table:owt}
\end{table}


\subsection{Details for PDE Solving}
We adopt the experiment setting and data generation of Navier-Stokes Equation from FNO~\citep{li2020fourier}. It considers the 2-d Navier-Stokes equation for a viscous, incompressible fliud in vorticity form on the unit tortus:
\begin{align}
    \partial_{t} w(x, t) + u(x, t) \cdot \nabla w(x, t) & = v \Delta w(x, t) + f(x), & x \in (0, 1)^2, t \in (0, T] \\
    \nabla w(x, t) & = 0, & x \in (0, 1)^2, t \in (0, T] \\
    w(x, 0) & = w_0(x), & x \in (0, 1)^2 \\
\end{align}
where $u \in C([, T0])$;$H_{per}((0, 1)^2; \mathbb{R}^2))$ for any $r>0$ is the velocity field, $w=\nabla \times u$ is the vorticity, $w_0 \in L^2_{per}((0, 1)^2; \mathbb{R})$ is the initial vorticity, $v \in \mathbb{R_{+}}$ is the viscosity coefficient, and $f \in L_{per}^2((0, 1)^2; \mathbb{R})$ is the forcing function. 
$T$ represents the time interval since it is time-dependent equation. $v$ represents the viscosity. N represents the number of training pairs or data. \cref{table:pde} shows the results for viscosities $v=1e-3, 1e-4, 1e-5$, $T=50, 30, 20$ respectively and use $N=1000$. 

\subsection{Details for GPT-2 Downstream Tasks}
We train Pixelfly-GPT2-small on a larger scale dataset, OpenWebText, and evaluate the downstream quality on zero-shot generation and classification tasks from~\citep{zhao2021calibrate}, achieving comparable and even better performance to the dense model. Specifically, the datasets contains five popular classification tasks: SST2, Trec, CB, Agnews, and Dbpedia. We also adapated the calibrated metric from~\citep{zhao2021calibrate} for evaluation. Results for each individual task are shown in~\cref{table:gpt_finetune_full}. 

\begin{table}[h]
  \small
  \centering
  \vspace{-3mm}
  \caption{\label{table:gpt_finetune_full}The performance (accuracy) of GPT-2-medium trained with Monarch reverse sparsification and with conventional dense training on text classification benchmarks.}
  \setlength{\tabcolsep}{5pt}
  \vspace{1em}
   \resizebox{0.7\linewidth}{!}{
  \begin{tabular}{@{}c||ccccc@{}}
    \specialrule{.15em}{.05em}{.05em}
    Model&\multicolumn{1}{c}{OpenWebText (ppl)}&\multicolumn{1}{c}{Speedup}& \multicolumn{1}{c}{Classification (avg acc)} \\
    \specialrule{.15em}{.05em}{.05em}
    GPT-2m& 68.3 & 37.0 & 10.7 & 52.0 & 26.6\\
    Monarch-GPT-2m& 72 & 38.6 & 12.5 & 47.3 & 23.0 \\
    \specialrule{.15em}{.05em}{.05em}
  \end{tabular}
  }
  \vspace{-3mm}
\end{table}

\subsection{Details for BERT Pretraining}
\label{subsec:bert_details}

We follow the training procedure and hyperparameters of the reference
implementation from Nvidia Deep Learning examples
(\url{https://github.com/NVIDIA/DeepLearningExamples}).
In particular, we use the LAMB optimizer with learning rate 4e-3.
We use as large a minibatch size as possible that still fits in the GPU memory
(A100-40GB), and use gradient accumulation to reach an effective batch size of
64k sequences for phase 1 (maximum sequence length 128) and 32k for phase 2
(maximum sequence legnth 512).
We train is mixed precision (fp16 and fp32).

We use all the optimizations that were in Nvidia's BERT implementation
in MLPerf 1.1:
\begin{enumerate}
  \item Only compute the prediction scores (last layer) for masked tokens as
  the outputs of other tokens are not used to compute the masked language
  modeling loss.
  \item Remove padding tokens and only compute the attention for non-padding
  tokens.
  \item Use a fused CUDA kernel (FMHA) that combines 4 steps into one kernel: computes
  $Q K^T$, take softmax, apply dropout, multiply by $V$, where $Q, K, V$ are the
  query, key, and value respectively.
  \item Fuse matrix multiplication and adding bias into one CUDA kernel in the feed-forward network
  (FFN) layers. The gradient of the bias is also fused with the matrix
  multiplication the backward pass.
  \item Fuse matrix multiplication and adding bias into one CUDA kernel in the
  attention output projection.
  \item Fuse dropout and adding residual in the residual connection at the end
  on the attention and FFN blocks.
\end{enumerate}

We train with DeepSpeed~\citep{rasley2020deepspeed} ZeRO optimizer stage 1 to
shard the optimizer states, thus reducing GPU memory usage and allowing us to
use larger batch sizes.
For the Nvidia MLPerf implementation, we report the speed for both Apex's
automatic mix-precision (AMP) level O2 (as in the original implementation), and
DeepSpeed ZeRO optimizer.

\subsection{Accelerated Multi-coil MRI Reconstruction}
\label{sec:experiment_details_mri}

\subsubsection{Background}
In multi-coil MRI, multiple receiver coils (i.e. sensors) acquire complex-valued measurements in the spatial frequency (a.k.a. \textit{k-space}) domain. These measurements are modulated by the spatially-varying sensitivity maps, which characterize the sensitivity of each coil to the imaging target. In accelerated MRI, scan times are reduced by decreasing the number of samples acquired in k-space. Because the data is sampled below the Nyquist rate, reconstructing the underlying image is an ill-posed problem.

The forward problem for accelerated multi-coil MRI can be written as the matrix equation
\begin{equation*}
    y = \Omega\boldsymbol{F}\boldsymbol{S}x + \epsilon
\end{equation*}
where $\Omega$ is the binary undersampling mask that indexes acquired samples in k-space, $y$ is the vectorized measured signal in k-space, $\boldsymbol{F}$ is the discrete Fourier transform matrix, $\boldsymbol{S}$ is the receiver coil sensitivity maps,  $x$ is the ground-truth signal in image-space, and $\epsilon$ is additive complex Gaussian noise. The acceleration factor is given by $R = \frac{\sum_i^{|N|} \Omega_i}{|\Omega|}$.

\subsubsection{Experimental Details}

\paragraph{Dataset.} We benchmark our method on the SKM-TEA Raw Data Track, which consists of dual-echo 3D MRI scans \citep{desai2021skm}. Scans are accelerated using Poisson Disc undersampling masks distributed with the dataset. During training, Poisson Disc masks are generated, cached, and applied to mask the k-space data to simulate accelerated scans.

\paragraph{Matrix Shape.} Like all matrices, Monarch matrices have an explicit shape constraint, which is a limitation of these matrices for MRI reconstruction tasks. Thus, the SKM-TEA dataset was filtered to include scans of shape $512 \times 512 \times 160$, which is the most frequently occuring scan shape. A total of 3 scans were dropped from the original 155 scans in the dataset. Our method and all baselines were trained on this filtered dataset.

\begin{table}[!ht]
    \vspace{-0.5cm}
\centering
\caption{Baseline configurations of the SKM-TEA MRI reconstruction experiments.}
\resizebox{0.6\linewidth}{!}{
\noindent\begin{tabular}{c||cccccc}
  \specialrule{.15em}{.05em}{.05em}
Model & Params & Optimizer & Weight Decay & Learning Rate & Epoch \\
  \specialrule{.15em}{.05em}{.05em}
SENSE & --- & --- & --- & --- & --- \\
U-Net &  7.8M & Adam & 1e-4 & 1e-3 & 20 \\
mSENSE  & 57.5K & Adam & 1e-4 & 1e-3 & 20 \\
  \specialrule{.15em}{.05em}{.05em}
\end{tabular}
}
\label{table:skmtea-config}
\end{table}


\paragraph{Baselines.} We compare our method to two baselines, SENSE and U-Net. Parameter count and hyperparameters are available in Table \ref{table:skmtea-config}.
\begin{itemize}
    \item \textit{SENSE}: SENSE performs a linear combination of the images acquired on each coil \citep{pruessmann1999sense}. Here, the inverse fsat Fourier transform (IFFT) is applied to the acquired k-space for each coil. The resulting images are combined into a single complex image by weighting each coil image by corresponding coil sensitivity maps. In accelerated MRI, the unsampled frequencies are zero-valued; thus, SENSE produces a \textit{zero-filled image}. Note, SENSE does not require any training.
    \item \textit{U-Net}: U-Net is a popular fully convolutional neural network baseline for MRI reconstruction \citep{ronneberger2015u}. We use the default implementation and hyperparameters used by \citet{desai2021skm} to benchmark the SKM-TEA dataset. In this approach, the SENSE-reconstructed zero-filled image is mapped to SENSE-reconstructed ground truth images.
\end{itemize}

\paragraph{Monarch-SENSE (mSENSE):} We propose a modification to the SENSE method, in which the (IFFT) is parameterized by a factorized Monarch matrix. This matrix is initialized to the IFFT but, unlike SENSE, is learnable. While mSENSE is trainable, it has 137x fewer trainable parameters than U-Net.

\paragraph{Metrics:} We evaluate reconstruction performance using peak signal-to-noise ratio (pSNR) and structural similarity (SSIM) on both echoes (echo1 - E1, echo2 - E2) separately. Both metrics were computed on the 3D volume of each echo.

\paragraph{Extended Results.} We provide sample reconstructions of SENSE, mSENSE, and U-Net in data-limited settings for first (Fig.~\ref{fig:mri-data-limited-echo1}) and second (Fig.~\ref{fig:mri-data-limited-echo2}) echoes. Both SENSE and U-Net reconstructed images have aliasing artifacts. Due to the random Poisson Disc undersampling pattern, these artifacts are incoherent, causing them to manifest as blurring around fine structures and edges. In contrast, mSENSE can recover these structures with higher fidelity. Even in the second echo, which has lower signal-to-noise ratio (SNR) than the first echo, mSENSE does not overblur the image.

\begin{figure}
    \centering
    \includegraphics[width=0.9\linewidth]{figures/sample-mri-echo1.pdf}
    \vspace{-1em}
    \caption{Sample reconstructions at 2x acceleration for the first echo in the SKM-TEA dataset using SENSE, Monarch-SENSE (mSENSE), and U-Net. Both mSENSE and U-Net are trained with 1 training scan. SENSE is an untrained method.}
    \label{fig:mri-data-limited-echo1}
\end{figure}

\begin{figure}
    \centering
    \includegraphics[width=6in]{figures/sample-mri-echo2.pdf}
    \vspace{-1em}
    \caption{Sample reconstructions at 2x acceleration for the second echo in the SKM-TEA dataset using SENSE, Monarch SENSE (mSENSE), and U-Net. Both mSENSE and U-Net are trained with 1 training scan. SENSE is an untrained method.}
    \label{fig:mri-data-limited-echo2}
\end{figure}








\end{document}
