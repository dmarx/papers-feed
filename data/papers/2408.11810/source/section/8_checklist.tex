\clearpage
\section*{AAAI Checklist}
\subsection*{Conceptual and Methodological Details}
\begin{itemize}
    \item[] Includes a conceptual outline and/or pseudocode description of AI methods introduced: \textbf{yes}
    \item[] Clearly delineates statements that are opinions, hypothesis, and speculation from objective facts and results: \textbf{yes}
    \item[] Provides well marked pedagogical references for less-familiar readers to gain background necessary to replicate the paper: \textbf{yes}
\end{itemize}

\subsection*{Theoretical Contributions}
Does this paper make theoretical contributions? \textbf{no}

If yes, please complete the list below:
\begin{itemize}
    \item[] All assumptions and restrictions are stated clearly and formally: \textbf{(yes/partial/no)}
    \item[] All novel claims are stated formally (e.g., in theorem statements): \textbf{(yes/partial/no)}
    \item[] Proofs of all novel claims are included: \textbf{(yes/partial/no)}
    \item[] Proof sketches or intuitions are given for complex and/or novel results: \textbf{(yes/partial/no)}
    \item[] Appropriate citations to theoretical tools used are given: \textbf{(yes/partial/no)}
    \item[] All theoretical claims are demonstrated empirically to hold: \textbf{(yes/partial/no/NA)}
    \item[] All experimental code used to eliminate or disprove claims is included: \textbf{(yes/no/NA)}
\end{itemize}

\subsection*{Dataset Usage}
Does this paper rely on one or more datasets? \textbf{yes}

If yes, please complete the list below:
\begin{itemize}
    \item[] A motivation is given for why the experiments are conducted on the selected datasets: \textbf{yes}
    \item[] All novel datasets introduced in this paper are included in a data appendix: \textbf{NA}
    \item[] All novel datasets introduced in this paper will be made publicly available upon publication of the paper with a license that allows free usage for research purposes: \textbf{NA}
    \item[] All datasets drawn from the existing literature (potentially including authors' own previously published work) are accompanied by appropriate citations: \textbf{yes}
    \item[] All datasets drawn from the existing literature (potentially including authors' own previously published work) are publicly available: \textbf{yes}
    \item[] All datasets that are not publicly available are described in detail, with explanation why publicly available alternatives are not scientifically satisficing: \textbf{NA}
\end{itemize}

\subsection*{Computational Experiments}
Does this paper include computational experiments? \textbf{yes}

If yes, please complete the list below:
\begin{itemize}
    \item[] Any code required for pre-processing data is included in the appendix: \textbf{no}
    \item[] All source code required for conducting and analyzing the experiments is included in a code appendix: \textbf{no}
    \item[] All source code required for conducting and analyzing the experiments will be made publicly available upon publication of the paper with a license that allows free usage for research purposes: \textbf{yes}
    \item[] All source code implementing new methods have comments detailing the implementation, with references to the paper where each step comes from: \textbf{yes}
    \item[] If an algorithm depends on randomness, then the method used for setting seeds is described in a way sufficient to allow replication of results: \textbf{partial}
    \item[] This paper specifies the computing infrastructure used for running experiments (hardware and software), including GPU/CPU models; amount of memory; operating system; names and versions of relevant software libraries and frameworks: \textbf{partial}
    \item[] This paper formally describes evaluation metrics used and explains the motivation for choosing these metrics: \textbf{yes}
    \item[] This paper states the number of algorithm runs used to compute each reported result: \textbf{partial}
    \item[] Analysis of experiments goes beyond single-dimensional summaries of performance (e.g., average; median) to include measures of variation, confidence, or other distributional information: \textbf{yes}
    \item[] The significance of any improvement or decrease in performance is judged using appropriate statistical tests (e.g., Wilcoxon signed-rank): \textbf{no}
    \item[] This paper lists all final (hyper-)parameters used for each model/algorithm in the paper's experiments: \textbf{yes}
    \item[] This paper states the number and range of values tried per (hyper-) parameter during development of the paper, along with the criterion used for selecting the final parameter setting: \textbf{partial}
\end{itemize}
