% Save time on writing our method
\newcommand{\ourmethod}{$\tt{T1}$}

\newcommand{\se}[1]{{\color{magenta} [SE: #1]}}
\newcommand{\fb}[1]{{\color{orange} [FB: #1]}}

\everypar{\looseness=-1}

 \allowdisplaybreaks[4]

\usepackage{tabularx}
\usepackage{fontawesome}

\newcolumntype{\CeX}{>{\centering\let\newline\\\arraybackslash}X}
\newcolumntype{\CeP}{>{\raggedleft\arraybackslash}p}

\newcommand{\TextAndSymbol}[2]{%
  \begin{tabularx}{\textwidth}{X >{\raggedleft}>{\raggedright\arraybackslash}X}%
    #1 & #2
  \end{tabularx}%
}
% \newcommand{\TwoSymbolsAndText}[3]{%
%   \begin{tabularx}{\textwidth}{c\CeX c}%
%     #1 & #2 & #3
%   \end{tabularx}%
% }
% \newcommand{\config}[1]{\TwoSymbolsAndText{\faCogs}{%
%     \textbf{#1}%
%   }{\faCogs}}
\newcommand{\config}[1]{\TextAndSymbol{%
    \textbf{#1}%
  }{\faCogs}}


\usepackage{thmtools} 
\usepackage{thm-restate}

\declaretheorem[name=Theorem,numberwithin=section]{thm}


% Clever references
\usepackage[nameinlink,capitalise]{cleveref}
\crefname{section}{§}{§§}
\Crefname{section}{§}{§§}
\crefname{lemma}{lemma}{lemmas}
\Crefname{lemma}{Lemma}{Lemmas}
\crefname{thm}{theorem}{theorems}
\Crefname{thm}{Theorem}{Theorems}

\title{\textit{Transform Once}\\ Efficient Operator Learning in Frequency Domain}

