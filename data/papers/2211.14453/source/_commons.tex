% %%%%%%%%%%% STYLE %%%%%%%%%%%%%%%%%%%%%%%%%%%%%%
\usepackage[utf8]{inputenc}
\usepackage{etoolbox}
\usepackage{fancyhdr}
\usepackage{stmaryrd}
\usepackage[dvipsnames]{xcolor}
\usepackage{booktabs}
\usepackage{wrapfig}
\usepackage{accents}
\usepackage{newunicodechar}
\usepackage{subcaption}
\usepackage{colortbl}
\usepackage{graphicx}
\usepackage{fontawesome}



% %
\def\appendixname{Appendix}

%%%%%%%%%%% MATH %%%%%%%%%%%%%%%%%%%%%%%%%%%%%%%
\usepackage[bb=boondox,bbscaled=.95]{mathalfa}
\usepackage{amsmath, amssymb, amsfonts, amsthm, bm}
\usepackage{physics}
\usepackage{cancel}
\usepackage{thmtools, thm-restate}

\usepackage{enumitem} 
\declaretheorem{theorem}
\declaretheorem{proposition}
\declaretheorem{corollary}
\newtheorem{definition}{Definition}
\newtheorem{remark}{Remark}
\newtheorem{prop}[definition]{Proposition}
\newtheorem{lemma}[definition]{Lemma}
\newtheorem{example}[definition]{Example}
\usepackage{tcolorbox}
\usepackage[frozencache,cachedir=minted-cache]{minted}
\tcbuselibrary{minted,breakable,xparse,skins}

\definecolor{bg}{gray}{0.95}


\DeclareTCBListing{mintedbox}{O{}m!O{}}{%
  breakable=true,
  listing engine=minted,
  listing only,
  minted language=#2,
  title=#3,
  minted style=default,
  minted options={%
    encoding=utf8,
    linenos,
    gobble=0,
    breaklines=true,
    breakafter=,,
    fontsize=\small,
    numbersep=8pt,
    #1},
  boxsep=0pt,
  left skip=0pt,
  right skip=0pt,
  left=25pt,
  right=0pt,
  top=3pt,
  bottom=3pt,
  arc=5pt,
  leftrule=0pt,
  rightrule=0pt,
  bottomrule=2pt,
  toprule=2pt,
  colback=bg,
  colframe=red!70!white,
  enhanced,
  overlay={%
    \begin{tcbclipinterior}
    \fill[red!20!white] (frame.south west) rectangle ([xshift=20pt]frame.north west);
    \end{tcbclipinterior}},
  #3}
\makeatletter
\AtBeginEnvironment{mintedbox}{\dontdofcolorbox}
\def\dontdofcolorbox{\renewcommand\fcolorbox[4][]{##4}}
\makeatother

% GRAPHICS


% COMMANDS
\newcommand{\ubar}[1]{\underaccent{\bar}{#1}}

\newcommand{\x}{\times}

\newcommand{\cB}{\mathcal{B}}
\newcommand{\cC}{\mathcal{C}}
\newcommand{\cD}{\mathcal{D}}
\newcommand{\cE}{\mathcal{E}}
\newcommand{\cL}{\mathcal{L}}
\newcommand{\cN}{\mathcal{N}}
\newcommand{\cO}{\mathcal{O}}
\newcommand{\cT}{\mathcal{T}}
\newcommand{\cU}{\mathcal{U}}
\newcommand{\cX}{\mathcal{X}}
\newcommand{\cF}{\mathcal{F}}
\newcommand{\cG}{\mathcal{G}}
\newcommand{\cR}{\mathcal{R}}
\newcommand{\cY}{\mathcal{Y}}
\newcommand{\cW}{\mathcal{W}}
\newcommand{\cZ}{\mathcal{Z}}

\newcommand{\zb}{\mathbf{z}}
\newcommand{\xb}{\mathbf{x}}

\newcommand{\sD}{\mathsf{D}}


\DeclareMathOperator{\argmin}{arg min}

\newcommand{\bC}{\mathbb{C}}
\newcommand{\bT}{\mathbb{T}}
\newcommand{\bE}{\mathbb{E}}
\newcommand{\bP}{\mathbb{P}}
\newcommand{\R}{\mathbb{R}}

\newcommand{\nU}{{n_u}}
\newcommand{\nX}{{n_x}}
\newcommand{\nZ}{{n_z}}
\newcommand{\nW}{{n_w}}
\newcommand{\nT}{{n_\theta}}
\newcommand{\nN}{{n_\xi}}

\newcommand{\la}{\left\langle}
\newcommand{\ra}{\right\rangle}
\newcommand{\lb}{\left[}
\newcommand{\rb}{\right]}
\newcommand{\lc}{\left(}
\newcommand{\rc}{\right)}

\newcommand{\0}{\mathbb{0}}
\newcommand{\Id}{\mathbb{I}}


\def\red{\color{red}}
\def\blue{\color{blue!70!white}}
\def\ocra{\color{ocra}}
\definecolor{olive}{rgb}{0.6, 0.6, 0.2}
\definecolor{sand}{rgb}{0.8666666666666667, 0.8, 0.4666666666666667}
\definecolor{wine}{rgb}{0.5333333333333333, 0.13333333333333333, 0.3333333333333333}
\definecolor{deblue}{RGB}{11,132,147}
\definecolor{degray}{RGB}{150,150,156}
\definecolor{ocra}{RGB}{204, 119, 34}

\newcommand{\fcircle}[2][red,fill=red]{\tikz[baseline=-0.5ex]\draw[#1,radius=#2] (0,0.03) circle ;}
\newcommand{\fsquare}[2][red,fill=red]{\tikz[baseline=-0.5ex]\draw[#1,radius=#2] (-0.1,-0.1) rectangle (0.1,0.1);}

% Syntax: \colorboxed[<color model>]{<color specification>}{<math formula>}
\newcommand*{\colorboxed}{}
\def\colorboxed#1#{%
  \colorboxedAux{#1}%
}
\newcommand*{\colorboxedAux}[3]{%
  % #1: optional argument for color model
  % #2: color specification
  % #3: formula
  \begingroup
    \colorlet{cb@saved}{.}%
    \color#1{#2}%
    \boxed{%
      \color{cb@saved}%
      #3%
    }%
  \endgroup
}

\newcommand{\inner}[3][]{\ensuremath{\left\langle #2, \, #3 \right\rangle_{#1}}}


\newunicodechar{λ}{{$\mathtt\lambda$}}
\newunicodechar{μ}{{$\mathtt\mu$}}

\newcommand{\se}[1]{{\color{magenta} [SE: #1]}}