\documentclass[sigconf]{acmart}
\renewcommand\footnotetextcopyrightpermission[1]{}
\settopmatter{printacmref=false}
\setcopyright{none}
\pagestyle{plain}

\usepackage{hyperref}
\usepackage[hyphenbreaks]{breakurl}
\usepackage{balance}
\usepackage{url}
\usepackage{color}
\usepackage{caption}
\usepackage{diagbox}
\usepackage{subfigure}
\usepackage{multirow}
\usepackage{booktabs}
\usepackage{epsfig,endnotes}
\usepackage{enumitem}
\usepackage{multirow}
\usepackage{booktabs}
\newcommand{\RNum}[1]{\uppercase\expandafter{\romannumeral #1\relax}}

\newcommand{\model}{{\mathcal{F}_{\theta}}}
\newcommand{\orgmodel}{{\mathcal{F}_{o}}}
\DeclareMathOperator{\sign}{sign}


\DeclareMathOperator*{\argmin}{argmin}

\newcommand{\para}[1]{{\vspace{2pt} \noindent \textbf{#1}
    \hspace{6pt}}}

\newcommand{\subpara}[1]{{\vspace{1.0pt} \textbf{#1}
    \hspace{4pt}}}

\newcommand{\fixme}[1]{{\color{red} #1}}
\newcommand{\rewrite}[1]{{\color{black} #1}}
\newcommand{\todo}[1]{{\color{red}TODO:  #1}}
\newcommand{\wip}[1]{{\color{gray} #1}}

\newcommand{\askben}[1]{{\color{red} Q:  #1}}
\newcommand{\emily}[1]{{\color{black} #1}}

\newcommand{\arjun}[1]{{\color{black} Arjun:  #1}}

\definecolor{applegreen}{rgb}{0.55, 0.71, 0.0}

\newcommand{\abedit}[1]{{\color{black} #1}}
\newcommand{\shawnmr}[1]{{\color{black} #1}}
\newcommand{\shawn}[1]{{\color{black} #1}}
\newcommand{\rev}[1]{{\color{black} #1}}


\newcommand{\htedit}[1]{{\color{black} #1}}

\newcommand{\bencheck}[1]{{\color{black} #1}}

\newcommand{\ssedit}[1]{{\color{black} #1}}
\newcommand{\outline}[1]{{\color{blue} #1}}
\newcommand{\ol}[1]{{\color{blue} #1}}

\newcommand{\etal}{{\em et al.\ }}
\newcommand{\eg}{{\em e.g.,\ }}
\newcommand{\ie}{{\em i.e.,\ }}

\newcommand{\secspace}{\vspace{-0.05in}}
\newcommand{\secspacesm}{\vspace{0.0in}}

\newcommand{\system}{{\em Neo}}

\newcommand{\ad}[1]{{$\mathcal{A}$}}
\newcommand{\service}[1]{{$\mathcal{S}$}}
\newcommand{\mcD}{\mathcal{D}}

\newcommand{\ytface}{{\tt YTFace}}
\newcommand{\cifarS}{{\tt CIFAR10}}
\newcommand{\cifar}{{\tt CIFAR10}}
\newcommand{\skin}{{\tt SkinCancer}}

\newcommand{\cifarL}{{\tt CIFAR100}}

\newcommand{\imagenet}{{\tt ImageNet}}

\newenvironment{packed_itemize}{
\begin{list}{\labelitemi}{\leftmargin=0.5em}
  \setlength{\itemsep}{1pt}
  \setlength{\parskip}{0pt}
  \setlength{\parsep}{0pt}
  \setlength{\headsep}{0pt}
  \setlength{\topskip}{0pt}
  \setlength{\topmargin}{0pt}
  \setlength{\topsep}{0pt}
  \setlength{\partopsep}{0pt}
}{\end{list}}

\newenvironment{packed_enumerate}{
\begin{enumerate}
 \setlength{\itemsep}{1pt}
 \setlength{\parskip}{0pt}
 \setlength{\parsep}{0pt}
 \setlength{\headsep}{0pt}
 \setlength{\topskip}{0pt}
 \setlength{\topmargin}{0pt}
 \setlength{\topsep}{0pt}
 \setlength{\partopsep}{0pt}
}{\end{enumerate}}

\begin{document}
\title{A Response to Glaze Purification via IMPRESS}

\author{Shawn Shan$^\dag$, Stanley Wu$^\dag$, Haitao Zheng, Ben Y. Zhao\\
$^\dag$ denotes authors with equal contribution\\
  {\em Department of Computer Science, University of Chicago}\\
  {\em \{shawnshan, stanleywu, htzheng, ravenben\}@cs.uchicago.edu}}

\begin{abstract}

  Recent work proposed a new mechanism to remove protective perturbation
  added by Glaze in order to again enable mimicry of art styles from images
  protected by Glaze.  Despite promising results shown in the original paper,
  our own tests with the authors' code demonstrated several limitations of
  the proposed purification approach.  The main limitations are 1)
  purification has a limited effect when tested on artists that are \textit{not
    well-known historical artists} already embedded in original training
  data, 2) problems in evaluation metrics, and 3) 
  collateral damage on mimicry result for clean images.  We believe these
  limitations should be carefully considered in order to understand real
  world usability of the purification attack.

\end{abstract}

\maketitle

\section{Introduction}
%
Neural network (NN) learning has underpinned state of the art empirical
results in numerous applied machine learning tasks (see for
instance~\cite{krizhevsky2012imagenet,lecun2015deep}). Nonetheless, neural
network learning remains rather poorly understood in several regards.
Notably, it remains unclear why training algorithms find good weights, how
learning is impacted by the network architecture and activations, what is
the role of random weight initialization, and how to choose a concrete
optimization procedure for a given architecture.

We start by analyzing the expressive power of NNs subsequent to the random
weight initialization. The motivation is the empirical success of training
algorithms despite inherent computational intractability, and the fact that
they optimize highly non-convex objectives with potentially many local minima.
Our key result shows that random initialization already positions learning
algorithms at a good starting point. We define an object termed a {\em
computation skeleton} that describes a distilled structure of feed-forward
networks. A skeleton induces a family of network architectures along with a
hypothesis class $\ch$ of functions obtained by certain non-linear
compositions according to the skeleton's structure.  We show that the
representation generated by random initialization is sufficiently rich to
approximately express the functions in $\ch$. Concretely, all functions in
$\ch$ can be approximated by tuning the weights of the last layer, which is
a convex optimization task.

In addition to explaining in part the success in finding good weights, our
study provides an appealing perspective on neural network learning.  We
establish a tight connection between network architectures and their dual
kernel spaces. This connection generalizes several previous constructions
(see Sec~\ref{sec:related}). As we demonstrate, our dual view gives rise to
design principles for NNs, supporting current practice and suggesting
new ideas. We outline below a few points.

\begin{itemize}

\item Duals of convolutional networks appear a more suitable fit for
	vision and acoustic tasks than those of fully connected networks.

\item Our framework surfaces a principled initialization scheme. It is
	very similar to common practice, but incorporates a small correction.

\item By modifying the activation functions, two consecutive fully connected
	layers can be replaced with one while preserving the network's dual kernel.

\item The ReLU activation, i.e. $x \mapsto \max(x,0)$, possesses favorable
	properties. Its dual kernel is expressive, and it can be well approximated by
	random initialization, even when the initialization's scale is moderately
	changed.

\item As the number of layers in a fully connected network becomes very
	large, its dual kernel converges to a degenerate form for any non-linear
	activation.

\item Our result suggests that optimizing the weights of the last layer can
	serve as a convex proxy for choosing among different architectures prior
	to training. This idea was advocated and tested empirically
	in~\cite{saxe2011random}.

\end{itemize}

\secspace
\section{Background}
\label{sec:back}

Here, we first summarize Glaze and then IMPRESS purification method. 

\para{Glaze protection against style mimicry. } Glaze seeks to protect artist's 
artwork from AI mimicry by adding small 
perturbations on these artwork to confuse diffusion models. 
Given an artwork $x$ and target style $T$ that is different from the artist's, 
Glaze first uses a pretrained style transfer model $\Omega$ to compute a style 
transferred version of the artwork. We denote such image as $\Omega(x, T)$. Then, Glaze 
computes a cloak $\delta_x$ that optimize the latent representation of Glazed artwork
($x + \delta_x$) to be similar to the style transferred artwork ($\Omega(x, T)$). 
The Glaze optimization effectively moves the original image to a new position in 
the high dimensional latent space, causing model to learn an incorrect art style. 
Glaze calculates the latent space using the 
feature extractor ($\mathcal{E}$) from a diffusion model.
Formally, we write the Glaze 
optimization as solving the following:
\begin{align}
    \min_{\delta_x} ||\mathcal{E}(x + \delta_x) - \mathcal{E}(\Omega(x, T))||_2 \\
    \text{subject to } \text{LPIPS}(x + \delta_x, x) < p_{G} \notag
\end{align}
\noindent We use LPIPS, a popular human-perceived visual distortion metric~\cite{zhang2018unreasonable}, to bound the perturbation 
within a budget $p_{G}$. 

\para{IMPRESS Purification Method. } IMPRESS adds additional perturbation on top of a Glazed artwork 
hoping to ``purify'' the Glaze
effect -- recovering the precise latent representation of original artwork. 
First, the authors empirically find that when passing Glazed images through an image autoencoder, 
the output image looks more different from the input image, compared to the output 
when inputting a clean image to the same autoencoder. Then authors assume removing this particular 
discrepancy would guide them to find the original (non-Glazed) image. 

IMPRESS purification  
optimizes perturbations on Glaze images such that purified images
behave similarly to clean images
when passing through an autoencoder. The authors assume the optimization process will guide 
the image to move back to the original latent space of the non-Glazed image. 
Formally, IMPRESS purification optimize a perturbation $\delta_{pur}$ on 
a Glaze image $x_{glazed}$:

\begin{align}
    \min_{\delta_{pur}} ||(x_{glazed} + \delta_{pur}) - \text{VAE}(x_{glazed} + \delta_{pur}) ||_2^2 \\
    \text{subject to } \text{LPIPS}(x_{glazed} + \delta_{pur}, x_{glazed}) < p_{I} \notag
\end{align}

\noindent $\text{VAE}$ is an image autoencoder, which consists of an encoder $\mathcal{E}$ followed by a decoder $D$. IMPRESS uses the same
autoencoder as the stable diffusion model. The 
perturbation $\delta_{pur}$ is bounded by a LPIPS 
perturbation budget $p_{I}$. 

\secspace
\section{Evaluation of IMPRESS}
\label{sec:issues}

Here, we first identify a few flaws in IMPRESS's experiment setup 
and then evaluate purification performance 
in several realistic mimicry scenarios. 

\para{Setup. } We follow the purification setup from the original IMPRESS paper and the source
code provided by the authors. We use default 
Glaze setup (perturbation budget = $0.07$) as described in the paper. 
We follow style mimicry setup in prior 
work~\cite{shan2023glaze}. Given a set of artworks, we mimic its style by
fine-tuning stable diffusion model (version 1.5) using DreamBooth 
for 600 to 1000 steps depending on the finetuning sample size for the artist.

\begin{figure*}
  \centering
  \includegraphics[width=0.99\linewidth]{nonhistorical.pdf}
  \vspace{-0.1in}
  \caption{Mimicry results on non-historical artists (Karla Ortiz and Kim van Deun). Left shows the images generated from a model trained on original images; right shows the images generated from a model trained on images that are first Glazed and then purified by IMPRESS. }
  \label{fig:non-historical}
  \vspace{-0.in}
\end{figure*}

\begin{figure*}
  \centering
  \includegraphics[width=0.99\linewidth]{smooth.pdf}
  \vspace{-0.1in}
  \caption{Mimicry results on smooth surface art styles (Ivan Shishkin and Ivan Aivazovsky). Left shows the images generated from a model trained on original images; right shows the images generated from a model trained on images that are first Glazed and then purified by IMPRESS. }
   \label{fig:smooth}
  \vspace{-0.in}
\end{figure*}

\begin{figure*}
  \centering
  \includegraphics[width=0.95\linewidth]{clean-degrade.pdf}
  \vspace{-0.1in}
  \caption{Original artwork and corresponding purified artwork. }
  \label{fig:clean}
  \vspace{-0.in}
\end{figure*}

\subsection{Generalizability to Realistic Scenarios}

In the original IMPRESS paper, the authors focus the evaluation on protecting the
art styles of famous historical artists (Monet, Van Gogh) -- whose style are already learned
by pretrained diffusion models prior to style mimicry. In the real-world, it is current artists
who are most concerned about AI mimicking their art style. Glaze is designed
to protect those artists, and they are not as heavily pretrained into the
base model as Monet or Van Gogh. 

We evaluate IMPRESS on non-historical artists and show that purification
has limited effectiveness at removing Glaze protection. 
Even for historical artists, we find the purification works poorly for 
certain art styles. Lastly, we find purification also degrades clean image quality where
it removes texture from art pieces. 

\para{Performance on non-historical artists. } We use artwork from Karla 
Ortiz and Kim Van Deun to simulate the mimicry attack on current artists. Karla is a fine-art artist 
with a similar style as some historical artists tested in the IMPRESS paper. 
Figure~\ref{fig:non-historical}
shows the mimicry results when model trained on artist's original art pieces (left)
or when model trained on art pieces that are first Glazed
and then purified by IMPRESS (right). We can observe significant amount of 
artifacts on mimicry results when the model is trained on purified Glazed images. 

\para{Poor performance on certain art styles. } IMPRESS works by adding artifacts
on Glazed images to recover the latent representation of original artwork. 
We found purification has more challenges recovering smooth surface 
art styles (realism art, symbolism, romanticism, etc) even for historical
artists already trained into the base model. 
We choose two historical artists: Ivan Aivazovsky (romanticism style) and Ivan Shishkin (realism style). 

Figure~\ref{fig:smooth} shows the mimicry results. We see IMPRESS 
introduces signficiant amount of artifacts to the mimicked images. 
The weaker performance is likely because purifying  the 
original smoother surface art requires the optimization to 
be very percise -- find the exact smooth surface. 

\para{Degrading image quality. } We found the purification process degrades 
clean image quality. Figure~\ref{fig:clean} shows original artwork 
and its corresponding purified artwork. The purified artwork is much more
blurry as purification removes textures from the images. 

\subsection{CLIP-based metrics are Inaccurate}

``CLIP genre accuracy'' quantifies if mimicked art is classified into the same art genre as the original 
art pieces according to a CLIP model. It has been used in prior work to evaluate mimicry 
success. However, in our own tests dating back to late September 2023, we
found CLIP genre accuracy is a poor indicator of end-to-end mimicry success.  
CLIP accuracy is especially poor when evaluating attacks against Glaze. The 
reason is that attacks (such as 
IMPRESS purification), as they seek to remove Glaze effect from art, often have signficant impact 
on the base image quaility of the artwork. 
The degradation in image quality is not captured by CLIP accuracy, \eg a very
low quality cubism painting is still classified as ``cubism'' with high
probability. But the result are not successful 
mimicry due to the low quaility of the mimicked images. 
Because of these poor properties as an accuracy indicator, we stopped using
CLIP distance as a success metric starting with the Glaze v1.1 update
(October 2023). 

\section{Discussion and conclusion}
\label{sec:discussion}

This work introduces a design for agents that assist users in generating images through an interactive process of proactive question asking and belief graph refinement. By dynamically updating its understanding of the user's intent, the agent facilitates a more collaborative and precise approach to image generation. Moreover, presenting the agent's belief graph can be a generalizable method for AI transparancy, which is an important factor given the increasing complexity of modern AI models. 

\textbf{Modular design.}  Our agent prototypes are highly modular: the agents use frozen T2I models to generate images based on the prompts that the agent updated. Therefore when a better off-the-shelf T2I model becomes available, it can be directly plugged into the agents and the system will achieve better performance without any additional adaptation\footnote{T2T scores in \Cref{tab:auto_eval} ablates the T2I model and only performs similarity on the captions. Our agents have achieved a 92\%+ T2T score, showing that their performance can be boosted by adopting better T2I models.}.  

\textbf{Personalized content.} By asking clarification questions, our agents enable a more customizable and personalized content creation experience. Because different groups of people may perceive helpfulness and harmfulness of contents differently, learning more about the user through clarification questions before generation can potentially mitigate risks of generating contents that are offensive to each specific user, and increase likelihoods of producing helpful outputs.


\textbf{Future work.} Alternative to the modular design, one can explore generating images directly from belief graphs and fine-tuning  LLM/VLMs on text/image trajectories that include asking questions. These may require a) collecting data such as gold-standard trajectories or annotations on the quality of trajectories of human-agent conversations and b) new approaches to fine-tune the model on multi-turn trajectories of images and text, which can potentially improve the performance of the agent.










\subsection*{Acknowledgements}
We would like to thank Jason Baldridge and Zoubin Ghahramani for insightful discussions on multi-turn T2I and belief states, Mahima Pushkarna for the help and consultation on user study. We would also like to thank Richard Song and Noah Fiedel for feedback on the paper.


\bibliographystyle{ACM-Reference-Format}
\bibliography{impress}
\balance

\end{document}

\typeout{get arXiv to do 4 passes: Label(s) may have changed. Rerun}
