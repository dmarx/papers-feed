\begin{table*}[ht]
\small
\centering
\def\arraystretch{1.2}
\begin{tabular}{|l|l|lll|}
\hline
\multirow{4}{*}{\begin{tabular}[c]{@{}l@{}}Originality\\ in Thought\end{tabular}} & NewYorker & \multicolumn{3}{l|}{\begin{tabular}[c]{@{}l@{}}The ideas in this piece are unique, and expressed with original language. The metaphorical \\ language referenced above is a list of good examples. Others include the moment when she\\ slides her sunglasses down and everything goes darker; Rabbi Adler's monotonous drone \\ rendered as--son...his...own...flesh; Barbara rocking like the overloaded boat she's become.\\ This piece is practically bursting with new, exciting ways of expressing familiar things.\end{tabular}} \\ \cline{2-5} 
 & Claude & \multicolumn{3}{l|}{While the piece avoids overused expressions, its ideas and themes are hackneyed.} \\ \cline{2-5} 
 & GPT4 & \multicolumn{3}{l|}{\begin{tabular}[c]{@{}l@{}}The characters in this piece are so defined by their religion and culture as to be flattened by \\ stereotype. The events of this piece feel arbitrary, almost random. While that does grant it an\\ unpredictability and a vague form of originality, it feels thoughtless.\end{tabular}} \\ \cline{2-5} 
 & GPT3.5 & \multicolumn{3}{l|}{\begin{tabular}[c]{@{}l@{}}The piece relies on cliched turns of phrase to express actions and thoughts. Reality hits \\ Barbara like a tidal wave; days turn to weeks (and weeks?) and months; she uses her \\ experience to "bridge divides" and "heal wounds".\end{tabular}} \\ \hline
\end{tabular} 
\vspace{2ex}
\caption{\label{expertfeedbackcluster}Expert explanations on stories from NewYorker, Claude, GPT3.5 and GPT4 for one of the Originality test.}
\vspace{-4ex}
\end{table*}