\documentclass[manuscript,screen,review]{acmart}
% \documentclass[manuscript,review,anonymous]{acmart}

%anonymous change
\AtBeginDocument{%
  \providecommand\BibTeX{{%
    \normalfont B\kern-0.5em{\scshape i\kern-0.25em b}\kern-0.8em\TeX}}}


\setcopyright{acmcopyright}
\copyrightyear{2023}
\acmYear{2023}
\acmDOI{XXXXXXX.XXXXXXX}

\acmConference[CHI '24]{Make sure to enter the correct
  conference title from your rights confirmation emai}{May 11--16,
  2024}{Honolulu, Hawaii}

\acmBooktitle{Hawaii' 24: ACM (Association of Computing Machinery) CHI conference on Human Factors in Computing Systems ,
 May 11--16,2024, Honolulu, Hawaii} 
\acmPrice{15.00}
\acmISBN{978-1-4503-XXXX-X/18/06}


\usepackage{multicol}
\usepackage{multirow}
\usepackage{array}
\usepackage{cleveref}
\usepackage{subcaption}
\usepackage{xcolor}
\usepackage{enumitem}
\usepackage{multirow}
\usepackage{subcaption}

%%
%% Submission ID.
%% Use this when submitting an article to a sponsored event. You'll
%% receive a unique submission ID from the organizers
%% of the event, and this ID should be used as the parameter to this command.
%%\acmSubmissionID{123-A56-BU3}

%%
%% For managing citations, it is recommended to use bibliography
%% files in BibTeX format.
%%
%% You can then either use BibTeX with the ACM-Reference-Format style,
%% or BibLaTeX with the acmnumeric or acmauthoryear sytles, that include
%% support for advanced citation of software artefact from the
%% biblatex-software package, also separately available on CTAN.
%%
%% Look at the sample-*-biblatex.tex files for templates showcasing
%% the biblatex styles.
%%

%%
%% The majority of ACM publications use numbered citations and
%% references.  The command \citestyle{authoryear} switches to the
%% "author year" style.
%%
%% If you are preparing content for an event
%% sponsored by ACM SIGGRAPH, you must use the "author year" style of
%% citations and references.
%% Uncommenting
%% the next command will enable that style.
%%\citestyle{acmauthoryear}

%%
%% end of the preamble, start of the body of the document source.
\begin{document}


\title{\textit{Is your AI really Creative?} Evaluating Creative Writing in the age of Large Language Models}


\author{Tuhin Chakrabarty}
\email{tuhin.chakr@cs.columbia.edu}
\affiliation{%
  \institution{Columbia University}
  \country{USA}
}

\author{Phillippe Laban}
\affiliation{%
  \institution{Salesforce AI Research}
  \country{USA}
}

\author{Divyansh Agarwal}
\affiliation{%
  \institution{Salesforce AI Research}
  \country{USA}
}

\author{Smaranda Muresan}
\email{smara@cs.columbia.edu}
\affiliation{%
  \institution{Columbia University}
  \country{USA}
}

\author{Chien-Sheng Wu}
\affiliation{%
  \institution{Salesforce AI Research}
  \country{USA}
}

% \renewcommand{\shortauthors}{Trovato and Tobin, et al.}


\begin{abstract}
Creativity is often associated with other markers of distinction, such as genius, imagination or originality.Eminent philosopher Vilém Flusser says "Only one who writes lines can think logically, calculate, criticize, pursue knowledge, philosophize.". Researchers have demonstrated that the current pile of large language models can  not only write academic essays, emails, cover letters but even poetry, fiction or screenplays. Unlike other forms of writing, creative writing places a much higher value on doing unexpected things that confound and delight the reader, typically contradicting the auto-regressive nature of LLMs. In spite of this there has been substantial discussion at the intersection of AI and creative writing, especially with researchers from Open-AI stating that creative writers and authors have extraordinarily high "exposure" to disruption from AI tools \cite{eloundou2023gpts}. Creativity is often difficult to evaluate because it is a complex, multifaceted concept that is deeply subjective, contextual, and hard to judge objectively. To tackle this, we interview 8 creative writing experts to elicit measures for evaluating creative writing. These measures are grounded in existing theoretical research on cognitive psychology for measuring creativity. We then recruit creative writing experts to utilize these measures to evaluate stories written by LLM's vs professional writers on the same plot. Our results show concerning trends that LLM generated stories are often of much poorer quality across several dimensions of creativity when compared to those written by professional writers. Additionally we also show that experts can easily identify stories written by AI vs those by professional writers. We also simulate the same evaluation using LLMs, demonstrating significant disagreement with creative writing experts thereby suggesting that LLMs are not aligned with expert humans across judgments about creativity.
\end{abstract}

%%
%% The code below is generated by the tool at http://dl.acm.org/ccs.cfm.
%% Please copy and paste the code instead of the example below.
%%

\begin{CCSXML}
<ccs2012>
   <concept>
       <concept_id>10003120.10003121.10011748</concept_id>
       <concept_desc>Human-centered computing~Empirical studies in HCI</concept_desc>
       <concept_significance>500</concept_significance>
       </concept>
   <concept>
       <concept_id>10003120.10003130.10011762</concept_id>
       <concept_desc>Human-centered computing~Empirical studies in collaborative and social computing</concept_desc>
       <concept_significance>500</concept_significance>
       </concept>
   <concept>
       <concept_id>10010147.10010178.10010179.10010182</concept_id>
       <concept_desc>Computing methodologies~Natural language generation</concept_desc>
       <concept_significance>300</concept_significance>
       </concept>
 </ccs2012>
\end{CCSXML}

\ccsdesc[500]{Human-centered computing~Empirical studies in HCI}
\ccsdesc[500]{Human-centered computing~Empirical studies in collaborative and social computing}
\ccsdesc[300]{Computing methodologies~Natural language generation}

% Author Keywords
\keywords{Human-AI collaboration, Large Language Models, Story Writing, Natural Language Generation, Evaluation, Creativity}% Print the classficiation codes

%% A "teaser" image appears between the author and affiliation
%% information and the body of the document, and typically spans the
%% page.
\begin{teaserfigure}
    \small
    \centering
  \includegraphics[width=0.35\textwidth]{sample-ai}
  \caption{\href{https://www.newyorker.com/culture/cultural-comment/the-computers-are-getting-better-at-writing}{The Computers Are Getting Better at Writing (The NewYorker)}}
  \Description{Enjoying the baseball game from the third-base
  seats. Ichiro Suzuki preparing to bat.}
  \label{fig:teaser}
\end{teaserfigure}

\received{20 February 2007}
\received[revised]{12 March 2009}
\received[accepted]{5 June 2009}


\maketitle

\usepackage{amsmath}
\usepackage{amsthm}
\usepackage{thmtools}

%\definecolor{blueboxedinner}{rgb}{.7,.7,1}
%\definecolor{blueboxedouter}{rgb}{.2,.2,1}
%\newcommand{\blueboxed}[1]{
%  \indent\fcolorbox{blueboxedouter}{blueboxedinner}{
%    \begin{minipage}{0.9\textwidth}
%      {#1}
%    \end{minipage}
%  } \\
%}
%\comment[1]{\blueboxed{\textcolor{blue} {#1}}}
%\newcommand\rfcomment[1]{\blueboxed{\textcolor{blue}{\bf RF:} {#1}}}
%\newcommand\yscomment[1]{\blueboxed{\textcolor{blue}{\bf YS:} {#1}}}
%\newcommand\adcomment[1]{\blueboxed{\textcolor{blue}{\bf AD:} {#1}}}

\newtheorem{theorem}{Theorem}
\newtheorem{definition}{Definition}
\newtheorem*{terminology}{Terminology}
\newtheorem{claim}{Claim}
\newtheorem{lemma}[theorem]{Lemma}
\newtheorem{proposition}[theorem]{Proposition}
\newtheorem{corollary}[theorem]{Corollary}
\newtheorem{conjecture}[theorem]{Conjecture}
\theoremstyle{definition}
\newtheorem{example}[theorem]{Example}
\newtheorem{remark}[theorem]{Comment}
\newtheorem{note}[theorem]{Note}
\newtheorem{counter-example}[theorem]{Counter example}
\newtheorem{hypothesis}[theorem]{Hypothesis}
\newtheorem{assumption}[theorem]{Assumption}
\newtheorem{open question}[theorem]{Open question}
\newtheorem{sublemma}[theorem]{Sublemma}

\newcommand{\co}{{\cal O}}
\newcommand{\ca}{{\cal A}}
\newcommand{\cb}{{\cal B}}
\newcommand{\cd}{{\cal D}}
\newcommand{\D}{\mathcal{D}}
\newcommand{\cc}{{\cal C}}
\newcommand{\ck}{{\cal K}}
\newcommand{\cq}{{\cal Q}}
\newcommand{\ce}{{\cal E}}
\newcommand{\ct}{{\cal T}}
\newcommand{\cg}{{\cal G}}
\newcommand{\ch}{{\cal H}}
\newcommand{\cm}{{\cal M}}
\newcommand{\ci}{{\cal I}}
\newcommand{\cj}{{\cal J}}
\newcommand{\cw}{{\cal W}}
\newcommand{\cl}{{\cal L}}
\newcommand{\cf}{{\cal F}}
\newcommand{\cv}{{\cal V}}
\newcommand{\cp}{{\cal P}}
\newcommand{\cu}{{\cal U}}
\newcommand{\cx}{{\cal X}}
\newcommand{\cy}{{\cal Y}}
\newcommand{\cz}{{\cal Z}}
\newcommand{\cs}{{\cal S}}
\newcommand{\cn}{{\cal N}}

\newcommand{\bh}{{\mathbb H}}
\newcommand{\x}{{\mathbf x}}
\newcommand{\y}{{\mathbf y}}
\newcommand{\z}{{\mathbf z}}
\newcommand{\w}{{\mathbf w}}
\newcommand{\bv}{{\mathbf v}}
\newcommand{\bb}{{\mathbf b}}

\DeclareMathOperator*{\erf}{erf}
\DeclareMathOperator*{\hinge}{hinge}
\DeclareMathOperator*{\sign}{sign}
\DeclareMathOperator*{\argmax}{argmax}

\newcommand{\mono}{{\mathbb M}}
\newcommand{\bool}{{\mathbb B}}
\newcommand{\reals}{{\mathbb R}}
\newcommand{\integers}{{\mathbb Z}}
\newcommand{\complex}{{\mathbb C}}
\newcommand{\sphere}{{\mathbb S}}
\newcommand{\csphere}{\complex\cup\{\infty\}}
\newcommand{\one}{{\vec 1}}
\newcommand{\zero}{{\vec 0}}
\newcommand{\rand}{\mathrm{rand}}
\newcommand{\width}{\mathrm{width}}
\newcommand{\out}{\mathrm{out}}
\newcommand{\depth}{\mathrm{depth}}
\newcommand{\np}{\mathrm{NP}}
\newcommand{\IN}{\mathrm{in}}
\newcommand{\rep}{\mathcal{R}}
\newcommand{\netrep}{\mathrm{rep}}
\newcommand{\ind}{\mathbf{1}}
\newcommand\by{\times}

\newcommand{\prob}[1]{{\rm Prob}\left\{ #1 \right\} }

\newcommand{\proofbox}{\qed}

\DeclareMathOperator{\Err}{Err}
\DeclareMathOperator{\poly}{poly}
\DeclareMathOperator*{\E}{\mathbb{E}}

\newcommand{\inner}[1]{\langle #1 \rangle}
\newcommand{\todo}[1]{{\bf \textcolor{red}{TODO: #1}}}

\newcommand\eg{\textit{e.g.}\ }
\newcommand\ie{\textit{i.e.}\ }

\newcommand\textif{\text{if }}
\newcommand\textelse{\text{else}}
\newcommand\textow{\text{otherwise}}
\newcommand\textwhere{\text{where }}
\newcommand\textsuchthat{\text{such that }}
\newcommand\textsubjto{\text{subject to }}
\newcommand\textand{\text{and }}
\newcommand\textforall{\text{for all }}
\newcommand\textfor{\text{for }}
\newcommand\textwhen{\text{when }}
\newcommand\textwhenever{\text{whenever }}

\newcommand\gaussian{\text{N}}

\section{Introduction}
%
Neural network (NN) learning has underpinned state of the art empirical
results in numerous applied machine learning tasks (see for
instance~\cite{krizhevsky2012imagenet,lecun2015deep}). Nonetheless, neural
network learning remains rather poorly understood in several regards.
Notably, it remains unclear why training algorithms find good weights, how
learning is impacted by the network architecture and activations, what is
the role of random weight initialization, and how to choose a concrete
optimization procedure for a given architecture.

We start by analyzing the expressive power of NNs subsequent to the random
weight initialization. The motivation is the empirical success of training
algorithms despite inherent computational intractability, and the fact that
they optimize highly non-convex objectives with potentially many local minima.
Our key result shows that random initialization already positions learning
algorithms at a good starting point. We define an object termed a {\em
computation skeleton} that describes a distilled structure of feed-forward
networks. A skeleton induces a family of network architectures along with a
hypothesis class $\ch$ of functions obtained by certain non-linear
compositions according to the skeleton's structure.  We show that the
representation generated by random initialization is sufficiently rich to
approximately express the functions in $\ch$. Concretely, all functions in
$\ch$ can be approximated by tuning the weights of the last layer, which is
a convex optimization task.

In addition to explaining in part the success in finding good weights, our
study provides an appealing perspective on neural network learning.  We
establish a tight connection between network architectures and their dual
kernel spaces. This connection generalizes several previous constructions
(see Sec~\ref{sec:related}). As we demonstrate, our dual view gives rise to
design principles for NNs, supporting current practice and suggesting
new ideas. We outline below a few points.

\begin{itemize}

\item Duals of convolutional networks appear a more suitable fit for
	vision and acoustic tasks than those of fully connected networks.

\item Our framework surfaces a principled initialization scheme. It is
	very similar to common practice, but incorporates a small correction.

\item By modifying the activation functions, two consecutive fully connected
	layers can be replaced with one while preserving the network's dual kernel.

\item The ReLU activation, i.e. $x \mapsto \max(x,0)$, possesses favorable
	properties. Its dual kernel is expressive, and it can be well approximated by
	random initialization, even when the initialization's scale is moderately
	changed.

\item As the number of layers in a fully connected network becomes very
	large, its dual kernel converges to a degenerate form for any non-linear
	activation.

\item Our result suggests that optimizing the weights of the last layer can
	serve as a convex proxy for choosing among different architectures prior
	to training. This idea was advocated and tested empirically
	in~\cite{saxe2011random}.

\end{itemize}

\vspace{-5pt}
\section{Related Work}
\label{sec:related}
\vspace{-8pt}
\noindent\textbf{Self-Supervised Learning.} 
Recent SSL approaches have shown performance comparable to their supervised learning equivalents~\cite{caron2020unsupervised, caron2021emerging, chen2020simple, he2020momentum, grill2020bootstrap, zbontar2021barlow, bardes2021vicreg, chen2020improved}. In a nutshell, most of these methods use image augmentation techniques to generate correlated views (positives) from a sample, and then learn a model that is invariant to these augmentations by enforcing the network to output similar representations for the positives. Initially, contrastive learning, based on instance discrimination~\cite{wu2018unsupervised} using noise-contrastive estimation~\cite{gutmann2010noise, oord2018representation}, was a popular strategy~\cite{chen2020simple, he2020momentum}. However, this learning paradigm 
requires large batch sizes or memory banks. A few methods that use a negative-free cosine similarity loss~\cite{grill2020bootstrap, chen2021exploring} have addressed such issues.

Concurrently, clustering-based methods (SwAV~\cite{caron2020unsupervised}, DeepCluster v2~\cite{caron2020unsupervised, caron2018deep} and DINO~\cite{caron2021emerging}) have also been proposed. They do not operate on the features directly, and instead compare positives through a cross-entropy loss using cluster prototypes as a proxy. Redundancy reduction-based methods have also been popular\cite{ermolov2021whitening, zbontar2021barlow, bardes2021vicreg}. Among them, BarlowTwins \cite{zbontar2021barlow} considers an objective function measuring the cross-correlation matrix between the features, and VicReg\cite{bardes2021vicreg} uses a mix of variance, invariance and covariance
regularizations. Methods such as~\cite{dwibedi2021little} have explored the use of nearest-neighbour retrieval and divide and conquer~\cite{tian2021divide}. However, none of these works studied the ability of SSL methods to learn continually and adaptively.


\noindent\textbf{Continual Learning.} A plethora of methods have been developed to counteract catastrophic forgetting~\cite{kirkpatrick2017overcoming, Rusu16progressive, shin2017continual, Lopez-Paz17, Chaudhry19, serra2018overcoming, chaudhry2018riemannian, Aljundi17, Zenke17, buzzega2020dark, fini2020online, douillard2020podnet, wu2019large, castro2018end, rebuffi2017icarl, hou2019learning, prabhu2020gdumb, Li17learning, ostapenko2019learning, cha2021co2l, Robins95}. Following~\cite{de2019continual}, these works can be organized into three macro-categories: replay-based~\cite{ostapenko2019learning, Robins95, rebuffi2017icarl, buzzega2020dark, Chaudhry19, Lopez-Paz17}, regularization-based~\cite{fini2020online, Li17learning, shin2017continual, kirkpatrick2017overcoming, Zenke17, Aljundi17, castro2018end, douillard2020podnet, hou2019learning, chaudhry2018riemannian, wu2019large, cha2021co2l}, and parameter isolation
methods~\cite{Rusu16progressive, serra2018overcoming}. All these works evaluate the effectiveness of CL methods using a linear classifier learned
sequentially over time. However, this evaluation does not reflect an important aspect, \textit{i.e.}, the internal dynamics of the hidden representations. Moreover, most CL methods tend to rely on supervision in order to mitigate catastrophic forgetting. A few of them can be adapted for the unsupervised setting, although their effectiveness is greatly reduced (see discussion in Sec.~\ref{sec:cassowary}, Sec.~\ref{sec:experiments} and the supplementary material). 

Works such as~\cite{rao2019continual, achille2018life, smith2019unsupervised} laid the foundations of unsupervised CL, but their studies are severely limited to digit-like datasets, \emph{e.g.}, MNIST and Omniglot, and the proposed methods are unfit for large-scale scenarios. Recently, \cite{gallardo2021self, caccia2021special} explored self-supervised pretraining for supervised continual learning with online and few-shot tasks, and \cite{cha2021co2l} presented a supervised contrastive CL approach. Two concurrent works~\cite{lin2021continual, madaan2021rethinking} have also attempted to address CSSL recently. The former~\cite{lin2021continual} extends~\cite{cha2021co2l} to the unsupervised setting, but is specifically designed for contrastive SSL, such as~\cite{chen2020simple,he2020momentum}, and lacks generalizability to other popular SSL paradigms. The latter~\cite{madaan2021rethinking} is also limited as it only shows small-scale experiments in the class-incremental setting and considers just two SSL methods. In contrast, we present a general framework for CSSL with superior performance, conduct large-scale experiments on three challenging settings, thereby presenting a deeper analysis of CSSL.
\section{Designing Tests for Creative Writing} \label{sec:approach}
\subsection{Rethinking Torrance Test of Creative Thinking}
Built on J.P. Guilford's work and created by Ellis Paul Torrance, the Torrance Tests of Creative Thinking, a test of creativity, originally involved simple tests of divergent thinking and other problem-solving skills, which were scored on four scales:

\begin{figure*}
\centering
\small
\includegraphics[width=0.8\textwidth]{figures/interface.png}
\caption{\label{fig:interface}Interface to collect Creativity measurements across the four-dimensional analytical framework derived from the Torrance Test of Creative Thinking}
\end{figure*}

\begin{itemize}
    \item Fluency. The total number of interpretable, meaningful, and relevant ideas generated in response to the stimulus.
    \item  Flexibility. The number of different categories of relevant responses.
    \item Originality. The statistical rarity of the responses.
    \item Elaboration. The amount of detail in the responses.
\end{itemize}

While prior work \cite{10.1145/3313831.3376495,10.1145/1978942.1979048,Beketayev2016ScoringDT} has utilized Torrance Test of Creative Thinking in other domains, it hasn't been used to evaluate creative writing. Even though the overall dimensions are generalizable across various forms of art, there are measures specific to creative writing along these dimensions that require the existing definitions to be revised and expanded. For this we recruit experts in creative writing. 

\subsection{Participant Recruitment}
In prior work \cite{gero2023social} has argued that the definition of an ‘expert’ or ‘amateur’ creative writer is difficult in a field that has unclear professional delineations.Though it could be feasible to enlist individuals self-identifying as creative professionals, our requirements necessitated the procurement of participants possessing an intricate and analytical comprehension of the creative writing process. Accordingly, we delimited our participant acquisition to those possessing either a structured educational background in creative writing (for instance, a Master of Fine Arts in Creative Writing), traditionally published authors \footnote{We do not recruit self-published authors}, or lecturers/professors instructing Fiction Writing at the university level.Our recruitment thus resulted in participants who have published novels with leading publishing giants, students enrolled in top MFA programs in the United States, University professors teaching Fiction Writing and screenwriters from prime-time network. Participants were recruited through \textit{User Interviews} (a professional freelancing website) and were paid 70\$ for a taking part in an hour long survey. Table \ref{surveyprof} shows our recruited participants.

\begin{table}[!ht]
\centering
\small
\def\arraystretch{1.35}
\parbox{.45\linewidth}{\begin{tabular}{ll}
\hline
ID & Profession                  \\ \hline
W1 & Professor of Creative Writing \\ \hline
W2 & Professor of Creative Writing \\ \hline
W3 & Lecturer in Creative Writing  \\ \hline
W4 & MFA Fiction Student           \\ \hline
W5 & MFA Fiction Student           \\ \hline
W6 & MFA Fiction Student           \\ \hline
W7 & Author                        \\ \hline
W8 & ScreenWriter                  \\ \hline
\end{tabular}
\vspace{2ex}
\caption{\label{surveyprof}Background of Participants recruited for collecting judgements about Creativity across the dimensions of Torrance Test}
}
\quad\quad\quad\quad
\parbox{.45\linewidth}
{\begin{tabular}{ll}
\hline
Cluster & Participants                  \\ \hline
Narrative Pacing & W4,W6,W7 \\ \hline
\begin{tabular}[c]{@{}l@{}}Understandability\\ \& Coherence\end{tabular}  & W3,W7 \\ \hline
\begin{tabular}[c]{@{}l@{}}Language Proficiency\\ \& Literary Devices\end{tabular}  & W4,W2  \\ \hline
Narrative Ending & W3           \\ \hline
Scene vs Summary & W2,W5,W8          \\ \hline\hline
Structural Flexibility & W3,W7,W8           \\ \hline
Perspective \& Voice Flexibility & W3,W6                        \\ \hline
Emotional Flexibility & W3                  \\ \hline\hline
Originality in Theme/Content & W3                  \\ \hline
Originality in Thought & W1,W2,W3,W5,W7                  \\ \hline
Originality in Form/Structure & W2,W4                  \\ \hline\hline
World Building \& Setting & W2,W6                  \\ \hline
Rhetorical Complexity & W3,W4                  \\ \hline
Character Development & W2,W3,W4,W5,W7,W8                  \\ \hline
\end{tabular}
\vspace{2ex}
\caption{\label{testsource}Background of Participants recruited for collecting judgements about Creativity across the dimensions of Torrance Test}}
\vspace{-5ex}
\end{table}


\subsection{Collecting Creativity Measures}

\begin{table*}[!ht]
\centering
\small
\def\arraystretch{1.5}
\begin{tabular}{|l|l|l|}
\hline
\multirow{5}{*}{Fluency}     & Narrative Pacing                                                                   & \textit{\textbf{\begin{tabular}[c]{@{}l@{}}Does the manipulation of time in terms of compression or stretching\\ feel appropriate and balanced?\end{tabular}}}                                                                                                    \\ \cline{2-3} 
                             & Scene vs Exposition                                                                & \textit{\textbf{\begin{tabular}[c]{@{}l@{}}Does the story display an awareness and insight into the balance \\ between scene and summary/exposition?\end{tabular}}}                                                                                               \\ \cline{2-3} 
                             & \begin{tabular}[c]{@{}l@{}}Language Proficiency \&\\ Literary Devices\end{tabular} & \textit{\textbf{\begin{tabular}[c]{@{}l@{}}Does the story make sophisticated use of idiom or metaphor or\\ literary allusion?\end{tabular}}}                                                                                                                     \\ \cline{2-3} 
                             & Narrative Ending                                                                   & \textit{\textbf{\begin{tabular}[c]{@{}l@{}}Does the end of the story feel natural and earned, as opposed to \\ arbitrary or abrupt?\end{tabular}}}                                                                                                               \\ \cline{2-3} 
                             & \begin{tabular}[c]{@{}l@{}}Understandability \&\\ Coherence\end{tabular}           & \textit{\textbf{\begin{tabular}[c]{@{}l@{}}Do the different elements of the story work together to form a \\ unified, engaging, and satisfying whole?\end{tabular}}}                                                                                              \\ \hline\hline
\multirow{3}{*}{Flexibility} & \begin{tabular}[c]{@{}l@{}}Perspective \& Voice \\ Flexibility\end{tabular}        & \textit{\textbf{\begin{tabular}[c]{@{}l@{}}Does the story provide diverse perspectives, and if there are unlikeable\\ characters, are their perspectives presented convincingly and accurately?\end{tabular}}}                                                    \\ \cline{2-3} 
                             & Emotional Flexibility                                                              & \textit{\textbf{\begin{tabular}[c]{@{}l@{}}Does the story achieve a good balance between interiority and exteriority, \\ in a way that feels emotionally flexible?\end{tabular}}}                                                                                 \\ \cline{2-3} 
                             & Structural Flexibility                                                             & \textit{\textbf{Does the story contain turns that are both surprising and appropriate?}}                                                                                                                                                                          \\ \hline\hline
\multirow{3}{*}{Originality} & Originality in Thought                                 & \textit{\textbf{Is the story an original piece of writing without any cliches?}}                                                                                                                                                                                  \\ \cline{2-3} 
                             & \begin{tabular}[c]{@{}l@{}}Originality in Theme \\ \& Content\end{tabular}                                      & \textit{\textbf{\begin{tabular}[c]{@{}l@{}}Will an average reader of this story obtain a unique and original idea \\ from reading it?\end{tabular}}}                                                                                                              \\ \cline{2-3} 
                             & \begin{tabular}[c]{@{}l@{}}Originality in Form/ \\ Structure\end{tabular}          & \textit{\textbf{Does the story show originality in its form and/or structure?}}                                                                                                                                                                                   \\ \hline\hline
\multirow{3}{*}{Elaboration} & \begin{tabular}[c]{@{}l@{}}World Building \& \\ Setting\end{tabular}               & \textit{\textbf{Does the writer make the fictional world believable at the sensory level?}}                                                                                                                                                                      \\ \cline{2-3} 
                             & Character Development                                                              & \textit{\textbf{\begin{tabular}[c]{@{}l@{}}Does each character in the story feel developed at the appropriate complexity\\ level, ensuring that no character feels like they are present simply to satisfy\\ a plot requirement?\end{tabular}}} \\ \cline{2-3} 
                             & Rhetorical Complexity                                                              & \textit{\textbf{Does the story operate at multiple 'levels' of meaning (surface and subtext)?}}                                                                                                                                                                  \\ \hline
\end{tabular}
\vspace{2ex}
\caption{\label{CreativityTest} Clusters with individual questions to empirically measure creativity across the four dimensions of Torrance Test}
\end{table*}

Following participant enlistment, we explained our objective aimed at assessing creativity in any piece fiction or creative non-fiction, utilizing the four-dimensional analytical framework derived from the Torrance Test of Creative Thinking. Explicit instructions were provided requesting participants to concisely articulate methodologies employed in gauging creativity across these dimensions. In formulating their responses, participants were exhorted to adopt an empirical mindset and eschew abstract or indeterminate terminologies that lack quantifiable attributes. Figure \ref{fig:interface} shows our interface to collect these responses. For each dimension we allowed our participants to state up to 5 measures. We received a total 126 measures from 8 participants across the 4 dimensions. On average each participant provided with 16 measures, 4 for each dimension of creativity.

The measures derived from the involved participants exhibited a considerable degree of semantic congruence. To consolidate these measures from them we prompt GPT-4 \cite{OpenAI2023GPT4TR} to distribute them into individual clusters, each of which encapsulates a generalized representation of a given measure.Each individual cluster was subsequently subjected to a rigorous review by a panel of four domain-specific experts to confirm both the validity of their classification and the comprehensiveness of the represented measures.Measures that could not be quantified empirically at all were discarded.In total we ended up with 14 distinct clusters with 5 representative measures for Fluency and 3 representative measures each for Flexibility, Originality and Elaboration.Some of these measures for creativity came from an individual participant while some measures were uniformly suggested by multiple participants as can be seen in Table \ref{testsource}.

\section{What do these measures tell about Creative Writing?}
Table \ref{CreativityTest} shows the representative measure from each individual cluster across the four dimensions of Creativity.These measures test several aspect of creative writing ranging from originality of thought to well-rounded character development.
\subsection{Fluency}
Compared with both reading and speaking fluency, writing fluency has always been traditionally harder to define \cite{abdel2013we}.Our 5 measures across this dimensions each look at individual aspects of creative writing. 
\subsubsection{Narrative Pacing} 
This measure refers to the manipulation of time in storytelling for dramatic effect. Essentially, it's about controlling the perceived speed and rhythm at which a story unfolds.A skilled writer can manipulate the relationship between these two to affect the pacing of the narrative, either speeding it up (compression) or slowing it down (stretching). This technique plays a crucial role in shaping the reader's experience and engagement with the story. 
\footnote{\url{https://www.writingclasses.com/toolbox/articles/stretching-and-shrinking-time}}
\subsubsection{Scene vs Exposition}
A 'Scene' is a moment in the story that is dramatized in real-time often featuring character interaction, dialogue, and action while 'Exposition', on the other hand, involves summarizing events or providing information like character history, setting details, or prior events. The right balance between scene and summary/exposition can vary depending on the story, but in general, it's essential for maintaining a good pace, keeping the reader engaged, and delivering necessary information \cite{burroway2019writing}. 
\footnote{\url{https://creativenonfiction.org/syllabus/scene-summary/}}
\subsubsection{Language Proficiency \& Literary Devices} 
Eminent novelist Milan Kundera said ``\textit{Metaphors are not to be trifled with. A single metaphor can give birth to love.}". Sophisticated use of literary allusion or figurative language such as metaphor/idioms often add depth, interest, and nuanced meaning to any creative writing. It allows for a richer reading experience, where the literal events are imbued with deeper symbolic or thematic significance. 
\subsubsection{Narrative Ending}
In her New Yorker essay ``On Bad Endings" \cite{BadEndings} Joan Accocela writes 
            \begin{quote}
                \centering
                ``Another possibility is that the author just gets tired. I review a lot of books, many of them non-fiction. Again and again, the last chapters are hasty and dull. `I’ve worked hard enough,' the author seems to be saying. `My advance wasn’t much. I already have the idea for my next book. Get me out of here.'"
            \end{quote}
If the writer ends the piece simply because they are ``tired of writing", the conclusion might feel abrupt, disjointed, or unfulfilling to the reader. This is one of the important factors of creative writing fluency.A strong ending offers a sense of closure, ties up the central conflicts or questions of the story, and generally leaves the reader feeling that the narrative journey was worthwhile and complete.
\subsubsection{Understandability \& Coherence}
Narrative coherence is the degree to which a story makes sense  \footnote{https://en.wikipedia.org/wiki/Narrative\_paradigm}.A well-crafted story usually follows a logical path, where the events in the beginning set up the middle, which then logically leads to the end. Every scene, character action, and piece of dialogue should serve the story and propel it forward. Well-written stories have an underlying unity that binds the elements together. The themes, plotlines, character arcs, and other elements of the story interweave to create a harmonious whole. A story with `disorder' might feel disjointed, with characters, scenes, etc that don't connect or contribute to the overall narrative.

\subsection{Flexibility}
Flexibility is often referred to as the ability to look at something from a different angle or point of view. In the context of creative writing our participants agreed on 3 distinct measures of Flexibility
\subsubsection{Perspective \& Voice Flexibility}
An \textit{omniscient} narrator is the all-knowing voice in a story that can convincingly and accurately depict a wide range of character viewpoints, including those of characters who may be morally ambiguous, difficult, or otherwise unappealing. This can also potentially involve diving into the mindset of characters who may act or think in ways that the reader, or even the writer, finds objectionable or repugnant.As stated in \cite{friedman1955point} an omniscient narrator enhances a sense of reliability or truth within literary works since readers are given deeper insights into many characters. The multiple viewpoints feel more objective because readers have access to multiple interpretations of events and can thus decide how they feel about each character’s perspective.
\subsubsection{Emotional Flexibility}
Emotional flexibility is asking whether the piece of writing effectively balances action and introspection, and if it portrays a broad and realistic spectrum of emotions. \textit{Exteriority} refers to the observable actions, behaviors, or dialogue of a character, and the physical or visible aspects of the setting, plot, and conflicts.\textit{Interiority}, on the other hand, pertains to the inner life of a character — their thoughts, feelings, memories, and subjective experiences. A balance between these two aspects is crucial in creating well-rounded characters and compelling narratives. As stated in \cite{campe2014rethinking} if a story is too heavy on exteriority, it may feel shallow or lack emotional depth. If it leans too much on interiority, it could become overly introspective and potentially lose the momentum of the plot.
\subsubsection{Structural Flexibility}
A good piece of creative writing often has plot twists, character developments, or thematic revelations that surprise the reader, subverting their expectations in a thrilling way. It's about keeping readers engaged and curious, never fully knowing what's going to happen next. However despite the surprises and twists, the turns in the story must also make sense within the established context of the story's universe, its characters, and its themes. This means that even though an event might be surprising, it should feel appropriate or fitting in hindsight. It shouldn't feel like the writer has broken the rules they've set up, or made a character behave inconsistently without reason, simply for the sake of shock value.
\subsection{Originality}
Creative writing requires originality, or the ability to generate unique ideas \cite{ward1999creative}. Originality can conveyed in several ways. Our participants suggested three unique ways in which they look for originality in creative writing.
\subsubsection{Originality in Theme and Content}
In his book ``Literature and the Brain" well-known literary critic and scholar Norman Holland discusses how stories stimulate the mind and impact readers \cite{holland2009literature}. A good story that offers a deeper understanding of human nature, cultural insights,unique viewpoints, or even the exploration of new ideas and themes has a lasting impact on its reader and society .It is meant to entertain, inform, provoke thought, challenge beliefs, provide comfort, or raise awareness on specific issues.In ``Poetic Justice", prominent philosophers Martha Nussbaum explores how the literary imagination is an essential ingredient of public discourse and a democratic society \cite{nussbaum1997poetic}.As such originality in theme and content is an important measure of creative writing.
\subsubsection{Originality in Thought}
A cliche is an idea, expression, character, or plot that has been overused to the point of losing its original meaning or impact \cite{fountain2012cliches}. They often become predictable and uninteresting for the reader.In his book \cite{clark2008writing} eminent American writer, editor, and a writing coach: Roy Peter Clark advised writers to strictly avoid cliches because they often indicate a lack of original thought or laziness in language use.Originality suggests that the piece isn't cliche.
\subsubsection{Originality in Form/Structure}
In his book \cite{boardman1992narrative} Michael M. Boardman discusses how innovation in narrative structure can serve ideological purposes and challenge conventional narrative forms. Frederic Jameson  highlighted the complexities of postmodern literature, where the blurring of genres and innovation in form and structure was a key characteristic\cite{jameson1991postmodernism}. Originality in form/structure has also been accomplished by unconventional use of format, genre or plot structure.For instance, the Pulitzer winning book \textit{The Color Purple} from Alice Walker is told through a series of letters written by the protagonist. Neil Gaiman's \textit{American Gods} on the other hand combines elements of fantasy, mystery, and mythic fiction in unexpected ways. \textit{The Sound and the Fury} by William Faulkner deviates from the traditional plot structure by presenting a narrative that unfolds through the stream of consciousness of different characters. The goal of originality in form or structure is often to provide a fresh reader experience, challenge conventional reading expectations, or to create a deeper or more complex exploration of the story's themes.
\subsection{Elaboration}
\subsubsection{World Building and Setting}
American poet and memoirist Mark Doty discusses the importance of create a vivid, immersive reality at the sensory level through the use of detailed, evocative description \cite{doty2014art84794531}.An effective writer often uses sensory details to paint a detailed picture of the story's environment, making it feel tangible and real to the reader.For example, describing the specific colors and shapes in a scene, the sounds that fill a space, the textures and temperatures that a character comes into contact with, the flavors of the food they eat, or the scents that fill the air, can all contribute to creating a sensory-rich and believable world.By stimulating the reader's senses, the writer can make the reader feel as though they're experiencing the events of the story firsthand. This level of detail contributes to the believability of the world, even if it's a completely fictional or fantastical setting. It helps the reader to suspend disbelief and become more deeply invested in the narrative.
\subsubsection{Character Development}
A 'flat character' is typically a minor character who is not thoroughly developed or who does not undergo significant change or growth throughout the story. They often embody or represent a single trait or idea, and they're only used to advance the plot or highlight certain qualities in other characters. A 'complex character' on the other hand also known as a round character, has depth in feelings and passions, has a variety of traits of a real human being, and evolves over time. They have their strengths, weaknesses, and they learn from their experiences. \cite{forster1927aspects,fishelov1990types,currie1990nature} highlights that any creative piece of fiction or non-fiction tend to be more engaging to the reader when authors can take a character who initially appears to be one-dimensional or stereotypical (flat) and add depth to them, as it mirror the complexity of real people.
\subsubsection{Rhetorical Complexity}
In Ernest Hemingway's short story "Hills Like White Elephants," the couple's conversation about seemingly unrelated topics implies a much deeper and more serious discussion about an abortion. Their actual dialogue never directly addresses this issue, but it's heavily suggested through what's left unsaid — the subtext. Effective writing often operates on both surface and subtext levels. The surface text keeps the reader engaged with the plot and characters, while the subtext provides depth, complexity, and additional layers of interpretation, contributing to a richer and more rewarding reading experience \cite{kochis2007baxter,phelan1996narrative}.


\documentclass{article}
\usepackage{graphicx} % Required for inserting images

\usepackage{xcolor}
\usepackage{colortbl}
\definecolor{gray0}{gray}{0.9}

\usepackage{multirow}
% for symbol x
\usepackage{bbding}
\usepackage{pifont}
\usepackage{utfsym}
\newcommand{\cmark}{\ding{51}\xspace}%
% \newcommand{\cmarkg}{\textcolor{lightgray}{\ding{51}}\xspace}%
\newcommand{\xmark}{\ding{55}\xspace}%
% \newcommand{\xmarkg}{\textcolor{lightgray}{\ding{55}}\xspace}%
\definecolor{raycolor}{RGB}{255,192,0}

\usepackage{booktabs}

% for mutiple table
\usepackage{floatrow}
\floatsetup[table]{capposition=top}
\newfloatcommand{capbtabbox}{table}[][\FBwidth]

\title{FinegrainDynamicCache}
\author{liufeng }
\date{September 2024}

\begin{document}

\maketitle

\section{Introduction}

% \begin{table*}[t]
  
%   \scriptsize
%   \centering
% \begin{tabular}{l  | c  c  cc }
% \toprule
% \textbf{Methods}    & \textbf{Vbench}   & \textbf{Lantency(s)}     & \textbf{Speedup}              \\
% \midrule
% OpenSora(30 steps) & 79.44 & 48.6  & 1.00x  \\
% Sparse4Dv2~\cite{lin2023sparse4d} & V2-99  & 900$\times$640 & 18.9 & 13.4 & 0.832 & 0.343 & 0.723 \\
% StreamPETR~\cite{Wang_2023_ICCV}  & V2-99  & 900$\times$640 & 20.3 & 14.6 & 0.843 & 0.321 & 0.650 \\
% \rowcolor{gray0}  RayDN (Ours) & V2-99      & 900$\times$640 &\textbf{22.3}      &\textbf{16.1}      & 0.825     & 0.325     & 0.629         \\

% \bottomrule
% \end{tabular}
% \caption{Comparisons on the Argoverse 2 validation set. We evaluate across 26 object categories within a range of 150 meters.}
% \label{tab:argoverse}
% \end{table*}

\begin{table}[]
\begin{tabular}{cccc}
\toprule
Method              & Vbench         & Lantency(s) & Speedup        \\
\midrule
Open Sora(30 steps) & 79.44          & 48.6        & 1.0x           \\
\midrule
delta DiT           & 78.21          & 47.2        & 1.03x          \\
T-GATE              & 77.61          & 40.8        & 1.19x          \\
PAB-246             & 78.51          & 37.6        & 1.29x          \\
PAB-579             & 76.95          & 35.4        & 1.37x          \\
\midrule
DynamicCache-0.2(Ours)     & \textbf{78.99} & 27.8        & 1.75x          \\
DynamicCache-0.25(Ours)    & 78.88          & \textbf{24.0}        & \textbf{2.03x} \\
\bottomrule
\end{tabular}
\end{table}



\begin{table}[]
\begin{tabular}{cccc}
\toprule
Method                    & Vbench         & Lantency(s)   & Speedup        \\
\midrule
Open Sora Plan(150 steps) & 80.39          & 107.2         & 1.0x           \\
\midrule
delta DiT                 & 77.55          & 106.2         & 1.01x          \\
T-GATE                    & 80.15          & 90.8          & 1.18x          \\
PAB-246                   & 80.30          & 81.6          & 1.31x          \\
PAB-579                   & 71.81          & 72.4          & 1.48x          \\
\midrule
DynamicCache-0.05(Ours)          & \textbf{80.34} & 32.6          & 3.29x          \\
DynamicCache-0.1(Ours)           & 79.68          & \textbf{23.2} & \textbf{4.62x}\\
\bottomrule
\end{tabular}
\end{table}



\begin{table}[]
\begin{tabular}{cccc}
\toprule
Method           & Vbench         & Lantency(s) & Speedup        \\
\midrule
Latte(50 steps)  & 77.40          & 27.8        & 1.0x           \\
\midrule
delta DiT        & 52.00          & 27.2        & 1.02x          \\
T-GATE           & 75.42          & 24.6        & 1.13x          \\
PAB-235          & 76.32          & 22.6        & 1.23x          \\
PAB-469          & 73.13          & 20.6        & 1.35x          \\
\midrule
DynamicCache-0.03(Ours) & \textbf{77.19} & 16.4        & 1.70x          \\
DynamicCache-0.05(Ours) & 76.79          & \textbf{12.0}        & \textbf{2.32x} \\
\bottomrule
\end{tabular}
\end{table}


\begin{table}[]
\begin{tabular}{cccc}
\toprule
Method              & Vbench         & Lantency(s)   & Speedup        \\
\midrule
Open Sora(30 steps) & 79.44          & 48.6          & 1.0x           \\
Open Sora(15 steps) & 77.34          & 26.4          & 1.84x          \\
DynamicCache-0.25(Ours)    & \textbf{78.88} & \textbf{24.0} & \textbf{2.03x} \\
\bottomrule
\end{tabular}
\end{table}


\begin{table}[]
\begin{tabular}{cccc}
\toprule
Method               & Vbench         & Speedup                \\
\midrule
Open Sora(30 steps)  & 79.44          & 1.0x                    \\
DynamicCache-timestep(Ours) & \textbf{79.14} & \textbf{1.75x}                    \\
DynamicCache-output(Ours)   & 78.99 & \textbf{1.75x}  \\
\bottomrule
\end{tabular}
\end{table}


\begin{table}[]
\begin{tabular}{cccc}
\toprule
Method               & Vbench         & Speedup               \\
\midrule
OpenSora(240p)       & 77.48          & 1.0x                      \\
DynamicCache-timestep(Ours) & 77.34 & \textbf{1.5x}                    \\
DynamicCache-output(Ours)   & \textbf{77.42} & 1.34x & \\
\bottomrule
\end{tabular}
\end{table}


\begin{table}[]
\begin{tabular}{cccc}
\toprule
Method               & Opensora 1.0         & Opensora 1.2 & Kling 1.5                \\
\midrule
Resolution       & 512x512          & 720p  & 1080p                      \\

\bottomrule
\end{tabular}
\end{table}

\end{document}

\section{Evaluation Protocol}
As mentioned earlier our evaluation protocol focuses on the absolute evaluation of the 48 stories on each of the 14 questions shown in Table \ref{CreativityTest} as well as a relative evaluation of 12 clusters for discerning whether a story on the same plot within an individual cluster has been produced by a human or an LLM( Turing Test).
\subsection{Absolute Evaluation}
\subsubsection{Absolute Evaluation with LLMs}
In prior work, \cite{gao2023human} has demonstrated the effectiveness of GPT3.5 over automated metrics in summarization evaluation.\cite{liu2023gpteval} employ the framework of using large language models with chain-of-thoughts (CoT) \cite{wei2022chain} to assess the quality of NLG outputs. Further \cite{rajani2023llm_labels} have shown that GPT4 as an evaluator has a higher correlation with humans for the tasks of brainstorming or creative generation. Following prior work we also use GPT3.5, GPT4, and Claude respectively to provide a verdict (Yes/No) to each of the 14 questions stated in Table \ref{CreativityTest} along with a plausible explanation using Chain-Of-Thought reasoning justifying the answer choice. More details about the evaluation prompts can be found in Section \ref{sec:prompting}. The purpose of our assessment utilizing Large Language Models (LLMs) is not to champion the prospect of a future in which LLMs can adjudicate creativity. Rather, our aim is to decipher the divergence between expert human evaluators and artificial intelligence in comprehending the essence of creativity. We posit that if such models lack the fundamental comprehension of the constituents of creativity, their proficiency in creative writing tasks could be deemed sub-optimal.
\subsubsection{Absolute Evaluation with Creative Writing Experts} A cogent argument posits that the engagement with artistic prose isn't confined solely to specialists, but non-experts also can be competent in assessing imaginative prowess. In prior work, numerous researchers have leveraged annotators sourced from prevalent crowd-sourcing platforms to gauge model outputs, particularly in the domain of creativity. Nevertheless, it is critical to acknowledge that during narrative evaluation trials involving both teachers in English and participants from Amazon Mechanical Turk, the study by \cite{karpinska-etal-2021-perils} exhibited that AMT contributors, even when shortlisted via rigorous eligibility parameters (unlike teachers), struggle to discriminate between model generated text and human-crafted references.In addition, the recent study by \cite{veselovsky2023artificial} highlighted the fact that approximately 33-46\% of crowdworkers on such platforms currently utilize large language models (LLMs) to complete any assigned task.

These complications create the necessity to enlist the expertise of creative writing professionals garnered through \textit{User Interviews}, a prominent freelancing digital platform. A total of eight individuals, adept in creative writing, were enlisted for this endeavor. Five of these experts are associated with the creative writing departments at leading American academic institutions, with considerable experience in conducting undergraduate and graduate level courses. Two of these experts function as literary agents at a top-tier, full-service US literary agency that represents well-recognized authors. The last expert is a Master of Fine Arts student specializing in fiction, who was nominated for the esteemed Pushcart Prize. Similar to LLM's we also ask these experts to provide a verdict (Yes/No) to each of the 14 questions stated in Table \ref{CreativityTest} and subsequently elicit 2-3 sentence explanations justifying the answer.

We conduct this study at a cluster level consisting of 4 stories based on the same plot, however we explicitly ask them to not read all the stories together, as this might potentially confound the ensuing analyses. Participants are explicitly instructed to approach each narrative in an isolated manner, answering a set of 14 targeted queries per individual narrative. All the four stories on the same plot are shuffled randomly based on their source and shared to the experts over a google doc. We use a google form to collect 56 answers and explanations from experts for all the 4 stories within a cluster and pay them 80\$ for a task estimate of 2-2.5 hrs.
\begin{figure*}
    \centering
     \includegraphics[width=\textwidth]{figures/rel.pdf}
    \caption{\label{relev} Relative Evaluation by Creative Writing Experts within a given cluster of four stories}
\end{figure*}
\subsubsection{Relative Evaluation with Creative Writing Experts} After providing verdict on each story inside a particular cluster we ask experts to rank the stories based on their personal preference \cite{ouyang2022training}. This evaluation was also designed with the hope that the most preferred story would also have the highest number of answers to individual questions from the absolute evaluation as `Yes'. Additionally we also ask them to guess the author of each story based on the Turing Test recommendation. We provide them 3 possible author choices to select between \textit{An expert, An amateur, AI} as can be seen in Figure \ref{relev}. This is further useful for us to study if there are certain patterns that leads experts to discriminative between model vs human written stories.
\section{Converting expert questions to quantifiable Natural language instructions} \label{sec:prompting}

\begin{table*}[!h]
\centering
\small
\def\arraystretch{1.15}
\begin{tabular}{|l|l|}
\hline
\begin{tabular}[c]{@{}l@{}}Expert \\ Measure\end{tabular}               & Does the writer make the fictional world believable at the sensory level?                                                                                                                                     \\ \hline
\begin{tabular}[c]{@{}l@{}}Expanded\\ Expert\\ Measure (M)\end{tabular} & \begin{tabular}[c]{@{}l@{}}Sensory details pertain to the five senses - sight, sound, touch, taste, and smell. An effective \\ writer can use these elements to paint a detailed picture of the story's environment, making\\ it feel tangible and real to the reader.\\ \\ For example, describing the specific colors and shapes in a scene, the sounds that fill a space,the \\ textures and temperatures that a character comes into contact with, the flavors of the food they \\ eat, or the scents that fill the air, can all contribute to creating a sensory-rich and believable world.\\ \\ By stimulating the reader's senses, the writer can make the reader feel as though they're \\ experiencing the events of the story firsthand.This level of detail contributes to the believability of\\ the world, even if it's a completely fictional or fantastical setting. It helps the reader to suspend\\ disbelief and become more deeply invested in the narrative.\end{tabular} \\ \hline
\begin{tabular}[c]{@{}l@{}}Human\\ Instruction\end{tabular}             & \begin{tabular}[c]{@{}l@{}}\{\{M\}\}\\ \\ Based on the story that you just read, answer the following question.\\ \textit{\color{blue}Does the writer make the fictional world believable at the sensory level?}\end{tabular}                                                                       \\ \hline
\begin{tabular}[c]{@{}l@{}}LLM\\ Instruction\end{tabular}               & \begin{tabular}[c]{@{}l@{}}\{\{M\}\}\\ \\ Given the story above, list out the elements in the story that call to each of the\\ five senses. Then overall, give your reasoning about the question below and give\\ an answer to it between 'Yes' or 'No' only\\ \\ \textit{\color{blue} Q) Does the writer make the fictional world believable at the sensory level?}\end{tabular}                                                                                                                                                                                                                                 \\ \hline
\end{tabular}
\vspace{2ex}
\caption{\label{prompting}Expert suggested question for World Building and setting (Row1) ; Elucidated prompt designed for other expert humans (Row2); Elucidated quantifiable prompt designed for Large Language Models that elicit Chain of Thought Reasoning(Row3) }
\vspace{-5ex}
\end{table*}

An expert suggested questions for empirically evaluating creative writing might frequently elicit ambiguity in Large Language Models or even other creative writing experts. In order for LLMs or other experts to comprehend the suggested questions in Table \ref{CreativityTest} we attempt at expanding them by adding more details. Recent pre-trained LLMs (e.g., GPT-4 \cite{OpenAI2023GPT4TR} GPT3.5 \cite{ChatGPT}) can engage in fluent, multi-turn conversations out of the box, substantially lowering the data and programming-skill barriers to creating passable conversational user experiences. People can improve LLM outputs by prepending prompts—textual instructions and examples of their desired interactions—to LLM inputs. The prompts steer the model towards generating the desired outputs, raising the ceiling of what conversational UX is achievable for non-AI experts.To elucidate these questions we prompt GPT4 with the following instruction \textit{What do creative experts mean when they say the following: \{\{expert question\}\}}. Once GPT4 gives a response 3 domain experts carefully verify the response and edit it where required in order to convert it into a detailed natural language instruction. Table \ref{prompting} (Row2) shows the human-verified GPT4 elucidated instruction in response to the input prompt.

Prior work \cite{wei2022chain} has shown how generating a \textit{chain of thought} -- a series of intermediate reasoning steps -- significantly improves the ability of large language models to perform complex reasoning. Taking advantage of this we design the prompts for large language models in a slightly different fashion as that of expert humans as can be seen in Table \ref{prompting}. To help the model make an informed decision we first ask it to list out elements specific to any given test such as ``elements in the story that call to each of the five senses" for the World Building and setting test followed by asking it decide overall and then provide its reasoning before choosing an answer between `Yes' or `No'.More examples of Human and LLM instructions for the remaining 13 tests are provided in Appendix \ref{appendix}
We evaluated the restoration performance of the Stochastic-Restoration-GAN (SR-GAN) \cite{lattner2021stochastic} and Apollo models across various bitrates and music genres on the combined test set of MUSDB18-HQ and MoisesDB (with 5000 samples for each case). The test set encompasses a wide range of music genres, including vocals, single instruments, and mixed instruments, aiming to comprehensively assess each model's restoration capabilities.

\textbf{Bitrate Impact Analysis.} Fig.\ref{fig:plot} compares the performance of the Apollo model and the Stochastic-Restoration-GAN (SR-GAN) at different bitrates (ranging from 24 kHz to 128 kHz). The experimental results demonstrated that Apollo consistently outperformed SR-GAN across all bitrates, particularly in addressing issues such as frequency band voids or reduced signal bandwidth, as reflected by SI-SNR and SDR scores. Additionally, Apollo significantly improved audio restoration quality as measured by VISQOL. Project page\footnote{\url{https://cslikai.cn/Apollo/}} for Apollo's reconstructed audio given multiple MP3 bitrates.

\textbf{Music Genre Impact Analysis.} Table~\ref{tab:stems} further illustrates the performance of both models across different music genres. In audio scenarios involving vocals, single instruments, mixed instruments, and a combination of instruments with vocals, Apollo consistently surpasses SR-GAN, with its advantage being especially pronounced in complex scenarios with mixed instruments and vocals. This is attributed to Apollo's alternating band and sequence modeling design, which emphasizes capturing and restoring complex spectral information. Compared to SR-GAN, Apollo delivers higher user ratings (VISQOL) with comparable inference speed while maintaining a more compact model size. This is especially important for real-time communications and live audio restoration, where low latency is critical to the user experience.


\section{Analysis}
\label{sec:analysis}

We provide some insights based on different ablations in this section. 
Llama-3.1$_{\text{70B}}$ were used as the base model for all the experiments in this section.


\subsection{Memory Configurations}



In this analysis, we explore the influence of memory configurations on factuality, based on experiments on 50 randomly sampled prompts from \lf. We examine this impact through two dimensions: the number of memory units and the shape of memory units.


\begin{figure}
\centering
\includegraphics[width=0.4\textwidth]{figures/factcheck_memory_config_ablation.png}
\includegraphics[width=0.4\textwidth]{figures/retrieval_memory_config_ablation.png}
    \caption{\vs F$_1$ over 50 prompts from \lf when varying number of memory units used for storing retrieved passages and fact-checking feedback. Each memory unit stores 128 tokens.}
    \label{fig:memory_unit_number_ablation}
\end{figure}

In Figure~\ref{fig:memory_unit_number_ablation}, we investigate how varying the numbers of memory units used for storing fact-checking feedback and retrieved passages may impact factuality. When adjusting the configuration for one, we keep the other constant to facilitate easier interpretation of the results. Overall, we observe that having a large amount of memory units for either fact-checking feedback or retrieved passages \textit{negatively} impacts factuality. 
This is likely because a significant amount of stale information remains in working memory for an extended period without being updated, as we adhere to the FIFO rule for updating working memory. 
Consequently, this information becomes outdated as the generation process continues.



\begin{table}
    \centering
    \begin{tabular}{lccc} \toprule
      Memory Shape ($M\times k$) & $128\times 20$ & $256\times 10$ & $512\times 4$ \\ \midrule
       F1  & \bf 74.0 & 69.0 & 68.2 \\\bottomrule
    \end{tabular}
    \caption{\vs F$_1$ over 50 prompts from \lf with different shapes of working memory. We allocate an equal number of memory units for both retrieval and fact-checking feedback.}
    \label{tab:memory_unit_shape_ablation}
\end{table}


In Table~\ref{tab:memory_unit_shape_ablation}, we examine the impact of varying memory unit shapes on factuality. To ensure a fair comparison, we maintained a consistent total number of tokens in working memory across different experimental setups. 
Notably, our findings suggest that models favor shorter, more memory units over longer, fewer ones. We hypothesize that this preference arises because 128 tokens approximately match the length of a retrieved passage, allowing the attention mechanism to effectively cover one individual passage at a time. 
In contrast, longer memory units combine multiple passages into a single unit, which may compel the attention to focus on less relevant passages when they are grouped with more relevant ones.



\subsection{Feedback Forms}




\begin{table}[t]
    \centering
    \begin{tabular}{cccccc} \toprule
         \multirow{2}{*}{\begin{minipage}{1.3in}Passages determining a claim is incorrect\end{minipage}} & \multirow{2}{*}{\begin{minipage}{1.3in}Passages determining a claim is correct\end{minipage}} & \multirow{2}{*}{\begin{minipage}{1.3in}Instructions\\for nonfactual claims\end{minipage}}   & Precision & Recall & F$_1$  \\
        & & & & & \\\midrule
         \checkmark & \checkmark & \checkmark &  77.3 & 64.0 & 66.8 \\
         \checkmark & & & 76.4 &\bf  67.4 & 67.9 \\
         & \checkmark & & 77.5 & 67.2 & \bf 69.4 \\
         & & \checkmark & 67.1 & 66.2 & 66.7 \\
         \checkmark & \checkmark &   &\bf  80.8 & 66.1 &\bf  69.3 \\
         & & & 72.5 & 65.9 &	66.2 \\\midrule
         \multicolumn{3}{r}{Llama-3.1$_{\text{70B}}$} & 65.8 &	67.1 &	65.5 \\
         \multicolumn{3}{r}{Llama-3.1$_{\text{70B}}$ + RA} & 70.1 & 66.1 & 65.9 \\ \bottomrule
    \end{tabular}
    \caption{Comparing different feedback forms for fact-checkers. We report \vs over 50 prompts from \lf.}
    \label{tab:feedback_form}
\end{table}


In this analysis, we explore various feedback formats utilized by fact-checkers. The models in \vs offers 2 types of information: a list of both factual and nonfactual claims, along with relevant passages that support these factual and nonfactual judgments. To examine the impact of these feedback formats, we conduct experiments using different combinations of these information types in the working memory. For the supporting passages, we combine them using new line symbols. For the list of claims, we apply an instruction template as follows to encode nonfactual claims:
\begin{TextBox}{}
{Please refrain from including the following imprecise statements: (1) nonfactual claim\textsubscript{1} (2) nonfactual claim\textsubscript{2} ...}
\end{TextBox}

Our results are shown in Table~\ref{tab:feedback_form}. Overall, fact-checking feedback is beneficial compared to the base model with and without retrieval augmentation. 
The specific types of feedback also play a crucial role. Incorporating all feedback forms does not enhance model performance, with supporting passages proving more effective than instructions. 
We notice that instructing models not to generate specific details often results in misunderstanding. Models might rephrase the instruction, include the nonfactual statement in their response, and then add a clarification indicating the previous statement is nonfactual, such as ``\textit{(Note: This is a nonfactual claim and may not be accurate.)}''. We leave a better design of feedback forms to future work. 
Interestingly, when we exclude all the textual feedback from fact-checkers and only pause and regenerate in the presence of nonfactual sentences, performance still slightly improves.


\subsection{Model Confidence}




One important question remains is when to refresh the working memory. To study it, we conducted a comparative analysis of different criteria for refreshing working memory and regeneration. 
Since the working memory consists of the retrieval memory and fact-checking memory, which can have interacting effects, we first investigate when to trigger the retriever alone (without fact-checking memory) and then investigate when to trigger the fact-checker (when retrieval interval $T_r$ is set to 1).

\paragraph{Fixed intervals for refreshing working memory}
As shown in Figure~\ref{fig:interval-fixed}, when using a fixed retrieval interval, an intermediate interval seems to perform well. 
This may be due to the fact that overly frequent retrieval can add irrelevant and conflicting information to the memory. 
With fact-checking feedback in memory, it seems frequent verification and regeneration is not always beneficial due to the fact that we only regenerate once and the regenerated sentence is not always better.

\paragraph{Model confidence for refreshing working memory}
In practice, fixed retrieval and verification intervals may be unnecessary and lead to sub-optimal performance. We explore whether model-confidence can serve as a signal for refreshing working memory. Specifically, we compare two different metrics for model confidence: (1) \textbf{Entropy}: average entropy of generated tokens in a sentence, and (2) \textbf{Min-prob}: minimum probability of tokens in a sentence. A higher threshold for entropy results in less frequent memory update, and a higher threshold for min-prob results in more frequent memory update. 
Since external fact-checkers can be computationally expensive, we first examine if we can use model confidence as a signal for retrieval and regeneration, without using an auxiliary fact-checking model to provide feedback. 
As shown in Figure~\ref{fig:interval-entropy} and \ref{fig:interval-prob} (blue line), we observe empirically intermediate thresholds for retrieval perform well, leading to to better F$_1$ when compared to the settings in Figure~\ref{fig:interval-fixed}, where we use different fixed intervals for retrieval. 
With external fact-checkers, we investigate if we can use model confidence as a signal to trigger verification and regeneration to improve generation efficiency. 
In Figure~\ref{fig:interval-entropy} and \ref{fig:interval-prob} (red line), when chosen at an appropriate threshold, both entropy and min-prob can outperform the baseline (using fixed verification interval $T_v=8$) despite with less frequent verification.



\begin{figure}[t!]
\centering
\begin{subfigure}[t]{0.32\textwidth}
        \centering
        \includegraphics[width=\textwidth]{figures/fixed_interval.pdf}
        \caption{\vs F$_1$ when using fixed retrieval and verification intervals.}
        \label{fig:interval-fixed}

    \end{subfigure}%
    \hfill
    \begin{subfigure}[t]{0.32\textwidth}
        \centering
            \includegraphics[width=\textwidth]{figures/entropy_threshold.pdf}
        \caption{F$_1$ when using entropy thresholds for triggering retrieval and verification.}
                \label{fig:interval-entropy}

    \end{subfigure}
    \hfill
    \begin{subfigure}[t]{0.32\textwidth}
        \centering
            \includegraphics[width=\textwidth]{figures/prob_threshold.pdf}
        \caption{F$_1$ when using min-prob thresholds for triggering retrieval and verification.}
            \label{fig:interval-prob}

    \end{subfigure}
\caption{Comparison of different criteria for refreshing working memory over 50 prompts from \lf. The baseline uses retrieval interval $T_r = 1$ and verification interval $T_v = 8$.}
\label{fig:interval}
\end{figure}


\subsection{Knowledge from Retrieval}

\begin{table}[t!]
    \centering
    \begin{tabular}{lcccc} \toprule
       Datastore  & \lf & \bio & \alpaca & \fava \\\midrule
       Wiki  & 63.0 & 39.1 & 55.4 & 50.5 \\
       C4 & 66.8 & \bf 45.4 & 58.6 & 53.6 \\
       C4 + Wiki & \bf 69.0 & 43.8 & \bf 58.8 & \bf 54.6 \\ \bottomrule
    \end{tabular}
    \caption{\vs F$_1$ over 50 prompts from \lf, \alpaca, \fava and \bio with different retrieval datastores.}
    \label{tab:retrieval_corpus_ablation}
\end{table}


We present the results of using different retrieval corpora in Table~\ref{tab:retrieval_corpus_ablation}, including Wikipedia, C4, or both of them together.
Likely due to its broader coverage, C4 is more effective than Wikipedia in helping the model produce more factual responses. 
Combining C4 with Wikipedia further enhances the factual accuracy (except for \bio), probably because they offer complementary sets of knowledge.












\section{Conclusion}
\section{Acknowledgments}
\section{Appendices}\label{appendix}



\bibliographystyle{ACM-Reference-Format}
\bibliography{sample-base}
\end{document}
\endinput
%%
%% End of file `sample-authordraft.tex'.
