\section{Design Considerations}\label{design}

One of the main contributions of our work is the collection of 14 tests, referred to as the \textit{Torrance Test for Creative Writing (TTCW)}, to evaluate creativity in short fictional stories. These tests were formulated through collaboration with domain experts which we detail in Section \ref{sec:approach}, but in this Section, we first present the design principles that shaped the methodology and provide the desiderata for the tests, which are then empirically validated.

\paragraph{Design Principle 1: Leveraging the Torrance Test Metrics.} The 14 tests we propose are grounded in the Torrance Test for Creative Thinking (TTCT) \cite{torrance1966torrance}, which has been a cornerstone in the evaluation of creativity.TTCT provides measures to understand the creative process through tasks that encompass unusual uses of objects, scenarios, and out-of-the-box problems. TTCT is centered around evaluating four dimensions of creativity

\begin{itemize}
\item \textbf{Fluency.} The sheer volume of meaningful ideas produced in reaction to a given stimulus.
\item \textbf{Flexibility.} The diversity of categories within the responses. 
\item \textbf{Originality.} The uniqueness or novelty of answers.
\item \textbf{Elaboration.} The depth or granularity of details within the responses.
\end{itemize}

While the direct applications of TTCT might exhibit limitations in terms of applicability across diverse creative domains \cite{amabile1982social,baer2009assessing}, its fundamental dimensions have proven adaptable. Researchers have repurposed these dimensions effectively in diverse sectors like science education \cite{trisnayanti2019development}, content strategies in marketing \cite{mcintyre2003individual}, and even in human-computer interaction, particularly interface design \cite{10.1145/3313831.3376495}.In Section \ref{sec:approach}, we show how using creative writing experts we design domain-specific tests grounded in each TTCT dimension. Feedback from these specialists further underscores the pertinence of these dimensions in assessing creative writing.

\paragraph{Design Principle 2: Artifact-centric Testing.}\footnote{In the context of creative writing we will refer to the "product" as artifact}  A key consideration when designing tests to evaluate creativity is whether to center the evaluation on the \textit{cognitive process} that leads to creativity or whether to evaluate the final artifact, which is a byproduct of the process \cite{mayers2007re}.
Much prior work -- including the TTCT -- takes a design-centric approach, as it includes richer observation of the evaluated individual, which might not be captured in the final artifact. However, process-oriented evaluation is limited in several ways. First, process-oriented evaluation is inherently limited by the quality of the observation of the individual's process. For instance, internal thoughts of the individual and other unrecorded activities can bias and lower the quality of the evaluation. Second, prior work has argued that neatly separating a process from an artifact is challenging, as the two are ``tightly integrated'' \cite{mayers2007re}, with ``the creative process leaving traces within the artifact'' \cite{murray2012craft}. Finally, observing the process is not always possible, particularly when evaluating the creativity of a preexisting artifact (e.g., a short story written years ago), or evaluating black-box agents such as LLMs, whose process cannot be observed in an interpretable way. We, therefore, follow prior work \citet{vaezi2019development,rodriguez2008problem} and design our creative writing evaluation to be artifact-centric.

\paragraph{Design Principle 3: Binary (Yes-No) Questions with Open-Ended Rationales.} In accordance with findings from prior work \cite{doi:10.1177/001316447103100307} which showed that reliability and validity are independent of the number of scale points
used for Likert-type items, we stick to a binary scale. The evaluation within each Torrance dimension follows a similar procedure. Each dimension is associated with multiple binary questions, that represent individual tests. Each binary question is formulated as having a Yes/No answer such that an artifact receiving a ``Yes'' answer to a question corresponds to the artifact passing the test. Additionally, the Yes/No answer should be accompanied by a free-text rationale written by the evaluator which justifies the chosen binary label, with a length expectation of at least 1-3 sentences. For each test, the combination of a structured binary assessment and an open-ended rationale are complementary. The binary assessment can be used for quantitative assessment, such as measuring agreement amongst evaluators, or comparative evaluation of a story collection (such as the one we perform in Section ~\ref{ref:humanresult}), whereas the rationale can be used for qualitative assessment, such as understanding concrete reasons for the passing or failing of a test, such as the analysis we perform in Section \ref{sec:analysis} centering around the most common themes that lead to the passing or failing of a given test.
\paragraph{Design Principle 4: Additive Nature of Tests.} Each question is intended to be independent of other questions (i.e., no question is a prerequisite to another question), but the creative assessment of a given artifact requires completing all the TTCW. 
The final assessment of a given artifact is the number of tests passed by the artifact, with the general expectation that passing more tests is directly proportional to the creativity of the artifact. In other words, the passing or failing of any single test cannot be interpreted as a final assessment of the creativity of an artifact, but rather the number of tests passed can paint a more complete picture of the creativity of the artifact. Analysis in Section ~\ref{ref:humanresult} confirms that our experts achieve on average moderate agreement on individual tests, but strong agreement when considering all tests in aggregate, confirming empirically the additive nature of the tests. In Section \ref{sec:approach}, we conduct a formative study with experts to formulate the fourteen TTCW that satisfy our design principles, which we then use in Section ~\ref{ref:humanresult} to run a evaluation of short stories using TTCW with experts as assessors. 

\section{Formative Study: Formulating the Torrance Tests For Creative Writing} \label{sec:approach}
We restricted the involvement of participants in our formative study to only those possessing either a structured educational background in creative writing (for instance, a Master of Fine Arts in Creative Writing), traditionally published authors \footnote{We do not recruit self-published authors}, or lecturers/professors instructing Fiction Writing at the university level. We specifically chose this filtering criterion to restrict our selection pool to experts in the field thereby aligning with the \textit{Consensual Assessment Technique}. Our recruitment resulted in participants who have published novels with leading publishing houses, students enrolled in top MFA programs in the United States, University professors teaching Fiction Writing, and screenwriters from prime-time networks. Participants were recruited through \textit{UserInterviews}~\footnote{\url{https://www.userinterviews.com}}, a professional freelancing website, and were paid \$70 for taking part in the hour-long survey. Table \ref{surveyprof} shows the background of the recruited participants. Our recruited participants span across different age groups, gender and professional expertise.

The formative study was structured in three parts. Initially, over video conferencing, experts were briefed on the study's primary objective, which aimed at devising actionable metrics for assessing creative writing, emphasizing fiction. They were also introduced to the Torrance Tests and the four specific dimensions encompassed by it. In the subsequent phase, participants were emailed the URL to a web app where they were asked to input their measures in text. They were instructed to allocate a 20-minute window to articulate up to five distinct measures corresponding to each of the Torrance dimensions first. This task was conducted without a sample fiction to maintain abstraction. The final phase involved presenting the participants with a sample fiction piece \footnote{\url{https://www.newyorker.com/books/flash-fiction/the-mirror}} on the same web app for evaluation, retrieved from The New Yorker. Participants were encouraged to use this as a tangible example to refine and augment their initial measures, ensuring they were grounded and practical. As an outcome of our study, we received 126 measures from the participants across the four Torrance dimensions. \footnote{The research was conducted at an institution that does not have an IRB approval process in place, but an Ethical Practices team reviewed the work and study protocols. We did not collect or share any PII during data collection, and participants could choose not to complete the survey and still receive a payment.}

\begin{table}[!ht]
\centering
\small
\begin{tabular}{|l|l|l|l|}
\hline
ID & Profession  & Gender & Age                \\ \hline
W1 & Professor of Creative Writing & Female & 45 \\ \hline
W2 & Professor of Creative Writing & Female & 56 \\ \hline
W3 & Lecturer in Creative Writing & Male & 40  \\ \hline
W4 & MFA Fiction Student & Male & 35        \\ \hline
W5 & MFA Fiction Student & Male & 31      \\ \hline
W6 & MFA Fiction Student & Female & 48         \\ \hline
W7 & Young Adult Fiction Writer  & Non-Binary & 39                      \\ \hline
W8 & ScreenWriter & Non-Binary & 34                  \\ \hline
\end{tabular}
\vspace{2ex}
\caption{\label{surveyprof}Background of Participants recruited for collecting judgments about Creativity across the dimensions of Torrance Test}
\end{table}

\begin{table}[!ht]
\centering
\small
\begin{tabular}{|l|l|l|}
\hline
Dimension & Expert Measure & Participants                  \\ \hline
\multirow{7}{*}{Fluency} & Narrative Pacing & W4,W6,W7 \\ \cline{2-3}
&\begin{tabular}[c]{@{}l@{}}Understandability\\ \& Coherence\end{tabular}  & W3,W7 \\ \cline{2-3}
&\begin{tabular}[c]{@{}l@{}}Language Proficiency\\ \& Literary Devices\end{tabular}  & W4,W2  \\ \cline{2-3}
&Narrative Ending & W3           \\ \cline{2-3}
&Scene vs Summary & W2,W5,W8          \\ \hline
\multirow{3}{*}{Flexibility} &Structural Flexibility & W1,W3,W8           \\ \cline{2-3}
& Perspective \& Voice Flexibility & W3,W6                        \\ \cline{2-3}
& Emotional Flexibility & W3                  \\ \hline
\multirow{5}{*}{Originality} & Originality in Theme/Content & W3,W6                 \\ \cline{2-3}
&Originality in Thought & \begin{tabular}[c]{@{}l@{}}W1,W2,W3,\\W5,W7\end{tabular}                  \\ \cline{2-3}
&Originality in Form & W2,W3,W4                  \\ \hline
\multirow{5}{*}{Elaboration} & World Building \& Setting & W2,W6                  \\ \cline{2-3}
&Rhetorical Complexity & W3,W4                  \\ \cline{2-3}
&Character Development & \begin{tabular}[c]{@{}l@{}}W2,W3,W4\\W5,W7,W8\end{tabular}                  \\ \hline
\end{tabular}
\vspace{2ex}
\caption{\label{testsource}Tests proposed and mapped by experts for evaluating story writing across TTCW dimensions}
\end{table}

\subsection{From Measures to Actionable Tests}
The measures derived from the participants exhibited a considerable degree of semantic congruence. For example, W2 proposed one way to measure Originality in Creative Writing as \textit{``Shows an innovative use of form/structure.''} while W4 proposed a similar measure \textit{Formal or stylistic novelty}.
W6 proposed one way to measure Elaboration depending on whether \textit{``The story has developed 3D characters.''} while W3 proposed an exactly similar measure but phrased it as \textit{``Does the piece make a flat character complex?''}. To consolidate these measures and develop a framework of the underlying tests for measuring creativity, we use a general inductive approach for analyzing qualitative data \cite{thomas2006general}. Following this method, three authors independently read all of the measures and assigned each measure an initial potential low-level group. Then, through repeated discussion, we reduced category overlap and created shared low-level groups associated. Finally, these low-level groups were collected into high-level groups, and a name was proposed for each group that encapsulates a generalized representation of the measures within the group. During the later meetings, an American Novelist and Creative Writing Professor were present to give further insights into the data.

In total, the tagging process yielded 14 distinct groups, 5 in the Fluency dimension, and 3 in Flexibility, Originality, and Elaboration. Table~\ref{testsource} presents the name we assigned to each group and the study participants that proposed a measure tagged within this group. For every group, we had a list of expert-suggested questions speaking about the same artifact. To choose a representative measure for every group, we selected the most well-articulated measure (in terms of word count). If the measure was already suggested as a question, we keep it intact; otherwise, we used the GPT4 model to convert a measure to a Yes/No question. For example, \textit{Originality suggests that the piece isn’t cliche $\rightarrow$ Is the story an original piece of writing without any
cliches?}
Next, we list each TTCW and provide the necessary background to contextualize the test. It should be noted that our tests are additive in nature. Failing a particular test should not be interpreted as the fact that the story doesn’t present any creative elements. Instead, it should be considered, by counting the number of distinct tests passed by a given story to obtain a more calibrated understanding of the creativity in a given piece.

\subsection{The Torrance Test for Creative Writing} \label{CreativityTest}

\paragraph{\textbf{{Fluency}}}: Compared with both reading and speaking fluency, writing fluency has always been traditionally harder to define \cite{abdel2013we}. Our 5 measures across this dimension each look at individual aspects of creative writing. 

\subsubsection{\textbf{{\color{blue} Narrative Pacing (TTCW Fluency1)}: Does the manipulation of time in terms of compression or stretching feel appropriate and balanced?}}This measure refers to the manipulation of time in storytelling for dramatic effect. Essentially, it is about controlling the perceived speed and rhythm at which a story unfolds. A skilled writer can manipulate the relationship between these two to affect the pacing of the narrative, either speeding it up (compression) or slowing it down (stretching). This technique plays a crucial role in shaping the reader's experience and engagement with the story. To assess narrative pacing, W4 suggested looking  ``\textit{Compression/stretching of time (story time vs. real world time)}'' while W6 and W7 advised to ``\textit{control the speed at which a story unfolds
\footnote{\url{https://www.writingclasses.com/toolbox/articles/stretching-and-shrinking-time}}.}''

\subsubsection{\textbf{{\color{blue}Scene vs Exposition (TTCW Fluency2)}: Does the story display awareness and insight into the balance between scene and summary/exposition?}} A `Scene' is a moment in the story that is dramatized in real-time, often featuring character interaction, dialogue, and action, while `Exposition', on the other hand, involves summarizing events or providing information like character history, setting details, or prior events. The right balance between scene and summary/exposition can vary depending on the story, but in general, it's essential for maintaining a good pace, keeping the reader engaged, and delivering necessary information \citet{burroway2019writing} 
\footnote{\url{https://creativenonfiction.org/syllabus/scene-summary/}}.
W2 strongly felt that fluent writing needs to ``\textit{display awareness and insight into the balance between scene and summary/exposition in the story.}'' while W5 and W8 emphasized the need for ``\textit{Enough dialogue to compensate for backstory.}''

\subsubsection{\textbf{{\color{blue}Language Proficiency \& Literary Devices (TTCW Fluency3)}: Does the story make sophisticated use of idiom or metaphor or literary allusion?}} Eminent novelist Milan Kundera said ``\textit{Metaphors are not to be trifled with. A single metaphor can give birth to love.}''. Sophisticated use of literary allusion or figurative language such as metaphor/idioms often add depth, interest, and nuanced meaning to any creative writing. It allows for a richer reading experience, where the literal events are imbued with deeper symbolic or thematic significance. W4 and W2 both emphasized the presence of ``\textit{Sophisticated use of idiom, metaphor, and literary allusion} or \textit{Surprising, skilled, and complex use of metaphor/simile/allusion} as a way to measure Fluency in creative writing.
\subsubsection{\textbf{{\color{blue}Narrative Ending (TTCW Fluency4)}: Does the end of the story feel natural and earned, as opposed to
arbitrary or abrupt?}} In her New Yorker essay ``On Bad Endings'' \cite{BadEndings} Accocela writes ``Another possibility is that the author just gets tired. I review a lot of books, many of them non-fiction. Again and again, the last chapters are hasty and dull. `I’ve worked hard enough,' the author seems to be saying. `My advance wasn’t much. I already have an idea for my next book. Get me out of here'.''
If the writer ends the piece simply because they are ``tired of writing'', the conclusion might feel abrupt, disjointed, or unfulfilling to the reader. This is one of the important factors of creative writing fluency. A strong ending offers a sense of closure, ties up the central conflicts or questions of the story, and generally leaves the reader feeling that the narrative journey was worthwhile and complete. For this measure of Fluency, W3 asked  ``\textit{Does the writer know how to end the piece not because they're tired of writing, but because they have come to the moment the entire piece has been leading us towards?}''
\subsubsection{\textbf{{\color{blue}Understandability \& Coherence (TTCW Fluency5)}: Do the different elements of the story work together to form a unified, engaging, and satisfying whole?}} Narrative coherence is the degree to which a story makes sense  \footnote{https://en.wikipedia.org/wiki/Narrative\_paradigm}. A well-crafted story usually follows a logical path, where the events in the beginning set up the middle, which then logically leads to the end. Every scene, character action, and piece of dialogue should serve the story and propel it forward. Well-written stories have an underlying unity that binds the elements together. W3 strongly advocated for this measure by saying ``\textit{Does the piece hold together? In other words, does the beginning lead through the middle to the end in a way that feels deliberate and intentional? This is the difference between several pages of writing and a PIECE of writing. Great writing errs on the side of unity over disorder.}'' while W7 suggested the importance of this measure through ``\textit{a logical flow.''}

\paragraph{\textbf{Flexibility}} Flexibility is often referred to as the ability to look at something from a different angle or point of view. In the context of creative writing, our participants agreed on 3 distinct measures of Flexibility.
\subsubsection{\textbf{{\color{blue}Perspective \& Voice Flexibility (TTCW Flexibility1)}: Does the story provide diverse perspectives, and if there are unlikeable characters, are their perspectives presented convincingly and accurately?}} An \textit{omniscient} narrator is the all-knowing voice in a story that can convincingly and accurately depict a wide range of character viewpoints, including those of characters who may be morally ambiguous, difficult, or otherwise unappealing. As stated in \citet{friedman1955point} an omniscient narrator enhances a sense of reliability or truth within literary works since readers are given deeper insights into many characters. The multiple viewpoints feel more objective because readers have access to multiple interpretations of events and can thus decide how they feel about each character’s perspective. W3 wanted ``\textit{a writer to be able to inhabit various perspectives, even unlikable ones.}'' while W6 suggested ``\textit{Flexibility of voice. Can the author inhabit the consciousness of different characters and not just likable ones?}"
\subsubsection{\textbf{{\color{blue}Emotional Flexibility (TTCW Flexibility2)} : Does the story achieve a good balance between interiority and exteriority, in a way that feels emotionally flexible?}}
Emotional flexibility is asking whether the piece of writing effectively balances action and introspection, and if it portrays a broad and realistic spectrum of emotions as corroborated by W3. \textit{Exteriority} refers to the observable actions, behaviors, or dialogue of a character, and the physical or visible aspects of the setting, plot, and conflicts.\textit{Interiority}, on the other hand, pertains to the inner life of a character — their thoughts, feelings, memories, and subjective experiences. A balance between these two aspects is crucial in creating well-rounded characters and compelling narratives. As stated in \citet{campe2014rethinking} if a story is too heavy on exteriority, it may feel shallow or lack emotional depth. If it leans too much on interiority, it could become overly introspective and potentially lose the momentum of the plot.
\subsubsection{\textbf{{\color{blue} Structural Flexibility(TTCW Flexibility3)}: Does the story contain turns that are both surprising and appropriate?}} A good piece of creative writing often has plot twists, character developments, or thematic revelations that surprise the reader, subverting their expectations in a thrilling way. However, despite the surprises and twists, the turns in the story must also make sense within the established context of the story's universe, its characters, and its themes. It shouldn't feel like the writer has broken the rules they've set up, or made a character behave inconsistently without reason, simply for the sake of shock value. In order for any writing to be structurally flexible W3 wanted to ensure that ``\textit{the writer capable of making turns in the work that are both surprising and appropriate} while W1 required the presence of ``\textit{New twist in the story that are believable} to measure structural flexibility
\paragraph{\textbf{{Originality}}} Creative writing requires originality, or the ability to generate unique ideas \cite{ward1999creative}.
Our participants suggested three unique ways in which they look for originality in creative writing.

\subsubsection{\textbf{{\color{blue}Originality in Theme and Content (TTCW Originality1)}: Will an average reader of this story obtain a unique and original idea from reading it?}} In his book ``Literature and the Brain'' well-known literary critic and scholar Norman Holland discusses how stories stimulate the mind and impact readers \citet{holland2009literature}. A good story that offers a deeper understanding of human nature, cultural insights, unique viewpoints, or even the exploration of new ideas and themes has a lasting impact on its reader and society. In ``Poetic Justice'', prominent philosophers Martha Nussbaum explores how the literary imagination is an essential ingredient of public discourse and a democratic society \citet{nussbaum1997poetic}. As such originality in theme and content is an important measure of creative writing. In the words of W6 originality in theme meant  ``\textit{New brilliant ideas about the future and humanity (mostly in speculative fiction). Does the writing have an original message?} while W3 mentioned ``\textit{Do I feel I am learning something new from the piece? What is the purpose of putting it into the world?}''
\subsubsection{\textbf{{\color{blue} Originality in Thought (TTCW Originality2)}: Is the story an original piece of writing without any cliches?}} A cliche is an idea, expression, character, or plot that has been overused to the point of losing its original meaning or impact \citet{fountain2012cliches}. They often become predictable and uninteresting for the reader. In his book \citet{clark2008writing} eminent American writer, editor, and writing coach: Roy Peter Clark advised writers to strictly avoid cliches because they often indicate a lack of original thought or laziness in language use. Originality suggests that the piece isn't cliche. Several experts agreed on this measure with W3 saying ``\textit{Originality suggests that the piece isn't cliche.} while W4 required ``\textit{Plot that is surprising rather than cliche}''. W6 emphasized originality in thought through ``\textit{writing that is not cliche, not recycled tropes, and does not contain stereotyped one-dimensional characters}
\subsubsection{\textbf{{\color{blue}TTCW Originality in Form \& Structure (Originality3)} : Does the story show originality in its form?}} In his book \citet{boardman1992narrative} Frederic Jameson highlighted the complexities of postmodern literature, where the blurring of genres and innovation in form was a key characteristic \citet{jameson1991postmodernism}. Originality in form has also been accomplished by the unconventional use of format, genre, or narrative structure or arc. For instance, the Pulitzer-winning book \textit{The Color Purple} by Alice Walker is told through a series of letters written by the protagonist. Neil Gaiman's \textit{American Gods} on the other hand combines elements of fantasy, mystery, and mythic fiction in unexpected ways. \textit{The Sound and the Fury} by William Faulkner deviates from the traditional plot structure by presenting a narrative that unfolds through the stream of consciousness of different characters. The goal of originality in form or structure is often to provide a fresh reader experience, challenge conventional reading expectations, or to create a deeper or more complex exploration of the story's themes. This was corroborated by W2 and W5 who wanted to see ``\textit{formal or stylistic novelty}. W3 also wanted to see if \textit{A piece demonstrates original ways to work in sometimes tired forms (such as the short story)}
\paragraph{\textbf{{{Elaboration}}}}
Elaboration in creative writing is the process of adding details and information to a story to make it more interesting and engaging. This can be done by describing the setting, characters, and action in more detail, or by adding dialogue, thoughts, and feelings to the story.

\subsubsection{\textbf{{\color{blue} World Building and Setting (TTCW Elaboration1)}: Does the writer make the fictional world believable at the sensory level?}} American poet and memoirist Mark Doty discuss the importance of creating a vivid, immersive reality at the sensory level through the use of detailed, evocative description \cite{doty2014art}. An effective writer often uses sensory details to paint a detailed picture of the story's environment, making it feel tangible and real to the reader. This level of detail contributes to the believability of the world, even if it is a completely fictional or fantastical setting. It helps the reader to suspend disbelief and become more deeply invested in the narrative. W2 suggested measuring elaboration through ``\textit{Use of setting to inform the action and atmosphere of the story without succumbing to pathetic fallacy}''. W6 further confirmed the same by saying ``\textit{There is a setting. The short story is rooted in a time and a place. If it's science fiction or fantasy there will be some world building.}''. W5 further mentioned ``\textit{the presence of a self-consistent world}'' as a measure of elaboration while W3 asked ``\textit{How does the writer make the fictional world believable at the sensory level?}
\subsubsection{\textbf{{\color{blue}Character Development (TTCW Elaboration2)}: Does each character in the story feel developed at the appropriate complexity level, ensuring that no character feels like they are present simply to satisfy a plot requirement?}} A 'flat character' is typically a minor character who is not thoroughly developed or who does not undergo significant change or growth throughout the story. They often embody or represent a single trait or idea, and they're only used to advance the plot or highlight certain qualities in other characters. A 'complex character' on the other hand also known as a round character, has depth in feelings and passions, has a variety of traits of a real human being, and evolves over time. \citet{forster1927aspects,fishelov1990types,currie1990nature} highlights that any creative piece of fiction or non-fiction tends to be more engaging to the reader when authors can take a character who initially appears to be one-dimensional or stereotypical (flat) and add depth to them, as it mirrors the complexity of real people. Multiple experts emphasized on the importance of character development. For instance, W6 thought ``\textit{The story should have developed 3D characters.} while W2 and W8 asked for ``\textit{Complex character development that avoids stereotype, generalization, trope, etc}''. W3 wanted to see ``\textit{Does the piece make a flat character complex?}
\subsubsection{\textbf{{\color{blue}Rhetorical Complexity (TTCW Elaboration3)}: Does the story operate at multiple ``levels'' of meaning (surface and subtext)?}}In Ernest Hemingway's short story ``Hills Like White Elephants,'' the couple's conversation about seemingly unrelated topics implies a much deeper and more serious discussion about abortion. Their actual dialogue never directly addresses this issue, but it's heavily suggested through what's left unsaid — the subtext. Effective writing often operates on both surface and subtext levels. The surface text keeps the reader engaged with the plot and characters, while the subtext provides depth, complexity, and additional layers of interpretation, contributing to a richer and more rewarding reading experience \citet{kochis2007baxter,phelan1996narrative}. W3 asked for rhetorical complexity by saying ``\textit{Is there a sense of the piece being complex in such a way that it needed to be written as a whole, not simply paraphrased or summed up?} while W4 mentioned that ``\textit{Text should operate at multiple levels of meaning (surface and subtext)}.''
