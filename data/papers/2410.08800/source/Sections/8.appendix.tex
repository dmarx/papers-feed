\section*{Appendix}\label{sec:appendix}

\subsection{Dataset}

%% JL: This is already mentioned in the text.
%% Please emove the table
%\newgeometry{top=1cm,bottom=1.5cm,left=1cm,right=1cm}

%%\begin{landscape}
%%\centering
\begin{table}[htb]
\centering
\resizebox{0.99\textwidth}{!}{%
\begin{tabular}{lllll}
\hline
\multicolumn{1}{l}{\textbf{Monolingual Dataset}}                                 
& \multicolumn{1}{l}{\textbf{Text Extraction}} 
& \multicolumn{1}{l}{\textbf{Language Identification}} 
& \multicolumn{1}{l}{\textbf{Filtering}}                               
& \textbf{Deduplication}                        
\\ \midrule

\multicolumn{1}{l}{C4 \cite{raffel_shazeer_etal2020}}                    
& \multicolumn{1}{l}{CC (WARC)}                
& \multicolumn{1}{l}{langdetect  ($\geq 0.99$)}          
& \multicolumn{1}{l}{Non-language, outliers, LID}                      
& 3-sentence spans                              \\
\multicolumn{1}{l}{The Pile \cite{gao_biderman_etal2020}}                      
& \multicolumn{1}{l}{jusText}                  
& \multicolumn{1}{l}{pycld2, fastText}                 
& \multicolumn{1}{l}{LID}                                              
& MinHashLSH                                    \\
\multicolumn{1}{l}{RefinedWeb \cite{penedo_malartic_etal2023}}           
& \multicolumn{1}{l}{trafilatura}              
& \multicolumn{1}{l}{fastText}                         
& \multicolumn{1}{l}{LID, DW/ LW heuristics}                           
& MinHashLSH, token matching, URL deduplication \\
\multicolumn{1}{l}{Dolma \cite{soldaini_kinney_etal2024}}                   
& \multicolumn{1}{l}{CC (WARC)}                
& \multicolumn{1}{l}{fastText  ($\geq 0.5$)}             
& \multicolumn{1}{l}{LID , DW/ L, Toxic content}                       
& URL matching, raw doc match, Bloom filter     \\
\multicolumn{1}{l}{FineWeb \cite{penedo_kydlicek_etal2024}}                 
& \multicolumn{1}{l}{trafilatura}              
& \multicolumn{1}{l}{fastText}                         
& \multicolumn{1}{l}{DW/ LW heuristics}                                
& MinHash                                       
\\ 
\midrule
\multicolumn{1}{l}{\textbf{Multilingual Dataset}}                                 
& \multicolumn{1}{l}{\textbf{Text Extraction}} 
& \multicolumn{1}{l}{\textbf{Language Identification}} 
& \multicolumn{1}{l}{\textbf{Filtering}}                               
& \textbf{Deduplication}       \\ 
\midrule
\multicolumn{1}{l}{CCNet \cite{wenzek_lachaux_etal2019}}                     
& \multicolumn{1}{l}{WET}                      
& \multicolumn{1}{l}{fastText ($\geq 0.5$)}              
& \multicolumn{1}{l}{Perplexity filtering, LID}                        
& PW SHA-1                                      \\
\multicolumn{1}{l}{mC4 \cite{xue_constant_etal2021}}                                
& \multicolumn{1}{l}{CC (WARC)}                
& \multicolumn{1}{l}{cld3}                             
& \multicolumn{1}{l}{C4, 3 lines of 200+ characters}                   
& 3-sentence spans                                           \\
\multicolumn{1}{l}{OSCAR 22.01 \cite{abadji_suarez_etal2022}}             
& \multicolumn{1}{l}{WET}                      
& \multicolumn{1}{l}{fastText ($\geq 0.8$)}              
& \multicolumn{1}{l}{LW LID,  WD Unicode rules, UT1 blocklist}         
& Line-wise deduplication (English only)        \\
\multicolumn{1}{l}{BigScience ROOTS \cite{laurenccon_saulnier_etal2022}} 
& \multicolumn{1}{l}{Custom extractor}         
& \multicolumn{1}{l}{Manual (by data source)}      
& \multicolumn{1}{l}{Word frequency heuristics}                        
& SimHash, substring deduplication              \\
\multicolumn{1}{l}{Glot500 \cite{imanigooghari_lin_etal2023}}          
& \multicolumn{1}{l}{Custom crawling}          
& \multicolumn{1}{l}{Manual (by data source)}                               
& \multicolumn{1}{l}{Language-script matching, BigScience ROOTS rules} & Sentences                                           \\
\multicolumn{1}{l}{RedPajama-v2 \cite{redpajama2023}}        
& \multicolumn{1}{l}{WET}           
& \multicolumn{1}{l}{fastText}  
& \multicolumn{1}{l}{Classifier, heuristic filtering}                  
& DW Bloom filter                               \\
\multicolumn{1}{l}{MADLAD-400 \cite{kudugunta_caswell_etal2024}}            
& \multicolumn{1}{l}{Unknown}                      
& \multicolumn{1}{l}{Semi-supervised LID}              
& \multicolumn{1}{l}{Prefiltering similar to C4}                       
& LW deduplication                              \\
\multicolumn{1}{l}{HPLT \cite{degilbert_nail_etal2024}}                            
& \multicolumn{1}{l}{warc2text}                
& \multicolumn{1}{l}{CLD2, fastText}                   
& \multicolumn{1}{l}{Two-stage LID, dictionary spell check}            
& DW MinHash                                    \\ \hline
\end{tabular}%
}
\caption{\label{tab:datasets}%
This table summarizes the text extraction, language identification, filtering, and deduplication methods used across various large-scale datasets discussed in the related work section (Section \ref{sec:rel_work}). It provides a comparison of the approaches taken for both monolingual and multilingual datasets, illustrating the diversity of tools and techniques employed in the preprocessing of data for language model training. Abbreviations: DW = Document-wise, LW = Line-wise, PW = Paragraph-wise, LID = Language Identification.}

\end{table}
%%\end{landscape}


\subsection{Metadata Content Example}
\label{ssec:appendix.metadataexample}
% \begin{verbatim}
% {
%     "meta": {
%         "docid": "wikipedia/en/12/56",
%         "url": "https://en.wikipedia.org/wiki/LLM",
%         "title": "LLM",      
%         "download_date": "2024-04-01",
%         "language": "en",
%         "language_score": 0.97
%         },
%     "text": "LLM may refer to:\n
% - Master of Laws (Latin: Legum Magister), a postgraduate degree\n 
% - LLM Communications, a defunct lobbying firm \n
% - LLM Lettering, a typeface\n
% - Large language model, the use of large neural networks for language modeling\n
% - Logic learning machine, a machine learning method\n"
% }
% \end{verbatim}

\begin{verbatim}
    {
    "meta": {
        "url": "https://en.wikipedia.org/wiki/Organic%20Chemistry/Cover", 
        "title": "Organic Chemistry/Cover", 
        "docid": "wikimedia/wikibooks/enwikibooks-20240401", 
        "language": "en", 
        "language_score": 1.0, 
        "download_data": "2024-04-01", 
        }
    "text": "Welcome to the world's foremost open contentOrganic
            Chemistry Textbookon the web!\n\n
            The Study of Organic Chemistry[...]", 
    }
\end{verbatim}


\subsection{Curated Data}
\label{sec:appendix.curated}

\subsubsection{Filtering}
\label{sec:appendix.curated.filtering}
%%JL why did we look at correlation?
The correlation between the percentage filtered and the total number of words in a dataset was relatively low at \( r = 0.26 \) (p = 0.022), suggesting only a weak relationship. Similarly, a slightly stronger correlation was observed between the percentage filtered and the number of documents, at \( r = 0.33 \) (p = 0.003), indicating that datasets with more documents tend to require somewhat more filtering, but the relationship is still far from definitive.

We also examined the impact of format on filtering, but found no significant correlation (\( r = 0.03 \), p = 0.79). However, when considering the language of the dataset, we found a more notable correlation (\( r = 0.37 \), p = 0.001), hinting that certain languages may be more prone to needing extensive filtering, possibly due to differences in data quality or availability across languages.

Finally, domain showed no significant correlation with the percentage of data filtered (\( r = -0.004 \), p = 0.97).

These results suggest that filtering operates somewhat independently from these broader dataset characteristics. Language differences, in particular, seem to warrant more focused attention in future analyses, as they exhibit a stronger connection to filtering needs compared to other factors. 

\subsubsection{List of Curated Datasets}
The complete list of curated dataset is presented in Table \ref{tab:curated_data_list} (next page).


% Please add the following required packages to your document preamble:
% \usepackage[normalem]{ulem}
% \usepackage{lscape}
% \usepackage{longtable}
% Note: It may be necessary to compile the document several times to get a multi-page table to line up properly
\newgeometry{top=1cm,bottom=1.5cm,left=0.5cm,right=0.5cm}

\small
% Increase row height
\renewcommand{\arraystretch}{1.5} 
%\begin{longtable}{|p{3.8cm}p{1.7cm}p{2cm}p{2.7cm}p{2.1cm}>{\raggedleft\arraybackslash}p{1.7cm}>{\raggedleft\arraybackslash}p{1.5cm}>{\centering\arraybackslash}p{1.3cm}|}
\begin{longtable}{p{3.8cm}p{1.7cm}p{2cm}p{2.7cm}p{2.1cm}>{\raggedleft\arraybackslash}p{1.7cm}>{\raggedleft\arraybackslash}p{1.5cm}>{\centering\arraybackslash}p{1.3cm}}
\caption{\label{tab:curated_data_list}%
This table provides an overview of various multilingual datasets utilized in OpenGPT-X project. Each entry begins with the \textbf{Corpus ID} of the dataset and a link to its project page. The \textbf{Language/s} column specifies the languages included within each dataset (see Section \ref{sec:analysis.curated}). The \textbf{Format} of the datasets is also noted, indicating the file type, such as TXT, JSON, or XML. The \textbf{License} column outlines the legal terms governing the use of each dataset, for uncommon licenses a link is provided. The \textbf{Domain} column reflects the specific field or subject area that the dataset pertains to, such as Law, Math, or Medical. The  \textbf{\# Docs} column presents the total number of documents contained in each dataset, while the \textbf{\# Words} column conveys the total word count in thousand. Finally, the \textbf{\% Filtered} (Filt.) column indicates the percentage of documents that have been filtered out during preprocessing. }\\%
\toprule
\textbf{Corpus ID} &
  \textbf{Language/s} &
  \textbf{Format} &
  \textbf{License} &
  \textbf{Domain} &
  \textbf{\#Docs} &
  \textbf{\#Words} &
  \textbf{\% Filt.} \\  \midrule %hline\hline
\endfirsthead
%
\multicolumn{8}{c}%
{{\bfseries Table \thetable\ continued from previous page}} \\ \hline
\textbf{Corpus ID} &
  \textbf{Language/s} &
  \textbf{Format} &
  \textbf{License} &
  \textbf{Domain} &
  \textbf{\#Docs} &
  \textbf{\#Words} &
  \textbf{\% Filt.} \\  \midrule % \hline\hline
\endhead%
\href{https://dcl.bas.bg/BulNC-registration/?lang=EN}{eu\_admin} & EU24 & TXT & CC BY-NC 3.0 & Law \& Admin. & 435.318 & 1.079 & 0\% 
\\
\href{https://github.com/hendrycks/math}{ampsmath\_khan} & EN & JSON / TXT & MIT & Math & 102.985 & 12.081 & 0\% 
\\
\href{https://github.com/hendrycks/math}{ampsmath\_mathematica} & EN & JSON / TXT & MIT & Math & 4.824.189 & 232.502 & 0\% 
\\
\href{https://zenodo.org/records/4643066}{cdrs\_bt} & DE & TXT & CC0 1.0 & Law \& Admin. & 124.487 & 515.053 & 8\% 
\\
\href{https://zenodo.org/records/4006645 }{ce\_bag} & DE & TXT & CC0 1.0 & Law \& Admin. & 5.290 & 18.487 & 0\% 
\\
\href{https://zenodo.org/records/7691841}{ce\_bfh} & DE & TXT & CC0 1.0 & Law \& Admin. & 10.310 & 22.635 & 0\% 
\\
\href{https://zenodo.org/records/7699032}{ce\_bgh} & DE & TXT & CC0 1.0 & Law \& Admin. & 71.976 & 104.002 & 0\% 
\\
\href{https://zenodo.org/records/7767295}{ce\_bpatg} & DE & TXT & CC0 1.0 & Law \& Admin. & 30.772 & 66.029 & 0\% 
\\
\href{https://zenodo.org/records/5910152}{ce\_bverfg} & DE & TXT & CC0 1.0 & Law \& Admin. & 8.449 & 20.679 & 0\% 
\\
\href{https://zenodo.org/records/7749683}{ce\_bverwg} & DE & TXT & CC0 1.0 & Law \& Admin. & 27.099 & 49.664 & 0\% 
\\
\href{https://zenodo.org/records/4542662}{cpp\_bt} & DE & TXT & CC0 1.0 & Law \& Admin. & 4.087 & 235.502 & 0\% 
\\
\href{https://www.gaois.ie/en/corpora/monolingual}{gaois\_corpus} & GA & XML / TMX & CC BY 4.0 & Law \& Admin. & 2 & 2.168 & 0\% 
\\
\href{https://zenodo.org/records/10030647}{cd\_icj} & EN, FR & TXT & CC0 1.0 & Law \& Admin. & 4.565 & 25.979 & 0\% 
\\
\href{https://zenodo.org/records/7051934 }{cd\_pcij} & EN & TXT & CC0 1.0 & Law \& Admin. & 518 & 2.117 & 0\% 
\\
\href{https://gigaword.dk/}{dagw} & DA & TXT & CC BY 4.0 & Web & 285.634 & 922.700 & 5\% 
\\
\href{https://www.deutschestextarchiv.de/}{dta} & DE & TXT & CC BY-SA 4.0 & Books & 5.305 & 197.339 & 0\% 
\\
\href{https://joint-research-centre.ec.europa.eu/language-technology-resources/dcep-digital-corpus-european-parliament_en}{dcep} & EU24 & TXT & \href{https://commission.europa.eu/content/european-union-public-licence_en}{EUPL} & Law \& Admin. & 1.227.996 & 1.117 & 5\% 
\\
\href{https://pub.cl.uzh.ch/wiki/public/pacoco/medi-notice}{medi\_notice} & DE, FR, IT & TSV & CC BY SA % licence checked
& Medical & 22.219 & 52.068 & 0\% 
\\
\href{https://www.sketchengine.eu/estonian-national-corpus/#:~:text=Estonian\%20National\%20Corpus\%202021\%20(Estonian\%20NC\%202021)\%20\%E2\%80\%93\%202.4\%20billion,academic\%20writing\%20(2020\%E2\%80\%932021)}{enc2021} & EU24 & VERT & CC BY-NC 4.0 & Web & 4.361.539 & 1.004 & 5\% 
\\
\href{https://www.cl.ut.ee/korpused/segakorpus/}{estonian\_reference\_corpus} & ET & XML & 
CC BY-NC\textsuperscript{\textdagger} 
& Web & 17.892 & 227.978 & 0\% 
\\
\href{https://opus.nlpl.eu/EMEA/corpus/version/EMEA}{emea} & EU24 & TXT & CC BY 4.0 & Medical & 17.960 & 152.252 & 4\% \\
\href{https://www.statmt.org/europarl/}{europarl} & EU24 & XML & CC0 1.0\textsuperscript{\textdagger} & Law \& Admin. & 63.937 & 660.716 & 0\% 
\\
\href{https://eur-lex.europa.eu/homepage.html?locale=en}{eurlex} & EU24 & JSONL & CC BY 4.0 & Law \& Admin. & 4.463.480 & 13.235 & 8\% 
\\
\href{https://opus.nlpl.eu/ECB/corpus/version/ECB}{ecb\_corpus} & EU24 & XML & CC0 & Finance & 19 & 53.894 & 0\%  % licence checked
\\
\href{https://github.com/nkrusch/fi-news-corpus}{fi\_news} & FI & CSV & MIT & News & 69.413 & 3.650 & 62\%\footnote{Very short documents (headlines)} 
\\
\href{https://openlegaldata.io/research/2019/02/19/court-decision-dataset.html}{german\_legal\_cases} & DE & JSONL & CC0 1.0\textsuperscript{\textdagger} & Law \& Admin. & 249.240 & 749.060 & 0\% 
\\
\href{https://doi.org/10.5281/zenodo.3611246}{german\_political\_speeches} & DE & XML & CC BY-SA 4.0 & Law \& Admin. & 6.659 & 11.366 & 0\% 
\\
\href{https://huggingface.co/datasets/greek_legal_code}{greek\_legal\_code} & EL & JSONL & CC BY 4.0 & Law \& Admin. & 40.929 & 28.791 & 1\% 
\\
\href{https://www.gaois.ie/en/corpora/parallel/data/}{irish\_legislation} & GA & XML / TMX & CC BY 4.0 & Law \& Admin. & 12 & 29.136 & 0\% 
\\
\href{https://kleineanfragen.de/info/daten}{kleine\_anfragen} & DE & SQL & \href{https://opendatacommons.org/licenses/odbl/1-0/}{ODbL 1.0} & Law \& Admin. & 3.401.475 & 6.151 & 18\%\footnote{Noise from PDF conversion.} 
\\
\href{https://huggingface.co/datasets/MLRS/korpus_malti}{korpus\_malti} & MA & JSONL & CC BY-NC-SA  & Web & 103.874 & 276.204 & 11\%\footnote{Noisy data and short document.} 
\\
\href{https://huggingface.co/datasets/joelniklaus/legal-mc4}{legal\_mc4} & EU24 & JSONL & CC BY 4.0 & Law \& Admin. & 4.853.498 & 13.892 & 1\% 
\\
\href{https://marcell-project.eu/}{marcell} & BG, HR, HU, RO, SL, SK, PL & CONLLUP / TXT & CC0 1.0 & Law \& Admin. & 280.629 & 373.283 & 5\% 
\\
\href{https://macocu.eu/}{macocu} & EU24 & XML & CC0 1.0 & Web & 79.164.225 & 27.930 & 12\%\footnote{Short documents.} \\
\href{https://www.euromatrixplus.net/multi-un/}{multi\_un} & EU24 & XML & CC BY-NC\textsuperscript{\textdagger} & Law \& Admin. & 262.391 & 1.096 & 0\% 
%% https://huggingface.co/datasets/Helsinki-NLP/multiun - licence unknown
\\
\href{https://huggingface.co/datasets/coastalcph/multi_eurlex#:~:text=MultiEURLEX\%20comprises\%2065k\%20EU\%20laws,\%2C\%20\%5B1115\%2C\%20fruit\%5D.}{multi\_eurlex} & EU24 & JSONL & CC BY-NC-SA 4.0 & Law \& Admin. & 1.038.305 & 1.184 & 1\% 
\\
\href{https://huggingface.co/datasets/HiTZ/Multilingual-Medical-Corpus}{medical\_t5} & DE, EN, ES, FR & JSON & Apache-2.0 & Medical & 94.269 & 90.741 & 1\% 
\\
\href{https://huggingface.co/datasets/NbAiLab/NCC}{ncc} & NO & JSON & \href{https://huggingface.co/datasets/NbAiLab/NCC#license}{Various}\footnote{The ncc corpus includes the following licenses : NLOD 2.0, CC0 1.0, CC BY-NC 2.0, CC BY-SA 3.0} & Knowledge Base & 1.241.694 & 2.920 & 60\%\footnote{Short documents.} 
\\
\href{http://openlegaldata.io/research/2019/02/19/court-decision-dataset.html}{openlegaldata} & DE & JSONL & CC0 1.0 & Law \& Admin. & 103.870 & 363.962 & 0\% 
\\
\href{http://www.opensubtitles.org/}{opensubtitles2018} & EU24 & XML & CC BY-NC\textsuperscript{\textdagger} & Culture & 2.257.513 & 1.398 & 22\%\footnote{Noisy documents.} 
%% license unclear
\\
\href{https://huggingface.co/datasets/open-web-math/open-web-math}{open\_web\_math} & EN & Parquet & ODC-By 1.0 & Math & 5.956.611 & 7.378 & 2\% 
\\
\href{https://www.corpusitaliano.it/en/}{paisa} & IT & TXT / XML & CC BY-NC-SA 3.0 & Web & 318.730 & 207.382 & 7\% 
\\
\href{https://www.clarin.eu/parlamint}{parlamint} & EU24 & XML / TEI & CC BY & Law \& Admin. & 40.995 & 1.095 & 3\% 
\\
\href{https://huggingface.co/datasets/allenai/peS2o}{pes2o} & EN & JSONL & ODC-By 1.0 & Academic & 38.958.175 & 42.172 & 0\% 
\\
\href{https://pile.eleuther.ai/}{pile\_v2\_freelawopinions} & EN & JSONL & CC BY-ND 4.0 & Law \& Admin. & 4.410.012 & 10.408 & 0\% 
\\
\href{https://github.com/EleutherAI/hn-scraper}{pile\_hackernews} & EN & JSONL & CC BY-NC & Web & 331.205 & 246.149 & 1\% 
\\
\href{https://pile.eleuther.ai/}{pile\_nih\_exporter} & EN & JSONL & CC0 1.0 & Medical & 930.048 & 307.055 & 0\% 
\\
\href{https://pile.eleuther.ai/}{pile\_openwebtext2} & EN & JSONL & MIT\textsuperscript{\textdagger} & Books & 15.752.217 & 10.633 & 1\% 
\\
\href{https://pile.eleuther.ai/}{pile\_v2\_philarchive} & EN & JSONL & Various\footnote{This Pile part is extracted from \href{https://arxiv.org/}{ArXiv} where the author can decide between a \href{https://info.arxiv.org/help/license/index.html}{variety of licenses}.} & Academic & 40.344 & 487.032 & 1\% 
\\
\href{https://pile.eleuther.ai/}{pile\_pmc\_abstracts} & EN & JSONL & Various\footnote{This Pile part is extracted from \href{https://pubmed.ncbi.nlm.nih.gov/}{PubMed} where the author can decide between a \href{https://www.ncbi.nlm.nih.gov/pmc/about/copyright/}{variety of licenses}.} & Medical & 15.476.085 & 3.173 & 0\% 
\\
\href{https://pile.eleuther.ai/}{pile\_pmc\_extracts} & EN & JSONL & Various\footnote{This Pile part is extracted from \href{https://pubmed.ncbi.nlm.nih.gov/}{PubMed} where the author can decide between a \href{https://www.ncbi.nlm.nih.gov/pmc/about/copyright/}{variety of licenses}.} & Medical & 2.808.849 & 12.113 & 1\% 
\\
\href{https://www.sketchengine.eu/polish-parliamentary-corpus/#:~:text=The\%20Polish\%20Parliamentary\%20Corpus\%20(PPC,segments\%20of\%20interpellations\%20and\%20questions.}{ppc} & PL & XML & CC BY 4.0 & Law \& Admin. & 38.304 & 472.193 & 1\% 
\\
\href{https://elrc-share.eu/repository/search/?q=PRINCIPLE&selected_facets=languageNameFilter_exact\%3AIrish}{principle\_ga} & TXT & GA, EN & CC BY 4.0 & Law \& Admin. & 11 & 25.776.098  & 3\% 
\\
\href{https://www.projekt-gutenberg.org/}{projekt\_gutenberg} & EU24 & TXT & \href{https://www.gutenberg.org/policy/license.html}{Custom Licence} & Books & 60.912 & 3.374 & 1\% 
\\
\href{https://huggingface.co/datasets/hoskinson-center/proof-pile}{proof\_pile} & EN & JSONL & Apache-2.0 & Math & 2.128.218 & 4.495 & 0\% 
\\
\href{https://github.com/togethercomputer/RedPajama-Data}{rp\_arxiv} & EN & JSONL & Apache-2.0 & Academic & 1.503.469 & 10.202 & 3\% 
\\
\href{https://opus.nlpl.eu/legacy/SETIMES.php}{setimes\_corpus} & BG, BS, EL, EN, HR, MK, RO, SQ & TXT & CC BY-SA 3.0 & News & 246.206 & 69.279 & 7\% 
\\
\href{https://data.europa.eu/data/datasets/elrc_1188?locale=en}{seimas} & EN, LT & XML / TMX & CC BY 4.0 & Law \& Admin. & 3540 & 47.319 & 0\% 
\\
\href{https://www.juls.savba.sk/justicecorp.html}{sk\_court\_decisions} & SVK & JSONL & CC0 1.0\textsuperscript{\textdagger} & Law \& Admin. & 1.647.736 & 2.174 & 74\%\footnote{Noisy documenta and conversion errors.} 
\\
\href{http://nlp.ffzg.hr/resources/corpora/slwac/}{slwac} & SL & CONLLU & CC BY-SA 4.0 & Web & 1.484.546 & 953.803 & 16\%\footnote{Low language score due to similar languages.} 
\\
\href{https://zenodo.org/records/5495529}{spanish\_legal\_corpora} & ES & TXT & CC BY 4.0 & Law \& Admin. & 15 & 1.383.749 & 0\% 
\\
\href{https://archive.org/details/stackexchange}{rp\_stackexchange} & EN & JSONL & CC BY-SA 4.0 & Forum & 23.909.364 & 7.311 & 19\%\footnote{Short documents.} 
\\
\href{https://huggingface.co/datasets/bigcode/starcoderdata}{starcoder\_data} & XX & JSONL & Various\footnote{The Stack is a collection of source code from repositories with various licenses. Any use of all or part of the code gathered in The Stack must abide by the terms of the original licenses, including attribution clauses when relevant.} & Source Code & 206.634.734 & 73.064 & 0\% 
\\
\href{https://huggingface.co/datasets/rcds/swiss_judgment_prediction }{swiss\_judgment\_prediction} & DE, FR & JSONL & CC BY-SA 4.0 & Law \& Admin. & 262.789 & 144.599 & 0\% 
\\
\href{https://pub.cl.uzh.ch/wiki/public/pacoco/swiss_legislation_corpus}{swiss\_legislation\_corpus} & DE, FR & TXT & CC BY-SA & Law \& Admin. & 2 & 10.407 & 0\% % Licence checked
\\
\href{https://opendata.swiss/en/dataset?keywords_en=environmental--policy&keywords_en=policy-analysis}{swiss\_policy\_documents} & DE, FR, IT & PARQUET & CC BY 4.0 & Law \& Admin. & 417.923 & 561.574 & 0\% 
\\
\href{https://wacky.sslmit.unibo.it/doku.php?id=start}{wacky} & EN, DE, IT & TXT & CC BY-NC-SA 4.0 & Web & 8.571.072 & 6.440 & 0\% 
\\
\href{https://ucsb.box.com/s/ap23l8gafpezf4tq3wapr6u8241zz358}{wikihow} & EN & CSV & CC BY-NC-SA & Knowledge Base & 210.526 & 109.611 & 0\% 
\\
\href{https://en.wikibooks.org/wiki/Main_Page}{wikibooks} & EU24 & XML & CC BY-SA 4.0 & Books & 190.332 & 168.473 & 6\% 
\\
\href{https://en.wikinews.org/wiki/Main_Page}{wikinews} & EU24 & XML & CC BY 2.5 & News & 197.265 & 46.936 & 3\% 
\\
\href{https://en.wikipedia.org/wiki/Main_Page}{wikipedia} & EU24 & XML & CC BY-SA 4.0  & Knowledge Base & 26.175.600 & 11.382 & 6\% 
\\
\href{https://www.wikiquote.org/}{wikiquote} & EU24 & XML & CC BY-SA 4.0 & Recreation & 208.881 & 146.497 & 2\% 
\\
\href{https://en.wikisource.org/wiki/Main_Page}{wikisource} & EU24 & XML & CC BY-SA 4.0 & Books & 428.157 & 798.322 & 7\% 
\\
\href{https://en.wikivoyage.org/wiki/Main_Page}{wikivoyage} & EU24 & XML & CC BY-SA 4.0 & Culture & 71.905 & 66.278 & 4\% 
\\
\href{https://www.wikiversity.org/}{wikiversity} & EU24 & XML & CC BY-SA 4.0 & Books & 61.397 & 65.482 & 0\%
\\ \bottomrule %\hline
\end{longtable}

%
%\begin{landscape}
\small
% Increase row height
%\renewcommand{\arraystretch}{1.5} 
\begin{longtable}{p{2cm}
p{3cm} p{2.2cm} p{2.7cm} p{2.4cm}>{\raggedleft\arraybackslash}p{1.7cm}>{\raggedleft\arraybackslash}p{1.5cm}>{\centering\arraybackslash}p{1.8cm}}

\toprule
\textbf{Corpus ID} & \textbf{Name} & \textbf{Language/s} & \textbf{Format} & \\
  \textbf{License} &
  \textbf{Domain} &
  \textbf{\# Docs} &
  \textbf{\# Words} &
  \textbf{\% Filtered} \\ 
  \midrule
\endfirsthead
%
\multicolumn{8}{c}%
{{\bfseries Table \thetable\ continued from previous page}} 
\\ 
\midrule
\textbf{Corpus ID} & \textbf{Name} & \textbf{Language/s} & \textbf{Format} & \\
  \textbf{License} &
  \textbf{Domain} &
  \textbf{\# Docs} &
  \textbf{\# Words} &
  \textbf{\% Filtered} \\  
  \midrule
\endhead
%
- &
Admin EUR corpus of EU legislation &
  EU24 &
  TXT & \\
  CC BY-NC 3.0 &
  Law and Admin. &
  435.318 &
  1.079 &
  9\% \\ \midrule
- & 
AmpsMath: Khan &
  EN &
  JSON, TXT & \\
  MIT &
  Math &
  102.985 &
  12 &
  0\% \\ \midrule
- & 
AmpsMath: Mathematica &
  EN &
  JSON, TXT & \\
  MIT &
  Math &
  4.824.189 &
  232 &
  0\% \\ \midrule
- & 
Corpus der Drucksachen des Deutschen Bundestages: CDRS-BT &
  DE &
  TXT & \\
  CC0 1.0 &
  Law and Admin. &
  124.487 &
  515 &
  8\% \\ \midrule
- & 
Corpus der Entscheidungen: Bundesarbeitsgerichts (CE-BAG) &
  DE &
  TXT & \\
  CC0 1.0 &
  Law and Admin. &
  5.290 &
  18 &
  0\% \\ \midrule
- & 
Corpus der Entscheidungen: Bundesfinanzhofs (CE-BFH) &
  DE &
  TXT & \\
  CC0 1.0 &
  Law and Admin. &
  10.310 &
  22 &
  0\% \\ \midrule
- & 
Corpus der Entscheidungen: Bundesgerichtshofs (CE-BGH) &
  DE &
  TXT & \\
  CC0 1.0 &
  Law and Admin. &
  71.976 &
  104 &
  0\% \\ \midrule
- & 
Corpus der Entscheidungen: Bundespatentgerichts (CE-BPatG) &
  DE &
  TXT & \\
  CC0 1.0 &
  Law and Admin. &
  30.772 &
  66 &
  0\% \\ \midrule
- & 
Corpus der Entscheidungen: Bundesverfassungsgerichts (CE-BverfG) &
  DE &
  TXT & \\
  CC0 1.0 &
  Law and Admin. &
  8.449 &
  20 &
  0\% \\ \midrule
- & 
Corpus der Entscheidungen: Bundesverwaltungsgerichts (CE-BverwG) &
  DE &
  TXT & \\
  CC0 1.0 &
  Law and Admin. &
  27.099 &
  49 &
  0\% \\ \midrule
- & 
Corpus der Plenarprotokolle des Deutschen Bundestages: CPP-BT &
  DE &
  TXT & \\
  CC0 1.0 &
  Law and Admin. &
  4.087 &
  235 &
  0\% \\ \midrule
- & 
Corpus of Contemporary Irish: Gaois corpus &
  GA &
  XML, TMX & \\
  CC BY 4.0 &
  Law and Admin. &
  2 &
  2 &
  0\% \\ \midrule
- & 
Corpus of Decisions: International Court of Justice (CD-ICJ) &
  EN, FR &
  TXT & \\
  CC0 1.0 &
  Law and Admin. &
  4.565 &
  25 &
  0\% \\ \midrule
- & 
Corpus of Decisions: Permanent Court of International Justice (CD-PCIJ) &
  EN &
  TXT & \\
  CC0 1.0 &
  Law and Admin. &
  518 &
  2 &
  0\% \\ \midrule
- & 
Danish Gigaword: DaGW &
  DA &
  TXT & \\
  CC BY 4.0 &
  Web &
  285.634 &
  922 &
  5\% \\ \midrule
- & 
Deutsches TextArchiv: DTA &
  DE &
  TXT & \\
  CC BY-SA 4.0 &
  Books &
  5.305 &
  197 &
  0\% \\ \midrule
- & 
Digital Corpus of the European Parliament: DCEP &
  EU24 &
  TXT & \\
  EUPL\footnote{European Union Public Licence} &
  Law and Admin. &
  1.227.996 &
  1.117 &
  5\% \\ \midrule
- & 
Drug Information Corpus: Medi-Notice &
  DE, FR, IT &
  TSV & \\
  CC0 1.0\textsuperscript{\textdagger} &
  Medical &
  22.219 &
  52 &
  0\% \\ \midrule
- & 
Estonian National Corpus 2021: ENC 2021 &
  EU24 &
  VERT & \\
  CC BY-NC 4.0 &
  Web &
  4.361.539 &
  1.004 &
  5\% \\ \midrule
- & 
Estonian Reference Corpus &
  ET &
  XML & \\
  CC BY-NC\textsuperscript{\textdagger} &
  Web &
  17.892 &
  227 &
  0\% \\ \midrule
- & 
European Medicines Agency Corpus: EMEA &
  EU24 &
  TXT & \\
  CC BY 4.0 &
  Medical &
  17.960 &
  152 &
  4\% \\ \midrule
- & 
European Parliament: EuroParl &
  EU24 &
  XML & \\
  CC0 1.0\textsuperscript{\textdagger} &
  Law and Admin. &
  63.937 &
  660 &
  0\% \\ \midrule
- & 
European Union Law: EurLex &
  EU24 &
  JSONL & \\
  CC BY 4.0 &
  Law and Admin. &
  4.463.480 &
  13.235 &
  8\% \\ \midrule
- & 
%%JL WHAT IS THIS?
EventCorefBank: EC &
  EU24 &
  XML & \\
  CC0 1.0\textsuperscript{\textdagger} &
  Finance &
  19 &
  53 &
  0\% \\ \midrule
- & 
Finnish Language Text Corpus: FI news &
  FI &
  CSV & \\
  MIT &
  News &
  69.413 &
  3 &
  62\%\footnote{Very short documents (headlines)} \\ \hline
- & 
German Court Decision Dataset &
  DE &
  JSONL & \\
  CC0 1.0 \textsuperscript{\textdagger} &
  Law and Admin. &
  249.240 &
  749 &
  0\% \\ \midrule
- & 
German Polticial Speeches Corpus &
  DE &
  XML & \\
  CC BY-SA 4.0 &
  Law and Admin. &
  6.659 &
  11 &
  0\% \\ \midrule
- & 
Greek legal code &
  EL &
  JSONL & \\
  CC BY 4.0 &
  Law and Admin. &
  40.929 &
  28 &
  1\% \\ \midrule
- & 
Irish legislation &
  GA &
  XML, TMX & \\
  CC BY 4.0 &
  Law and Admin. &
  12 &
  29 &
  0\% \\ \midrule
- & 
Kleine (und große) Anfragen &
  DE &
  SQL & \\
  ODbL\footnote{Open Database License} 1.0 &
  Law and Admin. &
  3.401.475 &
  6.151 &
  18\%\footnote{Noise from PDF conversion.} \\ \midrule
- & 
Korpus Malti &
  MA &
  JSONL & \\
  CC BY-NC-SA &
  Web &
  103.874 &
  27 &
  11\%\footnote{Noisy data and short documents.} \\ \midrule
- & 
Legal Multilingual Colossal Clean Crawled Corpus: Legal MC4 &
  EU24 &
  JSONL & \\
  CC BY 4.0 &
  Law and Admin. &
  4.853.498 &
  13.892 &
  1\% \\ \midrule
- & 
MARCELL CEF: MARCELL &
  BG, HR, HU, RO, SL, SK, PL &
  CONLLUP, TXT & \\
  CC0 1.0 &
  Law and Admin. &
  280.629 &
  373 &
  5\% \\ \midrule
- & 
Massive collection and curation of monolingual and bilingual data: focus on under-resourced languages: MaCoCu &
  EU24 &
  XML & \\
  CC0 1.0 &
  Web &
  79.164.225 &
  27.930 &
  12\%\footnote{Short documents.} \\ \midrule
- & 
Multi UN corpus: MultiUN &
  EU24 &
  XML & \\
  CC BY-NC\textsuperscript{\textdagger} &
  Law and Admin. &
  262.391 &
  1.096 &
  0\% \\ \midrule
- & 
Multilingual European Union Law: MultiEurLex &
  EU24 &
  JSONL & \\
  CC BY-NC-SA 4.0 &
  Law and Admin. &
  1.038.305 &
  1.184 &
  1\% \\ \midrule
- & 
Multilingual Medical Corpora &
  DE, EN, ES, FR &
  JSON & \\
  Apache-2.0 &
  Medical &
  94.269 &
  90 &
  1\% \\ \midrule
- & 
Norwegian Collossal Corpus (NCC) &
  NO &
  JSON & \\
  Various\footnote{Norwegian Licence for Open Government Data (NLOD 2.0), CC0 1.0, CC BY-NC 2.0, CC BY-SA 3.0} &
  Knowledge Base &
  1.241.694 &
  2.920 &
  60\%\footnote{Short documents.} \\ \midrule
- & 
OpenLegalData &
  DE &
  JSONL & \\
  CC0 1.0 &
  Law and Admin. &
  103.870 &
  363 &
  0\% \\ \midrule
- & 
Opensubtitles 2018 &
  EU24 &
  XML & \\
  CC BY-NC\textsuperscript{\textdagger} &
  Culture &
  2.257.513 &
  1.398 &
  22\%\footnote{Noisy documents.} \\ \midrule
- & 
OpenWebMath &
  EN &
  Parquet & \\
  ODC-BY 1.0 &
  Math &
  5.956.611 &
  7.378 &
  2\% \\ \midrule
- & 
Paisa &
  IT &
  TXT, XML & \\
  CC BY-NC-SA 3.0 &
  Web &
  318.730 &
  207 &
  7\% \\ \midrule
- & 
ParlaMint corpus of parliamentary debates &
  EU24 &
  XML, TEI & \\
  CC BY 4.0 &
  Law and Admin. &
  40.995 &
  1.095 &
  3\% \\ \midrule
- & 
peS2o &
  EN &
  JSONL & \\
  ODC-BY 1.0 &
  Academic &
  38.958.175 &
  42.172 &
  0\% \\ \midrule
- & 
Pile: Free law opinions V2 &
  EN &
  JSONL & \\
  CC BY-ND 4.0 &
  Law and Admin. &
  4.410.012 &
  10.408 &
  0\% \\ \midrule
- & 
Pile: HackerNews &
  EN &
  JSONL & \\
  CC BY-NC\textsuperscript{\textdagger} &
  Web &
  331.205 &
  246 &
  1\% \\ \midrule
- & 
Pile: NIH exporter &
  EN &
  JSONL & \\
  CC0 1.0 &
  Medical &
  930.048 &
  307 &
  0\% \\ \midrule
- & 
Pile: Openwebtext2 &
  EN &
  JSONL & \\
  MIT\textsuperscript{\textdagger} &
  Books &
  15.752.217 &
  10.633 &
  1\% \\ \midrule
- & 
Pile: Philarchive V2 &
  EN &
  JSONL & \\
  Various\textsuperscript{\textdagger} &
  Academic &
  40.344 &
  487 &
  1\% \\ \midrule
- & 
Pile: PMC abstracts &
  EN &
  JSONL & \\
  Various\textsuperscript{\textdagger} &
  Medical &
  15.476.085 &
  3.173 &
  0\% \\ \midrule
- & 
Pile: PMC extracts &
  EN &
  JSONL & \\
  Various\textsuperscript{\textdagger} &
  Medical &
  2.808.849 &
  12.113 &
  1\% \\ \midrule
- & 
Polish Parliamentary Corpus: PPC &
  PL &
  XML & \\
  CC BY 4.0 &
  Law and Admin. &
  38.304 &
  472 &
  1\% \\ \midrule
- & 
PRINCIPLE Anonymized English-Irish DCHG parallel translation memory dataset &
  GA, EN &
  TXT & \\
  CC BY 4.0 &
  Law and Admin. &
  11 &
  25 &
  3\% \\ \midrule
- & 
Projekt Gutenberg &
  EU24 &
  TXT & \\
  CC BY-NC\textsuperscript{\textdagger}\footnote{Custom license as described on \href{https://www.projekt-gutenberg.org/info/texte/info.html}{the project website}.} &
  Books &
  60.912 &
  3.374 &
  1\% \\ \midrule
- & 
Proof pile &
  EN &
  JSONL & \\
  Apache-2.0 &
  Math &
  2.128.218 &
  4.495 &
  0\% \\ \midrule
- & 
RedPajama: arXiv &
  EN &
  JSONL & \\
  Apache-2.0 &
  Academic &
  1.503.469 &
  10.202 &
  3\% \\ \midrule
- & 
SE times corpus &
  BG, BS, EL, EN, HR, MK, RO, SQ, TR &
  TXT & \\
  CC BY-SA 3.0 &
  News &
  246.206 &
  69 &
  7\% \\ \midrule
- & 
Seimas transcripts &
  EN, LT &
  XML, TMX & \\
  CC BY 4.0 &
  Law and Admin. &
  3540 &
  47 &
  0\% \\ \midrule
- & 
SK court decisions &
  SVK &
  JSONL & \\
  CC BY-NC\textsuperscript{\textdagger} &
  Law and Admin. &
  1.647.736 &
  2.174 &
  74\%\footnote{Noisy documenta and conversion errors.} \\ \midrule
- & 
slWaC &
  SL &
  CONLLU & \\
  CC BY-SA 4.0 &
  Web &
  1.484.546 &
  953 &
  16\%\footnote{Low language score due to similar languages.} \\ \midrule
- & 
Spanish legal corpra &
  ES &
  TXT & \\
  CC BY 4.0 &
  Law and Admin. &
  15 &
  1.383 &
  0\% \\ \midrule
- & 
StackExchange &
  EN &
  JSONL & \\
  CC BY-SA 4.0 &
  Forum &
  23.909.364 &
  7.311 &
  19\%\footnote{Short documents.} \\ \midrule
- & 
StarCoder data &
  XX &
  JSONL & \\
  Various\textsuperscript{\textdagger} &
  Source Code &
  206.634.734 &
  73.064 &
  0\% \\ \midrule
- & 
Swiss Judgment Prediction &
  DE, FR &
  JSONL & \\
  CC BY-SA 4.0 &
  Law and Admin. &
  262.789 &
  144 &
  0\% \\ \midrule
- & 
Swiss Legislation Corpus (SLC) &
  DE, FR &
  TXT & \\
  CC BY-SA\textsuperscript{\textdagger} &
  Law and Admin. &
  2 &
  10 &
  0\% \\ \midrule
- & 
Swiss policy documents &
  DE, FR, IT &
  PARQUET & \\
  CC BY 4.0 &
  Law and Admin. &
  417.923 &
  561 &
  0\% \\ \midrule
- & 
Wacky (ukWaC, deWaC, itWaC, frWaC) &
  EN, DE, IT &
  TXT & \\
  CC BY-NC-SA 4.0 &
  Web &
  8.571.072 &
  6.440 &
  0\% \\ \midrule
- & 
WikiHow &
  EN &
  CSV & \\
  CC BY-NC-SA &
  Knowledge Base &
  210.526 &
  109 &
  0\% \\ \midrule
- & 
Wikimedia: WikiBooks &
  EU24 &
  XML & \\
  CC BY-SA 4.0 &
  Books &
  190.332 &
  168 &
  6\% \\ \midrule
- & 
Wikimedia: WikiNews &
  EU24 &
  XML & \\
  CC BY 2.5 &
  News &
  197.265 &
  46 &
  3\% \\ \midrule
- & 
Wikimedia: Wikipedia &
  EU24 &
  XML & \\
  CC BY-SA 4.0 &
  Knowledge Base &
  26.175.600 &
  11.382 &
  6\% \\ \midrule
- & 
Wikimedia: WikiQuote &
  EU24 &
  XML & \\
  CC BY-SA 4.0 &
  Recreation &
  208.881 &
  146 &
  2\% \\ \midrule
- & 
Wikimedia: WikiSource &
  EU24 &
  XML & \\
  CC BY-SA 4.0 &
  Books &
  428.157 &
  798 &
  7\% \\ \midrule
- & 
Wikimedia: Wikivoyage &
  EU24 &
  XML & \\
  CC BY-SA 4.0 &
  Culture &
  71.905 &
  66 &
  4\% \\ \midrule
- & 
Wikiversity &
  EU24 &
  XML & \\
  CC BY-SA 4.0 &
  Books &
  61.397 &
  65 &
  3\% \\ 
  \bottomrule
\caption{\label{tab:curated_data_list}%
Overview over multilingual datasets utilized in OpenGPT-X, highlighting their essential characteristics. Each entry begins with the \textbf{Name} of the dataset, which identifies the title. The \textbf{Language/s} column specifies the languages included within each dataset, indicating the multilingual nature of the resources. The \textbf{Format} of the datasets is also noted, indicating the file type, such as TXT, JSON, or XML, which is crucial for understanding how the data can be processed. The \textbf{License} column outlines the legal terms governing the use of each dataset, which is vital for ensuring compliance and appropriate usage in research and application. The \textbf{Domain} column reflects the specific field or subject area that the dataset pertains to, such as Law, Math, or Medical, giving context to the type of data included. The \textbf{\# Docs} column presents the total number of documents contained in each dataset, while the \textbf{\# Words} column conveys the total word count in millions, offering a sense of the dataset size. Finally, the \textbf{\% Filtered} column indicates the percentage of documents that have been filtered out during preprocessing, shedding light on the dataset quality and refinement process. }
\end{longtable}
%\end{landscape}


% \newpage
% \include{tables/global\_langauges}
