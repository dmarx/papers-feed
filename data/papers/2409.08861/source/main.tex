\documentclass[]{fairmeta}

\usepackage{amsmath,amsthm,amssymb,bm}
\usepackage{color}
\usepackage{algorithm}
\usepackage[noend]{algpseudocode}
\usepackage[ruled,algo2e]{algorithm2e}%
\usepackage{derivative}
\usepackage{longtable}
\usepackage{pifont}
\usepackage{mdframed}
\usepackage{enumitem}
\usepackage{mathrsfs}
\usepackage{nicefrac}
\usepackage{subcaption}
\usepackage{float}
\newfloat{algorithm}{t}{lop}

\usepackage[normalem]{ulem}

\usepackage{siunitx}
\usepackage{tocloft}

\let\oldaddcontentsline\addcontentsline
\newcommand{\stoptocentries}{\renewcommand{\addcontentsline}[3]{}}
\newcommand{\starttocentries}{\let\addcontentsline\oldaddcontentsline}

\usepackage{tikz}
\usetikzlibrary{positioning, fit, calc}

\usepackage{booktabs}

\usepackage{arydshln} % Required for dashed lines

% \usepackage{amsmath}
% \usepackage{amssymb}
\usepackage{graphicx}
\usepackage{mathtools}
% \mathtoolsset{showonlyrefs}
\usepackage{amsfonts}
% \usepackage{amsthm,bm}
\usepackage{color}
% \usepackage{enumitem}
\usepackage{dsfont}
% \usepackage{pgfplots}
% \pgfplotsset{width=10cm,compat=1.9}
%  \usepackage{pifont}%http://ctan.org/pkg/pifont
\usepackage{thmtools} 
% \usepackage{thm-restate}
% \usepackage{sidecap}

\declaretheorem[name=Theorem]{thm}
\declaretheorem[name=Proposition]{prop}
\declaretheorem[name=Corollary]{cor}


\newcommand{\cmark}{\ding{51}}%
\newcommand{\xmark}{\ding{55}}%
\newcommand{\R}{\mathbb{R}}
\newcommand{\upmark}{\ding{218}}
\newcommand{\downmark}{\ding{216}}
\newcommand{\norm}[1]{\lVert#1\rVert}
\newcommand{\dotprod}[1]{\langle #1\rangle}
\newcommand{\E}{\mathbb{E}} 
\newcommand{\Et}[1]{\mathbf{E}_t\left[#1\right] } 
\newcommand{\Ev}[1]{\mathbf{E}_v\left[#1\right] } 
\newcommand{\EE}[2]{\mathbf{E}_{#1}\left[#2\right] } 
\newcommand{\Prb}[1]{\mathbf{P}\left[#1\right] }
\newcommand{\Tr}[1]{\mathrm{Tr}( #1)}
\newcommand{\Rea}[1]{\mathrm{Re}[ #1]}
\newcommand{\Ima}[1]{\mathrm{Im}[ #1]}
\newcommand{\eqdef}{\overset{\text{def}}{=}} 
\newcommand{\floor}[1]{\lfloor #1 \rfloor}
\newcommand{\Cov}[1]{\mathrm{Cov}\left[#1\right]}
\newcommand{\Var}[1]{\mathrm{Var}\left[#1\right]}
%\newcommand{\argmin}[1]{\underset{#1}{\text{argmin }}  } 
\newcommand{\breg}[2]{\mathcal{D}_{\Phi}\left(#1,#2\right) }
\newcommand{\xx}{\mathbf{x}}
\newcommand{\yy}{\mathbf{y}}
\newcommand{\zz}{\mathbf{z}}
\newcommand{\hmu}{\hat{\mu}}
\renewcommand{\phi}{\varphi}

%\usepackage{algorithm}
%\usepackage[noend]{algpseudocode}

\newcommand{\carles}[1]{{\color{red}{\bf[Carles:} #1{\bf]}}}
\newcommand{\ricky}[1]{{\color{magenta}{\bf[Ricky:} #1{\bf]}}}
\newcommand{\brian}[1]{{\color{orange}{\bf[Brian:} #1{\bf]}}}

% \newcommand{\carles}[1]{}
% \newcommand{\ricky}[1]{}
% \newcommand{\brian}[1]{}

\setlength{\parskip}{0.5em}
\setlength\parindent{0pt}

\graphicspath {{figures/}}

\newcommand{\N}{\mathbb{N}}
\newcommand{\Ff}{\mathcal{F}}
\newcommand{\Gg}{\mathcal{G}}
\usepackage{hyperref}
\usepackage{caption}
\usepackage[super]{nth}
\delimitershortfall-1sp

\newtheorem{lemma}{Lemma}
\newtheorem{definition}{Definition}
\newtheorem{theorem}{Theorem}
\newtheorem{note}{Note}
\newtheorem{assumption}{Assumption}
\newtheorem{proposition}{Proposition}
\newtheorem{example}{Example}
\newtheorem{remark}{Remark}
\newtheorem{corollary}{Corollary}
\newtheorem{observation}{Observation}
% \newtheorem{algorithm}{Algorithm}
\usepackage{subcaption}
%\newcommand{\corollaryautorefname}{Corollary}

\DeclareMathOperator*{\argmin}{argmin}
\DeclareMathOperator*{\argmax}{argmax}

\DeclarePairedDelimiter\abs{\lvert}{\rvert}

\renewcommand{\sectionautorefname}{Sec.}
\renewcommand{\subsectionautorefname}{Subsec.}
\renewcommand{\appendixautorefname}{App.}
\renewcommand{\theoremautorefname}{Thm.}
\renewcommand{\propositionautorefname}{Prop.}
\renewcommand{\corollaryautorefname}{Cor.}
% \renewcommand(\algorithmautorefname}{Alg.}

\newenvironment{talign*}
 {\let\displaystyle\textstyle\csname align*\endcsname}
 {\endalign}
\newenvironment{talign}
 {\let\displaystyle\textstyle\csname align\endcsname}
 {\endalign}

\makeatletter
\renewcommand{\thealgorithm}{\arabic{algorithm}}
% \@addtoreset{algorithm}{chapter}  % Remove or modify this line if you do not want to reset with chapters
\makeatother

\makeatletter
\DeclareRobustCommand{\cev}[1]{%
  {\mathpalette\do@cev{#1}}%
}
\newcommand{\do@cev}[2]{%
  \vbox{\offinterlineskip
    \sbox\z@{$\m@th#1 x$}%
    \ialign{##\cr
      \hidewidth\reflectbox{$\m@th#1\vec{}\mkern4mu$}\hidewidth\cr
      \noalign{\kern-\ht\z@}
      $\m@th#1#2$\cr
    }%
  }%
}
\makeatother

\definecolor{mygray}{gray}{0.95}
\newcommand{\greybox}[1]{
\vspace{-0.9em}
\begin{center}			% Centering minipage
\vspace{-0.5em}
\colorbox{mygray} {		% Set's the color of minipage
\begin{minipage}{0.987\linewidth} 	% Starts minipage
\centering
\vspace{-0.8em}
{#1}
\end{minipage}}			% End minipage
\end{center}
\vspace{-0.5em}
}

% Shorthands for this project
\newcommand{\fX}{\bm{X}}
\usepackage{xspace}
\newcommand*{\eg}{{\it e.g.}\@\xspace}
\newcommand*{\ie}{{\it i.e.}\@\xspace}
\DeclarePairedDelimiterX{\infdivx}[2]{(}{)}{%
  #1\;\delimsize\|\;#2%
}
\newcommand{\infdiv}{D_\text{KL}\infdivx}
\newcommand*{\matr}[1]{\mathbfit{#1}}
\newcommand*{\tran}{^{\mkern-1.5mu\mathsf{T}}}

\definecolor{mygray}{gray}{0.95}
\newcommand{\graybox}[1]{%
\vspace{-1em} 
\begin{center}			% Centering minipage
\colorbox{mygray} {		% Set's the color of minipage
\begin{minipage}{0.987\linewidth} 	% Starts minipage
\centering
\vspace{-1em}   
{#1}    
\end{minipage}}			% End minipage
\end{center}
}

% \theoremstyle{definition}
% \newtheorem{definition}{Definition}[section]

\title{Adjoint Matching: Fine-tuning Flow and Diffusion Generative Models with Memoryless Stochastic Optimal Control}

\author[1]{Carles Domingo-Enrich}
\author[1]{Michal Drozdzal}
\author[1]{Brian Karrer}
\author[1]{Ricky T. Q. Chen}
\affiliation[1]{FAIR, Meta}

\abstract{
    Dynamical generative models that produce samples through an iterative process, such as Flow Matching and denoising diffusion models, have seen widespread use, but there have not been many theoretically-sound methods for improving these models with reward fine-tuning.
    In this work, we cast reward fine-tuning as stochastic optimal control (SOC). 
    Critically, we prove that a very specific \emph{memoryless} noise schedule must be enforced during fine-tuning, in order to account for the dependency between the noise variable and the generated samples.
    We also propose a new algorithm named \emph{Adjoint Matching} which outperforms existing SOC algorithms, by casting SOC problems as a regression problem. 
    We find that our approach significantly improves over existing methods for reward fine-tuning, achieving better consistency, realism, and generalization to unseen human preference reward models, while retaining sample diversity.
}

\stoptocentries

% \date{\today}
\correspondence{Carles Domingo-Enrich at \email{cd2754@nyu.edu}}

% \metadata[Code]{\url{TBD}}

\begin{document}

\maketitle

\section{Introduction}
\label{sec:intro}

\begin{figure}[b!]
    \vspace{-1.5em}
    \centering
    \begin{subfigure}[t]{0.495\linewidth}
        \centering
        \caption*{Base model (Flow Matching) w/ Guidance}
        \includegraphics[width=0.32\linewidth]{figs/samples/cfg_prompt_49_image_1.jpg}\;%
        \includegraphics[width=0.32\linewidth]{figs/samples/cfg_prompt_49_image_5.jpg}\;%
        \includegraphics[width=0.32\linewidth]{figs/samples/cfg_prompt_49_image_7.jpg}\\
        \includegraphics[width=0.32\linewidth]{figs/samples/cfg_prompt_1_image_0.jpg}\;%
        \includegraphics[width=0.32\linewidth]{figs/samples/cfg_prompt_1_image_1.jpg}\;%
        \includegraphics[width=0.32\linewidth]{figs/samples/cfg_prompt_1_image_8.jpg}
    \end{subfigure}
    \begin{subfigure}[t]{0.495\linewidth}
        \centering
        \caption*{\textbf{Adjoint Matching (Ours)}}
        \includegraphics[width=0.32\linewidth]{figs/samples/adjmat_prompt_49_image_1.jpg}\;%
        \includegraphics[width=0.32\linewidth]{figs/samples/adjmat_prompt_49_image_5.jpg}\;%
        \includegraphics[width=0.32\linewidth]{figs/samples/adjmat_prompt_49_image_7.jpg}\\
        \includegraphics[width=0.32\linewidth]{figs/samples/adjmat_prompt_1_image_0.jpg}\;%
        \includegraphics[width=0.32\linewidth]{figs/samples/adjmat_prompt_1_image_1.jpg}\;%
        \includegraphics[width=0.32\linewidth]{figs/samples/adjmat_prompt_1_image_8.jpg}
    \end{subfigure}\\
    \begin{subfigure}[t]{0.495\linewidth}
        \centering
        \includegraphics[width=0.32\linewidth]{figs/samples/cfg_prompt_72_image_0.jpg}\;%
        \includegraphics[width=0.32\linewidth]{figs/samples/cfg_prompt_72_image_1.jpg}\;%
        \includegraphics[width=0.32\linewidth]{figs/samples/cfg_prompt_72_image_6.jpg}\\
        \includegraphics[width=0.32\linewidth]{figs/samples/cfg_prompt_75_image_0.jpg}\;%
        \includegraphics[width=0.32\linewidth]{figs/samples/cfg_prompt_75_image_3.jpg}\;%
        \includegraphics[width=0.32\linewidth]{figs/samples/cfg_prompt_75_image_9.jpg}
    \end{subfigure}
    \begin{subfigure}[t]{0.495\linewidth}
        \centering
        \includegraphics[width=0.32\linewidth]{figs/samples/adjmat_prompt_72_image_0.jpg}\;%
        \includegraphics[width=0.32\linewidth]{figs/samples/adjmat_prompt_72_image_1.jpg}\;%
        \includegraphics[width=0.32\linewidth]{figs/samples/adjmat_prompt_72_image_6.jpg}\\
        \includegraphics[width=0.32\linewidth]{figs/samples/adjmat_prompt_75_image_0.jpg}\;%
        \includegraphics[width=0.32\linewidth]{figs/samples/adjmat_prompt_75_image_3.jpg}\;%
        \includegraphics[width=0.32\linewidth]{figs/samples/adjmat_prompt_75_image_9.jpg}
    \end{subfigure}\\
    \caption{We introduce Adjoint Matching, a theoretically-driven yet simple algorithm for reward fine-tuning that works for a large family of dynamical generative models, including for the first time, Flow Matching models. 
    Text prompts: ``\textit{Beautiful colorful sunset midst of building in Bangkok Thailand}'', ``\textit{Beautiful grandma and granddaughter are mixing salad and smiling while cooking in kitchen}'', ``\textit{The beautiful young woman in sunglasses is standing at the background of field and hill. She is smiling and looking over shoulder}'', ``\textit{Chess, intellectual games, figure horse, chess board}''. 
    }
    \vspace{-6em}
    \label{fig:fig1}
\end{figure}

Flow Matching \citep{lipman2023flow,albergo2023building,liu2023flow} and denoising diffusion
\citep{song2019generative,ho2020denoising,song2021scorebased,kingma2021ondensity}
models are being used for many generative modeling applications, including text-to-image \citep{rombach2022high,esser2024scaling}, text-to-video \citep{singer2022make}, and text-to-audio \citep{le2024voicebox,vyas2023audiobox}. 
In most cases, the base generative model does not achieve the desired sample quality.
To improve the generated samples, it is common to resort to techniques such as classifier-free guidance \citep{ho2022classifier,zheng2023guided} to get better text-to-sample alignment, or to fine-tune using human preference reward models to improve sample quality and realism \citep{wallace2023diffusion,clark2024directly}.

In the adjacent field of large language models, the behavior of the model is aligned to human preferences through fine-tuning with reinforcement learning from human feedback (RLHF).  
Either explicitly or implicitly, RLHF methods \citep{ziegler2020finetuning,stiennon2020learning,ouyang2022training,bai2022training} assume a reward model $r(x)$ that captures human preferences, with the goal of modifying the base generative model such that it generates the following \emph{tilted distribution}: 
\begin{talign} \label{eq:p_star_info}
p^*(x) \propto p^{\mathrm{base}}(x) \exp(r(x)),
\end{talign}
where $p_{\mathrm{base}}$ is the base generative model's sample distribution. 

Inspired by this,
fine-tuning methods have been developed to improve denoising diffusion models based on human preference data; either using a reward-based approach \citep{fan2023optimizing,black2024training,fan2023dpok,xu2023imagereward,clark2024directly,uehara2024understanding,uehara2024finetuning},
or direct preference optimization \citep{wallace2023diffusion}. 
However, unlike the fine-tuning methods designed for large language models,
most of the existing methods to a large degree ignore $p^\text{base}$ and focus solely on the reward model.
Reward models can range from standard evaluation metrics such as ClipScore \citep{hessel2021clipscore,kirstain2023pickapic} to specialized models that have been trained on human preferences \citep{schuhmann2022laion,xu2023imagereward,wu2023humanpreferencescorev2}. As these are parameterized by neural networks, they fall pray to adversarial examples which lead to the generation of undesirable artifacts \citep{goodfellow2014explaining,mordvintsev2015inceptionism}. This has led some works to consider adding regularization during fine-tuning \citep{fan2024reinforcement,uehara2024finetuning} to incentivize staying close to the base model distribution; however, there does not yet exist a \emph{simple} approach which actually provably generates from the tilted distribution \eqref{eq:p_star_info}. 

The main contributions of our paper are as follows:
\begin{enumerate}[label=(\roman*)]
    \item We present a stochastic optimal control (SOC) formulation for reward fine-tuning of dynamical generative models. Importantly, we prove that the na\"ive approach considered by prior works lead to a \emph{value function bias} problem that biases the fine-tuned model away from the tilted distribution \eqref{eq:p_star_info}. 
    This problem has also been observed by \citet{uehara2024finetuning} but they propose a more complicated solution which involves training a separate generative model for the optimal noise distribution.
    \item Instead, we propose a very simple solution: the \emph{memoryless noise schedule}. This is a unique noise schedule that completely removes the dependency between noise variables and the generated samples, resulting in provable convergence to the tilted distribution. 
    This allows us to fine-tune dynamical generative models in full generality, including being the first to fine-tune noiseless Flow Matching models.
    \item We also propose a new method for solving SOC problems, called \emph{Adjoint Matching}, which combines the scalability of gradient-based methods and the simplicity of a least-squares regression objective. This is orthogonal to the reward fine-tuning application and can be applied to general SOC problems.
    \item We perform extensive comparisons to baseline approaches, and analyze them from multiple perspectives such as realism, consistency, and diversity. We find that our proposed method provides generalization to unseen human preference reward models, better text-to-sample consistency, and retains good diversity.
\end{enumerate}
In the following, sections are broken down as follows: \Cref{sec:prelim_generative_models} summarizes the algorithms used for sampling from pre-trained Flow Matching and diffusion models, while \Cref{sec:common_perspective} provides a common notation that we will use throughout. \Cref{sec:memoryless_SOC,sec:adjoint_matching} form the core of our contributions. \Cref{sec:memoryless_SOC} details the value function bias problem and our proposed solution via the memoryless noise schedule. \Cref{sec:adjoint_matching} details the new Adjoint Matching algorithm for solving SOC problems. 
%Within each section, we first present preliminaries that summarize the existing methods and their relation to our contributions. 

\section{Preliminaries on dynamical generative models}\label{sec:prelim_generative_models}

We are interested in fine-tuning base generative models $p^{\mathrm{base}}(X_1)$ where samples are generated through the simulation of a stochastic process. That is, these models transform noise variables into a sample through an iterative process. In particular, we discuss the specific constructions and sampling processes of Flow Matching~\citep{lipman2023flow,liu2023flow,liu2022rectified,albergo2023building} and Denoising Diffusion Models~\citep{ho2020denoising,song2021scorebased,song2021denoising}.
The goal of this section is to provide background information on these methods, which we will later unify into a single consistent notation in \Cref{sec:common_perspective}.

Given random variables from an initial distribution $\bar{X}_0 \sim p_0 = \mathcal{N}(0, I)$, and $\bar{X}_1$ which are distributed according to some data distribution, we define the reference flow $\bm{\bar{X}} = (\bar{X}_t)_{t\in[0,1]}$ where
\begin{talign} \label{eq:reference_flow}
\bar{X}_t = \beta_t \bar{X}_0 + \alpha_t \bar{X}_1,
\end{talign}
where $(\alpha_t)_{t \in [0,1]}, (\beta_t)_{t \in [0,1]}$ are functions such that $\alpha_0 = \beta_1 = 0$ and $\alpha_1 = \beta_0 = 1$. 
Diffusion models and Flow Matching construct generative Markov processes $X_t$ with initial distribution $X_0 \sim \mathcal{N}(0, I)$ that result in flows $\bm{X} = (X_t)_{t \in [0,1]}$ with the same time marginals as the reference flow $\bm{\bar{X}}$, \ie, the random variables $X_t$ and $\bar{X}_t$ have identical distribution for all times $t \in [0, 1]$. 
This implies $X_1$ has the same distribution as the data distribution, so simulating the Markov process from random noise $X_0$ is a way to generate artificial samples\footnote{In our derivations, we will simply assume the base model has been trained perfectly during the pre-training phase.}.

\subsection{Flow Matching}

In its simplest form, the generative Markov process of a Flow Matching model is an ordinary differential equation (ODE) of the form:
\begin{talign} \label{eq:FM_ode}
    \mathrm{d}X_t = v
    %^\text{base}
    (X_t,t) \, \mathrm{d}t, \qquad X_0 \sim \mathcal{N}(0,I).
\end{talign}
where $v(X_t, t)$ is a parametric velocity that is optimized to match the derivative of the reference flow, \ie,
%\begin{talign}
$v
% ^\text{base}
(X_t, t) = \argmin_{\hat{v}} \mathbb{E} \big\| \hat{v}(\bar{X}_t,t) - \frac{\mathrm{d}}{\mathrm{d}t}\bar{X}_t \big\|^2$ (see \eg \citet{lipman2023flow} for details on pre-training Flow Matching models).
%\end{talign}
It can then be proven that the solution of the generative process \eqref{eq:FM_ode} has the same time marginals as the reference flow \citep{lipman2023flow,liu2022rectified,albergo2023building}, and a commonly used choice is $\alpha_t = t$ and $\beta_t = 1-t$.
One can also consider a family of stochastic differential equations (SDEs) with an arbitrary state-independent diffusion coefficient%
\footnote{We use the common short-hand ``over-dot'' notation to denote the time derivative, \ie, $\dot{x}_t = \frac{\mathrm{d}}{\mathrm{d}t} x_t$.}%
:
\begin{talign} \label{eq:FM_general_diffusion_coeff}
    \mathrm{d}X_t = \left( v
    % ^\text{base}
    (X_t,t) + \frac{\sigma(t)^2}{2\beta_{t}(\frac{\dot{\alpha}_{t}}{\alpha_{t}} \beta_{t} -\dot{\beta}_{t})} \left( v
    % ^\text{base}
    (X_t,t) - \frac{\dot{\alpha}_{t}}{\alpha_{t}} X_t \right) \right) \, \mathrm{d}t + \sigma(t) \, \mathrm{d}B_t, \qquad X_0 \sim \mathcal{N}(0,I),
\end{talign}
where $(B_t)_{t\geq 0}$ is a Brownian motion.
The generative processes in \eqref{eq:FM_ode} and \eqref{eq:FM_general_diffusion_coeff} have the same time marginals. This can be seen by writing down the Fokker-Planck equations for \eqref{eq:FM_ode} and \eqref{eq:FM_general_diffusion_coeff}, and observing that they are the same up to a cancellation of terms~\citep{maoutsa2020interacting}.
The diffusion coefficient $\sigma(t)$ in \eqref{eq:FM_general_diffusion_coeff} is compensated by the second term in the drift which scales proportionally as $\sigma(t)^2$. 

% To simulate the SDE \eqref{eq:FM_general_diffusion_coeff}, a good approach is to consider a $K$-step discretization: for $k\in \{0,\dots,K\}$,
% \begin{talign} \label{eq:FM_discretization}
%     X_{k+1} = X_k + \frac{1}{K} v
%     % ^\text{base}
%     (X_k,\frac{k}{K}) + \frac{K \sigma_k^2}{2\beta_{k}\left(\frac{\alpha_{k+1} - \alpha_k}{\alpha_{k}} \beta_{k} - \beta_{k+1} + \beta_{k}\right)} \left( v
%     % ^\text{base}
%     (X_k,\frac{k}{K}) - \frac{\alpha_{k+1} - \alpha_k}{K\alpha_{k}} X_k \right) +
%     \frac{1}{\sqrt{K}} \sigma_{k} \varepsilon_{k},
%     \qquad \substack{\varepsilon_k \sim \mathcal{N}(0,I), \\ X_0 \sim \mathcal{N}(0,I),}
% \end{talign}
% where we've identified the continuous-time process $(X_t)_{t\in[0,1]}$ with a discrete-time process $(X_k)_{k\in\{0,\dots,K\}}$, where $\alpha_k = \alpha(\nicefrac{k}{K})$, $\beta_k = \beta(\nicefrac{k}{K})$, and $\sigma_k = \sigma(\nicefrac{k}{K})$, applied a first-order approximation.

\subsection{Denoising Diffusion Models}

We next discuss diffusion models, in particular the sampling scheme proposed by Denoising Diffusion Implicit Model (DDIM; \cite{song2021denoising}) which we will later relate to Denoising Diffusion Probabilistic Models (DDPM; \cite{ho2020denoising}) as a particular case of the former. 
For sampling from a diffusion model, the DDIM update rule%
\footnote{We slightly depart from the notation in \citet{song2021denoising} by flipping the direction of time and using $\bar{\alpha}_k$ which corresponds to the $\alpha_k$ in \citet{song2021denoising} while it corresponds to the $\bar{\alpha}_k$ in \citet{ho2020denoising}.}~(\citet{song2021denoising}, Eq. 12)%
, typically stated in discrete time with $k\in \{0,\dots,K\}$, is:
\begin{talign} \label{eq:DDIM_original_main}
    X_{k+1} = \sqrt{\bar{\alpha}_{k+1}}
    \big( \frac{X_{k} - \sqrt{1-\bar{\alpha}_k} \epsilon(X_k,k)}{\sqrt{\bar{\alpha}_k}} \big)
    + \sqrt{1-\bar{\alpha}_{k+1} - \sigma_{k}^2} \epsilon(X_k,k) +
    % \frac{1}{\sqrt{K}} 
    \sigma_{k} \varepsilon_{k},
    \qquad \varepsilon_k \sim \mathcal{N}(0,I), \ X_0 \sim \mathcal{N}(0,I),
\end{talign}
where $\bar{\alpha}_k$ is an increasing sequence such that $\bar{\alpha}_0 = 0$, $\bar{\alpha}_K = 1$, and the sequence $\sigma_k$ is arbitrary. That is, one samples an initial Gaussian random variable $x_0$, and applies the stochastic update \eqref{eq:DDIM_original_main} iteratively $K$ times in order to obtain an artificial sample $X_K$. Updates can be interpreted as progressively denoising the iterate: $x_0$ is completely noisy and $x_K$ is fully denoised. The noise predictor model $\epsilon(x_k,k)$ is trained to predict the noise of $x_k$ (see \eg \citet{ho2020denoising} for details on pre-training denoising diffusion models). 

\section{Flow Matching and diffusion models from a common perspective}\label{sec:common_perspective}

We formulate Flow Matching and diffusion models in a unified framework, which we will later use throughout the paper.
Firstly, to simplify notation, we will be using continuous-time formulations. This will also directly enable fine-tuning methods inspired by the continuous-time paradigm, which we find tends to perform better than discrete-time counterparts in our empirical validations. Secondly, by consolidating notation, we will be able to discuss fine-tuning of dynamical generative models that follow the same time marginals as the reference flow \eqref{eq:reference_flow}, pre-trained with either the Denoising Diffusion or Flow Matching framework, in full generality.

To convert DDIM to a continuous-time stochastic process, we can show that the DDIM update rule \eqref{eq:DDIM_original_main}, up to a first-order approximation, is equivalent to the Euler-Maruyama discretization of the following SDE:
\begin{talign} \label{eq:euler_maruyama_DDIM}
    \mathrm{d}X_t &= \big( \frac{\dot{\bar{\alpha}}_{t}}{2\bar{\alpha}_{t}} X_t - \big( \frac{\dot{\bar{\alpha}}_{t}}{2\bar{\alpha}_{t}} + \frac{\sigma(t)^2}{2} \big) \frac{\epsilon^\text{base}(X_{t},t)}{\sqrt{1-\bar{\alpha}_{t}}} \big) \mathrm{d}t + \sigma(t) \mathrm{d}B_t, \qquad X_{0} \sim \mathcal{N}(0,I).
\end{talign}
See \Cref{subsec:continuous_DDIM} for the full derivation.
To go from \eqref{eq:DDIM_original_main} to \eqref{eq:euler_maruyama_DDIM}, we assumed a uniform discretization of time, \ie $t=\tfrac{k}{K}$. 
This results in identifying the discrete-time process $(X_{k})_{k\in \{0,\dots,K\}}$ with a continuous-time process $(X_{t})_{t\in[0, 1]}$, where $\bar{\alpha}_k := \bar{\alpha}_{t}$, $\sigma_k := \frac{1}{\sqrt{K}} \sigma(t)$, and $\epsilon(X_k, k)$ with $\epsilon^\text{base}(X_k, t)$. 
In relation to the reference flow \eqref{eq:reference_flow}, the generative process in \eqref{eq:euler_maruyama_DDIM} has the same time marginals when $\alpha_t = \sqrt{\bar{\alpha}_t}$ and $\beta_t = \sqrt{1 - \bar{\alpha}_t}$~\citep{ho2020denoising}.

Furthermore, when viewed up to first order approximations, the DDPM sampling scheme~(\citet{ho2020denoising}; Algorithm 2) can be seen as special instance of the DDIM sampling scheme when $\sigma(t) = \sqrt{\nicefrac{\dot{\bar{\alpha}}_t}{\bar{\alpha}_t}}$. This results in the following generative process:
\begin{talign}\label{eq:euler_maruyama_DDPM}
    \mathrm{d}X_t &= \big( \frac{\dot{\bar{\alpha}}_{t}}{2\bar{\alpha}_{t}} X_t - \frac{\dot{\bar{\alpha}}_{t}}{\bar{\alpha}_{t}} \frac{\epsilon^\text{base}(X_{t},t)}{\sqrt{1-\bar{\alpha}_{t}}} \big) \mathrm{d}t + \sqrt{\frac{\dot{\bar{\alpha}}_{t}}{\bar{\alpha}_{t}}} \mathrm{d}B_t, \qquad X_{0} \sim \mathcal{N}(0,I),
\end{talign}
% We defer the derivation of the continuous-time limit of DDPM to \Cref{eq:euler_maruyama_DDIM}.

We can further consolidate notation by converting all quantities to the score function $\mathfrak{s}(x,t)$---defined as the gradient of the log density of the random variable $X_t$---which is possible when $X_0$ is Normal-distributed and under the affine reference flow \eqref{eq:reference_flow}. In particular, 
the velocity $v^\text{base}$ from Flow Matching can be expressed in terms of the score function (see \Cref{subsec:v_score}):
\begin{talign}
    v^\text{base}(x,t) = \frac{\dot{\alpha}_t}{\alpha_t} x + \beta_t(\frac{\dot{\alpha}_t}{\alpha_t} \beta_t - \dot{\beta}_t) \mathfrak{s}(x,t).
\end{talign}
And the noise predictor $\epsilon^\text{base}$ also admits an expression in terms of the score function (see \Cref{subsec:hat_epsilon_score}):
\begin{talign}
    \mathfrak{s}(x,t) = - \frac{\epsilon^\text{base}(x,t)}{\sqrt{1-\bar{\alpha}_t}}.
\end{talign}
Plugging these two equations into \eqref{eq:FM_general_diffusion_coeff} and \eqref{eq:euler_maruyama_DDIM}, respectively, and rewriting them in terms of only the $\alpha_t$ and $\beta_t$ in \eqref{eq:reference_flow}, we can unify both the Flow Matching and continuous-time DDIM generative processes as:
\begin{talign} \label{eq:gen_process_1}
    \mathrm{d}X_t &= b(X_t,t) \, \mathrm{d}t + \sigma(t) \, \mathrm{d}B_t, \qquad X_0 \sim \mathcal{N}(0,I), \\
    \text{where} \ b(x,t) &= \kappa_t x + \big(\frac{\sigma(t)^2}{2} + \eta_t\big) \mathfrak{s}(x,t), \quad \kappa_t = \frac{\dot{\alpha}_t}{\alpha_t}, \quad 
    \eta_t = \beta_t (\frac{\dot{\alpha}_t}{\alpha_t} \beta_t - \dot{\beta}_t)
    \label{eq:gen_process_2}
\end{talign}
where $(\alpha_t, \beta_t)$ are coefficients of the reference flow \eqref{eq:reference_flow}. We have hence expressed the generative process of a base model, whether it is a Flow Matching or a diffusion model, as an SDE of the form \eqref{eq:gen_process_1}-\eqref{eq:gen_process_2}, unified by the choice of reference flow. This expression has been written before for DDIM, e.g. \cite{bartosh2024neural,bartosh2024neural2}.
%Cref{tab:coefficients} summarizes the coefficients for each pre-training framework.

\section{Fine-tuning as ``memoryless'' stochastic optimal control}
\label{sec:memoryless_SOC}

We now discuss the crux of the problem: how to produce a fine-tuned generative model that produces samples $X_1$ which follow the tilted distribution involving a reward model \eqref{eq:p_star_info}. 
An obvious direction is to construct a \emph{fine-tuning objective} involving both the base generative model and the reward model, where the optimal solution results in a fine-tuned generative model for the tilted distribution. 
However, as we will explain, this turns out to be non-trivial, because a na\"ive formulation will introduce bias into the solution.

In \Cref{sec:SOC_formulation}, we discuss the problem formulation of stochastic optimal control, a general framework for optimizing SDEs, and its relation to the maximum entropy reinforcement learning framework commonly used for RLHF fine-tuning. 
Next, in \Cref{sec:value_function_bias_problem}, we discuss the \emph{initial value function bias} problem which plagues existing approaches and so far has seen no simple solution.
Finally, in \Cref{sec:memoryless_schedule}, we propose a novel simple solution that circumvents the bias problem, by enforcing a particular diffusion coefficient, the \emph{memoryless noise schedule}, to be used during fine-tuning. This results in an extremely simple fine-tuning objective that provably converges to a model which generates the tilted distribution \eqref{eq:p_star_info} without any statistical bias.

% However, this turns out to not be trivial. As a na\"ive formulation will add bias into the solution, or simply shift the problem into optimizing the base distribution of the generative model.

\subsection{Preliminaries on the stochastic optimal control problem formulation} \label{sec:SOC_formulation}

Stochastic optimal control (SOC; \cite{bellman1957,fleming2012deterministic,sethi2018optimal}) considers general optimization problems over stochastic differential equations, but we only need to consider a common instantiation, the quadratic cost control-affine problem formulation:
\graybox{
\begin{talign} \label{eq:control_problem_def}
    &\min\limits_{u \in \mathcal{U}} \mathbb{E} \big[ \int_0^1 
    \big(\frac{1}{2} \|u(X^u_t,t)\|^2 + f(X^u_t,t) \big) \, \mathrm{d}t + 
    g(X^u_1) \big], \\
    \begin{split}
    \text{s.t.}~ \mathrm{d}X^u_t =  \left( b(X^u_t,t) + \sigma(t) u(X^u_t,t) \right) \, \mathrm{d}t + 
    \sigma(t) \mathrm{d}B_t, \qquad X^u_0 \sim p_0
    \end{split} 
    \label{eq:controlled_SDE}
\end{talign}
}
where in \eqref{eq:controlled_SDE}, $X_t^u \in \R^d$ is the state of the stochastic process, $u : \R^d \times [0,1] \to \R^d$ is commonly referred to as the control vector field, $b : \R^d \times [0,1] \to \R^d$ is a base drift, and $\sigma : [0,1] \to \R^{d \times d}$ is the diffusion coefficient. These jointly define the \emph{controlled process} $\fX^u \sim p^u$ that we are interested in optimizing; often both $b$ and $\sigma$ are fixed and we only optimize over the control $u$. 

As part of the objective functional \eqref{eq:control_problem_def}, we have an affine control cost $\frac{1}{2} \|u(X^u_t,t)\|^2$, a running state cost $f : \R^d \times [0,1] \to \R$ and a terminal state cost $g : \R^d \to \R$. 
%For the application of RLHF to dynamical generative models, we only need to consider the case of $f=0$; however, we will keep $f$ in the discussion for full generality as our proposed Adjoint Matching algorithm (\Cref{sec:adjoint_matching}) solves any control-affine SOC problem formulation.

The stochastic optimal control (SOC) objective \eqref{eq:control_problem_def} can be decomposed recursively from the final time value. It is common to define the \emph{cost functional} which is the expected future cost starting from state $x$ at time $t$:
\begin{talign} \label{eq:cost_functional}
J(u;x,t) := \mathbb{E}_{\fX \sim p^u} \left[ \int_t^1 
\left(\frac{1}{2} \|u(X_s,s)\|^2  +  f(X_s,s) \right) \, \mathrm{d}s  +  
g(X_1) \;\big|\; X_t = x \right].
\end{talign}
From here, the \emph{value function} is the optimal value of the cost functional%
\footnote{Note that there is a slight difference in terminology between SOC and reinforcement learning, where our cost functional is referred to as the state value function and our value function is the optimal state value function in RL.}
:
\begin{talign}\label{eq:value_fn_defn}
V(x,t) := \min_{u\in \mathcal{U}} J(u;x,t) = J(u^*;x,t),
\end{talign}
where $u^*$ is the \emph{optimal control}, \ie, minimizer of \eqref{eq:control_problem_def}. Furthermore, a classical result is that the value function can be expressed in terms of the \emph{uncontrolled} base process $p^\text{base}$ (\cite{kappen2005path}, see \citealt[Eq.~8,~App.~B]{domingoenrich2023stochastic} for a self-contained proof):
\begin{talign}\label{eq:value_fn_from_uncontrolled}
    V(x, t) = - \log \E_{\fX \sim p^\text{base}} \left[ \exp( - \int_t^1 f(X_s, s) \mathrm{d}s - g(X_1) ) \;\big|\; X_t = x  \right]. 
\end{talign}
A useful expression for the optimal control (which we will make use of in deriving the Adjoint Matching objective in \Cref{sec:adjoint_matching}) is that it is related to the gradient of the value function:
\begin{talign} \label{eq:optimal_control}
u^*(x,t) = - \sigma(t)^{\top} \nabla_x V(x,t) = - \sigma(t)^{\top} \nabla_x J(u^*, x,t).
\end{talign}
%
\paragraph{Relation to MaxEnt RL.} Stochastic optimal control with the control-affine formulation \eqref{eq:control_problem_def} is the continuous-time equivalence of maximum entropy reinforcement learning (MaxEnt RL; \citet{todorov2006linearly,ziebart2008maximum}) with a KL regularization instead of only an entropy regularization. In particular, by the Girsanov theorem (\Cref{cor:girsanov_sdes}), 
% \ricky{cite}, 
the affine control cost is equivalent to a Kullback–Leibler (KL) divergence between the base process $p^\text{base}$, when $u=0$, and the controlled process $p^u$, when conditioned on the same initial state $X_0$ (see \Cref{subsec:proof_eq_cond_kl}):
\begin{talign}\label{eq:cond_kl}
    \infdiv*{p^u(\fX | X_0)}{p^{base}(\fX | X_0)} = \mathbb{E}_{\fX^u \sim p^u} \left[ \int_{0}^1 \frac{1}{2} \|u(X^u_t,t)\|^2 \mathrm{d}t \right],
\end{talign}
resulting in the KL-regularized RL interpretation of \eqref{eq:control_problem_def}:
\begin{talign}\label{eq:kl_regularized_interpretation}
    &\max\limits_{u \in \mathcal{U}}\;
    \E_{X_0 \sim p_0} \left[
    \mathbb{E}_{\fX \sim p^u(\cdot | X_0)} \big[ \int_0^1  -f(X_t^u , t) \mathrm{d} t - g(X_1^u) \big] - \infdiv{p^u(\fX | X_0)}{p^{base}(\fX | X_0)}
    \right],
\end{talign}
where the negative state costs correspond to intermediate and terminal rewards in the RL interpretation. The KL divergence incentivizes the optimal solution to stay close to the distribution of the base process. 
%This KL regularization is a common objective for RLHF of large language models \citep{ouyang2022training} but has seen seldom use in fine-tuning diffusion models, particularly due to a problem that we will discuss next.

\subsection{The initial value function bias problem} \label{sec:value_function_bias_problem}

We next discuss why na\"ively adding a KL regularization does not lead to the tilted distribution \eqref{eq:p_star_info}.
From \eqref{eq:kl_regularized_interpretation}, we can also show that the optimal distribution conditioned on $X_0$ is%
\footnote{Note \eqref{eq:cond_optimal_distribution_SOC} is informal because densities over continuous-time processes are ill-defined; the formal statement is $\frac{\mathrm{d}\mathbb{P}^{*}}{\mathrm{d}\mathbb{P}^{\mathrm{base}}}(\fX | X_0) = \exp ( - \int_0^1 f(X_t,t) \, \mathrm{d}t - g(X_1))$, where $\frac{\mathrm{d}\mathbb{P}^{*}}{\mathrm{d}\mathbb{P}^{\mathrm{base}}}$ denotes the Radon-Nikodym derivative. We treat this formally in the proofs.}
\begin{talign}\label{eq:cond_optimal_distribution_SOC}
    p^{*}(\fX | X_0) \propto 
    p^{\mathrm{base}}(\fX | X_0) \exp \big( - \int_0^1 f(X_t,t) \, \mathrm{d}t - g(X_1) \big).
\end{talign}
This is analogous to the exponentiated reward distribution in MaxEnt RL \citep{rawlik2013stochastic}, but since we generalize the entropy regularization to a KL regularization, $p^\text{base}$ acts as a prior distribution.

% Importantly, 
In order to relate this to the tilted distribution \eqref{eq:p_star_info} that we want to achieve for fine-tuning, 
% we need to marginalize all time values and check the distribution of $p^*(X_1)$. In order to do this, 
first notice that the normalization constant of the right-hand side (RHS) of \eqref{eq:cond_optimal_distribution_SOC} is exactly the value function at $t=0$:
\begin{talign}\label{eq:normalization_constant}
    \E_{\fX \sim p^\text{base}(\fX | X_0)} \left[ \exp \big( - \int_0^1 f(X_t,t) \, \mathrm{d}t - g(X_1) \big) \right] = \exp \left( - V(X_0, 0) \right),
\end{talign}
where the equality is due to \eqref{eq:value_fn_from_uncontrolled}.  
% Therefore, we see that this normalization constant depends on $X_0$. 
Dividing the RHS of \eqref{eq:cond_optimal_distribution_SOC} by \eqref{eq:normalization_constant} and multiplying by $p_0(X_0)$, we obtain the normalized distribution over the full path $\fX$,
\begin{talign} \label{eq:optimal_distribution_SOC} 
    p^{*}(\bm{X}) = 
    p^{\mathrm{base}}(\bm{X}) \exp \big( - \int_0^1 f(X_t,t) \, \mathrm{d}t - g(X_1) + V(X_0,0) \big).
\end{talign}
Setting $f=0$ and $g = -r$, we arrive at an expression for the optimal distribution
\begin{talign} \label{eq:optimal_distribution_SOC_RLHF} 
    p^{*}(X_0, X_1) = 
    p^{\mathrm{base}}(X_0, X_1) \exp \big( r(X_1) + V(X_0,0) \big).
\end{talign}
This unfortunately does not lead to the tilted distribution \eqref{eq:p_star_info} because we have a bias in the optimal distribution that is due to the value function of the initial distribution $V(X_0, 0)$. That is to say, na\"ively adding a KL regularization \eqref{eq:cond_kl} to the fine-tuning objective in the sense of \eqref{eq:kl_regularized_interpretation} leads to a biased distribution \eqref{eq:optimal_distribution_SOC} after fine-tuning and is \textit{not} equivalent to the tilted distribution \eqref{eq:p_star_info}. For instance, when the sampling procedure is noiseless, \ie, $\sigma(t) = 0$, fine-tuning na\"ively will not have any effect because $X_0$ completely determines $X_1$.

This is unlike the situation for large language models \citep{ouyang2022training,rafailov2023direct}, where there is no dynamical process that samples $X_1$ iteratively and hence no dependence on the initial noise variable $X_0$. 
Although this KL regularization is a common objective for RLHF of large language models, it has seen seldom use in fine-tuning diffusion models, likely due to this issue of the initial value function bias.

In the context of diffusion models, KL regularization \eqref{eq:kl_regularized_interpretation} has been explored in prior works \citep{fan2024reinforcement}, but its behavior was not well-understood and they did not relate the fine-tuned model to the tilted distribution \eqref{eq:p_star_info}. Another direction that has been proposed is to learn the initial distribution $p_0$ to cancel out the bias \citep{uehara2024finetuning,tang2024finetuning} but this simply shifts the work into tilting the initial distribution and requires an auxiliary model for parameterizing the optimal initial distribution. 
In contrast, we show in the next section that it is possible to remove the value function bias by simply choosing a very particular noise schedule during the fine-tuning procedure. 
% In Figure \Cref{fig:memorylessness_illustration}, we show that this value function bias can introduce arbitrary bias into the fine-tuned distribution depending on different choices of noise schedules.

\subsection{The memoryless noise schedule for fine-tuning dynamical generative models} \label{sec:memoryless_schedule}

\begin{table}
\centering
\begin{tabular}{lcccc}
    \toprule
     & $\kappa_t$ & $\eta_t$ & Diffusion coefficient $\sigma(t)$ & Memoryless $X_t$ \\
    \midrule
    \addlinespace
    Flow Matching \eqref{eq:FM_ode} & $\frac{\dot{\alpha}_t}{\alpha_t}$ & $\beta_t \big(\frac{\dot{\alpha}_t}{\alpha_t} \beta_t - \dot{\beta}_t\big)$ & General (commonly $0$) & No \\
    \addlinespace
    Memoryless Flow Matching \eqref{eq:FM_general_diffusion_coeff} & $\frac{\dot{\alpha}_t}{\alpha_t}$ & $\beta_t \big(\frac{\dot{\alpha}_t}{\alpha_t} \beta_t - \dot{\beta}_t\big)$ & $\sqrt{2\eta_t}$ & Yes \\
    \addlinespace
    DDIM \eqref{eq:euler_maruyama_DDIM} & 
    $\frac{\dot{\bar{\alpha}}_{t}}{2\bar{\alpha}_{t}}$
    & 
    $\frac{\dot{\bar{\alpha}}_{t}}{2\bar{\alpha}_{t}}$
    & General (commonly $0$) & No \\
    \addlinespace
    DDPM \eqref{eq:euler_maruyama_DDPM} & 
    $\frac{\dot{\bar{\alpha}}_{t}}{2\bar{\alpha}_{t}}$
    & 
    $\frac{\dot{\bar{\alpha}}_{t}}{2\bar{\alpha}_{t}}$
    & $\sqrt{2\eta_t}$ & Yes \\
    \bottomrule
\end{tabular}
\caption{Diffusion coefficient $\sigma(t)$ and the factors $\kappa_t$, $\eta_t$ for the Flow Matching, Memoryless Flow Matching, DDIM, and DDPM generative processes. When the diffusion coefficient is $\sigma(t) = \sqrt{2\eta_t}$, the generative process is memoryless, \ie, samples $X_1$ will be independent of the initial noise $X_0$.}
\label{tab:coefficients}
\end{table} 

In this section, we propose a very simple method of turning \eqref{eq:optimal_distribution_SOC_RLHF} into the tilted distribution \eqref{eq:p_star_info} through the use of a particular \emph{memoryless} noise schedule.
Throughout, we provide an intuitive explanation of why this noise schedule is sufficient for fine-tuning while discussing the full theoretical result where we show that the memoryless noise schedule is actually not only sufficient but also necessary.

Intuitively, the main reason we cannot arrive at the tilted distribution from \eqref{eq:optimal_distribution_SOC_RLHF} is due to the $p^{\text{base}}(X_0, X_1)$ distribution not factoring into $X_0$ and $X_1$. Hence, we define a memoryless generative process as follows:
%
\begin{definition}[Memoryless generative process]
A generative process of the form \eqref{eq:gen_process_1}-\eqref{eq:gen_process_2} is memoryless if $X_0$ and $X_1$ are independent, \ie, $p^\text{base}(X_0, X_1) = p^\text{base}(X_0) p^\text{base}(X_1)$.
\end{definition}
When the base generative process is memoryless, this implies:
\begin{talign}
    p^{*}(X_1) 
    = \int p^{\text{base}}(X_0) p^{\text{base}}(X_1) \exp( r(X_1) + V(X_0, 0)) \mathrm{d} X_0
    \propto p^\text{base}(X_1) \exp(r(X_1)).
\end{talign}
That is, solving the SOC problem \eqref{eq:control_problem_def}-\eqref{eq:controlled_SDE} with a memoryless base model will result in a fine-tuned model that generates samples $p^*(X_1)$ according to the tilted distribution \eqref{eq:p_star_info}. 
This memoryless property is not satisfied generally by the family of generative processes captured by \eqref{eq:control_problem_def}-\eqref{eq:controlled_SDE}.
For instance, the Flow Matching and DDIM generative processes with zero diffusion coefficient (\ie, $\sigma(t) = 0$) are definitely not memoryless due to $X_0$ and $X_1$ being theoretically invertible.
Below, we provide the sufficient and neccessary condition for the noise schedule in order to have a memoryless generative process.
%\textcolor{red}{Needs to be adapted to new prop. statement.}
%
\begin{proposition}[Memoryless noise schedules] \label{prop:memorylessness_noise_schedule}
    Within the family of generative processes \eqref{eq:gen_process_1}-\eqref{eq:gen_process_2}, a generative process is memoryless if and only if the noise schedule is chosen as: 
    \begin{talign} \label{eq:chi_condition}
        \sigma(t)^2 = 2 \eta_t + \chi(t), \text{ where } \chi : [0,1] \to \R \text{ is s.t. } \forall t \in (0,1], \quad \lim_{t' \to 0^{+}} \alpha_{t'} \exp \big( - \int_{t'}^t \frac{\chi(s)}{2 \beta_{s}^2} \, \mathrm{d}s \big) = 0.
    \end{talign}
    where $\eta_t$ is the coefficient defined in \eqref{eq:gen_process_2} (see also \autoref{tab:coefficients}).
    In particular, we refer to $\sigma(t) = \sqrt{2\eta_t}$ as the memoryless noise schedule.
\end{proposition}
%
Due to the endpoint constraints of $(\alpha_t, \beta_t)$ for the reference flow \eqref{eq:reference_flow}, the memoryless noise schedule $\sigma(t)$ is infinite at $t=0$ and approaches zero at $t=1$. This provides a way for the generative process to mix when close to noise $X_0$ while stay steadying when close to the sample $X_1$. Hence, the sample will have no information about $X_0$ due to the enormous amount of mixing with a large diffusion coefficient.
% We note that \Cref{prop:memorylessness_noise_schedule} is stronger than this simple explanation portrays: the memoryless noise schedule is the \textit{only} noise schedule that allows $X_0$ and $X_1$ to be independent. 
% Intuitively, when the diffusion coefficient is too large, the drift $b(X_t, t)$ must compensate and this can actually result in stronger dependency between states at different time values. 
Furthermore, while we have intuitively justified the memoryless noise schedule through its independence property, our theoretical result is actually even stronger: all generative models of the form \eqref{eq:gen_process_1}-\eqref{eq:gen_process_2} \textit{must} be fine-tuned using the memoryless noise schedule.
We formalize this in the following theorem, which we prove in \Cref{subsec:proof_prop_diff_finetuning}:
\begin{theorem}[Fine-tuning recipe for general noise schedule sampling] \label{thm:general_fine-tuning}
Within the family of generative processes \eqref{eq:gen_process_1}-\eqref{eq:gen_process_2}, in order to allow the use of arbitrary noise schedules and still generate samples according to the tilted distribution \eqref{eq:p_star_info}, the fine-tuning problem \eqref{eq:control_problem_def}-\eqref{eq:controlled_SDE} with $f=0$ and $g=-r$ must be done with the memoryless noise schedule $\sigma(t) = \sqrt{2\eta_t}$.
\end{theorem}
\Cref{thm:general_fine-tuning} states that we \textit{need} to use the memoryless noise schedule for fine-tuning with the SOC objective---or equivalently, the KL regularized reward objective \eqref{eq:kl_regularized_interpretation}.
This is the only noise schedule that retains the relationship between the velocity and score function, allowing the conversion to arbitrary noise schedules (\eg, $\sigma(t) = 0$) after fine-tuning.
It is worth noting that when using the memoryless noise schedule for DDIM, this recovers what we derived as the continuous-time limit of the DDPM generative process \eqref{eq:euler_maruyama_DDPM}. However, the DDPM sampler \citep{ho2020denoising} is not commonly used while the DDIM sampler \citep{song2021denoising} and Flow Matching models typically generate samples using $\sigma(t) = 0$, so an explicit conversion to the memoryless noise schedule is necessary for fine-tuning. 
To the best of our knowledge, we are not aware of any existing works that have proposed a time-varying diffusion coefficient with theoretical guarantees.
\Cref{tab:coefficients} summarizes the memoryless schedule for diffusion and Flow Matching models, which we refer to as Memoryless Flow Matching. In \Cref{fig:memorylessness_illustration}, we visualize fine-tuning a 1D model, where we see that constant $\sigma(t)$ leads to biased distributions whereas the memoryless noise schedule perfectly converges to the tilted distribution \eqref{eq:p_star_info}. 

\begin{figure}
    \centering
    \begin{subfigure}[t]{\linewidth}
        \centering
        \includegraphics[width=0.7\linewidth]{figs/finetune1d/legend.png}
    \end{subfigure}\\
    \begin{subfigure}[t]{0.25\linewidth}
        \centering
        \includegraphics[width=\linewidth]{figs/finetune1d/pretrained_model.png}
        \caption{Pre-trained FM $v^\mathrm{base}$}
    \end{subfigure}%
    \begin{subfigure}[t]{0.25\linewidth}
        \centering
        \includegraphics[width=\linewidth]{figs/finetune1d/finetuned_biased_model.png}
        \caption{Fine-tuned FM $v^\mathrm{finetune}$\\ with $\sigma(t) = 0.2$}
    \end{subfigure}%
    \begin{subfigure}[t]{0.25\linewidth}
        \centering
        \includegraphics[width=\linewidth]{figs/finetune1d/finetuned_biased_1.0_model.png}
        \caption{Fine-tuned FM $v^\mathrm{finetune}$\\ with $\sigma(t) = 1.0$}
    \end{subfigure}%
    \begin{subfigure}[t]{0.25\linewidth}
        \centering
        \includegraphics[width=\linewidth]{figs/finetune1d/finetuned_model.png}
        \caption{Fine-tuned FM $v^\mathrm{finetune}$\\ with memoryless $\sigma(t) = \sqrt{2\eta_t}$}
    \end{subfigure}
    \caption{Visualization of \Cref{thm:general_fine-tuning} showing that fine-tuning must be done with the memoryless noise schedule to ensure convergence to the tilted distribution \eqref{eq:p_star_info}. (a) Shows the base Flow Matching model. (b, c) Fine-tuning using a constant $\sigma(t)$ leads to biased distributions. (d) Fine-tuning using the memoryless noise schedule leads to the correct tilted distribution. Note that sample generation can use any noise schedule after fine-tuning, including $\sigma(t) = 0$.}
    \label{fig:memorylessness_illustration}
\end{figure}

For convenience, we plug the memoryless noise schedule into the controlled process for fine-tuning \eqref{eq:controlled_SDE}, and express them in terms of each respective framework.
Let $\epsilon^{\mathrm{base}}$, $v^{\mathrm{base}}$ denote the pre-trained vector fields and $\epsilon^{\mathrm{finetune}}$, $v^{\mathrm{finetune}}$ the fine-tuned vector fields. Then we have the following expressions for the full drift $b(x,t) + \sigma(t) u(x,t)$ and control $u(x,t)$ when $\sigma(t) = \sqrt{2\eta_t}$:

\quad \textit{DDIM / DDPM}:
\greybox{
\begin{talign} \label{eq:conversion_DDPM}
b(x,t) + \sigma(t) u(x,t) = \frac{\dot{\bar{\alpha}}_{t}}{2\bar{\alpha}_{t}} x - \frac{\dot{\bar{\alpha}}_{t}}{\bar{\alpha}_{t}} \frac{\epsilon^{\mathrm{finetune}}(x,t)}{\sqrt{1-\bar{\alpha}_{t}}} 
,\quad\quad
u(x,t) = 
- \sqrt{\frac{\dot{\bar{\alpha}}_t}{\bar{\alpha}_t(1-\bar{\alpha}_t)}} (\epsilon^{\mathrm{finetune}}(x,t) - \epsilon^{\mathrm{base}}(x,t)).
\end{talign}
}
\vspace{0.4em}
\quad \textit{Memoryless Flow Matching}:
\greybox{
\begin{talign} \label{eq:conversion_MFM}
b(x,t) + \sigma(t) u(x,t) = 2v^{\mathrm{finetune}}(x,t) - \frac{\dot{\alpha}_{t}}{\alpha_{t}} x 
,\quad\quad
u(x,t) = 
\sqrt{\frac{2}{\beta_{t}(\frac{\dot{\alpha}_{t}}{\alpha_{t}} \beta_{t} - \dot{\beta}_{t})}} (v^{\mathrm{finetune}}(x,t) - v^{\mathrm{base}}(x,t)).
\end{talign}
}
Thus, to solve the SOC problem \eqref{eq:control_problem_def}-\eqref{eq:controlled_SDE} in practice, we parameterize the control $u$ in terms of $\epsilon^{\mathrm{finetune}}$ or $v^{\mathrm{finetune}}$ and optimize these vector fields instead. After plugging in \eqref{eq:conversion_DDPM}-\eqref{eq:conversion_MFM}, the SOC problem \eqref{eq:control_problem_def}-\eqref{eq:controlled_SDE} can then be solved using any SOC algorithm in order to perform fine-tuning, and we proposed an especially effective algorithm next in \Cref{sec:adjoint_matching}. After fine-tuning, $\epsilon^{\mathrm{finetune}}$ and $v^{\mathrm{finetune}}$ can simply be plugged back into their respective generative processes \eqref{eq:FM_ode}-\eqref{eq:euler_maruyama_DDPM} to sample from the tilted distribution \eqref{eq:p_star_info} using any choice of diffusion coefficient. 

\section{Adjoint Matching for control-affine stochastic optimal control} \label{sec:adjoint_matching}

We discuss existing methods and also propose a new method for optimizing control-affine SOC problems. The new Adjoint Matching method is a combination of the time-tested continuous adjoint method \citep{pontryagin1962mathematical} with recent developments on constructing least-squares objectives for solving SOC problems \citep{domingoenrich2023stochastic}. In this section, we briefly discuss preliminaries on existing methods, their pros and cons, then detail the Adjoint Matching algorithm and its surprising connections to the prior methods. For numerical optimization, we now assume that the control $u$ is a parametric model with parameters $\theta$.

\subsection{Existing methods for stochastic optimal control}
\subsubsection{The adjoint method}\label{sec:adjoint_method}

The most basic method of optimizing the simulation of an SDE is to directly differentiate through the simulation using gradients from the SOC objective function \citep{han2016deep}. The adjoint method simply uses the objective:
\begin{talign} \label{eq:L_RE}
    \mathcal{L}(u ; \fX) := \int_0^1 \big(\frac{1}{2} \|u(X_t,t)\|^2 \! + \! f(X_t,t) \big) \, \mathrm{d}t \! + \! g(X_1), \qquad \fX \sim p^u.
\end{talign}
This is a stochastic estimate of the control objective in \eqref{eq:control_problem_def}, and the goal is to take compute the gradient of $\mathcal{L}(u ; \fX)$ with respect to the parameters $\theta$ of the control $u$.
Due to the continuous-time nature of SDEs, there are two main approaches to implementing this numerically. Firstly, the \emph{Discrete Adjoint} method uses a ``discretize-then-differentiate'' approach, where the numerical solver for simulating the SDE is simply stored in memory then differentiated through, and it has been studied extensively (\eg, \citet{bierkens2014explicit,gomez2014policy,hartmann2012efficient,kappen2012optimal,rawlik2013stochastic,haber2017stable}).
%The gradient of this loss with respect to the parameters $\theta$ of the control $u$ can be computed by keeping the computational graph throughout the computation of the trajectories, and then backpropagating, leveraging the power of automatic differentiation frameworks \ricky{cite}. 
This approach, however, uses an extremely large amount of memory as the full computational graph of the numerical solver must be stored in memory and implementations often must rely on gradient checkpointing \citep{chen2016training} to reduce memory usage.

Secondly, the \emph{Continuous Adjoint} method exploits the continuous-time nature of SDEs and uses an analytical expression for the gradient of the control objective with respect to the intermediate states $X_t$, expressed as an adjoint ODE, and then applies a numerical method to simulate this gradient itself, hence it is referred to as a ``differentiate-then-discretize'' approach \citep{pontryagin1962mathematical,chen2018neural,li2020scalable}. We first define the \emph{adjoint state} as: 
% the gradient of \eqref{eq:L_RE} 
% \brian{nit: $J$ is defined as the cost functional?  Is this second equality obvious / apparent?} 
% with respect to the state $X_t$:
\begin{talign}
\begin{split} \label{eq:adjoint_state_defn}
&a(t ; \fX, u) := \nabla_{X_t} 
\big(\int_t^1 \big(\frac{1}{2} \|u(X_{t'},t')\|^2 \! + \! f(X_{t'},t') \big) \, \mathrm{d}t' \! + \! g(X_1) \big), \\
&\text{where } \fX \text{ solves } \mathrm{d}X_t =  \left( b(X_t,t) + \sigma(t) u(X_t,t) \right) \, \mathrm{d}t + 
    \sigma(t) \mathrm{d}B_t.
% \\ &\text{implying }\;\; \mathbb{E}_{\fX \sim p^u} \left[ a(t ; \fX, u) \;|\; X_t = x \right] = \nabla_{x} J(u; x, t),
\end{split}
\end{talign}
This implies that $\mathbb{E}_{\fX \sim p^u} \left[ a(t ; \fX, u) \;|\; X_t = x \right] = \nabla_{x} J(u; x, t)$,
where $J$ denotes the cost functional defined in \eqref{eq:cost_functional}.
It can then be shown that this adjoint state satisfies 
% \brian{Will we show this in an appendix?}\ricky{TODO: appendix}%
\footnote{Note we use the convention that a Jacobian matrix $J = \nabla_x v(x)$ is defined as $J_{ij} = \frac{\partial v_i(x)}{\partial x_j}$.}%
:
\begin{talign} 
\begin{split} \label{eq:cont_adjoint_1}
    \frac{\mathrm{d}}{\mathrm{d}t} a(t;\fX,u)  &=  - \left[ a(t;\fX,u)\tran{} \left(\nabla_{X_t} (b (X_t,t) + \sigma(t) u(X_t,t))\right) 
    + \nabla_{X_t} \left( f(X_t,t) + \frac{1}{2}\|u(X_t,t)\|^2 \right) \right],
\end{split}
    \\ a(1;\fX,u) &= \nabla g(X_1). \label{eq:cont_adjoint_2}
\end{talign}
The adjoint state is solved backwards in time, starting from the terminal condition \eqref{eq:cont_adjoint_2}. Computation of \eqref{eq:cont_adjoint_1} can be efficiently done as a vector-Jacobian product on automatic differentiation software \citep{paszke2019pytorch}. Once the adjoint state has been solved for $t \in [0, 1]$, then the gradient of $\mathcal{L}(u ; \fX)$ with respect to the parameters $\theta$ can be obtained by integrating over the entire time interval: 
% \brian{Is this obvious / apparent?}
\begin{talign}\label{eq:continuous_adjoint_grads}
    \frac{\mathrm{d} \mathcal{L}}{\mathrm{d} \theta} =  
    \frac{1}{2}\int_0^1 \frac{\partial}{\partial \theta} \norm{u(X_t, t)}^2 \mathrm{d} t
    +\int_0^1 \frac{\partial u(X_t, t)}{\partial \theta}\tran{} \sigma(t)\tran{} a(t; \fX, u) \mathrm{d} t,
\end{talign}
where the first term is the partial derivative of $\mathcal{L}$ w.r.t. $\theta$ and the second term is the partial derivative through the sample trajectory $\fX$. See \Cref{prop:cont_adjoint_method} in \Cref{subsec:derivation_cont_adj_method} for a statement and proof of this result. 
The discrete and continuous adjoint methods converge to the same gradient as the step size of the numerical solvers go to zero. 
Both are scalable to high dimensions and have seen their fair share of usage in optimizing neural ODE/SDEs \citep{chen2018neural,chen2021learning,li2020scalable}. As the adjoint methods are essentially gradient-based optimization algorithms applied on a highly non-convex problem, many have also reported they can be unstable empirically \citep{mohamed2020monte,suh2022differentiable,domingoenrich2023stochastic}.

\subsubsection{Importance-weighted matching objectives for regressing onto the optimal control}
\label{sec:socm}

An alternative is to consider regressing onto the optimal control $u^*$, which is the approach of the cross-entropy method~\citep{rubinstein2013cross,zhang2014applications} and stochastic optimal control matching (SOCM; \citet{domingoenrich2023stochastic}). These methods make use of path integral theory \citep{kappen2005path} to express the optimal control through importance sampling, resulting in an \emph{importance-weighted} least-squares objective function
\begin{talign}\label{eq:socm_objective}
    \mathcal{L}_{\text{SOCM}}(u; \fX) := \int_0^1 \norm{u(X_t, t) - \hat{u}^*(X_t, t)}^2 \mathrm{d} t \times \omega(u, \fX), \qquad \fX \sim p^u,
\end{talign}
where $\omega$ is an importance weighting that approximates sampling from the optimal distribution $p^*$, and $\hat{u}^*$ is a stochastic estimator of the optimal control relying on having sampled from the optimal process. We defer to \citet{domingoenrich2023stochastic} for the exact details. The functional landscape of this objective is convex, which is argued to help yield stable training. However, the need for importance sampling renders this impractical for high dimensional applications: the variance of the importance weighting $\omega$ grows exponentially with dimension of the stochastic process, leading to catastrophic failure. This unfortunately means that such importance-weighted matching objectives are impractical for fine-tuning dynamical generative models; however, a least-squares objective is greatly coveted as it can lead to stable training and simple interpretations.

\subsection{Adjoint Matching} \label{subsec:adjoint_matching}

We make two important observations which lead to our proposed method: (\textit{i}) it is possible to construct a matching objective without any importance weighting, and (\textit{ii}) there are unnecessary terms in the adjoint differential equation \eqref{eq:cont_adjoint_1} that can lead to higher variance at convergence.

Firstly, we notice that we can simply match the gradient of the cost functional under the \textit{current} control. That is, while SOCM carefully constructs an importance-weighted estimator of the \textit{optimal} control $u^* = - \sigma(t)\tran{} \nabla J(u^*; x, t)$ \eqref{eq:optimal_control}, we claim that we can actually just regress onto the target vector field $- \sigma(t)\tran{} \nabla J(u; x, t)$ where $u$ is the current control, and furthermore, this results in a gradient equal in expectation to the continuous adjoint method. 
%
We formalize this in the following proposition, proven in \Cref{subsec:derivation_continuous}: 
\begin{proposition} \label{prop:continuous_adjoint_loss_main}
    Let us define, for now, the basic Adjoint Matching objective as:
    \begin{talign}
    \begin{split} \label{eq:cont_adjoint}
        \mathcal{L}_{\mathrm{Basic-Adj-Match}}(u; \fX) &:= \frac{1}{2} \int_0^{1} \big\| u(X_t,t)
        + \sigma(t)\tran{} a(t;\bm{X},\bar{u}) \big\|^2 \, \mathrm{d}t, \qquad \fX \sim p^{\bar{u}}, \quad \bar{u} = \texttt{stopgrad}(u),
    \end{split}
    \end{talign}
    where $\bar{u} = \texttt{stopgrad}(u)$ means that the gradients of $\bar{u}$ with respect to the parameters $\theta$ of the control $u$ are artificially set to zero. The gradient of $\mathcal{L}_{\mathrm{Basic-Adj-Match}}(u; \fX)$ with respect to $\theta$ is equal to the gradient $\frac{\mathrm{d} \mathcal{L}}{\mathrm{d} \theta}$ in equation \eqref{eq:continuous_adjoint_grads}.
    Importantly, the only critical point of $\E \left[\mathcal{L}_{\mathrm{Basic-Adj-Match}} \right]$ is the optimal control $u^*$.
\end{proposition}

Critical points of $\mathcal{L}$ are controls $u$ such that $\frac{\delta}{\delta u} \mathcal{L}(u) = 0$, where $\frac{\delta}{\delta u} \mathcal{L}$ denotes the first variation of the functional $\mathcal{L}$.
In other words, \Cref{prop:continuous_adjoint_loss_main} states that the only control that satisfies the first-order optimality condition for the basic Adjoint Matching objective is the optimal control, which provides theoretical grounding for gradient-based optimization algorithms. 

An intuitive way to understand the basic Adjoint Matching objective is that it is a \emph{consistency loss}. The Adjoint Matching objective is based off of the observation that the optimal control $u^*(x, t)$ is the unique fixed-point of the relation $u(x,t) = -\sigma(t)\tran{} \nabla_x J(u; x, t)$ (see \Cref{eq:lemma_cost_functional} in \Cref{subsec:derivation_continuous})
%\brian{Citation?  Is this obvious?  Or part of the proof of proposition 2?}, 
and so we are directly optimizing for a control that fits this relation, while using the adjoint state as a stochastic estimator of $\nabla_x J(u; x, t)$ \eqref{eq:adjoint_state_defn}. 
% This uniqueness property has previously been used to justify the adjoint method by showing that the optimal control is the unique critical point of the adjoint method \ricky{cite}; however, to the best of our knowledge, it has not been used to directly construct a consistency loss like the basic Adjoint Matching objective.

The basic Adjoint Matching objective in \Cref{prop:continuous_adjoint_loss_main} does not yet yield a novel algorithm for stochastic optimal control, because it produces the same gradient as the continuous adjoint method. This can be seen by taking the gradient w.r.t. $\theta$ after expanding the square in \eqref{eq:cont_adjoint} and removing terms that do not depend on $\theta$ to arrive exactly at the continuous adjoint method~\eqref{eq:continuous_adjoint_grads}. 
However, it provides the means of deriving a simpler \textit{leaner} objective function.

\paragraph{The ``Lean'' Adjoint.} The minimizer of a least-squares objective is the conditional expectation of the regression target, so for the Adjoint Matching objective, at the optimum we have that
\begin{talign}
    u^*(x, t) = \E_{\fX \sim p^*} \left[ -\sigma(t)\tran{} a(t; \fX, u^*) | X_t = x\right].
\end{talign}
Multiplying both sides by the Jacobian $\nabla_x u^*(x, t)$ and re-arranging, we get the relation
\begin{talign}\label{eq:cancellation_terms}
    \E_{\fX \sim p^*} \left[ u^*(x, t)\tran{} \nabla_x u^*(x, t) + a(t; \fX, u^*) \tran{} \sigma(t) \nabla_x u^*(x, t) \;|\; X_t = x\right] = 0.
\end{talign}
Notice that the terms inside the expectation in \eqref{eq:cancellation_terms} show up as part of the adjoint differential equation \eqref{eq:cont_adjoint_1}, which we have now shown to have expectation zero at the optimal solution. 
% Furthermore, the variance of the terms inside \eqref{eq:cancellation_terms} is non-zero even at the optimal solution, so the basic Adjoint Matching \eqref{eq:cont_adjoint} and hence \emph{the continuous adjoint method will also have non-vanishing gradients even when $u=u^*$}.
Therefore, we motivate the definition of a \emph{lean adjoint state} $\tilde{a}$ with the terms in \eqref{eq:cancellation_terms} removed. Plugging this lean adjoint back into the least-squares objective, we obtain our final proposed Adjoint Matching objective: 
\graybox{
\begin{talign}\label{eq:lean_adjoint_matching}
\mathcal{L}_{\mathrm{Adj-Match}}(u; \fX) 
:= \frac{1}{2} \int_0^{1} \big\| & u(X_t %^{\bar{u}}
,t)
+ \sigma(t)\tran{} \tilde{a}(t;\bm{X} %^{\bar{u}}
) \big\|^2 \, \mathrm{d}t, 
\qquad \fX \sim p^{\bar{u}}, \quad \bar{u} = \texttt{stopgrad}(u), \\
%
\label{eq:lean_adjoint_1}
\text{where }\quad \frac{\mathrm{d}}{\mathrm{d}t} \tilde{a}(t;\bm{X}) 
&= - (\tilde{a}(t;\bm{X})^{\top} \nabla_x b (X_t,t) + \nabla_x f(X_t,t)), \\ 
%
\label{eq:lean_adjoint_2}
\tilde{a}(1;\bm{X}) &= \nabla_x g(X_1).
\end{talign}
}
Equations \eqref{eq:lean_adjoint_1}-\eqref{eq:lean_adjoint_2} define the \emph{lean adjoint state}, and \eqref{eq:lean_adjoint_matching} is the complete Adjoint Matching objective.
\textit{The unique critical point of $\mathbb{E}[\mathcal{L}_{\mathrm{Adj-Match}}]$ is the optimal control}, which we prove relying on \Cref{prop:continuous_adjoint_loss_main} and equation \eqref{eq:cancellation_terms} (see \Cref{prop:lean_adjoint} in \Cref{subsec:proof_lean_adjoint}). 

Compared to the importance sampling methods (\Cref{sec:socm}), Adjoint Matching is a simple least-squares regression objective and has no importance weighting. This allows it to avoid the pitfalls of high variance importance weights and makes it as scalable as the adjoint methods while retaining the interpretation of matching a target vector field.

Compared to the adjoint method (\Cref{sec:adjoint_method}), Adjoint Matching produces a \emph{different gradient in expectation than the continuous adjoint}. This is because the lean adjoint state is not related to the gradient of the cost functional anymore, \ie, \eqref{eq:adjoint_state_defn} is not true, except at the optimum when $u=u^*$.
Even at the optimal solution, since Adjoint Matching removes terms that have expectation zero, it can potentially exhibit better convergence and lower variance than the continuous adjoint method. 
Additionally, computation of the lean adjoint state \eqref{eq:lean_adjoint_1} also exhibits a smaller computational cost due to the removal of the extra terms (no longer need the Jacobian of the control $\nabla_x u$).
We provide a rigorous derivation of Adjoint Matching and the above claims in \Cref{subsec:proof_lean_adjoint}.

Adjoint Matching can be applied to reward fine-tuning of dynamical generative models through the memoryless SOC formulation discussed in \Cref{sec:memoryless_SOC}. We provide pseudo-code for this in \Cref{alg:adjoint_matching_finetuning_FM} for Flow Matching models and in \Cref{alg:adjoint_matching_finetuning_DDIM} in \Cref{subsec:pseudocode_DDIM} for denoising diffusion models.

% \begin{algorithm}
% \SetAlgoNoLine % Disable line numbering
% \SetAlgoNlRelativeSize{0} %Set number line font to zero
% \small{
% \KwIn{Pre-trained FM velocity field $v^{\mathrm{base}}$ or noise predictor $\epsilon^{\mathrm{base}}$, number of fine-tuning iterations $N$.}

% Initialize fine-tuned vector fields: $v^{\mathrm{finetune}} = v^{\mathrm{base}}$ or $\epsilon^{\mathrm{finetune}} = \epsilon^{\mathrm{base}}$ with parameters $\theta$.

%   \For{$n \in \{0,\dots,N-1\}$}{
%     Sample $m$ trajectories $\bm{X} = (X_k)_{k\in\{0,\dots, K\}}$ with memoryless noise schedule (\Cref{prop:memorylessness_noise_schedule}), \eg using:
%     \begin{itemize}
%     \item the FM update \eqref{eq:FM_discretizatio} with $v = v^{\mathrm{finetune}}$ and $\sigma_k = \sqrt{2\beta_{k}(\frac{\alpha_{k+1} - \alpha_k}{\alpha_{k}} \beta_{k} - \beta_{k+1} + \beta_{k})}$, 
%     \item or the DDIM update \eqref{eq:DDIM_original_main} with $\epsilon = \epsilon^{\mathrm{finetune}}$ and $\sigma_k = \sqrt{\frac{\bar{\alpha}_{k+1} - \bar{\alpha}_{k}}{\bar{\alpha}_{k}}}$.
%     \end{itemize}

%     For each trajectory, solve the \textit{lean adjoint ODE} \eqref{eq:lean_adjoint_1}-\eqref{eq:lean_adjoint_2} backwards in time from $k={K}$ to $1$:
%     \begin{talign}
%         \tilde{a}_{k-1} 
%         &= \tilde{a}_{k} + \frac{1}{K} \tilde{a}_k\tran{} \nabla_{X_k} b_k (X_k), \qquad 
%         \tilde{a}_K = \nabla_{X_K} r(X_K), \qquad \text{where} \\
%         b_k(x) &=
%         \begin{cases}
%             2 v^\text{base}(x,\frac{k}{K}) - \frac{\alpha_{k+1} - \alpha_k}{K\alpha_{k}} x & \text{for FM} \\
%             \frac{\bar{\alpha}_{k+1} - \bar{\alpha}_{k}}{2\bar{\alpha}_{k}} x - \frac{\bar{\alpha}_{k+1} - \bar{\alpha}_k}{\bar{\alpha}_{k} \sqrt{1-\bar{\alpha}_{k}}}\epsilon^{\mathrm{base}}(x,k) & \text{for DDIM}.
%         \end{cases}
%     \end{talign}

%     Note that $X_k$ and $\tilde{a}_k$ should be computed without gradients, \ie, $X_k = \texttt{stopgrad}(X_k)$, $\tilde{a}_k = \texttt{stopgrad}(\tilde{a}_k)$. \vspace{0.5em}

%     For each trajectory, compute the Adjoint Matching objective \eqref{eq:lean_adjoint_matching}: 
%     \begin{talign}
%     \mathcal{L}_{\mathrm{Adj-Match}}(\theta) = 
%     \begin{cases}
%         \frac{1}{K} \sum_{k=0}^{K-1} \big\|\frac{2}{\sigma_k} \big(v^{\mathrm{finetune}}_{\theta}(X_k,\frac{k}{K}) - v^{\mathrm{base}}(X_k,\frac{k}{K}) \big) + \sigma_k \tilde{a}_{k} \big\|^2 & \text{for FM} \\
%         \frac{1}{K} \sum_{k=0}^{K-1} \| - \frac{\bar{\alpha}_{k+1} - \bar{\alpha}_k}{\sigma_k \bar{\alpha}_{k} \sqrt{1-\bar{\alpha}_{k}}} \big( \epsilon^{\mathrm{finetune}}_{\theta}(X_k,k) - \epsilon^{\mathrm{base}}(X_k,k) \big) + \sigma_k \tilde{a}_{k} \|^2 & \text{for DDIM}
%     \end{cases}
%     \end{talign}

%     Compute the gradient $\nabla_{\theta} \mathcal{L}(\theta)$ and update $\theta$ using favorite gradient descent algorithm.
%   }
% \KwOut{Fine-tuned vector field $v^{\mathrm{finetune}}$ or $\epsilon^{\mathrm{finetune}}$}}
% \caption{Adjoint Matching for Fine-tuning Flow Matching and Diffusion Models}
% \label{alg:adjoint_matching_finetuning}
% \end{algorithm}

\begin{algorithm}
\SetAlgoNoLine % Disable line numbering
\SetAlgoNlRelativeSize{0} %Set number line font to zero
\small{
\KwIn{Pre-trained FM velocity field $v^{\mathrm{base}}$, step size $h$, number of fine-tuning iterations $N$.}

Initialize fine-tuned vector fields: $v^{\mathrm{finetune}} = v^{\mathrm{base}}$ with parameters $\theta$.

  \For{$n \in \{0,\dots,N-1\}$}{
    Sample $m$ trajectories $\bm{X} = (X_t)_{t\in\{0, \dots, 1\}}$ with memoryless noise schedule $\sigma(t) = \sqrt{2 \beta_t (\frac{\dot{\alpha}_t}{\alpha_t} \beta_t - \dot{\beta}_t)}$, \eg :
    \begin{talign} \label{eq:EM_update_box}
    X_{t+h} = X_{t} + h \left(2v_\theta^{\mathrm{finetune}}(X_t, t) - \frac{\dot{\alpha}_t}{\alpha_t} X_t \right) + \sqrt{h} \sigma(t) \varepsilon_t, \quad\quad \varepsilon_t \sim \mathcal{N}(0, I), \quad\quad X_0 \sim \mathcal{N}(0, I).
    \end{talign}

    For each trajectory, solve the \textit{lean adjoint ODE} \eqref{eq:lean_adjoint_1}-\eqref{eq:lean_adjoint_2} backwards in time from $t={1}$ to $0$, \eg:
    \begin{talign} \label{eq:Euler_lean_adjoint}
        \tilde{a}_{t-h} 
        = \tilde{a}_{t} + h \tilde{a}_t\tran{} \nabla_{X_t} \left(2v^{\mathrm{base}}(X_t, t) - \frac{\dot{\alpha}_t}{\alpha_t} X_t \right), \qquad 
        \tilde{a}_1 = - \nabla_{X_1} r(X_1).
    \end{talign}

    Note that $X_t$ and $\tilde{a}_t$ should be computed without gradients, \ie, $X_t = \texttt{stopgrad}(X_t)$, $\tilde{a}_t = \texttt{stopgrad}(\tilde{a}_t)$. \vspace{0.5em}

    For each trajectory, compute the Adjoint Matching objective \eqref{eq:lean_adjoint_matching}: 
    \begin{talign} \label{eq:adj_matching_algorithm_box}
    \mathcal{L}_{\mathrm{Adj-Match}}(\theta) =
        \sum_{t\in\{0, \dots, 1 - h\}} \big\|\frac{2}{\sigma(t)} \big(v^{\mathrm{finetune}}_{\theta}(X_t, t) - v^{\mathrm{base}}(X_t, t) \big) + \sigma(t) \tilde{a}_t \big\|^2.
    \end{talign}

    Compute the gradient $\nabla_{\theta} \mathcal{L}(\theta)$ and update $\theta$ using favorite gradient descent algorithm.
  }
\KwOut{Fine-tuned vector field $v^{\mathrm{finetune}}$}}
\caption{Adjoint Matching for fine-tuning Flow Matching models}
\label{alg:adjoint_matching_finetuning_FM}
\end{algorithm}

\section{Related work} 

\paragraph{Fine-tuning from human feedback.}
There are two main overarching approaches to RLHF: the \textit{reward-based} approach \citep{ziegler2020finetuning,stiennon2020learning,ouyang2022training,bai2022training} and \textit{direct preference optimization} (DPO; \cite{rafailov2023direct}).
The reward-based approach \citep{ziegler2020finetuning,stiennon2020learning,ouyang2022training,bai2022training} consists in learning the reward model $r(x)$ from human preference data, and then solving a maximum entropy RL problem with rewards produced by $r(x)$. 
DPO merges the two previous steps into one: there is no need to learn $r(x)$ as human preference data is directly used to fine-tune the model. 
However, DPO is typically only applied with a filtered dataset, and does not work explicitly with a reward model.
Furthermore, for flow and diffusion models specifically, it is possible to differentiate the reward function, so there is a larger emphasis on reward-based approaches.

\paragraph{Fine-tuning for diffusion models.}
Among existing reward-based diffusion fine-tuning methods, \citet{fan2023optimizing} interpret the denoising process as a multi-step decision-making task and use policy gradient algorithms to fine-tune diffusion samplers. 
%They restrict their attention to rewards that are integral probability metrics between a target distribution and the generated distribution. 
\citet{black2024training} makes use of proximal policy gradients for fine-tuning but this does not make use of the differentiability of the reward model. 
\citet{fan2023dpok} also consider KL-regularized rewards \eqref{eq:kl_regularized_interpretation} but do not make the critical connection to the tilted distribution \eqref{eq:p_star_info} that we flesh out in \Cref{sec:value_function_bias_problem}. 
The fine-tuning algorithms of \cite{xu2023imagereward,clark2024directly} directly take gradients of the reward model and use heuristics to try to stay close to the original base generative model, but their behavior is not well understood and unrelated to the tilted distribution: \cite{xu2023imagereward} takes gradients of the reward applied on the denoised sample at different points in time, and \cite{clark2024directly} backpropagates the reward function through all or part of the diffusion trajectory. 
Finally, \cite{uehara2024finetuning} also fine-tune diffusion models with the goal of sampling from the tilted distribution \eqref{eq:p_star_info}, but their approach is much more involved than ours as it requires learning a value function, and solving two stochastic optimal control problems. Additional reward fine-tuning works include \cite{bruna2024posterior}, that provide theoretical guarantees to sample from the tilted distribution when the reward is a quadratic function, and \cite{zhang2024improving}, that propose a reward fine-tuning algorithm for the GFlowNet architecture. 

\paragraph{Inference-time optimization methods.} Some have proposed methods that do not update the base model but instead modify the generation process directly. One approach is to add a guidance term to the velocity \citep{chung2022diffusion,song2023pseudoinverse,pokle2023training}; however, this is a heuristic and it is not well-understood what particular distribution is being generated. Another approach is to directly optimize the initial noise distribution \citep{li2021differentiable,wallace2023endtoend,benhamu2024dflow}; this is taking an opposite approach to the inital value bias problem than us by moving all of the work into optimizing the initial distribution. 
A more computationally intensive approach is to perform online estimation of the optimal control, for the purpose of heuristically solving an optimal control problem within the sampling process~\citep{huang2024symbolic,rout2024rb}; these approaches aim to solve a separate control problem for each generated sample, instead of performing amortization \citep{amos2023tutorial} to learn a fine-tuned generative model.

\paragraph{Optimal control in generative modeling.} Methods from optimal control have been used to train dynamical generative models parameterized by ODEs \citep{chen2018neural}, SDEs \citep{li2020scalable}, and jump processes \citep{chen2021learning}, enabled through the adjoint method. 
They can be used to train arbitrary generative processes, but for simplified constructions these have fallen in favor of simulation-free matching objectives such as denoising score matching \citep{vincent2011connection} and Flow Matching \citep{lipman2023flow}. 
The optimal control formalism also has significance in sampling from un-normalized distributions \citep{zhang2022path,berner2023optimal,vargas2023denoising,vargas2022bayesian,richter2024improved,tzen2019theoretical}. 
The inclusion of a state cost has been used to solve transport problems where intermediate path distributions are of importance \citep{liu2023generalized,pooladian2024neural}.
These collective advances naturally lead to the consideration of the optimal control formalism for reward fine-tuning.

\paragraph{Conditional sampling in inverse problems.} \cite{denker2024deft} and \cite{wu2023practical} independently consider a pre-trained diffusion model $p(x)$, and an observation $y$ on the generated sample $x$, as well as the analytic likelihood $p(y|x)$. Their aim is to sample from the posterior $p(x) p(y|x)$, and their applications include inpainting, class-conditional generation, super-resolution, phase retrieval, non-linear deblurring, computed tomography, and protein design. Their setting reduces to a particular case of our reward fine-tuning framework by setting $r(x) = \log p(y|x)$. \cite{denker2024deft} formulate an SOC problem, and they solve it via the log-variance loss (\cite{richter2020vargrad,nüsken2023solving}), and the moment loss \citep{nüsken2023solving}\footnote{See also \cite{domingoenrich2024taxonomy} for a comparison among SOC losses.}, which they refer to as the trajectory balance loss \citep{malkin2023trajectory}. \cite{wu2023practical} propose Twisted Diffusion Sampler, an algorithm based on Sequential Monte Carlo that uses increased inference-time compute to reduce bias.
%can be understood as a poor man's SOC solver combined with importance reweighting to reduce the bias introduced by the suboptimal control. 
A third work that also tackles the conditional sampling problem is \cite{du2024doobs}, which use a Lagrangian formulation that they solve approximately using Gaussian paths. 

\section{Experiments} \label{sec:diff_finetuning_exp}

\begin{table}[t]
\centering
% \resizebox{\textwidth}{!}{%
\small
\begin{tabular}{llccccccc}
    \toprule
    & Fine-tuning & Fine-tuning & Sampling & \multirow{2}{*}{ClipScore$\, \uparrow$} & \multirow{2}{*}{PickScore$\, \uparrow$} & \multirow{2}{*}{HPS v2$\, \uparrow$} & DreamSim 
    % & Total time (s) / 
    \\
    & Method & $\sigma(t)$ & $\sigma(t)$ &  &  &  & Diversity$\, \uparrow$ 
    % & \# iterations 
    \\
    \midrule
    & None & \multirow{2}{*}{\color{gray}N/A} & $\sqrt{2 \eta_t}$ & 24.15{\tiny$\pm$0.26} & 17.25{\tiny$\pm$0.06} & 16.19{\tiny$\pm$0.17} & 53.60{\tiny$\pm$1.37} 
    % & \multirow{2}{*}{\color{gray}N/A} 
    \\ 
    & (Base model)
                                 &                     & 0                 & 28.32{\tiny$\pm$0.22} & 18.15{\tiny$\pm$0.07} & 17.89{\tiny$\pm$0.16} & \textbf{56.53{\tiny$\pm$1.52}} 
                                 % &  
                                 \\
    \midrule % \addlinespace
    \parbox[t]{2mm}{\multirow{8}{*}{\rotatebox[origin=c]{90}{Baselines \;\;\;}}}
    & \multirow{2}{*}{DRaFT-1}           & $\sqrt{2 \eta_t}$ & $\sqrt{2 \eta_t}$ & 30.18{\tiny$\pm$0.24} & 19.38{\tiny$\pm$0.08} & 24.61{\tiny$\pm$0.17} & 25.54{\tiny$\pm$0.99} 
    % & 140k{\tiny$\pm$5.9k} 
    \\
    &                                   & 0                 & 0                 & 30.95{\tiny$\pm$0.28} & 19.37{\tiny$\pm$0.06} & 24.37{\tiny$\pm$0.17} & 27.39{\tiny$\pm$1.14} 
    % & / 4000 
    \\
    \addlinespace
    & \multirow{2}{*}{DRaFT-40}          & $\sqrt{2 \eta_t}$ & $\sqrt{2 \eta_t}$ & 26.94{\tiny$\pm$0.28} & 18.34{\tiny$\pm$0.19} & 19.98{\tiny$\pm$1.02} & 41.98{\tiny$\pm$2.14} 
    % & 148k{\tiny$\pm$4.2k} 
    \\
    &                                   & 0                 & 0                 & 30.07{\tiny$\pm$0.39} & 19.45{\tiny$\pm$0.08} & 24.06{\tiny$\pm$0.24} & 36.53{\tiny$\pm$1.69} 
    % & / 1500 
    \\
    \addlinespace
    & \multirow{2}{*}{DPO}          & $\sqrt{2 \eta_t}$ & $\sqrt{2 \eta_t}$ & 24.11{\tiny$\pm$0.22} & 17.24{\tiny$\pm$0.06} & 16.15{\tiny$\pm$0.14} & 53.27{\tiny$\pm$1.36} 
    % & 118k{\tiny$\pm$0.6k} 
    \\
    &                                   & 0                 & 0                 & 27.77{\tiny$\pm$0.18} & 17.92{\tiny$\pm$0.07} & 17.30{\tiny$\pm$0.20} & 54.11{\tiny$\pm$1.50} 
    % & / 1000 
    \\
    \addlinespace
    & \multirow{2}{*}{ReFL}              & $\sqrt{2 \eta_t}$ & $\sqrt{2 \eta_t}$ & 28.59{\tiny$\pm$0.31} & 18.68{\tiny$\pm$0.10} & 22.24{\tiny$\pm$0.46} & 32.71{\tiny$\pm$2.76} 
    % & 173k{\tiny$\pm$10.9k} 
    \\
    &                                   & 0                 & 0                 & 30.06{\tiny$\pm$0.63} & 19.07{\tiny$\pm$0.21} & 23.06{\tiny$\pm$0.41} & 32.69{\tiny$\pm$1.28} 
    % & / 6000 
    \\
    \midrule % \addlinespace
    \parbox[t]{2mm}{\multirow{10}{*}{\rotatebox[origin=c]{90}{Memoryless SOC \;\;\;\; }}} 
    & Cont. Adjoint & \multirow{2}{*}{$\sqrt{2 \eta_t}$} & $\sqrt{2 \eta_t}$ & 26.99{\tiny$\pm$0.43} & 18.33{\tiny$\pm$0.16} & 20.83{\tiny$\pm$0.63} & 46.59{\tiny$\pm$1.40} 
    % & 153k{\tiny$\pm$0.9k} 
    \\
    & $\lambda = 12500$                     &                                    & 0                 & 29.49{\tiny$\pm$0.32} & 18.98{\tiny$\pm$0.16} & 21.34{\tiny$\pm$0.53} & 48.41{\tiny$\pm$1.44} 
    % & / 750 
    \\
    \addlinespace
    & Disc. Adjoint & \multirow{2}{*}{$\sqrt{2 \eta_t}$} & $\sqrt{2 \eta_t}$ & 28.04{\tiny$\pm$0.57} & 18.44{\tiny$\pm$0.21} & 20.04{\tiny$\pm$0.39} & 54.90{\tiny$\pm$2.03} 
    % & 152k{\tiny$\pm$1.5k} 
    \\
    & $\lambda = 12500$                    &                                    & 0                 & 29.28{\tiny$\pm$0.17} & 18.82{\tiny$\pm$0.14} & 19.73{\tiny$\pm$0.17} & 53.36{\tiny$\pm$2.48} 
    % & / 1000 
    \\
    \addlinespace
    \cline{2-9}\addlinespace
    & Adj.-Matching  & \multirow{2}{*}{$\sqrt{2 \eta_t}$} & $\sqrt{2 \eta_t}$ & 30.36{\tiny$\pm$0.22} & 19.29{\tiny$\pm$0.08} & 24.12{\tiny$\pm$0.17} & 40.89{\tiny$\pm$1.50} 
    % & 
    \\
    & $\lambda = 1000$                     &                                    & 0                 & 31.41{\tiny$\pm$0.22} & 19.57{\tiny$\pm$0.09} & 23.29{\tiny$\pm$0.18} & 43.10{\tiny$\pm$1.76} 
    % &  
    \\
    \addlinespace
    & Adj.-Matching & \multirow{2}{*}{$\sqrt{2 \eta_t}$} & $\sqrt{2 \eta_t}$ & 30.59{\tiny$\pm$0.40} & 19.49{\tiny$\pm$0.10} & 24.85{\tiny$\pm$0.23} & 37.07{\tiny$\pm$1.47} 
    % & 156k{\tiny$\pm$1.9k} 
    \\
    & $\lambda = 2500$                     &                                    & 0                 & 31.64{\tiny$\pm$0.21} & 19.71{\tiny$\pm$0.09} & 24.12{\tiny$\pm$0.27} & 39.88{\tiny$\pm$1.59} 
    % & / 1000 
    \\
    \addlinespace
    & Adj.-Matching  & \multirow{2}{*}{$\sqrt{2 \eta_t}$} & $\sqrt{2 \eta_t}$ & 30.62{\tiny$\pm$0.30} & 19.50{\tiny$\pm$0.09} & \textbf{24.95{\tiny$\pm$0.28}} & 34.50{\tiny$\pm$1.33}
    % &  
    \\
    & $\lambda = 12500$                    &                                    & 0                 & \textbf{31.65{\tiny$\pm$0.19}} & \textbf{19.76{\tiny$\pm$0.08}} & 24.49{\tiny$\pm$0.27} & 37.24{\tiny$\pm$1.57} 
    % &  
    \\
    \bottomrule
\end{tabular}
% }
\caption{Evaluation metrics of different fine-tuning methods for text-to-image generation. 
The second and third columns show the noise schedules $\sigma(t)$ used for fine-tuning and for sampling: $\sigma(t) = \sqrt{2\eta_t}$ corresponds to Memoryless Flow Matching, and $\sigma(t) = 0$ to the Flow Matching ODE \eqref{eq:FM_ode}. 
We report standard errors estimated over 3 runs of the fine-tuning algorithm on random sets of 40000 training prompts, each evaluated over a random set of 1000 test prompts. 
% \carles{Put wall clock time in the appendix table}
}
\label{tab:evaluation_metrics}
\end{table}

\begin{figure}[t!]
    \centering
    \begin{subfigure}[t]{0.49\linewidth}
        \centering
        \rotatebox{90}{\;\; $\lambda=1000$}\,%
        \includegraphics[width=0.24\linewidth]{figs/samples/adjmat1000_prompt_32_image_0.jpg}\,%
        \includegraphics[width=0.24\linewidth]{figs/samples/adjmat1000_prompt_32_image_2.jpg}\,%
        \includegraphics[width=0.24\linewidth]{figs/samples/adjmat1000_prompt_83_image_5.jpg}\,%
        \includegraphics[width=0.24\linewidth]{figs/samples/adjmat1000_prompt_83_image_8.jpg}\\
        \rotatebox{90}{\;\; $\lambda=2500$}\,%
        \includegraphics[width=0.24\linewidth]{figs/samples/adjmat2500_prompt_32_image_0.jpg}\,%
        \includegraphics[width=0.24\linewidth]{figs/samples/adjmat2500_prompt_32_image_2.jpg}\,%
        \includegraphics[width=0.24\linewidth]{figs/samples/adjmat2500_prompt_83_image_5.jpg}\,%
        \includegraphics[width=0.24\linewidth]{figs/samples/adjmat2500_prompt_83_image_8.jpg}\\
        \rotatebox{90}{\; $\lambda=12500$}\,%
        \includegraphics[width=0.24\linewidth]{figs/samples/adjmat12500_prompt_32_image_0.jpg}\,%
        \includegraphics[width=0.24\linewidth]{figs/samples/adjmat12500_prompt_32_image_2.jpg}\,%
        \includegraphics[width=0.24\linewidth]{figs/samples/adjmat12500_prompt_83_image_5.jpg}\,%
        \includegraphics[width=0.24\linewidth]{figs/samples/adjmat12500_prompt_83_image_8.jpg}
        \caption*{Adjoint Matching (Ours)}
    \end{subfigure}\hfill
    \begin{subfigure}[t]{0.49\linewidth}
        \centering
        \includegraphics[width=0.24\linewidth]{figs/samples/draft1000_prompt_32_image_0.jpg}\,%
        \includegraphics[width=0.24\linewidth]{figs/samples/draft1000_prompt_32_image_2.jpg}\,%
        \includegraphics[width=0.24\linewidth]{figs/samples/draft1000_prompt_83_image_5.jpg}\,%
        \includegraphics[width=0.24\linewidth]{figs/samples/draft1000_prompt_83_image_8.jpg}\,%
        \rotatebox[origin=r]{270}{$1000$ itrs. }\\
        \includegraphics[width=0.24\linewidth]{figs/samples/draft2000_prompt_32_image_0.jpg}\,%
        \includegraphics[width=0.24\linewidth]{figs/samples/draft2000_prompt_32_image_2.jpg}\,%
        \includegraphics[width=0.24\linewidth]{figs/samples/draft2000_prompt_83_image_5.jpg}\,%
        \includegraphics[width=0.24\linewidth]{figs/samples/draft2000_prompt_83_image_8.jpg}\,%
        \rotatebox[origin=r]{270}{$2000$ itrs. }\\
        \includegraphics[width=0.24\linewidth]{figs/samples/draft4000_prompt_32_image_0.jpg}\,%
        \includegraphics[width=0.24\linewidth]{figs/samples/draft4000_prompt_32_image_2.jpg}\,%
        \includegraphics[width=0.24\linewidth]{figs/samples/draft4000_prompt_83_image_5.jpg}\,%
        \includegraphics[width=0.24\linewidth]{figs/samples/draft4000_prompt_83_image_8.jpg}\,%
        \rotatebox[origin=r]{270}{$4000$ itrs. }
        \caption*{DRaFT-1}
    \end{subfigure}
    \caption{Our proposed Adjoint Matching using the memoryless SOC formulation introduces a much more principled way of trading off how close to stay to the base model while optimizing the reward model. In contrast, baseline methods such as DRaFT-1 only optimize the reward model and must rely on early stopping to perform this trade off, resulting in a much more sensitive hyperparameter. Samples are produced using $\sigma(t)=0$ with the same noise sample. Text prompts: ``\textit{Handsome Smiling man in blue jacket portrait}'' and ``\textit{Quinoa and Feta Stuffed Baby Bell Peppers}''.}
    \label{fig:ablation_tradeoff_lambda}
\end{figure}

\begin{figure}[t!]
    \centering
    \begin{subfigure}[t]{0.49\linewidth}
        \centering
        \rotatebox{90}{\;\;\; $w=0.0$}\,%
        \includegraphics[width=0.24\linewidth]{figs/samples/adjmat0.0_prompt_37_image_3.jpg}\,%
        \includegraphics[width=0.24\linewidth]{figs/samples/adjmat0.0_prompt_37_image_5.jpg}\,%
        \includegraphics[width=0.24\linewidth]{figs/samples/adjmat0.0_prompt_37_image_6.jpg}\,%
        \includegraphics[width=0.24\linewidth]{figs/samples/adjmat0.0_prompt_37_image_7.jpg}\\
        \rotatebox{90}{\;\;\; $w=1.0$}\,%
        \includegraphics[width=0.24\linewidth]{figs/samples/adjmat1.0_prompt_37_image_3.jpg}\,%
        \includegraphics[width=0.24\linewidth]{figs/samples/adjmat1.0_prompt_37_image_5.jpg}\,%
        \includegraphics[width=0.24\linewidth]{figs/samples/adjmat1.0_prompt_37_image_6.jpg}\,%
        \includegraphics[width=0.24\linewidth]{figs/samples/adjmat1.0_prompt_37_image_7.jpg}\\
        \rotatebox{90}{\;\;\; $w=4.0$}\,%
        \includegraphics[width=0.24\linewidth]
        {figs/samples/adjmat4.0_prompt_37_image_3.jpg}\,%
        \includegraphics[width=0.24\linewidth]{figs/samples/adjmat4.0_prompt_37_image_5.jpg}\,%
        \includegraphics[width=0.24\linewidth]{figs/samples/adjmat4.0_prompt_37_image_6.jpg}\,%
        \includegraphics[width=0.24\linewidth]{figs/samples/adjmat4.0_prompt_37_image_7.jpg}
        \caption*{Text prompt: ``\textit{Man sitting on sofa at home in front of fireplace and using laptop computer, rear view}''}
    \end{subfigure}\hfill
    \begin{subfigure}[t]{0.49\linewidth}
        \centering
        \includegraphics[width=0.24\linewidth]{figs/samples/adjmat0.0_prompt_74_image_6.jpg}\,%
        \includegraphics[width=0.24\linewidth]{figs/samples/adjmat0.0_prompt_74_image_7.jpg}\,%
        \includegraphics[width=0.24\linewidth]{figs/samples/adjmat0.0_prompt_74_image_8.jpg}\,%
        \includegraphics[width=0.24\linewidth]{figs/samples/adjmat0.0_prompt_74_image_9.jpg}\\
        \includegraphics[width=0.24\linewidth]{figs/samples/adjmat1.0_prompt_74_image_6.jpg}\,%
        \includegraphics[width=0.24\linewidth]{figs/samples/adjmat1.0_prompt_74_image_7.jpg}\,%
        \includegraphics[width=0.24\linewidth]{figs/samples/adjmat1.0_prompt_74_image_8.jpg}\,%
        \includegraphics[width=0.24\linewidth]{figs/samples/adjmat1.0_prompt_74_image_9.jpg}\\
        \includegraphics[width=0.24\linewidth]{figs/samples/adjmat4.0_prompt_74_image_6.jpg}\,%
        \includegraphics[width=0.24\linewidth]{figs/samples/adjmat4.0_prompt_74_image_7.jpg}\,%
        \includegraphics[width=0.24\linewidth]{figs/samples/adjmat4.0_prompt_74_image_8.jpg}\,%
        \includegraphics[width=0.24\linewidth]{figs/samples/adjmat4.0_prompt_74_image_9.jpg}
        \caption*{Text prompt: ``\textit{3D World Food Day Morocco}''}
    \end{subfigure}
    \caption{
    Generated samples from varying classifier-free guidance weight $w$, from an Adjoint Matching fine-tuned model. 
    Higher guidance increases text-to-image consistency but loses diversity and has use cases for generating highly structured images such as 3D renderings.
    Corresponding samples from the base model can be found in \Cref{fig:ablation_tradeoff_cfg_base}. 
    }
    \label{fig:ablation_tradeoff_cfg}
\end{figure}

\begin{figure}
    \centering
    \includegraphics[width=0.65\linewidth]{figs/metrics_relationship_plots/legend.pdf}\\
    \includegraphics[width=0.325\linewidth]{figs/metrics_relationship_plots/dreamsim_diversity_vs_clipscore.pdf}
    \includegraphics[width=0.325\linewidth]{figs/metrics_relationship_plots/dreamsim_diversity_vs_hps_v2.pdf}
    \includegraphics[width=0.325\linewidth]{figs/metrics_relationship_plots/hps_v2_vs_clipscore.pdf}
    \caption{Tradeoffs between different aspects of generative models: text-to-image consistency (ClipScore), sample diversity for each prompt (DreamSim Diversity), and generalization to unseen human preferences (HPS v2). Different points are obtained from varying values of $\lambda$ for Adjoint Matching and varying number of fine-tuning iterations for the DRaFT-1 baseline. Overall, we find our proposed method Adjoint Matching has the best Pareto fronts.}
    \label{fig:tradeoffs}
\end{figure}

We experimentally validate our proposed method on reward fine-tuning a Flow Matching base model \citep{lipman2023flow}. In particular, we use the usual setup of pre-training an autoencoder for 512$\times$512 resolution images, then training a text-conditional Flow Matching model on the latent variables with a U-net architecture \citep{long2015fully}, similar to the setup in \citet{rombach2022high}. We pre-trained our base model using a dataset of licensed text and image pairs. Then for fine-tuning, we consider the reward function:
\begin{talign}
    r(x) := \lambda \times \texttt{RewardModel}(x)
\end{talign}
corresponding to a scaled version of the reward model, which we take to be ImageReward~\citep{xu2023imagereward}. Different values of $\lambda$ provide different tradeoffs between the KL regularization and the reward model \eqref{eq:kl_regularized_interpretation}.

For evaluation and benchmarking purposes, we report metrics that separately quantify text-to-image consistency, human preference, and sample diversity, capturing the tradeoff between each aspect of generative models~\citep{astolfi2024consistency}. For consistency, we make use of the standard ClipScore \citep{hessel2021clipscore} and PickScore \citep{kirstain2023pickapic}; for generalization to unseen human preferences, we use the HPSv2 model \citep{wu2023human}; and for diversity, we compute averages of pairwise distances of the DreamSim features \citep{fu2023learning}. More details are provided in \Cref{subsec:evaluation_metrics}.
% \ricky{TODO appendix}.

As our baselines, we consider the DPO \citep{wallace2023diffusion}, ReFL \citep{xu2023imagereward}, and DRaFT-K algorithms \citep{clark2024directly}. DPO does not use gradients from the reward function, while ReFL and DRaFT make use of heuristic gradient stopping approaches to stay close to the base generative model. 
Out of these baseline methods, we find that DRaFT-1 performs the best, so we perform additional ablation experiments comparing to this method.
Within the same SOC formulation as our method, we also consider the discrete and continuous adjoint methods. We provide full experimental details in \Cref{sec:experimental_details}; an important implementation detail is that we slightly offset $\sigma(t)$ in order to avoid division by zero.

\paragraph{Evaluation results.} In \Cref{tab:evaluation_metrics} we report the evaluation metrics for the baselines as well as our proposed Adjoint Matching approach. We compare each method at roughly the same wall clock time (see the times and number of iterations in \Cref{table:metrics_multiprompt_alternative}, and comments in \Cref{subsec:remarks_computational_cost}).
We find that across all metrics, our proposed memoryless SOC formulation outperforms existing baseline methods. The choice of SOC algorithms also obviously favors Adjoint Matching over continuous and discrete adjoint methods, which result in poorer consistency and human preference metrics.

\paragraph{Ablation: base model vs. reward tradeoff.} We note that the scaling in front of the reward model $\lambda$ determines how strongly the we should prefer the reward model over the base model. As such, we see a natural tradeoff curve: higher $\lambda$ results in better consistency and human preference, but lower diversity in the generated samples. Overall, we find that Adjoint Matching performs stably across all values of $\lambda$. Our method of regularizing the fine-tuning procedure through memoryless SOC works much better than baseline methods which often must employ early stopping. We show the qualitative effect of varying $\lambda$ in \Cref{fig:ablation_tradeoff_lambda}, while for the DRaFT-1 baseline we show the effect of varying the number of fine-tuning iterations.

\paragraph{Ablation: classifier-free guidance.} We note that it is possible to apply classifier-free guidance (CFG; \citet{ho2022classifier,zheng2023guided}) after fine-tuning. We use the formula $(1+w) v(x, t | y) - w v(x, t)$, where $w$ is the guidance weight, $v(x, t | y)$ is a fine-tuned text-to-image model while $v(x, t)$ is an unconditional image model.
This is not principled as only the conditional model is fine-tuned, but generally it is unclear what distribution guided models sample from anyhow. 
In \Cref{fig:tradeoffs} we show the evaluation metrics with classifier-free guidance applied. Comparing three different guidance weight values, we see a higher weight does improve text-to-image consistency, and to some extent, human preference, but this comes at the cost of being worse in terms of diversity. We show qualitative differences in \Cref{fig:ablation_tradeoff_cfg}.

\section{Conclusion}
\label{sec:conclusion}

We investigate the problem of fine-tuning dynamical generative models such as Flow Matching and propose the use of a stochastic optimal control (SOC) formulation with a memoryless noise schedule. 
This ensures we converge to the same tilted distribution that the large language modeling literature uses for learning from human feedback. 
In particular, the memoryless noise schedule corresponds to DDPM sampling for diffusion models and a new Memoryless Flow Matching generative process for flow models. 
In conjunction, we propose a novel training algorithm for solving stochastic optimal control problems, by casting SOC as a regression problem, which we call the Adjoint Matching objective. 
Empirically, we find that our memoryless SOC formulation works better than multiple existing works on fine-tuning diffusion models, and our Adjoint Matching algorithm outperforms related gradient-based methods.
In summary, we are the first to provide a theoretically-driven algorithm for fine-tuning Flow Matching models, and we find that our approach significantly outperforms baseline methods across multiple axes of evaluation---text-to-image consistency, generalization to unseen human preference, and sample diversity---on large-scale text-to-image generation.

\bibliographystyle{assets/plainnat}
\bibliography{biblio}

\clearpage
\newpage
\beginappendix

\tableofcontents

\starttocentries

\clearpage
\newpage

\section{Additional Figures \& Tables}
\label{sec:additional_figures_tables}

\begin{figure}[h!]
    \centering
    \includegraphics[width=0.46\linewidth]{figs/training_plots/reward_plot.png}
    \includegraphics[width=0.46\linewidth]{figs/training_plots/control_term_plot.png}
    \includegraphics[width=0.46\linewidth]{figs/training_plots/clipscore_plot.png}
    \caption{Average values of ImageReward (reward function), control cost ($\int_0^t \frac{1}{2} \|u(X^u_t,t)\|^2 \, \mathrm{d}t$), and ClipScore vs. wall-clock time for Adjoint Matching and our baselines. Lines show averages over three fine-tuning runs, evaluating on separate test datasets of size 200. Confidence intervals show standard errors of estimates.}
    \label{fig:training_figures}
\end{figure}

\begin{table}[h!]
\centering
{\small
\begin{tabular}{lcccccc}
    \toprule
    Fine-tuning & Fine-tuning & Sampling & \multirow{2}{*}{ImageReward$\, \uparrow$} & ClipScore & PickScore & Total time (s) /
    \\
    loss & $\sigma(t)$ & $\sigma(t)$ &  & diversity$\, \uparrow$ & diversity$\, \uparrow$ & \# iterations \\
    \midrule
    None & \multirow{2}{*}{N/A} & $\sqrt{2 \eta_t}$ & $-$1.384{\tiny$\pm$0.040} & 28.07{\tiny$\pm$1.40} & 1.63{\tiny$\pm$0.08} & \multirow{2}{*}{N/A} 
    \\
    ($\mathrm{CFG}=1.0$)                    &                     & 0                 & $-$0.920{\tiny$\pm$0.042} & 30.29{\tiny$\pm$1.53} & 1.82{\tiny$\pm$0.09} \\
    \midrule
    \multirow{2}{*}{DRaFT-1}           & $\sqrt{2 \eta_t}$ & $\sqrt{2 \eta_t}$ & 1.357{\tiny$\pm$0.039} & 16.86{\tiny$\pm$0.98} & 1.21{\tiny$\pm$0.07} 
    & 140k{\tiny$\pm$5.9k}
    \\
                                       & 0                 & 0                 & 1.251{\tiny$\pm$0.040} & 16.76{\tiny$\pm$1.06} & 1.27{\tiny$\pm$0.07} 
    & / 4000 
    \\
    \addlinespace
    \multirow{2}{*}{DRaFT-40}          & $\sqrt{2 \eta_t}$ & $\sqrt{2 \eta_t}$ & $-$0.560{\tiny$\pm$0.138} & 24.07{\tiny$\pm$1.37} & 1.64{\tiny$\pm$0.12} 
    & 148k{\tiny$\pm$4.2k} 
    \\
                                       & 0                 & 0                 & 0.424{\tiny$\pm$0.042} & 20.99{\tiny$\pm$1.54} & 1.67{\tiny$\pm$0.08} 
    & / 1500 
    \\
    \addlinespace
    \multirow{2}{*}{DPO}          & $\sqrt{2 \eta_t}$ & $\sqrt{2 \eta_t}$ & $-$1.386{\tiny$\pm$0.033} & 27.80{\tiny$\pm$1.40} & 1.62{\tiny$\pm$0.08} 
    & 118k{\tiny$\pm$0.6k} 
    \\
                                       & 0                 & 0                 & $-$0.957{\tiny$\pm$0.040} & 29.81{\tiny$\pm$1.43} & 1.68{\tiny$\pm$0.10} 
    & / 1000
    \\
    \addlinespace
    \multirow{2}{*}{ReFL}              & $\sqrt{2 \eta_t}$ & $\sqrt{2 \eta_t}$ & 0.687{\tiny$\pm$0.085} & 19.49{\tiny$\pm$1.76} & 1.22{\tiny$\pm$0.08} 
    & 173k{\tiny$\pm$10.9k}
    \\
                                       & 0                 & 0                 & 0.709{\tiny$\pm$0.080} & 18.39{\tiny$\pm$1.11} & 1.31{\tiny$\pm$0.10} 
    & / 6000 
    \\
    \midrule % \addlinespace
    Cont. Adjoint & \multirow{2}{*}{$\sqrt{2 \eta_t}$} & $\sqrt{2 \eta_t}$ & $-$0.448{\tiny$\pm$0.135} & 26.97{\tiny$\pm$1.37} & 1.82{\tiny$\pm$0.09} 
    & 153k{\tiny$\pm$0.9k}
    \\
    $\lambda = 12500$                     &                                    & 0                 & $-$0.249{\tiny$\pm$0.116} & 26.25{\tiny$\pm$1.30} & 1.90{\tiny$\pm$0.10} 
    & / 750  
    \\
    \addlinespace
    Disc. Adjoint & \multirow{2}{*}{$\sqrt{2 \eta_t}$} & $\sqrt{2 \eta_t}$ & $-$0.557{\tiny$\pm$0.113} & 30.40{\tiny$\pm$2.39} & 1.91{\tiny$\pm$0.09} 
    & 152k{\tiny$\pm$1.5k}
    \\
    $\lambda = 12500$                    &                                    & 0                 & $-$0.552{\tiny$\pm$0.041} & 28.37{\tiny$\pm$2.26} & 1.97{\tiny$\pm$0.09} 
    & / 1000 
    \\
    \midrule % \addlinespace
    Adj.-Matching  & \multirow{2}{*}{$\sqrt{2 \eta_t}$} & $\sqrt{2 \eta_t}$ & 0.550{\tiny$\pm$0.043} & 23.00{\tiny$\pm$1.27} & 1.65{\tiny$\pm$0.08} 
    &  
    \\
    $\lambda = 1000$                     &                                    & 0                 & 0.454{\tiny$\pm$0.055} & 22.76{\tiny$\pm$1.40} & 1.73{\tiny$\pm$0.09} 
    % &  
    \\
    \addlinespace
    Adj.-Matching & \multirow{2}{*}{$\sqrt{2 \eta_t}$} & $\sqrt{2 \eta_t}$ & 0.755{\tiny$\pm$0.040} & 21.33{\tiny$\pm$1.71} & 1.55{\tiny$\pm$0.08} 
    & 156k{\tiny$\pm$1.9k}
    \\
    $\lambda = 2500$                     &                                    & 0                 & 0.671{\tiny$\pm$0.047} & 21.42{\tiny$\pm$1.54} & 1.64{\tiny$\pm$0.08}
    & / 1000 
    \\
    \addlinespace
    Adj.-Matching  & \multirow{2}{*}{$\sqrt{2 \eta_t}$} & $\sqrt{2 \eta_t}$ & 0.882{\tiny$\pm$0.058} & 20.49{\tiny$\pm$1.48} & 1.50{\tiny$\pm$0.09} 
    & 
    \\
    $\lambda = 12500$                    &                                    & 0                 & 0.778{\tiny$\pm$0.050} & 20.34{\tiny$\pm$1.49} & 1.57{\tiny$\pm$0.09}
    &  
    \\
    \bottomrule
\end{tabular}
}
\caption{Metrics for various fine-tuning methods for text-to-image generation. The second and third columns show the noise schedules $\sigma(t)$ used for fine-tuning and for inference: $\sigma(t) = \sqrt{2\eta_t}$ corresponds to Memoryless Flow Matching, %\eqref{eq:memoryless_FM_sde}, 
and $\sigma(t) = 0$ to the Flow Matching ODE \eqref{eq:FM_ode}. Confidence intervals show standard errors of estimates; computed over 3 runs of the fine-tuning algorithm on separate fine-tuning prompt datasets of size 40000 each. Test prompt sets are of size 1000, and also different for each run.}
\label{table:metrics_multiprompt_diversity}
\end{table}

\begin{table}[h!]
\centering
{\footnotesize
\begin{tabular}{lcccccccc}
    \toprule
    Fine-tun. & Fine-tun. & Generat. & \multirow{2}{*}{ImageReward$\, \uparrow$} & \multirow{2}{*}{ClipScore$\, \uparrow$} & \multirow{2}{*}{PickScore$\, \uparrow$} & \multirow{2}{*}{HPS v2$\, \uparrow$} & DreamSim & Runtime/ \\
    loss & $\sigma(t)$ & $\sigma(t)$ &  & &  &  & diversity$\, \uparrow$ & $\#$iter. \\
    % \midrule
    % \multirow{2}{*}{None (Base)} & \multirow{2}{*}{N/A} & $\sqrt{2 \eta_t}$ & $-\! \pm \!-$ & $-\! \pm \!-$ & $-\! \pm \!-$ & $-\! \pm \!-$ & \multirow{2}{*}{N/A} \\
    %                              &                     & 0                 & $-\! \pm \!-$ & $-\! \pm \!-$ & $-\! \pm \!-$ & $-\! \pm \!-$ &  \\
    \midrule % \addlinespace
    \multirow{2}{*}{ReFL}              & $\sqrt{2 \eta_t}$ & $\sqrt{2 \eta_t}$ & 0.459{\tiny$\pm$0.096} & 28.46{\tiny$\pm$0.25} & 18.77{\tiny$\pm$0.09} & 22.54{\tiny$\pm$0.17} & 37.51{\tiny$\pm$3.50} & 43k{\tiny$\pm$2.7k} \\
                                       & 0                 & 0                 & 0.330{\tiny$\pm$0.114} & 29.63{\tiny$\pm$0.61} & 19.08{\tiny$\pm$0.18} & 22.46{\tiny$\pm$0.77} & 39.51{\tiny$\pm$1.30} & / 1500 \\
    \addlinespace
    \multirow{2}{*}{DRaFT-1}           & $\sqrt{2 \eta_t}$ & $\sqrt{2 \eta_t}$ & 0.913{\tiny$\pm$0.068} & 29.80{\tiny$\pm$0.22} & 19.16{\tiny$\pm$0.06} & 23.63{\tiny$\pm$0.16} & 35.21{\tiny$\pm$1.93} & 35k{\tiny$\pm$1.5k} \\
                                       & 0                 & 0                 & 0.626{\tiny$\pm$0.195} & 30.48{\tiny$\pm$0.32} & 18.91{\tiny$\pm$0.34} & 21.92{\tiny$\pm$1.63} & 38.52{\tiny$\pm$2.01} & / 1000 \\
    \addlinespace
    \multirow{2}{*}{Draft-40}          & $\sqrt{2 \eta_t}$ & $\sqrt{2 \eta_t}$ & $-$1.427{\tiny$\pm$0.267} & 23.39{\tiny$\pm$1.72} & 17.24{\tiny$\pm$0.45} & 15.72{\tiny$\pm$1.80} & 41.98{\tiny$\pm$2.14} & 49k{\tiny$\pm$1.4k} \\
                                       & 0                 & 0                 & $-$0.097{\tiny$\pm$0.052} & 29.12{\tiny$\pm$0.41} & 18.97{\tiny$\pm$0.14} & 21.93{\tiny$\pm$0.20} & 46.35{\tiny$\pm$1.34} & / 500 \\
    \midrule % \addlinespace
    Adj.-Match.  & \multirow{2}{*}{$\sqrt{2 \eta_t}$} & $\sqrt{2 \eta_t}$ & 0.107{\tiny$\pm$0.046} & 29.37{\tiny$\pm$0.25} & 19.05{\tiny$\pm$0.07} & 22.79{\tiny$\pm$0.20} & 46.38{\tiny$\pm$1.36} &  \\
    $\lambda = 1000$                     &                                    & 0                 & 0.051{\tiny$\pm$0.044} & 30.58{\tiny$\pm$0.17} & 19.31{\tiny$\pm$0.07} & 21.93{\tiny$\pm$0.23} & 48.12{\tiny$\pm$1.56} &  \\
    \addlinespace
    Adj.-Match. & \multirow{2}{*}{$\sqrt{2 \eta_t}$} & $\sqrt{2 \eta_t}$ & 0.199{\tiny$\pm$0.068} & 29.27{\tiny$\pm$0.21} & 19.07{\tiny$\pm$0.10} & 22.98{\tiny$\pm$0.30} & 45.03{\tiny$\pm$1.61} & 39k{\tiny$\pm$0.5k} \\
    $\lambda = 2500$                     &                                    & 0                 & 0.106{\tiny$\pm$0.067} & 30.43{\tiny$\pm$0.24} & 19.32{\tiny$\pm$0.11} & 22.16{\tiny$\pm$0.33} & 47.61{\tiny$\pm$1.49} & / 250 \\
    \addlinespace
    Adj.-Match.  & \multirow{2}{*}{$\sqrt{2 \eta_t}$} & $\sqrt{2 \eta_t}$ & 0.299{\tiny$\pm$0.095} & 29.61{\tiny$\pm$0.37} & 19.26{\tiny$\pm$0.14} & 23.67{\tiny$\pm$0.27} & 43.36{\tiny$\pm$1.93} &  \\
    $\lambda = 12500$                    &                                    & 0                 & 0.224{\tiny$\pm$0.051} & 30.70{\tiny$\pm$0.23} & 19.52{\tiny$\pm$0.11} & 22.93{\tiny$\pm$0.21} & 44.62{\tiny$\pm$1.79} &  \\
    \midrule % \addlinespace
    Cont. Adj. & \multirow{2}{*}{$\sqrt{2 \eta_t}$} & $\sqrt{2 \eta_t}$ & $-$0.910{\tiny$\pm$0.116} & 26.29{\tiny$\pm$0.44} & 18.06{\tiny$\pm$0.16} & 18.86{\tiny$\pm$0.88} & 51.60{\tiny$\pm$1.97} & 51k{\tiny$\pm$0.3k} \\
    $\lambda = 12500$                     &                                    & 0                 & $-$0.681{\tiny$\pm$0.051} & 28.50{\tiny$\pm$0.19} & 18.69{\tiny$\pm$0.11} & 19.90{\tiny$\pm$0.50} & 50.87{\tiny$\pm$1.52} & / 250 \\
    \addlinespace
    Disc. Adj. & \multirow{2}{*}{$\sqrt{2 \eta_t}$} & $\sqrt{2 \eta_t}$ & $-$0.978{\tiny$\pm$0.123} & 26.68{\tiny$\pm$0.76} & 18.51{\tiny$\pm$0.11} & 18.53{\tiny$\pm$0.28} & 55.95{\tiny$\pm$1.70} & 38k{\tiny$\pm$0.4k} \\
    $\lambda = 12500$                    &                                    & 0                 & $-$0.791{\tiny$\pm$0.065} & 28.66{\tiny$\pm$0.33} & 18.51{\tiny$\pm$0.11} & 18.53{\tiny$\pm$0.28} & 54.78{\tiny$\pm$2.00} & / 250 \\
    \bottomrule
\end{tabular}
}
\caption{Additional metrics for various fine-tuning methods for text-to-image generation, which complement the ones in \Cref{tab:evaluation_metrics} (both tables correspond to the same runs). The second and third columns show the noise schedules $\sigma(t)$ used for fine-tuning and for inference: $\sigma(t) = \sqrt{2\eta_t}$ corresponds to Memoryless Flow Matching, 
% \eqref{eq:memoryless_FM_sde}, 
and $\sigma(t) = 0$ to the Flow Matching ODE \eqref{eq:FM_ode}.
}
\label{table:metrics_multiprompt_alternative}
\end{table}

\begin{table}[h!]
\centering
{\footnotesize
\begin{tabular}{lllccccccc}
    \toprule
    \multirow{2}{*}{$w$} & Fine-tuning & $\#$iter. & Fine-tun. & Sampl.  & \multirow{2}{*}{ImageReward$\, \uparrow$} & \multirow{2}{*}{ClipScore$\, \uparrow$} & \multirow{2}{*}{PickScore$\, \uparrow$} & \multirow{2}{*}{HPS v2$\, \uparrow$} & DreamSim \\
     & loss & / $\lambda$ & $\sigma(t)$ & $\sigma(t)$ &  &  &  &  & diversity$\, \uparrow$ \\
    \midrule
    \multirow{2}{*}{0.0} & \multirow{2}{*}{None} &  \multirow{2}{*}{N/A}        & \multirow{2}{*}{N/A}          & $\sqrt{2 \eta_t}$         & $-$1.384{\tiny$\pm$0.040} & 24.15{\tiny$\pm$0.26} & 17.25{\tiny$\pm$0.06} & 16.19{\tiny$\pm$0.17} & 53.60{\tiny$\pm$1.37} \\
     &  &       &                & 0                 & $-$0.920{\tiny$\pm$0.042} & 28.32{\tiny$\pm$0.22} & 18.15{\tiny$\pm$0.07} & 17.89{\tiny$\pm$0.16} & \textbf{56.53{\tiny$\pm$1.52}}       
                                    \\
    \midrule
    \multirow{7}{*}{0.0} & \multirow{7}{*}{DRaFT-1} &  \multirow{2}{*}{1000}       & $\sqrt{2 \eta_t}$ & $\sqrt{2 \eta_t}$ & 0.913{\tiny$\pm$0.068} & 29.80{\tiny$\pm$0.22} & 19.16{\tiny$\pm$0.06} & 23.63{\tiny$\pm$0.16} & 35.21{\tiny$\pm$1.93} \\
    & &  & 0                 & 0                 & 0.626{\tiny$\pm$0.195} & 30.48{\tiny$\pm$0.32} & 18.91{\tiny$\pm$0.34} & 21.92{\tiny$\pm$1.63} & 38.52{\tiny$\pm$2.01} \\
    % \midrule
    \addlinespace
     &  &  \multirow{2}{*}{2000}       & $\sqrt{2 \eta_t}$ & $\sqrt{2 \eta_t}$ & 1.204{\tiny$\pm$0.046} & 29.90{\tiny$\pm$0.43} & 19.29{\tiny$\pm$0.12} & 24.40{\tiny$\pm$0.27} & 28.51{\tiny$\pm$1.68} \\
                                   & &  & 0                 & 0                 & 1.052{\tiny$\pm$0.088} & 30.65{\tiny$\pm$0.24} & 19.27{\tiny$\pm$0.11} & 23.81{\tiny$\pm$0.44} & 32.11{\tiny$\pm$2.37} \\
    % \midrule
    \addlinespace
    & &  \multirow{2}{*}{3000}       & $\sqrt{2 \eta_t}$ & $\sqrt{2 \eta_t}$ & \textbf{1.307{\tiny$\pm$0.041}} & 29.96{\tiny$\pm$0.22} & 19.31{\tiny$\pm$0.06} & 24.42{\tiny$\pm$0.13} & 26.57{\tiny$\pm$1.32} \\
                                   & &  & 0                 & 0                 & 1.173{\tiny$\pm$0.058} & 30.86{\tiny$\pm$0.25} & 19.37{\tiny$\pm$0.06} & 24.17{\tiny$\pm$0.23} & 29.69{\tiny$\pm$1.30}
    \\
    \addlinespace
    & &  \multirow{2}{*}{4000}       & $\sqrt{2 \eta_t}$ & $\sqrt{2 \eta_t}$ & \textbf{1.357{\tiny$\pm$0.039}} & 30.18{\tiny$\pm$0.24} & 19.38{\tiny$\pm$0.08} & 24.61{\tiny$\pm$0.17} & 25.54{\tiny$\pm$0.99} \\
                                   & &  & 0                 & 0                 & 1.251{\tiny$\pm$0.040} & 30.95{\tiny$\pm$0.28} & 19.37{\tiny$\pm$0.06} & 24.37{\tiny$\pm$0.17} & 27.39{\tiny$\pm$1.14}
    \\
    \midrule
    \multirow{7}{*}{0.0} & \multirow{7}{*}{Adj.-Match.} & \multirow{2}{*}{1000} & $\sqrt{2 \eta_t}$ & $\sqrt{2 \eta_t}$  & 0.550{\tiny$\pm$0.043} & 30.36{\tiny$\pm$0.22} & 19.29{\tiny$\pm$0.08} & 24.12{\tiny$\pm$0.17} & 40.89{\tiny$\pm$1.50} \\
     & &  & 0                 & 0                 & 0.454{\tiny$\pm$0.055} &  31.41{\tiny$\pm$0.22} & 19.57{\tiny$\pm$0.09} & 23.29{\tiny$\pm$0.18} & 43.10{\tiny$\pm$1.76} \\
    \addlinespace
    & & \multirow{2}{*}{2500} & $\sqrt{2 \eta_t}$ & $\sqrt{2 \eta_t}$ & 0.755{\tiny$\pm$0.040} & 30.59{\tiny$\pm$0.40} & 19.49{\tiny$\pm$0.10} & 24.85{\tiny$\pm$0.23} & 37.07{\tiny$\pm$1.47} \\
    & &  & 0 & 0 & 0.671{\tiny$\pm$0.047} & 31.64{\tiny$\pm$0.21} & 19.71{\tiny$\pm$0.09} & 24.12{\tiny$\pm$0.27} & 39.88{\tiny$\pm$1.59} \\
    \addlinespace
    & & \multirow{2}{*}{12500} & $\sqrt{2 \eta_t}$ & $\sqrt{2 \eta_t}$ & 0.882{\tiny$\pm$0.058} & 30.62{\tiny$\pm$0.30} & 19.50{\tiny$\pm$0.09} & 24.95{\tiny$\pm$0.28} & 34.50{\tiny$\pm$1.33} \\
    & &  & 0 & 0 & 0.778{\tiny$\pm$0.050} & 31.65{\tiny$\pm$0.19} & 19.76{\tiny$\pm$0.08} & 24.49{\tiny$\pm$0.27} & 37.24{\tiny$\pm$1.57} \\
    \midrule
    \multirow{2}{*}{1.0} & \multirow{2}{*}{None} &  \multirow{2}{*}{N/A}        & \multirow{2}{*}{N/A}          & $\sqrt{2 \eta_t}$         & $-$0.269{\tiny$\pm$0.050} & 30.41{\tiny$\pm$0.22} & 18.74{\tiny$\pm$0.07} & 20.47{\tiny$\pm$0.18} & 43.82{\tiny$\pm$1.24} \\
     &  &       &                & 0                 & $-$0.123{\tiny$\pm$0.041} & 31.83{\tiny$\pm$0.17} & 19.28{\tiny$\pm$0.07} & 20.95{\tiny$\pm$0.16} & 42.59{\tiny$\pm$1.23}                             
                                    \\
    \midrule
    \multirow{6}{*}{1.0} & \multirow{6}{*}{DRaFT-1} &  \multirow{2}{*}{1000}        & $\sqrt{2 \eta_t}$          & $\sqrt{2 \eta_t}$         & 1.123{\tiny$\pm$0.051} & 32.06{\tiny$\pm$0.19} & 19.69{\tiny$\pm$0.06} & 24.56{\tiny$\pm$0.17} & 28.25{\tiny$\pm$1.55} \\
     &  &       & 0                 & 0                 & 0.856{\tiny$\pm$0.167} & 32.32{\tiny$\pm$0.25} & 19.38{\tiny$\pm$0.34} & 22.88{\tiny$\pm$1.54} & 29.98{\tiny$\pm$1.86} \\
    % \midrule
    \addlinespace
     &  &  2000        & 0                 & 0                 & 1.177{\tiny$\pm$0.053} & 32.36{\tiny$\pm$0.18} & 19.67{\tiny$\pm$0.08} & 24.48{\tiny$\pm$0.28} & 25.09{\tiny$\pm$1.82} \\
    % \midrule
    \addlinespace
    & & 3000 & 0                 & 0                 & 1.255{\tiny$\pm$0.038} & 32.36{\tiny$\pm$0.19} & 19.70{\tiny$\pm$0.06} & 24.64{\tiny$\pm$0.17} & 23.24{\tiny$\pm$1.19} \\
    \addlinespace
    & &  4000  & 0 & 0 & \textbf{1.296{\tiny$\pm$0.033}} & 32.30{\tiny$\pm$0.19} & 19.68{\tiny$\pm$0.06} & 24.71{\tiny$\pm$0.14} & 21.54{\tiny$\pm$0.96} \\
    \midrule
    \multirow{5}{*}{1.0} & \multirow{5}{*}{Adj.-Match.} & 1000 & 0                 & 0                 & 0.782{\tiny$\pm$0.044} & 33.05{\tiny$\pm$0.22} & 20.20{\tiny$\pm$0.09} & 24.81{\tiny$\pm$0.18} & 32.67{\tiny$\pm$1.26} \\
    \addlinespace
    & & \multirow{2}{*}{2500} & $\sqrt{2 \eta_t}$ & $\sqrt{2 \eta_t}$ & 1.027{\tiny$\pm$0.038} & 32.85{\tiny$\pm$0.21} & 20.08{\tiny$\pm$0.08} & \textbf{25.88{\tiny$\pm$0.20}} & 29.83{\tiny$\pm$1.00} \\
    & &  & 0 & 0 & 0.910{\tiny$\pm$0.040} & 33.20{\tiny$\pm$0.17} & 20.29{\tiny$\pm$0.09} & 25.39{\tiny$\pm$0.24} & 30.34{\tiny$\pm$1.51} \\
    \addlinespace
    & & 12500 & 0 & 0 & 0.985{\tiny$\pm$0.041} & 33.10{\tiny$\pm$0.18} & 20.28{\tiny$\pm$0.08} & \textbf{25.61{\tiny$\pm$0.27}} & 28.86{\tiny$\pm$1.37} \\
    \midrule
    \multirow{2}{*}{4.0} & \multirow{2}{*}{None} &  \multirow{2}{*}{N/A}        & \multirow{2}{*}{N/A}          & $\sqrt{2 \eta_t}$         & 0.277{\tiny$\pm$0.043} & 32.68{\tiny$\pm$0.18} & 19.50{\tiny$\pm$0.07} & 22.29{\tiny$\pm$0.16} & 35.12{\tiny$\pm$0.92} \\
     &  &       &                & 0                 & 0.209{\tiny$\pm$0.046} & 32.83{\tiny$\pm$0.17} & 19.79{\tiny$\pm$0.07} & 22.30{\tiny$\pm$0.17} & 32.05{\tiny$\pm$1.05}    
                                    \\
    \midrule
    \multirow{6}{*}{4.0} & \multirow{6}{*}{DRaFT-1} &  \multirow{2}{*}{1000}        & $\sqrt{2 \eta_t}$          & $\sqrt{2 \eta_t}$         & 1.062{\tiny$\pm$0.045} & 32.29{\tiny$\pm$0.16} & 19.48{\tiny$\pm$0.06} & 23.67{\tiny$\pm$0.13} & 25.03{\tiny$\pm$1.32} \\
     &  &       & 0                 & 0                 & 0.604{\tiny$\pm$0.395} & 31.80{\tiny$\pm$0.86} & 19.09{\tiny$\pm$0.53} & 21.69{\tiny$\pm$2.10} & 25.92{\tiny$\pm$2.57} \\
    % \midrule
    \addlinespace
     &  &  2000        & 0                 & 0                 & 1.112{\tiny$\pm$0.046} & 32.29{\tiny$\pm$0.20} & 19.34{\tiny$\pm$0.11} & 23.31{\tiny$\pm$0.22} & 21.02{\tiny$\pm$1.67} \\
    % \midrule
    \addlinespace
    & & 3000 & 0                 & 0                 & 1.151{\tiny$\pm$0.036} & 32.31{\tiny$\pm$0.21} & 19.36{\tiny$\pm$0.06} & 23.29{\tiny$\pm$0.14} & 19.53{\tiny$\pm$1.24} \\
    \addlinespace
    & &  4000  & 0 & 0 & 1.172{\tiny$\pm$0.040} & 32.20{\tiny$\pm$0.22} & 19.30{\tiny$\pm$0.07} & 23.20{\tiny$\pm$0.15} & 18.45{\tiny$\pm$1.06} \\
    \midrule
    \multirow{5}{*}{4.0} & \multirow{5}{*}{Adj.-Match.} & 1000 & 0                 & 0                 & 0.852{\tiny$\pm$0.046} & \textbf{33.50{\tiny$\pm$0.22}} & 20.31{\tiny$\pm$0.08} & 24.97{\tiny$\pm$0.19} & 25.83{\tiny$\pm$0.82} \\
    \addlinespace
    & & \multirow{2}{*}{2500} & $\sqrt{2 \eta_t}$ & $\sqrt{2 \eta_t}$ & 1.052{\tiny$\pm$0.039} & \textbf{33.51}{\tiny$\pm$0.19} & 20.15{\tiny$\pm$0.07} & \textbf{25.56{\tiny$\pm$0.18}} & 26.21{\tiny$\pm$0.73} \\
    & &  & 0 & 0 & 0.942{\tiny$\pm$0.042} & \textbf{33.61}{\tiny$\pm$0.19} & \textbf{20.35}{\tiny$\pm$0.08} & 25.34{\tiny$\pm$0.21} & 24.30{\tiny$\pm$0.86} \\
    \addlinespace
    & & 12500 & 0 & 0 & 1.007{\tiny$\pm$0.052} & \textbf{33.48{\tiny$\pm$0.20}} & 20.29{\tiny$\pm$0.08} & \textbf{25.50{\tiny$\pm$0.29}} & 23.48{\tiny$\pm$0.81} \\
    \bottomrule
\end{tabular}
}
\caption{ 
Evaluation metrics when using classifier-free guidance (CFG; \citet{ho2022classifier}).
}
\end{table}

\begin{table}[h!]
\centering
{\footnotesize
\begin{tabular}{llccccccc}
    \toprule
    LR / & Fine-tuning & Fine-tun. & Generat. & \multirow{2}{*}{ImageReward$\, \uparrow$} & \multirow{2}{*}{ClipScore$\, \uparrow$} & \multirow{2}{*}{PickScore$\, \uparrow$} & \multirow{2}{*}{HPS v2$\, \uparrow$} & DreamSim \\
    Adam $\beta_1$ & loss & $\sigma(t)$ & $\sigma(t)$ & & &  &  & diversity$\, \uparrow$ \\
    \midrule
    \num{3e-5} & DRaFT-1          & $\sqrt{2 \eta_t}$ & $\sqrt{2 \eta_t}$ & 1.467{\tiny$\pm$0.029} & 30.28{\tiny$\pm$0.56} & 19.37{\tiny$\pm$0.09} & 24.70{\tiny$\pm$0.15} & 21.20{\tiny$\pm$0.93} \\
               % &  &  & $\num{3e-5} / 0.97$ & $- \pm -$ & $- \pm -$ & $- \pm -$ & $- \pm -$ \\
    \addlinespace
    / \num{0.97} & Adj.-Match.  & \multirow{2}{*}{$\sqrt{2 \eta_t}$} & \multirow{2}{*}{$\sqrt{2 \eta_t}$} & \multirow{2}{*}{1.130{\tiny$\pm$0.034}} & \multirow{2}{*}{31.01{\tiny$\pm$0.27}} & \multirow{2}{*}{19.60{\tiny$\pm$0.08}} & \multirow{2}{*}{25.01{\tiny$\pm$0.25}} & \multirow{2}{*}{26.73{\tiny$\pm$0.88}} \\
    & $\lambda = 12500$                     &                                    &                 &  &  &  &  &  \\
    \midrule
    \num{2e-5} & Disc. Adj.  & $\sqrt{2 \eta_t}$ & $\sqrt{2 \eta_t}$ & $-$1.186{\tiny$\pm$0.553} & 21.95{\tiny$\pm$4.29} & 16.94{\tiny$\pm$0.95} & 12.34{\tiny$\pm$4.40} & 28.33{\tiny$\pm$10.26} \\
     / \num{0.95} & $\lambda = 12500$                     &  0                                  &  0               & $-$0.961{\tiny$\pm$0.653} & 24.07{\tiny$\pm$4.71} & 17.86{\tiny$\pm$1.17} & 15.93{\tiny$\pm$5.80} & 33.62{\tiny$\pm$7.80} \\
    \bottomrule
\end{tabular}
}
\caption{Metrics for alternative optimization hyperparameters (learning rate and Adam $\beta_1$). 
% learning rate $=\num{3e-5}$ and Adam $\beta_1=$ 0.97, in contrast to the values $\num{2e-5}$ and 0.95 used elsewhere.
}
\label{table:alternative_hyperparameters}
\end{table}

\begin{table}[h!]
\centering
{\footnotesize
\begin{tabular}{lccccccc}
    \toprule
    Fine-tuning & Fine-tuning & Generative & \multirow{2}{*}{ImageReward$\, \uparrow$} & \multirow{2}{*}{ClipScore$\, \uparrow$} & \multirow{2}{*}{PickScore$\, \uparrow$} & \multirow{2}{*}{HPS v2$\, \uparrow$} & DreamSim \\
    loss & $\sigma(t)$ & $\sigma(t)$ & & &  &  & diversity$\, \uparrow$ \\
    \midrule
    Adj.-Matching  & \multirow{2}{*}{1} & 1 & 0.009{\tiny$\pm$0.077} & 29.18{\tiny$\pm$0.51} & 18.66{\tiny$\pm$0.09} & 20.75{\tiny$\pm$0.32} & 41.33{\tiny$\pm$1.24} \\
    $\lambda = 12500$                     &                                    &  0               & 0.454{\tiny$\pm$0.055} & 31.41{\tiny$\pm$0.22} & 19.57{\tiny$\pm$0.09} & 23.29{\tiny$\pm$0.18} & 43.10{\tiny$\pm$1.76} \\
    \addlinespace
    Adj.-Matching  & \multirow{2}{*}{$\sqrt{2 \eta_t}$} & $\sqrt{2 \eta_t}$ & 0.882{\tiny$\pm$0.058} & 30.62{\tiny$\pm$0.30} & 19.50{\tiny$\pm$0.09} & 24.95{\tiny$\pm$0.28} & 34.50{\tiny$\pm$1.33}  \\
    $\lambda = 12500$                    &                                    & 0                 & 0.778{\tiny$\pm$0.050} & 31.65{\tiny$\pm$0.19} & 19.76{\tiny$\pm$0.08} & 24.49{\tiny$\pm$0.27} & 37.24{\tiny$\pm$1.57}  \\
    \bottomrule
\end{tabular}
}
\caption{Comparison with an alternative fine-tuning noise schedule $\sigma(t)=1$. We see that the initial value function bias (\Cref{sec:value_function_bias_problem}) results in the model not having a high reward function (ImageReward is the reward function used for fine-tuning). Its performance on other metrics are also lower than when fine-tuning with the memoryless noise schedule, except for diversity.}
\label{table:alternative_noise_schedule}
\end{table}

\begin{table}[h!]
\centering
{\footnotesize
\begin{tabular}{llccccccc}
    \toprule
    $\#$sampl. & Fine-tuning & Fine-tun. & Sampl. & \multirow{2}{*}{ImageReward$\, \uparrow$} & \multirow{2}{*}{ClipScore$\, \uparrow$} & \multirow{2}{*}{PickScore$\, \uparrow$} & \multirow{2}{*}{HPS v2$\, \uparrow$} & DreamSim \\
    timesteps & loss & $\sigma(t)$ & $\sigma(t)$ &  &  &  &  & diversity$\, \uparrow$ \\
    \midrule
    \multirow{6}{*}{$10$} & \multirow{2}{*}{None (Base)} & \multirow{2}{*}{N/A} & $\sqrt{2 \eta_t}$ & $-$2.279{\tiny$\pm$0.001} & 13.99{\tiny$\pm$0.12} & 14.98{\tiny$\pm$0.05} & 7.37{\tiny$\pm$0.10} & 5.07{\tiny$\pm$0.13} \\
                            &     &                     & 0                 & $-$1.386{\tiny$\pm$0.040} & 26.26{\tiny$\pm$0.24} & 17.64{\tiny$\pm$0.07} & 14.92{\tiny$\pm$0.17} & 51.26{\tiny$\pm$1.38} \\
    \addlinespace
     & \multirow{2}{*}{DRaFT-1}           & $\sqrt{2 \eta_t}$ & $\sqrt{2 \eta_t}$ & 1.033{\tiny$\pm$0.051} & 25.98{\tiny$\pm$0.25} & 18.28{\tiny$\pm$0.07} & 22.08{\tiny$\pm$0.18} & 14.47{\tiny$\pm$0.67} \\
                                    &   & 0                 & 0                 & 1.236{\tiny$\pm$0.038} & 31.54{\tiny$\pm$0.27} & 19.53{\tiny$\pm$0.07} & 24.47{\tiny$\pm$0.19} & 24.78{\tiny$\pm$0.88} \\
    \addlinespace
     & Adj.-Match.  & \multirow{2}{*}{$\sqrt{2 \eta_t}$} & $\sqrt{2 \eta_t}$ & $-$2.104{\tiny$\pm$0.074} & 17.12{\tiny$\pm$0.56} & 15.76{\tiny$\pm$0.20} & 11.48{\tiny$\pm$1.03} & 9.88{\tiny$\pm$0.81} \\
     & $\lambda = 12500$                     &                                    & 0                 & 0.607{\tiny$\pm$0.055} & 31.36{\tiny$\pm$0.20} & 19.56{\tiny$\pm$0.08} & 23.23{\tiny$\pm$0.28} & 33.75{\tiny$\pm$1.48} \\
    \midrule
    \multirow{6}{*}{$20$} & \multirow{2}{*}{None (Base)} & \multirow{2}{*}{N/A} & $\sqrt{2 \eta_t}$ & $-$2.275{\tiny$\pm$0.002} & 14.58{\tiny$\pm$0.13} & 15.07{\tiny$\pm$0.05} & 7.47{\tiny$\pm$0.10} & 11.27{\tiny$\pm$0.33} \\
                         &        &                     & 0                 &  $-$1.017{\tiny$\pm$0.055} & 27.92{\tiny$\pm$0.19} & 18.01{\tiny$\pm$0.07} & 17.17{\tiny$\pm$0.15} & 54.69{\tiny$\pm$1.45} \\
    \addlinespace
     & \multirow{2}{*}{DRaFT-1}           & $\sqrt{2 \eta_t}$ & $\sqrt{2 \eta_t}$ & \textbf{1.301{\tiny$\pm$0.039}} & 27.09{\tiny$\pm$0.24} & 18.93{\tiny$\pm$0.07} & 23.78{\tiny$\pm$0.20} & 21.05{\tiny$\pm$1.12} \\
                                    &   & 0                 & 0                 & 1.255{\tiny$\pm$0.038} & 31.14{\tiny$\pm$0.25} & 19.43{\tiny$\pm$0.06} & 24.52{\tiny$\pm$0.16} & 26.15{\tiny$\pm$1.11} \\
    \addlinespace
     & Adj.-Match.  & \multirow{2}{*}{$\sqrt{2 \eta_t}$} & $\sqrt{2 \eta_t}$ & $-$0.032{\tiny$\pm$0.072} & 25.07{\tiny$\pm$0.27} & 18.01{\tiny$\pm$0.07} & 20.75{\tiny$\pm$0.23} & 29.06{\tiny$\pm$2.34} \\
     & $\lambda = 12500$                     &                                    & 0                 & 0.768{\tiny$\pm$0.048} & \textbf{31.70{\tiny$\pm$0.17}} & \textbf{19.73{\tiny$\pm$0.08}} & 24.30{\tiny$\pm$0.26} & 35.90{\tiny$\pm$1.52} \\
     \midrule
    \multirow{6}{*}{$40$} & \multirow{2}{*}{None (Base)} & \multirow{2}{*}{N/A} & $\sqrt{2 \eta_t}$ & $-$1.384{\tiny$\pm$0.040} & 24.15{\tiny$\pm$0.26} & 17.25{\tiny$\pm$0.06} & 16.19{\tiny$\pm$0.17} & 53.60{\tiny$\pm$1.37} \\
                         &        &                     & 0                 & $-$0.920{\tiny$\pm$0.042} & 28.32{\tiny$\pm$0.22} & 18.15{\tiny$\pm$0.07} & 17.89{\tiny$\pm$0.16} & \textbf{56.53{\tiny$\pm$1.52}} \\
    \addlinespace
     & \multirow{2}{*}{DRaFT-1}           & $\sqrt{2 \eta_t}$ & $\sqrt{2 \eta_t}$ & \textbf{1.357{\tiny$\pm$0.039}} & 30.18{\tiny$\pm$0.24} & 19.38{\tiny$\pm$0.08} & 24.61{\tiny$\pm$0.17} & 25.54{\tiny$\pm$0.99} \\
                                    &   & 0                 & 0                 & 1.251{\tiny$\pm$0.040} & 30.95{\tiny$\pm$0.28} & 19.37{\tiny$\pm$0.06} & 24.37{\tiny$\pm$0.17} & 27.39{\tiny$\pm$1.14} \\
    \addlinespace
     & Adj.-Match.  & \multirow{2}{*}{$\sqrt{2 \eta_t}$} & $\sqrt{2 \eta_t}$ & 0.882{\tiny$\pm$0.058} & 30.62{\tiny$\pm$0.30} & 19.50{\tiny$\pm$0.09} & \textbf{24.95{\tiny$\pm$0.28}} & 34.50{\tiny$\pm$1.33} \\
     & $\lambda = 12500$                     &                                    & 0                 & 0.778{\tiny$\pm$0.050} & \textbf{31.65{\tiny$\pm$0.19}} & \textbf{19.76{\tiny$\pm$0.08}} & 24.49{\tiny$\pm$0.27} & 37.24{\tiny$\pm$1.57} \\
    \midrule
    \multirow{6}{*}{$100$} & \multirow{2}{*}{None (Base)} & \multirow{2}{*}{N/A} & $\sqrt{2 \eta_t}$ & $-$0.881{\tiny$\pm$0.041} & 27.83{\tiny$\pm$0.19} & 18.10{\tiny$\pm$0.07} & 18.43{\tiny$\pm$0.17} & \textbf{57.21{\tiny$\pm$1.50}} \\
                         &        &                     & 0                 & $-$0.881{\tiny$\pm$0.036} & 28.65{\tiny$\pm$0.18} & 18.22{\tiny$\pm$0.06} & 18.20{\tiny$\pm$0.17} & \textbf{57.73{\tiny$\pm$1.68}} \\
    \addlinespace
     & \multirow{2}{*}{DRaFT-1}           & $\sqrt{2 \eta_t}$ & $\sqrt{2 \eta_t}$ & \textbf{1.343{\tiny$\pm$0.040}} & 30.64{\tiny$\pm$0.20} & 19.38{\tiny$\pm$0.08} & 24.37{\tiny$\pm$0.17} & 25.51{\tiny$\pm$1.10} \\
                                    &   & 0                 & 0                 & 1.239{\tiny$\pm$0.037} & 30.74{\tiny$\pm$0.28} & 19.33{\tiny$\pm$0.06} & 24.24{\tiny$\pm$0.17} & 28.70{\tiny$\pm$1.11} \\
    \addlinespace
     & Adj.-Match.  & \multirow{2}{*}{$\sqrt{2 \eta_t}$} & $\sqrt{2 \eta_t}$ & 0.892{\tiny$\pm$0.044} & 31.23{\tiny$\pm$0.23} & \textbf{19.65{\tiny$\pm$0.08}} & \textbf{24.92{\tiny$\pm$0.23}} & 35.13{\tiny$\pm$1.40} \\
     & $\lambda = 12500$                     &                                    & 0                 & 0.779{\tiny$\pm$0.048} & \textbf{31.64{\tiny$\pm$0.17}} & \textbf{19.76{\tiny$\pm$0.08}} & 24.57{\tiny$\pm$0.25} & 38.26{\tiny$\pm$1.65} \\
    \midrule
    \multirow{6}{*}{$200$} & \multirow{2}{*}{None (Base)} & \multirow{2}{*}{N/A} & $\sqrt{2 \eta_t}$ & $-$0.848{\tiny$\pm$0.048} & 28.37{\tiny$\pm$0.21} & 18.27{\tiny$\pm$0.08} & 18.56{\tiny$\pm$0.19} & \textbf{58.00{\tiny$\pm$1.58}} \\
                         &        &                     & 0                 & $-$0.871{\tiny$\pm$0.036} & 28.50{\tiny$\pm$0.18} & 18.23{\tiny$\pm$0.06} & 18.25{\tiny$\pm$0.14} & \textbf{57.84{\tiny$\pm$1.60}} \\
    \addlinespace
     & \multirow{2}{*}{DRaFT-1}           & $\sqrt{2 \eta_t}$ & $\sqrt{2 \eta_t}$ & \textbf{1.331{\tiny$\pm$0.044}} & 30.69{\tiny$\pm$0.23} & 19.36{\tiny$\pm$0.07} & 24.21{\tiny$\pm$0.17} & 26.41{\tiny$\pm$1.18} \\
                                    &   & 0                 & 0                 & 1.222{\tiny$\pm$0.042} & 30.77{\tiny$\pm$0.27} & 19.32{\tiny$\pm$0.06} & 24.18{\tiny$\pm$0.16} & 29.09{\tiny$\pm$1.07} \\
    \addlinespace
     & Adj.-Match.  & \multirow{2}{*}{$\sqrt{2 \eta_t}$} & $\sqrt{2 \eta_t}$ & 0.869{\tiny$\pm$0.062} & 31.33{\tiny$\pm$0.21} & \textbf{19.68{\tiny$\pm$0.09}} & \textbf{24.81{\tiny$\pm$0.30}} & 35.90{\tiny$\pm$1.55} \\
     & $\lambda = 12500$                     &                                    & 0                 & 0.766{\tiny$\pm$0.050} & \textbf{31.61{\tiny$\pm$0.16}} & \textbf{19.75{\tiny$\pm$0.08}} & 24.52{\tiny$\pm$0.24} & 38.60{\tiny$\pm$1.38} \\
    \bottomrule
\end{tabular}
}
\caption{Performance metrics for different number of sampling steps. Only the number of sampling steps is ablated; the fine-tuned models used in all cases are the ones fine-tuned using 40 steps.
}
\label{table:metrics_multiprompt_sampling_steps}
\end{table}

% \begin{table}[h!]
% \centering
% {\footnotesize
% \begin{tabular}{lcccccccc}
%     \toprule
%     Fine-tun. & Fine-tun. & Generat. & \multirow{2}{*}{ImageReward$\, \uparrow$} & \multirow{2}{*}{ClipScore$\, \uparrow$} & \multirow{2}{*}{PickScore$\, \uparrow$} & \multirow{2}{*}{HPS v2$\, \uparrow$} & DreamSim & Runtime/ \\
%     loss & $\sigma(t)$ & $\sigma(t)$ &  & &  &  & diversity$\, \uparrow$ & $\#$iter. \\
%     % \midrule
%     % \multirow{2}{*}{None (Base)} & \multirow{2}{*}{N/A} & $\sqrt{2 \eta_t}$ & $-\! \pm \!-$ & $-\! \pm \!-$ & $-\! \pm \!-$ & $-\! \pm \!-$ & \multirow{2}{*}{N/A} \\
%     %                              &                     & 0                 & $-\! \pm \!-$ & $-\! \pm \!-$ & $-\! \pm \!-$ & $-\! \pm \!-$ &  \\
%     \midrule % \addlinespace
%     \multirow{2}{*}{ReFL}              & $\sqrt{2 \eta_t}$ & $\sqrt{2 \eta_t}$ & 0.459{\tiny$\pm$0.096} & 28.46{\tiny$\pm$0.25} & 18.77{\tiny$\pm$0.09} & 22.54{\tiny$\pm$0.17} & 37.51{\tiny$\pm$3.50} & 43k{\tiny$\pm$2.7k} \\
%                                        & 0                 & 0                 & 0.330{\tiny$\pm$0.114} & 29.63{\tiny$\pm$0.61} & 19.08{\tiny$\pm$0.18} & 22.46{\tiny$\pm$0.77} & 39.51{\tiny$\pm$1.30} & / 1500 \\
%     \addlinespace
%     \multirow{2}{*}{DRaFT-1}           & $\sqrt{2 \eta_t}$ & $\sqrt{2 \eta_t}$ & 0.913{\tiny$\pm$0.068} & 29.80{\tiny$\pm$0.22} & 19.16{\tiny$\pm$0.06} & 23.63{\tiny$\pm$0.16} & 35.21{\tiny$\pm$1.93} & 35k{\tiny$\pm$1.5k} \\
%                                        & 0                 & 0                 & 0.626{\tiny$\pm$0.195} & 30.48{\tiny$\pm$0.32} & 18.91{\tiny$\pm$0.34} & 21.92{\tiny$\pm$1.63} & 38.52{\tiny$\pm$2.01} & / 1000 \\
%     \addlinespace
%     \multirow{2}{*}{Draft-40}          & $\sqrt{2 \eta_t}$ & $\sqrt{2 \eta_t}$ & $-$1.427{\tiny$\pm$0.267} & 23.39{\tiny$\pm$1.72} & 17.24{\tiny$\pm$0.45} & 15.72{\tiny$\pm$1.80} & 41.98{\tiny$\pm$2.14} & 49k{\tiny$\pm$1.4k} \\
%                                        & 0                 & 0                 & $-$0.097{\tiny$\pm$0.052} & 29.12{\tiny$\pm$0.41} & 18.97{\tiny$\pm$0.14} & 21.93{\tiny$\pm$0.20} & 46.35{\tiny$\pm$1.34} & / 500 \\
%     \midrule % \addlinespace
%     Adj.-Match.  & \multirow{2}{*}{$\sqrt{2 \eta_t}$} & $\sqrt{2 \eta_t}$ & 0.107{\tiny$\pm$0.046} & 29.37{\tiny$\pm$0.25} & 19.05{\tiny$\pm$0.07} & 22.79{\tiny$\pm$0.20} & 46.38{\tiny$\pm$1.36} &  \\
%     $\lambda = 1000$                     &                                    & 0                 & 0.051{\tiny$\pm$0.044} & 30.58{\tiny$\pm$0.17} & 19.31{\tiny$\pm$0.07} & 21.93{\tiny$\pm$0.23} & 48.12{\tiny$\pm$1.56} &  \\
%     \addlinespace
%     Adj.-Match. & \multirow{2}{*}{$\sqrt{2 \eta_t}$} & $\sqrt{2 \eta_t}$ & 0.199{\tiny$\pm$0.068} & 29.27{\tiny$\pm$0.21} & 19.07{\tiny$\pm$0.10} & 22.98{\tiny$\pm$0.30} & 45.03{\tiny$\pm$1.61} & 39k{\tiny$\pm$0.5k} \\
%     $\lambda = 2500$                     &                                    & 0                 & 0.106{\tiny$\pm$0.067} & 30.43{\tiny$\pm$0.24} & 19.32{\tiny$\pm$0.11} & 22.16{\tiny$\pm$0.33} & 47.61{\tiny$\pm$1.49} & / 250 \\
%     \addlinespace
%     Adj.-Match.  & \multirow{2}{*}{$\sqrt{2 \eta_t}$} & $\sqrt{2 \eta_t}$ & 0.299{\tiny$\pm$0.095} & 29.61{\tiny$\pm$0.37} & 19.26{\tiny$\pm$0.14} & 23.67{\tiny$\pm$0.27} & 43.36{\tiny$\pm$1.93} &  \\
%     $\lambda = 12500$                    &                                    & 0                 & 0.224{\tiny$\pm$0.051} & 30.70{\tiny$\pm$0.23} & 19.52{\tiny$\pm$0.11} & 22.93{\tiny$\pm$0.21} & 44.62{\tiny$\pm$1.79} &  \\
%     \midrule % \addlinespace
%     Cont. Adj. & \multirow{2}{*}{$\sqrt{2 \eta_t}$} & $\sqrt{2 \eta_t}$ & $-$0.910{\tiny$\pm$0.116} & 26.29{\tiny$\pm$0.44} & 18.06{\tiny$\pm$0.16} & 18.86{\tiny$\pm$0.88} & 51.60{\tiny$\pm$1.97} & 51k{\tiny$\pm$0.3k} \\
%     $\lambda = 12500$                     &                                    & 0                 & $-$0.681{\tiny$\pm$0.051} & 28.50{\tiny$\pm$0.19} & 18.69{\tiny$\pm$0.11} & 19.90{\tiny$\pm$0.50} & 50.87{\tiny$\pm$1.52} & / 250 \\
%     \addlinespace
%     Disc. Adj. & \multirow{2}{*}{$\sqrt{2 \eta_t}$} & $\sqrt{2 \eta_t}$ & $-$0.978{\tiny$\pm$0.123} & 26.68{\tiny$\pm$0.76} & 18.51{\tiny$\pm$0.11} & 18.53{\tiny$\pm$0.28} & 55.95{\tiny$\pm$1.70} & 38k{\tiny$\pm$0.4k} \\
%     $\lambda = 12500$                    &                                    & 0                 & $-$0.791{\tiny$\pm$0.065} & 28.66{\tiny$\pm$0.33} & 18.51{\tiny$\pm$0.11} & 18.53{\tiny$\pm$0.28} & 54.78{\tiny$\pm$2.00} & / 250 \\
%     \bottomrule
% \end{tabular}
% }
% \caption{Additional metrics for various fine-tuning methods for text-to-image generation, which complement the ones in \Cref{tab:evaluation_metrics} (both tables correspond to the same runs). The second and third columns show the noise schedules $\sigma(t)$ used for fine-tuning and for inference: $\sigma(t) = \sqrt{2\eta_t}$ corresponds to Memoryless Flow Matching, 
% % \eqref{eq:memoryless_FM_sde}, 
% and $\sigma(t) = 0$ to the Flow Matching ODE \eqref{eq:FM_ode}.
% }
% \label{table:metrics_multiprompt_alternative}
% \end{table}

\begin{figure}[h!]
    \centering
    \begin{subfigure}[t]{0.49\linewidth}
        \centering
        \rotatebox{90}{\;\;\; $w=0.0$}\,%
        \includegraphics[width=0.24\linewidth]{figs/samples/cfg0.0_prompt_37_image_3.jpg}\,%
        \includegraphics[width=0.24\linewidth]{figs/samples/cfg0.0_prompt_37_image_5.jpg}\,%
        \includegraphics[width=0.24\linewidth]{figs/samples/cfg0.0_prompt_37_image_6.jpg}\,%
        \includegraphics[width=0.24\linewidth]{figs/samples/cfg0.0_prompt_37_image_7.jpg}\\
        \rotatebox{90}{\;\;\; $w=1.0$}\,%
        \includegraphics[width=0.24\linewidth]{figs/samples/cfg1.0_prompt_37_image_3.jpg}\,%
        \includegraphics[width=0.24\linewidth]{figs/samples/cfg1.0_prompt_37_image_5.jpg}\,%
        \includegraphics[width=0.24\linewidth]{figs/samples/cfg1.0_prompt_37_image_6.jpg}\,%
        \includegraphics[width=0.24\linewidth]{figs/samples/cfg1.0_prompt_37_image_7.jpg}\\
        \rotatebox{90}{\;\;\; $w=4.0$}\,%
        \includegraphics[width=0.24\linewidth]{figs/samples/cfg4.0_prompt_37_image_3.jpg}\,%
        \includegraphics[width=0.24\linewidth]{figs/samples/cfg4.0_prompt_37_image_5.jpg}\,%
        \includegraphics[width=0.24\linewidth]{figs/samples/cfg4.0_prompt_37_image_6.jpg}\,%
        \includegraphics[width=0.24\linewidth]{figs/samples/cfg4.0_prompt_37_image_7.jpg}
        \caption*{Text prompt: ``\textit{Man sitting on sofa at home in front of fireplace and using laptop computer, rear view}''}
    \end{subfigure}\hfill
    \begin{subfigure}[t]{0.49\linewidth}
        \centering
        \includegraphics[width=0.24\linewidth]{figs/samples/cfg0.0_prompt_74_image_6.jpg}\,%
        \includegraphics[width=0.24\linewidth]{figs/samples/cfg0.0_prompt_74_image_7.jpg}\,%
        \includegraphics[width=0.24\linewidth]{figs/samples/cfg0.0_prompt_74_image_8.jpg}\,%
        \includegraphics[width=0.24\linewidth]{figs/samples/cfg0.0_prompt_74_image_9.jpg}\\
        \includegraphics[width=0.24\linewidth]{figs/samples/cfg1.0_prompt_74_image_6.jpg}\,%
        \includegraphics[width=0.24\linewidth]{figs/samples/cfg1.0_prompt_74_image_7.jpg}\,%
        \includegraphics[width=0.24\linewidth]{figs/samples/cfg1.0_prompt_74_image_8.jpg}\,%
        \includegraphics[width=0.24\linewidth]{figs/samples/cfg1.0_prompt_74_image_9.jpg}\\
        \includegraphics[width=0.24\linewidth]{figs/samples/cfg4.0_prompt_74_image_6.jpg}\,%
        \includegraphics[width=0.24\linewidth]{figs/samples/cfg4.0_prompt_74_image_7.jpg}\,%
        \includegraphics[width=0.24\linewidth]{figs/samples/cfg4.0_prompt_74_image_8.jpg}\,%
        \includegraphics[width=0.24\linewidth]{figs/samples/cfg4.0_prompt_74_image_9.jpg}
        \caption*{Text prompt: ``\textit{3D World Food Day Morocco}''}
    \end{subfigure}
    \caption{
    Generated samples from varying classifier-free guidance weights, from the pre-trained Flow Matching model. 
    Corresponding samples from the fine-tuned model can be found in \Cref{fig:ablation_tradeoff_cfg}. 
    }
    \label{fig:ablation_tradeoff_cfg_base}
\end{figure}


\begin{figure}[h!]
    \centering
    \begin{subfigure}[t]{0.32\linewidth}
    \centering
    \caption*{Base Flow Matching model}
	\includegraphics[width=0.320\linewidth]{figs/samples_appendix_3/base_cfg_2_ode_prompt_6_image_1.jpg}\;%
	\includegraphics[width=0.320\linewidth]{figs/samples_appendix_3/base_cfg_2_ode_prompt_6_image_2.jpg}\;%
	\includegraphics[width=0.320\linewidth]{figs/samples_appendix_3/base_cfg_2_ode_prompt_6_image_3.jpg}\\ 
	\includegraphics[width=0.320\linewidth]{figs/samples_appendix_3/base_cfg_2_ode_prompt_14_image_1.jpg}\;%
	\includegraphics[width=0.320\linewidth]{figs/samples_appendix_3/base_cfg_2_ode_prompt_14_image_2.jpg}\;%
	\includegraphics[width=0.320\linewidth]{figs/samples_appendix_3/base_cfg_2_ode_prompt_14_image_3.jpg}\\ 
	\includegraphics[width=0.320\linewidth]{figs/samples_appendix_3/base_cfg_2_ode_prompt_19_image_1.jpg}\;%
	\includegraphics[width=0.320\linewidth]{figs/samples_appendix_3/base_cfg_2_ode_prompt_19_image_2.jpg}\;%
	\includegraphics[width=0.320\linewidth]{figs/samples_appendix_3/base_cfg_2_ode_prompt_19_image_3.jpg}\\ 
	\includegraphics[width=0.320\linewidth]{figs/samples_appendix_3/base_cfg_2_ode_prompt_28_image_1.jpg}\;%
	\includegraphics[width=0.320\linewidth]{figs/samples_appendix_3/base_cfg_2_ode_prompt_28_image_2.jpg}\;%
	\includegraphics[width=0.320\linewidth]{figs/samples_appendix_3/base_cfg_2_ode_prompt_28_image_3.jpg}\\ 
	\includegraphics[width=0.320\linewidth]{figs/samples_appendix_3/base_cfg_2_ode_prompt_34_image_1.jpg}\;%
	\includegraphics[width=0.320\linewidth]{figs/samples_appendix_3/base_cfg_2_ode_prompt_34_image_2.jpg}\;%
	\includegraphics[width=0.320\linewidth]{figs/samples_appendix_3/base_cfg_2_ode_prompt_34_image_3.jpg}\\ 
	\includegraphics[width=0.320\linewidth]{figs/samples_appendix_3/base_cfg_2_ode_prompt_65_image_1.jpg}\;%
	\includegraphics[width=0.320\linewidth]{figs/samples_appendix_3/base_cfg_2_ode_prompt_65_image_2.jpg}\;%
	\includegraphics[width=0.320\linewidth]{figs/samples_appendix_3/base_cfg_2_ode_prompt_65_image_3.jpg}\\ 
	\includegraphics[width=0.320\linewidth]{figs/samples_appendix_3/base_cfg_2_ode_prompt_69_image_1.jpg}\;%
	\includegraphics[width=0.320\linewidth]{figs/samples_appendix_3/base_cfg_2_ode_prompt_69_image_2.jpg}\;%
	\includegraphics[width=0.320\linewidth]{figs/samples_appendix_3/base_cfg_2_ode_prompt_69_image_3.jpg}\\ 
	\includegraphics[width=0.320\linewidth]{figs/samples_appendix_3/base_cfg_2_ode_prompt_73_image_1.jpg}\;%
	\includegraphics[width=0.320\linewidth]{figs/samples_appendix_3/base_cfg_2_ode_prompt_73_image_2.jpg}\;%
	\includegraphics[width=0.320\linewidth]{figs/samples_appendix_3/base_cfg_2_ode_prompt_73_image_3.jpg}\\ 
	\includegraphics[width=0.320\linewidth]{figs/samples_appendix_3/base_cfg_2_ode_prompt_75_image_1.jpg}\;%
	\includegraphics[width=0.320\linewidth]{figs/samples_appendix_3/base_cfg_2_ode_prompt_75_image_2.jpg}\;%
	\includegraphics[width=0.320\linewidth]{figs/samples_appendix_3/base_cfg_2_ode_prompt_75_image_3.jpg}\\ 
	\includegraphics[width=0.320\linewidth]{figs/samples_appendix_3/base_cfg_2_ode_prompt_90_image_1.jpg}\;%
	\includegraphics[width=0.320\linewidth]{figs/samples_appendix_3/base_cfg_2_ode_prompt_90_image_2.jpg}\;%
	\includegraphics[width=0.320\linewidth]{figs/samples_appendix_3/base_cfg_2_ode_prompt_90_image_3.jpg}
    \end{subfigure}\hfill
    \begin{subfigure}[t]{0.32\linewidth}
    \centering
    \caption*{Adjoint Matching (Ours)}
	\includegraphics[width=0.320\linewidth]{figs/samples_appendix_3/adjmat_cfg_2_ode_prompt_6_image_1.jpg}\;%
	\includegraphics[width=0.320\linewidth]{figs/samples_appendix_3/adjmat_cfg_2_ode_prompt_6_image_2.jpg}\;%
	\includegraphics[width=0.320\linewidth]{figs/samples_appendix_3/adjmat_cfg_2_ode_prompt_6_image_3.jpg}\\ 
	\includegraphics[width=0.320\linewidth]{figs/samples_appendix_3/adjmat_cfg_2_ode_prompt_14_image_1.jpg}\;%
	\includegraphics[width=0.320\linewidth]{figs/samples_appendix_3/adjmat_cfg_2_ode_prompt_14_image_2.jpg}\;%
	\includegraphics[width=0.320\linewidth]{figs/samples_appendix_3/adjmat_cfg_2_ode_prompt_14_image_3.jpg}\\ 
	\includegraphics[width=0.320\linewidth]{figs/samples_appendix_3/adjmat_cfg_2_ode_prompt_19_image_1.jpg}\;%
	\includegraphics[width=0.320\linewidth]{figs/samples_appendix_3/adjmat_cfg_2_ode_prompt_19_image_2.jpg}\;%
	\includegraphics[width=0.320\linewidth]{figs/samples_appendix_3/adjmat_cfg_2_ode_prompt_19_image_3.jpg}\\ 
	\includegraphics[width=0.320\linewidth]{figs/samples_appendix_3/adjmat_cfg_2_ode_prompt_28_image_1.jpg}\;%
	\includegraphics[width=0.320\linewidth]{figs/samples_appendix_3/adjmat_cfg_2_ode_prompt_28_image_2.jpg}\;%
	\includegraphics[width=0.320\linewidth]{figs/samples_appendix_3/adjmat_cfg_2_ode_prompt_28_image_3.jpg}\\ 
	\includegraphics[width=0.320\linewidth]{figs/samples_appendix_3/adjmat_cfg_2_ode_prompt_34_image_1.jpg}\;%
	\includegraphics[width=0.320\linewidth]{figs/samples_appendix_3/adjmat_cfg_2_ode_prompt_34_image_2.jpg}\;%
	\includegraphics[width=0.320\linewidth]{figs/samples_appendix_3/adjmat_cfg_2_ode_prompt_34_image_3.jpg}\\ 
	\includegraphics[width=0.320\linewidth]{figs/samples_appendix_3/adjmat_cfg_2_ode_prompt_65_image_1.jpg}\;%
	\includegraphics[width=0.320\linewidth]{figs/samples_appendix_3/adjmat_cfg_2_ode_prompt_65_image_2.jpg}\;%
	\includegraphics[width=0.320\linewidth]{figs/samples_appendix_3/adjmat_cfg_2_ode_prompt_65_image_3.jpg}\\ 
	\includegraphics[width=0.320\linewidth]{figs/samples_appendix_3/adjmat_cfg_2_ode_prompt_69_image_1.jpg}\;%
	\includegraphics[width=0.320\linewidth]{figs/samples_appendix_3/adjmat_cfg_2_ode_prompt_69_image_2.jpg}\;%
	\includegraphics[width=0.320\linewidth]{figs/samples_appendix_3/adjmat_cfg_2_ode_prompt_69_image_3.jpg}\\ 
	\includegraphics[width=0.320\linewidth]{figs/samples_appendix_3/adjmat_cfg_2_ode_prompt_73_image_1.jpg}\;%
	\includegraphics[width=0.320\linewidth]{figs/samples_appendix_3/adjmat_cfg_2_ode_prompt_73_image_2.jpg}\;%
	\includegraphics[width=0.320\linewidth]{figs/samples_appendix_3/adjmat_cfg_2_ode_prompt_73_image_3.jpg}\\ 
	\includegraphics[width=0.320\linewidth]{figs/samples_appendix_3/adjmat_cfg_2_ode_prompt_75_image_1.jpg}\;%
	\includegraphics[width=0.320\linewidth]{figs/samples_appendix_3/adjmat_cfg_2_ode_prompt_75_image_2.jpg}\;%
	\includegraphics[width=0.320\linewidth]{figs/samples_appendix_3/adjmat_cfg_2_ode_prompt_75_image_3.jpg}\\ 
	\includegraphics[width=0.320\linewidth]{figs/samples_appendix_3/adjmat_cfg_2_ode_prompt_90_image_1.jpg}\;%
	\includegraphics[width=0.320\linewidth]{figs/samples_appendix_3/adjmat_cfg_2_ode_prompt_90_image_2.jpg}\;%
	\includegraphics[width=0.320\linewidth]{figs/samples_appendix_3/adjmat_cfg_2_ode_prompt_90_image_3.jpg}
    \end{subfigure}\hfill
    \begin{subfigure}[t]{0.32\linewidth}
    \centering
    \caption*{DRaFT-1}
	\includegraphics[width=0.320\linewidth]{figs/samples_appendix_3/draft1k_cfg_2_ode_prompt_6_image_1.jpg}\;%
	\includegraphics[width=0.320\linewidth]{figs/samples_appendix_3/draft1k_cfg_2_ode_prompt_6_image_2.jpg}\;%
	\includegraphics[width=0.320\linewidth]{figs/samples_appendix_3/draft1k_cfg_2_ode_prompt_6_image_3.jpg}\\ 
	\includegraphics[width=0.320\linewidth]{figs/samples_appendix_3/draft1k_cfg_2_ode_prompt_14_image_1.jpg}\;%
	\includegraphics[width=0.320\linewidth]{figs/samples_appendix_3/draft1k_cfg_2_ode_prompt_14_image_2.jpg}\;%
	\includegraphics[width=0.320\linewidth]{figs/samples_appendix_3/draft1k_cfg_2_ode_prompt_14_image_3.jpg}\\ 
	\includegraphics[width=0.320\linewidth]{figs/samples_appendix_3/draft1k_cfg_2_ode_prompt_19_image_1.jpg}\;%
	\includegraphics[width=0.320\linewidth]{figs/samples_appendix_3/draft1k_cfg_2_ode_prompt_19_image_2.jpg}\;%
	\includegraphics[width=0.320\linewidth]{figs/samples_appendix_3/draft1k_cfg_2_ode_prompt_19_image_3.jpg}\\ 
	\includegraphics[width=0.320\linewidth]{figs/samples_appendix_3/draft1k_cfg_2_ode_prompt_28_image_1.jpg}\;%
	\includegraphics[width=0.320\linewidth]{figs/samples_appendix_3/draft1k_cfg_2_ode_prompt_28_image_2.jpg}\;%
	\includegraphics[width=0.320\linewidth]{figs/samples_appendix_3/draft1k_cfg_2_ode_prompt_28_image_3.jpg}\\ 
	\includegraphics[width=0.320\linewidth]{figs/samples_appendix_3/draft1k_cfg_2_ode_prompt_34_image_1.jpg}\;%
	\includegraphics[width=0.320\linewidth]{figs/samples_appendix_3/draft1k_cfg_2_ode_prompt_34_image_2.jpg}\;%
	\includegraphics[width=0.320\linewidth]{figs/samples_appendix_3/draft1k_cfg_2_ode_prompt_34_image_3.jpg}\\ 
	\includegraphics[width=0.320\linewidth]{figs/samples_appendix_3/draft1k_cfg_2_ode_prompt_65_image_1.jpg}\;%
	\includegraphics[width=0.320\linewidth]{figs/samples_appendix_3/draft1k_cfg_2_ode_prompt_65_image_2.jpg}\;%
	\includegraphics[width=0.320\linewidth]{figs/samples_appendix_3/draft1k_cfg_2_ode_prompt_65_image_3.jpg}\\ 
	\includegraphics[width=0.320\linewidth]{figs/samples_appendix_3/draft1k_cfg_2_ode_prompt_69_image_1.jpg}\;%
	\includegraphics[width=0.320\linewidth]{figs/samples_appendix_3/draft1k_cfg_2_ode_prompt_69_image_2.jpg}\;%
	\includegraphics[width=0.320\linewidth]{figs/samples_appendix_3/draft1k_cfg_2_ode_prompt_69_image_3.jpg}\\ 
	\includegraphics[width=0.320\linewidth]{figs/samples_appendix_3/draft1k_cfg_2_ode_prompt_73_image_1.jpg}\;%
	\includegraphics[width=0.320\linewidth]{figs/samples_appendix_3/draft1k_cfg_2_ode_prompt_73_image_2.jpg}\;%
	\includegraphics[width=0.320\linewidth]{figs/samples_appendix_3/draft1k_cfg_2_ode_prompt_73_image_3.jpg}\\ 
	\includegraphics[width=0.320\linewidth]{figs/samples_appendix_3/draft1k_cfg_2_ode_prompt_75_image_1.jpg}\;%
	\includegraphics[width=0.320\linewidth]{figs/samples_appendix_3/draft1k_cfg_2_ode_prompt_75_image_2.jpg}\;%
	\includegraphics[width=0.320\linewidth]{figs/samples_appendix_3/draft1k_cfg_2_ode_prompt_75_image_3.jpg}\\ 
	\includegraphics[width=0.320\linewidth]{figs/samples_appendix_3/draft1k_cfg_2_ode_prompt_90_image_1.jpg}\;%
	\includegraphics[width=0.320\linewidth]{figs/samples_appendix_3/draft1k_cfg_2_ode_prompt_90_image_2.jpg}\;%
	\includegraphics[width=0.320\linewidth]{figs/samples_appendix_3/draft1k_cfg_2_ode_prompt_90_image_3.jpg}
    \end{subfigure}
    \caption{
    Generated samples with classifier-free guidance ($w=1$) and $\sigma(t)=0$ across ten selected prompts.  Each row corresponds to a different prompt and each image corresponds to a different random seed consistent across models.
    }
    \label{fig:image_comparison_v2}
\end{figure}

\begin{figure}[h!]
\centering
    \begin{subfigure}[t]{0.32\linewidth}
    \centering
    \caption*{Base Flow Matching model}
    	\includegraphics[width=0.32\linewidth]{figs/samples_appendix_4/base_cfg_2_ode_prompt_12_image_0.jpg}\;%
    	\includegraphics[width=0.32\linewidth]{figs/samples_appendix_4/base_cfg_2_ode_prompt_12_image_1.jpg}\;%
    	\includegraphics[width=0.32\linewidth]{figs/samples_appendix_4/base_cfg_2_ode_prompt_12_image_2.jpg}\\ 
    	\includegraphics[width=0.32\linewidth]{figs/samples_appendix_4/base_cfg_2_ode_prompt_16_image_0.jpg}\;%
    	\includegraphics[width=0.32\linewidth]{figs/samples_appendix_4/base_cfg_2_ode_prompt_16_image_1.jpg}\;%
    	\includegraphics[width=0.32\linewidth]{figs/samples_appendix_4/base_cfg_2_ode_prompt_16_image_2.jpg}\\ 
    	\includegraphics[width=0.32\linewidth]{figs/samples_appendix_4/base_cfg_2_ode_prompt_17_image_0.jpg}\;%
    	\includegraphics[width=0.32\linewidth]{figs/samples_appendix_4/base_cfg_2_ode_prompt_17_image_1.jpg}\;%
    	\includegraphics[width=0.32\linewidth]{figs/samples_appendix_4/base_cfg_2_ode_prompt_17_image_2.jpg}\\ 
    	\includegraphics[width=0.32\linewidth]{figs/samples_appendix_4/base_cfg_2_ode_prompt_21_image_0.jpg}\;%
    	\includegraphics[width=0.32\linewidth]{figs/samples_appendix_4/base_cfg_2_ode_prompt_21_image_1.jpg}\;%
    	\includegraphics[width=0.32\linewidth]{figs/samples_appendix_4/base_cfg_2_ode_prompt_21_image_2.jpg}\\ 
    	\includegraphics[width=0.32\linewidth]{figs/samples_appendix_4/base_cfg_2_ode_prompt_33_image_0.jpg}\;%
    	\includegraphics[width=0.32\linewidth]{figs/samples_appendix_4/base_cfg_2_ode_prompt_33_image_1.jpg}\;%
    	\includegraphics[width=0.32\linewidth]{figs/samples_appendix_4/base_cfg_2_ode_prompt_33_image_2.jpg}\\ 
    	\includegraphics[width=0.32\linewidth]{figs/samples_appendix_4/base_cfg_2_ode_prompt_35_image_0.jpg}\;%
    	\includegraphics[width=0.32\linewidth]{figs/samples_appendix_4/base_cfg_2_ode_prompt_35_image_1.jpg}\;%
    	\includegraphics[width=0.32\linewidth]{figs/samples_appendix_4/base_cfg_2_ode_prompt_35_image_2.jpg}\\ 
    	\includegraphics[width=0.32\linewidth]{figs/samples_appendix_4/base_cfg_2_ode_prompt_51_image_0.jpg}\;%
    	\includegraphics[width=0.32\linewidth]{figs/samples_appendix_4/base_cfg_2_ode_prompt_51_image_1.jpg}\;%
    	\includegraphics[width=0.32\linewidth]{figs/samples_appendix_4/base_cfg_2_ode_prompt_51_image_2.jpg}\\ 
    	\includegraphics[width=0.32\linewidth]{figs/samples_appendix_4/base_cfg_2_ode_prompt_58_image_0.jpg}\;%
    	\includegraphics[width=0.32\linewidth]{figs/samples_appendix_4/base_cfg_2_ode_prompt_58_image_1.jpg}\;%
    	\includegraphics[width=0.32\linewidth]{figs/samples_appendix_4/base_cfg_2_ode_prompt_58_image_2.jpg}\\ 
    	\includegraphics[width=0.32\linewidth]{figs/samples_appendix_4/base_cfg_2_ode_prompt_61_image_0.jpg}\;%
    	\includegraphics[width=0.32\linewidth]{figs/samples_appendix_4/base_cfg_2_ode_prompt_61_image_1.jpg}\;%
    	\includegraphics[width=0.32\linewidth]{figs/samples_appendix_4/base_cfg_2_ode_prompt_61_image_2.jpg}\\ 
    	\includegraphics[width=0.32\linewidth]{figs/samples_appendix_4/base_cfg_2_ode_prompt_63_image_0.jpg}\;%
    	\includegraphics[width=0.32\linewidth]{figs/samples_appendix_4/base_cfg_2_ode_prompt_63_image_1.jpg}\;%
    	\includegraphics[width=0.32\linewidth]{figs/samples_appendix_4/base_cfg_2_ode_prompt_63_image_2.jpg}
    \end{subfigure}\hfill
    \begin{subfigure}[t]{0.32\linewidth}
    \centering
    \caption*{Adjoint Matching (Ours)}
    	\includegraphics[width=0.32\linewidth]{figs/samples_appendix_4/adjmat_cfg_2_ode_prompt_12_image_0.jpg}\;%
    	\includegraphics[width=0.32\linewidth]{figs/samples_appendix_4/adjmat_cfg_2_ode_prompt_12_image_1.jpg}\;%
    	\includegraphics[width=0.32\linewidth]{figs/samples_appendix_4/adjmat_cfg_2_ode_prompt_12_image_2.jpg}\\ 
    	\includegraphics[width=0.32\linewidth]{figs/samples_appendix_4/adjmat_cfg_2_ode_prompt_16_image_0.jpg}\;%
    	\includegraphics[width=0.32\linewidth]{figs/samples_appendix_4/adjmat_cfg_2_ode_prompt_16_image_1.jpg}\;%
    	\includegraphics[width=0.32\linewidth]{figs/samples_appendix_4/adjmat_cfg_2_ode_prompt_16_image_2.jpg}\\ 
    	\includegraphics[width=0.32\linewidth]{figs/samples_appendix_4/adjmat_cfg_2_ode_prompt_17_image_0.jpg}\;%
    	\includegraphics[width=0.32\linewidth]{figs/samples_appendix_4/adjmat_cfg_2_ode_prompt_17_image_1.jpg}\;%
    	\includegraphics[width=0.32\linewidth]{figs/samples_appendix_4/adjmat_cfg_2_ode_prompt_17_image_2.jpg}\\ 
    	\includegraphics[width=0.32\linewidth]{figs/samples_appendix_4/adjmat_cfg_2_ode_prompt_21_image_0.jpg}\;%
    	\includegraphics[width=0.32\linewidth]{figs/samples_appendix_4/adjmat_cfg_2_ode_prompt_21_image_1.jpg}\;%
    	\includegraphics[width=0.32\linewidth]{figs/samples_appendix_4/adjmat_cfg_2_ode_prompt_21_image_2.jpg}\\ 
    	\includegraphics[width=0.32\linewidth]{figs/samples_appendix_4/adjmat_cfg_2_ode_prompt_33_image_0.jpg}\;%
    	\includegraphics[width=0.32\linewidth]{figs/samples_appendix_4/adjmat_cfg_2_ode_prompt_33_image_1.jpg}\;%
    	\includegraphics[width=0.32\linewidth]{figs/samples_appendix_4/adjmat_cfg_2_ode_prompt_33_image_2.jpg}\\ 
    	\includegraphics[width=0.32\linewidth]{figs/samples_appendix_4/adjmat_cfg_2_ode_prompt_35_image_0.jpg}\;%
    	\includegraphics[width=0.32\linewidth]{figs/samples_appendix_4/adjmat_cfg_2_ode_prompt_35_image_1.jpg}\;%
    	\includegraphics[width=0.32\linewidth]{figs/samples_appendix_4/adjmat_cfg_2_ode_prompt_35_image_2.jpg}\\ 
    	\includegraphics[width=0.32\linewidth]{figs/samples_appendix_4/adjmat_cfg_2_ode_prompt_51_image_0.jpg}\;%
    	\includegraphics[width=0.32\linewidth]{figs/samples_appendix_4/adjmat_cfg_2_ode_prompt_51_image_1.jpg}\;%
    	\includegraphics[width=0.32\linewidth]{figs/samples_appendix_4/adjmat_cfg_2_ode_prompt_51_image_2.jpg}\\ 
    	\includegraphics[width=0.32\linewidth]{figs/samples_appendix_4/adjmat_cfg_2_ode_prompt_58_image_0.jpg}\;%
    	\includegraphics[width=0.32\linewidth]{figs/samples_appendix_4/adjmat_cfg_2_ode_prompt_58_image_1.jpg}\;%
    	\includegraphics[width=0.32\linewidth]{figs/samples_appendix_4/adjmat_cfg_2_ode_prompt_58_image_2.jpg}\\ 
    	\includegraphics[width=0.32\linewidth]{figs/samples_appendix_4/adjmat_cfg_2_ode_prompt_61_image_0.jpg}\;%
    	\includegraphics[width=0.32\linewidth]{figs/samples_appendix_4/adjmat_cfg_2_ode_prompt_61_image_1.jpg}\;%
    	\includegraphics[width=0.32\linewidth]{figs/samples_appendix_4/adjmat_cfg_2_ode_prompt_61_image_2.jpg}\\ 
    	\includegraphics[width=0.32\linewidth]{figs/samples_appendix_4/adjmat_cfg_2_ode_prompt_63_image_0.jpg}\;%
    	\includegraphics[width=0.32\linewidth]{figs/samples_appendix_4/adjmat_cfg_2_ode_prompt_63_image_1.jpg}\;%
    	\includegraphics[width=0.32\linewidth]{figs/samples_appendix_4/adjmat_cfg_2_ode_prompt_63_image_2.jpg}
    \end{subfigure}\hfill
    \begin{subfigure}[t]{0.32\linewidth}
    \centering
    \caption*{DRaFT-1}
    	\includegraphics[width=0.32\linewidth]{figs/samples_appendix_4/draft1k_cfg_2_ode_prompt_12_image_0.jpg}\;%
    	\includegraphics[width=0.32\linewidth]{figs/samples_appendix_4/draft1k_cfg_2_ode_prompt_12_image_1.jpg}\;%
    	\includegraphics[width=0.32\linewidth]{figs/samples_appendix_4/draft1k_cfg_2_ode_prompt_12_image_2.jpg}\\ 
    	\includegraphics[width=0.32\linewidth]{figs/samples_appendix_4/draft1k_cfg_2_ode_prompt_16_image_0.jpg}\;%
    	\includegraphics[width=0.32\linewidth]{figs/samples_appendix_4/draft1k_cfg_2_ode_prompt_16_image_1.jpg}\;%
    	\includegraphics[width=0.32\linewidth]{figs/samples_appendix_4/draft1k_cfg_2_ode_prompt_16_image_2.jpg}\\ 
    	\includegraphics[width=0.32\linewidth]{figs/samples_appendix_4/draft1k_cfg_2_ode_prompt_17_image_0.jpg}\;%
    	\includegraphics[width=0.32\linewidth]{figs/samples_appendix_4/draft1k_cfg_2_ode_prompt_17_image_1.jpg}\;%
    	\includegraphics[width=0.32\linewidth]{figs/samples_appendix_4/draft1k_cfg_2_ode_prompt_17_image_2.jpg}\\ 
    	\includegraphics[width=0.32\linewidth]{figs/samples_appendix_4/draft1k_cfg_2_ode_prompt_21_image_0.jpg}\;%
    	\includegraphics[width=0.32\linewidth]{figs/samples_appendix_4/draft1k_cfg_2_ode_prompt_21_image_1.jpg}\;%
    	\includegraphics[width=0.32\linewidth]{figs/samples_appendix_4/draft1k_cfg_2_ode_prompt_21_image_2.jpg}\\ 
    	\includegraphics[width=0.32\linewidth]{figs/samples_appendix_4/draft1k_cfg_2_ode_prompt_33_image_0.jpg}\;%
    	\includegraphics[width=0.32\linewidth]{figs/samples_appendix_4/draft1k_cfg_2_ode_prompt_33_image_1.jpg}\;%
    	\includegraphics[width=0.32\linewidth]{figs/samples_appendix_4/draft1k_cfg_2_ode_prompt_33_image_2.jpg}\\ 
    	\includegraphics[width=0.32\linewidth]{figs/samples_appendix_4/draft1k_cfg_2_ode_prompt_35_image_0.jpg}\;%
    	\includegraphics[width=0.32\linewidth]{figs/samples_appendix_4/draft1k_cfg_2_ode_prompt_35_image_1.jpg}\;%
    	\includegraphics[width=0.32\linewidth]{figs/samples_appendix_4/draft1k_cfg_2_ode_prompt_35_image_2.jpg}\\ 
    	\includegraphics[width=0.32\linewidth]{figs/samples_appendix_4/draft1k_cfg_2_ode_prompt_51_image_0.jpg}\;%
    	\includegraphics[width=0.32\linewidth]{figs/samples_appendix_4/draft1k_cfg_2_ode_prompt_51_image_1.jpg}\;%
    	\includegraphics[width=0.32\linewidth]{figs/samples_appendix_4/draft1k_cfg_2_ode_prompt_51_image_2.jpg}\\ 
    	\includegraphics[width=0.32\linewidth]{figs/samples_appendix_4/draft1k_cfg_2_ode_prompt_58_image_0.jpg}\;%
    	\includegraphics[width=0.32\linewidth]{figs/samples_appendix_4/draft1k_cfg_2_ode_prompt_58_image_1.jpg}\;%
    	\includegraphics[width=0.32\linewidth]{figs/samples_appendix_4/draft1k_cfg_2_ode_prompt_58_image_2.jpg}\\ 
    	\includegraphics[width=0.32\linewidth]{figs/samples_appendix_4/draft1k_cfg_2_ode_prompt_61_image_0.jpg}\;%
    	\includegraphics[width=0.32\linewidth]{figs/samples_appendix_4/draft1k_cfg_2_ode_prompt_61_image_1.jpg}\;%
    	\includegraphics[width=0.32\linewidth]{figs/samples_appendix_4/draft1k_cfg_2_ode_prompt_61_image_2.jpg}\\ 
    	\includegraphics[width=0.32\linewidth]{figs/samples_appendix_4/draft1k_cfg_2_ode_prompt_63_image_0.jpg}\;%
    	\includegraphics[width=0.32\linewidth]{figs/samples_appendix_4/draft1k_cfg_2_ode_prompt_63_image_1.jpg}\;%
    	\includegraphics[width=0.32\linewidth]{figs/samples_appendix_4/draft1k_cfg_2_ode_prompt_63_image_2.jpg}
    \end{subfigure}
    \caption{
    Generated samples with classifier-free guidance ($w=1$) and $\sigma(t) = 0$ across ten selected prompts with people.  Each row corresponds to a different prompt and each image corresponds to a different random seed consistent across models.
    }
    \label{fig:image_comparison_people}
\end{figure}

\begin{figure}[!htb]
    \centering
    \begin{tabular}{>{\centering\arraybackslash}m{0.3cm} m{0.96\linewidth}}
        \rotatebox{90}{\parbox{2cm}{\centering\footnotesize None (Base)}} &
        \includegraphics[width=0.135\linewidth]{figs/samples_appendix_5/none_ode/prompt_606.jpg}
    	\includegraphics[width=0.135\linewidth]{figs/samples_appendix_5/none_ode/prompt_576.jpg}
    	\includegraphics[width=0.135\linewidth]{figs/samples_appendix_5/none_ode/prompt_554.jpg}
    	\includegraphics[width=0.135\linewidth]{figs/samples_appendix_5/none_ode/prompt_345.jpg}
    	\includegraphics[width=0.135\linewidth]{figs/samples_appendix_5/none_ode/prompt_162.jpg}
    	\includegraphics[width=0.135\linewidth]{figs/samples_appendix_5/none_ode/prompt_94.jpg}
    	\includegraphics[width=0.135\linewidth]{figs/samples_appendix_5/none_ode/prompt_143.jpg} \\

        \rotatebox{90}{\parbox{2cm}{\centering\footnotesize DRaFT-1}} &
        \includegraphics[width=0.135\linewidth]{figs/samples_appendix_5/draft1_ode/prompt_606.jpg}
    	\includegraphics[width=0.135\linewidth]{figs/samples_appendix_5/draft1_ode/prompt_576.jpg}
    	\includegraphics[width=0.135\linewidth]{figs/samples_appendix_5/draft1_ode/prompt_554.jpg}
    	\includegraphics[width=0.135\linewidth]{figs/samples_appendix_5/draft1_ode/prompt_345.jpg}
    	\includegraphics[width=0.135\linewidth]{figs/samples_appendix_5/draft1_ode/prompt_162.jpg}
    	\includegraphics[width=0.135\linewidth]{figs/samples_appendix_5/draft1_ode/prompt_94.jpg}
    	\includegraphics[width=0.135\linewidth]{figs/samples_appendix_5/draft1_ode/prompt_143.jpg} \\

        \rotatebox{90}{\parbox{2cm}{\centering\footnotesize DRaFT-40}} &
        \includegraphics[width=0.135\linewidth]{figs/samples_appendix_5/draft40_ode/prompt_606.jpg}
    	\includegraphics[width=0.135\linewidth]{figs/samples_appendix_5/draft40_ode/prompt_576.jpg}
    	\includegraphics[width=0.135\linewidth]{figs/samples_appendix_5/draft40_ode/prompt_554.jpg}
    	\includegraphics[width=0.135\linewidth]{figs/samples_appendix_5/draft40_ode/prompt_345.jpg}
    	\includegraphics[width=0.135\linewidth]{figs/samples_appendix_5/draft40_ode/prompt_162.jpg}
    	\includegraphics[width=0.135\linewidth]{figs/samples_appendix_5/draft40_ode/prompt_94.jpg}
    	\includegraphics[width=0.135\linewidth]{figs/samples_appendix_5/draft40_ode/prompt_143.jpg} \\

        \rotatebox{90}{\parbox{2cm}{\centering\footnotesize ReFL}} &
        \includegraphics[width=0.135\linewidth]{figs/samples_appendix_5/refl_ode/prompt_606.jpg}
    	\includegraphics[width=0.135\linewidth]{figs/samples_appendix_5/refl_ode/prompt_576.jpg}
    	\includegraphics[width=0.135\linewidth]{figs/samples_appendix_5/refl_ode/prompt_554.jpg}
    	\includegraphics[width=0.135\linewidth]{figs/samples_appendix_5/refl_ode/prompt_345.jpg}
    	\includegraphics[width=0.135\linewidth]{figs/samples_appendix_5/refl_ode/prompt_162.jpg}
    	\includegraphics[width=0.135\linewidth]{figs/samples_appendix_5/refl_ode/prompt_94.jpg}
    	\includegraphics[width=0.135\linewidth]{figs/samples_appendix_5/refl_ode/prompt_143.jpg} \\

        \rotatebox{90}{\parbox{2cm}{\centering\footnotesize Cont. Adj.\\$\lambda = 12500$}} &
        \includegraphics[width=0.135\linewidth]{figs/samples_appendix_5/contadj_ode/prompt_606.jpg}
    	\includegraphics[width=0.135\linewidth]{figs/samples_appendix_5/contadj_ode/prompt_576.jpg}
    	\includegraphics[width=0.135\linewidth]{figs/samples_appendix_5/contadj_ode/prompt_554.jpg}
    	\includegraphics[width=0.135\linewidth]{figs/samples_appendix_5/contadj_ode/prompt_345.jpg}
    	\includegraphics[width=0.135\linewidth]{figs/samples_appendix_5/contadj_ode/prompt_162.jpg}
    	\includegraphics[width=0.135\linewidth]{figs/samples_appendix_5/contadj_ode/prompt_94.jpg}
    	\includegraphics[width=0.135\linewidth]{figs/samples_appendix_5/contadj_ode/prompt_143.jpg} \\

        \rotatebox{90}{\parbox{2cm}{\centering\footnotesize Disc. Adj.\\$\lambda = 12500$}} &
        \includegraphics[width=0.135\linewidth]{figs/samples_appendix_5/discadj_ode/prompt_606.jpg}
    	\includegraphics[width=0.135\linewidth]{figs/samples_appendix_5/discadj_ode/prompt_576.jpg}
    	\includegraphics[width=0.135\linewidth]{figs/samples_appendix_5/discadj_ode/prompt_554.jpg}
    	\includegraphics[width=0.135\linewidth]{figs/samples_appendix_5/discadj_ode/prompt_345.jpg}
    	\includegraphics[width=0.135\linewidth]{figs/samples_appendix_5/discadj_ode/prompt_162.jpg}
    	\includegraphics[width=0.135\linewidth]{figs/samples_appendix_5/discadj_ode/prompt_94.jpg}
    	\includegraphics[width=0.135\linewidth]{figs/samples_appendix_5/discadj_ode/prompt_143.jpg} \\
     
        \rotatebox{90}{\parbox{2cm}{\centering\footnotesize Adj. match.\\$\lambda = 1000$}} &
        \includegraphics[width=0.135\linewidth]{figs/samples_appendix_5/adjmat_1000_ode/prompt_606.jpg}
        \includegraphics[width=0.135\linewidth]{figs/samples_appendix_5/adjmat_1000_ode/prompt_576.jpg}
        \includegraphics[width=0.135\linewidth]{figs/samples_appendix_5/adjmat_1000_ode/prompt_554.jpg}
        \includegraphics[width=0.135\linewidth]{figs/samples_appendix_5/adjmat_1000_ode/prompt_345.jpg}
        \includegraphics[width=0.135\linewidth]{figs/samples_appendix_5/adjmat_1000_ode/prompt_162.jpg}
        \includegraphics[width=0.135\linewidth]{figs/samples_appendix_5/adjmat_1000_ode/prompt_94.jpg}
        \includegraphics[width=0.135\linewidth]{figs/samples_appendix_5/adjmat_1000_ode/prompt_143.jpg} \\

        \rotatebox{90}{\parbox{2cm}{\centering\footnotesize Adj. match.\\$\lambda = 2500$}} &
        \includegraphics[width=0.135\linewidth]{figs/samples_appendix_5/adjmat_2500_ode/prompt_606.jpg}
    	\includegraphics[width=0.135\linewidth]{figs/samples_appendix_5/adjmat_2500_ode/prompt_576.jpg}
    	\includegraphics[width=0.135\linewidth]{figs/samples_appendix_5/adjmat_2500_ode/prompt_554.jpg}
    	\includegraphics[width=0.135\linewidth]{figs/samples_appendix_5/adjmat_2500_ode/prompt_345.jpg}
    	\includegraphics[width=0.135\linewidth]{figs/samples_appendix_5/adjmat_2500_ode/prompt_162.jpg}
    	\includegraphics[width=0.135\linewidth]{figs/samples_appendix_5/adjmat_2500_ode/prompt_94.jpg}
    	\includegraphics[width=0.135\linewidth]{figs/samples_appendix_5/adjmat_2500_ode/prompt_143.jpg} \\

        \rotatebox{90}{\parbox{2cm}{\centering\footnotesize Adj. match.\\$\lambda = 12500$}} &
        \includegraphics[width=0.135\linewidth]{figs/samples_appendix_5/adjmat_12500_ode/prompt_606.jpg}
    	\includegraphics[width=0.135\linewidth]{figs/samples_appendix_5/adjmat_12500_ode/prompt_576.jpg}
    	\includegraphics[width=0.135\linewidth]{figs/samples_appendix_5/adjmat_12500_ode/prompt_554.jpg}
    	\includegraphics[width=0.135\linewidth]{figs/samples_appendix_5/adjmat_12500_ode/prompt_345.jpg}
    	\includegraphics[width=0.135\linewidth]{figs/samples_appendix_5/adjmat_12500_ode/prompt_162.jpg}
    	\includegraphics[width=0.135\linewidth]{figs/samples_appendix_5/adjmat_12500_ode/prompt_94.jpg}
    	\includegraphics[width=0.135\linewidth]{figs/samples_appendix_5/adjmat_12500_ode/prompt_143.jpg} 
     \end{tabular}
     \caption{Generated samples without guidance ($w=0$) and $\sigma(t) = 0$ across seven selected prompts. Each row corresponds to a different finetuning algorithm. Prompts: ``\textit{Seaside view poster with palm trees vector image}'', ``\textit{Cayucos Beach Inn}'', ``\textit{Happy Summer Life- Aloha Flowers and Melon - Pattern Metal Print}'', ``\textit{Castle Square, Warsaw Old Town}'', ``\textit{Funny girl blowing soap bubbles. High quality photo}'', ``\textit{Colombian man with sweatshirt over yellow wall listening to something by putting hand on the ear}'', ``\textit{man in the hood black mask masquerade}''.}
    \label{fig:one_image_per_prompt_ode}
\end{figure}

\begin{figure}[!htb]
    \centering
    \begin{tabular}{>{\centering\arraybackslash}m{0.3cm} m{0.96\linewidth}}
        \rotatebox{90}{\parbox{2cm}{\centering\footnotesize None (Base)}} &
        \includegraphics[width=0.135\linewidth]{figs/samples_appendix_5/none_sde/prompt_606.jpg}
    	\includegraphics[width=0.135\linewidth]{figs/samples_appendix_5/none_sde/prompt_576.jpg}
    	\includegraphics[width=0.135\linewidth]{figs/samples_appendix_5/none_sde/prompt_554.jpg}
    	\includegraphics[width=0.135\linewidth]{figs/samples_appendix_5/none_sde/prompt_345.jpg}
    	\includegraphics[width=0.135\linewidth]{figs/samples_appendix_5/none_sde/prompt_162.jpg}
    	\includegraphics[width=0.135\linewidth]{figs/samples_appendix_5/none_sde/prompt_94.jpg}
    	\includegraphics[width=0.135\linewidth]{figs/samples_appendix_5/none_sde/prompt_143.jpg} \\

        \rotatebox{90}{\parbox{2cm}{\centering\footnotesize DRaFT-1}} &
        \includegraphics[width=0.135\linewidth]{figs/samples_appendix_5/draft1_sde/prompt_606.jpg}
    	\includegraphics[width=0.135\linewidth]{figs/samples_appendix_5/draft1_sde/prompt_576.jpg}
    	\includegraphics[width=0.135\linewidth]{figs/samples_appendix_5/draft1_sde/prompt_554.jpg}
    	\includegraphics[width=0.135\linewidth]{figs/samples_appendix_5/draft1_sde/prompt_345.jpg}
    	\includegraphics[width=0.135\linewidth]{figs/samples_appendix_5/draft1_sde/prompt_162.jpg}
    	\includegraphics[width=0.135\linewidth]{figs/samples_appendix_5/draft1_sde/prompt_94.jpg}
    	\includegraphics[width=0.135\linewidth]{figs/samples_appendix_5/draft1_sde/prompt_143.jpg} \\

        \rotatebox{90}{\parbox{2cm}{\centering\footnotesize DRaFT-40}} &
        \includegraphics[width=0.135\linewidth]{figs/samples_appendix_5/draft40_sde/prompt_606.jpg}
    	\includegraphics[width=0.135\linewidth]{figs/samples_appendix_5/draft40_sde/prompt_576.jpg}
    	\includegraphics[width=0.135\linewidth]{figs/samples_appendix_5/draft40_sde/prompt_554.jpg}
    	\includegraphics[width=0.135\linewidth]{figs/samples_appendix_5/draft40_sde/prompt_345.jpg}
    	\includegraphics[width=0.135\linewidth]{figs/samples_appendix_5/draft40_sde/prompt_162.jpg}
    	\includegraphics[width=0.135\linewidth]{figs/samples_appendix_5/draft40_sde/prompt_94.jpg}
    	\includegraphics[width=0.135\linewidth]{figs/samples_appendix_5/draft40_sde/prompt_143.jpg}\\

        \rotatebox{90}{\parbox{2cm}{\centering\footnotesize ReFL}} &
        \includegraphics[width=0.135\linewidth]{figs/samples_appendix_5/refl_sde/prompt_606.jpg}
    	\includegraphics[width=0.135\linewidth]{figs/samples_appendix_5/refl_sde/prompt_576.jpg}
    	\includegraphics[width=0.135\linewidth]{figs/samples_appendix_5/refl_sde/prompt_554.jpg}
    	\includegraphics[width=0.135\linewidth]{figs/samples_appendix_5/refl_sde/prompt_345.jpg}
    	\includegraphics[width=0.135\linewidth]{figs/samples_appendix_5/refl_sde/prompt_162.jpg}
    	\includegraphics[width=0.135\linewidth]{figs/samples_appendix_5/refl_sde/prompt_94.jpg}
    	\includegraphics[width=0.135\linewidth]{figs/samples_appendix_5/refl_sde/prompt_143.jpg} \\

        \rotatebox{90}{\parbox{2cm}{\centering\footnotesize Cont. Adj.\\$\lambda = 12500$}} &
        \includegraphics[width=0.135\linewidth]{figs/samples_appendix_5/contadj_sde/prompt_606.jpg}
    	\includegraphics[width=0.135\linewidth]{figs/samples_appendix_5/contadj_sde/prompt_576.jpg}
    	\includegraphics[width=0.135\linewidth]{figs/samples_appendix_5/contadj_sde/prompt_554.jpg}
    	\includegraphics[width=0.135\linewidth]{figs/samples_appendix_5/contadj_sde/prompt_345.jpg}
    	\includegraphics[width=0.135\linewidth]{figs/samples_appendix_5/contadj_sde/prompt_162.jpg}
    	\includegraphics[width=0.135\linewidth]{figs/samples_appendix_5/contadj_sde/prompt_94.jpg}
    	\includegraphics[width=0.135\linewidth]{figs/samples_appendix_5/contadj_sde/prompt_143.jpg} \\

        \rotatebox{90}{\parbox{2cm}{\centering\footnotesize Disc. Adj.\\$\lambda = 12500$}} &
        \includegraphics[width=0.135\linewidth]{figs/samples_appendix_5/discadj_sde/prompt_606.jpg}
    	\includegraphics[width=0.135\linewidth]{figs/samples_appendix_5/discadj_sde/prompt_576.jpg}
    	\includegraphics[width=0.135\linewidth]{figs/samples_appendix_5/discadj_sde/prompt_554.jpg}
    	\includegraphics[width=0.135\linewidth]{figs/samples_appendix_5/discadj_sde/prompt_345.jpg}
    	\includegraphics[width=0.135\linewidth]{figs/samples_appendix_5/discadj_sde/prompt_162.jpg}
    	\includegraphics[width=0.135\linewidth]{figs/samples_appendix_5/discadj_sde/prompt_94.jpg}
    	\includegraphics[width=0.135\linewidth]{figs/samples_appendix_5/discadj_sde/prompt_143.jpg} \\

        \rotatebox{90}{\parbox{2cm}{\centering\footnotesize Adj. match.\\$\lambda = 1000$}} &
        \includegraphics[width=0.135\linewidth]{figs/samples_appendix_5/adjmat_1000_sde/prompt_606.jpg}
        \includegraphics[width=0.135\linewidth]{figs/samples_appendix_5/adjmat_1000_sde/prompt_576.jpg}
        \includegraphics[width=0.135\linewidth]{figs/samples_appendix_5/adjmat_1000_sde/prompt_554.jpg}
        \includegraphics[width=0.135\linewidth]{figs/samples_appendix_5/adjmat_1000_sde/prompt_345.jpg}
        \includegraphics[width=0.135\linewidth]{figs/samples_appendix_5/adjmat_1000_sde/prompt_162.jpg}
        \includegraphics[width=0.135\linewidth]{figs/samples_appendix_5/adjmat_1000_sde/prompt_94.jpg}
        \includegraphics[width=0.135\linewidth]{figs/samples_appendix_5/adjmat_1000_sde/prompt_143.jpg} \\

        \rotatebox{90}{\parbox{2cm}{\centering\footnotesize Adj. match.\\$\lambda = 2500$}} &
        \includegraphics[width=0.135\linewidth]{figs/samples_appendix_5/adjmat_2500_sde/prompt_606.jpg}
    	\includegraphics[width=0.135\linewidth]{figs/samples_appendix_5/adjmat_2500_sde/prompt_576.jpg}
    	\includegraphics[width=0.135\linewidth]{figs/samples_appendix_5/adjmat_2500_sde/prompt_554.jpg}
    	\includegraphics[width=0.135\linewidth]{figs/samples_appendix_5/adjmat_2500_sde/prompt_345.jpg}
    	\includegraphics[width=0.135\linewidth]{figs/samples_appendix_5/adjmat_2500_sde/prompt_162.jpg}
    	\includegraphics[width=0.135\linewidth]{figs/samples_appendix_5/adjmat_2500_sde/prompt_94.jpg}
    	\includegraphics[width=0.135\linewidth]{figs/samples_appendix_5/adjmat_2500_sde/prompt_143.jpg} \\

        \rotatebox{90}{\parbox{2cm}{\centering\footnotesize Adj. match.\\$\lambda = 12500$}} &
        \includegraphics[width=0.135\linewidth]{figs/samples_appendix_5/adjmat_12500_sde/prompt_606.jpg}
    	\includegraphics[width=0.135\linewidth]{figs/samples_appendix_5/adjmat_12500_sde/prompt_576.jpg}
    	\includegraphics[width=0.135\linewidth]{figs/samples_appendix_5/adjmat_12500_sde/prompt_554.jpg}
    	\includegraphics[width=0.135\linewidth]{figs/samples_appendix_5/adjmat_12500_sde/prompt_345.jpg}
    	\includegraphics[width=0.135\linewidth]{figs/samples_appendix_5/adjmat_12500_sde/prompt_162.jpg}
    	\includegraphics[width=0.135\linewidth]{figs/samples_appendix_5/adjmat_12500_sde/prompt_94.jpg}
    	\includegraphics[width=0.135\linewidth]{figs/samples_appendix_5/adjmat_12500_sde/prompt_143.jpg} 
     \end{tabular}
     \caption{Generated samples without guidance ($w=0$) and $\sigma(t) = \sqrt{2 \eta_t}$ across seven selected prompts. Each row corresponds to a different finetuning algorithm. The prompts are the same as in \Cref{fig:one_image_per_prompt_ode}.}
    \label{fig:one_image_per_prompt_sde}
\end{figure}

\clearpage
\newpage

\section{Appendix} \label{appendix}


\subsection{NewYorker Data for evaluation}

\begin{figure}[!ht]
\small
\centering
\includegraphics[width=0.4\textwidth]{figures/length.png}
\caption{\label{lengthdist} Distribution of word count of stories in our test set}
\end{figure}

Table \ref{teststories} shows the data used for conducting our evaluation. The 12 stories shown are taken from The New Yorker and summarized into single-sentence plots. These stories come from highly established literary experts acting as an upper bound for what it means to be creative. These stories span complex themes.

\begin{table*}[!ht]
\centering
\small
\def\arraystretch{1.35}
\begin{tabular}{|l|}
\hline
\begin{tabular}[c]{@{}l@{}}Write a New Yorker-style story given the plot below. Make sure it is atleast \textbf{\color{blue}\{\{word\_count\}\}} words. Directly start with the\\ story, do not say things like `Here's the story {[}...{]}:\end{tabular}                                                                                                                                                                                            \\ \hline\hline
\begin{tabular}[c]{@{}l@{}}You wrote the story I gave you below. I requested a story with \textbf{\color{blue}\{\{word\_count\}\}} words, but the story only has\\ \textbf{\color{blue}\{\{current\_word\_count\}\}} words. Can you rewrite the story to make it longer, and closer to the \textbf{\color{blue}\{\{word\_count\}\}} word target\\ I gave you. Directly start with the story, do not say things like `Here's the story {[}...{]}:`\\ \\ Current story: \{\{story\}\}\end{tabular} \\ \hline
\end{tabular}
\vspace{2ex}
\caption{\label{promptstory}Prompt to write the initial story (Row1) vs Prompt to rewrite the initial story to be longer. word\_count represents the number of words in the human written story on a given plot (P) while current\_word\_count represents the number of words in the LLM generated story on the same plot (P)}
\end{table*}

\begin{table*}[!ht]
\def\arraystretch{1.15}
\small
\begin{tabular}{|l|l|}
\hline
Story                                    & Plot                                                                                                                                                                                                                                                                                                                                                                                                                                                                                                                                   \\ \hline
\href{https://www.newyorker.com/books/flash-fiction/a-triangle}{A Triangle}                               & \begin{tabular}[c]{@{}l@{}}An observer becomes entranced by a seemingly ordinary couple on the street, follows them home, and then \\watches them from outside in the rising floodwaters, drawing an eerie connection between the woman and\\ a discarded, burned chair they'd noticed earlier.\end{tabular}                                                                                                                                                                    \\ \hline\hline
\href{https://www.newyorker.com/books/flash-fiction/barbara-detroit-1966}{\begin{tabular}[c]{@{}l@{}}Barbara\\ Detroit,1966\end{tabular}}                    & \begin{tabular}[c]{@{}l@{}}On Feb 12, 1966, a heavily pregnant woman named Barbara experienced a shocking incident in her synagogue\\in Southfield, Detroit, where a young man shot and killed the renowned Rabbi Adler before turning the gun\\ on himself, and though Barbara tried to reach the shooter, she was swept away by the fleeing crowd.\end{tabular}                                                                              \\ \hline\hline
\href{https://www.newyorker.com/books/flash-fiction/beyond-nature}{Beyond Nature}                           & \begin{tabular}[c]{@{}l@{}}A solitary man walking in a remote mountainous region comes across a car crash, and stays by the side\\ of the lifeless female victim, narrating stories of his past and reflecting on the impermanence of \\events and life itself, while awaiting emergency services amidst the looming presence of wilderness.\end{tabular}                                                                                                                \\ \hline\hline
\href{https://www.newyorker.com/books/flash-fiction/certain-european-movies}{\begin{tabular}[c]{@{}l@{}}Certain European\\ Movies\end{tabular}}                  & \begin{tabular}[c]{@{}l@{}}Two individuals, at a residency together, navigate the complexity of their ephemeral relationship during\\ their final beach trip, framed by misadventures, subtle tensions, unspoken desires, and looming departures.\end{tabular}                                                                                                                                                                                   \\ \hline\hline
\href{https://www.newyorker.com/books/flash-fiction/keys}{Keys}                                     & \begin{tabular}[c]{@{}l@{}}Daniel, struggling with recurring dreams of his ex-wife Rachel and a mysterious unused flat, eventually \\discusses them with his current partner Isabel, sparking various reflections and conversations about their\\ past relationships, until a real-life discovery of old keys triggers a nostalgic memory and helps him find a\\ way to reconnect with his present relationship through canoeing.\end{tabular}                                     \\ \hline\hline
\href{https://www.newyorker.com/books/flash-fiction/listening-for-the-click}{\begin{tabular}[c]{@{}l@{}}Listening For\\ the Click\end{tabular}}                  & \begin{tabular}[c]{@{}l@{}}Navigating a complex social landscape, the protagonist experiences a series of complex relationships \\and emotional turmoil in a student environment, and engages in self-discovery and self-reflection as she\\ interacts with the characters Carl, Martin, Lizzy, and Johan, resulting in a journey of introspection,\\ betrayal, love, and personal growth.\end{tabular}                                                          \\ \hline\hline
\href{https://www.newyorker.com/magazine/2023/05/15/maintenance-hvidovre-fiction-olga-ravn}{\begin{tabular}[c]{@{}l@{}}Maintenance,\\ Hvidovre\end{tabular}}                   & \begin{tabular}[c]{@{}l@{}}A woman experiences a disorienting night in a maternity ward where she encounters other similarly \\disoriented new mothers, leading to an uncanny mix-up where she leaves the hospital with a baby \\that she realizes is not her own, yet accepts the situation with an inexplicable sense of happiness.\end{tabular}                                                                                                  \\ \hline\hline
\href{https://www.newyorker.com/magazine/2022/11/14/returns}{Returns}                                  & \begin{tabular}[c]{@{}l@{}}The narrator visits their elderly mother in her small town, spending a day with her that is filled with \\nostalgia, conversation, and old habits, only to return a month later after her hospitalization due to\\ a sunstroke, finding remnants of their last visit.\end{tabular}                                                                                                                                                                      \\ \hline\hline
\href{https://www.newyorker.com/books/flash-fiction/the-facade-renovation-thats-going-well}{\begin{tabular}[c]{@{}l@{}}The Facade \\Renovation\\ That’s Going Well\end{tabular}} & \begin{tabular}[c]{@{}l@{}}An academic faculty housed in a building with a critical waterproofing layer missing experiences a series\\ of disruptive and problematic construction repairs, causing tension, inconvenience, and health concerns\\ among the tenants, ultimately leading to resignation and endurance in hopes of better future circumstances.\end{tabular}                                                        \\ \hline\hline
\href{https://www.newyorker.com/books/flash-fiction/the-kingdom-that-failed}{\begin{tabular}[c]{@{}l@{}}The Kingdom\\ That Failed\end{tabular}}                  & \begin{tabular}[c]{@{}l@{}}The narrator recounts their college friendship with the seemingly flawless Q, and after a decade apart, \\they accidentally cross paths at a pool, where the narrator anonymously observes Q's failed attempt to \\let down a woman about a work-related issue, demonstrating that Q, too, has his share of difficulties.\end{tabular}                                                                                                \\ \hline\hline
\href{https://www.newyorker.com/magazine/2022/06/13/trash }{Trash}                                    & \begin{tabular}[c]{@{}l@{}}A woman unexpectedly marries the son of a successful, ambitious woman named Miss Emily, finding both \\acceptance and critique from her mother-in-law as she navigates this new relationship and confronts the \\stark contrasts between her former life as a supermarket cashier and her new life as part of a well-off family.\end{tabular}                                                                                                            \\ \hline\hline
\href{https://www.newyorker.com/culture/personal-history/the-last-dance-with-my-dad}{\begin{tabular}[c]{@{}l@{}}The Last Dance\\ with my Dad \end{tabular}}               & \begin{tabular}[c]{@{}l@{}}A young teenager recounts her experiences of fitting into her father's gay lifestyle, highlighted by a\\ seven-day cruise with hundreds of gay men, where she experienced acceptance and connection, had her\\ first genuine interaction with a  boy, and shared a last dance with her terminally ill father.\end{tabular}                                                                                                       \\ \hline
\end{tabular}
\vspace{2ex}
\caption{\label{teststories} Expert-written short stories from the New Yorker along with their human-verified GPT4 generated summary as plots that are included as part of our test data for Creativity Evaluation}
\end{table*}


\subsection{Expert Perception on the TTCW tests}

\begin{figure*}[!ht]
    \centering
     \includegraphics[width=\textwidth]{figures/rel.pdf}
    \caption{\label{relev} Relative Evaluation by Creative Writing Experts within a given group of four stories}
\end{figure*}

\begin{table*}[!ht]
\small
\centering
\begin{tabular}{|l|l|}
\hline
E5 & \begin{tabular}[c]{@{}l@{}}It was a pretty effective rubric! I'm used to being more subjective in my work -- did you like a story? Did it connect with \\you? Did it make sense? Why or why not? It was often challenging to break it down into more regimented segments \\like the rubric asked for -- but I do think that it allowed me to express the subjective feelings in a pretty thorough and\\ structured way!\end{tabular}                                                                                                                                                                 \\ \hline
E3 & \begin{tabular}[c]{@{}l@{}}As for the rubric, I thought it was quite thorough. There were some categories where I would say the story didn’t ``pass,"\\ but which were excellent. This happened often with the categories about multiple points of view, and innovative\\ structure and form. Overall, I think the rubric was helpful in helping me think about the different aspects of storytelling.\end{tabular}                                                                                                                                                                                 \\ \hline
E4 & \begin{tabular}[c]{@{}l@{}}I thought the rubric felt pretty thorough; the only part I felt could be added was that suggestion about consistency in\\ voice \& diction!\end{tabular}                                                                                                                                                         \\ \hline
E2 & \begin{tabular}[c]{@{}l@{}}The rubric seemed great to me! It’s however hard to talk about something like pacing without talking about scene and \\summary, for instance. Or the difference between originality of thought and originality in theme/content—wouldn’t the \\latter make up the former and vice/versa? But it is also comprehensive and I can see the merits of this sort of repetition in\\ teasing out a fuller picture of things\end{tabular} \\ \hline
E1 & \begin{tabular}[c]{@{}l@{}}I thought the rubric was pretty good tbh. I think there is overlap in some of the different elements, like "language \\proficiency \& literary devices" and "originality in thought." it's tricky to use words like "satisfying" and "sophisticated" \\when assessing art, but there's always going to be a great deal of subjectivity in these matters.I think that voice is a crucial \\aspect of high-quality writing that is being overlooked by the rubric, and one that greatly informs how I as a reader\\ evaluate 
and appreciate literary writing.\end{tabular} \\ \hline
\end{tabular}
\vspace{2ex}
\caption{\label{expertfeedbackrubric}Expert perception and feedback on the TTCW tests they conducted as part of our data collection.}
\end{table*}

Since the experts listed in Table \ref{creativeexperts} were not involved in designing the rubric but evaluated several stories based on the rubric we asked them their \textit{overall thought about the rubric and any potentially crucial test we missed out on that they use to discriminate between good and bad writing}.As can be seen in Table \ref{expertfeedbackrubric} in Appendix overall almost every expert agreed on the thorough and effective nature of our rubric. Many of them agreed on the fact that our rubric helped them to think about different aspects of storytelling in a more structured way. One of the difficult things about coming up with a rubric for creativity is ensuring coverage. Even though our rubric covers most aspects of creative writing, some experts such as E1 and E4 emphasized on the utility of \textbf{Consistency of Voice and Diction} as a measurable test. In E4's words \textit{``Inconsistent voice and diction are sometimes/often notable in stories that aren't very good, and when voice \& diction are used beautifully, it enhances a story considerably"}. E1 similarly exclaimed \textit{``One of the most meaningful aspects of high-quality literary writing is voice, which conveys qualities of proficiency, artistry, personality, and identity."}. We hope future work can adapt this as a meaningful test in addition to the tests covered in our rubric. Finally, some of the tests from our rubric can have potential overlaps as pointed out by E2. This is further corroborated by the similar numbers for \textit{Narrative Pacing} and \textit{Scenes vs Exposition} suggesting a strong correlation between the two.
\begin{table*}[!ht]
\small
\centering
% \def\arraystretch{1.3}
\begin{tabular}{|l|l|l|}
\hline
Test & Passing Stories & Failing Stories \\ \hline
\begin{tabular}[c]{@{}l@{}}Originality in\\ Form\end{tabular} & \begin{tabular}[c]{@{}l@{}}Inventive techniques like time jumping, varied \\ perspectives, unconventional punctuation, and\\ delayed revelation of key information\end{tabular} & \begin{tabular}[c]{@{}l@{}}Conventional and linear in its form, language, \\ and narrative, with occasional attempts at \\ innovation that do not significantly contribute to \\ its overall originality or creativity\end{tabular} \\ \hline
\begin{tabular}[c]{@{}l@{}}Originality in\\ Thought\end{tabular} & \begin{tabular}[c]{@{}l@{}}Fresh language, unique plot and characters, subtle\\ emotional resonance, and inventive metaphors. Minor \\ familiar elements, but do not undermine the overall \\ sense of imagination and distinctiveness\end{tabular} & \begin{tabular}[c]{@{}l@{}}Stories relies heavily on cliches \& tired tropes.\\ Language does not feel fresh or original with \\ narrative arc following a predictable trajectory.\\ Metaphors, descriptions, and overall premise \\ cover familiar ground that lacks novelty or nuance\end{tabular} \\ \hline
\begin{tabular}[c]{@{}l@{}}Originality in\\ Theme/Content\end{tabular} & \begin{tabular}[c]{@{}l@{}}Unconventional, dreamlike exploration of emotions\\ such as love and loss, evoking empathy and reflection\\ through its distinct main character perspective, \\ eschewing simplistic meanings for ambiguity, and \\ valuing open-ended questions over singular messages,\\ thus providing a unique reading experience compared\\ to conventional stories.\end{tabular} & \begin{tabular}[c]{@{}l@{}}Disjointed narrative, underdeveloped themes, \\ inconsistent tone, vaguely defined characters, and\\ abrupt context shifts, lack depth and fail to provide \\ substantive insight or originality to the reader.\end{tabular} \\ \hline\hline
\begin{tabular}[c]{@{}l@{}}World Building\\ and Setting\end{tabular} & \begin{tabular}[c]{@{}l@{}}Strategic use of concrete, specific sensory details from\\ a particular character’s perspective balances narrative\\ momentum, making a fictional world feel real, vivid\\ and immersive for readers. Thoughtful depiction of\\ everyday objects, and idiosyncratic elements within\\ narrative and dialogue to balance exposition with \\ vivid scene-setting, creating authenticity and realism \\ that serves the plot and characters\end{tabular} & \begin{tabular}[c]{@{}l@{}}Fictional world is not always convincingly \\established through sensory details and language. \\Stories rely too heavily on overwrought imagery\\ and figurative language without grounding \\the reader in a tangible reality.\end{tabular} \\ \hline
\begin{tabular}[c]{@{}l@{}}Character\\ Development\end{tabular} & \begin{tabular}[c]{@{}l@{}}Fully realized characters with contradictions, \\ motivations, and backstories that make them\\ feel lifelike. Flatter, less developed characters\\ that feel appropriate for the narrative goals \\ and style is not necessarily a weakness\end{tabular} & \begin{tabular}[c]{@{}l@{}}Characters not well rounded. easily resorting to \\stereotypes. Predictable arcs not making them\\memorable. Actions or motivations unclear leading \\to disconnect\end{tabular} \\ \hline
\begin{tabular}[c]{@{}l@{}}Rhetorical\\ Complexity\end{tabular} & \begin{tabular}[c]{@{}l@{}}Rich subtext that emerges through contrasts between\\ characters and settings. Omissions that let readers \\ fill in meaning, metaphors with layered significance, \\ implicit characterization, ambiguous endings, \\ intertextual references, and rich language that goes\\ beyond straightforward storytelling\end{tabular} & \begin{tabular}[c]{@{}l@{}}Straightforward storytelling, lack of complexity,\\ nuance, and ambiguity.The absence of context and \\ ironic tensions, overtelling instead of showing, \\ Use of expository dialogue, blunt symbolism, \\ predictable characters, and definitive endings, all \\of which limit interpretation and undermine\\ their potential resonance\end{tabular} \\ \hline
\end{tabular}
\vspace{2ex}
\caption{\label{expertexpl2}Common themes and issues found in expert explanations for tests focusing on TTCW-Originality and TTCW-Elaboration}
\vspace{-8ex}
\end{table*}

\subsection{Common themes in TTCW of Originality and Elaboration}


Table~\ref{expertexpl2} provides common themes and issues found in the expert explanations for each TTCW in the Originality and Elaboration dimensions.


\subsection{Example LLM-generated and expert-written explanations for a TTCW assessment} 

\begin{table*}[!ht]
\centering
\small
\def\arraystretch{1.35}
\begin{tabular}{|l|l|}
\hline
GPT4 & \begin{tabular}[c]{@{}l@{}}The story incorporates numerous instances of subtext that serve to further develop the characters' internal emotions \\ and relationships, as well as accentuate the atmospheric tension and undertones of longing. The subtext, which is\\ subtly weaved into surface-level interactions, actions, and observations, is eloquently crafted and does not feel \\ forced, enriching both the narrative depth and setting.\\ One key passage with evident subtext is when he is in the ocean, swimming and jumping off the rock, while she\\ observes from the towel. The surface-level interpretation suggests a casual, relaxed beach day. However, the \\ subtext is rife with allusions of emotional undercurrents in their interaction. His unawareness of her going \\ into the water parallels the emotional disconnect between them, insinuating the vast, metaphorical space that \\ separates their respective experiences and understanding of their relationship.\\ Another instance of subtext occurs when he takes a photo of her with her swimsuit top off, indicating his attraction\\ to her. But the meaning deepens when we understand that he will delete the photo before going home, implicitly\\ revealing his guilt or fear of being discovered, and again highlighting the transient, secret nature of their relationship.\\ The line, "As in certain European movies" suggests the fleeting, impermanent element of their relationship and the\\ eventual abruptness and intensity of feelings that often feature in said films. The 'European movies' serve as a \\ metaphor for the unexpected and sudden end of their encounter. \\ In these examples, the subtext does not feel forced but instead flows naturally from the characters and \\ circumstances, subtly conveying deeper meanings that heighten both the narrative tension and emotional depth.\\ \\ So Yes.\end{tabular} \\ \hline
E3   & \begin{tabular}[c]{@{}l@{}}There is rich subtext, as the main character seems continually conflicted about whether she wants to be where she is, \\ doing what she is doing. On the surface, she is carefree, riding to the beach with the guy she met, skipping the ceramics\\ and the museum, and whatever else. And yet, she is unhappy and unsatisfied, longing for a beer, imagining that if their\\ relationship continued they would only hate each other. This tension is maintained throughout the story.\end{tabular}                                                                                                                                                                                                                                                                                                               \\ \hline
E1   & \begin{tabular}[c]{@{}l@{}}This piece has an iceberg of subtext floating underneath it. The entire story is conveyed through the successful \\ integration of subtext and text. The interactions between the protagonist and the man (Did you see me jump of the \\ rock? No, she hadn't.Did he notice she had gone in the water too, that her hair was dripping? No, he hadn't.)convey\\ a profound disconnect that causes the reader to wonder why the protagonist continues to suffer the presence of this\\ man she clearly disdains and seems to view as an incompetent man-child.\end{tabular}
               \\ \hline
E7   & \begin{tabular}[c]{@{}l@{}}Yes!!!!! Again, the idea of the story was fairly simple (the inevitability of age, parting, change), but it was illustrated\\ in a way that felt inspiring re: questioning how these ideas relate and resonate throughout our own lives. It was really \\ beautiful and I was left feeling changed at the end of it :)\end{tabular}                                                                                                                                                                                                                                                            \\ \hline
\end{tabular}
\vspace{2ex}
\caption{\label{llmvsexpertexpl}LLM explanation vs expert explanation for Rhetorical Complexity}
\end{table*}

In Table~\ref{llmvsexpertexpl}, we show examples of explanations that experts wrote in conjunction with a binary TTCW assessment they made on a story, as well as the corresponding LLM-generated explanations.

\subsection{Can non-experts administer TTCW tests?}

Recruiting experts for data annotation purposes is challenging, and costly, and must consider the time constraint put on the experts. Prior work has shown the potential of crowd-sourcing (through platforms such as Amazon Mechanical Turk) and the ability of non-experts to accomplish complex tasks as a crowd \cite{kittur2013future}, when following an appropriate workflow that iterates and validates the work on individual non-experts. Some prior work has even shown the validity of crowd-based feedback for writing tasks \cite{bernstein2010soylent,nebeling2016wearwrite}. 

In this work, we chose to rely on experts for annotation, to maximize the validity of our experiments, and confirm whether experts with domain knowledge would reach satisfying agreement levels when evaluating stories with TTCW. Future work can leverage our open-sourced annotations to explore whether non-experts correlate with experts when performing TTCW evaluation, which could lead to more cost-effective TTCW evaluation.

\subsection{Prompts for TTCW} \label{allprompts}

All the instructions shown to creative writing experts and LLMs are given in the tables below.


\begin{table*}[!ht]
\centering
\small
\begin{tabular}{|l|l|}
\hline
\begin{tabular}[c]{@{}l@{}}Expert \\ Measure\end{tabular}               & Does the manipulation of time in terms of compression or stretching feel appropriate and balanced?                                                                                                                                    \\ \hline
\begin{tabular}[c]{@{}l@{}}Expanded\\ Expert\\ Measure (M)\end{tabular} & \begin{tabular}[c]{@{}l@{}}`Compression/stretching of time' in fiction writing, also known as pacing, refers to the manipulation of time in \\storytelling for dramatic effect, pacing, or other narrative purposes. Essentially, it's about controlling the perceived \\speed and rhythm at which a story unfolds.\\ \\

Compression of time refers to when events that take a long time (hours, days, weeks, or even years) are summarized \\or condensed into a brief narrative span. For example, a writer might compress several years of a character's life \\into a few paragraphs to quickly convey important changes or developments.\\ \\

On the other hand, stretching of time is when a brief moment or event is drawn out over pages or chapters. It's often \\used to create suspense, emphasize details, or delve deeper into a character's thoughts and feelings. For example, \\the few seconds it takes for a dropped glass to hit the floor might be stretched out with detailed descriptions of the\\ action, reactions, and thoughts of characters involved.\\ \\

Storytime refers to the time within the world of the story, while real-world time refers to the time it takes for the \\reader to read the story. A skilled writer can manipulate the relationship between these two to affect the pacing of \\the narrative, either speeding it up (compression) or slowing it down (stretching). This technique plays a crucial role \\in shaping the reader's experience and engagement with the story.\end{tabular} \\ \hline
\begin{tabular}[c]{@{}l@{}}Human\\ Instruction\end{tabular}             & \begin{tabular}[c]{@{}l@{}}\{\{M\}\}\\ \\ Based on the story that you just read, answer the following question.\\ \textit{\color{blue}Does the manipulation of time in terms of compression or stretching feel appropriate and balanced?}\\ -Yes \\ -No \\\\ Reasoning : \end{tabular}                                                                       \\ \hline
\begin{tabular}[c]{@{}l@{}}LLM\\ Instruction\end{tabular}               & \begin{tabular}[c]{@{}l@{}}\{\{M\}\}\\ \\ Given the story above, list out the scenes in the story in which time compression or time stretching is used, and \\argue for each whether it is successfully implemented.  Then overall, give your reasoning about the question below \\and give an answer to it between 'Yes' or 'No' only \\ \\ \textit{\color{blue} Q) Does the manipulation of time in terms of compression or stretching feel appropriate and balanced?}\end{tabular}                                                                                                                                                                                                                    \\ \hline
\end{tabular}
\vspace{2ex}
\caption{\label{prompting}TTCW Fluency1 (Narrative Pacing) }
\vspace{-5ex}
\end{table*}


% ==================================================





\begin{table*}[!ht]
\centering
\small
% \def\arraystretch{1.15}
\begin{tabular}{|l|l|}
\hline
\begin{tabular}[c]{@{}l@{}}Expert \\ Measure\end{tabular}               & \begin{tabular}[c]{@{}l@{}}Does the story have an appropriate balance between scene and summary/exposition or it relies on one\\ of the elements heavily compared to the other?  \end{tabular}                                                                                                                                  \\ \hline
\begin{tabular}[c]{@{}l@{}}Expanded\\ Expert\\ Measure (M)\end{tabular} & \begin{tabular}[c]{@{}l@{}}'Scene' and 'summary/exposition' are two crucial elements of narrative storytelling, and balancing them \\appropriately is an important skill in fiction writing.\\ \\ 

A 'scene' is a moment in the story that is dramatized in real-time. Scenes are usually vivid and engaging, often \\featuring character interaction, dialogue, and action. They are the building blocks of the plot, and through them, \\the story unfolds.\\ \\ 

'Summary' or 'exposition', on the other hand, involves summarizing events or providing information. Instead of \\unfolding in real time, \\summaries compress time and tell the reader what happened. Exposition provides \\necessary background information, like character history, setting details, or prior events. \\ \\ 

A good writer knows when to use scenes to make the story come alive, show character development, or increase \\tension. They also know when to use summary or exposition to move the story forward, fill in background \\information, or bridge gaps between important scenes. \\ \\ 

The right balance between scene and summary/exposition can vary depending on the story, but in general, it's \\essential for maintaining a good pace, keeping the reader engaged, and delivering necessary information. \\A story with too many scenes and not enough summary might feel overwhelming or slow, while a story with \\too much exposition and not enough scenes could feel dry and unengaging.\end{tabular} \\ \hline
\begin{tabular}[c]{@{}l@{}}Human\\ Instruction\end{tabular}             & \begin{tabular}[c]{@{}l@{}}\{\{M\}\}\\ \\ Based on the story that you just read, answer the following question.\\ \textit{\color{blue} Does the story have an appropriate balance between scene and summary/exposition or it relies on one of the elements} \\\textit{\color{blue}heavily compared to the other?} \\ -Yes \\ -No \\\\ Reasoning : \end{tabular}    
\\ \hline
\begin{tabular}[c]{@{}l@{}}LLM\\ Instruction\end{tabular}               & \begin{tabular}[c]{@{}l@{}}\{\{M\}\}\\ \\ Given the story above, answer the following question. Please first explain your reasoning step by step \\and then given an answer between 'Yes' or 'No' only \\ \\ \textit{\color{blue} Does the story have an appropriate balance between scene and summary/exposition or it relies on one of the elements} \\\textit{\color{blue}heavily compared to the other?}\end{tabular}                                                                                                                                                                                                                    \\ \hline
\end{tabular}
\vspace{2ex}
\caption{\label{prompting}TTCW Fluency2 (Scene vs Exposition) }
\vspace{-5ex}
\end{table*}


% ==================================================


\begin{table*}[!ht]
\centering
\small
% \def\arraystretch{1.15}
\begin{tabular}{|l|l|}
\hline
\begin{tabular}[c]{@{}l@{}}Expert \\ Measure\end{tabular}               & Does the story make sophisticated use of idiom or metaphor or literary allusion?                                                                                                                                     \\ \hline
\begin{tabular}[c]{@{}l@{}}Expanded\\ Expert\\ Measure (M)\end{tabular} & \begin{tabular}[c]{@{}l@{}}`Idiom' refers to phrases or expressions that have a figurative, or sometimes literal, meaning that is \\comprehensible to a particular group of people. These can be cultural, regional, or specific to a certain group or \\profession.Sophisticated use of idiom suggests that the writer is skillfully using these expressions to lend \\authenticity to character voices or to convey specific meanings in a concise way.\\\\

`Metaphor' is a figure of speech that describes an object or action in a way that isn't literally true, but helps explain\\ an idea or make a comparison. Sophisticated use of metaphor suggests the
writer could create impactful, original \\comparisons that reveal deeper insights about themes,
characters, or situations in the story.\\\\

`Literary allusion' refers to a brief and indirect reference to a person, place, thing or idea of
historical, cultural,\\ literary, or political significance. It does not describe in detail the person or thing to which it refers. A sophisticated\\ use of literary allusion implies the writer can effectively incorporate these references to enhance the depth\\ and resonance of the story. They can provide contextual richness, evoke a specific tone, or draw parallels between\\ the narrative and the work alluded to.\\\\

Overall, when a writer uses these techniques well, they add depth, interest, and nuanced \\meaning
to their work. It allows for a richer reading experience, where the literal events are \\imbued with deeper symbolic or thematic significance.\end{tabular} \\ \hline
\begin{tabular}[c]{@{}l@{}}Human\\ Instruction\end{tabular}             & \begin{tabular}[c]{@{}l@{}}\{\{M\}\}\\ \\ Based on the story that you just read, answer the following question.\\ \textit{\color{blue}Does the story make sophisticated use of idiom or metaphor or literary allusion?}\\ -Yes \\ -No \\\\ Reasoning: \end{tabular}                                                                       \\ \hline
\begin{tabular}[c]{@{}l@{}}LLM\\ Instruction\end{tabular}               & \begin{tabular}[c]{@{}l@{}}\{\{M\}\}\\ \\ Given the story above, please list out all the metaphors, idioms and literary allusions, and for each decide \\whether it is successful vs it feels forced or too easy.  Then overall, give your reasoning about the question \\below and give an answer to it between 'Yes' or 'No' only\\ \\ \textit{\color{blue} Q) Does the story make sophisticated use of idiom or metaphor or literary allusion?}\end{tabular}                                                                                                                                                                                                                    \\ \hline
\end{tabular}
\vspace{2ex}
\caption{\label{prompting}TTCW Fluency3 (Language Proficiency \& Literary Devices) }
\vspace{-5ex}
\end{table*}


% ==================================================



\begin{table*}[!ht]
\centering
\small
% \def\arraystretch{1.15}
\begin{tabular}{|l|l|}
\hline
\begin{tabular}[c]{@{}l@{}}Expert \\ Measure\end{tabular}               & Does the end of the story feel natural and earned, as opposed to arbitrary or abrupt?                                                                                                                                    \\ \hline
\begin{tabular}[c]{@{}l@{}}Expanded\\ Expert\\ Measure (M)\end{tabular} & \begin{tabular}[c]{@{}l@{}}If the writer ends the piece simply because they are 'tired of writing', the conclusion might feel abrupt, disjointed, \\or unfulfilling to the reader. It suggests a rushed ending, where plot threads might be left unresolved and character \\arcs incomplete.\\ \\ 

Conversely, if the writer concludes because they've reached `the moment the entire piece has been leading readers \\towards', it implies a well-considered and purposeful ending. The events, character development, and themes \\throughout the story have built towards this climactic moment, providing a satisfying resolution to the reader.\\ \\ 

A strong ending offers a sense of closure, ties up the central conflicts or questions of the story, and generally \\leaves the reader feeling that the narrative journey was worthwhile and complete.\end{tabular} \\ \hline
\begin{tabular}[c]{@{}l@{}}Human\\ Instruction\end{tabular}             & \begin{tabular}[c]{@{}l@{}}\{\{M\}\}\\ \\ Based on the story that you just read, answer the following question.\\ \textit{\color{blue}Does the end of the story feel natural and earned, as opposed to arbitrary or abrupt?}\\ -Yes \\ -No \\\\ Reasoning : \end{tabular}                                                                       \\ \hline
\begin{tabular}[c]{@{}l@{}}LLM\\ Instruction\end{tabular}               & \begin{tabular}[c]{@{}l@{}}\{\{M\}\}\\ \\ Given the story above, answer the following question. Please first explain your reasoning step by step \\ and then given an answer between 'Yes' or 'No' only\\ \\ \textit{\color{blue} Q) Does the end of the story feel natural and earned, as opposed to arbitrary or abrupt?}\end{tabular}                                                                                                                                                                                                                    \\ \hline
\end{tabular}
\vspace{2ex}
\caption{\label{prompting}TTCW Fluency4 (Narrative Ending) }
\vspace{-5ex}
\end{table*}



% ==================================================



\begin{table*}[!ht]
\centering
\small
% \def\arraystretch{1.15}
\begin{tabular}{|l|l|}
\hline
\begin{tabular}[c]{@{}l@{}}Expert \\ Measure\end{tabular}               & Do the different elements of the story work together to form a unified, engaging, and satisfying whole?                                                                                                                                     \\ \hline
\begin{tabular}[c]{@{}l@{}}Expanded\\ Expert\\ Measure (M)\end{tabular} & \begin{tabular}[c]{@{}l@{}}A well-crafted story usually follows a logical path, where the events in the beginning set up the middle, which then\\ logically leads to the end. Every scene, character action, and piece of dialogue should serve the story and propel it \\forward. Well-written stories have an underlying the unity that binds the elements together. The themes, plotlines, \\character arcs, and other elements of the story interweave to create a harmonious whole. A story with 'disorder'\\ might feel disjointed, with characters, scenes, etc that don't connect or contribute to the overall narrative.\end{tabular} \\ \hline
\begin{tabular}[c]{@{}l@{}}Human\\ Instruction\end{tabular}             & \begin{tabular}[c]{@{}l@{}}\{\{M\}\}\\ \\ Based on the story that you just read, answer the following question.\\ \textit{\color{blue}Do the different elements of the story work together to form a unified, engaging, and satisfying whole?}\\ -Yes \\ -No \\\\ Reasoning : \end{tabular}                                                                       \\ \hline
\begin{tabular}[c]{@{}l@{}}LLM\\ Instruction\end{tabular}               & \begin{tabular}[c]{@{}l@{}}\{\{M\}\}\\ \\ Given the story above, answer the following question. Please first explain your reasoning step by step and then \\give an answer between 'Yes' or 'No' only\\ \\ \textit{\color{blue} Q) Do the different elements of the story work together to form a unified, engaging, and satisfying whole?}\end{tabular}                                                                                                                                                                                                                                 \\ \hline
\end{tabular}
\vspace{2ex}
\caption{\label{prompting}TTCW Fluency5 (Understandability \& Coherence) }
\vspace{-5ex}
\end{table*}


% ==================================================



\begin{table*}[!ht]
\centering
\small
% \def\arraystretch{1.15}
\begin{tabular}{|l|l|}
\hline
\begin{tabular}[c]{@{}l@{}}Expert \\ Measure\end{tabular}               & \begin{tabular}[c]{@{}l@{}}Does the story provide diverse perspectives, and if there are unlikeable characters, are their perspectives \\presented convincingly and accurately? \end{tabular}                                                                                                                                     \\ \hline
\begin{tabular}[c]{@{}l@{}}Expanded\\ Expert\\ Measure (M)\end{tabular} & \begin{tabular}[c]{@{}l@{}}A good writer can convincingly and accurately depict a wide range of character viewpoints, including those of\\ characters who may be morally ambiguous, difficult, or otherwise unappealing.\\ \\ 

This can involve diving into the mindset of characters who may act or think in ways that the reader, or even \\the writer, finds objectionable or repugnant. It involves understanding their motivations, their beliefs, and the \\reasons behind their actions, and then conveying these elements in a way that is believable and consistent.\\ \\ 

The purpose of doing so is not to justify or endorse these perspectives, but rather to create complex, three-\\dimensional characters who contribute to the richness and depth of the story. This can also serve to \\challenge the reader, provoke thought, and provide insights into different aspects of the human experience.\end{tabular} \\ \hline
\begin{tabular}[c]{@{}l@{}}Human\\ Instruction\end{tabular}             & \begin{tabular}[c]{@{}l@{}}\{\{M\}\}\\ \\ Based on the story that you just read, answer the following question.\\ \textit{\color{blue}Does the story provide diverse perspectives, and if there are unlikeable characters, are their perspectives presented} \\ \textit{\color{blue}convincingly and accurately?}\\ -Yes \\ -No \\\\ Reasoning : \end{tabular}                                                                       \\ \hline
\begin{tabular}[c]{@{}l@{}}LLM\\ Instruction\end{tabular}               & \begin{tabular}[c]{@{}l@{}}\{\{M\}\}\\ \\ Given the story above, answer the following question. Please first explain your reasoning step by step and then \\give an answer between 'Yes' or 'No' only\\ \\ \textit{\color{blue} Q) Does the story provide diverse perspectives, and if there are unlikeable characters, are their perspectives presented}\\\textit{\color{blue} convincingly and accurately?}\end{tabular}                                                                                                                                                                                                                                 \\ \hline
\end{tabular}
\vspace{2ex}
\caption{\label{prompting}TTCW Flexibility1 (Perspective \& Voice Flexibility) }
\vspace{-5ex}
\end{table*}


% ==================================================




\begin{table*}[!ht]
\centering
\small
% \def\arraystretch{1.15}
\begin{tabular}{|l|l|}
\hline
\begin{tabular}[c]{@{}l@{}}Expert \\ Measure\end{tabular}               & \begin{tabular}[c]{@{}l@{}}Does the story achieve a good balance between interiority and exteriority, in a way that feels \\emotionally flexible? \end{tabular}                                                                                                                                     \\ \hline
\begin{tabular}[c]{@{}l@{}}Expanded\\ Expert\\ Measure (M)\end{tabular} & \begin{tabular}[c]{@{}l@{}}`Emotional flexibility' is asking whether the piece of writing effectively balances action and introspection, and \\if it portrays a broad and realistic spectrum of emotions.\\ \\

`Exteriority' refers to the observable actions, behaviors, or dialogue of a character, and the physical or visible \\aspects of the setting, plot, and conflicts.\\ \\

`Interiority', on the other hand, pertains to the inner life of a character — their thoughts, feelings, memories, \\and subjective experiences.\\ \\

A balance between these two aspects is crucial in creating well-rounded characters and compelling narratives. \\If a piece is too heavy on exteriority, it may feel shallow or lack emotional depth. If it leans too much on\\ interiority, it could become overly introspective and potentially lose the momentum of the plot.
\end{tabular} \\ \hline
\begin{tabular}[c]{@{}l@{}}Human\\ Instruction\end{tabular}             & \begin{tabular}[c]{@{}l@{}}\{\{M\}\}\\ \\ Based on the story that you just read, answer the following question.\\ \textit{\color{blue}Does the story achieve a good balance between interiority and exteriority, in a way that feels emotionally flexible?}\\ -Yes \\ -No \\\\ Reasoning : \end{tabular}                                                                       \\ \hline
\begin{tabular}[c]{@{}l@{}}LLM\\ Instruction\end{tabular}               & \begin{tabular}[c]{@{}l@{}}\{\{M\}\}\\ \\ Given the story above, answer the following question. Please first explain your reasoning step by step and \\then give an answer between 'Yes' or 'No' only\\ \\ \textit{\color{blue}Q) Does the story achieve a good balance between interiority and exteriority, in a way that feels} \\\textit{\color{blue}emotionally flexible?}\end{tabular}                                                                                                                                                                                                                                 \\ \hline
\end{tabular}
\vspace{2ex}
\caption{\label{prompting}TTCW Flexibility2 (Emotional Flexibility) }
\vspace{-5ex}
\end{table*}


% ==================================================




\begin{table*}[!ht]
\centering
\small
% \def\arraystretch{1.15}
\begin{tabular}{|l|l|}
\hline
\begin{tabular}[c]{@{}l@{}}Expert \\ Measure\end{tabular}               & \begin{tabular}[c]{@{}l@{}}Does the story contain turns that are both surprising and appropriate? \end{tabular}                                                                                                                                     \\ \hline
\begin{tabular}[c]{@{}l@{}}Expanded\\ Expert\\ Measure (M)\end{tabular} & \begin{tabular}[c]{@{}l@{}}`Surprising': This refers to the element of unpredictability in a narrative. A good story often has plot twists, \\character developments, or thematic revelations that surprise the reader, subverting their expectations in a \\thrilling way.It's about keeping readers engaged and curious, never fully knowing what's going to happen next.\\ \\ 

`Appropriate': Despite the surprises and twists, the turns in the story must also make sense within the established \\context of the story's universe, its characters, and its themes. This means that even though an event might be \\surprising, it should feel appropriate or fitting in hindsight. It shouldn't feel like the writer has broken the rules \\they've set up, or made a character behave inconsistently without reason, simply for the sake of shock value.\\ \\ 

So when someone wonders if a writer can make turns that are 'both surprising and appropriate', they're asking \\if the writer can strike this balance between unexpectedness and coherence, keeping the reader on their toes \\while maintaining a believable, satisfying narrative. \end{tabular} \\ \hline
\begin{tabular}[c]{@{}l@{}}Human\\ Instruction\end{tabular}             & \begin{tabular}[c]{@{}l@{}}\{\{M\}\}\\ \\ Based on the story that you just read, answer the following question.\\ \textit{\color{blue}Does the story contain turns that are both surprising and appropriate?}\\ -Yes \\ -No \\\\ Reasoning: \end{tabular}                                                                       \\ \hline
\begin{tabular}[c]{@{}l@{}}LLM\\ Instruction\end{tabular}               & \begin{tabular}[c]{@{}l@{}}\{\{M\}\}\\ \\ Given the story above, list each element in the story that is intended to be surprising. For each, decide whether the\\ surprising element remains appropriate with respect to the entire story. Then overall, give your reasoning \\about the question below and give an answer to it between 'Yes' or 'No' only\\ \\ \textit{\color{blue} Q) Does the story contain turns that are both surprising and appropriate?}\end{tabular}                                                                                                                                                                                                                                 \\ \hline
\end{tabular}
\vspace{2ex}
\caption{\label{prompting}TTCW Flexibility3 (Structural Flexibility) }
\vspace{-5ex}
\end{table*}


% ==================================================






\begin{table*}[!ht]
\centering
\small
% \def\arraystretch{1.15}
\begin{tabular}{|l|l|}
\hline
\begin{tabular}[c]{@{}l@{}}Expert \\ Measure\end{tabular}               & \begin{tabular}[c]{@{}l@{}}Will an average reader of this story obtain a unique and original idea from reading it? \end{tabular}                                                                                                                                     \\ \hline
\begin{tabular}[c]{@{}l@{}}Expanded\\ Expert\\ Measure (M)\end{tabular} & \begin{tabular}[c]{@{}l@{}}If a story is good, the reader gains new insights, perspectives, or knowledge from it. This doesn't necessarily\\ mean factual information but could relate to a deeper understanding of human nature, cultural insights,\\ unique viewpoints, or even the exploration of new ideas and themes. Essentially, it's about what\\ the reader takes away from the story beyond just the plot.\\ \\ 

A good story has lasting impacts on its readers and the society. It is meant to entertain, inform, provoke \\thought, challenge beliefs, provide comfort, or raise awareness on specific issues.
 \end{tabular} \\ \hline
\begin{tabular}[c]{@{}l@{}}Human\\ Instruction\end{tabular}             & \begin{tabular}[c]{@{}l@{}}\{\{M\}\}\\ \\ Based on the story that you just read, answer the following question.\\ \textit{\color{blue}Will an average reader of this story obtain a unique and original idea from reading it?}\\ -Yes \\ -No \\\\ Reasoning : \end{tabular}                                                                       \\ \hline
\begin{tabular}[c]{@{}l@{}}LLM\\ Instruction\end{tabular}               & \begin{tabular}[c]{@{}l@{}}\{\{M\}\}\\ \\ Given the story above, list out elements that are unique takeaways of this story for the reader. Then overall, \\give your reasoning about the question below and give an answer to it between 'Yes' or 'No' only\\ \\ \textit{\color{blue} Q) Will an average reader of this story obtain a unique and original idea from reading it?}\end{tabular}                                                                                                                                                                                                                                 \\ \hline
\end{tabular}
\vspace{2ex}
\caption{\label{prompting}TTCW Originality1 (Originality in Theme and Content) }
\vspace{-3ex}
\end{table*}


% ==================================================








\begin{table*}[!ht]
\centering
\small
% \def\arraystretch{1.15}
\begin{tabular}{|l|l|}
\hline
\begin{tabular}[c]{@{}l@{}}Expert \\ Measure\end{tabular}               & \begin{tabular}[c]{@{}l@{}}Is the story an original piece of writing without any cliches?\end{tabular}                                                                                                                                     \\ \hline
\begin{tabular}[c]{@{}l@{}}Expanded\\ Expert\\ Measure (M)\end{tabular} & \begin{tabular}[c]{@{}l@{}}A cliche is an idea, expression, character, or plot that has been overused to the point of losing its original \\meaning or impact. They often become predictable and uninteresting for the reader. Originality suggests\\ that the piece isn't cliche.

 \end{tabular} \\ \hline
\begin{tabular}[c]{@{}l@{}}Human\\ Instruction\end{tabular}             & \begin{tabular}[c]{@{}l@{}}\{\{M\}\}\\ \\ Based on the story that you just read, answer the following question.\\ \textit{\color{blue}Is the story an original piece of writing without any cliches?}\\ -Yes \\ -No \\\\ Reasoning: \end{tabular}                                                                       \\ \hline
\begin{tabular}[c]{@{}l@{}}LLM\\ Instruction\end{tabular}               & \begin{tabular}[c]{@{}l@{}}\{\{M\}\}\\ \\ Given the story above, are there any cliches in the story? If so, list out all the elements in this story that \\are cliche. Then overall, give your reasoning if the piece is negatively impacted by the cliches and give \\an answer to the question below between 'Yes' or 'No' only\\ \\ \textit{\color{blue} Q) Is the story an original piece of writing without any cliches?}\end{tabular}                                                                                                                                                                                                                                 \\ \hline
\end{tabular}
\vspace{2ex}
\caption{\label{prompting}TTCW Originality2 (Originality in Thought) }
\vspace{-5ex}
\end{table*}


% ==================================================




\begin{table*}[!ht]
\centering
\small
% \def\arraystretch{1.15}
\begin{tabular}{|l|l|}
\hline
\begin{tabular}[c]{@{}l@{}}Expert \\ Measure\end{tabular}               & \begin{tabular}[c]{@{}l@{}}Does the story show originality in its form?\end{tabular}                                                                                                                                     \\ \hline
\begin{tabular}[c]{@{}l@{}}Expanded\\ Expert\\ Measure (M)\end{tabular} & \begin{tabular}[c]{@{}l@{}}When someone says that a piece of fiction 'shows an innovative use of form/structure', they're referring to\\ how the writer has chosen to shape and organize the story in an unusual, original, or inventive way. This could \\involve a variety of different elements, including:\\ \\ 

Narrative Structure: This could include unconventional timelines (e.g. a non-linear story, a story told in reverse)\\, multiple perspectives or narrators, or unusual narrative voices (e.g. a story told from the perspective of an \\inanimate object).\\ \\ 

Format: This could be something as simple as using unconventional punctuation or capitalization, or as complex \\as telling a story through a series of letters, diary entries, newspaper clippings, or other documents. In recent years,\\ some authors have even experimented with using social media posts or text messages as a form of narrative structure.\\ \\ 

Genre Hybridity: Combining elements from different genres or sub-genres in unexpected ways can also be seen\\ as an innovative use of form such as Horror-Mystery or Comic Fantasy.\\ \\ 

Plot Structure: Deviating from traditional plot structures such as three-act structure, or following them in unexpected\\ ways.For example, telling a story without a clear resolution, incorporating multiple climaxes or using revelation as a \\device where a surprising, and often shocking, development occurs that was previously kept hidden from the \\characters and/or the audience. It's typically designed to provide new context for interpreting what has previously \\occurred in the story. \\ \\ 

Language and Style: Innovative use of form can also come in the form of unique use of language, style, or \\even typography, such as concrete poetry or writing that visually represents its subject matter on the page.\\ \\ 

The goal of this innovation is often to provide a fresh reader experience, challenge conventional reading\\ expectations, or to create a deeper or more complex exploration of the story's themes.

 \end{tabular} \\ \hline
\begin{tabular}[c]{@{}l@{}}Human\\ Instruction\end{tabular}             & \begin{tabular}[c]{@{}l@{}}\{\{M\}\}\\ \\ Based on the story that you just read, answer the following question.\\ \textit{\color{blue}Does the story show originality in its form?}\\ -Yes \\ -No \\\\ Reasoning: \end{tabular}                                                                       \\ \hline
\begin{tabular}[c]{@{}l@{}}LLM\\ Instruction\end{tabular}               & \begin{tabular}[c]{@{}l@{}}\{\{M\}\}\\ \\ Given the story and the devices mentioned above, list each device used with a short explanation of whether it is \\successful or not. Then overall, give your reasoning about the question below and give an answer to it\\ between 'Yes' or 'No' only\\ \\ \textit{\color{blue} Q) Does the story show originality in its form?}\end{tabular}                                                                                                                                                                                                                                 \\ \hline
\end{tabular}
\vspace{2ex}
\caption{\label{prompting}TTCW Originality3 (Originality in Form) }
\vspace{-5ex}
\end{table*}


% ==================================================




\begin{table*}[!ht]
\centering
\small
% \def\arraystretch{1.15}
\begin{tabular}{|l|l|}
\hline
\begin{tabular}[c]{@{}l@{}}Expert \\ Measure\end{tabular}               & \begin{tabular}[c]{@{}l@{}}Does each character in the story feel developed at the appropriate complexity level, ensuring that no character \\feels like they are present simply to satisfy a plot requirement?\end{tabular}                                                                                                                                     \\ \hline
\begin{tabular}[c]{@{}l@{}}Expanded\\ Expert\\ Measure (M)\end{tabular} & \begin{tabular}[c]{@{}l@{}} A `flat character' is typically a minor character who is not thoroughly developed or who does not undergo \\significant change or growth throughout the story. They often embody or represent a single trait or idea, \\and they're used to advance the plot or highlight certain qualities in other characters.\\ \\ 

A `complex character', also known as a round character, has depth in feelings and passions, has a variety \\of traits of a real human being, and evolves over time. They have their strengths, weaknesses, \\and they learn from their experiences. They tend to be more engaging to the reader, as they mirror \\the complexity of real people.\\ \\ 

In good stories, authors take a character who initially appears to be one-dimensional or stereotypical (flat) and \\add depth to them. This could be done by revealing more about their backstory, introducing unexpected traits \\or motivations, or having them grow and change in response to the events of the story. \\This transformation from a flat to a complex character can make the narrative more engaging and believable.
 \end{tabular} \\ \hline
\begin{tabular}[c]{@{}l@{}}Human\\ Instruction\end{tabular}             & \begin{tabular}[c]{@{}l@{}}\{\{M\}\}\\ \\ Based on the story that you just read, answer the following question.\\  \textit{\color{blue} Q) Does each character in the story feel developed at the appropriate complexity level, ensuring that no character} \\ \textit{\color{blue}feels like they are present simply to satisfy a plot requirement?}\\ -Yes \\ -No \\\\ Reasoning: \end{tabular}                                                                       \\ \hline
\begin{tabular}[c]{@{}l@{}}LLM\\ Instruction\end{tabular}               & \begin{tabular}[c]{@{}l@{}}\{\{M\}\}\\ \\ Given the story above, list each character and the level of development. Then overall, give your reasoning \\about the question below and give an answer to it between 'Yes' or 'No' only\\ \\ 
 \textit{\color{blue} Q) Does each character in the story feel developed at the appropriate complexity level, ensuring that no character} \\ \textit{\color{blue}feels like they are present simply to satisfy a plot requirement?}\end{tabular}                                                                                                                                                                                                                                 \\ \hline
\end{tabular}
\vspace{2ex}
\caption{\label{prompting}TTCW Elaboration2 (Character Development) }
\vspace{-5ex}
\end{table*}


% ==================================================



\begin{table*}[!ht]
\centering
\small
% \def\arraystretch{1.15}
\begin{tabular}{|l|l|}
\hline
\begin{tabular}[c]{@{}l@{}}Expert \\ Measure\end{tabular}               & \begin{tabular}[c]{@{}l@{}}Are there passages in the story that involve subtext and when there is subtext, does it enrich the story's setting \\or does it feel forced?\end{tabular}                                                                                                                                     \\ \hline
\begin{tabular}[c]{@{}l@{}}Expanded\\ Expert\\ Measure (M)\end{tabular} & \begin{tabular}[c]{@{}l@{}} `Surface' level: This is the most apparent and straightforward level of a story. It includes the visible actions, \\explicit dialogue, and clear descriptions. This is what literally happens in the plot: the characters' actions, events, \\and the apparent consequences.\\ \\ 

`Subtext' level: This is the underlying or implicit meaning that isn't directly stated but can be inferred from \\the characters'  actions, dialogue, and other elements of the story. Subtext often reveals deeper truths about \\characters, themes, or the overall message of the piece. It could be a hidden motive, an unstated\\ emotion, a cultural commentary, or a symbolic meaning.\\ \\ 

For example, in a conversation between two characters, the surface text might be polite and cordial, but the \\subtext \\discerned from the characters' nonverbal cues, previous interactions, or the context of their conversation\\ — could suggest tension or hostility.\\ \\ 

Effective fiction often operates on both levels. The surface text keeps the reader engaged with the plot and \\characters, while the subtext provides depth, complexity, and additional layers of interpretation, \\contributing to a richer and more rewarding reading experience.
 \end{tabular} \\ \hline
\begin{tabular}[c]{@{}l@{}}Human\\ Instruction\end{tabular}             & \begin{tabular}[c]{@{}l@{}}\{\{M\}\}\\ \\ Based on the story that you just read, answer the following question.\\  \textit{\color{blue} Q) Are there passages in the story that involve subtext and when there is subtext, does it enrich the story's setting} \\ \textit{\color{blue} or does it feel forced?}\\ -Yes \\ -No \\\\ Reasoning: \end{tabular}                                                                       \\ \hline
\begin{tabular}[c]{@{}l@{}}LLM\\ Instruction\end{tabular}               & \begin{tabular}[c]{@{}l@{}}\{\{M\}\}\\ \\ Given the story above, answer the following question. Please first explain your reasoning step by step \\and then give an answer between 'Yes' or 'No' only\\ \\ 
 \textit{\color{blue} Q)Are there passages in the story that involve subtext and when there is subtext, does it enrich the story's setting} \\ \textit{\color{blue} or does it feel forced?}\end{tabular}                                                                                                                                                                                                                                 \\ \hline
\end{tabular}
\vspace{2ex}
\caption{\label{prompting}TTCW Elaboration3 (Rhetorical Complexity) }
\vspace{-5ex}
\end{table*}


% ==================================================


\end{document}