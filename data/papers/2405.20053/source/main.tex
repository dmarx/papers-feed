\documentclass{article}


% if you need to pass options to natbib, use, e.g.:
    \PassOptionsToPackage{numbers, compress}{natbib}
% before loading neurips_2024


% ready for submission
% \usepackage{neurips_2024}
% \newcommand{\anon}[1]{\censor{#1}}


% to compile a preprint version, e.g., for submission to arXiv, add add the
% [preprint] option:
   \usepackage[preprint]{neurips_2024}
   \newcommand{\anon}[1]{{#1}}


% to compile a camera-ready version, add the [final] option, e.g.:
%    \usepackage[final]{neurips_2024}
%    \newcommand{\anon}[1]{{#1}}

% to avoid loading the natbib package, add option nonatbib:
%    \usepackage[nonatbib]{neurips_2024}
%    \newcommand{\anon}[1]{{#1}}


\usepackage[utf8]{inputenc} % allow utf-8 input
\usepackage[T1]{fontenc}    % use 8-bit T1 fonts
\usepackage{hyperref}       % hyperlinks
\usepackage{url}            % simple URL typesetting
\usepackage{booktabs}       % professional-quality tables
\usepackage{amsfonts}       % blackboard math symbols
\usepackage{nicefrac}       % compact symbols for 1/2, etc.
\usepackage{microtype}      % microtypography
\usepackage{amsmath} 
\usepackage{amsthm}
\usepackage{caption}
\usepackage{subcaption}
\usepackage{graphicx}
\usepackage[export]{adjustbox}
\usepackage{wrapfig}

\usepackage{algorithm}
\usepackage{algpseudocode}
\usepackage{tabularx}
\usepackage{xfrac}
\usepackage{makecell}

\usepackage[dvipsnames]{xcolor}
\newcommand{\red}[1]{{\color{red}#1}}
\newcommand{\todo}[1]{{\color{red}#1}}
\newcommand{\TODO}[1]{\textbf{\color{red}[TODO: #1]}}

\usepackage{censor}

\title{Would I Lie To You? Inference Time Alignment of Language Models using Direct Preference Heads}


% The \author macro works with any number of authors. There are two commands
% used to separate the names and addresses of multiple authors: \And and \AND.
%
% Using \And between authors leaves it to LaTeX to determine where to break the
% lines. Using \AND forces a line break at that point. So, if LaTeX puts 3 of 4
% authors names on the first line, and the last on the second line, try using
% \AND instead of \And before the third author name.


\author{%
  Avelina Asada Hadji-Kyriacou \\
  Department of Computer Science \\
  University of St Andrews \\
  College Gate, St Andrews, KY16 9AJ \\
  \texttt{lhk3@st-andrews.ac.uk} \\
  \And
  Ognjen Arandjelović \\
  Department of Computer Science \\
  University of St Andrews \\
  College Gate, St Andrews, KY16 9AJ \\
  \texttt{oa7@st-andrews.ac.uk} \\
  % examples of more authors
  % \And
  % Coauthor \\
  % Affiliation \\
  % Address \\
  % \texttt{email} \\
  % \AND
  % Coauthor \\
  % Affiliation \\
  % Address \\
  % \texttt{email} \\
  % \And
  % Coauthor \\
  % Affiliation \\
  % Address \\
  % \texttt{email} \\
  % \And
  % Coauthor \\
  % Affiliation \\
  % Address \\
  % \texttt{email} \\
}


\begin{document}


\maketitle

\newtheorem{theorem}{Theorem}

\begin{abstract}
Pre-trained Language Models (LMs) exhibit strong zero-shot and in-context learning capabilities; however, their behaviors are often difficult to control. By utilizing Reinforcement Learning from Human Feedback (RLHF), it is possible to fine-tune unsupervised LMs to follow instructions and produce outputs that reflect human preferences. Despite its benefits, RLHF has been shown to potentially harm a language model's reasoning capabilities and introduce artifacts such as hallucinations where the model may fabricate facts. To address this issue we introduce \textit{Direct Preference Heads} (DPH), a fine-tuning framework that enables LMs to learn human preference signals through an auxiliary reward head without directly affecting the output distribution of the language modeling head. We perform a theoretical analysis of our objective function and find strong ties to Conservative Direct Preference Optimization (cDPO). Finally we evaluate our models on GLUE, RACE, and the GPT4All evaluation suite and demonstrate that our method produces models which achieve higher scores than those fine-tuned with Supervised Fine-Tuning (SFT) or Direct Preference Optimization (DPO) alone.
\end{abstract}

\section{Introduction}
\label{sec:intro}

Neural networks have undergone rapid development in various computer vision tasks such as image classification, detection and segmentation. While their impressive performance has powered many applications, a roaring trend is to pursue fast neural networks with low latency and high throughput for great user experiences, instant responses, safety reasons, etc.

How to be fast? Instead of asking for more costly computing devices, researchers and practitioners prefer to design cost-effective fast neural networks with reduced computational complexity, mainly measured in the number of {\bf fl}oating-point {\bf op}eration{\bf s} (FLOPs)\footnote{We follow a widely adopted definition of FLOPs, as the number of multiply-adds~\cite{zhang2018shufflenet,liu2022convnet}.}. MobileNets~\cite{howard2017mobilenets,sandler2018mobilenetv2,howard2019searching},
ShuffleNets~\cite{zhang2018shufflenet,ma2018shufflenet} and GhostNet~\cite{han2020ghostnet}, among others, leverage the depthwise convolution (DWConv)~\cite{sifre2014rigid} and/or group convolution (GConv)~\cite{krizhevsky2012imagenet} to extract spatial features. However, in the effort to reduce FLOPs, the operators often suffer from the side effect of increased memory access. MicroNet~\cite{li2021micronet} further decomposes and sparsifies the network to push its FLOPs to an extremely low level. Despite its improvement in FLOPs, this approach experiences inefficient fragmented computation. Besides, the above networks are often accompanied by additional data manipulations, such as concatenation, shuffling, and pooling, whose running time tends to be significant for tiny models.

\begin{figure}
    \centering
    \includegraphics[width=1\linewidth]{figures/PConv-cropped.pdf}
    \vspace{-0.2in}
    \caption{Our partial convolution (PConv) is fast and efficient by applying filters on only a few input channels while leaving the remaining ones untouched. PConv obtains lower FLOPs than the regular convolution and higher FLOPS than the depthwise/group convolution.}
    \label{fig: PConv}
    \vspace{-0.05in}
\end{figure}

\begin{figure*}
    \centering
    \includegraphics[width=1\linewidth]{figures/FLOPS_latency_vs_FLOPs-cropped.pdf}
    \vspace{-0.3in}
    \caption{(a) FLOPS under varied FLOPs on CPU. Many existing neural networks suffer from low computational speed issues. Their effective FLOPS are lower than the popular ResNet50. By contrast, our FasterNet attains higher FLOPS. (b) Latency under varied FLOPs on CPU. Our FasterNet obtains lower latency than others with the same amount of FLOPs.}
    \label{fig:FLOPS(latency)_vs_FLOPs}
    \vspace{-0.05in}
\end{figure*}

Apart from the above pure convolutional neural networks (CNNs),  there is an emerging interest in making vision transformers (ViTs)~\cite{dosovitskiy2020image} and multilayer perceptrons (MLPs) architectures~\cite{tolstikhin2021mlp} smaller and faster. For example, MobileViTs~\cite{mehta2021mobilevit,mehta2022separable,wadekar2022mobilevitv3} and MobileFormer~\cite{chen2022mobile} reduce the computational complexity by combining DWConv with a modified attention mechanism. However, they still suffer from the aforementioned issue with DWConv and also need dedicated hardware support for the modified attention mechanism. The use of advanced yet time-consuming normalization and activation layers may also limit their speed on devices.

All these issues together lead to the following question: Are
these ``fast'' neural networks really fast? To answer this, we examine the relationship between latency and FLOPs, which is captured by 
\begin{equation}
  Latency = \frac{FLOPs}{FLOPS},
  \label{eq:latency_FLOPs}
\end{equation}
where FLOPS is short for {\bf fl}oating-point {\bf op}erations per {\bf s}econd, as a measure of the effective computational speed. While there are many attempts to reduce FLOPs, they seldom consider optimizing FLOPS at the same time to achieve truly low latency. To better understand the situation, we compare the FLOPS of typical neural networks on an Intel CPU. The results in~\cref{fig:FLOPS(latency)_vs_FLOPs} show that many existing neural networks suffer from low FLOPS, and their FLOPS is generally lower than the popular ResNet50. With such low FLOPS, these ``fast'' neural networks are actually not fast enough.
Their reduction in FLOPs cannot be translated into the exact amount of reduction in latency. In some cases, there is no improvement, and it even leads to worse latency. For example, CycleMLP-B1~\cite{chen2021cyclemlp} has half of FLOPs of ResNet50~\cite{he2016deep} but runs more slowly (\ie, CycleMLP-B1 \vs ResNet50: 116.1ms \vs 73.0ms). Note that this discrepancy between FLOPs and latency has also been noticed in previous works~\cite{ma2018shufflenet,mehta2021mobilevit} but remains unresolved partially because they employ the DWConv/GConv and various data manipulations with low FLOPS. It is deemed there are no better alternatives available.

This paper aims to eliminate the discrepancy by developing a simple yet fast and effective operator that maintains high FLOPS with reduced FLOPs. Specifically, we reexamine existing operators, particularly  DWConv, in terms of the computational speed -- FLOPS. We uncover that the main reason causing the low FLOPS issue is \emph{frequent memory access}. We then propose a novel partial convolution (PConv) as a competitive alternative that reduces the computational redundancy as well as the number of memory access. \cref{fig: PConv} illustrates the design of our PConv. It takes advantage of redundancy within the feature maps and systematically applies a regular convolution (Conv) on only a part of the input channels while leaving the remaining ones untouched. By nature, PConv has lower FLOPs than the regular Conv while having higher FLOPS than the DWConv/GConv. In other words, PConv better exploits the on-device computational capacity. PConv is also effective in extracting spatial features as empirically validated later in the paper. 


We further introduce FasterNet, which is primarily built upon our PConv, as a new family of networks that run highly fast on various devices. In particular, our FasterNet achieves state-of-the-art performance for classification, detection, and segmentation tasks while having much lower latency and higher throughput. For example, our tiny FasterNet-T0 is $2.8\times$, $3.3\times$, and $2.4\times$ faster than MobileViT-XXS~\cite{mehta2021mobilevit} on GPU, CPU, and ARM processors, respectively, while being 2.9\% more accurate on ImageNet-1k. Our large FasterNet-L achieves 83.5\% top-1 accuracy, on par with the emerging Swin-B~\cite{liu2021swin}, while offering 36\% higher throughput on GPU and saving 37\% compute time on CPU. To summarize, our contributions are as follows:
\begin{itemize}
\itemsep0em 
\item We point out the importance of achieving higher FLOPS beyond simply reducing FLOPs for faster neural networks.
\item We introduce a simple yet fast and effective operator called PConv, which has a high potential to replace the existing go-to choice, DWConv.
\item We introduce FasterNet which runs favorably and universally fast on a variety of devices such as GPU, CPU, and ARM processors.
\item We conduct extensive experiments on various tasks and validate the high speed and effectiveness of our PConv and FasterNet.
\end{itemize}
\section{Prior Approaches}
Prior approaches to language model alignment involve directly optimizing the logits produced by the language modelling head to increase the likelihood of producing preferable responses while decreasing the likelihood of undesirable responses. %This is often realized through RLHF or contrastive methods which are outlined below.

\subsection{Reinforcement Learning from Human Feedback (RLHF)}
Reinforcement Learning from Human Feedback seeks to learn a reward model from human feedback on completions generated by a language model which can be used to align an LM with human preferences. A typical RLHF pipeline consists of 3 steps: (1) supervised fine-tuning, (2) preference sampling and reward modelling, and (3) RL fine-tuning.

\textbf{Supervised Fine-Tuning\ } The first step of a standard RLHF pipeline is fine-tuning a pre-trained LM on high quality data for downstream tasks to obtain a model $\pi^\text{SFT}$.

\textbf{Reward Modelling\ } Next, the SFT model is prompted with input tokens $x$ to produce completions $y$. These answers are then rated by human labelers which rate the answers based on one or more criteria. A reward model $r_\phi(x,y)$ is then trained to estimate the scores assigned by human labelers using maximum likelihood estimation.

\textbf{RL Fine-Tuning\ } During the RL phase the learned reward function is used to provide feedback to the language model using the following optimization problem
\begin{equation} \label{eq:rl-objective}
    \underset{\pi_\theta}{\max}\, \mathbb{E}_{x \sim \mathcal{D},y \sim \pi_\theta(y|x)}
    \left[ r_\phi(x,y) \right]
    - \beta D_{\text{KL}}\left[ \pi_\theta(y|x)||\pi_\text{ref}(y|x) \right]
\end{equation}
where $\beta$ controls the deviation from the base reference policy $\pi_{\text{ref}}$, which is typically initialized from $\pi^\text{SFT}$. Due to the non-differentiable nature of language generation this objective must be optimized using a reinforcement learning algorithm such as PPO \cite{schulman2017proximal}.

\subsection{Direct Preference Optimization (DPO)}
Direct Preference Optimization was introduced as a reparameterization of RLHF which eliminates both the sampling stage and the reward modelling stages and reformulates alignment procedure as a loss function which can be optimized directly on a dataset of pairs of preferred and dispreferred completions to given prompts. This allows DPO to stably and efficiently converge on an optimal policy using what is effectively a classification loss over positive and negative pairs.

Given a dataset $\{(x,y_w,y_l)\}$ where $x$ is the prompt and $y_w,y_l$ are the preferred and dispreferred completions, we introduce the following loss function:
%the following loss function is formulated:
\begin{equation}
    \mathcal{L}_\text{DPO}(x,y_w,y_l)=
    -\log\sigma
    \left(
        \beta\log \frac{\pi_\theta(y_w|x)}{\pi_\text{ref}(y_w|x)} -
        \beta\log \frac{\pi_\theta(y_l|x)}{\pi_\text{ref}(y_l|x)}
    \right)
    \label{eq:dpo}
\end{equation}
where $\pi_\theta(y_*|x)$ and $\pi_\text{ref}(y_*|x)$ are the probabilities of completions $y_*$ for prompt $x$ given by the policy model and reference models respectively, and the $\beta$ parameter controls the deviation from the reference policy.

There also exists an augmentation of DPO namely Conservative DPO (cDPO) \cite{cdpo} which is designed to be more robust to noisy labels through the introduction of label smoothing parameter $\epsilon$. The objective function for cDPO is given by:% the following formula:
\begin{equation}
    \mathcal{L}_\text{cDPO}(x,y_w,y_l)=
    (1-\epsilon) \mathcal{L}_\text{DPO}(x,y_w,y_l) +
    \epsilon\,\mathcal{L}_\text{DPO}(x,y_l,y_w)
    \label{eq:cdpo}
\end{equation}
\section{Direct Preference Heads}
The hypothesis underlying the Direct Preference Optimization framework of Rafailov et al. \cite{rafailov2023direct} is that a ``language model is secretly a reward model'' thereby making the purpose of Direct Preference Heads to exploit this and extract explicit reward signals without the need of an \emph{additional} reward model.

% Unlike other RLHF pipelines such as PPO \cite{schulman2017proximal}, these rewards are not used for RL fine-tuning. Instead, the DPH rewards are to be used to prune candidate generations sampled from the LM at inference time to select the candidate which aligns most with human preferences.

% This makes DPH an excellent choice for small language models which are (1) more lightweight -- and therefor can be efficiently used to generate multiple samples -- and, (2) are more prone to degradation when aligned using typical RL techniques \cite{bekbayev2023poison, bai2022training}.

\subsection{Reward Head}
% \begin{wrapfigure}{R}{5.5cm}
%     \centering
%     \includegraphics[width=5.5cm,trim={0 0 0 1cm}]{figures/mode_diagram.png}
%     \caption{Architecture of an LM augmented with DPH.}
% \end{wrapfigure}
To obtain the rewards from a sequence $x;y$ three components are required: an aggregated hidden state $h$ which is conditioned on the intermediate representations of the language model, a pooling function $f$ which transforms the hidden state, and a learnable vector $w_{dph}$ with the same dimension as the output of $f$. We then compute the reward $r$ as follows:
\begin{equation}
    r=f(h) \cdot w_{dph}
\end{equation}
To obtain the hidden state we take the output of the last transformer layer for the final token of the sequence, and we experiment with three choices of $f$: (1) the identity mapping following the convention established by OpenAI's GPT for sequence classification \cite{Radford2018ImprovingLU}, (2) a learnable affine projection with $\tanh$ nonlinearity following BERT's pooling function \cite{devlin2019bert}, and (3) an inverted bottleneck FFN with SwiGLU activation mirroring the FFN blocks used within the transformer backbone followed by $\tanh$ nonlinearity \cite{shazeer2020glu}.
% \begin{subequations} \label{eq:pooling_funcs}
%     \begin{alignat}{2}
%     f_{\text{GPT}}(h) &= h \\
%     f_{\text{BERT}}(h) &= \tanh{(Wx+b)} \\
%     \todo{f_{\text{SwiGLU}}(h)} &= \todo{\tanh{(W_3((W_2x + b_2)\otimes\text{SiLU}(W_1x + b_1)) + b_3)}}
%     \end{alignat}
% \end{subequations}

\subsection{Objective Function}
We formulate two novel objective functions for our method: a separable objective which maximises positive rewards and minimises negative rewards, and a contrastive objective which maximises the margin between positive and negative rewards. The loss landscapes are illustrated by Figure~\ref{fig:both_dph_loss} in the appendix.

\subsubsection{Separable DPH}
The Separable DPH loss function given by \eqref{eq:sep_dph} is a function of the preferred and dispreferred rewards $r_w,r_l$, and the label smoothing parameter $0 \leq \epsilon \leq 0.5$ which controls the reward margin.
\begin{equation} \label{eq:sep_dph}
    \mathcal{L}_\text{SepDPH}(r_w,r_l)=
    - \left[ (1-\epsilon) \log \sigma(r_w) + \epsilon\, \log \sigma(-r_w) \right]
    - \left[ \epsilon\, \log \sigma(r_l) + (1-\epsilon) \log \sigma(-r_l) \right]
\end{equation}

\begin{theorem} \label{thrm:sep_dph_convergence}
For all $\epsilon \in (0,0.5]$ the objective function $\mathcal{L}_\text{SepDPH}$ is convex and will optimize the policy $\pi_\theta$ such that the preferred rewards $r_w$ produced by the preference head converge towards $\log\tfrac{1-\epsilon}{\epsilon}$ and the dispreferred rewards $r_l$ converge to $\log\tfrac{\epsilon}{1-\epsilon}$. %\todo{When $\epsilon=0$ the objective function will never converge and gradients $\nabla_\theta \mathcal{L}_\text{SepDPH}$ will never be zero.}
\end{theorem}

This can be proven by observing the first and second partial derivatives of the loss function with respect to the rewards. The first partial derivative is equal to zero at the points $r_w=log\tfrac{1-\epsilon}{\epsilon}$ and $r_l=\log\tfrac{\epsilon}{1-\epsilon}$ respectively, and the second partial derivative is strictly positive for all values of $r_w,r_l$. A full proof is included in Appendix~\ref{sec:sep_dph_proof}.

\subsubsection{Contrastive DPH}
Like Separable DPH, the loss function for Contrastive DPH given by \eqref{eq:con_dph} is function of the preferred and dispreferred rewards $r_w,r_l$ and the label smoothing parameter $0 \leq \epsilon \leq 0.5$. This version of the loss function optimizes the \textit{relative} margin between the rewards rather than optimizing the \textit{absolute} positive and negative rewards as in Separable DPH.
\begin{equation}
    \mathcal{L}_\text{ConDPH}(r_w,r_l)=
    - (1-\epsilon) \log\sigma(r_w-r_l)
    -  \epsilon\, \log\sigma(r_l-r_w)
    \label{eq:con_dph}
\end{equation}

% \vspace{3pt}

\begin{theorem} \label{thrm:con_dph_convergence}
For all $\epsilon \in (0,0.5]$ the objective function $\mathcal{L}_\text{ConDPH}$ is convex and will optimize the policy $\pi_\theta$ such that the difference between preferred rewards $r_w$ and dispreferred rewards $r_l$ produced by the preference head will converge to a fixed margin, given by $r_{\Delta}=r_w-r_l=\log\tfrac{1-\epsilon}{\epsilon}$. %\todo{When $\epsilon=0$ the objective function will never converge and gradients $\nabla_\theta \mathcal{L}_\text{ConDPH}$ will never be zero.}
\end{theorem}

This can be proven by reparameterising the loss function such that $r_{\Delta}=r_w-r_l$ and by then considering the first and second partial derivatives with respect to this reward margin. It can be observed that the first partial derivative is equal to zero when $r_{\Delta}=\log\tfrac{1-\epsilon}{\epsilon}$, and the second partial derivative is strictly positive for all values of $r_{\Delta}$. A full proof is included in Appendix~\ref{sec:con_dph_proof}.

\subsubsection{Relation to cDPO}
The properties of both Contrastive DPH and Seperable DPH show a strong relationship with Conservative DPO: SepDPH will converge to optimal \textit{fixed reward margins} above zero for $r_w$ and below zero for $r_l$; ConDPH will converge to optimal \textit{fixed reward margins} between $r_w$ and $r_l$, and cDPO will converge to a \textit{fixed delta from the reference model} \cite{cdpo}. Like Conservative DPO, this makes both Seperable DPH and Contrastive DPH robust to preference label noise and makes training more stable than naive maximum likelihood estimation without label-smoothing.

% This establishes a strong relationship between the \textit{Contrastive DPH objective} and \textit{Conservative DPO} when label smoothing is employed, since ConDPH will converge to an optimal \textit{fixed reward margin} while cDPO will converge to a \textit{fixed delta from the reference model}. Like cDPO, this makes Contrastive DPH robust to preference label noise and will likely make training more stable \cite{cdpo}.

\subsection{Novelty over Traditional Reward Modelling}
Although similar to the reward modelling phase of an RLHF pipeline, DPH has some distinct differences which set it apart. DPH does not require an SFT sampling and human labelling stage meaning it can take advantage of pre-constructed preference datasets such as those used for DPO. Typical RLHF also requires multiple models -- a reward model, a reference model and a policy model -- while DPH requires only a single model to produce both responses and rewards. Unlike other RLHF pipelines such as PPO \cite{schulman2017proximal}, the rewards produced by DPH are not used for RL fine-tuning; instead, the DPH rewards are to be used to prune candidate generations sampled from the LM at inference time to select the candidate which aligns most with human preferences. This makes DPH an excellent choice for small language models which are (1) more lightweight -- and therefore can be efficiently used to generate multiple samples -- and, (2) are more prone to degradation when aligned using typical RL techniques \cite{bekbayev2023poison, bai2022training}.
\section{Experimental Setup and Data} \label{sec:methodology}

\subsection{Datasets}
We make use of a variety of datasets for fine-tuning and evaluation which are outlined below. The specific prompt templates used for fine-tuning and evaluation are described in Appendix~\ref{sec:datasets_cont}.

\textbf{Natural Language Understanding (NLU)\ }
For general NLU we make use of the standard \textbf{GLUE} benchmark \cite{wang2019glue}. The overall score for GLUE is computed by the macro-average of unweighted metric averages for all 9 tasks, however we also include a secondary score which does not included the `problematic' WNLI task following the evaluation used for BERT \cite{devlin2019bert}. We opted to omit WNLI during fine-tuning due to the low sample size. %, overlap of sentences in the training and validation splits, and different label distribution in the test set.

\textbf{Commonsense Reasoning\ }
In accordance with the \textbf{GPT4All} \cite{gpt4all} evaluation suite, we use the following datasets to evaluate commonsense reasoning abilities:
\textbf{HellaSwag} \cite{zellers2019hellaswag},
\textbf{OpenBookQA} \cite{mihaylov2018suit},
\textbf{WinoGrande} \cite{DBLP:journals/corr/abs-1907-10641},
\textbf{ARC} \cite{clark2018think},
\textbf{BoolQ} \cite{clark2019boolq},
and \textbf{PIQA} \cite{bisk2019piqa}.
% \textbf{HellaSwag} \cite{zellers2019hellaswag} -- an adversarially constructed sequence completion task,
% \textbf{OpenBookQA} \cite{mihaylov2018suit} -- a commonsense multiple-choice task,
% \textbf{WinoGrande} \cite{DBLP:journals/corr/abs-1907-10641} -- a fill-in-a-blank task with binary options,
% \textbf{ARC} \cite{clark2018think} -- a collection of multiple-choice grade-school science questions,
% \textbf{BoolQ} \cite{clark2019boolq} -- an open-book yes/no QA task, and
% \textbf{PIQA} \cite{bisk2019piqa} -- a QA task focused on reasoning over physical actions.

% \todo{These tasks are included in the fine-tuning and alignment mixes used to train our models. We use the macro-average of task scores as an indicator of average commonsense reasoning ability, where we use specific validation or test split used by the LM Evaluation Harness \cite{eval-harness} for consistency with other evaluations.}

\textbf{Reading Comprehension\ }
To evaluate reading comprehension abilities we use the \textbf{RACE} dataset \cite{lai2017race}, a multiple-choice task which requires reasoning over provided passages.

% \todo{We include both these tasks in our fine-tuning and alignment mixes, but opt to not include SQuAD as part of our model evaluation results; this is due to the need to reformulate SQuAD as a generative task for causal LMs which introduces a multitude of sampling and generation hyperparameters which must be ablated over. Never-the-less we do include results of a reformulated version of SQuAD V2 which compares the log-probabilities of ground-truth answers from the validation set with `distractor' spans extracted using SpaCy; this is included purely to evaluate the performance of DPH alignment and these results should not be compared with the scores of encoder-only models.}

\textbf{Instruction Following\ }
We include the \textbf{Alpaca} \cite{alpaca}, \textbf{OpenOrca} \cite{OpenOrca}, and \textbf{UltraFeedback} \cite{cui2023ultrafeedback} datasets to train our models for instruction following.
% \textbf{Alpaca} \cite{alpaca} -- a collection of 52,000 self-instruct question-answer pairs,
% \textbf{OpenOrca} \cite{OpenOrca} -- a large dataset of augmented question-answer pairs from the FLAN Collection \cite{longpre2023flan}, and
% \textbf{UltraFeedback} \cite{cui2023ultrafeedback} -- a large scale preference dataset generated from a variety of LLMs.
We make use of OpenOrca and a cleaned version of \href{https://huggingface.co/datasets/yahma/alpaca-cleaned}{Alpaca} for SFT, and binarized versions of \href{https://huggingface.co/datasets/Intel/orca_dpo_pairs}{OpenOrca} and \href{https://huggingface.co/datasets/argilla/ultrafeedback-binarized-preferences-cleaned}{UltraFeedback} for alignment.

\textbf{Auxiliary Datasets\ }
% We also make use of auxiliary train split from \textbf{MMLU} \cite{hendrycks2021measuring} to provide additional multiple-choice training data, but opt not to evaluate our models on this dataset due to requiring highly domain-specific knowledge. Additionally, we include \textbf{SQuAD V2} \cite{rajpurkar-etal-2018-know,rajpurkar-etal-2016-squad}, \textbf{Tiny Stories} \cite{eldan2023tinystories}, \textbf{CNN-Dailymail} \cite{nallapati2016abstractive} and \textbf{CoQA} \cite{reddy2019coqa} to provide signals for a wider range of tasks during SFT.
To provide additional training data for SFT we include the \textbf{MMLU} \cite{hendrycks2021measuring}, \textbf{SQuAD V2} \cite{rajpurkar-etal-2018-know,rajpurkar-etal-2016-squad}, \textbf{Tiny Stories} \cite{eldan2023tinystories}, \textbf{CNN-Dailymail} \cite{nallapati2016abstractive} and \textbf{CoQA} \cite{reddy2019coqa} training splits. For alignment we only include MMLU and SQuAD V2.

\subsection{Prompts and Sampling}
\textbf{Prompts\ } We make use of the ChatML prompt templating scheme \cite{chatml} with handcrafted \texttt{system}, \texttt{user} and \texttt{assistant} prompts specific to each task. During fine-tuning we mask out the loss for all tokens of the prompt and condition the model on the content of \texttt{assistant} messages including the final \texttt{<|im\_end|>} token. During evaluation we select the highest scoring answer using the average log-probabilities of the tokens in the final \texttt{assistant} message, or compute the reward scores on the final \texttt{<|im\_end|>} token when evaluating with DPH.

\textbf{SFT Sampling\ } \label{sec:sft-sampling} When sampling from the datasets for SFT we randomly shuffle each dataset and uniformly interleave samples from all tasks in the mix. To control the weighting of samples from each task we fill the context window with $n$ consecutive samples from the same task before sampling from a different task, where $n$ is chosen to be 5 in our experiments. To maximise compute utilisation and minimize unused portions of the context window we make us of Transformer-XL \cite{dai2019transformerxl} style training with a context window size of 2048 tokens and a recurrent memory size of 2048 tokens.

\textbf{DPH Sampling\ } \label{sec:dph-sampling} When sampling from datasets for DPH alignment we switch from the Transformer-XL style pipeline to typical SFT training, opting to only include single samples in the context window padded to a fixed maximum length. As some of the datasets we use for DPH are intended for SFT rather than alignment (namely GLUE, GPT4All, RACE, MMLU and SQuAD) we synthesise preference pairs where the `correct' answer is used as the preferred completion and we uniformly sample an `incorrect' answer from the available choices for the dispreferred completion. This is trivial for most datasets, however we use a special process for the SQuAD V2 dataset; for answerable questions we use ``unanswerable'' as the dispreferred completion, and for unanswerable questions we use SpaCy to randomly sample a noun span from the context to use as the dispreferred completion.

\subsection{Regularization} \label{sec:regularization}
The hidden states $h$ used to compute the reward scores are likely sub-optimal for computing rewards when initialising $\pi_\theta$ from $\pi^{\text{SFT}}$. As such, it may be desirable to fine-tune some or all parameters in the language model to learn better reward signals. This necessitates the use of regularization to prevent degradation of the models generative capabilities while learning to predict rewards.

\textbf{Prior Regularization\ }
Typical parameter regularization strategies such as weight decay make the assumption that parameters $\theta$ follow a zero-mean Normal distribution $p(\theta) \sim \mathcal{N}(0,\tfrac{1}{\beta}\text{I})$ leading to an auxiliary loss term $\tfrac{\beta}{2}||\theta||^2_2$. However, when performing transfer-learning or fine-tuning on a pre-trained model this assumption can be harmful and aid in catastrophic forgetting of the model's previously learnt abilities.

An alternative regularization scheme is Prior Regularization \cite{CHELBA2006382, daumé2009frustratingly, grachten2019strategies} which instead makes the assumption that the fine-tuned parameters are normally distributed around the original parameters $\theta_{\text{ref}}$, that is $\theta \sim \mathcal{N}(\theta_{\text{ref}},\tfrac{1}{\beta}\text{I})$, leading to the auxiliary loss term $\tfrac{\beta}{2}||\theta-\theta_{\text{ref}}||^2_2$.

We employ Prior Regularization to limit the divergence of $\pi_\theta$ from $\pi^{\text{SFT}}$ while still facilitating the learning of improved hidden state representations for the Direct Preference Head. Pseudocode for optimizer based decoupled prior regularization is included in Appendix~\ref{sec:decoupled-pr}.

% \paragraph{KL Divergence Regularization}
% Utilising KL Divergence as means to prevent a policy from diverging too far from the initial parameters is popular form of regularization: In TRPO the KL term is used to constrain the optimized policy \cite{schulman2017trust}, while PPO uses the KL Divergence as a penalty in the objective function \cite{schulman2017proximal}. \todo{Following PPO, we can form an auxiliary loss penalty with the following formula
% \begin{equation}
%     L_{penalty}(x,y) =
%     \beta D_{\text{KL}}\left[ \pi_\theta(y|x)||\pi_\text{ref}(y|x) \right]
% \end{equation}
% where $\beta$ is the regularization penalty. As noted in the PPO paper, the $\beta$ parameter requires careful tuning, and varying this coefficient throughout training may be required.}

\textbf{cDPO Regularization\ }
Rather than directly employing a KL Divergence penalty similar to that used in \eqref{eq:rl-objective} we find that it is possible -- and even beneficial -- to use Conservative DPO as a means of (1) limiting the divergence of the policy model to a fixed delta from the reference model, and (2) `nudging' the model towards generating more preferable outputs which increases the chance of generating a better candidate completion at inference time with fewer sampling steps.

% \paragraph{Head Regularization}
% \todo{Although both flavors of the DPH objective include label smoothing which regularizes the confidence of reward scores, it is still possible for the a preference head optimized with Contrastive DPH loss to produce rewards in divergent manor. This is due to ConDPH loss operating on the \textit{margin} of preferred and dispreferred rewards which may not necesserily be centered around zero. This can be prevented in two ways, by constraining the head weights $w_{dph}$ using weight decay or by applying a penalty to the rewards as they diverge from zero.}

\subsection{Training Pipeline}
We progressively fine-tune the models in 3 stages: vocab extension, supervised fine-tuning, and DPH alignment. The details of the pre-trained model are included in Appendix~\ref{sec:pretrained-model}.

\textbf{Vocab Extension\ } Since our model was pre-trained without a chat structure it is necessary to train the embeddings for additional \texttt{<|im\_start|>} and \texttt{<|im\_end|>} tokens: we freeze all non-embedding parameters and use the same datasets as SFT. We fine-tune the embeddings for 4096 steps with a batch size of 128, a max LR of 6e-5 which warms up over 200 steps followed by cosine decay down to zero, and clip the global gradient norm to 1.

\textbf{Supervised Fine-Tuning\ } After vocab extension we move onto the SFT step which conditions the model for NLU tasks and instruction following using the sampling and loss masking method described in section~\ref{sec:sft-sampling}. We fine-tune the model for 6144 steps with a batch size of 128, a max LR of 3e-5 which warms up over 200 steps followed by cosine decay down to zero, prior-regularization applied to all non-embedding parameters with coefficient 0.5, and clip the global gradient norm to 1.

\textbf{DPH Alignment\ } Using the sampling method described in section~\ref{sec:dph-sampling} we jointly learn DPH rewards and perform cDPO alignment. The goal here is to gently push the model towards producing preferable outputs without compromising the model's reasoning abilities, and the priority is to attain the highest validation metrics from the DPH rewards. This requires balancing the two objectives, and as such we introduce weighting parameters $\alpha_1, \alpha_2$ to our final joint objective in \eqref{eq:alignment_objective} where $\mathcal{L}_\text{DPH}$ is either $\mathcal{L}_\text{sepDPH}$ or $\mathcal{L}_\text{conDPH}$. We find $\alpha_1,\alpha_2=1$ to be a good blance between DPO and DPH in our experiments.
\begin{equation} \label{eq:alignment_objective}
    \mathcal{L}_\text{joint}(x,y_w,y_l,r_w,r_l) =
    \alpha_1 \mathcal{L}_\text{cDPO}(x,y_w,y_l) +
    \alpha_2 \mathcal{L}_\text{DPH}(r_w,r_l)
\end{equation}
We align the model for 23040 steps with a batch size of 64 pairs, a max LR of 3e-6 which warms up over 200 steps followed by cosine decay down to 3e-7, prior-regularization applied to all parameters with coefficient 0.5, and clip the global gradient norm to 1. Following the optimal DPO parameters for OpenHermes-7b-2.5 \cite{pref-tuning} we use $\beta=0.6$ and chose cDPO $\epsilon=0.25$ and DPH $\epsilon=0.1$ for regularisation. Additionally, we apply dropout with $p=0.1$ to the outputs of the pooler.

\subsection{Compute Resources}
All fine-tuning was performed using an NVIDIA A100 SXM4 80GB GPU on a compute cluster, with jobs allocated 24 cores and 160GB of memory. Each checkpoint is saved in FP16 format which consumes about 1.1GB of storage, and the datasets require minimal storage space.

For vocab extension we train for 4096 steps with an average of 7.99 seconds of compute per step which translates to about 9 hours. For supervised fine-tuning we train for 6144 steps with an average of 9.26 seconds of compute per step which translates to about 16 hours. For DPH alignment we train for 23040 steps with an average of 7.21 seconds of compute per step which translates to about 46 hours. The DPH ablations with our models use about 140 hours of compute, and the Qwen ablations use about 60 hours of compute. In total, we used approximately 270 hours of A100 compute to train our models and collect the results included in our paper. We used additional compute for preliminary tests and fixing bugs for silently failing experiments although this wasn't tracked.
\begin{table*}[ht]
    \centering
    \begin{tabular}{ccc|cc}
    \toprule
         & \multicolumn{2}{c}{AUC} & \multicolumn{2}{c}{$C_\text{max}$} \\
        $R$ & Random & Even & Random & Even\\
        \midrule
        $1$ & $0.893 \pm 0.022$ & $0.903 \pm 0.022$ & $91.43 \pm 1.60$ & $91.01 \pm 1.80$ \\ 
        \cmidrule{1-5}
        $2$ & $0.891 \pm 0.026$ & $0.890 \pm 0.025$ & $79.97 \pm 1.35$ & $73.46 \pm 1.86$ \\ 
        \cmidrule{1-5}
        $4$ & $0.869 \pm 0.022$ & $0.861 \pm 0.024$ & $61.40 \pm 1.77$ & $41.26 \pm 0.82$ \\ 
         \cmidrule{1-5}
        $8$ & $0.840 \pm 0.027$ & $0.834 \pm 0.029$ & $42.39 \pm 1.47$ & $21.51 \pm 0.51$ \\  
         \cmidrule{1-5}
        $16$ & $0.783 \pm 0.032$ & $0.754 \pm 0.032$ & $25.90 \pm 0.70$ & $10.90 \pm 0.23$ \\ 
         \cmidrule{1-5}
        $32$ & $0.682 \pm 0.022$ & $0.646 \pm 0.041$ & $13.54 \pm 0.42$ & $5.89 \pm 0.13$ \\ 
         \bottomrule
    \end{tabular}
    \caption{AUC and $C_\text{max}$ (mean and standard deviation) for \emph{Random} and \emph{Even} replacement position distribution.}
    \label{tab:uniform_vs_random_distr}
\end{table*}

Fig.~\ref{fig:auc_vs_r_main} shows how the AUC varies for increasing number of replacements $R$ made to the fuzzy trap sequences. We find that the AUC only drops slightly for smaller values of replacements $R$. For $R=4$, when $n_{\text{dup}}=10$ fuzzy trap sequences are injected, the mean AUC only drops from $0.90$ to $0.87$. Even at $R=32$, when we replace roughly one third of all tokens in the sequence, the mean AUC is $0.68$ and remains significantly higher than a single repetition $n_{\text{dup}}=1$ with a mean AUC of $0.59$. This demonstrates the mosaic memory in the target LLM, i.e. the overlapping fragments of multiple, slightly different sequences contribute to the memorization of the reference trap sequence.

\subsection{Replacement position distribution.} \label{section:spread_replacements}

We now aim to further reduce the chances of fuzzy trap sequences being removed by deduplication, and ensure that token replacements are evenly spread across the sequence. Above, we have uniformly at random sampled tokens to be replaced across fuzzy duplicates. Here, we instantiate the exact same setup, but we split the tokenized reference trap sequence using the MLM tokenizer: $T_{\text{MLM}}(X_{\text{ref}}) = \{t_1,\ldots,t_N\}$ in $R$ equally-sized chunks of size $\lceil\frac{N}{R}\rceil$. We then replace exactly one (selected uniformly at random) token for every chunk. Below we refer to this strategy as \emph{Even}, and to the previously used strategy as \emph{Random}.

We also compute the length of subsequences repeated exactly within clusters of fuzzy duplicates, simulating a sequence-level deduplication algorithm. We define $C_\text{max}$ as the maximum length (in tokens) of a substring shared by at least two fuzzy duplicate sequences within a cluster. We report the mean $C_\text{max}$ across 100 clusters used in our experiments.

Table~\ref{tab:uniform_vs_random_distr} shows that for smaller values of $R$ (up to $R=8$) the AUC for \textit{randomly} and \textit{evenly} distributed token replacements remains highly similar. For larger values of $R$ ($R \geq 16$), the impact of uniform spreading becomes more apparent, leading to a slightly lower AUC. At the same time, $C_\text{max}$ is significantly lower for the evenly distributed token replacements, with the impact more pronounced at higher values of $R$. This demonstrates how a reasonable number of token replacements ($R=4$ or $R=8$) would evade even the sequence-level deduplication with the most aggressive threshold (e.g. $50$ tokens), while retaining significant memorization.

\subsection{MIAs adapted to fuzzy trap sequences.} \label{section:adapted_mia}

So far we have used the unmodified \textit{Ratio} attack~\cite{carlini2021extracting} to infer whether the reference trap sequence has been seen by the target model. Specifically, we compute a single $\alpha(X_{\text{ref}}) = \alpha(X_1)$ for each trap sequence, and do not utilize our knowledge of the fuzzy counterparts $\{X_i \mid i=2 \ldots n_{\text{dup}}\}$. To evaluate whether this could further improve detectability, we now compute the membership score for each of the fuzzy trap sequences, i.e. $\{\alpha(X_i) \mid i=1 \ldots n_{\text{dup}}\}$, and aggregate the membership scores with aggregation function $\mathcal{A}(\cdot)$, i.e. $\alpha_{\mathcal{A}}(X_{\text{ref}}) = \mathcal{A}\left( \{\alpha(X_i) \mid i=1 \ldots n_{\text{dup}}\}\right)$. We then compute $\alpha_{\mathcal{A}}(X_{\text{ref}})$ for each reference trap sequence. We compute AUC on a balanced set of \emph{members} and \emph{non-members}, and so generate fuzzy trap sequences for all non-member sequences too. As aggregation function $\mathcal{A}$ we consider the mean, median, minimum and maximum. We report the MIA AUC for all aggregation functions, and for $R=\{2, 8, 32\}$ for models trained in the main experiment (Fig.~\ref{fig:auc_vs_r_main}). As a baseline, we also provide the previously used MIA based on the reference trap sequence $\alpha(X_{\text{ref}})$ alone.

 Table~\ref{tab:custom_MIA} shows that aggregating the membership score on all fuzzy trap sequences $\alpha_{\mathcal{A}}(X_{\text{ref}})$ does not provide substantial benefits compared to the baseline. We attribute this to the fact that all fuzzy trap sequences $X_i$ only differ from $X_{\text{ref}}$ by $R$ replacements, and therefore the target model loss computed on the reference trap sequence likely captures an aggregation across its fuzzy trap sequences already.

\begin{table*}[ht]
    \centering
    \begin{tabular}{cccc}
    \toprule
         & \multicolumn{3}{c}{AUC} \\
        Aggregation $\mathcal{A}$ & $R=2$ & $R=8$ & $R=32$ \\
        \midrule
        $\alpha(X_{\text{ref}})$ - no aggregation & $0.891 \pm 0.026$ & $0.840 \pm 0.027$ & $0.682 \pm 0.022$ \\ 
         \midrule
         \midrule
        Mean & $0.870 \pm 0.021$ & $0.828 \pm 0.029$ & $0.683 \pm 0.026$ \\ 
         \cmidrule{1-4}
        Median & $0.869 \pm 0.028$ & $0.824 \pm 0.030$ & $0.692 \pm 0.039$ \\  
         \cmidrule{1-4}
        Minimum & $0.881 \pm 0.021$ & $0.823 \pm 0.030$ & $0.624 \pm 0.040$  \\ 
         \cmidrule{1-4}
        Maximum & $0.879 \pm 0.030$ & $0.821 \pm 0.034$ & $0.679 \pm 0.041$  \\ 
         \bottomrule
    \end{tabular}
    \label{tab:custom_MIA}
    \caption{AUC (mean and standard deviation) for MIA methodologies adapted to fuzzy trap sequences.}
\end{table*}



\subsection{Ablation studies.} \label{section:ablations}

\textbf{Token replacement hyperparameters.} We here explore whether the semantic coherence of fuzzy trap sequences $X_i$ affect memorization. For this, we vary the value $k$ when sampling replacements from the top-$k$ tokens predicted by the MLM when a fixed number $R=8$ of replacements are made (see Sec.~\ref{sec:method_gen_fuzzy}). Recall that thus far we only considered $k=50$ and that for $k=|\mathcal{V}_{\text{MLM}}|$ -which is $50,000$ for RoBERTa~\cite{liu2019roberta}-, we effectively randomly sample a token from the entire MLM vocabulary $\mathcal{V}_{\text{MLM}}$. In addition to sampling uniformly from the top $k$ tokens, we also consider sampling directly from the full probability distribution predicted by the MLM (\textit{Sample directly}).

Fig.~\ref{fig:robustness}(a) shows that the AUC decreases as $k$ increases. We find $k$ and AUC to be strongly correlated with a Spearman coefficient of $-0.47$ and a p-value of \num{4e-11}, suggesting that semantic coherence is important for mosaic memorization. Further, Fig.~\ref{fig:robustness}(b) compares the AUC for $k=50$ and $k=|\mathcal{V}_{\text{MLM}}|$ for increasing values of $R$. For a smaller amount of replacements $R$, the AUC for $k=50$ and random token replacement remains very similar. For larger values of $R$ the mean estimations start to diverge, yet with no statistically significant difference.

\textbf{Impact of learning rate.} We here show, that the key outcome of our experiments - the \emph{relative} memorization of fuzzy duplicates compared to exact duplicates holds regardless of the \emph{absolute} level of memorization. At a fixed number of training steps we can control the baseline memorization with varying learning rate. So far in previous experiments, we have considered a fixed learning rate of $\num{3e-6}$. Fig.~\ref{fig:robustness} (b) shows how for increasing learning rate the AUC for fuzzy trap sequences remains consistently slightly lower than the upper bound ($n_{dup} = 10$), while remaining significantly higher than the lower bound ($n_{dup} = 1$). This suggests that fuzzy trap sequences are also memorized in lower and higher memorization regimes. 

\begin{figure*}[ht]
\centering
\subfigure{
\includegraphics[width=0.3\linewidth]{figures/vary_k.pdf}
}
\subfigure{
\includegraphics[width=0.3\linewidth]{figures/AUC_vs_R_both_strategies.pdf}
}
\subfigure{
\includegraphics[width=0.3\linewidth]{figures/vary_lr.pdf}
}

    \caption{\textbf{Ablation.} MIA AUC (mean and standard deviation) for fuzzy trap sequences for (a) varying $k$ for $R=8$, (b) $k=50$ and $k=50,000$ across $R$ and (c) varying learning rate used in fine-tuning.} 
\label{fig:robustness}
\end{figure*} 

\section{Discussion and Conclusion}
\noindent \textbf{Limitations.} 
Our data curation pipeline and trained model have limitations. 
The quality of the long-range 3D motion tracks depends on the accuracy of optical flow and 2D point tracking and may degrade for distant background regions or objects occluded for long periods.
Additionally, \method is a non-generative model that only operates on two-frame inputs. 
Extending our model to video input by adopting an extra global optimization~\cite{zhang2024monst3r} or integrating generative priors for modeling ambiguous motion content is a promising future direction.

\bfpar{Conclusion.}
We presented a pipeline for mining high-quality 4D data from Internet stereoscopic videos. Our framework automatically annotates each real-world video sequence with camera parameters, 3D point clouds, and long-range 3D motion trajectories by consolidating different noisy structure and motion estimates derived from videos.  Furthermore, we show that training a variant of \duster on our real-world 4D data enables more accurate learning of 3D structure and motion in dynamic scenes, outperforming other baselines.



\begin{ack}
% \todo{Do we need acknowledgements?}
\end{ack}

{
    \small
    \bibliographystyle{ieeenat_fullname}
    \bibliography{main}
}

%%%%%%%%%%%%%%%%%%%%%%%%%%%%%%%%%%%%%%%%%%%%%%%%%%%%%%%%%%%%

\section{Appendix} \label{appendix}


\subsection{NewYorker Data for evaluation}

\begin{figure}[!ht]
\small
\centering
\includegraphics[width=0.4\textwidth]{figures/length.png}
\caption{\label{lengthdist} Distribution of word count of stories in our test set}
\end{figure}

Table \ref{teststories} shows the data used for conducting our evaluation. The 12 stories shown are taken from The New Yorker and summarized into single-sentence plots. These stories come from highly established literary experts acting as an upper bound for what it means to be creative. These stories span complex themes.

\begin{table*}[!ht]
\centering
\small
\def\arraystretch{1.35}
\begin{tabular}{|l|}
\hline
\begin{tabular}[c]{@{}l@{}}Write a New Yorker-style story given the plot below. Make sure it is atleast \textbf{\color{blue}\{\{word\_count\}\}} words. Directly start with the\\ story, do not say things like `Here's the story {[}...{]}:\end{tabular}                                                                                                                                                                                            \\ \hline\hline
\begin{tabular}[c]{@{}l@{}}You wrote the story I gave you below. I requested a story with \textbf{\color{blue}\{\{word\_count\}\}} words, but the story only has\\ \textbf{\color{blue}\{\{current\_word\_count\}\}} words. Can you rewrite the story to make it longer, and closer to the \textbf{\color{blue}\{\{word\_count\}\}} word target\\ I gave you. Directly start with the story, do not say things like `Here's the story {[}...{]}:`\\ \\ Current story: \{\{story\}\}\end{tabular} \\ \hline
\end{tabular}
\vspace{2ex}
\caption{\label{promptstory}Prompt to write the initial story (Row1) vs Prompt to rewrite the initial story to be longer. word\_count represents the number of words in the human written story on a given plot (P) while current\_word\_count represents the number of words in the LLM generated story on the same plot (P)}
\end{table*}

\begin{table*}[!ht]
\def\arraystretch{1.15}
\small
\begin{tabular}{|l|l|}
\hline
Story                                    & Plot                                                                                                                                                                                                                                                                                                                                                                                                                                                                                                                                   \\ \hline
\href{https://www.newyorker.com/books/flash-fiction/a-triangle}{A Triangle}                               & \begin{tabular}[c]{@{}l@{}}An observer becomes entranced by a seemingly ordinary couple on the street, follows them home, and then \\watches them from outside in the rising floodwaters, drawing an eerie connection between the woman and\\ a discarded, burned chair they'd noticed earlier.\end{tabular}                                                                                                                                                                    \\ \hline\hline
\href{https://www.newyorker.com/books/flash-fiction/barbara-detroit-1966}{\begin{tabular}[c]{@{}l@{}}Barbara\\ Detroit,1966\end{tabular}}                    & \begin{tabular}[c]{@{}l@{}}On Feb 12, 1966, a heavily pregnant woman named Barbara experienced a shocking incident in her synagogue\\in Southfield, Detroit, where a young man shot and killed the renowned Rabbi Adler before turning the gun\\ on himself, and though Barbara tried to reach the shooter, she was swept away by the fleeing crowd.\end{tabular}                                                                              \\ \hline\hline
\href{https://www.newyorker.com/books/flash-fiction/beyond-nature}{Beyond Nature}                           & \begin{tabular}[c]{@{}l@{}}A solitary man walking in a remote mountainous region comes across a car crash, and stays by the side\\ of the lifeless female victim, narrating stories of his past and reflecting on the impermanence of \\events and life itself, while awaiting emergency services amidst the looming presence of wilderness.\end{tabular}                                                                                                                \\ \hline\hline
\href{https://www.newyorker.com/books/flash-fiction/certain-european-movies}{\begin{tabular}[c]{@{}l@{}}Certain European\\ Movies\end{tabular}}                  & \begin{tabular}[c]{@{}l@{}}Two individuals, at a residency together, navigate the complexity of their ephemeral relationship during\\ their final beach trip, framed by misadventures, subtle tensions, unspoken desires, and looming departures.\end{tabular}                                                                                                                                                                                   \\ \hline\hline
\href{https://www.newyorker.com/books/flash-fiction/keys}{Keys}                                     & \begin{tabular}[c]{@{}l@{}}Daniel, struggling with recurring dreams of his ex-wife Rachel and a mysterious unused flat, eventually \\discusses them with his current partner Isabel, sparking various reflections and conversations about their\\ past relationships, until a real-life discovery of old keys triggers a nostalgic memory and helps him find a\\ way to reconnect with his present relationship through canoeing.\end{tabular}                                     \\ \hline\hline
\href{https://www.newyorker.com/books/flash-fiction/listening-for-the-click}{\begin{tabular}[c]{@{}l@{}}Listening For\\ the Click\end{tabular}}                  & \begin{tabular}[c]{@{}l@{}}Navigating a complex social landscape, the protagonist experiences a series of complex relationships \\and emotional turmoil in a student environment, and engages in self-discovery and self-reflection as she\\ interacts with the characters Carl, Martin, Lizzy, and Johan, resulting in a journey of introspection,\\ betrayal, love, and personal growth.\end{tabular}                                                          \\ \hline\hline
\href{https://www.newyorker.com/magazine/2023/05/15/maintenance-hvidovre-fiction-olga-ravn}{\begin{tabular}[c]{@{}l@{}}Maintenance,\\ Hvidovre\end{tabular}}                   & \begin{tabular}[c]{@{}l@{}}A woman experiences a disorienting night in a maternity ward where she encounters other similarly \\disoriented new mothers, leading to an uncanny mix-up where she leaves the hospital with a baby \\that she realizes is not her own, yet accepts the situation with an inexplicable sense of happiness.\end{tabular}                                                                                                  \\ \hline\hline
\href{https://www.newyorker.com/magazine/2022/11/14/returns}{Returns}                                  & \begin{tabular}[c]{@{}l@{}}The narrator visits their elderly mother in her small town, spending a day with her that is filled with \\nostalgia, conversation, and old habits, only to return a month later after her hospitalization due to\\ a sunstroke, finding remnants of their last visit.\end{tabular}                                                                                                                                                                      \\ \hline\hline
\href{https://www.newyorker.com/books/flash-fiction/the-facade-renovation-thats-going-well}{\begin{tabular}[c]{@{}l@{}}The Facade \\Renovation\\ That’s Going Well\end{tabular}} & \begin{tabular}[c]{@{}l@{}}An academic faculty housed in a building with a critical waterproofing layer missing experiences a series\\ of disruptive and problematic construction repairs, causing tension, inconvenience, and health concerns\\ among the tenants, ultimately leading to resignation and endurance in hopes of better future circumstances.\end{tabular}                                                        \\ \hline\hline
\href{https://www.newyorker.com/books/flash-fiction/the-kingdom-that-failed}{\begin{tabular}[c]{@{}l@{}}The Kingdom\\ That Failed\end{tabular}}                  & \begin{tabular}[c]{@{}l@{}}The narrator recounts their college friendship with the seemingly flawless Q, and after a decade apart, \\they accidentally cross paths at a pool, where the narrator anonymously observes Q's failed attempt to \\let down a woman about a work-related issue, demonstrating that Q, too, has his share of difficulties.\end{tabular}                                                                                                \\ \hline\hline
\href{https://www.newyorker.com/magazine/2022/06/13/trash }{Trash}                                    & \begin{tabular}[c]{@{}l@{}}A woman unexpectedly marries the son of a successful, ambitious woman named Miss Emily, finding both \\acceptance and critique from her mother-in-law as she navigates this new relationship and confronts the \\stark contrasts between her former life as a supermarket cashier and her new life as part of a well-off family.\end{tabular}                                                                                                            \\ \hline\hline
\href{https://www.newyorker.com/culture/personal-history/the-last-dance-with-my-dad}{\begin{tabular}[c]{@{}l@{}}The Last Dance\\ with my Dad \end{tabular}}               & \begin{tabular}[c]{@{}l@{}}A young teenager recounts her experiences of fitting into her father's gay lifestyle, highlighted by a\\ seven-day cruise with hundreds of gay men, where she experienced acceptance and connection, had her\\ first genuine interaction with a  boy, and shared a last dance with her terminally ill father.\end{tabular}                                                                                                       \\ \hline
\end{tabular}
\vspace{2ex}
\caption{\label{teststories} Expert-written short stories from the New Yorker along with their human-verified GPT4 generated summary as plots that are included as part of our test data for Creativity Evaluation}
\end{table*}


\subsection{Expert Perception on the TTCW tests}

\begin{figure*}[!ht]
    \centering
     \includegraphics[width=\textwidth]{figures/rel.pdf}
    \caption{\label{relev} Relative Evaluation by Creative Writing Experts within a given group of four stories}
\end{figure*}

\begin{table*}[!ht]
\small
\centering
\begin{tabular}{|l|l|}
\hline
E5 & \begin{tabular}[c]{@{}l@{}}It was a pretty effective rubric! I'm used to being more subjective in my work -- did you like a story? Did it connect with \\you? Did it make sense? Why or why not? It was often challenging to break it down into more regimented segments \\like the rubric asked for -- but I do think that it allowed me to express the subjective feelings in a pretty thorough and\\ structured way!\end{tabular}                                                                                                                                                                 \\ \hline
E3 & \begin{tabular}[c]{@{}l@{}}As for the rubric, I thought it was quite thorough. There were some categories where I would say the story didn’t ``pass,"\\ but which were excellent. This happened often with the categories about multiple points of view, and innovative\\ structure and form. Overall, I think the rubric was helpful in helping me think about the different aspects of storytelling.\end{tabular}                                                                                                                                                                                 \\ \hline
E4 & \begin{tabular}[c]{@{}l@{}}I thought the rubric felt pretty thorough; the only part I felt could be added was that suggestion about consistency in\\ voice \& diction!\end{tabular}                                                                                                                                                         \\ \hline
E2 & \begin{tabular}[c]{@{}l@{}}The rubric seemed great to me! It’s however hard to talk about something like pacing without talking about scene and \\summary, for instance. Or the difference between originality of thought and originality in theme/content—wouldn’t the \\latter make up the former and vice/versa? But it is also comprehensive and I can see the merits of this sort of repetition in\\ teasing out a fuller picture of things\end{tabular} \\ \hline
E1 & \begin{tabular}[c]{@{}l@{}}I thought the rubric was pretty good tbh. I think there is overlap in some of the different elements, like "language \\proficiency \& literary devices" and "originality in thought." it's tricky to use words like "satisfying" and "sophisticated" \\when assessing art, but there's always going to be a great deal of subjectivity in these matters.I think that voice is a crucial \\aspect of high-quality writing that is being overlooked by the rubric, and one that greatly informs how I as a reader\\ evaluate 
and appreciate literary writing.\end{tabular} \\ \hline
\end{tabular}
\vspace{2ex}
\caption{\label{expertfeedbackrubric}Expert perception and feedback on the TTCW tests they conducted as part of our data collection.}
\end{table*}

Since the experts listed in Table \ref{creativeexperts} were not involved in designing the rubric but evaluated several stories based on the rubric we asked them their \textit{overall thought about the rubric and any potentially crucial test we missed out on that they use to discriminate between good and bad writing}.As can be seen in Table \ref{expertfeedbackrubric} in Appendix overall almost every expert agreed on the thorough and effective nature of our rubric. Many of them agreed on the fact that our rubric helped them to think about different aspects of storytelling in a more structured way. One of the difficult things about coming up with a rubric for creativity is ensuring coverage. Even though our rubric covers most aspects of creative writing, some experts such as E1 and E4 emphasized on the utility of \textbf{Consistency of Voice and Diction} as a measurable test. In E4's words \textit{``Inconsistent voice and diction are sometimes/often notable in stories that aren't very good, and when voice \& diction are used beautifully, it enhances a story considerably"}. E1 similarly exclaimed \textit{``One of the most meaningful aspects of high-quality literary writing is voice, which conveys qualities of proficiency, artistry, personality, and identity."}. We hope future work can adapt this as a meaningful test in addition to the tests covered in our rubric. Finally, some of the tests from our rubric can have potential overlaps as pointed out by E2. This is further corroborated by the similar numbers for \textit{Narrative Pacing} and \textit{Scenes vs Exposition} suggesting a strong correlation between the two.
\begin{table*}[!ht]
\small
\centering
% \def\arraystretch{1.3}
\begin{tabular}{|l|l|l|}
\hline
Test & Passing Stories & Failing Stories \\ \hline
\begin{tabular}[c]{@{}l@{}}Originality in\\ Form\end{tabular} & \begin{tabular}[c]{@{}l@{}}Inventive techniques like time jumping, varied \\ perspectives, unconventional punctuation, and\\ delayed revelation of key information\end{tabular} & \begin{tabular}[c]{@{}l@{}}Conventional and linear in its form, language, \\ and narrative, with occasional attempts at \\ innovation that do not significantly contribute to \\ its overall originality or creativity\end{tabular} \\ \hline
\begin{tabular}[c]{@{}l@{}}Originality in\\ Thought\end{tabular} & \begin{tabular}[c]{@{}l@{}}Fresh language, unique plot and characters, subtle\\ emotional resonance, and inventive metaphors. Minor \\ familiar elements, but do not undermine the overall \\ sense of imagination and distinctiveness\end{tabular} & \begin{tabular}[c]{@{}l@{}}Stories relies heavily on cliches \& tired tropes.\\ Language does not feel fresh or original with \\ narrative arc following a predictable trajectory.\\ Metaphors, descriptions, and overall premise \\ cover familiar ground that lacks novelty or nuance\end{tabular} \\ \hline
\begin{tabular}[c]{@{}l@{}}Originality in\\ Theme/Content\end{tabular} & \begin{tabular}[c]{@{}l@{}}Unconventional, dreamlike exploration of emotions\\ such as love and loss, evoking empathy and reflection\\ through its distinct main character perspective, \\ eschewing simplistic meanings for ambiguity, and \\ valuing open-ended questions over singular messages,\\ thus providing a unique reading experience compared\\ to conventional stories.\end{tabular} & \begin{tabular}[c]{@{}l@{}}Disjointed narrative, underdeveloped themes, \\ inconsistent tone, vaguely defined characters, and\\ abrupt context shifts, lack depth and fail to provide \\ substantive insight or originality to the reader.\end{tabular} \\ \hline\hline
\begin{tabular}[c]{@{}l@{}}World Building\\ and Setting\end{tabular} & \begin{tabular}[c]{@{}l@{}}Strategic use of concrete, specific sensory details from\\ a particular character’s perspective balances narrative\\ momentum, making a fictional world feel real, vivid\\ and immersive for readers. Thoughtful depiction of\\ everyday objects, and idiosyncratic elements within\\ narrative and dialogue to balance exposition with \\ vivid scene-setting, creating authenticity and realism \\ that serves the plot and characters\end{tabular} & \begin{tabular}[c]{@{}l@{}}Fictional world is not always convincingly \\established through sensory details and language. \\Stories rely too heavily on overwrought imagery\\ and figurative language without grounding \\the reader in a tangible reality.\end{tabular} \\ \hline
\begin{tabular}[c]{@{}l@{}}Character\\ Development\end{tabular} & \begin{tabular}[c]{@{}l@{}}Fully realized characters with contradictions, \\ motivations, and backstories that make them\\ feel lifelike. Flatter, less developed characters\\ that feel appropriate for the narrative goals \\ and style is not necessarily a weakness\end{tabular} & \begin{tabular}[c]{@{}l@{}}Characters not well rounded. easily resorting to \\stereotypes. Predictable arcs not making them\\memorable. Actions or motivations unclear leading \\to disconnect\end{tabular} \\ \hline
\begin{tabular}[c]{@{}l@{}}Rhetorical\\ Complexity\end{tabular} & \begin{tabular}[c]{@{}l@{}}Rich subtext that emerges through contrasts between\\ characters and settings. Omissions that let readers \\ fill in meaning, metaphors with layered significance, \\ implicit characterization, ambiguous endings, \\ intertextual references, and rich language that goes\\ beyond straightforward storytelling\end{tabular} & \begin{tabular}[c]{@{}l@{}}Straightforward storytelling, lack of complexity,\\ nuance, and ambiguity.The absence of context and \\ ironic tensions, overtelling instead of showing, \\ Use of expository dialogue, blunt symbolism, \\ predictable characters, and definitive endings, all \\of which limit interpretation and undermine\\ their potential resonance\end{tabular} \\ \hline
\end{tabular}
\vspace{2ex}
\caption{\label{expertexpl2}Common themes and issues found in expert explanations for tests focusing on TTCW-Originality and TTCW-Elaboration}
\vspace{-8ex}
\end{table*}

\subsection{Common themes in TTCW of Originality and Elaboration}


Table~\ref{expertexpl2} provides common themes and issues found in the expert explanations for each TTCW in the Originality and Elaboration dimensions.


\subsection{Example LLM-generated and expert-written explanations for a TTCW assessment} 

\begin{table*}[!ht]
\centering
\small
\def\arraystretch{1.35}
\begin{tabular}{|l|l|}
\hline
GPT4 & \begin{tabular}[c]{@{}l@{}}The story incorporates numerous instances of subtext that serve to further develop the characters' internal emotions \\ and relationships, as well as accentuate the atmospheric tension and undertones of longing. The subtext, which is\\ subtly weaved into surface-level interactions, actions, and observations, is eloquently crafted and does not feel \\ forced, enriching both the narrative depth and setting.\\ One key passage with evident subtext is when he is in the ocean, swimming and jumping off the rock, while she\\ observes from the towel. The surface-level interpretation suggests a casual, relaxed beach day. However, the \\ subtext is rife with allusions of emotional undercurrents in their interaction. His unawareness of her going \\ into the water parallels the emotional disconnect between them, insinuating the vast, metaphorical space that \\ separates their respective experiences and understanding of their relationship.\\ Another instance of subtext occurs when he takes a photo of her with her swimsuit top off, indicating his attraction\\ to her. But the meaning deepens when we understand that he will delete the photo before going home, implicitly\\ revealing his guilt or fear of being discovered, and again highlighting the transient, secret nature of their relationship.\\ The line, "As in certain European movies" suggests the fleeting, impermanent element of their relationship and the\\ eventual abruptness and intensity of feelings that often feature in said films. The 'European movies' serve as a \\ metaphor for the unexpected and sudden end of their encounter. \\ In these examples, the subtext does not feel forced but instead flows naturally from the characters and \\ circumstances, subtly conveying deeper meanings that heighten both the narrative tension and emotional depth.\\ \\ So Yes.\end{tabular} \\ \hline
E3   & \begin{tabular}[c]{@{}l@{}}There is rich subtext, as the main character seems continually conflicted about whether she wants to be where she is, \\ doing what she is doing. On the surface, she is carefree, riding to the beach with the guy she met, skipping the ceramics\\ and the museum, and whatever else. And yet, she is unhappy and unsatisfied, longing for a beer, imagining that if their\\ relationship continued they would only hate each other. This tension is maintained throughout the story.\end{tabular}                                                                                                                                                                                                                                                                                                               \\ \hline
E1   & \begin{tabular}[c]{@{}l@{}}This piece has an iceberg of subtext floating underneath it. The entire story is conveyed through the successful \\ integration of subtext and text. The interactions between the protagonist and the man (Did you see me jump of the \\ rock? No, she hadn't.Did he notice she had gone in the water too, that her hair was dripping? No, he hadn't.)convey\\ a profound disconnect that causes the reader to wonder why the protagonist continues to suffer the presence of this\\ man she clearly disdains and seems to view as an incompetent man-child.\end{tabular}
               \\ \hline
E7   & \begin{tabular}[c]{@{}l@{}}Yes!!!!! Again, the idea of the story was fairly simple (the inevitability of age, parting, change), but it was illustrated\\ in a way that felt inspiring re: questioning how these ideas relate and resonate throughout our own lives. It was really \\ beautiful and I was left feeling changed at the end of it :)\end{tabular}                                                                                                                                                                                                                                                            \\ \hline
\end{tabular}
\vspace{2ex}
\caption{\label{llmvsexpertexpl}LLM explanation vs expert explanation for Rhetorical Complexity}
\end{table*}

In Table~\ref{llmvsexpertexpl}, we show examples of explanations that experts wrote in conjunction with a binary TTCW assessment they made on a story, as well as the corresponding LLM-generated explanations.

\subsection{Can non-experts administer TTCW tests?}

Recruiting experts for data annotation purposes is challenging, and costly, and must consider the time constraint put on the experts. Prior work has shown the potential of crowd-sourcing (through platforms such as Amazon Mechanical Turk) and the ability of non-experts to accomplish complex tasks as a crowd \cite{kittur2013future}, when following an appropriate workflow that iterates and validates the work on individual non-experts. Some prior work has even shown the validity of crowd-based feedback for writing tasks \cite{bernstein2010soylent,nebeling2016wearwrite}. 

In this work, we chose to rely on experts for annotation, to maximize the validity of our experiments, and confirm whether experts with domain knowledge would reach satisfying agreement levels when evaluating stories with TTCW. Future work can leverage our open-sourced annotations to explore whether non-experts correlate with experts when performing TTCW evaluation, which could lead to more cost-effective TTCW evaluation.

\subsection{Prompts for TTCW} \label{allprompts}

All the instructions shown to creative writing experts and LLMs are given in the tables below.


\begin{table*}[!ht]
\centering
\small
\begin{tabular}{|l|l|}
\hline
\begin{tabular}[c]{@{}l@{}}Expert \\ Measure\end{tabular}               & Does the manipulation of time in terms of compression or stretching feel appropriate and balanced?                                                                                                                                    \\ \hline
\begin{tabular}[c]{@{}l@{}}Expanded\\ Expert\\ Measure (M)\end{tabular} & \begin{tabular}[c]{@{}l@{}}`Compression/stretching of time' in fiction writing, also known as pacing, refers to the manipulation of time in \\storytelling for dramatic effect, pacing, or other narrative purposes. Essentially, it's about controlling the perceived \\speed and rhythm at which a story unfolds.\\ \\

Compression of time refers to when events that take a long time (hours, days, weeks, or even years) are summarized \\or condensed into a brief narrative span. For example, a writer might compress several years of a character's life \\into a few paragraphs to quickly convey important changes or developments.\\ \\

On the other hand, stretching of time is when a brief moment or event is drawn out over pages or chapters. It's often \\used to create suspense, emphasize details, or delve deeper into a character's thoughts and feelings. For example, \\the few seconds it takes for a dropped glass to hit the floor might be stretched out with detailed descriptions of the\\ action, reactions, and thoughts of characters involved.\\ \\

Storytime refers to the time within the world of the story, while real-world time refers to the time it takes for the \\reader to read the story. A skilled writer can manipulate the relationship between these two to affect the pacing of \\the narrative, either speeding it up (compression) or slowing it down (stretching). This technique plays a crucial role \\in shaping the reader's experience and engagement with the story.\end{tabular} \\ \hline
\begin{tabular}[c]{@{}l@{}}Human\\ Instruction\end{tabular}             & \begin{tabular}[c]{@{}l@{}}\{\{M\}\}\\ \\ Based on the story that you just read, answer the following question.\\ \textit{\color{blue}Does the manipulation of time in terms of compression or stretching feel appropriate and balanced?}\\ -Yes \\ -No \\\\ Reasoning : \end{tabular}                                                                       \\ \hline
\begin{tabular}[c]{@{}l@{}}LLM\\ Instruction\end{tabular}               & \begin{tabular}[c]{@{}l@{}}\{\{M\}\}\\ \\ Given the story above, list out the scenes in the story in which time compression or time stretching is used, and \\argue for each whether it is successfully implemented.  Then overall, give your reasoning about the question below \\and give an answer to it between 'Yes' or 'No' only \\ \\ \textit{\color{blue} Q) Does the manipulation of time in terms of compression or stretching feel appropriate and balanced?}\end{tabular}                                                                                                                                                                                                                    \\ \hline
\end{tabular}
\vspace{2ex}
\caption{\label{prompting}TTCW Fluency1 (Narrative Pacing) }
\vspace{-5ex}
\end{table*}


% ==================================================





\begin{table*}[!ht]
\centering
\small
% \def\arraystretch{1.15}
\begin{tabular}{|l|l|}
\hline
\begin{tabular}[c]{@{}l@{}}Expert \\ Measure\end{tabular}               & \begin{tabular}[c]{@{}l@{}}Does the story have an appropriate balance between scene and summary/exposition or it relies on one\\ of the elements heavily compared to the other?  \end{tabular}                                                                                                                                  \\ \hline
\begin{tabular}[c]{@{}l@{}}Expanded\\ Expert\\ Measure (M)\end{tabular} & \begin{tabular}[c]{@{}l@{}}'Scene' and 'summary/exposition' are two crucial elements of narrative storytelling, and balancing them \\appropriately is an important skill in fiction writing.\\ \\ 

A 'scene' is a moment in the story that is dramatized in real-time. Scenes are usually vivid and engaging, often \\featuring character interaction, dialogue, and action. They are the building blocks of the plot, and through them, \\the story unfolds.\\ \\ 

'Summary' or 'exposition', on the other hand, involves summarizing events or providing information. Instead of \\unfolding in real time, \\summaries compress time and tell the reader what happened. Exposition provides \\necessary background information, like character history, setting details, or prior events. \\ \\ 

A good writer knows when to use scenes to make the story come alive, show character development, or increase \\tension. They also know when to use summary or exposition to move the story forward, fill in background \\information, or bridge gaps between important scenes. \\ \\ 

The right balance between scene and summary/exposition can vary depending on the story, but in general, it's \\essential for maintaining a good pace, keeping the reader engaged, and delivering necessary information. \\A story with too many scenes and not enough summary might feel overwhelming or slow, while a story with \\too much exposition and not enough scenes could feel dry and unengaging.\end{tabular} \\ \hline
\begin{tabular}[c]{@{}l@{}}Human\\ Instruction\end{tabular}             & \begin{tabular}[c]{@{}l@{}}\{\{M\}\}\\ \\ Based on the story that you just read, answer the following question.\\ \textit{\color{blue} Does the story have an appropriate balance between scene and summary/exposition or it relies on one of the elements} \\\textit{\color{blue}heavily compared to the other?} \\ -Yes \\ -No \\\\ Reasoning : \end{tabular}    
\\ \hline
\begin{tabular}[c]{@{}l@{}}LLM\\ Instruction\end{tabular}               & \begin{tabular}[c]{@{}l@{}}\{\{M\}\}\\ \\ Given the story above, answer the following question. Please first explain your reasoning step by step \\and then given an answer between 'Yes' or 'No' only \\ \\ \textit{\color{blue} Does the story have an appropriate balance between scene and summary/exposition or it relies on one of the elements} \\\textit{\color{blue}heavily compared to the other?}\end{tabular}                                                                                                                                                                                                                    \\ \hline
\end{tabular}
\vspace{2ex}
\caption{\label{prompting}TTCW Fluency2 (Scene vs Exposition) }
\vspace{-5ex}
\end{table*}


% ==================================================


\begin{table*}[!ht]
\centering
\small
% \def\arraystretch{1.15}
\begin{tabular}{|l|l|}
\hline
\begin{tabular}[c]{@{}l@{}}Expert \\ Measure\end{tabular}               & Does the story make sophisticated use of idiom or metaphor or literary allusion?                                                                                                                                     \\ \hline
\begin{tabular}[c]{@{}l@{}}Expanded\\ Expert\\ Measure (M)\end{tabular} & \begin{tabular}[c]{@{}l@{}}`Idiom' refers to phrases or expressions that have a figurative, or sometimes literal, meaning that is \\comprehensible to a particular group of people. These can be cultural, regional, or specific to a certain group or \\profession.Sophisticated use of idiom suggests that the writer is skillfully using these expressions to lend \\authenticity to character voices or to convey specific meanings in a concise way.\\\\

`Metaphor' is a figure of speech that describes an object or action in a way that isn't literally true, but helps explain\\ an idea or make a comparison. Sophisticated use of metaphor suggests the
writer could create impactful, original \\comparisons that reveal deeper insights about themes,
characters, or situations in the story.\\\\

`Literary allusion' refers to a brief and indirect reference to a person, place, thing or idea of
historical, cultural,\\ literary, or political significance. It does not describe in detail the person or thing to which it refers. A sophisticated\\ use of literary allusion implies the writer can effectively incorporate these references to enhance the depth\\ and resonance of the story. They can provide contextual richness, evoke a specific tone, or draw parallels between\\ the narrative and the work alluded to.\\\\

Overall, when a writer uses these techniques well, they add depth, interest, and nuanced \\meaning
to their work. It allows for a richer reading experience, where the literal events are \\imbued with deeper symbolic or thematic significance.\end{tabular} \\ \hline
\begin{tabular}[c]{@{}l@{}}Human\\ Instruction\end{tabular}             & \begin{tabular}[c]{@{}l@{}}\{\{M\}\}\\ \\ Based on the story that you just read, answer the following question.\\ \textit{\color{blue}Does the story make sophisticated use of idiom or metaphor or literary allusion?}\\ -Yes \\ -No \\\\ Reasoning: \end{tabular}                                                                       \\ \hline
\begin{tabular}[c]{@{}l@{}}LLM\\ Instruction\end{tabular}               & \begin{tabular}[c]{@{}l@{}}\{\{M\}\}\\ \\ Given the story above, please list out all the metaphors, idioms and literary allusions, and for each decide \\whether it is successful vs it feels forced or too easy.  Then overall, give your reasoning about the question \\below and give an answer to it between 'Yes' or 'No' only\\ \\ \textit{\color{blue} Q) Does the story make sophisticated use of idiom or metaphor or literary allusion?}\end{tabular}                                                                                                                                                                                                                    \\ \hline
\end{tabular}
\vspace{2ex}
\caption{\label{prompting}TTCW Fluency3 (Language Proficiency \& Literary Devices) }
\vspace{-5ex}
\end{table*}


% ==================================================



\begin{table*}[!ht]
\centering
\small
% \def\arraystretch{1.15}
\begin{tabular}{|l|l|}
\hline
\begin{tabular}[c]{@{}l@{}}Expert \\ Measure\end{tabular}               & Does the end of the story feel natural and earned, as opposed to arbitrary or abrupt?                                                                                                                                    \\ \hline
\begin{tabular}[c]{@{}l@{}}Expanded\\ Expert\\ Measure (M)\end{tabular} & \begin{tabular}[c]{@{}l@{}}If the writer ends the piece simply because they are 'tired of writing', the conclusion might feel abrupt, disjointed, \\or unfulfilling to the reader. It suggests a rushed ending, where plot threads might be left unresolved and character \\arcs incomplete.\\ \\ 

Conversely, if the writer concludes because they've reached `the moment the entire piece has been leading readers \\towards', it implies a well-considered and purposeful ending. The events, character development, and themes \\throughout the story have built towards this climactic moment, providing a satisfying resolution to the reader.\\ \\ 

A strong ending offers a sense of closure, ties up the central conflicts or questions of the story, and generally \\leaves the reader feeling that the narrative journey was worthwhile and complete.\end{tabular} \\ \hline
\begin{tabular}[c]{@{}l@{}}Human\\ Instruction\end{tabular}             & \begin{tabular}[c]{@{}l@{}}\{\{M\}\}\\ \\ Based on the story that you just read, answer the following question.\\ \textit{\color{blue}Does the end of the story feel natural and earned, as opposed to arbitrary or abrupt?}\\ -Yes \\ -No \\\\ Reasoning : \end{tabular}                                                                       \\ \hline
\begin{tabular}[c]{@{}l@{}}LLM\\ Instruction\end{tabular}               & \begin{tabular}[c]{@{}l@{}}\{\{M\}\}\\ \\ Given the story above, answer the following question. Please first explain your reasoning step by step \\ and then given an answer between 'Yes' or 'No' only\\ \\ \textit{\color{blue} Q) Does the end of the story feel natural and earned, as opposed to arbitrary or abrupt?}\end{tabular}                                                                                                                                                                                                                    \\ \hline
\end{tabular}
\vspace{2ex}
\caption{\label{prompting}TTCW Fluency4 (Narrative Ending) }
\vspace{-5ex}
\end{table*}



% ==================================================



\begin{table*}[!ht]
\centering
\small
% \def\arraystretch{1.15}
\begin{tabular}{|l|l|}
\hline
\begin{tabular}[c]{@{}l@{}}Expert \\ Measure\end{tabular}               & Do the different elements of the story work together to form a unified, engaging, and satisfying whole?                                                                                                                                     \\ \hline
\begin{tabular}[c]{@{}l@{}}Expanded\\ Expert\\ Measure (M)\end{tabular} & \begin{tabular}[c]{@{}l@{}}A well-crafted story usually follows a logical path, where the events in the beginning set up the middle, which then\\ logically leads to the end. Every scene, character action, and piece of dialogue should serve the story and propel it \\forward. Well-written stories have an underlying the unity that binds the elements together. The themes, plotlines, \\character arcs, and other elements of the story interweave to create a harmonious whole. A story with 'disorder'\\ might feel disjointed, with characters, scenes, etc that don't connect or contribute to the overall narrative.\end{tabular} \\ \hline
\begin{tabular}[c]{@{}l@{}}Human\\ Instruction\end{tabular}             & \begin{tabular}[c]{@{}l@{}}\{\{M\}\}\\ \\ Based on the story that you just read, answer the following question.\\ \textit{\color{blue}Do the different elements of the story work together to form a unified, engaging, and satisfying whole?}\\ -Yes \\ -No \\\\ Reasoning : \end{tabular}                                                                       \\ \hline
\begin{tabular}[c]{@{}l@{}}LLM\\ Instruction\end{tabular}               & \begin{tabular}[c]{@{}l@{}}\{\{M\}\}\\ \\ Given the story above, answer the following question. Please first explain your reasoning step by step and then \\give an answer between 'Yes' or 'No' only\\ \\ \textit{\color{blue} Q) Do the different elements of the story work together to form a unified, engaging, and satisfying whole?}\end{tabular}                                                                                                                                                                                                                                 \\ \hline
\end{tabular}
\vspace{2ex}
\caption{\label{prompting}TTCW Fluency5 (Understandability \& Coherence) }
\vspace{-5ex}
\end{table*}


% ==================================================



\begin{table*}[!ht]
\centering
\small
% \def\arraystretch{1.15}
\begin{tabular}{|l|l|}
\hline
\begin{tabular}[c]{@{}l@{}}Expert \\ Measure\end{tabular}               & \begin{tabular}[c]{@{}l@{}}Does the story provide diverse perspectives, and if there are unlikeable characters, are their perspectives \\presented convincingly and accurately? \end{tabular}                                                                                                                                     \\ \hline
\begin{tabular}[c]{@{}l@{}}Expanded\\ Expert\\ Measure (M)\end{tabular} & \begin{tabular}[c]{@{}l@{}}A good writer can convincingly and accurately depict a wide range of character viewpoints, including those of\\ characters who may be morally ambiguous, difficult, or otherwise unappealing.\\ \\ 

This can involve diving into the mindset of characters who may act or think in ways that the reader, or even \\the writer, finds objectionable or repugnant. It involves understanding their motivations, their beliefs, and the \\reasons behind their actions, and then conveying these elements in a way that is believable and consistent.\\ \\ 

The purpose of doing so is not to justify or endorse these perspectives, but rather to create complex, three-\\dimensional characters who contribute to the richness and depth of the story. This can also serve to \\challenge the reader, provoke thought, and provide insights into different aspects of the human experience.\end{tabular} \\ \hline
\begin{tabular}[c]{@{}l@{}}Human\\ Instruction\end{tabular}             & \begin{tabular}[c]{@{}l@{}}\{\{M\}\}\\ \\ Based on the story that you just read, answer the following question.\\ \textit{\color{blue}Does the story provide diverse perspectives, and if there are unlikeable characters, are their perspectives presented} \\ \textit{\color{blue}convincingly and accurately?}\\ -Yes \\ -No \\\\ Reasoning : \end{tabular}                                                                       \\ \hline
\begin{tabular}[c]{@{}l@{}}LLM\\ Instruction\end{tabular}               & \begin{tabular}[c]{@{}l@{}}\{\{M\}\}\\ \\ Given the story above, answer the following question. Please first explain your reasoning step by step and then \\give an answer between 'Yes' or 'No' only\\ \\ \textit{\color{blue} Q) Does the story provide diverse perspectives, and if there are unlikeable characters, are their perspectives presented}\\\textit{\color{blue} convincingly and accurately?}\end{tabular}                                                                                                                                                                                                                                 \\ \hline
\end{tabular}
\vspace{2ex}
\caption{\label{prompting}TTCW Flexibility1 (Perspective \& Voice Flexibility) }
\vspace{-5ex}
\end{table*}


% ==================================================




\begin{table*}[!ht]
\centering
\small
% \def\arraystretch{1.15}
\begin{tabular}{|l|l|}
\hline
\begin{tabular}[c]{@{}l@{}}Expert \\ Measure\end{tabular}               & \begin{tabular}[c]{@{}l@{}}Does the story achieve a good balance between interiority and exteriority, in a way that feels \\emotionally flexible? \end{tabular}                                                                                                                                     \\ \hline
\begin{tabular}[c]{@{}l@{}}Expanded\\ Expert\\ Measure (M)\end{tabular} & \begin{tabular}[c]{@{}l@{}}`Emotional flexibility' is asking whether the piece of writing effectively balances action and introspection, and \\if it portrays a broad and realistic spectrum of emotions.\\ \\

`Exteriority' refers to the observable actions, behaviors, or dialogue of a character, and the physical or visible \\aspects of the setting, plot, and conflicts.\\ \\

`Interiority', on the other hand, pertains to the inner life of a character — their thoughts, feelings, memories, \\and subjective experiences.\\ \\

A balance between these two aspects is crucial in creating well-rounded characters and compelling narratives. \\If a piece is too heavy on exteriority, it may feel shallow or lack emotional depth. If it leans too much on\\ interiority, it could become overly introspective and potentially lose the momentum of the plot.
\end{tabular} \\ \hline
\begin{tabular}[c]{@{}l@{}}Human\\ Instruction\end{tabular}             & \begin{tabular}[c]{@{}l@{}}\{\{M\}\}\\ \\ Based on the story that you just read, answer the following question.\\ \textit{\color{blue}Does the story achieve a good balance between interiority and exteriority, in a way that feels emotionally flexible?}\\ -Yes \\ -No \\\\ Reasoning : \end{tabular}                                                                       \\ \hline
\begin{tabular}[c]{@{}l@{}}LLM\\ Instruction\end{tabular}               & \begin{tabular}[c]{@{}l@{}}\{\{M\}\}\\ \\ Given the story above, answer the following question. Please first explain your reasoning step by step and \\then give an answer between 'Yes' or 'No' only\\ \\ \textit{\color{blue}Q) Does the story achieve a good balance between interiority and exteriority, in a way that feels} \\\textit{\color{blue}emotionally flexible?}\end{tabular}                                                                                                                                                                                                                                 \\ \hline
\end{tabular}
\vspace{2ex}
\caption{\label{prompting}TTCW Flexibility2 (Emotional Flexibility) }
\vspace{-5ex}
\end{table*}


% ==================================================




\begin{table*}[!ht]
\centering
\small
% \def\arraystretch{1.15}
\begin{tabular}{|l|l|}
\hline
\begin{tabular}[c]{@{}l@{}}Expert \\ Measure\end{tabular}               & \begin{tabular}[c]{@{}l@{}}Does the story contain turns that are both surprising and appropriate? \end{tabular}                                                                                                                                     \\ \hline
\begin{tabular}[c]{@{}l@{}}Expanded\\ Expert\\ Measure (M)\end{tabular} & \begin{tabular}[c]{@{}l@{}}`Surprising': This refers to the element of unpredictability in a narrative. A good story often has plot twists, \\character developments, or thematic revelations that surprise the reader, subverting their expectations in a \\thrilling way.It's about keeping readers engaged and curious, never fully knowing what's going to happen next.\\ \\ 

`Appropriate': Despite the surprises and twists, the turns in the story must also make sense within the established \\context of the story's universe, its characters, and its themes. This means that even though an event might be \\surprising, it should feel appropriate or fitting in hindsight. It shouldn't feel like the writer has broken the rules \\they've set up, or made a character behave inconsistently without reason, simply for the sake of shock value.\\ \\ 

So when someone wonders if a writer can make turns that are 'both surprising and appropriate', they're asking \\if the writer can strike this balance between unexpectedness and coherence, keeping the reader on their toes \\while maintaining a believable, satisfying narrative. \end{tabular} \\ \hline
\begin{tabular}[c]{@{}l@{}}Human\\ Instruction\end{tabular}             & \begin{tabular}[c]{@{}l@{}}\{\{M\}\}\\ \\ Based on the story that you just read, answer the following question.\\ \textit{\color{blue}Does the story contain turns that are both surprising and appropriate?}\\ -Yes \\ -No \\\\ Reasoning: \end{tabular}                                                                       \\ \hline
\begin{tabular}[c]{@{}l@{}}LLM\\ Instruction\end{tabular}               & \begin{tabular}[c]{@{}l@{}}\{\{M\}\}\\ \\ Given the story above, list each element in the story that is intended to be surprising. For each, decide whether the\\ surprising element remains appropriate with respect to the entire story. Then overall, give your reasoning \\about the question below and give an answer to it between 'Yes' or 'No' only\\ \\ \textit{\color{blue} Q) Does the story contain turns that are both surprising and appropriate?}\end{tabular}                                                                                                                                                                                                                                 \\ \hline
\end{tabular}
\vspace{2ex}
\caption{\label{prompting}TTCW Flexibility3 (Structural Flexibility) }
\vspace{-5ex}
\end{table*}


% ==================================================






\begin{table*}[!ht]
\centering
\small
% \def\arraystretch{1.15}
\begin{tabular}{|l|l|}
\hline
\begin{tabular}[c]{@{}l@{}}Expert \\ Measure\end{tabular}               & \begin{tabular}[c]{@{}l@{}}Will an average reader of this story obtain a unique and original idea from reading it? \end{tabular}                                                                                                                                     \\ \hline
\begin{tabular}[c]{@{}l@{}}Expanded\\ Expert\\ Measure (M)\end{tabular} & \begin{tabular}[c]{@{}l@{}}If a story is good, the reader gains new insights, perspectives, or knowledge from it. This doesn't necessarily\\ mean factual information but could relate to a deeper understanding of human nature, cultural insights,\\ unique viewpoints, or even the exploration of new ideas and themes. Essentially, it's about what\\ the reader takes away from the story beyond just the plot.\\ \\ 

A good story has lasting impacts on its readers and the society. It is meant to entertain, inform, provoke \\thought, challenge beliefs, provide comfort, or raise awareness on specific issues.
 \end{tabular} \\ \hline
\begin{tabular}[c]{@{}l@{}}Human\\ Instruction\end{tabular}             & \begin{tabular}[c]{@{}l@{}}\{\{M\}\}\\ \\ Based on the story that you just read, answer the following question.\\ \textit{\color{blue}Will an average reader of this story obtain a unique and original idea from reading it?}\\ -Yes \\ -No \\\\ Reasoning : \end{tabular}                                                                       \\ \hline
\begin{tabular}[c]{@{}l@{}}LLM\\ Instruction\end{tabular}               & \begin{tabular}[c]{@{}l@{}}\{\{M\}\}\\ \\ Given the story above, list out elements that are unique takeaways of this story for the reader. Then overall, \\give your reasoning about the question below and give an answer to it between 'Yes' or 'No' only\\ \\ \textit{\color{blue} Q) Will an average reader of this story obtain a unique and original idea from reading it?}\end{tabular}                                                                                                                                                                                                                                 \\ \hline
\end{tabular}
\vspace{2ex}
\caption{\label{prompting}TTCW Originality1 (Originality in Theme and Content) }
\vspace{-3ex}
\end{table*}


% ==================================================








\begin{table*}[!ht]
\centering
\small
% \def\arraystretch{1.15}
\begin{tabular}{|l|l|}
\hline
\begin{tabular}[c]{@{}l@{}}Expert \\ Measure\end{tabular}               & \begin{tabular}[c]{@{}l@{}}Is the story an original piece of writing without any cliches?\end{tabular}                                                                                                                                     \\ \hline
\begin{tabular}[c]{@{}l@{}}Expanded\\ Expert\\ Measure (M)\end{tabular} & \begin{tabular}[c]{@{}l@{}}A cliche is an idea, expression, character, or plot that has been overused to the point of losing its original \\meaning or impact. They often become predictable and uninteresting for the reader. Originality suggests\\ that the piece isn't cliche.

 \end{tabular} \\ \hline
\begin{tabular}[c]{@{}l@{}}Human\\ Instruction\end{tabular}             & \begin{tabular}[c]{@{}l@{}}\{\{M\}\}\\ \\ Based on the story that you just read, answer the following question.\\ \textit{\color{blue}Is the story an original piece of writing without any cliches?}\\ -Yes \\ -No \\\\ Reasoning: \end{tabular}                                                                       \\ \hline
\begin{tabular}[c]{@{}l@{}}LLM\\ Instruction\end{tabular}               & \begin{tabular}[c]{@{}l@{}}\{\{M\}\}\\ \\ Given the story above, are there any cliches in the story? If so, list out all the elements in this story that \\are cliche. Then overall, give your reasoning if the piece is negatively impacted by the cliches and give \\an answer to the question below between 'Yes' or 'No' only\\ \\ \textit{\color{blue} Q) Is the story an original piece of writing without any cliches?}\end{tabular}                                                                                                                                                                                                                                 \\ \hline
\end{tabular}
\vspace{2ex}
\caption{\label{prompting}TTCW Originality2 (Originality in Thought) }
\vspace{-5ex}
\end{table*}


% ==================================================




\begin{table*}[!ht]
\centering
\small
% \def\arraystretch{1.15}
\begin{tabular}{|l|l|}
\hline
\begin{tabular}[c]{@{}l@{}}Expert \\ Measure\end{tabular}               & \begin{tabular}[c]{@{}l@{}}Does the story show originality in its form?\end{tabular}                                                                                                                                     \\ \hline
\begin{tabular}[c]{@{}l@{}}Expanded\\ Expert\\ Measure (M)\end{tabular} & \begin{tabular}[c]{@{}l@{}}When someone says that a piece of fiction 'shows an innovative use of form/structure', they're referring to\\ how the writer has chosen to shape and organize the story in an unusual, original, or inventive way. This could \\involve a variety of different elements, including:\\ \\ 

Narrative Structure: This could include unconventional timelines (e.g. a non-linear story, a story told in reverse)\\, multiple perspectives or narrators, or unusual narrative voices (e.g. a story told from the perspective of an \\inanimate object).\\ \\ 

Format: This could be something as simple as using unconventional punctuation or capitalization, or as complex \\as telling a story through a series of letters, diary entries, newspaper clippings, or other documents. In recent years,\\ some authors have even experimented with using social media posts or text messages as a form of narrative structure.\\ \\ 

Genre Hybridity: Combining elements from different genres or sub-genres in unexpected ways can also be seen\\ as an innovative use of form such as Horror-Mystery or Comic Fantasy.\\ \\ 

Plot Structure: Deviating from traditional plot structures such as three-act structure, or following them in unexpected\\ ways.For example, telling a story without a clear resolution, incorporating multiple climaxes or using revelation as a \\device where a surprising, and often shocking, development occurs that was previously kept hidden from the \\characters and/or the audience. It's typically designed to provide new context for interpreting what has previously \\occurred in the story. \\ \\ 

Language and Style: Innovative use of form can also come in the form of unique use of language, style, or \\even typography, such as concrete poetry or writing that visually represents its subject matter on the page.\\ \\ 

The goal of this innovation is often to provide a fresh reader experience, challenge conventional reading\\ expectations, or to create a deeper or more complex exploration of the story's themes.

 \end{tabular} \\ \hline
\begin{tabular}[c]{@{}l@{}}Human\\ Instruction\end{tabular}             & \begin{tabular}[c]{@{}l@{}}\{\{M\}\}\\ \\ Based on the story that you just read, answer the following question.\\ \textit{\color{blue}Does the story show originality in its form?}\\ -Yes \\ -No \\\\ Reasoning: \end{tabular}                                                                       \\ \hline
\begin{tabular}[c]{@{}l@{}}LLM\\ Instruction\end{tabular}               & \begin{tabular}[c]{@{}l@{}}\{\{M\}\}\\ \\ Given the story and the devices mentioned above, list each device used with a short explanation of whether it is \\successful or not. Then overall, give your reasoning about the question below and give an answer to it\\ between 'Yes' or 'No' only\\ \\ \textit{\color{blue} Q) Does the story show originality in its form?}\end{tabular}                                                                                                                                                                                                                                 \\ \hline
\end{tabular}
\vspace{2ex}
\caption{\label{prompting}TTCW Originality3 (Originality in Form) }
\vspace{-5ex}
\end{table*}


% ==================================================




\begin{table*}[!ht]
\centering
\small
% \def\arraystretch{1.15}
\begin{tabular}{|l|l|}
\hline
\begin{tabular}[c]{@{}l@{}}Expert \\ Measure\end{tabular}               & \begin{tabular}[c]{@{}l@{}}Does each character in the story feel developed at the appropriate complexity level, ensuring that no character \\feels like they are present simply to satisfy a plot requirement?\end{tabular}                                                                                                                                     \\ \hline
\begin{tabular}[c]{@{}l@{}}Expanded\\ Expert\\ Measure (M)\end{tabular} & \begin{tabular}[c]{@{}l@{}} A `flat character' is typically a minor character who is not thoroughly developed or who does not undergo \\significant change or growth throughout the story. They often embody or represent a single trait or idea, \\and they're used to advance the plot or highlight certain qualities in other characters.\\ \\ 

A `complex character', also known as a round character, has depth in feelings and passions, has a variety \\of traits of a real human being, and evolves over time. They have their strengths, weaknesses, \\and they learn from their experiences. They tend to be more engaging to the reader, as they mirror \\the complexity of real people.\\ \\ 

In good stories, authors take a character who initially appears to be one-dimensional or stereotypical (flat) and \\add depth to them. This could be done by revealing more about their backstory, introducing unexpected traits \\or motivations, or having them grow and change in response to the events of the story. \\This transformation from a flat to a complex character can make the narrative more engaging and believable.
 \end{tabular} \\ \hline
\begin{tabular}[c]{@{}l@{}}Human\\ Instruction\end{tabular}             & \begin{tabular}[c]{@{}l@{}}\{\{M\}\}\\ \\ Based on the story that you just read, answer the following question.\\  \textit{\color{blue} Q) Does each character in the story feel developed at the appropriate complexity level, ensuring that no character} \\ \textit{\color{blue}feels like they are present simply to satisfy a plot requirement?}\\ -Yes \\ -No \\\\ Reasoning: \end{tabular}                                                                       \\ \hline
\begin{tabular}[c]{@{}l@{}}LLM\\ Instruction\end{tabular}               & \begin{tabular}[c]{@{}l@{}}\{\{M\}\}\\ \\ Given the story above, list each character and the level of development. Then overall, give your reasoning \\about the question below and give an answer to it between 'Yes' or 'No' only\\ \\ 
 \textit{\color{blue} Q) Does each character in the story feel developed at the appropriate complexity level, ensuring that no character} \\ \textit{\color{blue}feels like they are present simply to satisfy a plot requirement?}\end{tabular}                                                                                                                                                                                                                                 \\ \hline
\end{tabular}
\vspace{2ex}
\caption{\label{prompting}TTCW Elaboration2 (Character Development) }
\vspace{-5ex}
\end{table*}


% ==================================================



\begin{table*}[!ht]
\centering
\small
% \def\arraystretch{1.15}
\begin{tabular}{|l|l|}
\hline
\begin{tabular}[c]{@{}l@{}}Expert \\ Measure\end{tabular}               & \begin{tabular}[c]{@{}l@{}}Are there passages in the story that involve subtext and when there is subtext, does it enrich the story's setting \\or does it feel forced?\end{tabular}                                                                                                                                     \\ \hline
\begin{tabular}[c]{@{}l@{}}Expanded\\ Expert\\ Measure (M)\end{tabular} & \begin{tabular}[c]{@{}l@{}} `Surface' level: This is the most apparent and straightforward level of a story. It includes the visible actions, \\explicit dialogue, and clear descriptions. This is what literally happens in the plot: the characters' actions, events, \\and the apparent consequences.\\ \\ 

`Subtext' level: This is the underlying or implicit meaning that isn't directly stated but can be inferred from \\the characters'  actions, dialogue, and other elements of the story. Subtext often reveals deeper truths about \\characters, themes, or the overall message of the piece. It could be a hidden motive, an unstated\\ emotion, a cultural commentary, or a symbolic meaning.\\ \\ 

For example, in a conversation between two characters, the surface text might be polite and cordial, but the \\subtext \\discerned from the characters' nonverbal cues, previous interactions, or the context of their conversation\\ — could suggest tension or hostility.\\ \\ 

Effective fiction often operates on both levels. The surface text keeps the reader engaged with the plot and \\characters, while the subtext provides depth, complexity, and additional layers of interpretation, \\contributing to a richer and more rewarding reading experience.
 \end{tabular} \\ \hline
\begin{tabular}[c]{@{}l@{}}Human\\ Instruction\end{tabular}             & \begin{tabular}[c]{@{}l@{}}\{\{M\}\}\\ \\ Based on the story that you just read, answer the following question.\\  \textit{\color{blue} Q) Are there passages in the story that involve subtext and when there is subtext, does it enrich the story's setting} \\ \textit{\color{blue} or does it feel forced?}\\ -Yes \\ -No \\\\ Reasoning: \end{tabular}                                                                       \\ \hline
\begin{tabular}[c]{@{}l@{}}LLM\\ Instruction\end{tabular}               & \begin{tabular}[c]{@{}l@{}}\{\{M\}\}\\ \\ Given the story above, answer the following question. Please first explain your reasoning step by step \\and then give an answer between 'Yes' or 'No' only\\ \\ 
 \textit{\color{blue} Q)Are there passages in the story that involve subtext and when there is subtext, does it enrich the story's setting} \\ \textit{\color{blue} or does it feel forced?}\end{tabular}                                                                                                                                                                                                                                 \\ \hline
\end{tabular}
\vspace{2ex}
\caption{\label{prompting}TTCW Elaboration3 (Rhetorical Complexity) }
\vspace{-5ex}
\end{table*}


% ==================================================


%%%%%%%%%%%%%%%%%%%%%%%%%%%%%%%%%%%%%%%%%%%%%%%%%%%%%%%%%%%%

% \input{sections/neurips-checklist}

\end{document}