We evaluated the restoration performance of the Stochastic-Restoration-GAN (SR-GAN) \cite{lattner2021stochastic} and Apollo models across various bitrates and music genres on the combined test set of MUSDB18-HQ and MoisesDB (with 5000 samples for each case). The test set encompasses a wide range of music genres, including vocals, single instruments, and mixed instruments, aiming to comprehensively assess each model's restoration capabilities.

\textbf{Bitrate Impact Analysis.} Fig.\ref{fig:plot} compares the performance of the Apollo model and the Stochastic-Restoration-GAN (SR-GAN) at different bitrates (ranging from 24 kHz to 128 kHz). The experimental results demonstrated that Apollo consistently outperformed SR-GAN across all bitrates, particularly in addressing issues such as frequency band voids or reduced signal bandwidth, as reflected by SI-SNR and SDR scores. Additionally, Apollo significantly improved audio restoration quality as measured by VISQOL. Project page\footnote{\url{https://cslikai.cn/Apollo/}} for Apollo's reconstructed audio given multiple MP3 bitrates.

\textbf{Music Genre Impact Analysis.} Table~\ref{tab:stems} further illustrates the performance of both models across different music genres. In audio scenarios involving vocals, single instruments, mixed instruments, and a combination of instruments with vocals, Apollo consistently surpasses SR-GAN, with its advantage being especially pronounced in complex scenarios with mixed instruments and vocals. This is attributed to Apollo's alternating band and sequence modeling design, which emphasizes capturing and restoring complex spectral information. Compared to SR-GAN, Apollo delivers higher user ratings (VISQOL) with comparable inference speed while maintaining a more compact model size. This is especially important for real-time communications and live audio restoration, where low latency is critical to the user experience.

