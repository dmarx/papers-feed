\FloatBarrier

\section{From \mup\ to \umup} \label{app:from_mup_to_umup}

Here we outline additional details to help readers follow the process of deriving \umup\ from the combination of Unit Scaling and \mup. Our first step of dropping $\sigma_W$ and $\basefanin$, and moving $\alpha_W$s to functions, results in \Cref{table:mup_to_umup_1}. This intermediate scheme does not yet satisfy Unit Scaling, but simplifies the HP rules in preparation for further changes. Note that we have also removed $\hat \eta_\mathrm{emb}$ as we don't include this HP in our \umup\ extended HP set. We have included residual scaling rules here, in accordance with \depthmup, which we intend \umup\ to satisfy, though our standard \mup\ implementation doesn't use it.

\begin{table}[h]
  \centering
  \caption{An intermediate scheme resulting from dropping those HPs from \mup\ which are not needed under \umup.}
  \label{table:mup_to_umup_1}
  \begin{tabular}{l @{\hspace{0.8\tabcolsep}} lcccc}
      \toprule
      \multicolumn{2}{c}{\multirow{2}{*}[-0.2em]{ABC-multiplier}} & & Weight Type \vspace{0.2em} & & \multirow{2}{*}[-0.2em]{Residual}
      \\\cline{3-5}
      \rule{0pt}{1em} & & Input & Hidden & Output &
      \\
      \midrule
      parameter & ($A_W$) & $1$ & $1$ & $\frac 1 {\fanin}$ & $\frac 1 {\sqrt{\depth}}$\textsuperscript{*}
      \rule{0pt}{1.2em}\\[0.75em]
      initialization & ($B_W$) & $1$ & $\frac 1 {\sqrt{\fanin}}$ & $1 $ & ---
      \\[0.65em]
      Adam LR & ($C_W$) & $\eta$ & $\eta \, \frac 1 {\fanin}$ & $\eta$ & $\frac 1 {\sqrt{\depth}}$\phantom{\textsuperscript{*}}
      \\[0.45em]
      \bottomrule
      \vspace{-0.4cm}
      \\
  \end{tabular}
\end{table}
