\section{Qualitative Analysis}
\label{appendix_sec:qualitative_analysis}

\begin{table*}[!thbp]
\centering
\small
\begin{tabular}{@{}L{3.5cm}L{7.3cm}L{4cm}@{}}
\toprule
\rowcolor{white} % Example coloring for header row
\textbf{Query} & \textbf{\gear} & \textbf{HippoRAG w$/$ IRCoT} \\
\midrule

% First row
\cellcolor{white}\begin{tabular}[l]{@{}l@{}}What team is the player who\\ led the league in sacks in\\ 2017 a member of?\end{tabular} & 
\cellcolor{customGreen}\begin{tabular}[c]{@{}l@{}}("Chandler Jones", "team",\ "Arizona Cardinals")\\ ("Chandler Jones", "led the\ league in sacks in", "2017")\end{tabular} & 
\cellcolor{customRed}To answer the question about the player who led the league in sacks in 2017, I need to identify the player and their team from external sources, as the provided documents do not contain that specific information \\
\midrule

% Second row
\cellcolor{white}\begin{tabular}[l]{@{}l@{}}Who is the CEO of\\ FleetBoston Financial's\\ purchaser?\end{tabular} & 
\cellcolor{customGreen}\begin{tabular}[c]{@{}l@{}}("FleetBoston Financial",\ "purchaser", "Bank of America")\\ ("Bank of America", "CEO",\ "Brian Moynihan")\end{tabular} & 
\cellcolor{customYellow}The purchaser of FleetBoston Financial is Bank of America. To find the CEO of Bank of America, I need to look for information about the current leadership of the bank. Next, I will check for the current CEO of Bank of America to answer the question accurately.\\
\midrule

% Third row
\cellcolor{white}\begin{tabular}[l]{@{}l@{}}What UK label was purchased\\ by the company that, along\\ with the distributor of The\\ Cube, and ABC, is the other\\ major New York-based\\ broadcaster?\end{tabular} & 
\cellcolor{customGreen}\begin{tabular}[c]{@{}l@{}}("CBS", "purchased",\ "Oriole Records")\\ ("CBS", "is a major broadcaster\ based in", "New York")\\ ("NBC", "is a major broadcaster\ based in", "New York")\\ ("The Cube", "was distributed\ by", "NBC")\end{tabular} & 
\cellcolor{customYellow}The original network of Undercovers is NBC, which is one of the major broadcasters based in New York. Now, I need to find out which UK label was bought by NBC ... \\
\midrule

% Fourth row
\cellcolor{white}\begin{tabular}[l]{@{}l@{}}What was the 2018\\ population of the Italian\\ city that's underwater?\end{tabular} & 
\cellcolor{customGreen}\begin{tabular}[c]{@{}l@{}}("Venice", "population in 2018",\ "260,897")\end{tabular} & 
\cellcolor{customRed}The Italian city that is underwater is Krag, British Columbia, which is a ghost town... \\
\bottomrule
\end{tabular}
\caption{Comparison of MuSiQue queries where \gear achieves 100\% recall at R@15 in a single iteration, while HippoRAG w$/$ IRCoT shows lower performance despite using all four available iterations. Cell colors indicate recall performance: \colorbox{customGreen}{green} for 100\% recall, \colorbox{customRed}{red} for 0\% recall, and \colorbox{customYellow}{yellow} for any intermediate value. Cell values in \gear represent the proximal triples stored in the Gist Triple Memory. Cell values in HippoRAG w$/$ IRCoT represent IRCoT's thought process.}
\label{tab:triple_extraction_comparison}
\end{table*}


Table \ref{tab:triple_extraction_comparison} showcases some query instances where \gear achieves perfect recall in a single iteration, while HippoRAG w$/$ IRCoT achieves lower recall and consumes all available iterations. The presented examples illustrate how \gear's Gist Memory $\mathcal{G}^{(n)}$ precisely captures the essential information needed to answer MuSiQue's queries, maintaining the appropriate level of granularity without including superfluous details. In contrast, HippoRAG w/ IRCoT struggles to retrieve crucial information—whether due to limitations in its triple extraction step or retriever functionality—such as the exact population of Venice, which is necessary for accurate responses. Furthermore, the verbose nature of IRCoT's thought process component contrasts with \gear's streamlined approach. The lack of such verbose component in our approach contributes to the fact that \gear requires fewer LLM tokens than the competition, as explained in subsection \ref{subsec:gear_efficient}. 
