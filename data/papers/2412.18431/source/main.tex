% This must be in the first 5 lines to tell arXiv to use pdfLaTeX, which is strongly recommended.
\pdfoutput=1
% In particular, the hyperref package requires pdfLaTeX in order to break URLs across lines.

\documentclass[11pt]{article}
% \usepackage{showframe}
% Change "review" to "final" to generate the final (sometimes called camera-ready) version.
% Change to "preprint" to generate a non-anonymous version with page numbers.
\usepackage[preprint]{acl}

% Standard package includes
\usepackage{times}
\usepackage{latexsym}

% For proper rendering and hyphenation of words containing Latin characters (including in bib files)
\usepackage[T1]{fontenc}
% For Vietnamese characters
% \usepackage[T5]{fontenc}
% See https://www.latex-project.org/help/documentation/encguide.pdf for other character sets

% This assumes your files are encoded as UTF8
\usepackage[utf8]{inputenc}

% This is not strictly necessary, and may be commented out,
% but it will improve the layout of the manuscript,
% and will typically save some space.
\usepackage{microtype}

% This is also not strictly necessary, and may be commented out.
% However, it will improve the aesthetics of text in
% the typewriter font.
\usepackage{inconsolata}

% Including images in your LaTeX document requires adding
% additional package(s)
\usepackage{graphicx}

% Our LaTeX packages
\usepackage{array}
\usepackage{amsfonts}
\usepackage{amsmath}
\usepackage{booktabs}
\usepackage{xspace}
\usepackage{paralist}
\usepackage{algorithm}
\usepackage{colortbl}
\usepackage[noend]{algpseudocode}
\usepackage{multirow}
\usepackage{cuted}
\usepackage{tcolorbox}
\usepackage{caption}
\tcbuselibrary{breakable}  % Load the breakable library
\usepackage{multicol}
% \usepackage[table]{xcolor}
% \usepackage{wasysym} % for \twonote

\definecolor{customGreen}{RGB}{223, 242, 223}
\definecolor{customYellow}{RGB}{250, 249, 214}
\definecolor{customRed}{RGB}{242, 223, 223}
\definecolor{customBlue}{RGB}{214, 224, 250}

\newtcolorbox[list inside=prompt,auto counter,number within=section]{prompt}[1][]{
    colbacktitle=black!60,
    coltitle=white,
    fontupper=\footnotesize,
    boxsep=5pt,
    left=0pt,
    right=0pt,
    top=0pt,
    bottom=0pt,
    boxrule=1pt,
    title={#1},
    breakable,
    #1, % Additional options
}

% If the title and author information does not fit in the area allocated, uncomment the following
%
%\setlength\titlebox{<dim>}
%
% and set <dim> to something 5cm or larger.

% Our LaTeX commands
\newcolumntype{L}[1]{>{\raggedright\let\newline\\\arraybackslash\hspace{0pt}}m{#1}}
\newcolumntype{C}[1]{>{\centering\let\newline\\\arraybackslash\hspace{0pt}}m{#1}}
\newcolumntype{R}[1]{>{\raggedleft\let\newline\\\arraybackslash\hspace{0pt}}m{#1}}

\DeclareMathOperator*{\argmax}{\arg\max}
\newcommand{\zhili}[1]{\textcolor{blue}{(Zhili: #1)}}
\newcommand{\chenxin}[1]{\textcolor{red}{(Chenxin: #1)}}
\newcommand{\pavlos}[1]{\textcolor{orange}{(Pavlos: #1)}}
\newcommand{\pascual}[1]{\textcolor{brown}{(Pascual: #1)}}
\newcommand{\gear}{\textsc{GeAR}\xspace}

\title{\gear: Graph-enhanced Agent for Retrieval-augmented Generation}

\author{{\bf Zhili Shen}$^{\mathbf{\dagger}}$ \quad {\bf Chenxin Diao}$^{\mathbf{\dagger}}$ \quad {\bf Pavlos Vougiouklis}$^{\mathbf{\dagger}}$ \quad {\bf Pascual Merita}$^{\mathbf{\dagger}}$ \\
{\bf Shriram Piramanayagam} \quad {\bf Damien Graux} \quad {\bf Dandan Tu}\\ {\bf Zeren Jiang} \quad {\bf Ruofei Lai} \quad {\bf Yang Ren} \quad \quad {\bf Jeff Z. Pan}\\ 
Huawei Technologies Co., Ltd.
\\Edinburgh, United Kingdom\\\texttt{\{}{\href{mailto:zhilishen@huawei.com}{\texttt{zhilishen}}}\texttt{,} 
\href{mailto:chenxindiao@huawei.com}{\texttt{chenxindiao}}\texttt{,} \href{mailto:pavlos.vougiouklis@huawei.com}{\texttt{pavlos.vougiouklis}}\texttt{,}
\href{mailto:pascual.merita@h-partners.com}{\texttt{pascual.merita}}\texttt{\}@huawei.com}\\\texttt{\{}{\href{mailto:renyang1@huawei.com}{\texttt{renyang1}}}\texttt{,} 
\href{mailto:jeff.pan@huawei.com}{\texttt{jeff.pan}}\texttt{\}@huawei.com}\\
}


\begin{document}
\maketitle
\begin{abstract}
Retrieval-augmented generation systems rely on effective document retrieval capabilities. By design, conventional sparse or dense retrievers face challenges in multi-hop retrieval scenarios. In this paper, we present \gear, which advances RAG performance through two key innovations: \begin{inparaenum}[(i)] \item graph expansion, which enhances any conventional base retriever, such as BM25, and \item an agent framework that incorporates graph expansion\end{inparaenum}. Our evaluation demonstrates \gear 's superior retrieval performance on three multi-hop question answering datasets. Additionally, our system achieves state-of-the-art results with improvements exceeding $10\%$ on the challenging MuSiQue dataset, while requiring fewer tokens and iterations compared to other multi-step retrieval systems.  
\end{abstract}
\renewcommand{\thefootnote}
{\fnsymbol{footnote}}
\setcounter{footnote}{2}
\footnotetext{The authors contributed equally to this work.}
\renewcommand{\thefootnote}
{\arabic{footnote}}
\setcounter{footnote}{0}

\section{Introduction}
%
Neural network (NN) learning has underpinned state of the art empirical
results in numerous applied machine learning tasks (see for
instance~\cite{krizhevsky2012imagenet,lecun2015deep}). Nonetheless, neural
network learning remains rather poorly understood in several regards.
Notably, it remains unclear why training algorithms find good weights, how
learning is impacted by the network architecture and activations, what is
the role of random weight initialization, and how to choose a concrete
optimization procedure for a given architecture.

We start by analyzing the expressive power of NNs subsequent to the random
weight initialization. The motivation is the empirical success of training
algorithms despite inherent computational intractability, and the fact that
they optimize highly non-convex objectives with potentially many local minima.
Our key result shows that random initialization already positions learning
algorithms at a good starting point. We define an object termed a {\em
computation skeleton} that describes a distilled structure of feed-forward
networks. A skeleton induces a family of network architectures along with a
hypothesis class $\ch$ of functions obtained by certain non-linear
compositions according to the skeleton's structure.  We show that the
representation generated by random initialization is sufficiently rich to
approximately express the functions in $\ch$. Concretely, all functions in
$\ch$ can be approximated by tuning the weights of the last layer, which is
a convex optimization task.

In addition to explaining in part the success in finding good weights, our
study provides an appealing perspective on neural network learning.  We
establish a tight connection between network architectures and their dual
kernel spaces. This connection generalizes several previous constructions
(see Sec~\ref{sec:related}). As we demonstrate, our dual view gives rise to
design principles for NNs, supporting current practice and suggesting
new ideas. We outline below a few points.

\begin{itemize}

\item Duals of convolutional networks appear a more suitable fit for
	vision and acoustic tasks than those of fully connected networks.

\item Our framework surfaces a principled initialization scheme. It is
	very similar to common practice, but incorporates a small correction.

\item By modifying the activation functions, two consecutive fully connected
	layers can be replaced with one while preserving the network's dual kernel.

\item The ReLU activation, i.e. $x \mapsto \max(x,0)$, possesses favorable
	properties. Its dual kernel is expressive, and it can be well approximated by
	random initialization, even when the initialization's scale is moderately
	changed.

\item As the number of layers in a fully connected network becomes very
	large, its dual kernel converges to a degenerate form for any non-linear
	activation.

\item Our result suggests that optimizing the weights of the last layer can
	serve as a convex proxy for choosing among different architectures prior
	to training. This idea was advocated and tested empirically
	in~\cite{saxe2011random}.

\end{itemize}

\section{Related Work}
\label{sec:related_work}

\paragraph{Attention variants and distributed attention}
Ever since attention became popular with the Transformer
architecture~\citep{vaswani2017attention}, there has been a large body of work
on approximating attention to scale it to longer sequences.
These approximation methods can generally be categorized into two classes:
sparse and low-rank.
Sparse attention only computes some entries of the attention matrix ($\mathrm{softmax}(\vQ
\vK^T)$) and assumes that other entries are zero.
Different methods have different ways of choosing which entries should be zero,
either with a fixed pattern~\citep{child2019generating}, with a sliding
window~\citep{beltagy2020longformer}, or with a dynamic pattern through
hashing~\citep{kitaev2020reformer} or routing~\citep{roy2020efficient}.
The low-rank approach instead assumes that the attention matrix has a low-rank
structure, and apply a pointwise nonlinearity to the query and
key~\citep{katharopoulos2020transformers} with random
projection~\citep{choromanski2021rethinking, peng2021random, xiong2021nystromformer}.
One can also combine the sparse and low-rank approximation for better
quality~\citep{zaheer2020bigbird,scatterbrain}.
However, these approximation methods typically do not offer the same model
quality as standard attention~\citep{tay2020efficient}, and so most large-scale
models do not employ these techniques.

There are other variants of attention aimed at reducing the size of the KV cache
to improve inference efficiency. Multi-query attention~\citep{shazeer2019fast} and grouped query
attention~\citep{ainslie2023gqa} tie different heads of $\vK$ and $\vV$, and
multiple query heads interact with the same key and value head.
Multi-head latent attention~\citep{deepseekv2} parameterizes the $\vK$ and $\vV$
as low-rank projections of a shared matrix to further reduce the KV cache size.
However, all of these approaches do not change the core computation
$\mathrm{softmax}(\vQ \vK^T) \vV$ during training and simply change how $\vQ, \vK, \vV$ are
obtained.
As a result, any efficiency or accuracy improvement to the standard attention
computation benefits these methods.

To extend to even longer context, attention computation can be distributed
across multiple GPUs.
Methods such as Ring attention~\citep{liu2023ring,liu2024world} and
variants~\citep{brandon2023striped} can reach a context length of up to 1
million.
They use \fa (or \faa) as a primitive, and so the improvement from \fat would
benefit these distributed attention methods as well.

\paragraph{Alternative architectures}
Motivated by the limitations of attention, a variety of alternative
architectures have been proposed.
They build on the connection between linear
attention~\citep{katharopoulos2020transformers} and recurrent neural networks
(RNNs).
RWKV~\citep{peng2023rwkv}, H3~\citep{dao2023hungry}, MEGA~\citep{ma2023mega},
Retnet~\citep{sun2023retentive}  enhance the expressivity of the simple
cumulative sum in linear attention with more sophisticated recurrences.
Mamba~\citep{gu2023mamba} and xLSTM~\citep{beck2024xlstm} use learnable
weighting for the recurrence and can match the quality of Transformers in
language modeling at small or medium scale.
These approaches can be connected to generalizations of linear attention through
the lens of the structure of the token-mixing matrix~\citep{dao2024transformers}.
These models have started to see some traction, seeing usage in some medium to
large-scale models such as Jamba~\citep{jamba}, Zamba~\citep{zamba},
Megalodon~\citep{ma2024megalodon}, and Mamba2-hybrid~\citep{waleffe2024empirical}.
For the highest quality, these SSM- and RNN-based models still employ
many layers of attention.
We expect that techniques to speed up attention presented in this work will be
useful to speedup these alternative architectures.

\paragraph{Low-precision attention}
Quantization is a promising approach to speed up attention, but they have mostly
focused on reducing the space for KV cache for inference efficiency.
QuIP~\citep{chee2024quip} and QuIP\#\citep{tseng2024quip} use incoherent processing to reduce the quantization,
and we adapted this technique for FP8 \fat.
Recent work suggests that for inference the KV cache is highly compressible down to 4-, 3-, or
even 2-bits~\citep{hooper2024kvquant, liu2024kivi}.
However, quantization during training is still challenging as higher precision
is typically required for stable training.

\paragraph{Hardware-aware Algorithms}
Our work presented in this paper focuses on the micro-architecture
specific tuning to leverage new instruction sets and adopt a natively
asynchronous programming model. There are other orthogonal axes for
hardware-aware algorithm co-design being explored.
A recent example of this is LeanAttention~\citep{sanovar2024-leanattention},
which recognizes the poor GPU occupancy and high memory bandwidth requirements
of the sequential token generation phase as primary bottlenecks for inference
and optimizes it via a smarter load balancing strategy similar to Stream-K
load balancing~\citep{streamk} to achieve nearly peak occupancy.
There is a large literature on optimizing GEMM for specific hardware that employs
many of the same techniques.
As an example, \citet{abdel2016batched} presents a high performance batched GEMM kernel on
K40c Graphics Processing Units (GPU) for both fixed and variable sizes,
proposing specialized GEMM designs
and a comprehensive autotuning process to deliver state-of-the-art 
performance.


\section{Hybrid Retrieval Strategy}
\label{appendix_sec:preliminaries}

A list of $k$ passages by merging returned passages from both a $\mathbf{C} = \left \{c_1, c_2, \ldots, c_C \right \}$ index of textual passages and a $\mathbf{T} = \left \{t_1, t_2, \ldots,t_T: t_j = \left ( s_j, p_j, o_j \right ) \right \}$ index representing a set of triples associated with the passages in $\mathbf{C}$, using Reciprocal Rank Fusion (RRF) \cite{Cormack2009}, can be obtained, as follows:
\begin{align}
h^k_{\text{base}}\left( \mathbf{q}, {\mathbf{C} \cup \mathbf{T}} \right) 
    &= \text{RRF} \Big ( h^k_{\text{base}}\left( \mathbf{q}, {\mathbf{C}}\right), \nonumber \\
    &\quad\quad h^k_{\text{base}}\left( \mathbf{q}, {\mathbf{T}} \right) \Big ),
    \label{eq:triple_passage_index_retrieve}
\end{align}
% we multiply $k$ for the $T$ index by 5.
where $h^k_{\text{base}}\left( \mathbf{q}, {\mathbf{C}}\right)$ and $h^k_{\text{base}}\left( \mathbf{q}, {\mathbf{T}}\right)$ are the passages retrieved from $\mathbf{C}$ and $\mathbf{T}$, after a base retrieval step on each index separately. In this case, each triple $\in h^k_{\text{base}}\left( \mathbf{q}, {\mathbf{T}} \right)$ is mapped to its corresponding passage, ensuring that top-$k$ unique passages are returned after considering the triple scores in $\mathbf{T}$.


\iffalse
Given an input query $\mathbf{q'}$ \pascual{Shouldn't be use q instead of $\mathbf{q'}$?}, a \textit{baseline} retrieval step includes selecting the most relevant passages, using a combination of hybrid retrieval steps. Each hybrid retrieval search step returns top-$k$ items from an index of interest $\mathbf{R} = \left \{r_1, \ldots, r_R \right \}$ s.t. $\mathbf{R} \in \left (\mathbf{C} \cup \mathbf{T} \right ) \setminus \left (\mathbf{C} \cap \mathbf{T} \right )$ \pascual{Isn't it unnecessary to include $(\mathbf{C} \cap \mathbf{T})$? (Almost) by definition there won't be any intersection, although I guess a passage could be just a triple? Note that in the following sections we only use $\mathbf{C} \cup \mathbf{T}$} by aggregating the results of semantic search and $\text{score}_{\text{BM25}}$ using Reciprocal Rank Fusion (RRF) \pascual{Should we explain what the RRF function does, or should be cite a paper that does so?}, as follows: 
\begin{align}
h^k_{\text{hybrid}}\left( \mathbf{q'}, {\mathbf{R}}\right ) = \text{RRF}\left(h^k_{\text{dense}}, h^k_{\text{BM25}}\right),
\end{align}
where $h^k_{\text{dense}} \subseteq \mathbf{R}$ and $h^k_{\text{BM25}}\subseteq \mathbf{R}$ are functions
returning sets of items from $\mathbf{R}$, in descending order according to $\text{score}_{\text{dense}}$ and $\text{score}_{\text{BM25}}$ respectively, s.t. 
$\text{score}_{\text{dense}}  \left( \mathbf{q'}, r_j \right ) \geq\text{score}_{\text{dense}}  \left( \mathbf{q'}, r_{j+1} \right )$ $\forall r_j \in h^k_{\text{dense}}$ and $j \in \left [ 1, k - 1\right ]$ and $\text{score}_{\text{BM25}}  \left( \mathbf{q'}, r_j \right ) \geq\text{score}_{\text{BM25}}  \left( \mathbf{q'}, r_{j+1} \right )$ $\forall r_j \in h^k_{\text{BM25}}$ and $j \in \left [ 1, k - 1\right ]$ \pascual{Maybe we can simplify all this math by just saying the lists are ordered in descending order of their respective scores}.
\fi

\section{Retrieval with Synchronised Graph Expansion}
\label{sec:graph_retrieval}

\def\Tqinit{\mathbf{T}_\mathbf{q}}


\begin{figure}[thbp]
  \includegraphics[width=\columnwidth]{figures/gear-sys-fig.pdf}
  \caption{\label{fig:system_diagram}System Architecture}
\end{figure}

% Start: Zhili --------------------------


Given an input query $\mathbf{q}$, let $\mathbf{C}_\mathbf{q}' = h^k_{\text{base}}\left( \mathbf{q}, {\mathbf{C}}\right )$  be a list of passages returned by the base retriever\footnote{The choice of a base retriever within our framework is flexible, without requiring any multi-hop capabilities.}.
Given this initially retrieved list of passages, $\mathbf{C}_\mathbf{q}'$, our goal is to derive relevant multi-hop contexts (passages) by retrieving a sub-graph of triples that interconnect their source passages. There are two challenges for materialising such sub-graph retrieval: \begin{inparaenum}[(i)]\item how to locate initial triples (i.e. starting nodes) $\Tqinit$, and \item how to expand the graph based on initial triples while reducing the search space\end{inparaenum}. The following sections address these challenges respectively, within \gear.



\subsection{Knowledge Synchronisation}
\label{subsection:knowledge_syncro}
\def\linkTriple{\texttt{tripleLink}}

We describe a knowledge \textbf{Sync}hronisation (\textbf{Sync}) process for locating initial nodes for graph expansion. We first employ an LLM to \texttt{read} $\mathbf{C}_\mathbf{q}'$ (see Appendix~\ref{subsec:online_retrieval_prompts}) and summarise knowledge triples that can support answering the current query $\mathbf{q}$, as defined:
\begin{align}
    \mathbf{T}_\mathbf{q}' = \texttt{read}\left (\mathbf{C}_\mathbf{q}', \mathbf{q}\right ).
    \label{eq:proximal_read}
\end{align}
$\mathbf{T}_\mathbf{q}'$ is a collection of triples to which we refer as \textit{proximal triples}. Initial nodes $\Tqinit$ for graph expansion can then be identified by linking each triple in $\mathbf{T}_\mathbf{q}'$ to a triple in $\mathbf{T}$, using the \linkTriple{} function:
\begin{align}
    \Tqinit =\left \{t_i | t_i = \linkTriple(t_i') ~ \forall t_i' \in \mathbf{T}_\mathbf{q}'\right \}.
\end{align}
The implementation of \linkTriple{} can vary. However, in this paper we consider it to be simply retrieving the most similar triple from $\mathbf{T}$.



\begin{algorithm}[ht]
\textbf{Input:} $\mathbf{q}$: query \\
\hspace*{3em} $b$: beam size \\
\hspace*{3em} $l$: maximum length \\
\hspace*{3em} $\mathrm{score}(\cdot, \cdot)$: scoring function \\
\hspace*{3em} $\{t_1, t_2, \ldots, t_n\}$: initial triples \\
\hspace*{3em} $\gamma$: hyperparameter for diversity


\begin{algorithmic}[1]
\State $B_0 \gets [\;]$
\For{$t \in \{t_1, t_2, ..., t_n\}$}
    \State $s \gets \mathrm{score}(\mathbf{q}, [t])$
    \State $B_0.\mathrm{add}(\langle s, [t] \rangle)$
\EndFor

\State $B_0 \gets \mathrm{top}(B_0, b)$


\For{$i \in \{1, \dots, l - 1\}$}
    \State $B \gets [\;]$
    
    \For{$\langle s, T \rangle \in B_{i-1}$}
        \State $V \gets [\;]$

        \For{$t \in \mathrm{get\_neighbours}(T.\mathrm{last}())$}
            \If{$\mathrm{exists}(t, B_{i-1})$}
                \State \textbf{continue}
            \EndIf
            
            \State $s' \gets s + \mathrm{score}(\mathbf{q}, T \circ t)$ ~ \texttt{\# concat} 
            \State $V.\mathrm{add}(\langle s', T \circ t \rangle)$
        \EndFor

        \State $\mathrm{sort}(V, \mathrm{descending})$

        \For{$n \in \{0, \dots, V.\mathrm{length()} - 1\}$}
            \State $\langle s', T \circ t \rangle \gets V[n]$
            \State $s' \gets s' \times e^{- \frac{\mathrm{min}(n, \gamma)}{\gamma}}$
            \State $B.\mathrm{add}(\langle s', T \circ t \rangle)$
        \EndFor
        
    \EndFor
    \State $B_i \gets \mathrm{top}(B, b)$
    
\EndFor

\State \Return $B_i$
\end{algorithmic}

\caption{Diverse Triple Beam Search}
\label{alg:beam_search}
\end{algorithm}

\subsection{Diverse Triple Beam Search}

We borrow the idea of constructing reasoning triple chains \cite{Fang2024} for expanding the graph, and present a retrieval algorithm: \textit{Diverse Triple Beam Search} (see Alg.~\ref{alg:beam_search}). 

We maintain top-$b$ sequences (beams) of triples and the scores at each step are determined by a scoring function. In this paper, we focus on leveraging a dense embedding model to compute the cosine similarity between embeddings of the query and a candidate sequence of triples, leaving other implementations of the scoring function for future work (see Section~\ref{sec:limitations}).

Considering all possible triple extensions at each step, in a Viterbi decoding fashion, would be intractable due to the size of $\mathbf{T}$. Consequently, we define the neighbourhood of a triple as the set of triples with shared head or tail entities (i.e. $\mathrm{get\_neighbours}$ in Alg.~\ref{alg:beam_search}). During each expansion step, we only consider neighbours of the last triple in the sequence, and avoid selecting previously visited triples (i.e. $\mathrm{exists}$ in Alg.~\ref{alg:beam_search}) to further reduce the search space.

While regular beam search can reduce the search space, it is prone to producing high-likelihood sequences that differ only slightly from one another \cite{Ippolito2019, Vijayakumar2018}. Our algorithm increases the diversity across beams to improve the recall for retrieval. In detail, for each beam, we sort candidate sequences extended from that beam in descending order, and weight their scores based on their relative positions. Candidate sequences that are ranked lower, within a beam, will receive smaller weights. Consequently, the resulting top-$b$ beams at each step are less likely to share the same starting sequence. 

The top-$b$ returned sequences are flattened in a breadth-first order. Each triple in the resulting list is then mapped to its source passage. This alignment between triples and passages is described in more detail in Section~\ref{sec:preliminaries}. Let $\widetilde{\mathbf{C}}_\mathbf{q}$ be the list of unique passages after alignment. The output of our graph expansion is then given by the Reciprocal Rank Fusion (RRF) \cite{Cormack2009} of $\widetilde{\mathbf{C}}_\mathbf{q}$ and the initial $\mathbf{C}_\mathbf{q}'$ list of passages :
\begin{align}
    \mathbf{C}_{\mathbf{q}} = \mathrm{RRF}\left(\widetilde{\mathbf{C}}_\mathbf{q}, \mathbf{C}_\mathbf{q}'\right ).
\end{align}
We refer to this graph-based method of retrieving relevant passages as \textbf{Sync}ronised \textbf{G}raph \textbf{E}xpansion (\textbf{SyncGE}).


\section{Multi-step Extension}


While SyncGE can enhance a base retriever with multi-hop context, some queries inherently require multiple steps to gather all necessary evidence. We materialise \gear by incorporating an agent with multi-turn capabilities, capable of interacting with the graph-retriever described above. We focus on:
\begin{itemize}
\item maintaining a gist memory of proximal knowledge obtained throughout the different steps 
\item incorporating a similar synchronisation process 
that summarises retrieved passages in proximal triples to be stored in this multi-turn gist memory
\item determining if additional steps are needed for answering the original input question
\end{itemize}
%
Within this multi-turn setting, the original input question $\mathbf{q}$ is iteratively decomposed into simpler queries: $\mathbf{q}^{(1)}, \ldots, \mathbf{q}^{(n)}$, where $\mathbf{q}^{(1)} = \mathbf{q}$ and $n \in \mathbb{N}$ represents the number of the current step.
For each query $\mathbf{q}^{(n)}$, we use the graph retrieval method introduced in Section~\ref{sec:graph_retrieval} in order to retrieve relevant passages $\mathbf{C}_{\mathbf{q}^{(n)}}$.



\subsection{Gist Memory Constructor}
To facilitate the multi-step capabilities of our agent, we introduce a \textit{gist memory}, $\mathcal{G}^{(n)}$, which is used for storing knowledge as an array of proximal triples. At the beginning of the first iteration, the gist memory is empty. During the $n$-th iteration, similar to the knowledge synchronisation module explained in Section~\ref{subsection:knowledge_syncro}, we employ an LLM to read a collection of retrieved paragraphs $\mathbf{C}_{\mathbf{q}^{(n)}}$ and summarise their content with proximal triples:

\begin{align}
\mathbf{T}_{\mathbf{q}^{(n)}}^{\mathcal{G}} = 
\begin{cases} 
    \texttt{read}\left(\mathbf{C}_{\mathbf{q}^{(n)}}, \mathbf{q} \right), & \text{if } n = 1 \\
    \texttt{read}\left(\mathbf{C}_{\mathbf{q}^{(n)}}, \mathbf{q}\textcolor{blue}{, \mathcal{G}^{(n-1)}} \right), & \text{if } n \geq 2
\end{cases}
\label{eq:proximal_read_agent}
\end{align}


Apart from the first iteration where Eq.~\ref{eq:proximal_read} and ~\ref{eq:proximal_read_agent} are identical, the inclusion of the memory in the \texttt{read} operation differentiates the construction of proximal triples produced at the subsequent steps compared to the ones from Eq.~\ref{eq:proximal_read}. $\mathcal{G}^{(n)}$ maintains the aggregated content of proximal triples s.t. 
\begin{align}
\mathcal{G}^{(n)} = \left[ \mathbf{T}_{\mathbf{q}^{(1)}}^{\mathcal{G}}  \circ \cdots \circ \mathbf{T}_{\mathbf{q}^{(n)}}^{\mathcal{G}} \right],
\end{align}where $\circ$ defines the concatenation operation. The triple memory serves as a concise representation of all the accumulated evidence, up to the $n$-th step. 

We believe the process introduced by the \texttt{read} step along with the information storage paradigm served by the gist memory, aligns well with the communication between the hippocampus and neocortex. The combination of the two establishes the synergetic behaviour between our graph retriever and the LLM that we seek to achieve within \gear.



\subsection{Reasoning for Termination}
After $\mathcal{G}^{(n)}$ is updated, we check the sufficiency of the accumulated evidence, within it, for answering the original question. This is achieved with the following LLM reasoning step:
\begin{align}
\mathbf{a}^{(n)}, \mathbf{r}^{(n)}   = \texttt{reason}(\mathcal{G}^{(n)}, \mathbf{q}),
\end{align}
% We can also call it 'sufficiency' instead of 'answerability'. I do not really have a preference.
where $\mathbf{a}^{(n)}$ denotes the query's answerability given the available evidence in $\mathcal{G}^{(n)}$, and $\mathbf{r}^{(n)}$ represents the reasoning behind this determination. When the query is deemed answerable, the system concludes its iterative process.



\subsection{Query Re-writing}
The query re-writing process leverages an LLM that incorporates three key inputs: the original query $\mathbf{q}$, the accumulated memory, and crucially, the reasoning output $\mathbf{r}^{(n)}$ from the previous step. This process can be formally expressed as:
\begin{align}
\mathbf{q}^{(n+1)} = \texttt{rewrite}\left (\mathcal{G}^{(n)}, \mathbf{q}, \mathbf{r}^{(n)} \right),
\end{align}
where $\mathbf{q}^{(n+1)}$ represents the updated query, which serves as input for the retriever in the next iteration.\\
\subsection{After Termination}
\gear aims to return a single ranked list of passages. Given the final gist memory $\mathcal{G}^{(n)}$ upon termination, we link each proximal triple in $\mathcal{G}^{(n)}$ to a list of passages as follows:
\begin{align}
    \mathbf{C}_{t_j} = \texttt{passageLink}\left(t_j, k\right),
\end{align}
where $j \in \left \{1, \dots, \vert\mathcal{G}^{(n)}\vert \right \}$. Similar to \texttt{tripleLink}, \texttt{passageLink} is implemented by retrieving passages with a triple as the query (see Appendix~\ref{appendixpara:passage_link}). The final list of passages returned by \gear is the RRF of the resulting linked passages and passages retrieved across iterations:
\begin{align}
\mathbf{C}_\mathbf{q}^{(n)} = \mathrm{RRF}\big(&\mathbf{C}_{t_1}, \ldots,\mathbf{C}_{t_{\vert\mathcal{G}^{(n)}\vert}}, \nonumber\\
    &\mathbf{C}_{\mathbf{q}^{(1)}}, \ldots, \mathbf{C}_{\mathbf{q}^{(n)}} \big).
\end{align}

All relevant prompts for the \texttt{read}, \texttt{reason} and \texttt{rewrite} steps are provided in Appendix~\ref{subsec:online_retrieval_prompts}.

\section{Experimental Setup}

We evaluate our proposed framework on three multi-hop QA datasets in the open-domain setting: \textbf{MuSiQue} (answerable subset) \cite{Trivedi2022}, \textbf{HotpotQA} \cite{Yang2018}, and \textbf{2WikiMultiHopQA} (2Wiki) \cite{Ho2020}.
For MuSiQue and 2Wiki, we use the data splits provided in IRCoT \cite{Trivedi2023}, while for HotpotQA we follow the same data setting as in HippoRAG \cite{Gutierrez2024}. Dataset-specific statistics can be found in Appendix \ref{appendix:dataset_stats}.

We measure both retrieval and QA performance, with our primary contributions focused on the retrieval component. For retrieval evaluation, we use Recall@$k$ (R@$k$) metrics for $k \in \left \{5, 10, 15\right \}$, showing the percentage of questions where the correct entries are found within the top-$k$ retrieved passages. We include an analysis about the selected recall ranks in Appendix \ref{appendix:reasoning_behind_retrieval_metrics}. Following standard practices, QA performance is evaluated with Exact Match (EM) and F1 scores \cite{Trivedi2023}.


\begin{table*}[t]
\small
\centering
\small
\begin{tabular}{@{}l@{\hspace{2pt}}lccccccccc@{}}
\toprule
& \multirow{2.5}{*}{\textbf{Retriever}} & \multicolumn{3}{c}{\textbf{MuSiQue}} & \multicolumn{3}{c}{\textbf{2Wiki}} & \multicolumn{3}{c}{\textbf{HotpotQA}}\\ 
\cmidrule{3-11}
& & R@5 & R@10 & R@15 & R@5 & R@10 & R@15 & R@5 & R@10 & R@15 \\ \midrule
\multirow{11}{*}{\parbox{2cm}{\textbf{Single-step\\Retrieval}}}
& ColBERTv2 & $39.4$ & $44.8$ & $47.7$ & $59.1$ & $64.3$ & $66.2$ & $79.3$ & $87.1$ & $90.1$ \\
& HippoRAG & $41.0$ & $47.0$ & $51.4$ & $\mathbf{75.1}$ & $\mathbf{83.2}$ & $\mathbf{86.4}$ & $79.8$ & $89.0$ & $92.4$ \\ 
& BM25 & $33.8$ & $38.5$ & $41.3$ & $59.5$ & $62.7$ & $64.1$ & $74.2$ & $83.6$ & $86.3$ \\ 
& \hspace{2mm} + NaiveGE & $37.5$ & $45.5$ & $48.4$ & $65.0$ & $70.7$ & $71.8$ & $79.1$ & $89.1$ & $91.9$ \\ 
& \hspace{2mm} + SyncGE & $\underline{44.7}$ & $\underline{52.6}$ & $\underline{57.4}$ & $70.5$ & $76.1$ & $79.3$ & $\underline{87.4}$ & $\underline{93.0}$ & $\underline{94.0}$ \\ 
& SBERT & $31.1$ & $37.9$ & $41.6$ & $41.2$ & $48.1$ & $51.5$ & $72.1$ & $79.3$ & $84.0$ \\
& {\hspace{2mm} + NaiveGE} & $32.2$ & $41.4$ & $45.4$ & $45.1$ & $54.0$ & $57.3$ & $76.1$ & $84.7$ & $88.8$ \\
& \hspace{2mm} + SyncGE & $41.6$ & $51.3$ & $54.2$ & $54.8$ & $64.9$ & $70.7$ & $84.1$ & $89.6$ & $92.8$ \\ 
& Hybrid & $39.9$ & $46.3$ & $49.1$ & $60.0$ & $65.8$ & $66.6$ & $77.8$ & $85.8$ & $89.7$ \\
& \hspace{2mm} + NaiveGE & $41.8$ & $49.4$ & $53.0$ & $63.0$ & $70.8$ & $72.6$ & $80.6$ & $89.4$ & $92.7$ \\
& {\hspace{2mm} + SyncGE} & $\mathbf{48.7}$ & $\mathbf{57.7}$ & $\mathbf{61.2}$ & $\underline{72.6}$ & $\underline{80.9}$ & $\underline{82.4}$ & $\mathbf{87.4}$ & $\mathbf{93.3}$ & $\mathbf{95.2}$ \\ 
\midrule
\multirow{4}{*}{\parbox{2cm}{\textbf{Multi-step}\\ \textbf{Retrieval}}}
& IRCoT (BM25) & $46.1$ & $\underline{54.9}$ & $57.9$ & $67.9$ & $75.5$ & $76.1$ & $87.0$ & $92.6$ & $92.9$ \\
& IRCoT (ColBERTv2) & $47.9$ & $54.3$ & $56.4$ & $60.3$ & $86.6$ & $69.7$ & $86.9$ & $92.5$ & $92.8$ \\
& HippoRAG w$/$ IRCoT
& $\underline{48.8}$ & $54.5$ & $\underline{58.9}$ & $\underline{82.9}$ & $\underline{90.6}$ & $\underline{93.0}$ & $\underline{90.1}$ & $\underline{94.7}$ & $\underline{95.9}$ \\
& \gear & $\mathbf{58.4}$ & $\mathbf{67.6}$ & $\mathbf{71.5}$ & $\mathbf{89.1}$ & $\mathbf{95.3}$ & $\mathbf{95.9}$ & $\mathbf{93.4}$ & $\mathbf{96.8}$ & $\mathbf{97.3}$ \\ \bottomrule
\end{tabular}
 \caption{Retrieval performance for single- and multi-step retrievers on MuSiQue, 2Wiki, and HotpotQA. Results are reported using Recall@$k$ (R@$k$) metrics for $k \in \left \{5, 10, 15\right \}$, showing the percentage of questions where the correct entries are found within the top-$k$ retrieved passages.}
 \label{tab:recall_main_table}
\end{table*}

\subsection{Baselines}
We evaluate \gear against strong, multi-step baselines, including IRCoT \cite{Trivedi2023} and a combination of HippoRAG w$/$ IRCoT \cite{Gutierrez2024} which, similar to our framework, includes a graph-retrieval component and a multi-step agent. To showcase the benefits of our graph retriever (i.e. SyncGE), we evaluate it against several stand-alone, single-step retrievers: \begin{inparaenum}[(i)]\item BM25, \item Sentence-BERT (SBERT), \item a hybrid approach that combines BM25 and SBERT results through RRF and \item HippoRAG\end{inparaenum}. Throughout the experiments, we refer to the single-step setup when an approach does not support several iterations and is not equipped with an LLM agent.





\subsection{Implementation Details}
To maintain consistency and validity in comparisons with the baselines on the splits used in this study, we conducted all experiments locally using their corresponding codebases.

In addition to our proposed single-step retriever, SyncGE, we evaluate a more \textit{naive} implementation of GE (i.e. NaiveGE) in order to explore the generality of the method when in resource-constrained setting, where no LLM is involved. In NaiveGE, we use all triples that are associated with $\mathbf{C}_\mathbf{q}'$ (see Section~\ref{sec:graph_retrieval}) for diverse triple beam search.


For all models using an LLM, we employ GPT-4o mini (\texttt{gpt-4o-mini-2024-07-18}) as the backbone model with a temperature of 0, both for offline triple extraction (i.e. how the $\mathbf{T}$ index in Section~\ref{sec:preliminaries} is formed) and online retrieval operations. Our triple extraction prompt (in Appendix \ref{sec:offline_prompts}) is adapted\footnote{Our approach uses a modified version of HippoRAG's triple extraction prompt that combines entity and triple extraction into a single step, while incorporating an additional demonstration and updated in-context examples.} from the ones used by \citeauthor{Gutierrez2024}. To ensure a fair comparison against \citeauthor{Gutierrez2024}, the closest work to ours, we run experiments with HippoRAG using our prompting setup\footnote{For transparency, we also compare against HippoRAG's original triple extraction prompt in Appendix \ref{appendix_sec:hipporag_results_original_prompt}, where we observe only minor differences across the two configurations.} for triple extraction. For evaluating QA performance, we use the prompts provided in Appendix~\ref{subsec:online_qa_prompts}. Further implementation details are provided in Appendix~\ref{appendix:detailed_implementation_details}.

\section{Results}
\label{section:results}

We performed an extensive series of evaluations of Llama 3, investigating the performance of: \textbf{(1)} the pre-trained language model, \textbf{(2)} the post-trained language model, and \textbf{(3)} the safety characteristics of Llama 3. We present the results of these evaluations in separate subsections below. 

\input{results/pretrained.tex}
\input{results/finetuned.tex}
LLMs can propagate harmful content, reinforce biases, or amplify misinformation. While users are responsible for assessing the potential risks of generated content, developers must prioritize legal and safety considerations, strengthening models against attacks that may bypass safety protocols. 

In line with the Biden-Harris US Executive Order on AI \citep{whitehouse2023fact}, we curated the Biden-Harris Redteam Dataset, consisting of 5000 instruction-response pairs, addressing key concerns such as harm, cyber-attacks, CNBR risks, illegal acts, and privacy infringement. This dataset was created using a combination of filtering human preference data on harmlessness and template-based methods, with responses reviewed and edited for quality and safety. We used this dataset to instruction-tune \system\ and evaluated its safety levels before and after tuning. Details are provided in Section \ref{sec:experiments}, with further dataset insights in Appendix \ref{ap:safety}.

\section{Discussion and conclusion}
\label{sec:discussion}

This work introduces a design for agents that assist users in generating images through an interactive process of proactive question asking and belief graph refinement. By dynamically updating its understanding of the user's intent, the agent facilitates a more collaborative and precise approach to image generation. Moreover, presenting the agent's belief graph can be a generalizable method for AI transparancy, which is an important factor given the increasing complexity of modern AI models. 

\textbf{Modular design.}  Our agent prototypes are highly modular: the agents use frozen T2I models to generate images based on the prompts that the agent updated. Therefore when a better off-the-shelf T2I model becomes available, it can be directly plugged into the agents and the system will achieve better performance without any additional adaptation\footnote{T2T scores in \Cref{tab:auto_eval} ablates the T2I model and only performs similarity on the captions. Our agents have achieved a 92\%+ T2T score, showing that their performance can be boosted by adopting better T2I models.}.  

\textbf{Personalized content.} By asking clarification questions, our agents enable a more customizable and personalized content creation experience. Because different groups of people may perceive helpfulness and harmfulness of contents differently, learning more about the user through clarification questions before generation can potentially mitigate risks of generating contents that are offensive to each specific user, and increase likelihoods of producing helpful outputs.


\textbf{Future work.} Alternative to the modular design, one can explore generating images directly from belief graphs and fine-tuning  LLM/VLMs on text/image trajectories that include asking questions. These may require a) collecting data such as gold-standard trajectories or annotations on the quality of trajectories of human-agent conversations and b) new approaches to fine-tune the model on multi-turn trajectories of images and text, which can potentially improve the performance of the agent.










\subsection*{Acknowledgements}
We would like to thank Jason Baldridge and Zoubin Ghahramani for insightful discussions on multi-turn T2I and belief states, Mahima Pushkarna for the help and consultation on user study. We would also like to thank Richard Song and Noah Fiedel for feedback on the paper.


Hyperbolic embeddings embed hierarchical information with high
fidelity and few dimensions. We explored the limits of this approach
by describing scalable, high quality algorithms. We hope the
techniques here encourage more follow-on work on the exciting
techniques of \citet{fb, ucl}. As future work, we hope to explore how
hyperbolic embeddings can be most effectively incorporated into downstream
tasks and applications.

\bibliography{main}

\appendix
\section{Hybrid Retrieval Strategy}
\label{appendix_sec:preliminaries}

A list of $k$ passages by merging returned passages from both a $\mathbf{C} = \left \{c_1, c_2, \ldots, c_C \right \}$ index of textual passages and a $\mathbf{T} = \left \{t_1, t_2, \ldots,t_T: t_j = \left ( s_j, p_j, o_j \right ) \right \}$ index representing a set of triples associated with the passages in $\mathbf{C}$, using Reciprocal Rank Fusion (RRF) \cite{Cormack2009}, can be obtained, as follows:
\begin{align}
h^k_{\text{base}}\left( \mathbf{q}, {\mathbf{C} \cup \mathbf{T}} \right) 
    &= \text{RRF} \Big ( h^k_{\text{base}}\left( \mathbf{q}, {\mathbf{C}}\right), \nonumber \\
    &\quad\quad h^k_{\text{base}}\left( \mathbf{q}, {\mathbf{T}} \right) \Big ),
    \label{eq:triple_passage_index_retrieve}
\end{align}
% we multiply $k$ for the $T$ index by 5.
where $h^k_{\text{base}}\left( \mathbf{q}, {\mathbf{C}}\right)$ and $h^k_{\text{base}}\left( \mathbf{q}, {\mathbf{T}}\right)$ are the passages retrieved from $\mathbf{C}$ and $\mathbf{T}$, after a base retrieval step on each index separately. In this case, each triple $\in h^k_{\text{base}}\left( \mathbf{q}, {\mathbf{T}} \right)$ is mapped to its corresponding passage, ensuring that top-$k$ unique passages are returned after considering the triple scores in $\mathbf{T}$.


\iffalse
Given an input query $\mathbf{q'}$ \pascual{Shouldn't be use q instead of $\mathbf{q'}$?}, a \textit{baseline} retrieval step includes selecting the most relevant passages, using a combination of hybrid retrieval steps. Each hybrid retrieval search step returns top-$k$ items from an index of interest $\mathbf{R} = \left \{r_1, \ldots, r_R \right \}$ s.t. $\mathbf{R} \in \left (\mathbf{C} \cup \mathbf{T} \right ) \setminus \left (\mathbf{C} \cap \mathbf{T} \right )$ \pascual{Isn't it unnecessary to include $(\mathbf{C} \cap \mathbf{T})$? (Almost) by definition there won't be any intersection, although I guess a passage could be just a triple? Note that in the following sections we only use $\mathbf{C} \cup \mathbf{T}$} by aggregating the results of semantic search and $\text{score}_{\text{BM25}}$ using Reciprocal Rank Fusion (RRF) \pascual{Should we explain what the RRF function does, or should be cite a paper that does so?}, as follows: 
\begin{align}
h^k_{\text{hybrid}}\left( \mathbf{q'}, {\mathbf{R}}\right ) = \text{RRF}\left(h^k_{\text{dense}}, h^k_{\text{BM25}}\right),
\end{align}
where $h^k_{\text{dense}} \subseteq \mathbf{R}$ and $h^k_{\text{BM25}}\subseteq \mathbf{R}$ are functions
returning sets of items from $\mathbf{R}$, in descending order according to $\text{score}_{\text{dense}}$ and $\text{score}_{\text{BM25}}$ respectively, s.t. 
$\text{score}_{\text{dense}}  \left( \mathbf{q'}, r_j \right ) \geq\text{score}_{\text{dense}}  \left( \mathbf{q'}, r_{j+1} \right )$ $\forall r_j \in h^k_{\text{dense}}$ and $j \in \left [ 1, k - 1\right ]$ and $\text{score}_{\text{BM25}}  \left( \mathbf{q'}, r_j \right ) \geq\text{score}_{\text{BM25}}  \left( \mathbf{q'}, r_{j+1} \right )$ $\forall r_j \in h^k_{\text{BM25}}$ and $j \in \left [ 1, k - 1\right ]$ \pascual{Maybe we can simplify all this math by just saying the lists are ordered in descending order of their respective scores}.
\fi

\section{Dataset Choices and Statistics}
\label{appendix:dataset_stats}

\begin{table}[ht]
\centering
\small
\begin{tabular}{@{}lrrr@{}}
\toprule
& \textbf{MuSiQue} & \textbf{2Wiki} & \textbf{HotpotQA} \\ \midrule
Split Source & IRCoT & IRCoT & HippoRAG \\ \midrule
\# Hops  & $2-4$ & $2$ & $2$ \\
\# Documents & $139,416$ & $430,225$ & $9,221$ \\
\# Test Queries & $500$ & $500$ & $1,000$\\ \midrule
\# Chunks ($\mathbf{C}$) & $148,793$  & $490,454$ & $10,293$ \\
\# Triples ($\mathbf{T}$) & $1,521,136$  & $4,993,637$ & $122,492$ \\
Av. \# $\mathbf{T}/\mathbf{C}$ & $10.2$  & $10.2$ & $11.9$ \\
\bottomrule
\end{tabular}
\caption{Dataset characteristics and preprocessing statistics, where triples are extracted from chunks, and Av. \# $\mathbf{T}$$/$$\mathbf{C}$ represents the average number of triples per chunk.}
\label{tab:dataset statistics}
\end{table}

Table \ref{tab:dataset statistics} serves as a summary of various facts and statistics related to the employed datasets and the chunking and triple extraction process introduced in Section \ref{sec:preliminaries}.


\paragraph{Reasoning behind dataset split choices}
For MuSiQue and 2Wiki, we use the data provided by \citeauthor{Trivedi2023}, including the full corpus and sub-sampled test cases for each dataset. To limit the experimental cost for HotpotQA, we follow \citeauthor{Gutierrez2024} setting where both the corpus and test split are smaller than IRCoT's counterpart.

\paragraph{Reasoning behind retrieval metrics}
\label{appendix:reasoning_behind_retrieval_metrics}
Our evaluation employs recall at ranks 5, 10, and 15 (R@5, R@10, R@15). While previous work like HippoRAG evaluate R@2, we choose higher rank thresholds since many questions in MuSiQue require information from more than two documents. Additionally, given modern LLMs' expanding context length capabilities \cite{Ding2024}, examining recall beyond R@5 (HippoRAG's highest evaluated rank) provides valuable insights. Following IRCoT's approach, we measure up to R@15 and include R@10 as an intermediate point, offering a comprehensive view of model performance across retrieval depths.

\section{More Implementation Details}
\label{appendix:detailed_implementation_details}



\subsection{Baselines Details}
\label{appendix:retrievers_implementation_details}
We implement all proposed approaches using Elasticsearch\footnote{\url{https://www.elastic.co}}. For SBERT, we employ the \texttt{all-mpnet-base-v2} model with approximate k-nearest neighbours and cosine similarity for vector comparisons. In IRCoT experiments, we evaluate both ColBERTv2 and BM25 retrievers — ColBERTv2 for alignment with HippoRAG's baselines, and BM25 for consistency with the original IRCoT implementation.

For all multi-step approaches, including ours, we follow \citeauthor{Gutierrez2024} with respect to the maximum number of retrieval iterations, which vary based on the hop requirements of each dataset. Thus, we use a maximum of 4 iterations for MuSiQue and 2 iterations for HotpotQA and 2Wiki.

\subsection{\gear Details}
\gear involves several hyperparameters, such as the beam size inside graph expansion. 
We randomly sampled $500$ questions from the MuSiQue development set, which we ensure not to overlap with the relevant test set. We select our hyperparameters based on this sample without performing a grid search across all possible configurations. Our goal is to demonstrate our method is ability to achieve state-of-the-art results without extensive parameter tuning. We acknowledge that a more thorough hyperparameter tuning may result in further improvements.


The initial retrieval phase utilises the chunks index $\mathbf{C}$ as the information source, while leaving the triple index $\mathbf{T}$ unused. Our graph expansion component implements beam search with length 2, width 10, and 100 neighbours per beam. The hyperparameter $\gamma$ employed in diverse triple beam search is set to twice the beam search width. For the scoring function, we use the cosine similarity score and the SBERT embedding model.

For the single-step configurations (i.e. any base retriever with NaiveGE or SyncGE), we set the base retriever's maximum number of returned chunks to match our evaluation recall threshold. With the multi-step setup, we maintain a consistent maximum of 10 retrieved chunks before knowledge synchronisation for the purpose of matching IRCoT's implementation. While this 10-chunk limitation applies to individual retrieval rounds, please note that the total number of accessible chunks can exceed this threshold through graph expansion and multiple \gear iterations.


\paragraph{\textrm{\texttt{passageLink}} Details\label{appendixpara:passage_link}}
We use \texttt{passageLink} to link each triple $t_j \in \mathcal{G}^{(n)}$ to its corresponding passages in $\mathbf{C}$ by running a retrieval step as follows:
\begin{align}
\mathbf{C}_{t_j} = h^k_{\text{base}}\left( t_j, {\mathbf{C} \cup \mathbf{T}} \right),
\end{align}where $j \in \left \{1, \dots, \vert\mathcal{G}^{(n)}\vert \right \}$ and $h^k_{\text{base}}\left( t_j, {\mathbf{C} \cup \mathbf{T}} \right)$ is the RRF of passages returned by both $\mathbf{T}$ and $\mathbf{C}$ when queried with $t_j$ (as defined in Eq.~\ref{eq:triple_passage_index_retrieve}).

\section{Compatibility with Open-weight Models}
\label{appendix_sec:open_source_model_experiments}

\paragraph{\gear Results}
As shown in Table \ref{tab:open_source_recall}, we evaluate \gear using popular 7-8B parameter open-weight models, comparing them against a closed-source alternative. On HotpotQA, Llama-3.1-7B surpasses the closed-source alternative, achieving higher recall rates at R@10 and R@15. For MuSiQue and 2Wiki, while the closed-source model maintains a slight superior edge in performance, the margin is narrow. Importantly, all tested open-weight models consistently outperform the previous state-of-the-art, HippoRAG w$/$IRCoT. This decouples \gear from the need to use closed-source models, suggesting that state-of-the-art multi-step retrieval can be achieved using more accessible models.

\begin{table*}[tbhp]
\small
\centering
\resizebox{\textwidth}{!}{%
\begin{tabular}{l L{2cm} ccc ccc ccc}
\toprule
& \multirow{2.5}{*}{\textbf{LLM}} & \multicolumn{3}{c}{\textbf{MuSiQue}} & \multicolumn{3}{c}{\textbf{2Wiki}} & \multicolumn{3}{c}{\textbf{HotpotQA}} \\ 
\cmidrule{3-11}
& & R@5 & R@10 & R@15 & R@5 & R@10 & R@15 & R@5 & R@10 & R@15 \\ 
\midrule
\multirow{1}{*}{\textbf{Closed-source}} 
& GPT-4o mini & $\mathbf{58.4}$ & $\mathbf{67.6}$ & $\mathbf{71.5}$ & $\mathbf{89.1}$ & $\mathbf{95.3}$ & $\mathbf{95.9}$ & $\mathbf{93.4}$ & $96.8$ & $97.3$ \\ 
\midrule
\multirow{2}{*}{\textbf{Open-weight}}
& Llama-3.1-7B & $52.4$ & $62.3$ & $66.7$ & $81.6$ & $91.0$ & $93.7$ & $92.2$ & $\textbf{97.4}$ & $\textbf{98.1}$ \\ 
& Qwen-2.5-8B & $53.7$ & $63.7$ & $66.7$ & $85.9$ & $91.6$ & $93.0$ & $91.7$ & $96.2$ & $96.9$ \\ 
\bottomrule
\end{tabular}
}
\caption{Retrieval performance of \gear across different closed-source and open-weight models on MuSiQue, 2Wiki and HotpotQA. Results are reported using Recall@$k$ (R@$k$) metrics for $k \in \left \{5, 10, 15 \right\}$, showing the percentage of questions for which the correct entries are found within the top-$k$ retrieved passages. The included open-weight models are Llama-3.1-8B-Instruct and Qwen-2.5-7B-Instruct, and the closed-source model is GPT-4o mini.}
\label{tab:open_source_recall}
\end{table*}


\begin{table*}[thbp]
\small
\centering
\resizebox{\textwidth}{!}{%
\begin{tabular}{l l ccc ccc ccc}
\toprule
& & \multicolumn{3}{c}{\textbf{MuSiQue}} & \multicolumn{3}{c}{\textbf{2Wiki}} & \multicolumn{3}{c}{\textbf{HotpotQA}} \\ 
\cmidrule(lr){3-5} \cmidrule(lr){6-8} \cmidrule(lr){9-11}
& & R@5 & R@10 & R@15 & R@5 & R@10 & R@15 & R@5 & R@10 & R@15 \\ 
\midrule
\multirow{2}{*}{GPT-4o mini}
& w$/$ diversity & \textbf{48.7} & \textbf{57.7} & \textbf{61.2} & \textbf{72.6} & \textbf{80.9} & \textbf{82.4} & \textbf{87.4} & \textbf{93.3} & \textbf{95.2} \\ 
& w$/$o diversity & 47.0 & 53.9 & 58.4 & 68.2 & 76.0 & 77.4 & 85.0 & 92.2 & 94.3 \\ 
\midrule
\multirow{2}{*}{Llama-3.1-8B-Instruct}
& w$/$ diversity & $\textbf{46.2}$ & $\textbf{54.3}$ & $\textbf{57.4}$ & $\textbf{69.1}$ & $\textbf{78.1}$ & $\textbf{81.6}$ & $\textbf{87.3}$ & $\textbf{92.8}$ & $\textbf{95.1}$ \\ 
& w$/$o diversity & $44.9$ & $52.7$ & $55.0$ & $66.9$ & $75.9$ & $78.2$ & $85.0$ & $91.7$ & $94.4$ \\ 
\bottomrule
\end{tabular}}
\caption{Retrieval performance of the Hybrid + SyncGE method with different LLMs for the \texttt{read} step (see Eq.~\ref{eq:proximal_read}) w$/$ and w$/$o diversity for triple beam search. Results are reported using Recall@$k$ (R@$k$) metrics for $k \in \left \{5, 10, 15 \right\}$, showing the percentage of questions for which the correct entries are found within the top-$k$ retrieved passages.}
\label{tab:diverse_beam_search_expanded}
\end{table*}


\paragraph{Diverse Beam Search Results}
\label{appendix_sec:diverse_beam_search_results_expanded}Expanding upon Table \ref{tab:diverse_beam_search}, Table \ref{tab:diverse_beam_search_expanded} demonstrates that diverse beam search consistently improves retrieval performance across both closed-source and open-weight models when using our proposed Hybrid + SyncGE setup. This further confirms the broader applicability of this approach.

\section{Correlation between Question Hops and Agent Iterations}
\label{appendix_sec:correlation_between_hops_and_agent_iterations}

\begin{figure*}[thbp]
  \includegraphics[width=\textwidth]{figures/experiments/hop_vs_agent_iteration_correlation.pdf}
  \caption{Analysis of the relationship between the number of hops in questions and the required number of agent iterations on the MuSiQue dataset. For each hop count, we analyse the number of iterations required by \gear to determine question answerability. The maximum iteration limit was set to 4, with ``4+'' indicating cases where the agent could not determine answerability within this limit. The visualization presents two complementary perspectives on the same data: the left panel shows a box plot emphasizing the median and distribution of stopping iterations, while the right panel focuses on the mean number of iterations across different hop counts.}
  \label{fig:hop_vs_agent_iteration_correlation}
\end{figure*}

The left panel in Figure \ref{fig:hop_vs_agent_iteration_correlation} demonstrates that the median stopping iteration remains consistently at 1 across all hop counts. Additionally, the upper quartile shows a clear upward trend as the number of hops increases. This suggests greater variability in processing time for more complex questions. The right panel illustrates two concurrent trends: as the question hop count increases, the number of questions in the dataset decreases, and the mean number of iterations \gear requires to determine question answerability increases. This pattern indicates that higher-hop questions not only appear less frequently but also typically demand more computational effort to process.

\section{Qualitative Analysis}
\label{appendix_sec:qualitative_analysis}

\begin{table*}[!thbp]
\centering
\small
\begin{tabular}{@{}L{3.5cm}L{7.3cm}L{4cm}@{}}
\toprule
\rowcolor{white} % Example coloring for header row
\textbf{Query} & \textbf{\gear} & \textbf{HippoRAG w$/$ IRCoT} \\
\midrule

% First row
\cellcolor{white}\begin{tabular}[l]{@{}l@{}}What team is the player who\\ led the league in sacks in\\ 2017 a member of?\end{tabular} & 
\cellcolor{customGreen}\begin{tabular}[c]{@{}l@{}}("Chandler Jones", "team",\ "Arizona Cardinals")\\ ("Chandler Jones", "led the\ league in sacks in", "2017")\end{tabular} & 
\cellcolor{customRed}To answer the question about the player who led the league in sacks in 2017, I need to identify the player and their team from external sources, as the provided documents do not contain that specific information \\
\midrule

% Second row
\cellcolor{white}\begin{tabular}[l]{@{}l@{}}Who is the CEO of\\ FleetBoston Financial's\\ purchaser?\end{tabular} & 
\cellcolor{customGreen}\begin{tabular}[c]{@{}l@{}}("FleetBoston Financial",\ "purchaser", "Bank of America")\\ ("Bank of America", "CEO",\ "Brian Moynihan")\end{tabular} & 
\cellcolor{customYellow}The purchaser of FleetBoston Financial is Bank of America. To find the CEO of Bank of America, I need to look for information about the current leadership of the bank. Next, I will check for the current CEO of Bank of America to answer the question accurately.\\
\midrule

% Third row
\cellcolor{white}\begin{tabular}[l]{@{}l@{}}What UK label was purchased\\ by the company that, along\\ with the distributor of The\\ Cube, and ABC, is the other\\ major New York-based\\ broadcaster?\end{tabular} & 
\cellcolor{customGreen}\begin{tabular}[c]{@{}l@{}}("CBS", "purchased",\ "Oriole Records")\\ ("CBS", "is a major broadcaster\ based in", "New York")\\ ("NBC", "is a major broadcaster\ based in", "New York")\\ ("The Cube", "was distributed\ by", "NBC")\end{tabular} & 
\cellcolor{customYellow}The original network of Undercovers is NBC, which is one of the major broadcasters based in New York. Now, I need to find out which UK label was bought by NBC ... \\
\midrule

% Fourth row
\cellcolor{white}\begin{tabular}[l]{@{}l@{}}What was the 2018\\ population of the Italian\\ city that's underwater?\end{tabular} & 
\cellcolor{customGreen}\begin{tabular}[c]{@{}l@{}}("Venice", "population in 2018",\ "260,897")\end{tabular} & 
\cellcolor{customRed}The Italian city that is underwater is Krag, British Columbia, which is a ghost town... \\
\bottomrule
\end{tabular}
\caption{Comparison of MuSiQue queries where \gear achieves 100\% recall at R@15 in a single iteration, while HippoRAG w$/$ IRCoT shows lower performance despite using all four available iterations. Cell colors indicate recall performance: \colorbox{customGreen}{green} for 100\% recall, \colorbox{customRed}{red} for 0\% recall, and \colorbox{customYellow}{yellow} for any intermediate value. Cell values in \gear represent the proximal triples stored in the Gist Triple Memory. Cell values in HippoRAG w$/$ IRCoT represent IRCoT's thought process.}
\label{tab:triple_extraction_comparison}
\end{table*}


Table \ref{tab:triple_extraction_comparison} showcases some query instances where \gear achieves perfect recall in a single iteration, while HippoRAG w$/$ IRCoT achieves lower recall and consumes all available iterations. The presented examples illustrate how \gear's Gist Memory $\mathcal{G}^{(n)}$ precisely captures the essential information needed to answer MuSiQue's queries, maintaining the appropriate level of granularity without including superfluous details. In contrast, HippoRAG w/ IRCoT struggles to retrieve crucial information—whether due to limitations in its triple extraction step or retriever functionality—such as the exact population of Venice, which is necessary for accurate responses. Furthermore, the verbose nature of IRCoT's thought process component contrasts with \gear's streamlined approach. The lack of such verbose component in our approach contributes to the fact that \gear requires fewer LLM tokens than the competition, as explained in subsection \ref{subsec:gear_efficient}. 

\section{Increasing Number of Agent Iterations}
\label{appendix_sec:increasing_n_iterations}
\begin{figure*}[thbp]
\centering
\includegraphics[width=\textwidth]{figures/experiments/recall_evolution_across_agent_iterations_20_iters.pdf}
\caption{Evolution of R@15 over 20 iterations on MuSiQue. Recall is computed at each iteration using the cumulative set of retrieved documents, with prior recall values carried forward for questions that terminated in earlier iterations. The horizontal line indicates the single-step performance of Hybrid + SyncGE.}
\label{fig:recall_across_iterations_20_iters}
\end{figure*}

Figure \ref{fig:recall_across_iterations_20_iters} expands upon the analysis shown in Figure \ref{fig:recall_across_iterations} by evaluating retrieval performance over 20 iterations, rather than the initial 4 iterations. The results demonstrate a consistent pattern across all methods: retrieval performance stabilises after approximately 4 iterations, with no substantial improvements or degradation in subsequent iterations. While some minor fluctuations occur beyond this point, they are negligible.

This performance plateau can be attributed to two key factors. First, the query re-writing mechanisms in all investigated approaches struggle to generate effective subsequent queries. Second, our analysis has identified several cases of unanswerable queries within MuSiQue's answerable subset. A representative example is provided in Table~\ref{tab:musique_problematic_example}.


\begin{table*}
\centering
\footnotesize
\begin{tabular}{L{3cm}L{12cm}}

\toprule
\textbf{Question} & Who did the \textcolor{red}{producer} of \textcolor{purple}{Big Jim McLain} play in \textcolor{purple}{True Grit}? \\
\midrule
\multirow{3.5}{*}{\textbf{Gold Passages}} & {1. \textcolor{purple}{Big Jim McLain}: \textcolor{purple}{Big Jim McLain} is a 1952 political thriller film starring John Wayne and James Arness as HUAC investigators.}\\\cmidrule{2-2}
& {2. \textcolor{purple}{True Grit} is a 1969 American western film. It is the first film adaptation of Charles Portis' 1968 novel of the same name. The screenplay was written by Marguerite Roberts. The film was directed by Henry Hathaway and starred Kim Darby as Mattie Ross and John Wayne as U.S. Marshal Rooster Cogburn. Wayne won his only Academy Award for his performance in this film and reprised his role for the 1975 sequel Rooster Cogburn.}\\
\midrule
\textbf{Comment} & \textcolor{red}{No information about who was the producer of} \textcolor{purple}{Big Jim McLain} \textcolor{red}{is provided in the gold passages}\\
\bottomrule
\end{tabular}
\caption{\label{tab:musique_problematic_example}Example of a query from MuSiQue that in not answerable solely based on the provided gold passages.}
\end{table*}

\section{Comparison of Triple Extraction Prompting Strategies}
\label{appendix_sec:hipporag_results_original_prompt}

\begin{table*}[thbp]
\small
\centering
\resizebox{\textwidth}{!}{%
\begin{tabular}{l l ccc ccc ccc}
\toprule
& & \multicolumn{3}{c}{\textbf{MuSiQue}} & \multicolumn{3}{c}{\textbf{2Wiki}} & \multicolumn{3}{c}{\textbf{HotpotQA}} \\ 
\cmidrule(lr){3-5} \cmidrule(lr){6-8} \cmidrule(lr){9-11}
& & R@5 & R@10 & R@15 & R@5 & R@10 & R@15 & R@5 & R@10 & R@15 \\ 
\midrule
\multirow{2}{*}{\parbox[t]{4cm}{\textbf{HippoRAG}}}
& original prompt & \textbf{41.9} & 46.9 & 51.1 & \textbf{75.4} & \textbf{83.5} & \textbf{86.9} & 79.7 & 88.4 & 91.4 \\ 
& our prompt & 41.0 & \textbf{47.0} & \textbf{51.4} & 75.1 & 83.2 & 86.4 & \textbf{79.8} & \textbf{89.0} & \textbf{92.4} \\ 
\midrule
\multirow{2}{*}{\parbox[t]{4cm}{\textbf{HippoRAG w/ IRCoT}}}
& original prompt & \textbf{49.9} & \textbf{56.4} & \textbf{59.3} & 81.5 & 90.2 & 92.3 & \textbf{90.2} & \textbf{94.7} & 95.8 \\ 
& our prompt & 48.8 & 54.5 & 58.9 & \textbf{82.9} & \textbf{90.6} & \textbf{93.0} & 90.1 & \textbf{94.7} & \textbf{95.9} \\ 
\bottomrule
\end{tabular}}
\caption{Retrieval performance comparison between HippoRAG's sequential triple extraction method and our joint extraction approach across three datasets.}
\label{tab:hippo_prompt_vs_our_prompt}
\end{table*}

HippoRAG employs a sequential approach to triple extraction: it first identifies named entities from a text chunk, and then uses these entities to guide triple extraction in a second step. In contrast, our method extracts both entities and triples simultaneously. Table \ref{tab:hippo_prompt_vs_our_prompt} shows that both approaches achieve comparable retrieval performance across all datasets, with each method excelling in different scenarios. These results validate that joint entity and triple extraction can match the effectiveness of sequential extraction while reducing the number of required processing steps.

\clearpage
\onecolumn  % Switch to one column before strip environment
\section{Prompts}
\label{appendix_sec:agent_prompts}

\subsection{Offline Prompts}
\label{sec:offline_prompts}
\begin{prompt}[title={Reader}]
 \textbf{\# Instruction} \\
\\
Your task is to construct an RDF (Resource Description Framework) graph from the given passages and named entity lists. \\
Respond with a JSON list of triples, with each triple representing a relationship in the RDF graph. \\
Pay attention to the following requirements: \\
- Each triple should contain at least one, but preferably two, of the named entities in the list for each passage. \\
- Clearly resolve pronouns to their specific names to maintain clarity. \\
\\
Convert the paragraph into a JSON dict containing a named entity list and a triple list. \\
\\
\ \textbf{\# Demonstration \#1} \\
\\
Paragraph: \\
``` \\
Magic Johnson \\
\\
After winning a national championship with Michigan State in 1979, Johnson was selected first overall in the 1979 NBA draft by the Lakers, leading the team to five NBA championships during their "Showtime" era. \\
``` \\
\{\{"named\_entities": ["Michigan State", "national championship", "1979", "Magic Johnson", \\ "National Basketball Association", "Los Angeles Lakers", "NBA Championship"]\}\} \\
\{\{ \\
    "triples": [ \\
        ("Magic Johnson", "member of sports team", "Michigan State"), \\
        ("Michigan State", "award", "national championship"), \\
        ("Michigan State", "award date", "1979"), \\
        ("Magic Johnson", "draft pick number", "1"), \\
        ("Magic Johnson", "drafted in", "1979"), \\
        ("Magic Johnson", "drafted by", "Los Angeles Lakers"), \\
        ("Magic Johnson", "member of sports team", "Los Angeles Lakers"), \\
        ("Magic Johnson", "league", "National Basketball Association"), \\
        ("Los Angeles Lakers", "league", "National Basketball Association"), \\
        ("Los Angeles Lakers", "award received", "NBA Championship"), \\
    ] \\
\}\} \\
``` \\
\\
\ \textbf{\# Demonstration \#2} \\
\\
Paragraph: \\
``` \\
Elden Ring \\
\\
Elden Ring is a 2022 action role-playing game developed by FromSoftware. It was directed by Hidetaka Miyazaki with worldbuilding provided by American fantasy writer George R. R. Martin. \\
``` \\
\{\{"named\_entities": ["Elden Ring", "2022", "Role-playing video game", "FromSoftware", "Hidetaka Miyazaki", "United States of America", "fantasy", "George R. R. Martin"]\}\} \\
\{\{ \\
    "triples": [ \\
        ("Elden Ring", "publication", "2022"), \\
        ("Elden Ring", "genre", "action role-playing game"), \\
        ("Elden Ring", "publisher", "FromSoftware"), \\
        ("Elden Ring", "director", "Hidetaka Miyazaki"), \\
        ("Elden Ring", "screenwriter", "George R. R. Martin"), \\
        ("George R. R. Martin", "country of citizenship", "United States of America"), \\
        ("George R. R. Martin", "genre", "fantasy"), \\
    ] \\
\}\} \\
\\
\\
\ \textbf{\# Input} \\
\\
Convert the paragraph into a JSON dict, it has a named entity list and a triple list. \\
\\
Paragraph: \\
``` \\
\textbf{$\{$wiki\_title$\}$} \\
\\
\textbf{$\{$passage$\}$}\\
\end{prompt}

\subsection{Online Retrieval Prompts}
\label{subsec:online_retrieval_prompts}

The \textcolor{blue}{blue}-highlighted portions of the Reader prompt below indicate additional text that is only required when the Gist Memory $\mathcal{G}^{(n)}$ is active. When Gist Memory is inactive, these blue sections should be omitted, and the $\{$triples$\}$ parameter should be left empty.

\begin{prompt}[title={Reader with and without Gist Memory }]
Your task is to find facts that help answer an input question. \\
\\
You should present these facts as knowlege triples, which are structured as ("subject", "predicate", "object"). \\
Example: \\
Question: When was Neville A. Stanton's employer founded? \\
Facts: ("Neville A. Stanton", "employer", "University of Southampton"), ("University of Southampton", "founded in", "1862") \\
\\
\\
Now you are given some documents:\\
\textbf{$\{$docs$\}$} \\
\\
\\
Based on these documents \textcolor{blue}{and some preliminary facts provided below}, \\ find additional supporting fact(s) that may help answer the following question. \\
 \\
Note: if the information you are given is insufficient, output only the relevant facts you can find.\\
\\
Question: \textbf{$\{$query$\}$} \\
Facts: \textcolor{blue}{\textbf{$\{$triples$\}$}} \\
\end{prompt}

\begin{prompt}[title={Reasoning for Termination}]
\ \textbf{\# Task Description:} \\
You are given an input question and a set of known facts:\\
Question: \textbf{$\{$query$\}$} \\
Facts: \textbf{$\{$triples$\}$} \\
\\
Your reply must follow the required format:\\
1. If the provided facts contain the answer to the question, your should reply as follows:\\
Answerable: Yes\\
Answer: ...\\
\\
2. If not, you should explain why and reply as follows:\\
Answerable: No\\
Why: ...\\
\\
\ \textbf{\# Your reply:} \\
\end{prompt}


\begin{prompt}[title={Query Re-writing}]
\ \textbf{\# Task Description:} \\
You will be presented with an input question and a set of known facts. \\
These facts might be insufficient for answering the question for some reason. \\
Your task is to analyze the question given the provided facts and 
determine what else information is needed for the next step. \\
\\
\ \textbf{\# Example:} \\
Question: What region of the state where Guy Shepherdson was born, contains SMA Negeri 68?\\
Facts: ("Guy Shepherdson", "born in", "Jakarta")\\
Reason: The provided facts only indicate that Guy Shepherdson was born in Jakarta, but they do not provide any information about the region of the state that contains SMA Negeri 68. \\
Next Question: What region of Jakarta contains SMA Negeri 68? \\
\\
\ \textbf{\# Your Task:} \\
Question: \textbf{$\{$query$\}$} \\
Facts: \textbf{$\{$triples$\}$} \\
Reason: \textbf{$\{$reason$\}$} \\
\\
Next Question:
\end{prompt}

\subsection{Online Question Answering Prompts}

The following prompt with retrieved passages combines the QA generation prompts from \citeauthor{Gutierrez2024} and \citeauthor{Wang2024}. For the variation without retrieved passages, we omit the preamble and only include the instruction, highlighted in \textcolor{purple}{purple} .

\begin{prompt}[title={Retrieved Passages with In-context Example}]
As an advanced reading comprehension assistant, your task is to analyze text passages and corresponding questions meticulously, with the aim of providing the correct answer. \\
==================\\
For example:\\
==================\\
Wikipedia Title: Edward L. Cahn \\
Edward L. Cahn (February 12, 1899 – August 25, 1963) was an American film director.\\
\\
Wikipedia Title: Laughter in Hell \\
Laughter in Hell is a 1933 American Pre-Code drama film directed by Edward L. Cahn and starring Pat O'Brien. The film's title was typical of the sensationalistic titles of many Pre-Code films. Adapted from the 1932 novel of the same name buy Jim Tully, the film was inspired in part by "I Am a Fugitive from a Chain Gang" and was part of a series of films depicting men in chain gangs following the success of that film. O'Brien plays a railroad engineer who kills his wife and her lover in a jealous rage and is sent to prison. The movie received a mixed review in "The New York Times" upon its release. Although long considered lost, the film was recently preserved and was screened at the American Cinematheque in Hollywood, CA in October 2012. The dead man's brother ends up being the warden of the prison and subjects O'Brien's character to significant abuse. O'Brien and several other characters revolt, killing the warden and escaping from the prison. The film drew controversy for its lynching scene where several black men were hanged. Contrary to reports, only blacks were hung in this scene, though the actual executions occurred off-camera (we see instead reaction shots of the guards and other prisoners). The "New Age" (an African American weekly newspaper) film critic praised the scene for being courageous enough to depict the atrocities that were occurring in some southern states. \\
\\
Wikipedia Title: Theodred II (Bishop of Elmham) \\
Theodred II was a medieval Bishop of Elmham. The date of Theodred's consecration unknown, but the date of his death was sometime between 995 and 997. \\
\\
Wikipedia Title: Etan Boritzer \\
Etan Boritzer( born 1950) is an American writer of children 's literature who is best known for his book" What is God?" first published in 1989. His best selling" What is?" illustrated children's book series on character education and difficult subjects for children is a popular teaching guide for parents, teachers and child- life professionals. Boritzer gained national critical acclaim after" What is God?" was published in 1989 although the book has caused controversy from religious fundamentalists for its universalist views. The other current books in the" What is?" series include What is Love?, What is Death?, What is Beautiful?, What is Funny?, What is Right?, What is Peace?, What is Money?, What is Dreaming?, What is a Friend?, What is True?, What is a Family?, What is a Feeling?" The series is now also translated into 15 languages. Boritzer was first published in 1963 at the age of 13 when he wrote an essay in his English class at Wade Junior High School in the Bronx, New York on the assassination of John F. Kennedy. His essay was included in a special anthology by New York City public school children compiled and published by the New York City Department of Education. \\
\\
Wikipedia Title: Peter Levin \\
Peter Levin is an American director of film, television and theatre. \\
\\
Question: When did the director of film Laughter In Hell die? \\
Answer: August 25, 1963. \\
================== \\
\textcolor{purple}{Given the following text passages and questions, please present a concise, definitive answer, devoid of additional elaborations, and of maximum length of 6 words.} \\
================== \\
\\
Wikipedia Title : \textbf{$\{$title$\}$}
\textbf{$\{$text$\}$} \texttt{for each retrieved passage} ...  \\
Question: \textbf{$\{$question$\}$} \\
\\
Answer:
\end{prompt}

\label{subsec:online_qa_prompts}
\begin{prompt}[title={No Retrieved Passages}]
\textcolor{purple}{Given the following question, please present a concise, definitive answer, devoid of additional elaborations, and of maximum length of 6 words.} \\
\\
Question: \textbf{$\{$question$\}$} \\
\\
Answer:
\end{prompt}


\end{document}
