\documentclass[letterpaper,twocolumn,10pt]{article}
\usepackage{usenix}
\usepackage{etoolbox}
\usepackage{xurlshawn}
% \usepackage[hyphenbreaks]{breakurl}

\makeatletter
\patchcmd{\maketitle}
  {\@maketitle}
  {\vspace{-0.3in}\@maketitle\vspace{-0.4in}}% change the value as needed
  {}
  {}
\makeatother

\usepackage{balance}
\usepackage{url}
\usepackage{color}
\usepackage{caption}
\usepackage{diagbox}
\usepackage{subfigure}
\usepackage{mathptmx}
\usepackage{algorithm,algpseudocode}
\usepackage{amsmath,amssymb}

\usepackage{multirow}
\usepackage{booktabs}
\usepackage{epsfig,endnotes}
\usepackage{enumitem}
\usepackage[font=footnotesize,labelfont=bf, labelsep=period]{caption}
\interfootnotelinepenalty=1000
\newcommand{\para}[1]{{\vspace{2pt} \noindent \textbf{#1}
    \hspace{6pt}}}

\newcommand{\subpara}[1]{{\vspace{1.0pt} \textbf{#1}
    \hspace{4pt}}}

\newcommand{\fixme}[1]{{\color{red} #1}}
\newcommand{\todo}[1]{{\color{red}TODO:  #1}}

\newcommand{\askben}[1]{{\color{red} Q:  #1}}
\newcommand{\emily}[1]{{\color{teal} #1}}

\newcommand{\shawn}[1]{{\color{black} #1}}
\newcommand{\jenna}[1]{{\color{purple} #1}}
\newcommand{\htedit}[1]{{\color{black} #1}}

\newcommand{\outline}[1]{{\color{blue} #1}}
\newcommand{\revise}[1]{{\color{black} #1}}
\newcommand{\ol}[1]{{\color{blue} #1}}
\newcommand{\logic}[1]{{\color{brown}LOGIC: #1}}

\newcommand{\etal}{{\em et al.\ }}
\newcommand{\eg}{{\em e.g.,\ }}
\newcommand{\ie}{{\em i.e.,\ }}

\newcommand{\secspace}{\vspace{-0.1in}}
\newcommand{\secspacesm}{\vspace{0.0in}}

\newcommand{\system}{{\em Glaze}}
\newcommand{\dalle}{DALL$\cdot$E-2}
\newcommand{\dalleM}{DALL$\cdot$E-m}
\newcommand\inappendix{\textcolor{blue}{\textbf{in Appendix}}}


\newenvironment{packed_itemize}{
\begin{list}{\labelitemi}{\leftmargin=0.5em}
  \setlength{\itemsep}{1pt}
  \setlength{\parskip}{0pt}
  \setlength{\parsep}{0pt}
  \setlength{\headsep}{0pt}
  \setlength{\topskip}{0pt}
  \setlength{\topmargin}{0pt}
  \setlength{\topsep}{0pt}
  \setlength{\partopsep}{0pt}
}{\end{list}}

\newenvironment{packed_enumerate}{
\begin{enumerate}
 \setlength{\itemsep}{1pt}
 \setlength{\parskip}{0pt}
 \setlength{\parsep}{0pt}
 \setlength{\headsep}{0pt}
 \setlength{\topskip}{0pt}
 \setlength{\topmargin}{0pt}
 \setlength{\topsep}{0pt}
 \setlength{\partopsep}{0pt}
}{\end{enumerate}}

\pagenumbering{gobble}

\begin{document}


\title{Glaze: Protecting Artists from Style Mimicry by Text-to-Image Models}

\author{Shawn Shan, Jenna Cryan, Emily Wenger, Haitao Zheng, Rana Hanocka, Ben Y. Zhao\\
	{\em Department of Computer Science, University of Chicago}\\
	{\em \{shawnshan, jennacryan, ewillson, htzheng, ranahanocka, ravenben\}@cs.uchicago.edu}}
\maketitle


\begin{abstract}
  Recent text-to-image diffusion models such as MidJourney and Stable
  Diffusion threaten to displace many in the professional artist
  community. In particular, models can learn to mimic the artistic style of
  specific artists after ``fine-tuning'' on samples of their art. In this
  paper, we describe the design, implementation and evaluation of \system{},
  a tool that enables artists to apply ``style cloaks'' to their art before
  sharing online. These cloaks apply barely perceptible perturbations to
  images, and when used as training data, mislead generative models that try
  to mimic a specific artist. In coordination with the professional artist
  community, we deploy user studies to more than 1000 artists, assessing
  their views of AI art, as well as the efficacy of our tool, its usability
  and tolerability of perturbations, and robustness across different
  scenarios and against adaptive countermeasures. Both surveyed artists and
  empirical CLIP-based scores show that even at low perturbation levels ($p$=0.05),
  \system{} is highly successful at disrupting mimicry under normal
  conditions (>92\%) and against adaptive countermeasures (>85\%).

\end{abstract}

\section{Introduction}
%
Neural network (NN) learning has underpinned state of the art empirical
results in numerous applied machine learning tasks (see for
instance~\cite{krizhevsky2012imagenet,lecun2015deep}). Nonetheless, neural
network learning remains rather poorly understood in several regards.
Notably, it remains unclear why training algorithms find good weights, how
learning is impacted by the network architecture and activations, what is
the role of random weight initialization, and how to choose a concrete
optimization procedure for a given architecture.

We start by analyzing the expressive power of NNs subsequent to the random
weight initialization. The motivation is the empirical success of training
algorithms despite inherent computational intractability, and the fact that
they optimize highly non-convex objectives with potentially many local minima.
Our key result shows that random initialization already positions learning
algorithms at a good starting point. We define an object termed a {\em
computation skeleton} that describes a distilled structure of feed-forward
networks. A skeleton induces a family of network architectures along with a
hypothesis class $\ch$ of functions obtained by certain non-linear
compositions according to the skeleton's structure.  We show that the
representation generated by random initialization is sufficiently rich to
approximately express the functions in $\ch$. Concretely, all functions in
$\ch$ can be approximated by tuning the weights of the last layer, which is
a convex optimization task.

In addition to explaining in part the success in finding good weights, our
study provides an appealing perspective on neural network learning.  We
establish a tight connection between network architectures and their dual
kernel spaces. This connection generalizes several previous constructions
(see Sec~\ref{sec:related}). As we demonstrate, our dual view gives rise to
design principles for NNs, supporting current practice and suggesting
new ideas. We outline below a few points.

\begin{itemize}

\item Duals of convolutional networks appear a more suitable fit for
	vision and acoustic tasks than those of fully connected networks.

\item Our framework surfaces a principled initialization scheme. It is
	very similar to common practice, but incorporates a small correction.

\item By modifying the activation functions, two consecutive fully connected
	layers can be replaced with one while preserving the network's dual kernel.

\item The ReLU activation, i.e. $x \mapsto \max(x,0)$, possesses favorable
	properties. Its dual kernel is expressive, and it can be well approximated by
	random initialization, even when the initialization's scale is moderately
	changed.

\item As the number of layers in a fully connected network becomes very
	large, its dual kernel converges to a degenerate form for any non-linear
	activation.

\item Our result suggests that optimizing the weights of the last layer can
	serve as a convex proxy for choosing among different architectures prior
	to training. This idea was advocated and tested empirically
	in~\cite{saxe2011random}.

\end{itemize}

\secspace
\section{Background: AI Art and Style Mimicry}
\label{sec:motivation}

In this section, we provide critical context in the form of
basic background on current AI art models and style mimicry.

\secspace
\subsection{Text-to-Image Generation}

Since Text-to-image generation was first proposed in
2015~\cite{mansimov2015generating}, a stream of research has proposed newer
model architectures and training methods enabling generation of
higher-quality
images~\cite{radford2015unsupervised,zhang2017stackgan,xu2018attngan,li2019object,zhu2019dm}.  The high level design of
recent models used for AI art
generation~\cite{rombach2022high,ramesh2021zero,d-mini} is shown in
Figure~\ref{fig:diffusion-arch}. During training, the model takes in an image
$x$ and uses a feature extractor $\Phi$ to extract its features, producing
$\Phi(x)$. Simultaneously, a conditional image generator $G$ takes in a
corresponding text caption ($s$) and outputs a predicted feature vector
$G(s)$. Then the parameters of $G$ are optimized so the text feature vector
$G(s)$ matches the image feature vector $\Phi(x)$. At generation time, a user
gives $G$ a text prompt $s_0$, and $G$ outputs an image feature vector
$G(s_0)$. A decoder $D$ then decodes $G(s_0)$ to produce the final generated
image. %Below, we describe $G$, $\Phi$, and $D$ in more detail.

Compared to earlier models based on generative adversarial networks
(GANs) or variational autoencoders
(VAE)~\cite{radford2015unsupervised,zhu2019dm,tao2022df}, more recent
models~\cite{rombach2022high,ramesh2022hierarchical}
leveraging \textit{diffusion} models
produce significantly higher quality images. Feature extractor ($\Phi$) is
used to reduce the dimensionality of the input image to facilitate the
generation process. The extractor $\Phi$ and decoder $D$ are often a pair of
variational autoencoder (VAE)~\cite{rombach2022high,ramesh2021zero}, \ie
extractor (encoder) extracts image features and decoder map features back to
images.

\para{Training Data Sources. } The training datasets of these models
typically contain image/ALT text pairs scraped from the Internet. They
are extremely large, e.g. LAION~\cite{schuhmann2022laion} contains 5 billion
images collected from 3 billion webpages.  

These datasets are subject to minimal curation and governance. Data
collectors typically only filter out data with extremely short or incorrect text captions
(based on an automated text/image alignment metric~\cite{schuhmann2022laion}). % Some
Since copyrighted images are not filtered~\cite{schuhmann2022laion}, these
datasets are rife with private, sensitive content, including copyrighted
artworks.

\secspace
\subsection{Style Mimicry}
\label{sec:mimicry-back}

\begin{figure}[t]
  \centering
  \includegraphics[width=0.95\columnwidth]{plots/overview/hollie-mimic.pdf}
  \vspace{-0.1in}
  \caption{Real-world incident of AI plagiarizing the style of artist Hollie Mengert~\cite{hollie-steal}. {\bf Left}: original artwork by Hollie Mengert. {\bf Right}: plagiarized artwork generated by a model trained to mimic Hollie's style. }
  \label{fig:hollie-mimic}
\end{figure}

In a {\em style mimicry} attack, a bad actor uses an AI art model to create
art in a particular artist's style without their consent. % Style mimicry can
More than 67\% of 
art pieces showcased on a popular AI-art-sharing website leverage style
mimicry~\cite{mid-top-artistname}.

\para{Style mimicry techniques.} Today, a ``mimic'' can easily copy the style of
a victim artist with only an open-source text-to-image model and a few
samples of artwork from the artist. A naive mimicry attack directly queries a
generic text-to-image model using the name of the victim artist. For example,
the prompt ``a painting in the style of Greg Rutkowski'' would cause the
model to generate images in the style of Polish artist Greg Rutkowski. This
is because many of Rutkowski's artworks appear in training datasets of these
generic models labeled with his name.

Naive mimicry can succeed when the artist is well-known and
has a significant amount of art online, but fail on other artists. In more recent
mimicry attacks, a mimic {\em fine-tunes} a
generic text-to-image model on samples of a target artist's work
(as few as $20$ unique pieces) downloaded from online sources. This calibrates the model to
the victim artist's style, identifying important features related to the
victim style and associating these regions in the feature space with the
victim artist's name~\cite{ruiz2022dreambooth,gal2022image}. This enables
style mimicry with impressive accuracy.  The entire fine-tuning process takes
less than 20 minutes on a low-end consumer GPU\footnote{It takes an average
  of 18.3 minutes on a GTX 1080 GPU}.

\para{Real-work mimicry incidents. }
The first well-known incident of mimicry was when a Reddit user stole
American artist Hollie Mengert's style and open-sourced the style-specific
model on Reddit~\cite{hollie-steal}. Figure~\ref{fig:hollie-mimic} has a
side-by-side comparison of Hollie's original artwork and plagiarized artwork
generated via style mimicry. Later, famous cartoonist Sarah Andersen reported
that AI art models can mimic her cartoon drawings~\cite{sarah-andersen}, and
other similar incidents abound~\cite{lensa-steal,sam-steal}.

Several companies~\cite{aigame} have even hosted style mimicry
as a service, allowing users to upload a few art pieces painted by victim
artists and producing new art in the victim styles. CivitAI~\cite{civitai}
built a large online marketplace where people share their customized stable
diffusion models, fine-tuned on certain artwork.  

\begin{figure}[t]
  \centering
  \includegraphics[width=1\columnwidth]{plots/overview/diffusion-arch.pdf}
  \vspace{-0.25in}
  \caption{High level model architecture of text-to-image models. }
  \label{fig:diffusion-arch}
\end{figure}

\secspace
\section{Collaborating with Artists}
\label{sec:artists}
Next, we explain our collaborative relationship with professional artists,
and its significant impact on our key evaluation metrics in this paper. We
also summarize key results from our first user study on views of AI art and
mimicry by members of the artist community.


Artists have spoken out against style mimicry in numerous venues, focusing
particularly on how it violates their intellectual property rights and
threatens their
livelihoods~\cite{guardian-artical,artical-1,artical-2, noai-protest}. Others
have taken direct action. The Concept Art Association raised over \$200K to
fight AI art, and filed a class action lawsuit in the US
against AI art companies~\cite{class-action}. In November 2022, artists
organized a large protest against ArtStation~\cite{noai-protest}, the large
digital art sharing platform that allowed users to post AI artwork without
identification. Anti-AI images flooded the site for several weeks, until
ArtStation banned the protest images~\cite{noai-result}. \revise{Recently, the
Writers Guild of America (WGA) went on strike demanding contractual changes
to ban generative AI~\cite{AIstrike}.}

Members of the professional art community reached out to us in Sept 2022. We
joined online town halls and meetings alongside hundreds of professionals,
including Emmy winners and artists at major film studios. After learning
more, we began an active collaboration with multiple professional artists,
including award-winning artist Karla Ortiz, who leads efforts defending
artists and is lead plaintiff in the class action suit.
The artists helped this project in multiple ways, by 1) sharing experiences
about specific ways AI-art has impacted them and their colleagues; 2) sharing
domain knowledge about what is acceptable to artists in terms of
perturbations on their art; and 3) helping to widely disseminate our user study to
members of their professional organizations, including the Concept Art
Association and the Animation Guild (TAG839).

\para{Evaluation via Direct Feedback from Artists.} Our goal is to help artists
disrupt AI models trying to mimic their artistic style, without adversely
impacting their own artwork. Because ``success'' in this context is highly
subjective (``Did this AI-art successfully mimic Karla's painting style?''),
we believe the only reliable evaluation metric is direct feedback by
professional artists themselves. Therefore, wherever possible, the evaluation
of \system{} is done via detailed user studies engaging members of the
professional artist community, augmented by an empirical score we develop based on
genre prediction using CLIP models.

We deployed two user studies during the course of this project (see
Table~\ref{tab:study-details}). Both are IRB-approved by our institution.  Both
draw participants from professional artists informed via their social circles
and professional networks. The first (Survey 1, \S\ref{sec:user-study},
\S\ref{sec:cloaking-results}), asked participants
about their broad views of AI style mimicry, and then presented them with a
number of inputs and outputs of our tool, and asked them to give ratings
corresponding to key metrics we wanted to evaluate. We select a subset 
of participants from the first study to participate in a 
longer and more in-depth study (Survey 2) where 
they were asked to evaluate the performance of \system{} in 
additional settings (\S\ref{sec:cloaking-results}, \S\ref{sec:robust-eval}, 
\S\ref{sec:counter}, and Appendix~\ref{sec:appendix}). 


\begin{table}[t]
  \centering
  \resizebox{0.49\textwidth}{!}{
  \centering
    \begin{tabular}{ccl}
    \hline
    \textbf{Survey} & \textbf{\begin{tabular}[c]{@{}c@{}} \# of artists\end{tabular}} & \multicolumn{1}{c}{\textbf{Content}} \\ \hline
    Survey 1 & 1156 & \begin{tabular}[c]{@{}l@{}} 1) Broad views of AI art
                        and style mimicry(\S\ref{sec:user-study}) \\
                        2) Glaze's usability, i.e. acceptable levels of cloaking (\S\ref{sec:cloaking-results}) \\
                        3) Glaze performance in disrupting style mimicry (\S\ref{sec:cloaking-results}) \end{tabular} \\ \hline
    
\begin{tabular}[c]{@{}c@{}}
    Survey 2\\(Extension to Survey 1)
    \end{tabular}
    & 151 & \begin{tabular}[c]{@{}l@{}}
    1) Additional performance tests (\S\ref{sec:cloaking-results}) \\
    2) Robustness to advanced scenarios (\S\ref{sec:robust-eval}) \\ and countermeasures (\S\ref{sec:counter}) \\ 
    3) Additional system evaluation (Appendix~\ref{sec:appendix}) \end{tabular} \\ \hline
    \end{tabular}
  }
  \vspace{-0.1in}
  \caption{Information on our user studies: the number of artist participants
    and where we report the results of the studies. We sent Survey 2 to
    some specific participants from survey 1 who volunteered to participate in a
    followup study.}
  \label{tab:study-details}
\end{table}


\secspace
\subsection{Artists' Opinions on Style Mimicry}
\label{sec:user-study}

While we expected artists to view style mimicry negatively, we wanted to
better understand how much individual artists understood this topic and how
many perceived it as a threat. Here we describe results from Survey 1 to
gather perceptions of the potential impact of AI art on existing artists.

\para{Survey Design.} Our survey consisted of both multiple choice and free
response questions to understand how well people understand the concept of AI
art, and how well the models successfully imitate the style of artists.
Additionally, we asked artists about the extent to which they anticipate the
emergence of AI art to impact their artistic activities, such as posting
their art online and their job security.  A handful of professional artists
helped disseminate our survey to their respective artist community groups.
Overall, we collected responses from 1,207 participants, consisting primarily
of professional artists (both full-time (46\%) and part-time/freelancer
(50\%)) and some non-artist members of the art community who felt invested in
the impact of AI art (4\%). Of the participants who consider themselves
artists, their experience varied: <1 year (13\%), 1-5 years (49\%), 5-10
years (19\%), 10+ years (19\%).  Participants' primary art style varied
widely, including: animation, concept art, abstract, anime, game art, digital
2D/3D, illustration, character artwork, storyboarding, traditional
painting/drawing, graphic design, and others.

\para{Key Results.} Our study found that 91\% of the artists have read about
AI art extensively, and either know of or worry about their art being used to
train the models. Artists expect AI mimicry to have a significant impact on artist
community: $97\%$ artists state it will decrease some artists' job security;
$88\%$ artists state it will discourage new students from studying art; and
$70\%$ artists state it will diminish creativity. ``Junior positions will
become extinct,'' as stated by one participant.

Many artists (> 89\% artists) have already or plan to take actions because of
AI mimicry. Over $95\%$ of artists post their artwork online. Out of these
artists, $53\%$ of them anticipate reducing or removing their online artwork,
if they haven't already. Out of these artists, $55\%$ of them believe
reducing their online presence will significantly impact their careers. One
participant stated ``AI art has unmotivated myself from uploading more art
and made me think about all the years I spent learning art.'' $78\%$ of
artists anticipate AI mimicry would impact their job security, and this
percentage increases to $94\%$ for the job security of newer
artists. Further, $24\%$ of artists believe AI art has \textit{already}
impacted their job security, and an additional $53\%$ expect to be affected
within the next 3 years. Over $51\%$ of artists expressed interest in
proactive measures, such as personally joining class action lawsuits against
AI companies.  

Professional artists thought AI mimicry was very successful at mimicking the
style of specific artists.  We showed the artists examples of original
artwork from $23$ artists, and the artwork generated by a model
attempting to mimic their styles (detailed mimicry setup in
\S\ref{sec:eval-cloak}).  $77\%$ of artists found the AI model
\textit{successfully} or \textit{very successfully} mimic the styles of
victim artists, with one stating ``it's shocking how well AI can mimic the
original artwork.''  Additionally, $19\%$ of participants thought the AI
mimicry is somewhat successful, leaving only $< 5\%$ of artists rating the
mimicry as unsuccessful.  Several artists also pointed out that, as artists,
upon close inspection they could spot differences between the AI art and
originals, but were skeptical the general public would notice them.
%the same disparities.

A significant concern of most participants, surprisingly, is not just the
existence of AI art, but rather scraping of existing artworks without
permission or compensation.  As one participant stated: ``If artists are paid
to have their pieces be used and asked permission, and if people had to pay
to use that AI software with those pieces in it, I would have no problem.''
However, without consent to use their artwork to train the models, ``it's
incredibly disrespectful to the artist to have their work `eaten' by a
machine [after] many years to grow our skills and develop our styles.''
%\secspace
\section{Background}
\label{sec:back}

Here, we first summarize Glaze and then IMPRESS purification method. 

\para{Glaze protection against style mimicry. } Glaze seeks to protect artist's 
artwork from AI mimicry by adding small 
perturbations on these artwork to confuse diffusion models. 
Given an artwork $x$ and target style $T$ that is different from the artist's, 
Glaze first uses a pretrained style transfer model $\Omega$ to compute a style 
transferred version of the artwork. We denote such image as $\Omega(x, T)$. Then, Glaze 
computes a cloak $\delta_x$ that optimize the latent representation of Glazed artwork
($x + \delta_x$) to be similar to the style transferred artwork ($\Omega(x, T)$). 
The Glaze optimization effectively moves the original image to a new position in 
the high dimensional latent space, causing model to learn an incorrect art style. 
Glaze calculates the latent space using the 
feature extractor ($\mathcal{E}$) from a diffusion model.
Formally, we write the Glaze 
optimization as solving the following:
\begin{align}
    \min_{\delta_x} ||\mathcal{E}(x + \delta_x) - \mathcal{E}(\Omega(x, T))||_2 \\
    \text{subject to } \text{LPIPS}(x + \delta_x, x) < p_{G} \notag
\end{align}
\noindent We use LPIPS, a popular human-perceived visual distortion metric~\cite{zhang2018unreasonable}, to bound the perturbation 
within a budget $p_{G}$. 

\para{IMPRESS Purification Method. } IMPRESS adds additional perturbation on top of a Glazed artwork 
hoping to ``purify'' the Glaze
effect -- recovering the precise latent representation of original artwork. 
First, the authors empirically find that when passing Glazed images through an image autoencoder, 
the output image looks more different from the input image, compared to the output 
when inputting a clean image to the same autoencoder. Then authors assume removing this particular 
discrepancy would guide them to find the original (non-Glazed) image. 

IMPRESS purification  
optimizes perturbations on Glaze images such that purified images
behave similarly to clean images
when passing through an autoencoder. The authors assume the optimization process will guide 
the image to move back to the original latent space of the non-Glazed image. 
Formally, IMPRESS purification optimize a perturbation $\delta_{pur}$ on 
a Glaze image $x_{glazed}$:

\begin{align}
    \min_{\delta_{pur}} ||(x_{glazed} + \delta_{pur}) - \text{VAE}(x_{glazed} + \delta_{pur}) ||_2^2 \\
    \text{subject to } \text{LPIPS}(x_{glazed} + \delta_{pur}, x_{glazed}) < p_{I} \notag
\end{align}

\noindent $\text{VAE}$ is an image autoencoder, which consists of an encoder $\mathcal{E}$ followed by a decoder $D$. IMPRESS uses the same
autoencoder as the stable diffusion model. The 
perturbation $\delta_{pur}$ is bounded by a LPIPS 
perturbation budget $p_{I}$. 

\section{Retrieval with Synchronised Graph Expansion}
\label{sec:graph_retrieval}

\def\Tqinit{\mathbf{T}_\mathbf{q}}


\begin{figure}[thbp]
  \includegraphics[width=\columnwidth]{figures/gear-sys-fig.pdf}
  \caption{\label{fig:system_diagram}System Architecture}
\end{figure}

% Start: Zhili --------------------------


Given an input query $\mathbf{q}$, let $\mathbf{C}_\mathbf{q}' = h^k_{\text{base}}\left( \mathbf{q}, {\mathbf{C}}\right )$  be a list of passages returned by the base retriever\footnote{The choice of a base retriever within our framework is flexible, without requiring any multi-hop capabilities.}.
Given this initially retrieved list of passages, $\mathbf{C}_\mathbf{q}'$, our goal is to derive relevant multi-hop contexts (passages) by retrieving a sub-graph of triples that interconnect their source passages. There are two challenges for materialising such sub-graph retrieval: \begin{inparaenum}[(i)]\item how to locate initial triples (i.e. starting nodes) $\Tqinit$, and \item how to expand the graph based on initial triples while reducing the search space\end{inparaenum}. The following sections address these challenges respectively, within \gear.



\subsection{Knowledge Synchronisation}
\label{subsection:knowledge_syncro}
\def\linkTriple{\texttt{tripleLink}}

We describe a knowledge \textbf{Sync}hronisation (\textbf{Sync}) process for locating initial nodes for graph expansion. We first employ an LLM to \texttt{read} $\mathbf{C}_\mathbf{q}'$ (see Appendix~\ref{subsec:online_retrieval_prompts}) and summarise knowledge triples that can support answering the current query $\mathbf{q}$, as defined:
\begin{align}
    \mathbf{T}_\mathbf{q}' = \texttt{read}\left (\mathbf{C}_\mathbf{q}', \mathbf{q}\right ).
    \label{eq:proximal_read}
\end{align}
$\mathbf{T}_\mathbf{q}'$ is a collection of triples to which we refer as \textit{proximal triples}. Initial nodes $\Tqinit$ for graph expansion can then be identified by linking each triple in $\mathbf{T}_\mathbf{q}'$ to a triple in $\mathbf{T}$, using the \linkTriple{} function:
\begin{align}
    \Tqinit =\left \{t_i | t_i = \linkTriple(t_i') ~ \forall t_i' \in \mathbf{T}_\mathbf{q}'\right \}.
\end{align}
The implementation of \linkTriple{} can vary. However, in this paper we consider it to be simply retrieving the most similar triple from $\mathbf{T}$.



\begin{algorithm}[ht]
\textbf{Input:} $\mathbf{q}$: query \\
\hspace*{3em} $b$: beam size \\
\hspace*{3em} $l$: maximum length \\
\hspace*{3em} $\mathrm{score}(\cdot, \cdot)$: scoring function \\
\hspace*{3em} $\{t_1, t_2, \ldots, t_n\}$: initial triples \\
\hspace*{3em} $\gamma$: hyperparameter for diversity


\begin{algorithmic}[1]
\State $B_0 \gets [\;]$
\For{$t \in \{t_1, t_2, ..., t_n\}$}
    \State $s \gets \mathrm{score}(\mathbf{q}, [t])$
    \State $B_0.\mathrm{add}(\langle s, [t] \rangle)$
\EndFor

\State $B_0 \gets \mathrm{top}(B_0, b)$


\For{$i \in \{1, \dots, l - 1\}$}
    \State $B \gets [\;]$
    
    \For{$\langle s, T \rangle \in B_{i-1}$}
        \State $V \gets [\;]$

        \For{$t \in \mathrm{get\_neighbours}(T.\mathrm{last}())$}
            \If{$\mathrm{exists}(t, B_{i-1})$}
                \State \textbf{continue}
            \EndIf
            
            \State $s' \gets s + \mathrm{score}(\mathbf{q}, T \circ t)$ ~ \texttt{\# concat} 
            \State $V.\mathrm{add}(\langle s', T \circ t \rangle)$
        \EndFor

        \State $\mathrm{sort}(V, \mathrm{descending})$

        \For{$n \in \{0, \dots, V.\mathrm{length()} - 1\}$}
            \State $\langle s', T \circ t \rangle \gets V[n]$
            \State $s' \gets s' \times e^{- \frac{\mathrm{min}(n, \gamma)}{\gamma}}$
            \State $B.\mathrm{add}(\langle s', T \circ t \rangle)$
        \EndFor
        
    \EndFor
    \State $B_i \gets \mathrm{top}(B, b)$
    
\EndFor

\State \Return $B_i$
\end{algorithmic}

\caption{Diverse Triple Beam Search}
\label{alg:beam_search}
\end{algorithm}

\subsection{Diverse Triple Beam Search}

We borrow the idea of constructing reasoning triple chains \cite{Fang2024} for expanding the graph, and present a retrieval algorithm: \textit{Diverse Triple Beam Search} (see Alg.~\ref{alg:beam_search}). 

We maintain top-$b$ sequences (beams) of triples and the scores at each step are determined by a scoring function. In this paper, we focus on leveraging a dense embedding model to compute the cosine similarity between embeddings of the query and a candidate sequence of triples, leaving other implementations of the scoring function for future work (see Section~\ref{sec:limitations}).

Considering all possible triple extensions at each step, in a Viterbi decoding fashion, would be intractable due to the size of $\mathbf{T}$. Consequently, we define the neighbourhood of a triple as the set of triples with shared head or tail entities (i.e. $\mathrm{get\_neighbours}$ in Alg.~\ref{alg:beam_search}). During each expansion step, we only consider neighbours of the last triple in the sequence, and avoid selecting previously visited triples (i.e. $\mathrm{exists}$ in Alg.~\ref{alg:beam_search}) to further reduce the search space.

While regular beam search can reduce the search space, it is prone to producing high-likelihood sequences that differ only slightly from one another \cite{Ippolito2019, Vijayakumar2018}. Our algorithm increases the diversity across beams to improve the recall for retrieval. In detail, for each beam, we sort candidate sequences extended from that beam in descending order, and weight their scores based on their relative positions. Candidate sequences that are ranked lower, within a beam, will receive smaller weights. Consequently, the resulting top-$b$ beams at each step are less likely to share the same starting sequence. 

The top-$b$ returned sequences are flattened in a breadth-first order. Each triple in the resulting list is then mapped to its source passage. This alignment between triples and passages is described in more detail in Section~\ref{sec:preliminaries}. Let $\widetilde{\mathbf{C}}_\mathbf{q}$ be the list of unique passages after alignment. The output of our graph expansion is then given by the Reciprocal Rank Fusion (RRF) \cite{Cormack2009} of $\widetilde{\mathbf{C}}_\mathbf{q}$ and the initial $\mathbf{C}_\mathbf{q}'$ list of passages :
\begin{align}
    \mathbf{C}_{\mathbf{q}} = \mathrm{RRF}\left(\widetilde{\mathbf{C}}_\mathbf{q}, \mathbf{C}_\mathbf{q}'\right ).
\end{align}
We refer to this graph-based method of retrieving relevant passages as \textbf{Sync}ronised \textbf{G}raph \textbf{E}xpansion (\textbf{SyncGE}).


\section{Multi-step Extension}


While SyncGE can enhance a base retriever with multi-hop context, some queries inherently require multiple steps to gather all necessary evidence. We materialise \gear by incorporating an agent with multi-turn capabilities, capable of interacting with the graph-retriever described above. We focus on:
\begin{itemize}
\item maintaining a gist memory of proximal knowledge obtained throughout the different steps 
\item incorporating a similar synchronisation process 
that summarises retrieved passages in proximal triples to be stored in this multi-turn gist memory
\item determining if additional steps are needed for answering the original input question
\end{itemize}
%
Within this multi-turn setting, the original input question $\mathbf{q}$ is iteratively decomposed into simpler queries: $\mathbf{q}^{(1)}, \ldots, \mathbf{q}^{(n)}$, where $\mathbf{q}^{(1)} = \mathbf{q}$ and $n \in \mathbb{N}$ represents the number of the current step.
For each query $\mathbf{q}^{(n)}$, we use the graph retrieval method introduced in Section~\ref{sec:graph_retrieval} in order to retrieve relevant passages $\mathbf{C}_{\mathbf{q}^{(n)}}$.



\subsection{Gist Memory Constructor}
To facilitate the multi-step capabilities of our agent, we introduce a \textit{gist memory}, $\mathcal{G}^{(n)}$, which is used for storing knowledge as an array of proximal triples. At the beginning of the first iteration, the gist memory is empty. During the $n$-th iteration, similar to the knowledge synchronisation module explained in Section~\ref{subsection:knowledge_syncro}, we employ an LLM to read a collection of retrieved paragraphs $\mathbf{C}_{\mathbf{q}^{(n)}}$ and summarise their content with proximal triples:

\begin{align}
\mathbf{T}_{\mathbf{q}^{(n)}}^{\mathcal{G}} = 
\begin{cases} 
    \texttt{read}\left(\mathbf{C}_{\mathbf{q}^{(n)}}, \mathbf{q} \right), & \text{if } n = 1 \\
    \texttt{read}\left(\mathbf{C}_{\mathbf{q}^{(n)}}, \mathbf{q}\textcolor{blue}{, \mathcal{G}^{(n-1)}} \right), & \text{if } n \geq 2
\end{cases}
\label{eq:proximal_read_agent}
\end{align}


Apart from the first iteration where Eq.~\ref{eq:proximal_read} and ~\ref{eq:proximal_read_agent} are identical, the inclusion of the memory in the \texttt{read} operation differentiates the construction of proximal triples produced at the subsequent steps compared to the ones from Eq.~\ref{eq:proximal_read}. $\mathcal{G}^{(n)}$ maintains the aggregated content of proximal triples s.t. 
\begin{align}
\mathcal{G}^{(n)} = \left[ \mathbf{T}_{\mathbf{q}^{(1)}}^{\mathcal{G}}  \circ \cdots \circ \mathbf{T}_{\mathbf{q}^{(n)}}^{\mathcal{G}} \right],
\end{align}where $\circ$ defines the concatenation operation. The triple memory serves as a concise representation of all the accumulated evidence, up to the $n$-th step. 

We believe the process introduced by the \texttt{read} step along with the information storage paradigm served by the gist memory, aligns well with the communication between the hippocampus and neocortex. The combination of the two establishes the synergetic behaviour between our graph retriever and the LLM that we seek to achieve within \gear.



\subsection{Reasoning for Termination}
After $\mathcal{G}^{(n)}$ is updated, we check the sufficiency of the accumulated evidence, within it, for answering the original question. This is achieved with the following LLM reasoning step:
\begin{align}
\mathbf{a}^{(n)}, \mathbf{r}^{(n)}   = \texttt{reason}(\mathcal{G}^{(n)}, \mathbf{q}),
\end{align}
% We can also call it 'sufficiency' instead of 'answerability'. I do not really have a preference.
where $\mathbf{a}^{(n)}$ denotes the query's answerability given the available evidence in $\mathcal{G}^{(n)}$, and $\mathbf{r}^{(n)}$ represents the reasoning behind this determination. When the query is deemed answerable, the system concludes its iterative process.



\subsection{Query Re-writing}
The query re-writing process leverages an LLM that incorporates three key inputs: the original query $\mathbf{q}$, the accumulated memory, and crucially, the reasoning output $\mathbf{r}^{(n)}$ from the previous step. This process can be formally expressed as:
\begin{align}
\mathbf{q}^{(n+1)} = \texttt{rewrite}\left (\mathcal{G}^{(n)}, \mathbf{q}, \mathbf{r}^{(n)} \right),
\end{align}
where $\mathbf{q}^{(n+1)}$ represents the updated query, which serves as input for the retriever in the next iteration.\\
\subsection{After Termination}
\gear aims to return a single ranked list of passages. Given the final gist memory $\mathcal{G}^{(n)}$ upon termination, we link each proximal triple in $\mathcal{G}^{(n)}$ to a list of passages as follows:
\begin{align}
    \mathbf{C}_{t_j} = \texttt{passageLink}\left(t_j, k\right),
\end{align}
where $j \in \left \{1, \dots, \vert\mathcal{G}^{(n)}\vert \right \}$. Similar to \texttt{tripleLink}, \texttt{passageLink} is implemented by retrieving passages with a triple as the query (see Appendix~\ref{appendixpara:passage_link}). The final list of passages returned by \gear is the RRF of the resulting linked passages and passages retrieved across iterations:
\begin{align}
\mathbf{C}_\mathbf{q}^{(n)} = \mathrm{RRF}\big(&\mathbf{C}_{t_1}, \ldots,\mathbf{C}_{t_{\vert\mathcal{G}^{(n)}\vert}}, \nonumber\\
    &\mathbf{C}_{\mathbf{q}^{(1)}}, \ldots, \mathbf{C}_{\mathbf{q}^{(n)}} \big).
\end{align}

All relevant prompts for the \texttt{read}, \texttt{reason} and \texttt{rewrite} steps are provided in Appendix~\ref{subsec:online_retrieval_prompts}.

\vspace{-2mm}
\section{\sysname{} Overview} \label{sec:design}

Figure~\ref{fig:architecture} depicts the architecture of \sysname{}, which consists of three major components: 
\textit{Hybrid Programming Model}, \textit{3D-HybridEngine} and \textit{Auto-Mapping algorithm}.
The hybrid programming model includes a set of hierarchical APIs to enable flexible expression of the RLHF dataflow and efficient computation of models in the dataflow (\textsection\ref{sec:programming_model}). The 3D-HybridEngine is particularly designed for efficient training and generation of the actor model, allowing different 3D parallel configurations 
in the two stages and %
enabling zero memory redundancy and minimized communication overhead during the transition between two stages (\textsection\ref{sec:hybrid_engine}).
The auto- mapping algorithm determines optimized device %
placement of each model to maximize the throughput of RLHF (\textsection\ref{sec:auto_mapping}).





\begin{figure}[t]
    \includegraphics[width=\linewidth]{figs/fig_architecture.pdf}
    \vspace{-5mm}
    \caption{Architecture of HybridFlow. 
    }
    \vspace{-6mm}
    \label{fig:architecture}
\end{figure}


The workflow of our RLHF system goes as follows. A user provides the following inputs to start the RLHF system: (i) model specifications, including
{the architecture and size}
of the actor/critic/reference policy/reward models in the RLHF dataflow;
(ii) device placement of the models in the dataflow, as obtained by running the auto-mapping algorithm under given GPU cluster configurations; (iii) parallelism strategy for running each model in each stage, e.g., a tuple of (p, t, d) for 3D parallelism, where p, t, d represent PP size, TP size and DP size, respectively. %
The single controller program takes these inputs to initialize models in the RLHF dataflow and virtualized resource pool, dispatches operations/models to devices according to the placement plan, and invokes functions run by the multiple controllers on devices to carry out distributed computation of each model.

The multi-controller program implements the ParallelWorker class: it %
constructs parallel groups of each model among allocated devices %
according to its parallelism strategies, 
invokes the 3D-HybridEngine for actor training and generation, and can be integrated seamlessly with existing LLM engines~\cite{shoeybi2019megatron, rasley2020deepspeed, paszke2019pytorch, kwon2023efficient} for training, inference and generation of other models.
The transfer protocols are coordinated by the single controller program to support resharding of data (including prompts, responses, and other model outputs in RLHF) between models with distinct parallelism strategies. The data resharding of the actor between training and generation is handled by 3D-HybridEngine.




\secspace
\section{Evaluation} 
\label{sec:eval}
In this section, we evaluate \system's ability to protect individual video
frames from style mimicry. \S\ref{sec:setup} describes our video
datasets and experimental setup. \S\ref{sec:eval-metrics} introduces
our metrics for evaluation. \S\ref{sec:protection-robustness}-\ref{sec:video-types}
present results on \system's protection and efficiency. Due to the subjective
nature of interpreting successful style mimicry and visual nature of videos,
we evaluate protection using both automated metrics from existing work, as
well as visual judgement in a user study.

\para{Summary of results.}
Naive perturbations protect videos against style mimicry attempts 64.3\% of
the time according to surveyed users. However, perturbation removal attacks
on naive perturbations successfully recover video style, causing protection
rate to drop to 23.8\% (compared to completely unprotected videos at
17.7\%). \system{} is able to maintain protection even against removal
attacks, restoring protection against style mimicry attempts to 64.5\%. 
We verify that \system~ is able to preserve image perturbations across
consecutive image frames on a diverse set of videos genres and types, and is
robust against our averaging attack introduced in \S\ref{sec:eval-limitations}. 

\subsection{Experimental Setup} 
\label{sec:setup}
\para{Video datasets.}
We evaluate the effectiveness of \system~ on five diverse datasets, covering
different animations, as well as human actions and scenery. On average, each
video contains 6289 total frames. Our experiments conducted on YouTube videos
are for research purposes only, and trained models are deleted at the
conclusion of the study~\cite{fairuse}.
\begin{enumerate}
\item \textbf{Video Games}: 20 randomly selected YouTube videos from a single
  channel showcasing in-game content of different video games, ranging from 2
  to 6 minutes long.
\item \textbf{Human Actions}: 20 randomly selected untrimmed videos of human
  actions from the THUMOS-15 training dataset originally designed for action
  classification~\cite{Idrees_2017}, ranging from 1 to 4 minutes long.
\item \textbf{Japanese Anime}: 20 randomly selected YouTube videos from
  different channels of Japanese animated movies and tv-shows, ranging from 1
  to 5 minutes long.
\item \textbf{Animated Movies}: 10 randomly selected compilations of animated
  (Disney, Pixar, Sony) movie clips, ranging from 5 to 13 minutes long.
\item \textbf{Nature and Wildlife}: 10 randomly selected YouTube videos of
  animals and scenery in the wild, shot in a documentary format. Original
  videos are 30-60 minutes long, we clip each video to the first 2 minutes.
\end{enumerate}

We include three unique animation styles, Video Game (3D
rigs~\cite{videogameanimation}), Japanese Anime (hand
drawn~\cite{animeanimation}), and Animated Movies
(CGI~\cite{amoviesanimation}). The remaining two video datasets are selected
to cover the other types of video imagery, including real human presence
and photography/aerial footage. As a preprocessing step, we maintain
consistent quality among all videos by center cropping videos to square and
resizing to 512x512.  

\para{Defense configuration.}  To showcase the generalizability of \system~
to multiple image-based protection systems, we test \system~ on three such
algorithms: Mist, Anti-DB, and Glaze. We re-implement each defense using
$l_{\infty}$ bounded projected gradient descent (PGD), and use a consistent
targeted image generation method for all three~\cite{shan2023glaze}. As is
standard in image-based PGD methods, we constrain maximum absolute change
in each image pixel to 0.07.

\para{Mimicry configuration.}  We base our style mimicry setup on existing
work~\cite{mist,shan2023glaze,shan2023prompt,antidb} and adapt it to video
frames:
\begin{packed_enumerate}
\item Split each video into partitions using our scene splitting algorithm from \S\ref{sec:method}.
\item Use the CLIP aesthetic model to identify the frame with the highest image quality score within a scene, and select the highest quality images based on scenes with the highest image quality score.
\end{packed_enumerate}
    
We find that all datasets could successfully train style mimicry models with
30 images, except for Animated Movies, which required 60. Here, we can apply the pixel 
averaging adaptive mimicry attack, finetuning on 80\% of extracted images
and saving the remaining 20\% for testing. We use Dreambooth to
finetune a Stable Diffusion 2.1 model on the finetuning dataset for 1000
steps and learning rate of 1e-5. To generate matching synthetic images, we
generate captions from testing images with BLIP~\cite{li2022blip}, and query
the finetuned model to create a set of style mimicked images. 

We evaluate \system~ across several combinations of our proposed defense
system and style mimicry configurations. For each existing perturbation
algorithm, we evaluate the efficacy when perturbed images are untouched, and
when our image averaging attack is used. We evaluate the efficacy of \system~
against style mimicry only when the image averaging attack is used. We also
train a style mimicry model on every video where the frames are left
untouched.

\subsection{Evaluation Metrics}
\label{sec:eval-metrics}
The two key contributions of \system~ are robustness under potential
adversary attack, and reduction in computation costs, which we will evaluate
using the following metrics.

\para{Robustness.}  We measure robustness in two ways. The first two metrics
capture how well image perturbations can withstand adversarial attack. The
last two measure the impact that \system~ has on the quality of images
generated by style mimicry models.
\begin{packed_itemize}
\item \textbf{Latent $L_2$ Norm:} We employ the image encoder used in
  diffusion models to calculate latent representations of perturbed and
  non-perturbed (original) frames, and then calculate the $L_2$ distance
  between them as a measurement for image closeness. For example, a
  successful averaging attack on consecutively perturbed frames would lead to
  low latent $L_2$ norm, while a robust system should maintain a high latent
  $L_2$ norm. In particular, this metric isolates how diffusion models
  capture image differences. 
\item \textbf{Mean Pixel Difference:} We also measure the differences between
  images at a pixel level, motivated by the $l_{\infty}$ bounded pixel
  changes that all three of our defense algorithms are constrained by. Mean
  Pixel Difference (MPD) is model-agnostic, and defined as the average of all
  pixel differences between a perturbed image and original image. Similar to
  the latent $L_2$ norm, a higher MPD between perturbed and original frames
  signals higher protection.
\item \textbf{CLIP-Genre Shift:} We adapt a metric used in existing
  work~\cite{shan2023glaze} to demonstrate the effectiveness
  of \system~ at disrupting style mimicry. Intuitively, existing image
  perturbation algorithms are designed to cause diffusion models to learn the
  wrong artistic style. Thus, we can measure the success of \system's
  protection by calculating the percentage of generated images in which a
  CLIP image classifier unsuccessfully predicts the ground-truth style from
  training images. A higher CLIP-based Genre Shift score equates to stronger
  protection, while the opposite equates to stronger style mimicry. However, it 
  is a granular metric and its correlation to visual properties might be weak.
\item \textbf{Human-rated protection success rate (PSR):} To accurately capture 
  end-to-end visual properties, we perform
  two IRB-approved user studies (more details on participants in
  Appendix~\ref{app:detailed-user-study}) where participants look at images
  generated by 
  mimicry models and rate if the style mimicry was successful. The first asks
  artist volunteers to 
  rate performance on our two ``Art'' datasets: Anime and Video Games. The
  second asks general users to rate performance on all 5 datasets. For
  each video, we train 10 style mimicry models, matching our experiment
  configuration. For each mimicry model, we show participants 5 frames from
  the original video next to 5 frames generated by the mimicry model. Each
  participant is shown one example from each experiment configuration,
  randomly selected from the set of 80 videos (3 datasets x 20 videos + 2
  datasets x 10 videos each).

  \indent We ask participants to compare original
  video frames to those generated by mimicry, and rate the success on a 5-point
  Likert scale (ranging from ``Not successful at all'' to ``Very
  successful''). Following prior work, we define protection success rate as
  the percent of participants who rated how well generated images mimic
  original style as ``Not very well'' or ``Not well at all.'' We also show
  artists short 10 second clips of videos glazed naively and with
  \system. Here, we ask how noticeable the perturbations are on a 5 point
  Likert scale ranging from ``Very noticeable'' to ``Not noticeable at all.''
  We define \textit{Noticeability Rate} (NR) as the \% of users that think
  perturbations are ``Noticeable'' or ``Very noticeable.''
\end{packed_itemize}

\para{Computation efficiency:} 
Finally, we also measure the computation speed of \system. Here, we only experiment on one image perturbation algorithm due to computational constraints. Glaze is chosen due to its balance of speed (fastest of the three) and strong robustness. For each video, we conduct our measurement on a single A100 GPU.
\begin{packed_itemize}
\item \textbf{Speedup Factor:} We compute speedup factor as time taken to apply (naive)
  protection to every frame, divided by time taken to protect the video using
  \system.  Because of the compute time involved, we estimate full (naive) protection by estimating per frame protection time, scaled up by number of frames in the video.
\item \textbf{Seconds per Frame:} We also report the average time it takes to
  perturb each frame in a video. This provides a grounding metric that is not
  relative like speedup factor.
\end{packed_itemize}

\begin{table*}[t]
    \centering
    \resizebox{0.88\textwidth}{!}{
      \begin{tabular}{cccccccccc}
        & \multicolumn{3}{c}{Glaze}                                                                                                                                               & \multicolumn{3}{c}{Mist}                                                                                                                                                & \multicolumn{3}{c}{Anti-DB}                                                                                                                        \\ \cline{2-10} 
        & Naive              & \begin{tabular}[c]{@{}c@{}}Naive\\ + Attack\end{tabular} & \multicolumn{1}{c|}{\textbf{\begin{tabular}[c]{@{}c@{}}\system\\ + Attack\end{tabular}}} & Naive              & \begin{tabular}[c]{@{}c@{}}Naive\\ + Attack\end{tabular} & \multicolumn{1}{c|}{\textbf{\begin{tabular}[c]{@{}c@{}}\system\\ + Attack\end{tabular}}} & Naive              & \begin{tabular}[c]{@{}c@{}}Naive\\ + Attack\end{tabular} & \textbf{\begin{tabular}[c]{@{}c@{}}\system\\ + Attack\end{tabular}} \\ \hline
\multicolumn{1}{c|}{Video Game}          & 406.69 $\pm$ 20.09 & 262.82 $\pm$ 32.39                                       & \multicolumn{1}{c|}{425.19 $\pm$ 26.84}                                                 & 468.51 $\pm$ 40.68 & 296.29 $\pm$ 38.53                                       & \multicolumn{1}{c|}{496.04 $\pm$ 43.52}                                                 & 400.48 $\pm$ 35.12 & 240.38 $\pm$ 37.83                                       & 421.66 $\pm$ 37.01                                                 \\
\multicolumn{1}{c|}{Japanese Anime}      & 395.86 $\pm$ 20.32 & 284.51 $\pm$ 53.59                                       & \multicolumn{1}{c|}{406.93 $\pm$ 26.87}                                                 & 467.00 $\pm$ 43.44 & 319.77 $\pm$ 56.92                                       & \multicolumn{1}{c|}{493.53 $\pm$ 47.91}                                                 & 406.04 $\pm$ 41.29 & 272.34 $\pm$ 59.09                                       & 424.51 $\pm$ 41.96                                                 \\
\multicolumn{1}{c|}{Animated Movies}     & 380.04 $\pm$ 17.52 & 308.06 $\pm$ 47.66                                       & \multicolumn{1}{c|}{406.30 $\pm$ 26.05}                                                 & 475.07 $\pm$ 39.40 & 349.04 $\pm$ 51.88                                       & \multicolumn{1}{c|}{496.81 $\pm$ 42.17}                                                 & 404.55 $\pm$ 37.85 & 299.84 $\pm$ 55.27                                       & 422.11 $\pm$ 38.46                                                 \\
\multicolumn{1}{c|}{Nature and Wildlife} & 402.10 $\pm$ 17.24 & 319.23 $\pm$ 53.86                                       & \multicolumn{1}{c|}{408.62 $\pm$ 25.91}                                                 & 466.05 $\pm$ 41.64 & 346.28 $\pm$ 56.05                                       & \multicolumn{1}{c|}{496.48 $\pm$ 41.38}                                                 & 393.73 $\pm$ 30.02 & 297.58 $\pm$ 58.51                                       & 412.73 $\pm$ 30.82                                                 \\
\multicolumn{1}{c|}{Human Actions}       & 370.77 $\pm$ 27.41 & 316.53 $\pm$ 50.39                                       & \multicolumn{1}{c|}{397.97 $\pm$ 27.03}                                                 & 494.19 $\pm$ 39.40 & 369.70 $\pm$ 48.57                                       & \multicolumn{1}{c|}{508.08 $\pm$ 39.06}                                                 & 415.92 $\pm$ 36.58 & 316.19 $\pm$ 50.97                                       & 426.26 $\pm$ 35.29                                                
\end{tabular}
    }
    \caption{Latent $L_2$ norm between original and perturbed frames across all datasets and image perturbation algorithms. Averaging attack reduces perturbation effectiveness from naive video cloaking (attacked naive $<<$ naive), but is unsuccessful against \system~.}
    \label{tab:adv-algorithm-robustness-loss}
\end{table*}

\begin{table*}[t]
    \centering
    \resizebox{0.88\textwidth}{!}{
      \begin{tabular}{cccccccccc}
        & \multicolumn{3}{c}{Glaze}                                                                                                                                               & \multicolumn{3}{c}{Mist}                                                                                                                                                & \multicolumn{3}{c}{Anti-DB}                                                                                                                        \\ \cline{2-10} 
        & Naive              & \begin{tabular}[c]{@{}c@{}}Naive\\ + Attack\end{tabular} & \multicolumn{1}{c|}{\textbf{\begin{tabular}[c]{@{}c@{}}\system\\ + Attack\end{tabular}}} & Naive              & \begin{tabular}[c]{@{}c@{}}Naive\\ + Attack\end{tabular} & \multicolumn{1}{c|}{\textbf{\begin{tabular}[c]{@{}c@{}}\system\\ + Attack\end{tabular}}} & Naive              & \begin{tabular}[c]{@{}c@{}}Naive\\ + Attack\end{tabular} & \textbf{\begin{tabular}[c]{@{}c@{}}\system\\ + Attack\end{tabular}} \\ \hline
\multicolumn{1}{c|}{Video Game}          & 111.99 $\pm$ 11.50 & 85.00 $\pm$ 12.37                                        & \multicolumn{1}{c|}{108.54 $\pm$ 12.66}                                                 & 120.88 $\pm$ 6.79  & 102.91 $\pm$ 14.20                                       & \multicolumn{1}{c|}{119.26 $\pm$ 7.52}                                                  & 124.64 $\pm$ 5.94  & 98.21 $\pm$ 14.69                                        & 121.10 $\pm$ 7.49                                                  \\
\multicolumn{1}{c|}{Japanese Anime}      & 112.31 $\pm$ 12.92 & 91.03 $\pm$ 14.44                                        & \multicolumn{1}{c|}{108.25 $\pm$ 14.26}                                                 & 120.84 $\pm$ 8.71  & 106.25 $\pm$ 15.08                                       & \multicolumn{1}{c|}{119.03 $\pm$ 9.88}                                                  & 124.34 $\pm$ 7.94  & 103.01 $\pm$ 15.91                                       & 120.27 $\pm$ 10.27                                                 \\
\multicolumn{1}{c|}{Animated Movies}     & 109.43 $\pm$ 13.14 & 89.92 $\pm$ 12.16                                        & \multicolumn{1}{c|}{107.57 $\pm$ 13.56}                                                 & 122.22 $\pm$ 8.59  & 109.44 $\pm$ 15.08                                       & \multicolumn{1}{c|}{121.37 $\pm$ 9.19}                                                  & 125.89 $\pm$ 7.98  & 106.27 $\pm$ 15.99                                       & 122.42 $\pm$ 9.21                                                  \\
\multicolumn{1}{c|}{Nature and Wildlife} & 115.14 $\pm$ 5.39  & 98.65 $\pm$ 12.56                                        & \multicolumn{1}{c|}{109.96 $\pm$ 8.23}                                                  & 121.54 $\pm$ 3.13  & 110.49 $\pm$ 10.65                                       & \multicolumn{1}{c|}{120.80 $\pm$ 4.81}                                                  & 124.79 $\pm$ 2.30  & 108.15 $\pm$ 13.53                                       & 120.92 $\pm$ 5.25                                                  \\
\multicolumn{1}{c|}{Human Actions}       & 109.61 $\pm$ 16.02 & 90.18 $\pm$ 15.91                                        & \multicolumn{1}{c|}{108.22 $\pm$ 16.41}                                                 & 120.85 $\pm$ 14.49 & 108.19 $\pm$ 21.66                                       & \multicolumn{1}{c|}{119.06 $\pm$ 16.02}                                                 & 124.52 $\pm$ 14.13 & 105.51 $\pm$ 23.66                                       & 120.09 $\pm$ 16.64                                                
\end{tabular}
    }
    \caption{Mean pixel difference (MPD) between original and perturbed frames across all datasets and image perturbation algorithms. Averaging attack reduces perturbation effectiveness from naive video cloaking (attacked naive $<<$ naive), but is unsuccessful against \system~.}
    \label{tab:adv-algorithm-robustness-pd}
\end{table*}


\begin{table*}[t]
    \centering
    \resizebox{0.88\textwidth}{!}{
      \begin{tabular}{cccccccccccc}
        \multirow{2}{*}{}                                                        & \multirow{2}{*}{}                         &                                       & \multicolumn{3}{c}{Glaze}                                                                                                                                             & \multicolumn{3}{c}{Mist}                                                                                                                                              & \multicolumn{3}{c}{Anti-DB}                                                                                                                      \\ \cline{3-12} 
                                                                                 &                                           & \multicolumn{1}{c|}{Clean}            & Naive            & \begin{tabular}[c]{@{}c@{}}Naive\\ + Attack\end{tabular} & \multicolumn{1}{c|}{\textbf{\begin{tabular}[c]{@{}c@{}}\system\\ + Attack\end{tabular}}} & Naive            & \begin{tabular}[c]{@{}c@{}}Naive\\ + Attack\end{tabular} & \multicolumn{1}{c|}{\textbf{\begin{tabular}[c]{@{}c@{}}\system\\ + Attack\end{tabular}}} & Naive            & \begin{tabular}[c]{@{}c@{}}Naive\\ + Attack\end{tabular} & \textbf{\begin{tabular}[c]{@{}c@{}}\system\\ + Attack\end{tabular}} \\ \hline
        \multirow{2}{*}{Artist}                                                  & \multicolumn{1}{c|}{Video Game}           & \multicolumn{1}{c|}{15.05 $\pm$ 1.10} & 92.23 $\pm$ 0.72 & 19.35 $\pm$ 1.13                                         & \multicolumn{1}{c|}{97.98 $\pm$ 0.52}                                                   & 82.88 $\pm$ 0.85 & 14.81 $\pm$ 1.08                                         & \multicolumn{1}{c|}{91.01 $\pm$ 0.69}                                                   & 93.26 $\pm$ 0.69 & 23.71 $\pm$ 1.25                                         & 91.74 $\pm$ 0.67                                                   \\
                                                                                 & \multicolumn{1}{c|}{Japanese Anime}       & \multicolumn{1}{c|}{24.27 $\pm$ 1.13} & 82.88 $\pm$ 0.97 & 27.03 $\pm$ 1.13                                         & \multicolumn{1}{c|}{72.32 $\pm$ 1.00}                                                   & 68.69 $\pm$ 0.97 & 34.74 $\pm$ 1.18                                         & \multicolumn{1}{c|}{65.77 $\pm$ 1.10}                                                   & 79.44 $\pm$ 0.83 & 36.19 $\pm$ 1.14                                         & 83.84 $\pm$ 0.83                                                   \\ \hline
        \multirow{5}{*}{\begin{tabular}[c]{@{}c@{}}General\\ Users\end{tabular}} & \multicolumn{1}{c|}{Video Game}           & \multicolumn{1}{c|}{20.14 $\pm$ 1.13} & 85.59 $\pm$ 0.82 & 18.26 $\pm$ 1.11                                         & \multicolumn{1}{c|}{88.99 $\pm$ 0.76}                                                   & 86.84 $\pm$ 0.87 & 16.67 $\pm$ 1.06                                         & \multicolumn{1}{c|}{95.24 $\pm$ 0.65}                                                   & 73.87 $\pm$ 1.02 & 25.20 $\pm$ 1.17                                         & 80.39 $\pm$ 0.97                                                   \\
                                                                                 & \multicolumn{1}{c|}{Japanese Anime}       & \multicolumn{1}{c|}{20.95 $\pm$ 1.07} & 66.07 $\pm$ 1.05 & 21.21 $\pm$ 1.09                                         & \multicolumn{1}{c|}{54.63 $\pm$ 1.07}                                                   & 59.46 $\pm$ 1.18 & 26.50 $\pm$ 1.14                                         & \multicolumn{1}{c|}{68.80 $\pm$ 1.03}                                                   & 49.00 $\pm$ 1.07 & 19.61 $\pm$ 0.99                                         & 52.53 $\pm$ 1.21                                                   \\
                                                                                 & \multicolumn{1}{c|}{Animated Movies}      & \multicolumn{1}{c|}{14.81 $\pm$ 1.19} & 44.83 $\pm$ 1.15 & 14.81 $\pm$ 1.04                                         & \multicolumn{1}{c|}{36.51 $\pm$ 1.16}                                                   & 35.00 $\pm$ 1.04 & 16.67 $\pm$ 1.06                                         & \multicolumn{1}{c|}{50.91 $\pm$ 1.19}                                                   & 26.79 $\pm$ 1.14 & 12.24 $\pm$ 1.03                                         & 35.94 $\pm$ 1.12                                                   \\
                                                                                 & \multicolumn{1}{c|}{Nature and Wildlife}  & \multicolumn{1}{c|}{8.89 $\pm$ 1.02}  & 78.43 $\pm$ 0.80 & 18.06 $\pm$ 1.10                                         & \multicolumn{1}{c|}{65.38 $\pm$ 1.02}                                                   & 68.63 $\pm$ 0.98 & 20.41 $\pm$ 1.07                                         & \multicolumn{1}{c|}{85.00 $\pm$ 0.73}                                                   & 69.49 $\pm$ 1.16 & 31.34 $\pm$ 1.34                                         & 64.15 $\pm$ 1.01                                                   \\
                                                                                 & \multicolumn{1}{c|}{Generic Video Scenes} & \multicolumn{1}{c|}{15.05 $\pm$ 1.03} & 70.64 $\pm$ 1.05 & 42.42 $\pm$ 1.18                                         & \multicolumn{1}{c|}{49.02 $\pm$ 1.17}                                                   & 54.00 $\pm$ 1.05 & 38.74 $\pm$ 1.15                                         & \multicolumn{1}{c|}{60.00 $\pm$ 1.06}                                                   & 62.73 $\pm$ 1.15 & 21.51 $\pm$ 1.19                                         & 61.21 $\pm$ 1.20                                                  
        \end{tabular}
    }
    \caption{Percentage of \textit{artists} and \textit{general users} who
      deem protection is successful on different videos. Artists rated
      more art-driven videos (Video Games and Japanese Anime), while general
      users rated all videos. Column include naive protection,
      pixel-averaging on naive protection, and pixel-averaging on
      \system{}. Pixel-averaging breaks protection; \system{} restores it.}
    \label{tab:adv-algorithm-robustness-artist}
\end{table*}

\begin{table*}[t]
    \centering
    \resizebox{0.88\textwidth}{!}{
      \begin{tabular}{ccccccccccc}
        &                                      & \multicolumn{3}{c}{Glaze}                                                                                                                                            & \multicolumn{3}{c}{Mist}                                                                                                                                             & \multicolumn{3}{c}{Anti-DB}                                                                                                                     \\ \cline{2-11} 
        & \multicolumn{1}{c|}{Clean}           & Naive           & \begin{tabular}[c]{@{}c@{}}Naive\\ + Attack\end{tabular} & \multicolumn{1}{c|}{\textbf{\begin{tabular}[c]{@{}c@{}}\system\\ + Attack\end{tabular}}} & Naive           & \begin{tabular}[c]{@{}c@{}}Naive\\ + Attack\end{tabular} & \multicolumn{1}{c|}{\textbf{\begin{tabular}[c]{@{}c@{}}\system\\ + Attack\end{tabular}}} & Naive           & \begin{tabular}[c]{@{}c@{}}Naive\\ + Attack\end{tabular} & \textbf{\begin{tabular}[c]{@{}c@{}}\system\\ + Attack\end{tabular}} \\ \hline
\multicolumn{1}{c|}{Video Games}         & \multicolumn{1}{c|}{0.49 $\pm$ 0.23} & 0.85 $\pm$ 0.23 & 0.71 $\pm$ 0.26                                          & \multicolumn{1}{c|}{0.97 $\pm$ 0.07}                                                    & 0.85 $\pm$ 0.20 & 0.75 $\pm$ 0.22                                          & \multicolumn{1}{c|}{0.97 $\pm$ 0.06}                                                    & 0.77 $\pm$ 0.26 & 0.61 $\pm$ 0.26                                          & 0.94 $\pm$ 0.13                                                    \\
\multicolumn{1}{c|}{Japanese Anime}      & \multicolumn{1}{c|}{0.25 $\pm$ 0.18} & 0.76 $\pm$ 0.23 & 0.58 $\pm$ 0.19                                          & \multicolumn{1}{c|}{0.88 $\pm$ 0.13}                                                    & 0.77 $\pm$ 0.24 & 0.60 $\pm$ 0.21                                          & \multicolumn{1}{c|}{0.92 $\pm$ 0.11}                                                    & 0.64 $\pm$ 0.30 & 0.41 $\pm$ 0.23                                          & 0.86 $\pm$ 0.13                                                    \\
\multicolumn{1}{c|}{Animated Movies}     & \multicolumn{1}{c|}{0.45 $\pm$ 0.22} & 0.65 $\pm$ 0.29 & 0.52 $\pm$ 0.28                                          & \multicolumn{1}{c|}{0.68 $\pm$ 0.25}                                                    & 0.61 $\pm$ 0.28 & 0.53 $\pm$ 0.30                                          & \multicolumn{1}{c|}{0.73 $\pm$ 0.21}                                                    & 0.58 $\pm$ 0.31 & 0.46 $\pm$ 0.29                                          & 0.65 $\pm$ 0.28                                                    \\
\multicolumn{1}{c|}{Nature and Wildlife} & \multicolumn{1}{c|}{0.32 $\pm$ 0.15} & 0.80 $\pm$ 0.24 & 0.61 $\pm$ 0.20                                          & \multicolumn{1}{c|}{0.91 $\pm$ 0.12}                                                    & 0.82 $\pm$ 0.18 & 0.53 $\pm$ 0.30                                          & \multicolumn{1}{c|}{0.98 $\pm$ 0.02}                                                    & 0.69 $\pm$ 0.32 & 0.41 $\pm$ 0.22                                          & 0.94 $\pm$ 0.09                                                    \\
\multicolumn{1}{c|}{Human Actions}       & \multicolumn{1}{c|}{0.22 $\pm$ 0.23} & 0.70 $\pm$ 0.32 & 0.50 $\pm$ 0.31                                          & \multicolumn{1}{c|}{0.78 $\pm$ 0.22}                                                    & 0.80 $\pm$ 0.25 & 0.66 $\pm$ 0.27                                          & \multicolumn{1}{c|}{0.94 $\pm$ 0.07}                                                    & 0.68 $\pm$ 0.33 & 0.45 $\pm$ 0.27                                          & 0.81 $\pm$ 0.24                                                   
\end{tabular}
  }
  \caption{CLIP-Genre Shift across all datasets and image perturbation algorithms. Averaging attack decreases the number of images generated of a different style to the original video (attacked naive $\approx$ clean), but fails to do the same against \system~.}
    \label{table:adv-algorithm-style-mimicry}
\end{table*}

\subsection{Robustness against Pixel-Averaging Mimicry}
\label{sec:protection-robustness}
\para{\system~ prevents style mimicry. } 
We showed in \S\ref{sec:eval-limitations} that pixel-averaging attacks can
remove image protection by smoothing them out across similar
frames. We begin by looking at the ability of pixel-averaging methods to extract
frames similar to the original, as measured by pixel level metrics in
Table~\ref{tab:adv-algorithm-robustness-loss} and
\ref{tab:adv-algorithm-robustness-pd}.

For all protection tools, across each category of videos, we see that the
pixel-averaging attack significantly reduces the distance between the
protected frames compared to the originals, measured by both latent $L_2$ norm and
MPD. More importantly, we see that that same pixel-averaging attack fail when
applied to frames protected by \system{}, and it actually increases the
distances from the original.

Next, we turn our attention to the ability of adaptive mimicry attacks and
their ability to produce accurate end-to-end mimicry models. Table~\ref{tab:adv-algorithm-robustness-artist} shows the results of our
two user studies, involving both a population of artists and general users
(participant details in Appendix~\ref{app:detailed-user-study}). While
artists' views varied somewhat from general users across categories, all
users consistently provided the same feedback, that pixel-averaging broke the
protection provided by naive anti-mimicry tools, but \system{} restored that
protection (and in many cases increased protection higher than naive
protection levels). Table~\ref{table:adv-algorithm-style-mimicry} quantified
the same issue of end to end protection, but using the CLIP
CLIP-Genre shift metric. Results are very consistent with those from user
studies. \system{} restored protection broken by pixel-averaging attack and
in many cases, improved protection beyond the original naive levels.

Finally, Figure~\ref{fig:core-style-mimicry-results} shows some examples of
images generated by style mimicry models under our experiment configurations.

\begin{figure}[t]
    \centering
    \includegraphics[width=0.95\columnwidth]{plots/core-style-mimicry-results-eps-converted-to.pdf}
    \vspace{-0.1in}
    \caption{Some visual examples of style mimicry on \system~ showing it is
      robust to pixel averaging attack.}
    \label{fig:core-style-mimicry-results}
    \vspace{-0.1in}    
  \end{figure}
\begin{figure}[t]
    \centering
    \includegraphics[width=3in]{plots/eval/eps-robustness-eps-converted-to.pdf}
    \vspace{-0.2in}
    \caption{Average robustness (CLIP-Genre Shift) increases as $l_{\infty}$
      perturbation budget increases on 4 Video Game videos. For comparison,
      mimicry on original frames produces CLIP-Genre Shift of 0.44.}
    \label{fig:eps-robustness}
\end{figure}

\para{Impact of perturbation budget on robustness.} 
In Figure~\ref{fig:eps-robustness}, we show that robustness under \system{}
increases with perturbation budget, quickly maxing out after 0.05. This
follows the same trend we see in existing image-only perturbation algorithms. 


\subsection{Computational Costs and Video Quality}
\label{subsec:protection-usability}
In \S\ref{subsec:limitations}, we identified two additional limitations
of a naive application of anti-mimicry tools: high computation overhead and
poor video quality (randomness across per-frame perturbations appear as
flickering snow when viewed at regular speeds). Here, we show that
\system~ significantly improves on both the computation overhead of
perturbing video frames, and increases video quality over naively protected
videos.

\para{\system~ reduces computation overhead. } 
We measure the computation efficiency of \system~ on three of our
datasets. As shown in Table~\ref{tab:adv-algorithm-efficiency}, the Video
Game and Human Actions datasets achieve 7-8x speedup factor when using \system~ over
its naive video protection, while the Japanese Anime dataset achieves just over
4x speedup factor. \system~ would improve computation efficiency of a 5 minute 30
fps video from 88 to 11 hours. This magnitude of speedup factor greatly improves the
usability of applying image-based perturbations to videos, and makes video
protection more reasonable for video creators, particularly smaller or
independent creator groups.

The speedup factor is lowest for Japanese anime videos, likely due to the high movement 
across frames. In practice, we can further improve the speedup by changing our
system parameters, \eg changing second threshold $\tau_2$
from 0.45 to 0.8, which increases speedup factor of the four slowest Japanese Anime videos from 2.12 to 3.90 with a small tradeoff in robustness (more details
in Appendix~\ref{app:num-images-average} and \ref{app:eps-robustness}).  

\begin{table}[t]
    \centering
    \resizebox{0.4\textwidth}{!}{
        \begin{tabular}{c|cc}
            & Avg. Speedup Factor    & Avg. Seconds per Frame \\ \hline
            Video Game     & 7.87 $\pm$ 2.81 & 5.16 $\pm$ 2.13        \\
            Japanese Anime & 4.16 $\pm$ 3.19 & 11.25 $\pm$ 4.55       \\
            Human Action   & 7.37 $\pm$ 4.34 & 7.84 $\pm$ 6.43       
        \end{tabular}
    }
    \caption{\system~ with Glaze enables significant computation speedup factor
      across all Video Game, Japanese Anime, and Human action videos when
      compared to naive video protection. For comparison, naive video
      protection takes on average 35 seconds per frame on a single A100 GPU.}
    \label{tab:adv-algorithm-efficiency}
\end{table}


\begin{table}[t]
    \centering
    \resizebox{0.4\textwidth}{!}{
        \begin{tabular}{c|cc}
            & \multicolumn{2}{c}{\% of Users Notice Perturbations} \\
            & Naive Glaze               & \system~ w/ Glaze           \\ \hline
            Video Game     & 70.30 $\pm$ 1.03          & 21.29 $\pm$ 1.06         \\
            Japanese Anime & 48.73 $\pm$ 1.16          & 18.27 $\pm$ 1.0         
        \end{tabular}
    }
    \caption{Our user study shows that the percentage of users who notice
      perturbations generated by \system~ is
      significantly less than those who notice perturbations in naive video
      protection. (Tests implemented using Glaze).} 
    \label{tab:adv-algorithm-style-mimicry-visual}
\end{table}

\para{\system~ is less visually disruptive/noticeable. }  Even though
image-based perturbation methods are designed to be imperceptible, applying
them naively to videos leads to protected videos that flicker and are
noticeably grainy. This is a result of perturbation masks changing
drastically from frame to frame, a problem that \system~ addresses
directly. In our user study, we show participants 10-second clips of
protected videos from our three datasets. The first video is naively protected
using our implementation of Glaze, and the second video is protected with 
\system{}. Participants are asked to identify how visible the
perturbations are in both 
videos. Table~\ref{tab:adv-algorithm-style-mimicry-visual} presents
\textit{noticeability rate}, showing artists are able to notice perturbations
significantly more when videos are protected naively (70\%) than with
\system~ (21\%).

\begin{table}[t]
    \centering
    \resizebox{0.3\textwidth}{!}{
        \begin{tabular}{c|cc}
            & CLIP-Genre Shift & Speedup Factor          \\ \hline
            15fps & 0.97 $\pm$ 0.04  & 6.11 $\pm$ 1.90  \\
            30fps & 0.98 $\pm$ 0.02  & 9.57 $\pm$ 2.47  \\
            60fps & 1.00 $\pm$ 0.00  & 15.96 $\pm$ 4.06
        \end{tabular}
    }
    \caption{Different framerates do not impact protection
      robustness (CLIP-Genre Shift), but higher framerates lead to higher
      speedup factor under \system. (4 Video Game videos.)} 
    \label{table:fps-test}
\end{table}

\begin{figure}[t]
  \centering
  \includegraphics[width=0.9\columnwidth]{plots/eval/consecutive-frames-speedup-eps-converted-to.pdf}
  \vspace{-0.25in}
  \caption{Speedup factor decreases as movement increases between frames (latent $L_2$
    norm) on all Video Game videos.}
  \label{fig:action-across-datasets}
\end{figure}


\subsection{Impact of Video Types on Robustness and Efficiency}
\label{sec:video-types}
Beyond examining the efficacy of \system~ on a variety of video genres, we
also examine four aspects of videos creation/distribution irrespective to
genre that real-world content creators are likely to deal with while posting
videos online. We investigate the effects that framerate, movement between
frames, duration of scenes, and compression, have on robustness and
computation efficiency. We anticipate these as questions that video content
creators are likely to ask with respect to the efficacy of \system. In this
section, we show that \system's robustness holds steady across video types,
while computational efficiency is the most prone to change. 


\para{FPS}
Here, we evaluate the change in robustness and computation efficiency of
\system~ when videos are encoded with a different number of frames per second
(fps). We evaluate \system~ with Glaze on three different framerates, 15fps,
30fps, and 60fps, and measure its performance on four videos from the Video
Game Dataset. 30fps and 60fps are widely used in the video
community~\cite{ytvideosettings}, and we also include 15 to show the effects
that much a slower framerate has on robustness and computation
efficiency. Our results can be found in Table~\ref{table:fps-test}, which
show that framerate does not have significant impact on robustness, but that
increasing framerate from 15fps to 60fps increases speedup factor from 6x to
16x. Thus, we show that \system's computation performance is at its worst
with low framerate, but scales well when content creators choose to increase
framerate in their videos. 


\para{Action within scenes}
Movement within videos is difficult to capture, but we use latent $L_2$ norm between
two image latent representations as a good approximation. Intuitively, minor
movement between frames should lead to minor changes in image latent
representation. In this section, we investigate the effect that latent $L_2$
norm between consecutive frames in video scenes has on computation
efficiency. We measure the latent $L_2$ norm between consecutive frames in a
scene for every video in the Video Game dataset, and compare the median of
the latent $L_2$ norm across scenes to the speedup factor \system~ with Glaze
provides. In Figure~\ref{fig:action-across-datasets}, we show that there is a
negative linear correlation between $L_2$ latent norm and
speedup factor. Intuitively, videos with higher action within scenes are more likely
to require additional optimization in order to better align image
perturbations across fast changing frames. 

Finally, we also analyze the impact of {\bf scene duration} (number of frames per
scene) and {\bf video compression} factor (bitrate of video) on robustness and
speedup factor. We found that these factors have no observable impact.




\section{Scene Splitting Adaptive Attack}
\label{sec:counter}

\system~ removes uncontrolled randomness in perturbations between similar
frames, thereby disabling adaptive attacks that leverage cross frame pixel
correlations. However, does its own design give rise to new countermeasures?
We carefully considered this question, and discuss what we consider the
strongest possible countermeasure to \system{}. We describe the potential
countermeasure and evaluate its efficacy against \system.

The intuition for the countermeasure is to manipulate the scene
identification process to force a scene break between highly similar
frames. If this can be achieved, then the attacker could obtain two
consecutive (and similar) frames that have been perturbed towards different
targets. They could then apply a version of the pixel-averaging attack to
restore the original, unprotected frames. We call this the scene-splitting
attack, and assume a powerful attacker who can somehow force \system{} to
insert a scene break in the middle of a sequence of similar frames.

\para{The scene splitting attack.} 
We simulate a strong adaptive attack by dividing a single scene from a
video into two subscenes, Scene $S_1$ and $S_2$ each containing $M$ frames. We
use each subscene (Scene$_i$) to generate a target tensor $T_{i}$ with the
aim of maximizing distance between target tensors, which should maximize the
difference between perturbations generated from these targets. 
Now that we have obtained $T_1, T_2$, we apply \system~ to perturb the two
consecutive frames around the bad scene split. Subsequently, we perform a
pixel averaging attack on these two perturbed frames, following the
implementation detailed in \S\ref{sec:eval-limitations}. 

We test our adaptive attack on 20 videos from the Japanese Anime and Video
Game datasets (10 videos from each dataset). We choose these two datasets
because they are best aligned with current threats as explained in
\S\ref{sec:intro}. We limit ourselves to only 10 videos from each category
because of the computation time involved.


\begin{figure}[t]
  \centering
  \includegraphics[width=1\columnwidth]{plots/countermeasures-eps-converted-to.pdf}
  \caption{Visual examples of mimicry attempts on Clean frames and frames
    protected by \system~ across the standard Adaptive Attack and the
    Scene-Splitting Adaptive Attack. CLIP-Genre Shift score demonstrates
    robust protection against Mimicry attacks under both adaptive attack
    scenarios.} 
  \label{fig:style-mimicry-ctr}
\end{figure}

\para{\system~ is robust to scene-splitting attack.}
Tables~\ref{tab:countermeasure-loss-score}
and~\ref{tab:countermeasure-pd-score} show results that quantify the
image-level difference between the original frames and the frames produced by
the scene-splitting attack. They show that \system~ is robust to the
scene-splitting attack combined with pixel averaging attacks attempting to
recover original frames. There is only a minimal decrease in $L_2$ norm and
$MPD$ value when \system~ is applied to videos under the new Adaptive
Scene-Splitting Attack. In Figure~\ref{fig:style-mimicry-ctr}, we 
further verify that style mimicry using a combination of
scene-splitting and pixel-averaging is still unsuccessful, through both
visual examples of generated images and associated CLIP-genre shift scores,
which remain unchanged under the scene splitting attack.

\para{Why does the attack fail?} The failure of the scene-splitting attack
makes sense, once we consider its limitations. Regardless of where the
attacker forces a new scene break, frames inside the two new scenes ($S_1$
and $S_2$) are bounded in their maximum difference from each other. Thus the
two source frames for $S_1$ and $S_2$ (each computed as an average of frames
in the scene) is also bounded in their dissimilarity. Thus, it is likely
their resulting target tensors, and consequently the perturbations generated
from them, are also small. This intuition holds even if the attacker could
break a single long scene into several scenes of their choosing.

  \begin{table}[t]
    \centering
      \resizebox{0.5\textwidth}{!}{
      \centering
  \begin{tabular}{c|ccc}
    & Naive         & \multicolumn{1}{c}{\begin{tabular}[c]{@{}c@{}}\system\\ + Adaptive Attack\end{tabular}} & \multicolumn{1}{c}{\begin{tabular}[c]{@{}c@{}}\system~  + Adaptive \\ Scene-Splitting Attack\end{tabular}} \\ \hline
    Video Game     & 405.38 $\pm$ 19.43 & 421.21 $\pm$ 25.52 & 394.80 $\pm$ 25.16       \\
    Japanese Anime & 396.76 $\pm$ 19.05 & 406.43 $\pm$ 24.07 & 387.06 $\pm$ 23.98
\end{tabular}
    }\caption{Latent $L_2$ norm between original and perturbed frames across Video Games and Japanese Anime datasets. Demonstrates Scene-Splitting Adaptive Attack is not able to significantly reduce protection \system~ offers.}
  \label{tab:countermeasure-loss-score}
  \end{table}

  \begin{table}[t]
    \centering
      \resizebox{0.5\textwidth}{!}{
      \centering
  \begin{tabular}{c|ccc}
    & Naive         & \multicolumn{1}{c}{\begin{tabular}[c]{@{}c@{}}\system\\ + Adaptive Attack\end{tabular}} & \multicolumn{1}{c}{\begin{tabular}[c]{@{}c@{}}\system~  + Adaptive \\ Scene-Splitting Attack\end{tabular}} \\ \hline
    Video Game     & 111.58 $\pm$ 12.38 & 107.42 $\pm$ 13.47 & 100.28 $\pm$ 11.81 \\
    Japanese Anime & 112.66 $\pm$ 11.05 & 108.57 $\pm$ 12.59 & 102.02 $\pm$ 11.43       
\end{tabular}
    }\caption{MPD between original and perturbed frames across Video Games and Japanese Anime datasets. Demonstrates Scene-Splitting Adaptive Attack is not able to significantly reduce protection \system~ offers.}
  \label{tab:countermeasure-pd-score}
  \end{table}


\section{Discussion and conclusion}
\label{sec:discussion}

This work introduces a design for agents that assist users in generating images through an interactive process of proactive question asking and belief graph refinement. By dynamically updating its understanding of the user's intent, the agent facilitates a more collaborative and precise approach to image generation. Moreover, presenting the agent's belief graph can be a generalizable method for AI transparancy, which is an important factor given the increasing complexity of modern AI models. 

\textbf{Modular design.}  Our agent prototypes are highly modular: the agents use frozen T2I models to generate images based on the prompts that the agent updated. Therefore when a better off-the-shelf T2I model becomes available, it can be directly plugged into the agents and the system will achieve better performance without any additional adaptation\footnote{T2T scores in \Cref{tab:auto_eval} ablates the T2I model and only performs similarity on the captions. Our agents have achieved a 92\%+ T2T score, showing that their performance can be boosted by adopting better T2I models.}.  

\textbf{Personalized content.} By asking clarification questions, our agents enable a more customizable and personalized content creation experience. Because different groups of people may perceive helpfulness and harmfulness of contents differently, learning more about the user through clarification questions before generation can potentially mitigate risks of generating contents that are offensive to each specific user, and increase likelihoods of producing helpful outputs.


\textbf{Future work.} Alternative to the modular design, one can explore generating images directly from belief graphs and fine-tuning  LLM/VLMs on text/image trajectories that include asking questions. These may require a) collecting data such as gold-standard trajectories or annotations on the quality of trajectories of human-agent conversations and b) new approaches to fine-tune the model on multi-turn trajectories of images and text, which can potentially improve the performance of the agent.










\subsection*{Acknowledgements}
We would like to thank Jason Baldridge and Zoubin Ghahramani for insightful discussions on multi-turn T2I and belief states, Mahima Pushkarna for the help and consultation on user study. We would also like to thank Richard Song and Noah Fiedel for feedback on the paper.


{
 \footnotesize
 \bibliographystyle{acm}
 \bibliography{glaze}
}

\balance

\appendix
\section{Appendix} \label{appendix}


\subsection{NewYorker Data for evaluation}

\begin{figure}[!ht]
\small
\centering
\includegraphics[width=0.4\textwidth]{figures/length.png}
\caption{\label{lengthdist} Distribution of word count of stories in our test set}
\end{figure}

Table \ref{teststories} shows the data used for conducting our evaluation. The 12 stories shown are taken from The New Yorker and summarized into single-sentence plots. These stories come from highly established literary experts acting as an upper bound for what it means to be creative. These stories span complex themes.

\begin{table*}[!ht]
\centering
\small
\def\arraystretch{1.35}
\begin{tabular}{|l|}
\hline
\begin{tabular}[c]{@{}l@{}}Write a New Yorker-style story given the plot below. Make sure it is atleast \textbf{\color{blue}\{\{word\_count\}\}} words. Directly start with the\\ story, do not say things like `Here's the story {[}...{]}:\end{tabular}                                                                                                                                                                                            \\ \hline\hline
\begin{tabular}[c]{@{}l@{}}You wrote the story I gave you below. I requested a story with \textbf{\color{blue}\{\{word\_count\}\}} words, but the story only has\\ \textbf{\color{blue}\{\{current\_word\_count\}\}} words. Can you rewrite the story to make it longer, and closer to the \textbf{\color{blue}\{\{word\_count\}\}} word target\\ I gave you. Directly start with the story, do not say things like `Here's the story {[}...{]}:`\\ \\ Current story: \{\{story\}\}\end{tabular} \\ \hline
\end{tabular}
\vspace{2ex}
\caption{\label{promptstory}Prompt to write the initial story (Row1) vs Prompt to rewrite the initial story to be longer. word\_count represents the number of words in the human written story on a given plot (P) while current\_word\_count represents the number of words in the LLM generated story on the same plot (P)}
\end{table*}

\begin{table*}[!ht]
\def\arraystretch{1.15}
\small
\begin{tabular}{|l|l|}
\hline
Story                                    & Plot                                                                                                                                                                                                                                                                                                                                                                                                                                                                                                                                   \\ \hline
\href{https://www.newyorker.com/books/flash-fiction/a-triangle}{A Triangle}                               & \begin{tabular}[c]{@{}l@{}}An observer becomes entranced by a seemingly ordinary couple on the street, follows them home, and then \\watches them from outside in the rising floodwaters, drawing an eerie connection between the woman and\\ a discarded, burned chair they'd noticed earlier.\end{tabular}                                                                                                                                                                    \\ \hline\hline
\href{https://www.newyorker.com/books/flash-fiction/barbara-detroit-1966}{\begin{tabular}[c]{@{}l@{}}Barbara\\ Detroit,1966\end{tabular}}                    & \begin{tabular}[c]{@{}l@{}}On Feb 12, 1966, a heavily pregnant woman named Barbara experienced a shocking incident in her synagogue\\in Southfield, Detroit, where a young man shot and killed the renowned Rabbi Adler before turning the gun\\ on himself, and though Barbara tried to reach the shooter, she was swept away by the fleeing crowd.\end{tabular}                                                                              \\ \hline\hline
\href{https://www.newyorker.com/books/flash-fiction/beyond-nature}{Beyond Nature}                           & \begin{tabular}[c]{@{}l@{}}A solitary man walking in a remote mountainous region comes across a car crash, and stays by the side\\ of the lifeless female victim, narrating stories of his past and reflecting on the impermanence of \\events and life itself, while awaiting emergency services amidst the looming presence of wilderness.\end{tabular}                                                                                                                \\ \hline\hline
\href{https://www.newyorker.com/books/flash-fiction/certain-european-movies}{\begin{tabular}[c]{@{}l@{}}Certain European\\ Movies\end{tabular}}                  & \begin{tabular}[c]{@{}l@{}}Two individuals, at a residency together, navigate the complexity of their ephemeral relationship during\\ their final beach trip, framed by misadventures, subtle tensions, unspoken desires, and looming departures.\end{tabular}                                                                                                                                                                                   \\ \hline\hline
\href{https://www.newyorker.com/books/flash-fiction/keys}{Keys}                                     & \begin{tabular}[c]{@{}l@{}}Daniel, struggling with recurring dreams of his ex-wife Rachel and a mysterious unused flat, eventually \\discusses them with his current partner Isabel, sparking various reflections and conversations about their\\ past relationships, until a real-life discovery of old keys triggers a nostalgic memory and helps him find a\\ way to reconnect with his present relationship through canoeing.\end{tabular}                                     \\ \hline\hline
\href{https://www.newyorker.com/books/flash-fiction/listening-for-the-click}{\begin{tabular}[c]{@{}l@{}}Listening For\\ the Click\end{tabular}}                  & \begin{tabular}[c]{@{}l@{}}Navigating a complex social landscape, the protagonist experiences a series of complex relationships \\and emotional turmoil in a student environment, and engages in self-discovery and self-reflection as she\\ interacts with the characters Carl, Martin, Lizzy, and Johan, resulting in a journey of introspection,\\ betrayal, love, and personal growth.\end{tabular}                                                          \\ \hline\hline
\href{https://www.newyorker.com/magazine/2023/05/15/maintenance-hvidovre-fiction-olga-ravn}{\begin{tabular}[c]{@{}l@{}}Maintenance,\\ Hvidovre\end{tabular}}                   & \begin{tabular}[c]{@{}l@{}}A woman experiences a disorienting night in a maternity ward where she encounters other similarly \\disoriented new mothers, leading to an uncanny mix-up where she leaves the hospital with a baby \\that she realizes is not her own, yet accepts the situation with an inexplicable sense of happiness.\end{tabular}                                                                                                  \\ \hline\hline
\href{https://www.newyorker.com/magazine/2022/11/14/returns}{Returns}                                  & \begin{tabular}[c]{@{}l@{}}The narrator visits their elderly mother in her small town, spending a day with her that is filled with \\nostalgia, conversation, and old habits, only to return a month later after her hospitalization due to\\ a sunstroke, finding remnants of their last visit.\end{tabular}                                                                                                                                                                      \\ \hline\hline
\href{https://www.newyorker.com/books/flash-fiction/the-facade-renovation-thats-going-well}{\begin{tabular}[c]{@{}l@{}}The Facade \\Renovation\\ That’s Going Well\end{tabular}} & \begin{tabular}[c]{@{}l@{}}An academic faculty housed in a building with a critical waterproofing layer missing experiences a series\\ of disruptive and problematic construction repairs, causing tension, inconvenience, and health concerns\\ among the tenants, ultimately leading to resignation and endurance in hopes of better future circumstances.\end{tabular}                                                        \\ \hline\hline
\href{https://www.newyorker.com/books/flash-fiction/the-kingdom-that-failed}{\begin{tabular}[c]{@{}l@{}}The Kingdom\\ That Failed\end{tabular}}                  & \begin{tabular}[c]{@{}l@{}}The narrator recounts their college friendship with the seemingly flawless Q, and after a decade apart, \\they accidentally cross paths at a pool, where the narrator anonymously observes Q's failed attempt to \\let down a woman about a work-related issue, demonstrating that Q, too, has his share of difficulties.\end{tabular}                                                                                                \\ \hline\hline
\href{https://www.newyorker.com/magazine/2022/06/13/trash }{Trash}                                    & \begin{tabular}[c]{@{}l@{}}A woman unexpectedly marries the son of a successful, ambitious woman named Miss Emily, finding both \\acceptance and critique from her mother-in-law as she navigates this new relationship and confronts the \\stark contrasts between her former life as a supermarket cashier and her new life as part of a well-off family.\end{tabular}                                                                                                            \\ \hline\hline
\href{https://www.newyorker.com/culture/personal-history/the-last-dance-with-my-dad}{\begin{tabular}[c]{@{}l@{}}The Last Dance\\ with my Dad \end{tabular}}               & \begin{tabular}[c]{@{}l@{}}A young teenager recounts her experiences of fitting into her father's gay lifestyle, highlighted by a\\ seven-day cruise with hundreds of gay men, where she experienced acceptance and connection, had her\\ first genuine interaction with a  boy, and shared a last dance with her terminally ill father.\end{tabular}                                                                                                       \\ \hline
\end{tabular}
\vspace{2ex}
\caption{\label{teststories} Expert-written short stories from the New Yorker along with their human-verified GPT4 generated summary as plots that are included as part of our test data for Creativity Evaluation}
\end{table*}


\subsection{Expert Perception on the TTCW tests}

\begin{figure*}[!ht]
    \centering
     \includegraphics[width=\textwidth]{figures/rel.pdf}
    \caption{\label{relev} Relative Evaluation by Creative Writing Experts within a given group of four stories}
\end{figure*}

\begin{table*}[!ht]
\small
\centering
\begin{tabular}{|l|l|}
\hline
E5 & \begin{tabular}[c]{@{}l@{}}It was a pretty effective rubric! I'm used to being more subjective in my work -- did you like a story? Did it connect with \\you? Did it make sense? Why or why not? It was often challenging to break it down into more regimented segments \\like the rubric asked for -- but I do think that it allowed me to express the subjective feelings in a pretty thorough and\\ structured way!\end{tabular}                                                                                                                                                                 \\ \hline
E3 & \begin{tabular}[c]{@{}l@{}}As for the rubric, I thought it was quite thorough. There were some categories where I would say the story didn’t ``pass,"\\ but which were excellent. This happened often with the categories about multiple points of view, and innovative\\ structure and form. Overall, I think the rubric was helpful in helping me think about the different aspects of storytelling.\end{tabular}                                                                                                                                                                                 \\ \hline
E4 & \begin{tabular}[c]{@{}l@{}}I thought the rubric felt pretty thorough; the only part I felt could be added was that suggestion about consistency in\\ voice \& diction!\end{tabular}                                                                                                                                                         \\ \hline
E2 & \begin{tabular}[c]{@{}l@{}}The rubric seemed great to me! It’s however hard to talk about something like pacing without talking about scene and \\summary, for instance. Or the difference between originality of thought and originality in theme/content—wouldn’t the \\latter make up the former and vice/versa? But it is also comprehensive and I can see the merits of this sort of repetition in\\ teasing out a fuller picture of things\end{tabular} \\ \hline
E1 & \begin{tabular}[c]{@{}l@{}}I thought the rubric was pretty good tbh. I think there is overlap in some of the different elements, like "language \\proficiency \& literary devices" and "originality in thought." it's tricky to use words like "satisfying" and "sophisticated" \\when assessing art, but there's always going to be a great deal of subjectivity in these matters.I think that voice is a crucial \\aspect of high-quality writing that is being overlooked by the rubric, and one that greatly informs how I as a reader\\ evaluate 
and appreciate literary writing.\end{tabular} \\ \hline
\end{tabular}
\vspace{2ex}
\caption{\label{expertfeedbackrubric}Expert perception and feedback on the TTCW tests they conducted as part of our data collection.}
\end{table*}

Since the experts listed in Table \ref{creativeexperts} were not involved in designing the rubric but evaluated several stories based on the rubric we asked them their \textit{overall thought about the rubric and any potentially crucial test we missed out on that they use to discriminate between good and bad writing}.As can be seen in Table \ref{expertfeedbackrubric} in Appendix overall almost every expert agreed on the thorough and effective nature of our rubric. Many of them agreed on the fact that our rubric helped them to think about different aspects of storytelling in a more structured way. One of the difficult things about coming up with a rubric for creativity is ensuring coverage. Even though our rubric covers most aspects of creative writing, some experts such as E1 and E4 emphasized on the utility of \textbf{Consistency of Voice and Diction} as a measurable test. In E4's words \textit{``Inconsistent voice and diction are sometimes/often notable in stories that aren't very good, and when voice \& diction are used beautifully, it enhances a story considerably"}. E1 similarly exclaimed \textit{``One of the most meaningful aspects of high-quality literary writing is voice, which conveys qualities of proficiency, artistry, personality, and identity."}. We hope future work can adapt this as a meaningful test in addition to the tests covered in our rubric. Finally, some of the tests from our rubric can have potential overlaps as pointed out by E2. This is further corroborated by the similar numbers for \textit{Narrative Pacing} and \textit{Scenes vs Exposition} suggesting a strong correlation between the two.
\begin{table*}[!ht]
\small
\centering
% \def\arraystretch{1.3}
\begin{tabular}{|l|l|l|}
\hline
Test & Passing Stories & Failing Stories \\ \hline
\begin{tabular}[c]{@{}l@{}}Originality in\\ Form\end{tabular} & \begin{tabular}[c]{@{}l@{}}Inventive techniques like time jumping, varied \\ perspectives, unconventional punctuation, and\\ delayed revelation of key information\end{tabular} & \begin{tabular}[c]{@{}l@{}}Conventional and linear in its form, language, \\ and narrative, with occasional attempts at \\ innovation that do not significantly contribute to \\ its overall originality or creativity\end{tabular} \\ \hline
\begin{tabular}[c]{@{}l@{}}Originality in\\ Thought\end{tabular} & \begin{tabular}[c]{@{}l@{}}Fresh language, unique plot and characters, subtle\\ emotional resonance, and inventive metaphors. Minor \\ familiar elements, but do not undermine the overall \\ sense of imagination and distinctiveness\end{tabular} & \begin{tabular}[c]{@{}l@{}}Stories relies heavily on cliches \& tired tropes.\\ Language does not feel fresh or original with \\ narrative arc following a predictable trajectory.\\ Metaphors, descriptions, and overall premise \\ cover familiar ground that lacks novelty or nuance\end{tabular} \\ \hline
\begin{tabular}[c]{@{}l@{}}Originality in\\ Theme/Content\end{tabular} & \begin{tabular}[c]{@{}l@{}}Unconventional, dreamlike exploration of emotions\\ such as love and loss, evoking empathy and reflection\\ through its distinct main character perspective, \\ eschewing simplistic meanings for ambiguity, and \\ valuing open-ended questions over singular messages,\\ thus providing a unique reading experience compared\\ to conventional stories.\end{tabular} & \begin{tabular}[c]{@{}l@{}}Disjointed narrative, underdeveloped themes, \\ inconsistent tone, vaguely defined characters, and\\ abrupt context shifts, lack depth and fail to provide \\ substantive insight or originality to the reader.\end{tabular} \\ \hline\hline
\begin{tabular}[c]{@{}l@{}}World Building\\ and Setting\end{tabular} & \begin{tabular}[c]{@{}l@{}}Strategic use of concrete, specific sensory details from\\ a particular character’s perspective balances narrative\\ momentum, making a fictional world feel real, vivid\\ and immersive for readers. Thoughtful depiction of\\ everyday objects, and idiosyncratic elements within\\ narrative and dialogue to balance exposition with \\ vivid scene-setting, creating authenticity and realism \\ that serves the plot and characters\end{tabular} & \begin{tabular}[c]{@{}l@{}}Fictional world is not always convincingly \\established through sensory details and language. \\Stories rely too heavily on overwrought imagery\\ and figurative language without grounding \\the reader in a tangible reality.\end{tabular} \\ \hline
\begin{tabular}[c]{@{}l@{}}Character\\ Development\end{tabular} & \begin{tabular}[c]{@{}l@{}}Fully realized characters with contradictions, \\ motivations, and backstories that make them\\ feel lifelike. Flatter, less developed characters\\ that feel appropriate for the narrative goals \\ and style is not necessarily a weakness\end{tabular} & \begin{tabular}[c]{@{}l@{}}Characters not well rounded. easily resorting to \\stereotypes. Predictable arcs not making them\\memorable. Actions or motivations unclear leading \\to disconnect\end{tabular} \\ \hline
\begin{tabular}[c]{@{}l@{}}Rhetorical\\ Complexity\end{tabular} & \begin{tabular}[c]{@{}l@{}}Rich subtext that emerges through contrasts between\\ characters and settings. Omissions that let readers \\ fill in meaning, metaphors with layered significance, \\ implicit characterization, ambiguous endings, \\ intertextual references, and rich language that goes\\ beyond straightforward storytelling\end{tabular} & \begin{tabular}[c]{@{}l@{}}Straightforward storytelling, lack of complexity,\\ nuance, and ambiguity.The absence of context and \\ ironic tensions, overtelling instead of showing, \\ Use of expository dialogue, blunt symbolism, \\ predictable characters, and definitive endings, all \\of which limit interpretation and undermine\\ their potential resonance\end{tabular} \\ \hline
\end{tabular}
\vspace{2ex}
\caption{\label{expertexpl2}Common themes and issues found in expert explanations for tests focusing on TTCW-Originality and TTCW-Elaboration}
\vspace{-8ex}
\end{table*}

\subsection{Common themes in TTCW of Originality and Elaboration}


Table~\ref{expertexpl2} provides common themes and issues found in the expert explanations for each TTCW in the Originality and Elaboration dimensions.


\subsection{Example LLM-generated and expert-written explanations for a TTCW assessment} 

\begin{table*}[!ht]
\centering
\small
\def\arraystretch{1.35}
\begin{tabular}{|l|l|}
\hline
GPT4 & \begin{tabular}[c]{@{}l@{}}The story incorporates numerous instances of subtext that serve to further develop the characters' internal emotions \\ and relationships, as well as accentuate the atmospheric tension and undertones of longing. The subtext, which is\\ subtly weaved into surface-level interactions, actions, and observations, is eloquently crafted and does not feel \\ forced, enriching both the narrative depth and setting.\\ One key passage with evident subtext is when he is in the ocean, swimming and jumping off the rock, while she\\ observes from the towel. The surface-level interpretation suggests a casual, relaxed beach day. However, the \\ subtext is rife with allusions of emotional undercurrents in their interaction. His unawareness of her going \\ into the water parallels the emotional disconnect between them, insinuating the vast, metaphorical space that \\ separates their respective experiences and understanding of their relationship.\\ Another instance of subtext occurs when he takes a photo of her with her swimsuit top off, indicating his attraction\\ to her. But the meaning deepens when we understand that he will delete the photo before going home, implicitly\\ revealing his guilt or fear of being discovered, and again highlighting the transient, secret nature of their relationship.\\ The line, "As in certain European movies" suggests the fleeting, impermanent element of their relationship and the\\ eventual abruptness and intensity of feelings that often feature in said films. The 'European movies' serve as a \\ metaphor for the unexpected and sudden end of their encounter. \\ In these examples, the subtext does not feel forced but instead flows naturally from the characters and \\ circumstances, subtly conveying deeper meanings that heighten both the narrative tension and emotional depth.\\ \\ So Yes.\end{tabular} \\ \hline
E3   & \begin{tabular}[c]{@{}l@{}}There is rich subtext, as the main character seems continually conflicted about whether she wants to be where she is, \\ doing what she is doing. On the surface, she is carefree, riding to the beach with the guy she met, skipping the ceramics\\ and the museum, and whatever else. And yet, she is unhappy and unsatisfied, longing for a beer, imagining that if their\\ relationship continued they would only hate each other. This tension is maintained throughout the story.\end{tabular}                                                                                                                                                                                                                                                                                                               \\ \hline
E1   & \begin{tabular}[c]{@{}l@{}}This piece has an iceberg of subtext floating underneath it. The entire story is conveyed through the successful \\ integration of subtext and text. The interactions between the protagonist and the man (Did you see me jump of the \\ rock? No, she hadn't.Did he notice she had gone in the water too, that her hair was dripping? No, he hadn't.)convey\\ a profound disconnect that causes the reader to wonder why the protagonist continues to suffer the presence of this\\ man she clearly disdains and seems to view as an incompetent man-child.\end{tabular}
               \\ \hline
E7   & \begin{tabular}[c]{@{}l@{}}Yes!!!!! Again, the idea of the story was fairly simple (the inevitability of age, parting, change), but it was illustrated\\ in a way that felt inspiring re: questioning how these ideas relate and resonate throughout our own lives. It was really \\ beautiful and I was left feeling changed at the end of it :)\end{tabular}                                                                                                                                                                                                                                                            \\ \hline
\end{tabular}
\vspace{2ex}
\caption{\label{llmvsexpertexpl}LLM explanation vs expert explanation for Rhetorical Complexity}
\end{table*}

In Table~\ref{llmvsexpertexpl}, we show examples of explanations that experts wrote in conjunction with a binary TTCW assessment they made on a story, as well as the corresponding LLM-generated explanations.

\subsection{Can non-experts administer TTCW tests?}

Recruiting experts for data annotation purposes is challenging, and costly, and must consider the time constraint put on the experts. Prior work has shown the potential of crowd-sourcing (through platforms such as Amazon Mechanical Turk) and the ability of non-experts to accomplish complex tasks as a crowd \cite{kittur2013future}, when following an appropriate workflow that iterates and validates the work on individual non-experts. Some prior work has even shown the validity of crowd-based feedback for writing tasks \cite{bernstein2010soylent,nebeling2016wearwrite}. 

In this work, we chose to rely on experts for annotation, to maximize the validity of our experiments, and confirm whether experts with domain knowledge would reach satisfying agreement levels when evaluating stories with TTCW. Future work can leverage our open-sourced annotations to explore whether non-experts correlate with experts when performing TTCW evaluation, which could lead to more cost-effective TTCW evaluation.

\subsection{Prompts for TTCW} \label{allprompts}

All the instructions shown to creative writing experts and LLMs are given in the tables below.


\begin{table*}[!ht]
\centering
\small
\begin{tabular}{|l|l|}
\hline
\begin{tabular}[c]{@{}l@{}}Expert \\ Measure\end{tabular}               & Does the manipulation of time in terms of compression or stretching feel appropriate and balanced?                                                                                                                                    \\ \hline
\begin{tabular}[c]{@{}l@{}}Expanded\\ Expert\\ Measure (M)\end{tabular} & \begin{tabular}[c]{@{}l@{}}`Compression/stretching of time' in fiction writing, also known as pacing, refers to the manipulation of time in \\storytelling for dramatic effect, pacing, or other narrative purposes. Essentially, it's about controlling the perceived \\speed and rhythm at which a story unfolds.\\ \\

Compression of time refers to when events that take a long time (hours, days, weeks, or even years) are summarized \\or condensed into a brief narrative span. For example, a writer might compress several years of a character's life \\into a few paragraphs to quickly convey important changes or developments.\\ \\

On the other hand, stretching of time is when a brief moment or event is drawn out over pages or chapters. It's often \\used to create suspense, emphasize details, or delve deeper into a character's thoughts and feelings. For example, \\the few seconds it takes for a dropped glass to hit the floor might be stretched out with detailed descriptions of the\\ action, reactions, and thoughts of characters involved.\\ \\

Storytime refers to the time within the world of the story, while real-world time refers to the time it takes for the \\reader to read the story. A skilled writer can manipulate the relationship between these two to affect the pacing of \\the narrative, either speeding it up (compression) or slowing it down (stretching). This technique plays a crucial role \\in shaping the reader's experience and engagement with the story.\end{tabular} \\ \hline
\begin{tabular}[c]{@{}l@{}}Human\\ Instruction\end{tabular}             & \begin{tabular}[c]{@{}l@{}}\{\{M\}\}\\ \\ Based on the story that you just read, answer the following question.\\ \textit{\color{blue}Does the manipulation of time in terms of compression or stretching feel appropriate and balanced?}\\ -Yes \\ -No \\\\ Reasoning : \end{tabular}                                                                       \\ \hline
\begin{tabular}[c]{@{}l@{}}LLM\\ Instruction\end{tabular}               & \begin{tabular}[c]{@{}l@{}}\{\{M\}\}\\ \\ Given the story above, list out the scenes in the story in which time compression or time stretching is used, and \\argue for each whether it is successfully implemented.  Then overall, give your reasoning about the question below \\and give an answer to it between 'Yes' or 'No' only \\ \\ \textit{\color{blue} Q) Does the manipulation of time in terms of compression or stretching feel appropriate and balanced?}\end{tabular}                                                                                                                                                                                                                    \\ \hline
\end{tabular}
\vspace{2ex}
\caption{\label{prompting}TTCW Fluency1 (Narrative Pacing) }
\vspace{-5ex}
\end{table*}


% ==================================================





\begin{table*}[!ht]
\centering
\small
% \def\arraystretch{1.15}
\begin{tabular}{|l|l|}
\hline
\begin{tabular}[c]{@{}l@{}}Expert \\ Measure\end{tabular}               & \begin{tabular}[c]{@{}l@{}}Does the story have an appropriate balance between scene and summary/exposition or it relies on one\\ of the elements heavily compared to the other?  \end{tabular}                                                                                                                                  \\ \hline
\begin{tabular}[c]{@{}l@{}}Expanded\\ Expert\\ Measure (M)\end{tabular} & \begin{tabular}[c]{@{}l@{}}'Scene' and 'summary/exposition' are two crucial elements of narrative storytelling, and balancing them \\appropriately is an important skill in fiction writing.\\ \\ 

A 'scene' is a moment in the story that is dramatized in real-time. Scenes are usually vivid and engaging, often \\featuring character interaction, dialogue, and action. They are the building blocks of the plot, and through them, \\the story unfolds.\\ \\ 

'Summary' or 'exposition', on the other hand, involves summarizing events or providing information. Instead of \\unfolding in real time, \\summaries compress time and tell the reader what happened. Exposition provides \\necessary background information, like character history, setting details, or prior events. \\ \\ 

A good writer knows when to use scenes to make the story come alive, show character development, or increase \\tension. They also know when to use summary or exposition to move the story forward, fill in background \\information, or bridge gaps between important scenes. \\ \\ 

The right balance between scene and summary/exposition can vary depending on the story, but in general, it's \\essential for maintaining a good pace, keeping the reader engaged, and delivering necessary information. \\A story with too many scenes and not enough summary might feel overwhelming or slow, while a story with \\too much exposition and not enough scenes could feel dry and unengaging.\end{tabular} \\ \hline
\begin{tabular}[c]{@{}l@{}}Human\\ Instruction\end{tabular}             & \begin{tabular}[c]{@{}l@{}}\{\{M\}\}\\ \\ Based on the story that you just read, answer the following question.\\ \textit{\color{blue} Does the story have an appropriate balance between scene and summary/exposition or it relies on one of the elements} \\\textit{\color{blue}heavily compared to the other?} \\ -Yes \\ -No \\\\ Reasoning : \end{tabular}    
\\ \hline
\begin{tabular}[c]{@{}l@{}}LLM\\ Instruction\end{tabular}               & \begin{tabular}[c]{@{}l@{}}\{\{M\}\}\\ \\ Given the story above, answer the following question. Please first explain your reasoning step by step \\and then given an answer between 'Yes' or 'No' only \\ \\ \textit{\color{blue} Does the story have an appropriate balance between scene and summary/exposition or it relies on one of the elements} \\\textit{\color{blue}heavily compared to the other?}\end{tabular}                                                                                                                                                                                                                    \\ \hline
\end{tabular}
\vspace{2ex}
\caption{\label{prompting}TTCW Fluency2 (Scene vs Exposition) }
\vspace{-5ex}
\end{table*}


% ==================================================


\begin{table*}[!ht]
\centering
\small
% \def\arraystretch{1.15}
\begin{tabular}{|l|l|}
\hline
\begin{tabular}[c]{@{}l@{}}Expert \\ Measure\end{tabular}               & Does the story make sophisticated use of idiom or metaphor or literary allusion?                                                                                                                                     \\ \hline
\begin{tabular}[c]{@{}l@{}}Expanded\\ Expert\\ Measure (M)\end{tabular} & \begin{tabular}[c]{@{}l@{}}`Idiom' refers to phrases or expressions that have a figurative, or sometimes literal, meaning that is \\comprehensible to a particular group of people. These can be cultural, regional, or specific to a certain group or \\profession.Sophisticated use of idiom suggests that the writer is skillfully using these expressions to lend \\authenticity to character voices or to convey specific meanings in a concise way.\\\\

`Metaphor' is a figure of speech that describes an object or action in a way that isn't literally true, but helps explain\\ an idea or make a comparison. Sophisticated use of metaphor suggests the
writer could create impactful, original \\comparisons that reveal deeper insights about themes,
characters, or situations in the story.\\\\

`Literary allusion' refers to a brief and indirect reference to a person, place, thing or idea of
historical, cultural,\\ literary, or political significance. It does not describe in detail the person or thing to which it refers. A sophisticated\\ use of literary allusion implies the writer can effectively incorporate these references to enhance the depth\\ and resonance of the story. They can provide contextual richness, evoke a specific tone, or draw parallels between\\ the narrative and the work alluded to.\\\\

Overall, when a writer uses these techniques well, they add depth, interest, and nuanced \\meaning
to their work. It allows for a richer reading experience, where the literal events are \\imbued with deeper symbolic or thematic significance.\end{tabular} \\ \hline
\begin{tabular}[c]{@{}l@{}}Human\\ Instruction\end{tabular}             & \begin{tabular}[c]{@{}l@{}}\{\{M\}\}\\ \\ Based on the story that you just read, answer the following question.\\ \textit{\color{blue}Does the story make sophisticated use of idiom or metaphor or literary allusion?}\\ -Yes \\ -No \\\\ Reasoning: \end{tabular}                                                                       \\ \hline
\begin{tabular}[c]{@{}l@{}}LLM\\ Instruction\end{tabular}               & \begin{tabular}[c]{@{}l@{}}\{\{M\}\}\\ \\ Given the story above, please list out all the metaphors, idioms and literary allusions, and for each decide \\whether it is successful vs it feels forced or too easy.  Then overall, give your reasoning about the question \\below and give an answer to it between 'Yes' or 'No' only\\ \\ \textit{\color{blue} Q) Does the story make sophisticated use of idiom or metaphor or literary allusion?}\end{tabular}                                                                                                                                                                                                                    \\ \hline
\end{tabular}
\vspace{2ex}
\caption{\label{prompting}TTCW Fluency3 (Language Proficiency \& Literary Devices) }
\vspace{-5ex}
\end{table*}


% ==================================================



\begin{table*}[!ht]
\centering
\small
% \def\arraystretch{1.15}
\begin{tabular}{|l|l|}
\hline
\begin{tabular}[c]{@{}l@{}}Expert \\ Measure\end{tabular}               & Does the end of the story feel natural and earned, as opposed to arbitrary or abrupt?                                                                                                                                    \\ \hline
\begin{tabular}[c]{@{}l@{}}Expanded\\ Expert\\ Measure (M)\end{tabular} & \begin{tabular}[c]{@{}l@{}}If the writer ends the piece simply because they are 'tired of writing', the conclusion might feel abrupt, disjointed, \\or unfulfilling to the reader. It suggests a rushed ending, where plot threads might be left unresolved and character \\arcs incomplete.\\ \\ 

Conversely, if the writer concludes because they've reached `the moment the entire piece has been leading readers \\towards', it implies a well-considered and purposeful ending. The events, character development, and themes \\throughout the story have built towards this climactic moment, providing a satisfying resolution to the reader.\\ \\ 

A strong ending offers a sense of closure, ties up the central conflicts or questions of the story, and generally \\leaves the reader feeling that the narrative journey was worthwhile and complete.\end{tabular} \\ \hline
\begin{tabular}[c]{@{}l@{}}Human\\ Instruction\end{tabular}             & \begin{tabular}[c]{@{}l@{}}\{\{M\}\}\\ \\ Based on the story that you just read, answer the following question.\\ \textit{\color{blue}Does the end of the story feel natural and earned, as opposed to arbitrary or abrupt?}\\ -Yes \\ -No \\\\ Reasoning : \end{tabular}                                                                       \\ \hline
\begin{tabular}[c]{@{}l@{}}LLM\\ Instruction\end{tabular}               & \begin{tabular}[c]{@{}l@{}}\{\{M\}\}\\ \\ Given the story above, answer the following question. Please first explain your reasoning step by step \\ and then given an answer between 'Yes' or 'No' only\\ \\ \textit{\color{blue} Q) Does the end of the story feel natural and earned, as opposed to arbitrary or abrupt?}\end{tabular}                                                                                                                                                                                                                    \\ \hline
\end{tabular}
\vspace{2ex}
\caption{\label{prompting}TTCW Fluency4 (Narrative Ending) }
\vspace{-5ex}
\end{table*}



% ==================================================



\begin{table*}[!ht]
\centering
\small
% \def\arraystretch{1.15}
\begin{tabular}{|l|l|}
\hline
\begin{tabular}[c]{@{}l@{}}Expert \\ Measure\end{tabular}               & Do the different elements of the story work together to form a unified, engaging, and satisfying whole?                                                                                                                                     \\ \hline
\begin{tabular}[c]{@{}l@{}}Expanded\\ Expert\\ Measure (M)\end{tabular} & \begin{tabular}[c]{@{}l@{}}A well-crafted story usually follows a logical path, where the events in the beginning set up the middle, which then\\ logically leads to the end. Every scene, character action, and piece of dialogue should serve the story and propel it \\forward. Well-written stories have an underlying the unity that binds the elements together. The themes, plotlines, \\character arcs, and other elements of the story interweave to create a harmonious whole. A story with 'disorder'\\ might feel disjointed, with characters, scenes, etc that don't connect or contribute to the overall narrative.\end{tabular} \\ \hline
\begin{tabular}[c]{@{}l@{}}Human\\ Instruction\end{tabular}             & \begin{tabular}[c]{@{}l@{}}\{\{M\}\}\\ \\ Based on the story that you just read, answer the following question.\\ \textit{\color{blue}Do the different elements of the story work together to form a unified, engaging, and satisfying whole?}\\ -Yes \\ -No \\\\ Reasoning : \end{tabular}                                                                       \\ \hline
\begin{tabular}[c]{@{}l@{}}LLM\\ Instruction\end{tabular}               & \begin{tabular}[c]{@{}l@{}}\{\{M\}\}\\ \\ Given the story above, answer the following question. Please first explain your reasoning step by step and then \\give an answer between 'Yes' or 'No' only\\ \\ \textit{\color{blue} Q) Do the different elements of the story work together to form a unified, engaging, and satisfying whole?}\end{tabular}                                                                                                                                                                                                                                 \\ \hline
\end{tabular}
\vspace{2ex}
\caption{\label{prompting}TTCW Fluency5 (Understandability \& Coherence) }
\vspace{-5ex}
\end{table*}


% ==================================================



\begin{table*}[!ht]
\centering
\small
% \def\arraystretch{1.15}
\begin{tabular}{|l|l|}
\hline
\begin{tabular}[c]{@{}l@{}}Expert \\ Measure\end{tabular}               & \begin{tabular}[c]{@{}l@{}}Does the story provide diverse perspectives, and if there are unlikeable characters, are their perspectives \\presented convincingly and accurately? \end{tabular}                                                                                                                                     \\ \hline
\begin{tabular}[c]{@{}l@{}}Expanded\\ Expert\\ Measure (M)\end{tabular} & \begin{tabular}[c]{@{}l@{}}A good writer can convincingly and accurately depict a wide range of character viewpoints, including those of\\ characters who may be morally ambiguous, difficult, or otherwise unappealing.\\ \\ 

This can involve diving into the mindset of characters who may act or think in ways that the reader, or even \\the writer, finds objectionable or repugnant. It involves understanding their motivations, their beliefs, and the \\reasons behind their actions, and then conveying these elements in a way that is believable and consistent.\\ \\ 

The purpose of doing so is not to justify or endorse these perspectives, but rather to create complex, three-\\dimensional characters who contribute to the richness and depth of the story. This can also serve to \\challenge the reader, provoke thought, and provide insights into different aspects of the human experience.\end{tabular} \\ \hline
\begin{tabular}[c]{@{}l@{}}Human\\ Instruction\end{tabular}             & \begin{tabular}[c]{@{}l@{}}\{\{M\}\}\\ \\ Based on the story that you just read, answer the following question.\\ \textit{\color{blue}Does the story provide diverse perspectives, and if there are unlikeable characters, are their perspectives presented} \\ \textit{\color{blue}convincingly and accurately?}\\ -Yes \\ -No \\\\ Reasoning : \end{tabular}                                                                       \\ \hline
\begin{tabular}[c]{@{}l@{}}LLM\\ Instruction\end{tabular}               & \begin{tabular}[c]{@{}l@{}}\{\{M\}\}\\ \\ Given the story above, answer the following question. Please first explain your reasoning step by step and then \\give an answer between 'Yes' or 'No' only\\ \\ \textit{\color{blue} Q) Does the story provide diverse perspectives, and if there are unlikeable characters, are their perspectives presented}\\\textit{\color{blue} convincingly and accurately?}\end{tabular}                                                                                                                                                                                                                                 \\ \hline
\end{tabular}
\vspace{2ex}
\caption{\label{prompting}TTCW Flexibility1 (Perspective \& Voice Flexibility) }
\vspace{-5ex}
\end{table*}


% ==================================================




\begin{table*}[!ht]
\centering
\small
% \def\arraystretch{1.15}
\begin{tabular}{|l|l|}
\hline
\begin{tabular}[c]{@{}l@{}}Expert \\ Measure\end{tabular}               & \begin{tabular}[c]{@{}l@{}}Does the story achieve a good balance between interiority and exteriority, in a way that feels \\emotionally flexible? \end{tabular}                                                                                                                                     \\ \hline
\begin{tabular}[c]{@{}l@{}}Expanded\\ Expert\\ Measure (M)\end{tabular} & \begin{tabular}[c]{@{}l@{}}`Emotional flexibility' is asking whether the piece of writing effectively balances action and introspection, and \\if it portrays a broad and realistic spectrum of emotions.\\ \\

`Exteriority' refers to the observable actions, behaviors, or dialogue of a character, and the physical or visible \\aspects of the setting, plot, and conflicts.\\ \\

`Interiority', on the other hand, pertains to the inner life of a character — their thoughts, feelings, memories, \\and subjective experiences.\\ \\

A balance between these two aspects is crucial in creating well-rounded characters and compelling narratives. \\If a piece is too heavy on exteriority, it may feel shallow or lack emotional depth. If it leans too much on\\ interiority, it could become overly introspective and potentially lose the momentum of the plot.
\end{tabular} \\ \hline
\begin{tabular}[c]{@{}l@{}}Human\\ Instruction\end{tabular}             & \begin{tabular}[c]{@{}l@{}}\{\{M\}\}\\ \\ Based on the story that you just read, answer the following question.\\ \textit{\color{blue}Does the story achieve a good balance between interiority and exteriority, in a way that feels emotionally flexible?}\\ -Yes \\ -No \\\\ Reasoning : \end{tabular}                                                                       \\ \hline
\begin{tabular}[c]{@{}l@{}}LLM\\ Instruction\end{tabular}               & \begin{tabular}[c]{@{}l@{}}\{\{M\}\}\\ \\ Given the story above, answer the following question. Please first explain your reasoning step by step and \\then give an answer between 'Yes' or 'No' only\\ \\ \textit{\color{blue}Q) Does the story achieve a good balance between interiority and exteriority, in a way that feels} \\\textit{\color{blue}emotionally flexible?}\end{tabular}                                                                                                                                                                                                                                 \\ \hline
\end{tabular}
\vspace{2ex}
\caption{\label{prompting}TTCW Flexibility2 (Emotional Flexibility) }
\vspace{-5ex}
\end{table*}


% ==================================================




\begin{table*}[!ht]
\centering
\small
% \def\arraystretch{1.15}
\begin{tabular}{|l|l|}
\hline
\begin{tabular}[c]{@{}l@{}}Expert \\ Measure\end{tabular}               & \begin{tabular}[c]{@{}l@{}}Does the story contain turns that are both surprising and appropriate? \end{tabular}                                                                                                                                     \\ \hline
\begin{tabular}[c]{@{}l@{}}Expanded\\ Expert\\ Measure (M)\end{tabular} & \begin{tabular}[c]{@{}l@{}}`Surprising': This refers to the element of unpredictability in a narrative. A good story often has plot twists, \\character developments, or thematic revelations that surprise the reader, subverting their expectations in a \\thrilling way.It's about keeping readers engaged and curious, never fully knowing what's going to happen next.\\ \\ 

`Appropriate': Despite the surprises and twists, the turns in the story must also make sense within the established \\context of the story's universe, its characters, and its themes. This means that even though an event might be \\surprising, it should feel appropriate or fitting in hindsight. It shouldn't feel like the writer has broken the rules \\they've set up, or made a character behave inconsistently without reason, simply for the sake of shock value.\\ \\ 

So when someone wonders if a writer can make turns that are 'both surprising and appropriate', they're asking \\if the writer can strike this balance between unexpectedness and coherence, keeping the reader on their toes \\while maintaining a believable, satisfying narrative. \end{tabular} \\ \hline
\begin{tabular}[c]{@{}l@{}}Human\\ Instruction\end{tabular}             & \begin{tabular}[c]{@{}l@{}}\{\{M\}\}\\ \\ Based on the story that you just read, answer the following question.\\ \textit{\color{blue}Does the story contain turns that are both surprising and appropriate?}\\ -Yes \\ -No \\\\ Reasoning: \end{tabular}                                                                       \\ \hline
\begin{tabular}[c]{@{}l@{}}LLM\\ Instruction\end{tabular}               & \begin{tabular}[c]{@{}l@{}}\{\{M\}\}\\ \\ Given the story above, list each element in the story that is intended to be surprising. For each, decide whether the\\ surprising element remains appropriate with respect to the entire story. Then overall, give your reasoning \\about the question below and give an answer to it between 'Yes' or 'No' only\\ \\ \textit{\color{blue} Q) Does the story contain turns that are both surprising and appropriate?}\end{tabular}                                                                                                                                                                                                                                 \\ \hline
\end{tabular}
\vspace{2ex}
\caption{\label{prompting}TTCW Flexibility3 (Structural Flexibility) }
\vspace{-5ex}
\end{table*}


% ==================================================






\begin{table*}[!ht]
\centering
\small
% \def\arraystretch{1.15}
\begin{tabular}{|l|l|}
\hline
\begin{tabular}[c]{@{}l@{}}Expert \\ Measure\end{tabular}               & \begin{tabular}[c]{@{}l@{}}Will an average reader of this story obtain a unique and original idea from reading it? \end{tabular}                                                                                                                                     \\ \hline
\begin{tabular}[c]{@{}l@{}}Expanded\\ Expert\\ Measure (M)\end{tabular} & \begin{tabular}[c]{@{}l@{}}If a story is good, the reader gains new insights, perspectives, or knowledge from it. This doesn't necessarily\\ mean factual information but could relate to a deeper understanding of human nature, cultural insights,\\ unique viewpoints, or even the exploration of new ideas and themes. Essentially, it's about what\\ the reader takes away from the story beyond just the plot.\\ \\ 

A good story has lasting impacts on its readers and the society. It is meant to entertain, inform, provoke \\thought, challenge beliefs, provide comfort, or raise awareness on specific issues.
 \end{tabular} \\ \hline
\begin{tabular}[c]{@{}l@{}}Human\\ Instruction\end{tabular}             & \begin{tabular}[c]{@{}l@{}}\{\{M\}\}\\ \\ Based on the story that you just read, answer the following question.\\ \textit{\color{blue}Will an average reader of this story obtain a unique and original idea from reading it?}\\ -Yes \\ -No \\\\ Reasoning : \end{tabular}                                                                       \\ \hline
\begin{tabular}[c]{@{}l@{}}LLM\\ Instruction\end{tabular}               & \begin{tabular}[c]{@{}l@{}}\{\{M\}\}\\ \\ Given the story above, list out elements that are unique takeaways of this story for the reader. Then overall, \\give your reasoning about the question below and give an answer to it between 'Yes' or 'No' only\\ \\ \textit{\color{blue} Q) Will an average reader of this story obtain a unique and original idea from reading it?}\end{tabular}                                                                                                                                                                                                                                 \\ \hline
\end{tabular}
\vspace{2ex}
\caption{\label{prompting}TTCW Originality1 (Originality in Theme and Content) }
\vspace{-3ex}
\end{table*}


% ==================================================








\begin{table*}[!ht]
\centering
\small
% \def\arraystretch{1.15}
\begin{tabular}{|l|l|}
\hline
\begin{tabular}[c]{@{}l@{}}Expert \\ Measure\end{tabular}               & \begin{tabular}[c]{@{}l@{}}Is the story an original piece of writing without any cliches?\end{tabular}                                                                                                                                     \\ \hline
\begin{tabular}[c]{@{}l@{}}Expanded\\ Expert\\ Measure (M)\end{tabular} & \begin{tabular}[c]{@{}l@{}}A cliche is an idea, expression, character, or plot that has been overused to the point of losing its original \\meaning or impact. They often become predictable and uninteresting for the reader. Originality suggests\\ that the piece isn't cliche.

 \end{tabular} \\ \hline
\begin{tabular}[c]{@{}l@{}}Human\\ Instruction\end{tabular}             & \begin{tabular}[c]{@{}l@{}}\{\{M\}\}\\ \\ Based on the story that you just read, answer the following question.\\ \textit{\color{blue}Is the story an original piece of writing without any cliches?}\\ -Yes \\ -No \\\\ Reasoning: \end{tabular}                                                                       \\ \hline
\begin{tabular}[c]{@{}l@{}}LLM\\ Instruction\end{tabular}               & \begin{tabular}[c]{@{}l@{}}\{\{M\}\}\\ \\ Given the story above, are there any cliches in the story? If so, list out all the elements in this story that \\are cliche. Then overall, give your reasoning if the piece is negatively impacted by the cliches and give \\an answer to the question below between 'Yes' or 'No' only\\ \\ \textit{\color{blue} Q) Is the story an original piece of writing without any cliches?}\end{tabular}                                                                                                                                                                                                                                 \\ \hline
\end{tabular}
\vspace{2ex}
\caption{\label{prompting}TTCW Originality2 (Originality in Thought) }
\vspace{-5ex}
\end{table*}


% ==================================================




\begin{table*}[!ht]
\centering
\small
% \def\arraystretch{1.15}
\begin{tabular}{|l|l|}
\hline
\begin{tabular}[c]{@{}l@{}}Expert \\ Measure\end{tabular}               & \begin{tabular}[c]{@{}l@{}}Does the story show originality in its form?\end{tabular}                                                                                                                                     \\ \hline
\begin{tabular}[c]{@{}l@{}}Expanded\\ Expert\\ Measure (M)\end{tabular} & \begin{tabular}[c]{@{}l@{}}When someone says that a piece of fiction 'shows an innovative use of form/structure', they're referring to\\ how the writer has chosen to shape and organize the story in an unusual, original, or inventive way. This could \\involve a variety of different elements, including:\\ \\ 

Narrative Structure: This could include unconventional timelines (e.g. a non-linear story, a story told in reverse)\\, multiple perspectives or narrators, or unusual narrative voices (e.g. a story told from the perspective of an \\inanimate object).\\ \\ 

Format: This could be something as simple as using unconventional punctuation or capitalization, or as complex \\as telling a story through a series of letters, diary entries, newspaper clippings, or other documents. In recent years,\\ some authors have even experimented with using social media posts or text messages as a form of narrative structure.\\ \\ 

Genre Hybridity: Combining elements from different genres or sub-genres in unexpected ways can also be seen\\ as an innovative use of form such as Horror-Mystery or Comic Fantasy.\\ \\ 

Plot Structure: Deviating from traditional plot structures such as three-act structure, or following them in unexpected\\ ways.For example, telling a story without a clear resolution, incorporating multiple climaxes or using revelation as a \\device where a surprising, and often shocking, development occurs that was previously kept hidden from the \\characters and/or the audience. It's typically designed to provide new context for interpreting what has previously \\occurred in the story. \\ \\ 

Language and Style: Innovative use of form can also come in the form of unique use of language, style, or \\even typography, such as concrete poetry or writing that visually represents its subject matter on the page.\\ \\ 

The goal of this innovation is often to provide a fresh reader experience, challenge conventional reading\\ expectations, or to create a deeper or more complex exploration of the story's themes.

 \end{tabular} \\ \hline
\begin{tabular}[c]{@{}l@{}}Human\\ Instruction\end{tabular}             & \begin{tabular}[c]{@{}l@{}}\{\{M\}\}\\ \\ Based on the story that you just read, answer the following question.\\ \textit{\color{blue}Does the story show originality in its form?}\\ -Yes \\ -No \\\\ Reasoning: \end{tabular}                                                                       \\ \hline
\begin{tabular}[c]{@{}l@{}}LLM\\ Instruction\end{tabular}               & \begin{tabular}[c]{@{}l@{}}\{\{M\}\}\\ \\ Given the story and the devices mentioned above, list each device used with a short explanation of whether it is \\successful or not. Then overall, give your reasoning about the question below and give an answer to it\\ between 'Yes' or 'No' only\\ \\ \textit{\color{blue} Q) Does the story show originality in its form?}\end{tabular}                                                                                                                                                                                                                                 \\ \hline
\end{tabular}
\vspace{2ex}
\caption{\label{prompting}TTCW Originality3 (Originality in Form) }
\vspace{-5ex}
\end{table*}


% ==================================================




\begin{table*}[!ht]
\centering
\small
% \def\arraystretch{1.15}
\begin{tabular}{|l|l|}
\hline
\begin{tabular}[c]{@{}l@{}}Expert \\ Measure\end{tabular}               & \begin{tabular}[c]{@{}l@{}}Does each character in the story feel developed at the appropriate complexity level, ensuring that no character \\feels like they are present simply to satisfy a plot requirement?\end{tabular}                                                                                                                                     \\ \hline
\begin{tabular}[c]{@{}l@{}}Expanded\\ Expert\\ Measure (M)\end{tabular} & \begin{tabular}[c]{@{}l@{}} A `flat character' is typically a minor character who is not thoroughly developed or who does not undergo \\significant change or growth throughout the story. They often embody or represent a single trait or idea, \\and they're used to advance the plot or highlight certain qualities in other characters.\\ \\ 

A `complex character', also known as a round character, has depth in feelings and passions, has a variety \\of traits of a real human being, and evolves over time. They have their strengths, weaknesses, \\and they learn from their experiences. They tend to be more engaging to the reader, as they mirror \\the complexity of real people.\\ \\ 

In good stories, authors take a character who initially appears to be one-dimensional or stereotypical (flat) and \\add depth to them. This could be done by revealing more about their backstory, introducing unexpected traits \\or motivations, or having them grow and change in response to the events of the story. \\This transformation from a flat to a complex character can make the narrative more engaging and believable.
 \end{tabular} \\ \hline
\begin{tabular}[c]{@{}l@{}}Human\\ Instruction\end{tabular}             & \begin{tabular}[c]{@{}l@{}}\{\{M\}\}\\ \\ Based on the story that you just read, answer the following question.\\  \textit{\color{blue} Q) Does each character in the story feel developed at the appropriate complexity level, ensuring that no character} \\ \textit{\color{blue}feels like they are present simply to satisfy a plot requirement?}\\ -Yes \\ -No \\\\ Reasoning: \end{tabular}                                                                       \\ \hline
\begin{tabular}[c]{@{}l@{}}LLM\\ Instruction\end{tabular}               & \begin{tabular}[c]{@{}l@{}}\{\{M\}\}\\ \\ Given the story above, list each character and the level of development. Then overall, give your reasoning \\about the question below and give an answer to it between 'Yes' or 'No' only\\ \\ 
 \textit{\color{blue} Q) Does each character in the story feel developed at the appropriate complexity level, ensuring that no character} \\ \textit{\color{blue}feels like they are present simply to satisfy a plot requirement?}\end{tabular}                                                                                                                                                                                                                                 \\ \hline
\end{tabular}
\vspace{2ex}
\caption{\label{prompting}TTCW Elaboration2 (Character Development) }
\vspace{-5ex}
\end{table*}


% ==================================================



\begin{table*}[!ht]
\centering
\small
% \def\arraystretch{1.15}
\begin{tabular}{|l|l|}
\hline
\begin{tabular}[c]{@{}l@{}}Expert \\ Measure\end{tabular}               & \begin{tabular}[c]{@{}l@{}}Are there passages in the story that involve subtext and when there is subtext, does it enrich the story's setting \\or does it feel forced?\end{tabular}                                                                                                                                     \\ \hline
\begin{tabular}[c]{@{}l@{}}Expanded\\ Expert\\ Measure (M)\end{tabular} & \begin{tabular}[c]{@{}l@{}} `Surface' level: This is the most apparent and straightforward level of a story. It includes the visible actions, \\explicit dialogue, and clear descriptions. This is what literally happens in the plot: the characters' actions, events, \\and the apparent consequences.\\ \\ 

`Subtext' level: This is the underlying or implicit meaning that isn't directly stated but can be inferred from \\the characters'  actions, dialogue, and other elements of the story. Subtext often reveals deeper truths about \\characters, themes, or the overall message of the piece. It could be a hidden motive, an unstated\\ emotion, a cultural commentary, or a symbolic meaning.\\ \\ 

For example, in a conversation between two characters, the surface text might be polite and cordial, but the \\subtext \\discerned from the characters' nonverbal cues, previous interactions, or the context of their conversation\\ — could suggest tension or hostility.\\ \\ 

Effective fiction often operates on both levels. The surface text keeps the reader engaged with the plot and \\characters, while the subtext provides depth, complexity, and additional layers of interpretation, \\contributing to a richer and more rewarding reading experience.
 \end{tabular} \\ \hline
\begin{tabular}[c]{@{}l@{}}Human\\ Instruction\end{tabular}             & \begin{tabular}[c]{@{}l@{}}\{\{M\}\}\\ \\ Based on the story that you just read, answer the following question.\\  \textit{\color{blue} Q) Are there passages in the story that involve subtext and when there is subtext, does it enrich the story's setting} \\ \textit{\color{blue} or does it feel forced?}\\ -Yes \\ -No \\\\ Reasoning: \end{tabular}                                                                       \\ \hline
\begin{tabular}[c]{@{}l@{}}LLM\\ Instruction\end{tabular}               & \begin{tabular}[c]{@{}l@{}}\{\{M\}\}\\ \\ Given the story above, answer the following question. Please first explain your reasoning step by step \\and then give an answer between 'Yes' or 'No' only\\ \\ 
 \textit{\color{blue} Q)Are there passages in the story that involve subtext and when there is subtext, does it enrich the story's setting} \\ \textit{\color{blue} or does it feel forced?}\end{tabular}                                                                                                                                                                                                                                 \\ \hline
\end{tabular}
\vspace{2ex}
\caption{\label{prompting}TTCW Elaboration3 (Rhetorical Complexity) }
\vspace{-5ex}
\end{table*}


% ==================================================


\end{document}
