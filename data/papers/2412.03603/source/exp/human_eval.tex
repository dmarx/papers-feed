\subsection{Comparison with SOTA Models}
To evaluate the performance of \nameofmethod{}, we selected five strong baselines from closed-source video generation models. In total, we utilized 1,533 text prompts, generating an equal number of video samples with \nameofmethod{} in a single run. For a fair comparison, we conducted inference only once, avoiding any cherry-picking of results. When comparing with the baseline methods, we maintained the default settings for all selected models, ensuring consistent video resolution. 60 professional evaluators performed the evaluation and the results are presented in Table \ref{tab:compare}. Videos were assessed based on three criteria: Text Alignment, Motion Quality, and Visual Quality. Notably, \nameofmethod{} demonstrated the best overall performance, particularly excelling in motion quality. We randomly sample 600 videos out of 1533 for public access\footnote{https://github.com/Tencent/HunyuanVideo}.
\begin{table}[h]
    \centering
    \footnotesize
    \caption{Model Performance Evaluation}
    \begin{tabular}{@{}llccccc@{}}
        \toprule
        Model Name                             &  Duration & Text Alignment  & Motion Quality  & Visual Quality  & Overall  & Ranking \\ \midrule
        \nameofmethod{} (Ours)               & 5s       & 61.8\%               & 66.5\%          & 95.7\%          & 41.3\%      & 1              \\
        CNTopA (API)       & 5s       & 62.6\%               & 61.7\%          & 95.6\%          & 37.7\%      & 2              \\
        CNTopB (Web)          & 5s       & 60.1\%               & 62.9\%          & 97.7\%          & 37.5\%      & 3              \\
        GEN-3 alpha (Web)           & 6s       & 47.7\%               & 54.7\%          & 97.5\%          & 27.4\%      & 4              \\
        Luma1.6 (API)                & 5s       & 57.6\%               & 44.2\%          & 94.1\%          & 24.8\%      & 5              \\
        CNTopC (Web)        & 5s       & 48.4\%               & 47.2\%          & 96.3\%          & 24.6\%      & 6              \\
         \bottomrule
    \end{tabular}
    \label{tab:compare}

\end{table}


