\section{The OneGraph Model}
\label{sec:data-model}
\begin{table}[]
    \centering
    \resizebox{\textwidth}{!}{
    \begin{tabular}{l|l}
    \toprule
     & Interpretation \\
     \midrule
     Object & The things included in the OneGraph data, such as named entities. \\
     Head Object (Head) & The first element of a triple.  \\
     Relation Object (Relation) & The second element of a triple. \\
     Tail Object (Tail) & The third element of a triple. \\
     Name & The name of an entity. \\
     Description &  \\
     Document & \\
     \bottomrule
    \end{tabular}
    }
    \caption{Vocabulary.}
    \label{tab:vocabulary}
\end{table}
\subsection{Structure}
\paragraph{All objects are represented as text.} In the OneGraph model, all things including beings (e.g. people, organization, concepts) and relations are represented as text. For example, $\texttt{Albert Einstein}$ denotes the famous scientist Albert Einstein. $\texttt{has won prize}$ denotes a relation that some one has won a prize. 

\paragraph{Two text can have relations, composing a triple.} For example, the following triple
$$ \texttt{Albert Einstein} \sep \texttt{has won prize} \sep \texttt{Nobel Prize}  $$
denotes the fact that Albert Einstein has won the Nobel Prize. We use $\square$ to separate the text, which is less likely to be used in a normal natural language text. 

\paragraph{Each text in the OneGraph model is labeled as \texttt{Name, Description, Document} from the perspective of text.} \texttt{Name} denotes a few words that named the beings and relations. For example, if we want to say that $\texttt{Albert Einstein}$ is the name of the scientist Albert Einstein,  it could be expressed as
$$ \texttt{Albert Einstein} \sep \texttt{text label} \sep \texttt{Name}$$
We use the $\texttt{text label}$ to denote the text label of the object.
\texttt{Description} is a few words that could describe the object we want to store. \texttt{Red and Long Dress} describe a kind of dress, and we will label it by the following triple
$$ \texttt{Red and Long Dress} \sep \texttt{text label}\sep \texttt{Description}$$ 
\texttt{Document} is a long text composed by one or more sentences. For example, we could label a well-known sentences from fictions as $\texttt{Document}$ that 
$$\texttt{To be, or not to be, that is the question.} \sep \texttt{text label} \sep \texttt{Document}$$
\paragraph{Each text is as also labeled as \texttt{String}, \texttt{time}, \texttt{Number} from the perspective of formate.} For example, to state that Albert Einstein is born in 1879, we write:
\begin{align}
& \texttt{Albert Einstein} \sep \texttt{born in} \sep \texttt{1879} \nonumber\\
&\texttt{1879} \sep \texttt{format label} \sep \texttt{time} \nonumber
\end{align}

To deal with synonymy and ambiguity, for example, both \texttt{ZJU} and \texttt{Zhejiang University} could represent Zhejiang University. We define a specific text type \texttt{Abstract} to represent the unique objects, for example
\begin{align}
    & \texttt{ZJU} \sep \texttt{abstract to} \sep \texttt{Chinese University-ZJU} \nonumber\\
    & \texttt{Chinese University-ZJU} \sep \texttt{text type} \sep \texttt{Abstract} \nonumber
\end{align}
\texttt{Chinese University-ZJU} is an abstract text used to represent the Zhejiang University. Different to other types of text, in all abstract text, there will not be two abstract texts represent the same object. Each object has a unique and only one abstract text. Thus the function of abstract text is similar to the URI in the traditional data model of ontology such as RDF. 

\paragraph{Texts could be grouped into classes represented as text.} For example, \texttt{University} is composed by all universities in the world. We use \texttt{type} to denote the class-instance relation that 
$$ \texttt{Zhejiang University} \sep \texttt{type} \sep \texttt{University} $$
The class could also be in instance of another class. For example
$$ \texttt{University} \sep \texttt{type} \sep \texttt{Educational Institution}$$
With the \texttt{type} relation, classes and entities are organized in a hierarchy. We do not distinguish the type relation between entities and classes and between classes and classes.  
For each class, we label it as \texttt{Complete} or \texttt{Incomplete}. \texttt{Complete} denotes the instances or subclasses of the class is stored in the data. For example
\begin{align}
& \ttt{Francesca Rossi} \sep \ttt{type} \sep \ttt{ISWC2022 Keynot Speaker} \nonumber \\
& \ttt{Ilaria Capua} \sep \ttt{type} \sep \ttt{ISWC2022 Keynot Speaker} \nonumber \\
& \ttt{Markus Krötzsch} \sep \ttt{type} \sep \ttt{ISWC2022 Keynot Speaker} \nonumber \\
& \ttt{ISWC2022 Keynot Speaker} \sep \ttt{class type} \sep \ttt{Complete} \nonumber
\end{align}
These triples means that there are 3 keynote speaker in ISWC2022 and all of them are stored in the data. 

\texttt{Incomplete} means not all instances or subclasses are recorded. Each class is labeled as \texttt{Incomplete} by default. For example if only one keynote speaker is known in the previous example, the data will be 
\begin{align}
& \ttt{Francesca Rossi} \sep \ttt{type} \sep \ttt{ISWC2022 Keynot Speaker} \nonumber \\
& \ttt{ISWC2022 Keynote Speaker} \sep \ttt{class label} \sep \ttt{Incomplete} \nonumber
\end{align}

In the OneGraph model, all texts could be the subject, relation, and object in the triple. This allows to represent the properties of relations and relations between relations. For example, we could define one relation is a sub-relation of the other one like
$$ \texttt{has father} \sep \texttt{sub-relation of} \sep \texttt{has parents}$$

For complex events that could not be expressed in a single fact. We make a triple also as   

We use multiple triples with abstract text to represented it. To express the event that OpenAI introduced ChatGPT, we make a triple also as an object, and link it to the subject, relation, object. 
\begin{align}
& \ttt{OpenAI} \sep \ttt{announced} \sep \ttt{ChatGPT} \nonumber \\
& \ttt{OpenAI announced ChatGPT} \sep \ttt{subject} \sep \ttt{OpenAI} \\
& \ttt{OpenAI announced ChatGPT} \sep \ttt{relation} \sep \ttt{announced} \\
& \ttt{OpenAI announced ChatGPT} \sep \ttt{object} \sep \ttt{ChatGPT} \\
& \ttt{OpenAI announced ChatGPT} \sep \ttt{date} \sep \ttt{2022.11.30} \\
& \ttt{OpenAI announced ChatGPT} \sep \ttt{introducing blog} \sep \ttt{https://openai.com/blog/chatgpt} \nonumber \\ 
& \ttt{OpenAI announced ChatGPT} \sep \ttt{text type} \sep \ttt{description} \\
\end{align}


 

\subsection{Discussion}
Previous data model in the KG projects are designed to be language-independent. While the OneGraph model is language-dependent and language-friendly.

We do not require each entity to be an instance of at least of one class. 

OneGraph Model supports: edge property, which is not supported by RDF and RDF-star. 






\subsection{Semantics}
In this section, we will give a model-theoretic semantics to OneGraph. We first precribe that the set of objects $\mathcal{O}$ for any OneGraph model must contain  \ttt{type, text label, format label, class label, abstract to} as relation objects, \ttt{Name, Description, Document}, \ttt{Abstract, Complete, Incomplete} for head and tail object. 

\subsection{Relation to Other Formalisms}
