\documentclass[11pt]{book}              % Book class in 11 points
%\documentclass[justified,twoside]{tufte-book}


\parindent0pt  \parskip10pt             % make block paragraphs
%\raggedright                            % do not right justify
\usepackage{xcolor}
\usepackage{amsmath}
\usepackage{amssymb}
\usepackage[round]{natbib}
\usepackage{graphicx}
\usepackage{amsthm}
\usepackage{tgpagella}
\usepackage{emptypage}
\usepackage{multirow}
\usepackage{booktabs}
\usepackage[LGR,T1]{fontenc}
\newcommand{\textgreek}[1]{\begingroup\fontencoding{LGR}\selectfont#1\endgroup}

\usepackage{tcolorbox}


\usepackage{microtype}


\definecolor{echodrk}{HTML}{0099cc}
\definecolor{mymauve}{rgb}{0.58,0,0.82}
\usepackage[colorlinks,linkcolor=mymauve,citecolor=echodrk]{hyperref}

\definecolor{boxgray}{rgb}{0.9,0.9,0.9}
\definecolor{boxpink}{rgb}{0.9,0.7,0.7}

\renewcommand\vec{\mathbf}
\newcommand\mat{\mathbf}

\makeatletter
\renewcommand{\thesection}{%
  \ifnum\c@chapter<1 \@arabic\c@section
  \else \thechapter.\@arabic\c@section
  \fi
}
\makeatother



% margin notes
%\usepackage[top=1.5cm, bottom=1.5cm, outer=5cm, inner=2cm, heightrounded, marginparwidth=2.5cm, marginparsep=2cm]{geometry}
\usepackage{marginnote}
\renewcommand*{\marginfont}{\footnotesize} %\justify} <- Taco: including \justify will throw errors.

 \newcommand{\joan}[1]{{\color{green}#1}}
 \newcommand{\michael}[1]{{\color{magenta}#1}} \newcommand{\taco}[1]{{\color{blue}#1}}
 \newcommand{\petar}[1]{{\color{cyan}#1}}
%\newcommand{\joan}[1]{{#1}}
%\newcommand{\michael}[1]{{#1}}
%\newcommand{\taco}[1]{{#1}}
%\newcommand{\petar}[1]{{#1}}



\theoremstyle{plain}
\newtheorem{theorem}{Theorem}
\newtheorem{lemma}{Lemma}[section]
\newtheorem{proposition}[lemma]{Proposition}
\newtheorem{corollary}[lemma]{Corollary}
\theoremstyle{definition}
\newtheorem{definition}[lemma]{Definition}
\newtheorem{assumption}[lemma]{Assumption}
\theoremstyle{remark}
\newtheorem{remark}[lemma]{Remark}


\newcommand{\diag}{\mathrm{diag}}
\newcommand{\norm}[1]{\left\|{#1}\right\|} %
\newcommand{\rank}{\mathrm{rank}}
\newcommand{\B}{\mathcal{B}}
\newcommand{\M}{\mathcal{M}}
\newcommand{\BS}{\B^*}
\newcommand{\BBS}{\B\B^*}
\newcommand{\BSB}{\B^*\B}
\newcommand{\MMS}{\M\M^*}
\newcommand{\MSM}{\M^*\M}
\newcommand{\BF}{\mathcal{B}\mathcal{F}}
\newcommand{\BD}{\mathcal{B}\mathcal{D}}
\newcommand{\DB}{\mathcal{D}\mathcal{B}}
\newcommand{\ind}[1]{^{(#1)}}

\newcommand{\vzero}{\mathbf{0}}
\newcommand{\vone}{\mathbf{1}}
\newcommand{\va}{\mathbf{a}}
\newcommand{\vb}{\mathbf{b}}
\newcommand{\vc}{\mathbf{c}}
\newcommand{\vd}{\mathbf{d}}
\newcommand{\ve}{\mathbf{e}}
\newcommand{\vf}{\mathbf{f}}
\newcommand{\vg}{\mathbf{g}}
\newcommand{\vh}{\mathbf{h}}
\newcommand{\vi}{\mathbf{i}}
\newcommand{\vj}{\mathbf{j}}
\newcommand{\vk}{\mathbf{k}}
\newcommand{\vl}{\mathbf{l}}
\newcommand{\vm}{\mathbf{m}}
\newcommand{\vn}{\mathbf{n}}
\newcommand{\vo}{\mathbf{o}}
\newcommand{\vp}{\mathbf{p}}
\newcommand{\vq}{\mathbf{q}}
\newcommand{\vr}{\mathbf{r}}
\newcommand{\vs}{\mathbf{s}}
\newcommand{\vt}{\mathbf{t}}
\newcommand{\vu}{\mathbf{u}}
\newcommand{\vv}{\mathbf{v}}
\newcommand{\vw}{\mathbf{w}}
\newcommand{\vx}{\mathbf{x}}
\newcommand{\vy}{\mathbf{y}}
\newcommand{\vz}{\mathbf{z}}
\newcommand{\vA}{\mathbf{A}}
\newcommand{\vB}{\mathbf{B}}
\newcommand{\vC}{\mathbf{C}}
\newcommand{\vD}{\mathbf{D}}
\newcommand{\vE}{\mathbf{E}}
\newcommand{\vF}{\mathbf{F}}
\newcommand{\vG}{\mathbf{G}}
\newcommand{\vH}{\mathbf{H}}
\newcommand{\vI}{\mathbf{I}}
\newcommand{\vJ}{\mathbf{J}}
\newcommand{\vK}{\mathbf{K}}
\newcommand{\vL}{\mathbf{L}}
\newcommand{\vM}{\mathbf{M}}
\newcommand{\vN}{\mathbf{N}}
\newcommand{\vO}{\mathbf{O}}
\newcommand{\vP}{\mathbf{P}}
\newcommand{\vQ}{\mathbf{Q}}
\newcommand{\vR}{\mathbf{R}}
\newcommand{\vS}{\mathbf{S}}
\newcommand{\vT}{\mathbf{T}}
\newcommand{\vU}{\mathbf{U}}
\newcommand{\vV}{\mathbf{V}}
\newcommand{\vW}{\mathbf{W}}
\newcommand{\vX}{\mathbf{X}}
\newcommand{\vY}{\mathbf{Y}}
\newcommand{\vZ}{\mathbf{Z}}

\newcommand{\cA}{\mathcal{A}}
\newcommand{\cB}{\mathcal{B}}
\newcommand{\cC}{\mathcal{C}}
\newcommand{\cD}{\mathcal{D}}
\newcommand{\cF}{\mathcal{F}}
\newcommand{\cG}{\mathcal{G}}
\newcommand{\cH}{\mathcal{H}}
\newcommand{\cI}{\mathcal{I}}
\newcommand{\cK}{\mathcal{K}}
\newcommand{\cL}{\mathcal{L}}
\newcommand{\cM}{\mathcal{M}}
\newcommand{\cN}{\mathcal{N}}
\newcommand{\cP}{\mathcal{P}}
\newcommand{\cS}{\mathcal{S}}
\newcommand{\cT}{\mathcal{T}}
\newcommand{\cX}{\mathcal{X}}

\newcommand{\R}{\mathbb{R}}
\newcommand{\F}{\mathbb{F}}

\newcommand{\Z}{\mathbb{Z}}
\newcommand{\ep}{\epsilon}
\newcommand{\g}{\gamma}
\newcommand{\Y}{\infty}
\newcommand{\f}[2]{\dfrac{#1}{#2}}
\newcommand{\ff}[2]{\tfrac{#1}{#2}}
\newcommand{\lm}[2]{\lim_{#1\rightarrow #2}}
\newcommand{\de}{\delta}
\newcommand{\T}{\theta}
\newcommand{\tm}{\times}
\newcommand{\su}[2]{\mathlarger{\sum\limits_{#1}^{#2}}}
\newcommand{\pd}[2]{\mathlarger{\prod\limits_{#1}^{#2}}}
\renewcommand{\sec}[1]{\section*{#1}}
\newcommand{\st}[1]{\subsection*{#1}}
\newcommand{\sst}[1]{\subsubsection*{#1}}
\renewcommand{\b}{\textbf}
\newcommand{\lessim}{\lesssim}
\newcommand{\E}{\mathbb{E}}
\newcommand{\p}{\partial}
\newcommand{\lt}{\left(}
\newcommand{\rt}{\right)}
\newcommand{\Lt}{\left[}
\newcommand{\Rt}{\right]}
\newcommand{\A}{\alpha}
\renewcommand{\b}{\beta}
\newcommand{\I}[2]{\mathlarger{\int_{#1}^{#2}}}
\newcommand{\G}{\nabla}
\newcommand{\Om}{\Omega}
\newcommand{\y}{\tau}
\newcommand{\K}{\mathcal{K}}
\newcommand{\C}{\mathbb{C}}
\newcommand{\om}{\omega}
\newcommand{\D}{\Delta}
\newcommand{\N}{\mathcal{N}}
\newcommand{\ra}{\rightarrow}
\newcommand{\Ra}{\Rightarrow}
\newcommand{\floor}[1]{\left\lfloor#1\right\rfloor}
\newcommand{\ceil}[1]{\left\lceil#1\right\rceil}
\newcommand{\ip}[1]{\left\langle#1\right\rangle}
\renewcommand{\mod}{\text{ mod }}
\newcommand{\sign}{\text{sign}}
\newcommand{\defeq}{:=}
\renewcommand{\a}{\bar{a}}
\newcommand{\MM}{\widetilde{\vM}}

\DeclareMathOperator*{\argmin}{argmin}


\definecolor{olivegreen}{HTML}{006400}
\definecolor{echoblue}{HTML}{0099CC}
\definecolor{gold}{HTML}{F09A00}
\definecolor{vividred}{HTML}{E60B42}
\definecolor{echonavy}{HTML}{0054B2}
\definecolor{darkgry}{HTML}{333333}
\definecolor{echopurple}{HTML}{9400D1}


\newcommand*{\horzbar}{\rule[.5ex]{2.5ex}{0.5pt}}


\newtheoremstyle{break}
  {\topsep}{\topsep}%
  {\itshape}{}%
  {\bfseries}{}%
  {\newline}{}%
%\theoremstyle{break}

\newtheorem{theorem}{Theorem}
\newtheorem{remark}{Remark}
\newtheorem{example}{Example}
\newtheorem{remarks}{Remarks}
\newtheorem{consequence}{Consequence}
\newtheorem{consequences}{Consequences}


\usepackage{fancyhdr}

\fancypagestyle{plain}{ %
  \fancyhf{} % remove everything
  \renewcommand{\headrulewidth}{0pt} % remove lines as well
  \renewcommand{\footrulewidth}{0pt}
}

\fancypagestyle{fancybook}{%
    \fancyhf{}%
    % Note the ## here. It's required because \fancypagestyle is making a macro (\ps@fancybook).
    % If we just wrote #1, TeX would think that it's the argument to \ps@fancybook, but
    % \ps@fancybook doesn't take any arguments, so TeX would complain with an error message.
    % You are not expected to understand this.
    \renewcommand*{\sectionmark}[1]{ \markright{\thesection\ ##1} }%
    %\renewcommand*{\chaptermark}[1]{ \markboth{\chaptername\ \thechapter: ##1}{} }%
    % Increase the length of the header such that the folios 
    % (typography jargon for page numbers) move into the margin
    \fancyhfoffset[LE]{6mm}% slightly less than 0.25in
    \fancyhfoffset[RO]{6mm}%
    % Put some space and a vertical bar between the folio and the rest of the header
    \fancyhead[LE]{\thepage\hskip3mm\vrule\hskip3mm BRONSTEIN, BRUNA, COHEN \& VELI\v{C}KOVI\'{C}}%
    \fancyhead[RO]{\rightmark\hskip3mm\vrule\hskip3mm\thepage}%
}
\pagestyle{fancybook}


%\usepackage{appendix}

\usepackage[titletoc]{appendix}


%\fancyhf[HLE,HRO]{Bruna et al.}



\title{\bf Geometric Deep Learning \\
Grids, Groups, Graphs,\\ Geodesics, and Gauges}    % Supply information
\author{Michael M. Bronstein\footnote{Imperial College London / USI IDSIA / Twitter}, Joan Bruna\footnote{New York University}, Taco Cohen\footnote{Qualcomm AI Research. Qualcomm AI Research is an initiative of Qualcomm Technologies, Inc.}, Petar Veli\v{c}kovi\'{c}\footnote{DeepMind}}              %   for the title page.
\date{\today}                           %   Use current date. 

% Note that book class by default is formatted to be printed back-to-back.
\begin{document}                        % End of preamble, start of text.
\frontmatter                            % only in book class (roman page #s)
\maketitle                              % Print title page.
{\hypersetup{linkcolor=black}
\tableofcontents
}        % Print table of contents
\mainmatter                             % only in book class (arabic page #s)
%\part{A Part Heading}                   % Print a "part" heading




\section*{Preface}  
\addcontentsline{toc}{section}{Preface}
\sectionmark{Preface}


For nearly two millenia since Euclid's \emph{Elements}, the word\marginnote{
\includegraphics[width=0.9\linewidth]{figures/ml_erlangen.png}
According to a popular belief, the Erlangen Programme was delivered in Klein's inaugural address in October 1872. Klein indeed gave such a talk (though on December 7 of the same year), but it was for a non-mathematical audience and concerned primarily his ideas of mathematical education. What is now called the `Erlangen Programme' was actually a research prospectus brochure {\em Vergleichende Betrachtungen über neuere geometrische Forschungen} (``A comparative review of recent researches in geometry'') he prepared as part of his professor appointment. See \cite{tobies2019felix}. 
%, subtitled {\em Programm zum Eintritt in die philosophische Fakultät und den Senat der k. Friedrich-Alexanders-Universität zu Erlangen} (``Program for entry into the Philosophical Faculty and the Senate of the Emperor Friedrich-Alexander University of Erlangen''). 
%While Erlangen claims the credit, Klein has stayed there for only three years, moving in 1875 to the Technical University of Munich (then called {\em Technische Hochschule}), followed by Leipzig (1880), and finally settling down in Göttingen from 1886 until his retirement. See R. Tobies, Felix Klein—Mathematician, Academic Organizer, Educational Reformer, In: H. G. Weigand et al. (eds) {\em The Legacy of Felix Klein}, Springer, 2019.
} `geometry' has been synonymous with \emph{Euclidean geometry}, as no other types of geometry existed. 
%
Euclid's monopoly came to an end in the nineteenth century, with examples of non-Euclidean geometries constructed by Lobachevesky, Bolyai, Gauss, and Riemann. 
%
%By the late nineteenth century however, various alternative non-Euclidean geometries such as hyperbolic-, elliptic- and projective geometry had emerged. 
%
Towards the end of that century, these studies had diverged into disparate fields, with 
mathematicians and philosophers debating the validity of and relations between these geometries as well as the nature of the ``one true geometry''. 


A way out of this pickle was shown by a young mathematician Felix Klein, appointed in 1872 as professor in the small Bavarian University of Erlangen. In a research prospectus, which entered the annals of mathematics as the \emph{Erlangen Programme}, Klein proposed approaching geometry as the study of {\em invariants}, i.e. properties unchanged under some class of transformations, called the {\em symmetries} of the geometry.
%and used the formalism of group theory. 
%
This approach created clarity by showing that various geometries known at the time could be defined by an appropriate choice of symmetry transformations, formalized using the language of group theory. 
%
%understood as the study of those and only those properties that are invariant under some group of symmetries associated with the geometry.
For instance, Euclidean geometry is concerned with lengths and angles, because these properties are preserved by the group of Euclidean transformations (rotations and translations), while affine geometry studies parallelism, which is preserved by the group of affine transformations. 
%
The relation between these geometries is immediately apparent when considering the respective groups, because the Euclidean group is a subgroup of the affine group, which in turn is a subgroup of the group of projective transformations. 
%, so certain figures which are Euclidean-equivalent may not be affine- or projectively equivalent.


The impact of the Erlangen Programme on geometry was very profound.
Furthermore, it spilled to other fields, especially physics, where symmetry principles allowed to derive conservation laws from first principles of symmetry (an astonishing result known as Noether's Theorem), 
and even enabled the classification of elementary particles as irreducible representations of the symmetry group.
%
{\em Category theory}, 
%a formalization of abstract similarities between mathematical structures, 
now pervasive in pure mathematics,
can be ``regarded as a continuation of the Klein Erlangen Programme, in the sense that a geometrical space with its group of transformations is generalized to a category with its algebra of mappings'', in the words of its creators Samuel Eilenber and Saunders Mac Lane. \marginnote{See \cite{marquis2009category}. }


%In mathematics, the Erlangen program ultimately inspired the creation of categories by Eilenberg \& Mac Lane\footnote{Indeed, in the first paper on categories, the authors write that ``This may be regarded as a continuation of the Klein Erlanger Program, in the sense that a geometrical space with its group of transformations is generalized to a category with its algebra of mappings.''}, which are widely prized for their ability to bring forth abstract similarities between mathematical structures and to reveal many ideas as ``just'' a special case of a more general one.


At the time of writing, the state of the field of deep learning is somewhat reminiscent of the field of geometry in the nineteenth century. 
There is a veritable zoo of neural network architectures for various kinds of data, but few unifying principles. 
As in times past, this makes it difficult to understand the relations between various methods, inevitably resulting in the reinvention and re-branding of the same concepts in different application domains.  
%
For a novice trying to learn the field, absorbing the sheer volume of redundant ideas is a true nightmare.  


In this text, we make a modest attempt to apply the Erlangen Programme mindset to the domain of deep learning, with the ultimate goal of obtaining a systematisation of this field and `connecting the dots'. 
%
We call this geometrisation attempt `Geometric Deep Learning', and true to the spirit of Felix Klein, propose to derive different inductive biases and network architectures implementing them from first principles of symmetry and invariance. 
%
%more systematic understanding of the landscape of geometric deep learning methods.
%
In particular, we focus on a large class of neural networks designed for analysing unstructured sets, grids, graphs, and manifolds, and show that they can be understood in a unified manner as methods that respect the structure and symmetries of these domains. 
%
%The archetypes of these domains, grid, graphs, groups, and  `four G'


We believe this text would appeal to a broad audience of deep learning researchers, practitioners, and enthusiasts. 
%
A novice may use it as an overview and introduction to Geometric Deep Learning. 
%
A seasoned deep learning expert may discover new ways of deriving familiar architectures from basic principles and perhaps some surprising connections.  
%
Practitioners may get new insights on how to solve problems in their respective fields. 


With such a fast-paced field as modern machine learning, the risk of writing a text like this is that it becomes obsolete and irrelevant before it sees the light of day. Having focused on foundations, our hope is that the key concepts we discuss will transcend their specific realisations\marginnote{``The knowledge of certain principles easily compensates the lack of knowledge of certain facts.'' \citep{helvetius1759esprit}} --- or, as Claude Adrien Helvétius put it, {\em ``la connaissance de certains principes supplée facilement à la connoissance de certains faits.'' }

%

\section*{Notation}
\addcontentsline{toc}{section}{Notation}
\sectionmark{Notation}

\begin{minipage}{\textwidth}
\def\arraystretch{1.5}
\begin{tabular}{lp{3.25in}}
$\Omega,u$ & Domain, point on domain\\
$x(u) \in \mathcal{X}(\Omega,\mathcal{C})$ & Signal on the domain of the form $x:\Omega\rightarrow \mathcal{C}$\\
$f(x) \in \mathcal{F}(\mathcal{X}(\Omega))$ & Functions on signals on the domain of the form $f:\mathcal{X}(\Omega) \rightarrow \mathcal{Y}$\\
%
$\fG,\fg$ & Group, element of the group\\
$\fg.u, \rho(\fg)$ & Group action, group representation\\
%
$\vec{X}\in\mathcal{C}^{|\Omega|\times s}$ & Matrix representing a signal on a discrete domain\\
$\vec{x}_u\in\mathcal{C}^{s}$ & Vector representing a discrete domain signal $\vec{X}$ on element $u\in\Omega$\\
$x_{uj}\in\mathcal{C}$ & Scalar representing the $j$th component of a discrete domain signal $\vec{X}$ on element $u\in\Omega$\\
$\vec{F}(\vec{X})$ & Function on discrete domain signals that returns another discrete domain signal, as a matrix\\
%
$\tau:\Omega\rightarrow\Omega$ & Automorphism of the domain \\
$\eta:\Omega\rightarrow\Omega'$ & Isomorphism between two different domains \\
$\sigma: \mathcal{C}\rightarrow\mathcal{C}'$ & Activation function (point-wise non-linearity)\\
%
$G=(\mathcal{V},\mathcal{E})$ & Graph with nodes $\mathcal{V}$ and edges $\mathcal{E}$ \\
$\mathcal{T}=(\mathcal{V},\mathcal{E},\mathcal{F})$ & Mesh with nodes $\mathcal{V}$,  edges $\mathcal{E}$, and faces $\mathcal{F}$ \\
%
$x\star \theta$ & Convolution with filter $\theta$ \\
$S_v$ & Shift operator \\
%
$\varphi_i$ & Basis function \\
%
%
$T_u\Omega, T\Omega$ & Tangent space at $u$, tangent bundle \\
$X \in T_u\Omega$ & Tangent vector\\
$g_u(X,Y) = \langle X, Y\rangle_u$ & Riemannian metric\\
%
$\ell(\gamma), \ell_{uv}$ & Length of a curve $\gamma$, discrete metric on edge $(u,v)$ \\

\end{tabular}
\end{minipage}
\pagebreak

\section{Introduction}
\label{sec:introduction}

\begin{wrapfigure}{r}{0.5\textwidth}
\vspace{-6mm}
\begin{center}
    \includegraphics[width=0.5\textwidth]{images/cover.pdf}
  \end{center}
  \vspace{-4mm}
  \caption{\textbf{Overview of \implname.} In training, we tune the singular values of the weight matrices to generate a set of ``expert'' vectors specializing in different tasks. In inference, a two-pass process is adopted where the first applies the expert and the second generates the answer.}
  \label{fig:cover}
  \vspace{-4mm}
\end{wrapfigure}

Self-adaptive large language models (LLMs) would represent a significant advancement in artificial intelligence, enabling real-time adaptation to various tasks and contexts.
While compositionality and scalability are crucial for effective adaptation, current LLM training methodologies fall short of achieving both these properties simultaneously.
Our research aims to present a solution to address these gaps.

In principle, the first step toward achieving self-adaptive LLMs can be realized through the development of specialized expert modules, each fine-tuned~\citep{kaplan2020scaling} via techniques such as low-rank adaptation (LoRA)~\citep{hu2021lora}. 
However, several challenges need to be addressed to make this approach both scalable and compositional: (1) multiple expert modules significantly increase the number of parameters; (2) expert modules are often prone to overfitting; and (3) flexible composition of these experts is still an open problem.

To overcome these limitations, we first propose \svdacro, a novel parameter-efficient fine-tuning (PEFT) method to obtain effective building blocks for self-adaptation.
\svdacro works by extracting and selectively tuning only the singular values within the model's weight matrices.
By focusing on this essential and principled parameterization, our approach mitigates the risk of overfitting, drastically reduces computational demands, and allows for inherent compositionality.

We then introduce our full \implname framework, which entails a two-pass inference mechanism to produce dynamically adapted weights targeted for the test-time conditions (Figure~\ref{fig:cover}).
We design three different adaptation strategies that can be used within \implname, which we show provide monotonic performance benefits with increasing access to the test-time conditions.
We evaluate \svdacro and the full \implname framework through extensive experiments across a diverse range of LLMs and tasks.
\svdacro outperforms traditional efficient fine-tuning methods like LoRA on domain-specific datasets with far fewer parameters. 
\implname further improves performance, even for out-of-distribution tasks like visual QA. 
Our analysis even shows that \implname allows the reuse of \svdacro experts across different LLMs. In summary, our key technical contributions are: 
\vspace{-2mm}
\begin{itemize}
\item The development of \implname as a pivotal self-adaptation framework for LLMs, providing a blueprint to adapt the behavior of LLMs from a growing set of pre-trained skills.
\item The introduction of \svdacro, a novel PEFT method trainable with RL on small datasets, producing compact expert vectors with inherent compositionality.
\item The implementation of three adaptation strategies, effectively dispatching \svdacro-trained experts with properties designed to cope with different deployment scenarios.
\end{itemize}

\vspace{-2mm}

\bibliographystyle{plainnat}
\bibliography{references}





\end{document}                          % The required last line
