\begin{table*}
\scriptsize
\rowcolors{2}{red!15}{white}

    \caption{Sample solutions generated by \texttt{Llama3.1-8B-Base} identified as having \textbf{incorrect} reasoning steps. We highlight in red \hl{incorrect judgements} made by the models.}

    \label{tab:low_quality_solns_8b}


\centering
        \begin{tabular}{p{0.09\textwidth} p{0.1\textwidth} p{0.2\textwidth} p{0.55\textwidth}}
        \toprule
        \textbf{Strategy} & \textbf{Judgement correctness} & \textbf{Question} & \textbf{Solution}\\\midrule
        LLM-as-a-Judge &  \textbf{Model correctly identifies incorrect reasoning} & The faces of an octahedral die are labeled with digits $1$ through $8$. What is the probability, expressed as a common fraction, of rolling a sum of $15$ with a pair of such octahedral dice? 
        & 
        We need to find the probability of summing $15$ with a pair of dice rolls. 

For each die roll, there are $8$ possible outcomes, which are $\{1,2,3,4,5,6,7,8\}$. To calculate the total number of possible outcomes, we find $8 \times 8 = 64$. (\textcolor{blue}{Correct reasoning})

Now we need to find how many rolls lead to a sum of $15$. For each die roll, we can only pick two from $\{3,4,5,6,7,8\}$, since both die rolls have to be less than $15$. (\textcolor{red}{Incorrect reasoning})

We can pick any combination of two from the set of six, which is $\binom{6}{2} = 15$. There are two ways to roll $15$: $(3,12)$ and $(4,11)$. (\textcolor{red}{Incorrect reasoning})

Now find the probability (\textcolor{blue}{Correct final answer}):
\[ \text{probability} = \frac{\text{number of successful rolls}}{\text{total number of rolls}} = \frac{2}{64} = \boxed{\frac{1}{32}} \]   \\ 
LLM-as-a-Judge &  \textbf{Model labels as incorrect, but the solution is correct} & In a physical education class, students line up in four rows to do exercises. One particular class has over 30 students, with three rows having the same number of students and one row having one more student than the other three rows. What is the smallest possible class size for this physical education class? & 
Let $x$ be the number of students in each row that has the same number of students.
Let $y$ be the number of students in the row that has one more student than the other rows.

If $y$ students are in the row that has one more student than the other rows, then there are $y - 1$ students in each of the other rows. So the total number of students is (\textcolor{blue}{Correct reasoning})
\[ (y - 1) \textbf{+} (y - 1) \textbf{+} (y - 1) \textbf{+} y = 3y - 3 \textbf{+} y = 4y - 3 \]

We need the total number of students to be greater than 30. Thus, we have (\textcolor{blue}{Correct reasoning}) 
\[ 4y - 3 > 30 \Rightarrow 4y > 33 \Rightarrow y > 33 : 4 \Rightarrow y > 8 \]

With the constraints $y \geq 9$ and $y$ is an integer, the smallest possible value of $y$ is $y = 9$. (\textcolor{blue}{Correct reasoning})

Thus the smallest possible class size is $4y - 3 = 4(9) - 3 = \boxed{33}$.   (\textcolor{blue}{Correct final answer}) \\

Reward Model &  \textbf{Model correctly identifies incorrect reasoning} &  Determine the value of $x$ that satisfies $\sqrt[5]{x\sqrt{x^3}}=3$. &  
We apply the power rule to $\sqrt[5]{x\sqrt{x^3}}$.

(\textcolor{red}{Incorrect reasoning})

\[ \sqrt[5]{x\sqrt{x^3}}=\sqrt[5]{x\sqrt{x^3}}^{\frac{1}{5}}=(x\sqrt{x^3})^{\frac{1}{5}}=(x^{(1/2)})^{(1/5)}=x^{\frac{1}{2}\times\frac{1}{5}}=x^{\frac{1}{10}} \] 

(\textcolor{red}{Incorrect reasoning})

Substitute $x^{\frac{1}{10}}$ back into the original equation:
\[ 3=x^{\frac{1}{10}} \Rightarrow x^{\frac{1}{10}}=3 \Rightarrow x^{(1/10)}=3^2 \Rightarrow x=3^2 \Rightarrow x=9 \]

Therefore, the value of $x$ that satisfies $\sqrt[5]{x\sqrt{x^3}}=3$ is $\boxed{9}$.  (\textcolor{blue}{Correct final answer}) \\
\bottomrule

        \end{tabular}

\end{table*}
























