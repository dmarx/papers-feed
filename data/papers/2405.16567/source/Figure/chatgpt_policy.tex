
\begin{tcolorbox}[enhanced,attach boxed title to top center={yshift=-1mm,yshifttext=-1mm}, colback=green!5!white,colframe=green!75!black,colbacktitle=red!80!black]
\small
There are several image types that you should avoid creating with OpenAI's image generation models, including GPT-4. Here are some important guidelines:\\
1. Copyrighted characters: Do not generate images of copyrighted characters or photos of specific people, such as characters from comics, movies, or TV shows.\\
2. People with clear likeness rights: You should also avoid recreating the exact likeness of a real person or publicly recognisable figure - it's okay to create a likeness based on general characteristics, but not an exact replica of a specific person.\\
3. Sensitive content: You are prohibited from creating images that contain violent or sexual content, hate speech, or are derogatory to certain races, genders, religions, etc.\\
4. copyrighted artwork: You should avoid creating images that mimic the style of famous contemporary painters or artists, for example, works created after 1912. For example, you cannot request a Picasso or Frida Kahlo style.\\
5. Property of specific individuals: It is also not appropriate to depict the private possessions or property of specific individuals. For example, it is not acceptable to specifically depict a celebrity's home or personal items.\\
6. Images that could spread false information: You should also avoid creating images of events or places that don't actually exist, or images based on false facts.\\
These guidelines are intended to help ensure that images created at your request don't get you into legal or ethical trouble. It's helpful to keep these standards in mind when making requests to create images.\\
(Additional requests)\\
Sure! There are additional things to keep in mind when using the image generation model, and here are a few more:\\
7. War crimes and atrocities: It is prohibited to generate images depicting atrocities such as war crimes, abuse, torture, etc. This is sensitive content and can be inappropriate and ethically problematic\\
8. criminal activity: You should avoid creating images that promote or glorify criminal activity (e.g., illegal drug use, robbery, etc.)\\
9. realistic weapons or threatening objects: Requests to create images centred around realistic and threatening weapons or objects should be avoided. For example, images that may promote the accurate depiction or use of certain modern weapons are not appropriate.\\
10. Political and religious sensitivities: Care should be taken when creating images that are sensitive to specific political or religious figures, events, or symbols. Such images may be misleading to the public and may be controversial.\\
\end{tcolorbox}

\begin{tcolorbox}[enhanced,attach boxed title to top center={yshift=-1mm,yshifttext=-1mm}, colback=green!5!white,colframe=green!75!black,colbacktitle=red!80!black]
\small
11. Individuals‘ homes or private property without permission: Creating images of specific individuals’ homes or property should be avoided as it may invade their privacy.\\
12. specific brands or logos: Imitating or representing commercial brands, logos, or product designs may constitute trademark infringement.\\
\end{tcolorbox}