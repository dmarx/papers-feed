\vspace{-0.15in}
\section{Limitation}\vspace{-0.1in}
Our approach has the limitation that the violation rate does not always reproduce the same due to the randomness of the commercial T2I systems. In addition, depending on the trial, content that was blocked may be generated again or the prompt that was generated may be blocked in other trials. Thus, multiple trials can eventually generate all copyright content. Moreover, the results may change when the commercial T2I service is updated.\footnote{The recently released GPT-4o seems to be more vulnerable to copyright infringement than GPT-4.} Although our approach relies on non-deterministic commercial T2I systems, we believe that the most significant contribution of this paper is to highlight the risk of copyright infringement, which many commercial T2I systems currently violate. One of the other limitations is that this paper analyzes copyright infringement from a technical point of view, so we could not confirm the extent to which commercial systems actually cause copyright infringement from a legal perspective. Despite the conduct of human evaluations, discrepancies may arise between the views of non-experts participants who are lack of expertise in copyright and actual legal judgments in court. However, we believe that this paper presents an opportunity for commercial companies to reconsider legal perspectives in depth.
%there may be discrepancies between the views of ordinary people, who are not experts in copyright issues, and legal judgments. However, we believe that this paper provides an opportunity for commercial companies to reconsider the legal perspective in depth.

\vspace{-0.1in}
\section{Conclusion}
\vspace{-0.1in}
In this paper, we have demonstrated that commercial T2I systems currently underestimate the risk of copyright infringement, even with naive prompts. Although several systems have implemented internal censorship mechanisms to prevent such violations, our Automated Prompt Generation Pipeline (APGP) easily circumvents these safeguards. The APGP utilizes a novel approach by integrating a self-generated QA score and a keyword penalty score within the LLM optimizer, without necessitating weight updates or gradient computations. Our empirical results show that APGP-generated prompts resulted in 76.0\% content violations in ChatGPT, a model previously considered 84.0\% secure against copyright issues. We conclude that our approach not only streamlines the process of red-teaming T2I models to expose risks at reduced costs but also aids intellectual property owners in more effectively claiming their rights.

\section*{Broader Impact}

\begin{wrapfigure}[9]{r}{0.28\textwidth}
  \vspace{-0.4in}
  \begin{center}
    \includegraphics[width=0.98\linewidth]{figure_folder/possible_risk.png}
  \end{center}
  \vspace{-0.1in}
  \caption{\small Violation of character copyright and publicity right}
  \label{figure:possible_risk}
\end{wrapfigure}
Our approach involves searching for prompts that may lead to copyright infringement in commercial text-to-image (T2I) systems. There is a concern that our work could enable adversaries to exploit these systems. Additionally, we are worried about the potential misuse and abuse of our approach, as we have identified instances of not only copyright infringement but also violations of publicity rights (Figure~\ref{figure:possible_risk}). However, these risks have been inherent in commercial T2I systems from the beginning. We believe it is crucial to acknowledge these issues and investigate ways to enhance the safety of real-world AI applications in the future.

\section*{Acknowledgement}
This work was supported by the 2023 Microsoft Research Asia Collaborative Research grant funded by Microsoft, Institute of Information \& communications Technology Planning \& Evaluation (IITP) grant funded by the Korea government (MSIT) (No.2020-0-00153) and by Institute of Information \& communications Technology Planning \& Evaluation (IITP) grant funded by the Korea government(MSIT) (No.2019-0-00075, Artificial Intelligence Graduate School Program(KAIST)). We are grateful to Jin Myung Kwak for providing valuable feedback and to Jihoon Tack for offering constructive feedback and manual trial results in support of this research.ort of this research. 