\chapter*{Preface}




\subsection*{The importance of studying the linear model}


A central task in statistics is to use data to build models to make inferences about the underlying data-generating processes or make predictions of future observations. Although real problems are very complex, the linear model can often serve as a good approximation to the true data-generating process. Sometimes, although the true data-generating process is nonlinear, the linear model can be a useful approximation if we properly transform the data based on prior knowledge. Even in highly nonlinear problems, the linear model can still be a useful first attempt in the data analysis process. 



Moreover, the linear model has many elegant algebraic and geometric properties. Under the linear model, we can derive many explicit formulas to gain insights about various aspects of statistical modeling. In more complicated models, deriving explicit formulas may be impossible. Nevertheless, we can use the linear model to build intuition and make conjectures about more complicated models. 


Pedagogically, the linear model serves as a building block in the whole statistical training. This book builds on my lecture notes for a master's level ``Linear Model'' course at UC Berkeley, taught over the past eight years. Most students are master's students in statistics. Some are undergraduate students with strong technical preparations. Some are Ph.D. students in statistics. Some are master's or Ph.D. students in other departments. This book requires the readers to have basic training in linear algebra, probability theory, and statistical inference. 
 
 
 
 


\subsection*{Recommendations for instructors}



This book has twenty-seven chapters in the main text and four chapters as the appendices. As I mentioned before, this book grows out of my teaching of ``Linear Model'' at UC Berkeley. In different years, I taught the course in different ways, and this book is a union of my lecture notes over the past eight years. Below I make some recommendations for instructors based on my own teaching experience. Since UC Berkeley is on the semester system, instructors on the quarter system should make some adjustments to my recommendations below.



\paragraph*{Version 1: a basic linear model course assuming minimal technical preparations}
If you want to teach a basic linear model course without assuming strong technical preparations from the students, you can start with the appendices by reviewing basic linear algebra, probability theory, and statistical inference. Then you can cover Chapters \ref{chapter::ols-1d}--\ref{chapter::interaction}. If time permits, you can consider covering Chapter \ref{chapter::binary-logit} due to the importance of the logistic model for binary data. 


\paragraph*{Version 2: an advanced linear model course assuming strong technical preparations}
If you want to teach an advanced linear model course assuming strong technical preparations from the students, you can start with the main text directly. When I did this, I asked my teaching assistants to review the appendices in the first two lab sessions and assigned homework problems from the appendices to remind the students to review the background materials. Then you can cover Chapters \ref{chapter::ols-1d}--\ref{chapter::sandwich}. You can omit Chapter \ref{chapter::rols} and some sections in other chapters due to their technical complications. If time permits, you can consider covering Chapter \ref{chapter::gee} due to the importance of the generalized estimating equation as well as its byproduct called the ``cluster-robust standard error'', which is important for many social science applications. 
Furthermore, you can consider covering Chapter \ref{chapter::survival-analysis} due to the importance of the Cox proportional hazards model. 


\paragraph*{Version 3: an advanced generalized linear models course} 
If you want to teach a course on generalized linear models, you can use Chapters \ref{chapter::binary-logit}--\ref{chapter::survival-analysis}. 







\subsection*{Additional recommendations for readers and students}



Readers and students can first read my recommendations for instructors above.  In addition, I have three other recommendations. 


\paragraph*{More simulation studies}
This book contains some basic simulation studies. I encourage the readers to conduct more intensive simulation studies to deepen their understanding of the theory and methods. 


\paragraph*{Practical data analysis}
Box wrote wisely that ``all models are wrong but some are useful.'' The usefulness of models strongly depends on the applications. When teaching ``Linear Model'', I sometimes replaced the final exam with the final project to encourage students to practice data analysis and make connections between the theory and applications.


\paragraph*{Homework problems}
This book contains many homework problems. It is important to try some homework problems. Moreover, some homework problems contain useful theoretical results. Even if you do not have time to figure out the details for those problems,  it is helpful to at least read the statements of the problems.  



\subsection*{Omitted topics}



Although ``Linear Model'' is a standard course offered by most statistics departments, it is not entirely clear what we should teach as the field of statistics is evolving. Although I made some suggestions to the instructors above, you may still feel that this book has omitted some important topics related to the linear model.   




\paragraph*{Advanced econometric models}
After the linear model, many econometric textbooks cover the instrumental variable models and panel data models. For these more specialized topics, \citet{wooldridge2010econometric} is a canonical textbook. 




\paragraph*{Advanced biostatistics models}
This book covers the generalized estimating equation in Chapter \ref{chapter::gee}. For analyzing longitudinal data, linear and generalized linear mixed effects models are powerful tools. \citet{fitzmaurice2012applied} is a canonical textbook on applied longitudinal data analysis.  
This book also covers the Cox proportional hazards model in Chapter \ref{chapter::survival-analysis}. For more advanced methods for survival analysis, \citet{kalbfleisch2011statistical} is a canonical textbook. 



\paragraph*{Causal inference}
I do not cover causal inference in this book intentionally. To minimize the overlap of the materials, I wrote another textbook on causal inference \citep{ding2023first}. However, I did teach a version of ``Linear Model'' with a causal inference unit after introducing the basics of linear model and logistic model. Students seemed to like it because of the connections between statistical models and causal inference. 



 

\subsection*{Features of the book}

 
The linear model is an old topic in statistics. There are already many excellent textbooks on the linear model. This book has the following features.


\begin{itemize}
\item
This book provides an intermediate-level introduction to the linear model. It balances rigorous proofs and heuristic arguments. 

\item
This book provides not only theory but also simulation studies and case studies.

\item
This book provides the \texttt{R} code to replicate all simulation studies and case studies. 

\item
This book covers the theory of the linear model related to not only social sciences but also biomedical studies. 

\item
This book provides homework problems with different technical difficulties. The solutions to the problems are available to instructors upon request. 
\end{itemize}




Other textbooks may also have one or two of the above features. This book has the above features simultaneously. I hope that instructors and readers find these features attractive.  
 
 
 




\subsection*{Acknowledgments}



 

Many students at UC Berkeley made critical and constructive comments on early versions of my lecture notes.  As teaching assistants for my ``Linear Model'' course,  Sizhu Lu,
Chaoran Yu, 
and
Jason Wu read early versions of my book carefully and helped me to improve the book a lot. 


Professors Hongyuan Cao and Zhichao Jiang taught related courses based on an early version of the book. They made very valuable suggestions. 


I am also very grateful for the suggestions from Nianqiao Ju. 


When I was a student, I took a linear model course based on \citet{weisberg2005applied}. 
In my early years of teaching, I used \citet{christensen2002plane} and \citet{agresti2015foundations} as reference books. 
I also sat in Professor Jim Powell's econometrics courses and got access to his wonderful lecture notes. They all heavily impacted my understanding and formulation of the linear model. 


If you identify any errors, please feel free to email me.

 
 
