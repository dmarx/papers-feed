\section{Method}
\section{Retrieval with Synchronised Graph Expansion}
\label{sec:graph_retrieval}

\def\Tqinit{\mathbf{T}_\mathbf{q}}


\begin{figure}[thbp]
  \includegraphics[width=\columnwidth]{figures/gear-sys-fig.pdf}
  \caption{\label{fig:system_diagram}System Architecture}
\end{figure}

% Start: Zhili --------------------------


Given an input query $\mathbf{q}$, let $\mathbf{C}_\mathbf{q}' = h^k_{\text{base}}\left( \mathbf{q}, {\mathbf{C}}\right )$  be a list of passages returned by the base retriever\footnote{The choice of a base retriever within our framework is flexible, without requiring any multi-hop capabilities.}.
Given this initially retrieved list of passages, $\mathbf{C}_\mathbf{q}'$, our goal is to derive relevant multi-hop contexts (passages) by retrieving a sub-graph of triples that interconnect their source passages. There are two challenges for materialising such sub-graph retrieval: \begin{inparaenum}[(i)]\item how to locate initial triples (i.e. starting nodes) $\Tqinit$, and \item how to expand the graph based on initial triples while reducing the search space\end{inparaenum}. The following sections address these challenges respectively, within \gear.



\subsection{Knowledge Synchronisation}
\label{subsection:knowledge_syncro}
\def\linkTriple{\texttt{tripleLink}}

We describe a knowledge \textbf{Sync}hronisation (\textbf{Sync}) process for locating initial nodes for graph expansion. We first employ an LLM to \texttt{read} $\mathbf{C}_\mathbf{q}'$ (see Appendix~\ref{subsec:online_retrieval_prompts}) and summarise knowledge triples that can support answering the current query $\mathbf{q}$, as defined:
\begin{align}
    \mathbf{T}_\mathbf{q}' = \texttt{read}\left (\mathbf{C}_\mathbf{q}', \mathbf{q}\right ).
    \label{eq:proximal_read}
\end{align}
$\mathbf{T}_\mathbf{q}'$ is a collection of triples to which we refer as \textit{proximal triples}. Initial nodes $\Tqinit$ for graph expansion can then be identified by linking each triple in $\mathbf{T}_\mathbf{q}'$ to a triple in $\mathbf{T}$, using the \linkTriple{} function:
\begin{align}
    \Tqinit =\left \{t_i | t_i = \linkTriple(t_i') ~ \forall t_i' \in \mathbf{T}_\mathbf{q}'\right \}.
\end{align}
The implementation of \linkTriple{} can vary. However, in this paper we consider it to be simply retrieving the most similar triple from $\mathbf{T}$.



\begin{algorithm}[ht]
\textbf{Input:} $\mathbf{q}$: query \\
\hspace*{3em} $b$: beam size \\
\hspace*{3em} $l$: maximum length \\
\hspace*{3em} $\mathrm{score}(\cdot, \cdot)$: scoring function \\
\hspace*{3em} $\{t_1, t_2, \ldots, t_n\}$: initial triples \\
\hspace*{3em} $\gamma$: hyperparameter for diversity


\begin{algorithmic}[1]
\State $B_0 \gets [\;]$
\For{$t \in \{t_1, t_2, ..., t_n\}$}
    \State $s \gets \mathrm{score}(\mathbf{q}, [t])$
    \State $B_0.\mathrm{add}(\langle s, [t] \rangle)$
\EndFor

\State $B_0 \gets \mathrm{top}(B_0, b)$


\For{$i \in \{1, \dots, l - 1\}$}
    \State $B \gets [\;]$
    
    \For{$\langle s, T \rangle \in B_{i-1}$}
        \State $V \gets [\;]$

        \For{$t \in \mathrm{get\_neighbours}(T.\mathrm{last}())$}
            \If{$\mathrm{exists}(t, B_{i-1})$}
                \State \textbf{continue}
            \EndIf
            
            \State $s' \gets s + \mathrm{score}(\mathbf{q}, T \circ t)$ ~ \texttt{\# concat} 
            \State $V.\mathrm{add}(\langle s', T \circ t \rangle)$
        \EndFor

        \State $\mathrm{sort}(V, \mathrm{descending})$

        \For{$n \in \{0, \dots, V.\mathrm{length()} - 1\}$}
            \State $\langle s', T \circ t \rangle \gets V[n]$
            \State $s' \gets s' \times e^{- \frac{\mathrm{min}(n, \gamma)}{\gamma}}$
            \State $B.\mathrm{add}(\langle s', T \circ t \rangle)$
        \EndFor
        
    \EndFor
    \State $B_i \gets \mathrm{top}(B, b)$
    
\EndFor

\State \Return $B_i$
\end{algorithmic}

\caption{Diverse Triple Beam Search}
\label{alg:beam_search}
\end{algorithm}

\subsection{Diverse Triple Beam Search}

We borrow the idea of constructing reasoning triple chains \cite{Fang2024} for expanding the graph, and present a retrieval algorithm: \textit{Diverse Triple Beam Search} (see Alg.~\ref{alg:beam_search}). 

We maintain top-$b$ sequences (beams) of triples and the scores at each step are determined by a scoring function. In this paper, we focus on leveraging a dense embedding model to compute the cosine similarity between embeddings of the query and a candidate sequence of triples, leaving other implementations of the scoring function for future work (see Section~\ref{sec:limitations}).

Considering all possible triple extensions at each step, in a Viterbi decoding fashion, would be intractable due to the size of $\mathbf{T}$. Consequently, we define the neighbourhood of a triple as the set of triples with shared head or tail entities (i.e. $\mathrm{get\_neighbours}$ in Alg.~\ref{alg:beam_search}). During each expansion step, we only consider neighbours of the last triple in the sequence, and avoid selecting previously visited triples (i.e. $\mathrm{exists}$ in Alg.~\ref{alg:beam_search}) to further reduce the search space.

While regular beam search can reduce the search space, it is prone to producing high-likelihood sequences that differ only slightly from one another \cite{Ippolito2019, Vijayakumar2018}. Our algorithm increases the diversity across beams to improve the recall for retrieval. In detail, for each beam, we sort candidate sequences extended from that beam in descending order, and weight their scores based on their relative positions. Candidate sequences that are ranked lower, within a beam, will receive smaller weights. Consequently, the resulting top-$b$ beams at each step are less likely to share the same starting sequence. 

The top-$b$ returned sequences are flattened in a breadth-first order. Each triple in the resulting list is then mapped to its source passage. This alignment between triples and passages is described in more detail in Section~\ref{sec:preliminaries}. Let $\widetilde{\mathbf{C}}_\mathbf{q}$ be the list of unique passages after alignment. The output of our graph expansion is then given by the Reciprocal Rank Fusion (RRF) \cite{Cormack2009} of $\widetilde{\mathbf{C}}_\mathbf{q}$ and the initial $\mathbf{C}_\mathbf{q}'$ list of passages :
\begin{align}
    \mathbf{C}_{\mathbf{q}} = \mathrm{RRF}\left(\widetilde{\mathbf{C}}_\mathbf{q}, \mathbf{C}_\mathbf{q}'\right ).
\end{align}
We refer to this graph-based method of retrieving relevant passages as \textbf{Sync}ronised \textbf{G}raph \textbf{E}xpansion (\textbf{SyncGE}).


\section{Multi-step Extension}


While SyncGE can enhance a base retriever with multi-hop context, some queries inherently require multiple steps to gather all necessary evidence. We materialise \gear by incorporating an agent with multi-turn capabilities, capable of interacting with the graph-retriever described above. We focus on:
\begin{itemize}
\item maintaining a gist memory of proximal knowledge obtained throughout the different steps 
\item incorporating a similar synchronisation process 
that summarises retrieved passages in proximal triples to be stored in this multi-turn gist memory
\item determining if additional steps are needed for answering the original input question
\end{itemize}
%
Within this multi-turn setting, the original input question $\mathbf{q}$ is iteratively decomposed into simpler queries: $\mathbf{q}^{(1)}, \ldots, \mathbf{q}^{(n)}$, where $\mathbf{q}^{(1)} = \mathbf{q}$ and $n \in \mathbb{N}$ represents the number of the current step.
For each query $\mathbf{q}^{(n)}$, we use the graph retrieval method introduced in Section~\ref{sec:graph_retrieval} in order to retrieve relevant passages $\mathbf{C}_{\mathbf{q}^{(n)}}$.



\subsection{Gist Memory Constructor}
To facilitate the multi-step capabilities of our agent, we introduce a \textit{gist memory}, $\mathcal{G}^{(n)}$, which is used for storing knowledge as an array of proximal triples. At the beginning of the first iteration, the gist memory is empty. During the $n$-th iteration, similar to the knowledge synchronisation module explained in Section~\ref{subsection:knowledge_syncro}, we employ an LLM to read a collection of retrieved paragraphs $\mathbf{C}_{\mathbf{q}^{(n)}}$ and summarise their content with proximal triples:

\begin{align}
\mathbf{T}_{\mathbf{q}^{(n)}}^{\mathcal{G}} = 
\begin{cases} 
    \texttt{read}\left(\mathbf{C}_{\mathbf{q}^{(n)}}, \mathbf{q} \right), & \text{if } n = 1 \\
    \texttt{read}\left(\mathbf{C}_{\mathbf{q}^{(n)}}, \mathbf{q}\textcolor{blue}{, \mathcal{G}^{(n-1)}} \right), & \text{if } n \geq 2
\end{cases}
\label{eq:proximal_read_agent}
\end{align}


Apart from the first iteration where Eq.~\ref{eq:proximal_read} and ~\ref{eq:proximal_read_agent} are identical, the inclusion of the memory in the \texttt{read} operation differentiates the construction of proximal triples produced at the subsequent steps compared to the ones from Eq.~\ref{eq:proximal_read}. $\mathcal{G}^{(n)}$ maintains the aggregated content of proximal triples s.t. 
\begin{align}
\mathcal{G}^{(n)} = \left[ \mathbf{T}_{\mathbf{q}^{(1)}}^{\mathcal{G}}  \circ \cdots \circ \mathbf{T}_{\mathbf{q}^{(n)}}^{\mathcal{G}} \right],
\end{align}where $\circ$ defines the concatenation operation. The triple memory serves as a concise representation of all the accumulated evidence, up to the $n$-th step. 

We believe the process introduced by the \texttt{read} step along with the information storage paradigm served by the gist memory, aligns well with the communication between the hippocampus and neocortex. The combination of the two establishes the synergetic behaviour between our graph retriever and the LLM that we seek to achieve within \gear.



\subsection{Reasoning for Termination}
After $\mathcal{G}^{(n)}$ is updated, we check the sufficiency of the accumulated evidence, within it, for answering the original question. This is achieved with the following LLM reasoning step:
\begin{align}
\mathbf{a}^{(n)}, \mathbf{r}^{(n)}   = \texttt{reason}(\mathcal{G}^{(n)}, \mathbf{q}),
\end{align}
% We can also call it 'sufficiency' instead of 'answerability'. I do not really have a preference.
where $\mathbf{a}^{(n)}$ denotes the query's answerability given the available evidence in $\mathcal{G}^{(n)}$, and $\mathbf{r}^{(n)}$ represents the reasoning behind this determination. When the query is deemed answerable, the system concludes its iterative process.



\subsection{Query Re-writing}
The query re-writing process leverages an LLM that incorporates three key inputs: the original query $\mathbf{q}$, the accumulated memory, and crucially, the reasoning output $\mathbf{r}^{(n)}$ from the previous step. This process can be formally expressed as:
\begin{align}
\mathbf{q}^{(n+1)} = \texttt{rewrite}\left (\mathcal{G}^{(n)}, \mathbf{q}, \mathbf{r}^{(n)} \right),
\end{align}
where $\mathbf{q}^{(n+1)}$ represents the updated query, which serves as input for the retriever in the next iteration.\\
\subsection{After Termination}
\gear aims to return a single ranked list of passages. Given the final gist memory $\mathcal{G}^{(n)}$ upon termination, we link each proximal triple in $\mathcal{G}^{(n)}$ to a list of passages as follows:
\begin{align}
    \mathbf{C}_{t_j} = \texttt{passageLink}\left(t_j, k\right),
\end{align}
where $j \in \left \{1, \dots, \vert\mathcal{G}^{(n)}\vert \right \}$. Similar to \texttt{tripleLink}, \texttt{passageLink} is implemented by retrieving passages with a triple as the query (see Appendix~\ref{appendixpara:passage_link}). The final list of passages returned by \gear is the RRF of the resulting linked passages and passages retrieved across iterations:
\begin{align}
\mathbf{C}_\mathbf{q}^{(n)} = \mathrm{RRF}\big(&\mathbf{C}_{t_1}, \ldots,\mathbf{C}_{t_{\vert\mathcal{G}^{(n)}\vert}}, \nonumber\\
    &\mathbf{C}_{\mathbf{q}^{(1)}}, \ldots, \mathbf{C}_{\mathbf{q}^{(n)}} \big).
\end{align}

All relevant prompts for the \texttt{read}, \texttt{reason} and \texttt{rewrite} steps are provided in Appendix~\ref{subsec:online_retrieval_prompts}.


Given a monocular video containing camera motions, a moving human, and a static scene, our method automatically disentangles and represents the human and the static scene with 3D Gaussians.
%
The human Gaussians are initialized using the \smpl body model and the scene Gaussians are initialized from the structure-from-motion point cloud from COLMAP~\cite{colmapschoenberger2016mvs, colmapschoenberger2016sfm}.
%
In the following, we first quickly review 3D Gaussian splatting and the \smpl body model. Then, we introduce the proposed method to address challenges when modeling and animating humans in the 3D Gaussian framework.  

\subsection{Preliminaries}

\paragraph{3D Gaussian Splatting (3DGS)~\cite{kerbl3Dgaussians}}
%3D Gaussians are parameterized by means $\bm\mu$, rotation $\bm{R}$, and scale $\bm{S}$. 
%
represents a scene by arranging 3D Gaussians. 
%
The $i$-th Gaussian is defined as 
%
\begin{equation}
    G(\mathbf{p}) = o_i \, e^{-\frac{1}{2} (\mathbf{p} - \bm\mu_i)^T \bm\Sigma_{i}^{-1} (\mathbf{p} - \bm\mu_i)},
    \label{eq:gauss3d}
\end{equation}
%
where $\mathbf{p} \, {\in} \, \mathbb{R}^3$ is a xyz location, $o_i \, {\in} \, [0, 1]$ is the opacity modeling the ratio of radiance the Gaussian absorbs, $\bm\mu_i \, {\in} \, \mathbb{R}^3$ is the center/mean of the Gaussian, and the covariance matrix $\bm\Sigma_{i}$ is parameterized by the scale $\mathbf{S}_{i} \, {\in} \, \mathbb{R_+}^3$ along each of the three Gaussian axes and the rotation $\mathbf{R}_{i} \, {\in} \, SO(3)$ with $\bm\Sigma_{i} = \mathbf{R}_{i} \mathbf{S}_{i} \mathbf{S}_{i}^\top \mathbf{R}_{i}^\top$.
%
Each Gaussian is also paired with spherical harmonics~\cite{ramamoorthi2001efficient} to model the radiance emit towards various directions. 
% 
%
% The $i$-th 3D Gaussian is parameterized by its mean $\bm\mu_i \in \mathbb{R}^3$, scales along each dimension $\mathbf{S}_{i} \in \mathbb{R_+}^3$, and rotation $\mathbf{R}_{i}\in SO(3)$.
%
% The covariance matrix is calculated by $\bm\Sigma_{i} = \mathbf{R}_{i} \mathbf{S}_{i} \mathbf{S}_{i}^\top \mathbf{R}_{i}^\top$.
%
% Each Gaussian is also paired with an opacity $o_i \in [0, 1]$, to model the ratio of radiance it absorbs, and a 4th-degree spherical harmonics~\cite{ramamoorthi2001efficient}, to model the radiance it emits toward various directions. 
%
% A 3D Gaussian is parameterized by its mean $\bm\mu_i \in \mathbb{R}^3$ and its covariance matrix $\bm\Sigma_{i} \in $, which is parameterized by rotation $\mathbf{R}_{i}\in SO(3)$ and scales $\mathbf{S}_{i} \in \mathbb{R}^3$. 

% the $\bm\Sigma_{i}$ into a rotation component $\mathbf{R}_{i}\in SO(3)$ and a scale component $\mathbf{S}_{i} \in \mathbb{R}^3$ where $\bm\Sigma_{i} = \mathbf{R}_{i} \mathbf{S}_{i} \mathbf{S}_{i}^\top \mathbf{R}_{i}^\top$.
%
% Each gaussian represents a physical space, we can compute the influence of a Gaussian to a point $\mathbf{p} \in \mathbb{R}^3$ by evaluating the probability density function

% \begin{equation}
%     G(\mathbf{p}) = \sigma({o_i}) e^{-\frac{1}{2} (\mathbf{p} - \bm\mu_i)^T \bm\Sigma_{i}^{-1} (\mathbf{p} - \bm\mu_i)},
%     \label{eq:gauss3d}
% \end{equation}
%
%where $o_i \in \mathbb{R}$ is the opacity of each gaussian and $\sigma$ is the sigmoid function. The view-dependent appearance of individual Gaussians is represented with 4th-degree spherical harmonics~\cite{ramamoorthi2001efficient}.

% represents an area of 3D physical space which is occupied by solid matter with a probability density function. %
% Each Gaussian influences a point in physical 3D space $p \in \mathbb{R}^3$ weighted by its opacity $o_i \in \mathbb{R}$ according to the standard (unnormalized) Gaussian equation,
% \begin{equation}
% f_{i}(p) = \sigma({o_i}) \exp\left( -\frac{1}{2} (p - \mu_{i})^T \Sigma_{i}^{-1} (p - \mu_{i}) \right),
% \label{eq:gauss3d}
% \end{equation}

% where ${\mu}_{i} \in \mathbb{R}^3$
% % = \begin{bmatrix} x_{i} &  y_{i} &  z_{i} \end{bmatrix}^T$ 
% is the center, and $\Sigma_{i} = R_{i} S_i S_i^T R_{i}^T$ is the covariance matrix of Gaussian $i$, with the scaling component $S_i \in \mathbb{R}^3 $ 
% %= \text{diag}\left(\begin{bmatrix} sx_i & sy_i & sz_i \end{bmatrix}\right)$,
% and the rotation component $R_{i} \in SO(3)$,
% %= \textrm{q2R}\left(\begin{bmatrix} qw_{i} & qx_{i} & qy_{i} & qz_{i} \end{bmatrix}\right)$, where $\textrm{q2R}()$ is the formula for constructing a rotation matrix from a quaternion. 
% and $\sigma$ is the standard sigmoid function.

% Each Gaussian's influence $f$ is both inherently local (being able to represent a small area of space), while also theoretically having infinite extent, such that gradients can flow to them even from a long distance, which is crucial for gradient-based differentiable rendering that optimizes for the position of 3D Gaussians. The softness of the 3D Gaussian representation also means that Gaussians typically need to significantly overlap in order to represent a physically solid object. As well as physical density, each Gaussian contributes its own appearance to each of the 3D points it influences.


During rendering, the 3D Gaussians are projected onto the image plane and form 2D Gaussians~\cite{zwicker2001surface} with the covariance matrix $\bm\Sigma_{i}^{\textrm{2D}} = \bm{J} \bm{W} \bm\Sigma_{i} \bm{W}^\top \bm{J}^\top$,
%following Zwicker \etal~\cite{zwicker2001surface} 
%
% \begin{equation}
%     \bm\Sigma_{i}^{\textrm{2D}} = J W \bm\Sigma_{i} W^\top J^\top,
% \end{equation}
%
where $\bm{J}$ is the Jacobian of the affine approximation of the projective transformation and $\bm{W}$ is the viewing transformation. 
% to We splat 3D Gaussians on the image plane by approximating the projection of the integral of $G$ function along the depth dimension of the 3D Gaussian into a 2D Gaussian influence function in pixel coordinates. The center of the Gaussian is splatted using the standard point rendering formula,
% $$\mu^{\textrm{2D}} = K \left((E \mu) / (E \mu)_z\right)$$
% where the 3D Gaussian center $\mu$ is projected into a 2D image by multiplication with the world-to-camera extrinsic matrix $E$, z-normalization, and multiplication by the intrinsic projection matrix $K$. The 3D covariance matrix is splatted into 2D using the formula from \cite{zwicker2001surface}:
% $$\Sigma^{\textrm{2D}} = J E \Sigma E^T J^T$$
% where $J$ is the Jacobian of the point projection formula above, i.e. $\partial \mu^{\textrm{2D}} / \partial \mu$.
%
The color of a pixel is calculated via alpha blending the $N$ Gaussians contributing to a given pixel: 
%
\begin{equation}
    C = \sum_{j = 1}^{N} c_j \alpha_{j} \prod_{k=1}^{j-1} (1 - \alpha_{k}),
\end{equation}
%
where the Gaussians are sorted from close to far, $c_j$ is the color obtained by evaluating the spherical harmonics given viewing transform $W$, and $\alpha_j$ is calculated from the 2D Gaussian formulation (with the covariance $\bm\Sigma_{j}^{2D}$) multiplied by its opacity $o_j$.
%
The rendering process is differentiable, which we take advantage of to learn our human model.

% The influence function $f$ can now be evaluated in 2D for each pixel for each Gaussian. The influence of all Gaussians on this pixel can be combined by sorting the Gaussians in depth order and performing front-to-back volume rendering using the volume rendering formula (the same as is used in NeRF \cite{mildenhall2020nerf}):
% $$C_{\text{pix}} = \sum_{i \in \mathcal{S}} c_i f^{\textrm{2D}}_{i, \textrm{pix}} \prod_{j=1}^{i-1} (1 - f^{\textrm{2D}}_{j, \textrm{pix}})$$
% where the final rendered color ($C_{\text{pix}}$) for each pixel is a weighted sum over the colors of each Gaussian ($c_i = \begin{bmatrix} r_i &  g_i &  b_i \end{bmatrix}^T$), weighted by the Gaussian's influence on that pixel $f^{\textrm{2D}}_{i, \textrm{pix}}$ (the equivalent of the formula for $f_i$ in 3D except with the 3D means and covariance matrices replaced with the 2D splatted versions), and down-weighted by an occlusion (transmittance) term taking into account the effect of all Gaussians in front of the current Gaussian. 

% \ar{Move this $\downarrow$ to implementation section}
% We use the CUDA implementation by Kerbl \etal~\cite{kerbl3Dgaussians} which enables fast training and rendering speeds $\sim$50 FPS for 1080p images.
%
%
% The implementation of \cite{kerbl20233d} uses a number of graphics and CUDA optimization techniques to achieve fast rendering speeds (\eg, $50$ FPS for our scenes), which therefore also enables very fast training.


\paragraph{SMPL~\cite{SMPL:2015}} is a parametric human body model which allows pose and shape control. 
%
The \smpl model comes with a template human mesh $(\template, \bm{F})$ in the rest pose (\ie, T-pose) in the template coordinate space. 
%
$\template \, {\in} \, \mathbb{R}^{\numverts \times 3}$ are the $\numverts$ vertices on the mesh, and $\bm{F} \, {\in} \, \mathbb{N}^{\numfaces\times 3}$ are the $\numfaces$ triangles with a fixed topology. 
%
Given the body shape parameters, $\shapecoeff \, {\in} \, \shapespaceexpl$, and the pose parameters, $\posecoeff \, {\in} \, \posespaceexpl$, \smpl transforms the vertices $\template$ from the template coordinate space to the shaped space via
%
\begin{equation}
    T_S(\shapecoeff, \posecoeff) = \template + B_{S}(\shapecoeff) +  B_{P}(\posecoeff),
    \label{eq:smpl vertex}
\end{equation}
%
where $T_S(\shapecoeff, \posecoeff)$ are the vertex locations in the shaped space, $B_{S}(\shapecoeff) \, \in \, \mathbb{R}^{\numverts \times 3}$ and $B_{S}(\posecoeff) \, \in \, \mathbb{R}^{\numverts \times 3}$ are the xyz offsets to individual vertices.
%
The mesh in the shaped space fits the identity (\eg, body type) of the human shape in the rest pose. 
%
%The mesh contains $\numverts$ vertices $\bm{V}\in\mathbb{R}^{\numverts\times 3}$ and $\numfaces$ triangles $\bm{F}\in\mathbb{R}^{\numfaces\times 3}$ with a fixed topology. 
%
To animate the human mesh to a certain pose (\ie, transforming the mesh to the posed space), \smpl utilizes $\numjoints$ predefined joints and Linear Blend Skinning (LBS).
%
The LBS weights $\bm{W} \, {\in} \, \mathbb{R}^{\numjoints {\times} \numverts}$ are provided by the \smpl model.
%
Given the $i$-th vertex location on the resting human mesh, $\bm{p}_i \, {\in} \, \mathbb{R}^3$, and individual posed joints' configuration (\ie, their rotation and translation in the world coordinate), $\bm{G} = [\bm{G}_1, \dots, \bm{G}_{\numjoints}]$, where $\bm{G}_k \, {\in} \, SE(3)$, the posed vertex location $\bm{v}_i$ is calculated as $\bm{v}_i = \left( \sum_{k=1}^{n_k} W_{k,i} \, \bm{G}_k \right) \bm{p}_i$, 
%
% \begin{equation}
% \small
% \begin{aligned}
%     \underbrace{\bm{v}_i}_{\textnormal{Posed vertex}} = \underbrace{\sum_{k=1}^{n_k}\bm{W}_{k,i} \, G_k }_{\textnormal{Transformation to the posed space}} {\bm{p}_i},
% \end{aligned}
% \end{equation}
%
where $W_{k,i} \, {\in} \, \mathbb{R}$ is the element in $\bm{W}$ corresponding to the $k$-th joint and the $i$-th vertex. 
%
While the \smpl model provides an animatable human body mesh, it does not model hair and clothing. 
%
Our method utilizes \smpl mesh and LBS only during the initialization phase and allows Gaussians to deviate from the human mesh to model details like hairs and clothing. 
%
% Our method only uses \smpl to initialize the human Gaussians, allowing the Gaussians to capture those details.
%
% To capture more geometric details, we use an upsampled version of \smpl with $\numverts \, {=} \, 110,210$ vertices and $\numfaces \, {=} \, 220,416$ faces.


\subsection{Human Gaussian Splats}

Given $T$ captured images and their camera poses, we first use a pretrained \smpl regressor~\cite{goel2023humans4d} to estimate the \smpl pose parameters for each image, $\bm{\theta}_1, \dots, \bm{\theta}_T$, and the body shape parameters, $\bm\beta$, that is shared across images.\footnote{We also obtain a coordinate transformation from \smpl's posed space to the world coordinate (used by the camera poses) for each frame, following Jiang \etal~\cite{jiang2022neuman}. For simplicity, we will ignore the coordinate transformation in the discussions.} 
%
Our method represents the human with 3D Gaussians and drive the Gaussians using a learned LBS. 
%
Our method outputs the Gaussian locations, rotations, scales, spherical harmonics coefficients, and their LBS weights with respect to the $\numjoints$ joints. 
%
% We also jointly optimize a different set of Gaussians representing the static scene. 
%
% \jg{This sentence needs to be clearer. Tie back to our intro about how separation of the human is a feature. Or is this "therefore in this method we focus on the human part"}The static scene is modeled using the standard 3DGS, and thus we focus on introducing the Human Gaussians. 
%
An overview of our method is illustrated in \cref{fig:overview}. 


The human Gaussians are constructed from their center locations in a canonical space, a feature triplane~\cite{Peng2020ECCV,Chan2022} $\bm{F} \, {\in} \, \mathbb{R}^{3{\times}h{\times}w{\times}d}$, and three Multi-Layer Perceptrons (MLPs) which predict properties of the Gaussians.
%
All of them are optimized per person.
%
The Human Gaussians live in a canonical space, which is a posed space of \smpl where the human mesh performs a predefined Da-pose.
%


\paragraph{Rendering process.} Given a joint configuration $\bm G$, to render an image, for each Gaussian, we first interpolate the triplane at its center location $\bm{\mu}_i$ and get feature vectors $\bm{f}_x^i, \bm{f}_y^i, \bm{f}_z^i \, {\in} \, \mathbb{R}^d$.
%
The feature $\bm{f}^i$ representing the $i$-th Gaussians is the concatenation of $\bm{f}_x^i, \bm{f}_y^i, \bm{f}_z^i$.
%
Taking $\bm{f}^i$ as input, an appearance MLP, $D_A$, outputs the RGB color and the opacity of the $i$-th Gaussian; a geometry MLP, $D_G$, outputs an offset to the center location, $\Delta \bm{\mu}_i$, the rotation matrix $\bm{R}_i$ (parameterized by the first two columns), and the scale of three axes $\bm{S}_i$; a deformation MLP, $D_D$, outputs the LBS weights, $\bm{W}_i \, {\in} \, \mathbb{R}^{\numjoints}$ for this Gaussian.
%
The LBS uses $\bm{W}$ and the joint transformation $\bm G$ to transform the Human Gaussians, which are then combined with the Scene Gaussians and splat onto the image plane.
%
The rendering process is end-to-end differentiable. 


\paragraph{Optimization.} We optimize the center locations of the Gaussians $\bm{\mu}$, the feature triplane, and the parameters of the three MLPs.\footnote{We also follow Jiang \etal~\cite{jiang2022neuman} and adjust the per image \smpl pose parameters $\bm{\theta}$ during the optimization, since $\bm{\theta}$ are initialized by an off-the-shelf \smpl regressor~\cite{goel2023humans4d} and may contain errors (see the details in the supplementary material).}
%
The rendered image is compared with the ground-truth captured image using $\mathcal{L}_1$ loss, the SSIM loss~\cite{ssim} $\mathcal{L}_{\text{ssim}}$, and the perceptual loss~\cite{simonyan2015vgg} $\mathcal{L}_{\text{vgg}}$. 
%\ot{citation for lvgg}
%
We also render a human-only image (using only the human Gaussians on a random solid background) and compare regions containing the human in the ground-truth image using the above losses.
%
The human regions are obtained using a pretrained segmentation model~\cite{kirillov2023segmentanything}. 
%
We also regularize the learned LBS weights $\bm{W}$ to be close to those from SMPL with an $\ell_2$ loss.
%
Specifically, to regularize the LBS weights $\bm{W}$, for individual Gaussians we retrieve their $k=6$ nearest vertices on the \smpl mesh and take a distance-weighted average of their LBS weights to get $\hat{\bm{W}}$. The loss is $\mathcal{L}_{\text{LBS}} = \| \bm{W} - \hat{\bm{W}} \|_{\text{F}}^2$.

%
% We use the same losses to optimize the scene Gaussians jointly.
%
Specifically, our loss is composed of
\begin{multline}
\mathcal{L} = \underbrace{\lambda_1 \mathcal{L}_1 + \lambda_2 \mathcal{L}_{\text{ssim}} + \lambda_3 \mathcal{L}_{\text{vgg}}}_{\text{scene + human}} \\ + \underbrace{\lambda_1 \mathcal{L}^h_1 + \lambda_2 \mathcal{L}^h_{\text{ssim}} + \lambda_3 \mathcal{L}^h_{\text{vgg}}}_{\text{human}} + \lambda_4 \mathcal{L}_{\text{LBS}},
\end{multline}
%
where $\lambda_1 = 0.8$, $\lambda_2 = 0.2$, $\lambda_3 = 1.0$, $\lambda_4 = 1000$ for all scenes in the experiments.
%
We employ the Adam optimizer~\cite{KingBa15} with a learning rate of $10^{-3}$, coupled with a cosine learning rate schedule.




%
We initialize the center of the Gaussians, $\bm\mu$, at the canonical-posed \smpl mesh vertices (so we have the same number of Gaussians as the \smpl vertices at the beginning of the optimization). We pretrain the feature triplane and the MLPs to output RGB color as $[0.5, 0.5, 0.5]$, opacity $o = 0.1$, $\Delta \bm{\mu} = 0$, rotation $\bm{R}$ so that z-axis of the Gaussians align with the corresponding \smpl vertex normal, scale $\bm{S}$ as the average incoming edges' lengths, and LBS weights $\bm{W}$ as those from \smpl (since the Gaussians lie exactly on the \smpl vertices).
%
The pretraining takes 5000 iterations (1 minute on a  GeForce 3090Ti GPU).
%
We use an upsampled version of \smpl with $\numverts \, {=} \, 110,210$ vertices and $\numfaces \, {=} \, 220,416$ faces.
%
Note that the \smpl mesh and LBS weights are only used in the initialization and regularization, \ie, they are not used during testing.


During the optimization, similar to the standard 3DGS, we clone, split, and prune Gaussians, every 600 iterations, based on their loss gradient and opacity.
%
They are important steps to avoid local minima during the optimization.
%
To clone and split, we simply add additional entries in $\bm{\mu}$ by repeating existing centers (cloning) and randomly sampling within the Gaussians with respect to their current shapes (\ie, $\bm{R}$ and $\bm{S}$) (splitting).
%
To prune a Gaussian, we remove it from $\bm{\mu}$.
%
Since the new Gaussians' centers are close to the original ones on the triplane, their features are similar and thus the new Gaussians have similar shape as the originals, allowing the optimization to proceed normally.
%
To make the split Gaussians smaller, we record a base scale $s \in \mathbb{R}_+$ for each Gaussian. The base scale is initialized as 1 and every time a Gaussian is split, we divide the base scale by 1.6. The actual scale of the Gaussian is $s$ multiplied with the MLP estimate $\bm{S}$.
%
The entire optimization takes 12K iterations, and 30 minutes on a GeForce 3090Ti GPU.  At the end of the optimization, the human is represented by 200K Gaussians on average.
%

\paragraph{Test-time rendering.} Importantly, after the optimization, the 3D Gaussians can be explicitly constructed, allowing direct animation of the human Gaussians using the LBS weights.
%
In other words, we \textit{do not need to evaluate} the triplane and MLPs to render new human poses. 
%
This is a big advantage compared to methods that represent human as implicit neural fields.


% \ot{We should mention that, after we learn the model we convert the representation to explicit therefore we do not need to evaluate the NNs during rendering new poses. This is a big adventage.}

% \ar{More details.., splitting the gaussians, mlp architecture in supp mat?}

% In order to model a human body using 3DGS, we start from 3D points on a template mesh of a human body and treat them as a 3D Gaussians. 
% %Our main idea is to start from Gaussians initialized from SMPL template and learn the
% We learn the geometry, appearance, and deformation over the set of Gaussians on the template. To this end, we represent the human in canonical space using the SMPL template $\template$ which is subdivided to $\numverts$ vertices. Each vertex is associated by a 3D Gaussian modeled using Eq.~\ref{eq:gauss3d}. The gaussian means $\bm\mu_i$ are initialized from SMPL vertices directly. We initialize the Gaussian rotations $\bm{R}_i$  using the SMPL vertex normals. For scale $\bm{S}_i$ initialization, we compute the mean of edge lengths $l_{edge}$ for a given mesh vertex and set $\bm{S}_i = [l_{edge}, l_{edge}, 10^{-4}]$. Following the Kerbl \etal~\cite{kerbl3Dgaussians} opacity $o_i$ is initialized as $0.1$. 

% \ar{need to switch the GCN part}
% In order to model the personalized geometry, clothing, and deformation details, we use graph convolutional network (GCN)-based modules that operates on the surface of our Gaussian mesh and the surface triangulation of the template mesh $(\mu, R, S, F)$. Our model uses three graph convolutional decoder networks -- a geometry decoder $D_G$ that models geometric offsets in the canonical space, an appearance decoder $D_A$, and a deformation decoder $D_D$. The input to these decoder networks is learnable Gaussian embeddings $G_{embed} \in \mathbb{R}^{72}$. The network $D_G$ predicts the deformation of the Gaussian mean $\Delta \bm\mu$, rotation $\Delta \bm{R}$ and scales $\Delta \bm{S}$ in the canonical space. The appearance network $D_A$ predicts the opacity $o_i$ and the spherical harmonics coefficients which is converted to $c_i$ following Ramamoorthi et al.~\cite{ramamoorthi2001efficient} for each of the Gaussians. Finally, the deformation network $D_D$ estimates pose-correctives~\cite{SMPL:2015} $B_P$ and the weights $W$ of the linear blend skinning function. The deformation of the Gaussians from canonical space to the observation is then given by,

% % \begin{align}
% %     \bm{T}_{G} &=  {\sum_{k=1}^{n_k}\bm{W}_{k,i}G_k(\bm{\theta},J(\bm{\beta}))} \\
% %     {\bm{\mu}_i^o} &= \bm{T}_{G} \cdot (\bm\mu_i + \Delta \bm\mu_i) \\
% %     {\bm{R}_i^o &= \bm{T}_{G} \cdot (\bm{R}_i \cdot \Delta \bm{R}_i)}
% % \end{align}

% \begin{align*} 
%     \bm{T}_{G} &= \sum_{k=1}^{n_k}\bm{W}_{k,i}G_k(\bm{\theta},J(\bm{\beta})) \\
%     \bm{\mu}_i^o &= \bm{T}_{G} \cdot (\bm\mu_i + \Delta \bm\mu_i) \\
%     \bm{R}_i^o &= \bm{T}_{G} \cdot (\bm{R}_i \cdot \Delta \bm{R}_i)
% \end{align*}

% where $\bm{T}_G$ is the global transformation matrix obtained by linearly blending the SMPL joint transformations. Here LBS weights $\bm W$ and pose-correctives $\bm{B}_P$ are estimated by deformation decoder $D_D$. $\bm{\mu}_i^o$ and $\bm{R}_i^o$ denote the deformed Gaussian parameters.

% \paragraph{Splitting the Gaussians}
% \ar{TBA}


% \subsection{Problems to deforming 3D Gaussians}

% \rick{we need to highlight the problems of using 3D gaussians to model/animate humans. In other words, we need to show why it is not a simple A+B method. It allows us to emphasize our contributions on pose-dependent deformation correction, GCN, and regularizations.  Please consider adding a figure to showcase with and without the addons.}

% \textcolor{red}{mention spherical harmonics rotation}

% \paragraph{GCN.}
% In order to human body deformation, we use a graph convolutional network (GCN) that operates on the surface of our gaussian mesh parametrizes by Gaussian means and the surface triangulation of the template mesh $(\mu, R, S, F)$. Our model uses three graph convolutional decoder networks -- a geometry decoder $D_G$ that models geometric offsets in the canonical space, an appearance decoder $D_A$, and a deformation decoder $D_D$. The network $G$ predicts the deformation of the Gaussian mean $\Delta \mu$, rotation $\Delta R$ and scales $\Delta S$ in the canonical space. The appearance network predicts the opacity $o_i$ Spherical Harmonics coefficients ... such that $c_i = SH_basis (\bm{n})$ following Ramamoorthi et al.~\cite{ramamoorthi2001efficient} for each of the Gaussians.  Finally, the deformation network $D_D$ pose-correctives~\cite{SMPL:2015} $B_P$ and the weights $W$ of the linear blend skinning function. The deformation of the Gaussians from canonical space to the observation is then given by,

% \begin{equation}
%     {\bm{\mu}_i^o} = {\sum_{k=1}^{n_k}\bm{W}_{k,i}G_k(\bm{\theta},J(\bm{\beta}))\cdot
%         \begin{bmatrix}
%      \bm{I} &  \bm{o}_i+ \bm{b}_i \\
%       \bm{0} &  1 
%   \end{bmatrix}}\cdot{\mu_i + \Delta \mu_i},
% \end{equation}


% \subsection{Joint Human-Scene Optimization}


% \paragraph{Initialization.}
% We use COLMAP~\cite{colmapschoenberger2016mvs, colmapschoenberger2016sfm} structure-from-motion to get camera parameters and a sparse scene point cloud. We use an off-the-shelf \smpl regressor~\cite{which_regressor} to obtain \smpl pose parameters $\posecoeff$ for each frame and the shared shape parameters $\shapecoeff$. Since the estimated $\posecoeff$ are not accurate, we optimize them using 2D joints estimated from \cite{xu2022vitpose} as detailed in the supplementary material. We then align the SMPL posed space to the world coordinate with a per-frame coodinate transformation, following Jiang \etal~\cite{jiang2022neuman}.

% \paragraph{Optimization.}
% We optimize the scene and human avatar jointly. We observed that joint optimization helps to alleviate the floating Gaussian artifacts for the scene and get sharper boundaries for the human model ~\ar{which we will discuss in \S 4.X?/suppmat}. Following Kerbl ~\etal~\cite{kerbl3Dgaussians} we use the adaptive control of Gaussians to get a denser set that better represents the scene. Please refer to Section 5.2 of \cite{kerbl3Dgaussians} for more details.

% During optimization, we employ image-based rendering losses together with human-specific regularizations during training. 

% \begin{equation}
%     \mathcal{L} = \mathcal{L}_1 + \mathcal{L}_{ssim} + \mathcal{L}_{vgg} + \mathcal{L}_{proj} + \mathcal{L}_{rep}
% \end{equation}

% Here $\mathcal{L}_1$ is the $\mathcal{L}_1$ loss between the rendered and ground-truth image, $\mathcal{L}_{ssim}$ is the $\mathcal{L}_{ssim}$ loss between the rendered and ground-truth image, and $\mathcal{L}_{vgg}$ is the perceptual loss between the rendered and ground-truth image. We use two regularizer losses on the human Gaussians $\mathcal{L}_{proj}$ and $\mathcal{L}_{rep}$. $\mathcal{L}_{proj}$ essentially enforces the Gaussian means to be $\bm{\mu}$ close to the local tangent plane of neighboring points by computing the PCA in a local neighborhood. \ar{$\mathcal{L}_{proj} equation?$}
% $\mathcal{L}_{rep}$ enforces Gaussians to be close to each other in a local neighborhood. \ar{$\mathcal{L}_{rep}$ equation?} We have $\lambda$ coefficients for each loss term presented here, but we removed them from the equation for brevity.

% Finally, we use Adam optimizer with learning rate $10^{-3}$ for decoder networks and $G_{embed}$ with a cosine learning rate scheduling~\cite{}.




% 3D Gaussian splatting is recently proposed by Kerbl \etal~\cite{kerbl3Dgaussians}. 

% We follow the \gstd~\cite{kerbl3Dgaussians} pipeline which is summarized below. Different from \gstd, we represent the humans in canonical space with \gauss We add a -- human deformation module (?) and .. ... , to improve .. .. .
% Our \gauss are defined by a full 3D covariance matrix $\Sigma$ defined in world space centered at point (mean) $\mu$:

% Following \cite{kerbl3Dgaussians}, the image formation model is given by
% \begin{equation}
%     C = \sum_{i \in N} c_i \alpha_i \prod_{j=1}^{i-1} 1 - \alpha_j
% \end{equation}
% where $C$ is the accumulated color at a pixel by blending $N$ points that overlap the pixel; and $\alpha_i$ is a function of opacity per point and the covariance of the 