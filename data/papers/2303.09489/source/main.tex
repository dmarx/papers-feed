\documentclass[11pt]{article}
\pdfoutput=1

% Recommended, but optional, packages for figures and better typesetting:
\usepackage{microtype}
\usepackage{graphicx}
\usepackage{subfigure}
\usepackage{booktabs} % for professional tables
\usepackage{amssymb}
\usepackage{bm}
\usepackage{bbm}
\usepackage{amsmath}  % Define \boldsymbol (in amsbsy too) and align
\usepackage{amsthm}
\usepackage{xcolor}
\usepackage{textcomp}
\usepackage{multirow}
\usepackage{mathtools}
\usepackage[margin=1in]{geometry}
\usepackage{authblk}
\usepackage[numbers, sort]{natbib}
\usepackage{xspace}
\usepackage{comment}
\usepackage{thmtools, thm-restate}  % for repeating thms in appendix
\usepackage{verbatim}
\usepackage[T1]{fontenc}

\usepackage{tikz}
\usetikzlibrary{arrows}
\usetikzlibrary{positioning}


% \usepackage{hyperref}
\definecolor{tabblue}{HTML}{5555CC}
\usepackage[hidelinks,colorlinks=true,linkcolor=tabblue,citecolor=tabblue]{hyperref}

% \usepackage[hidelinks,colorlinks=true,linkcolor=tabblue,citecolor=tabblue]{hyperref}

% % \usepackage[hidelinks,colorlinks=true,linkcolor=blue,citecolor=blue]{hyperref}
\usepackage{url}

\usepackage[capitalize,noabbrev]{cleveref}
% \usepackage{subcaption}
\usepackage{caption}
\usepackage{pifont}
\usepackage{makecell}
\usepackage{tabularx}
\usepackage{float}
% % Tables
\usepackage{adjustbox}
\usepackage{varwidth}
%

\DeclareMathOperator*{\argmax}{arg\,max}
\DeclareMathOperator*{\argmin}{arg\,min}
\DeclareMathOperator*{\maxpool}{Max-Pool}
\DeclareMathOperator*{\sech}{sech}
\DeclareMathOperator*{\softmax}{Softmax}
\DeclareMathOperator*{\pool}{Pool}

\DeclareMathOperator{\diag}{diag}
\DeclareMathOperator{\sign}{sign}
\DeclareMathOperator{\Cov}{Cov}
\DeclareMathOperator{\Det}{Det}
\DeclareMathOperator{\parents}{Pa}  %
\DeclareMathOperator{\Tr}{Tr}
\DeclareMathOperator{\Var}{Var}

\DeclareRobustCommand{\norm}[1]{\ensuremath{\left\lVert#1\right\rVert}}
\DeclareRobustCommand{\pnorm}[2]{\ensuremath{\left\lVert#1\right\rVert}_{#2}}
\DeclarePairedDelimiter{\ceil}{\lceil}{\rceil}
\DeclarePairedDelimiter{\floor}{\lfloor}{\rfloor}

\DeclareRobustCommand{\KLD}[2]{\ensuremath{\textnormal{KLD}\left(#1\;\|\;#2\right)}}
\DeclareRobustCommand{\JSD}[2]{\ensuremath{\textnormal{JSD}\left(#1\;\|\;#2\right)}}
\DeclareRobustCommand{\cossim}[2]{\ensuremath{\textnormal{cos}\left(#1,#2\right)}}

\newcommand{\E}{\mathbb{E}}
\newcommand{\Ls}{\mathcal{L}}
\newcommand{\R}{\mathbb{R}}
\newcommand{\emp}{\tilde{p}}
\newcommand{\lr}{\alpha}
\newcommand{\reg}{\lambda}
\newcommand{\rect}{\textnormal{rectifier}}
\newcommand{\sigmoid}{\sigma}
\newcommand{\softplus}{\zeta}
\newcommand{\standarderror}{\textnormal{SE}}
\newcommand{\normlzero}{L^0}
\newcommand{\normlone}{L^1}
\newcommand{\normltwo}{L^2}
\newcommand{\normlp}{L^p}
\newcommand{\normmax}{L^\infty}
\newcommand{\fullconv}{\,*\,}

\newcommand{\bb}[1]{{\boldsymbol{#1}}}
\newcommand{\mc}[1]{{\mathcal{#1}}}

\newcommand{\train}{\mathcal{D}}
\newcommand{\valid}{\mathcal{D_{\mathrm{valid}}}}
\newcommand{\test}{\mathcal{D_{\mathrm{test}}}}
\def\eps{{\epsilon}}

\newcommand{\pdata}{p_{\rm{data}}}

\newcommand{\ptrain}{\hat{p}_{\rm{data}}}
\newcommand{\Ptrain}{\hat{P}_{\rm{data}}}

\newcommand{\pmodel}{p_{\rm{model}}}
\newcommand{\Pmodel}{P_{\rm{model}}}
\newcommand{\ptildemodel}{\tilde{p}_{\rm{model}}}

\newcommand{\pencode}{p_{\rm{encoder}}}
\newcommand{\pdecode}{p_{\rm{decoder}}}
\newcommand{\precons}{p_{\rm{reconstruct}}}

\newcommand{\attn}{\mathrm{Attn}}
\newcommand{\sigmoidattn}{\mathrm{SigmoidAttn}}
\newcommand{\softmaxattn}{\mathrm{SoftmaxAttn}}
\newcommand{\mlp}{\mathrm{MLP}}

\newcommand{\reading}[2]{{#1}{\pm{#2}}}

% \newcommand{\ourmethod}{{\tt S4caster}}
\newcommand{\ourmethod}{\textsc{SpaceTime}}
\newcommand{\ourmethodunit}{\textsc{SpaceTime} layer}
\newcommand{\numberMonashTasks}{34}  % {58}
\newcommand{\numberInformerTasks}{16}

% Style to highlight edits, comments, todos, suggestions
\newcommand{\working}[1]{\textcolor{purple}{\authnote{}{#1}}}
\newcommand{\MZ}[1]{\textcolor{cyan}{\authnote{(MZ: }{#1})}}
\newcommand{\KS}[1]{\textcolor{orange}{\authnote{(KS: }{#1})}}
\newcommand{\tempcite}[1]{\textcolor{red}{(cite)}}

% Additional header / formatting
\newcommand{\header}[1]{\textbf{#1.}}
\newcommand{\subheader}[1]{\textbf{\textit{#1.}}}
\usepackage{enumitem}  % margins in lists

\usepackage{minitoc,wrapfig}
\renewcommand \thepart{}
\renewcommand \partname{}

% Other text abbreviations
\makeatletter
\DeclareRobustCommand\onedot{\futurelet\@let@token\@onedot}
\def\@onedot{\ifx\@let@token.\else.\null\fi\xspace}

\def\eg{\emph{e.g.},} 
\def\Eg{\emph{E.g.},}
\def\ie{\emph{i.e.},} 
\def\Ie{\emph{I.e.},}
\def\cf{\emph{c.f.},} 
\def\Cf{\emph{C.f.},}
\def\st{\emph{s.t}\onedot}
\def\etc{\emph{etc}\onedot} 
\def\vs{\emph{vs}\onedot}
\def\wrt{w.r.t.} 
\def\dof{d.o.f\onedot}
\def\iid{i.i.d\onedot} 
\def\wolog{w.l.o.g\onedot}
\def\etal{\emph{et al}\onedot}

% tikz
% colors and graphics
\usepackage{graphicx}
\usepackage{tikz}
\usepackage{tikz-cd}
\usepackage{hf-tikz}
\usepackage{pgfplots} 
\pgfplotsset{compat=1.17} 
\pgfplotsset{
        table/search path={figures/drawings},
    }
\usepackage{rotating}
\usetikzlibrary{fadings}
\usetikzlibrary{shapes, arrows, fit, backgrounds, arrows.meta}

\usetikzlibrary{matrix}
\usetikzlibrary{shadows.blur}
\usetikzlibrary{patterns, tikzmark}
\usetikzlibrary{decorations.pathreplacing, calc, decorations.markings,}

% Tables
\usepackage{adjustbox}
\usepackage{varwidth}

%% Tables
\usepackage{booktabs}
\usepackage{multirow}
\usepackage[normalem]{ulem}
\useunder{\uline}{\ul}{}
%% Table colors
\usepackage{colortbl} 
\definecolor{Gray}{gray}{0.90}  
% 0.85
\definecolor{LightCyan}{rgb}{0.7,1,1}
\definecolor{White}{rgb}{1,1,1}
\newcolumntype{a}{>{\columncolor{Gray}}c}
\newcolumntype{b}{>{\columncolor{LightCyan}}c}
\newcolumntype{n}{>{\columncolor{White}}c}
% \usepackage[table,xcdraw]{xcolor}
\usepackage{rotating}



\newcommand{\fcircle}[2][red,fill=red]{\tikz[baseline=-0.5ex]\draw[#1,radius=#2] (0,0.03) circle ;}


% Algorithms
\usepackage{algorithm}
\usepackage{algpseudocode}

% Figures
% \usepackage{subfig}

% Checkmark and Xmark
\newcommand{\cmark}{\ding{51}}%
\newcommand{\xmark}{\ding{55}}%

\newcommand*{\ShowNotes}{}

\newif\ifarxiv

%
\title{Effectively Modeling Time Series with \\Simple Discrete State Spaces}
%
\author{Michael Zhang$^*$, Khaled Saab\thanks{ Equal Contribution. Order determined by forecasting competition.}\;, Michael Poli,  Tri Dao, Karan Goel, and Christopher R\'{e} \\
% about author (webpage, alternative address)---\emph{not} for acknowledging
% funding agencies.  Funding acknowledgements go at the end of the paper.} \\
Stanford University \\
\vspace{0.25cm}
\texttt{mzhang@cs.stanford.edu}, \texttt{\{ksaab,poli\}@stanford.edu}, \texttt{\{tridao,kgoel,chrismre\}@cs.stanford.edu}
% \AND
% Coauthor \\
% Affiliation \\
% Address \\
% \texttt{email}
}
%
% \iclrfinalcopy % Uncomment for camera-ready version, but NOT for submission.
\begin{document}
%
\maketitle

%
\doparttoc
\faketableofcontents
%

\begin{abstract}
Language models (LMs), like other neural networks, often favor shortcut heuristics based on surface-level patterns.
Although LMs behave like n-gram models early in training, they must eventually learn hierarchical syntactic representations to correctly apply grammatical rules out-of-distribution (OOD).
In this work, we use case studies of English grammar to explore how complex, diverse training data drives models to generalize OOD. We construct a framework that unifies our understanding of random variation with training dynamics, rule selection with memorization, and data diversity with complexity. 
We show that these factors are nuanced, and that intermediate levels of diversity and complexity lead to inconsistent behavior across random seeds and to unstable training dynamics. 
Our findings emphasize the critical role of training data in shaping generalization patterns and illuminate how competing model strategies lead to inconsistent generalization outcomes across random seeds. Code is available at \url{https://github.com/sunnytqin/concept_comp.git}.

\end{abstract}

\section{Introduction}
\label{sec:introduction}

\begin{wrapfigure}{r}{0.5\textwidth}
\vspace{-6mm}
\begin{center}
    \includegraphics[width=0.5\textwidth]{images/cover.pdf}
  \end{center}
  \vspace{-4mm}
  \caption{\textbf{Overview of \implname.} In training, we tune the singular values of the weight matrices to generate a set of ``expert'' vectors specializing in different tasks. In inference, a two-pass process is adopted where the first applies the expert and the second generates the answer.}
  \label{fig:cover}
  \vspace{-4mm}
\end{wrapfigure}

Self-adaptive large language models (LLMs) would represent a significant advancement in artificial intelligence, enabling real-time adaptation to various tasks and contexts.
While compositionality and scalability are crucial for effective adaptation, current LLM training methodologies fall short of achieving both these properties simultaneously.
Our research aims to present a solution to address these gaps.

In principle, the first step toward achieving self-adaptive LLMs can be realized through the development of specialized expert modules, each fine-tuned~\citep{kaplan2020scaling} via techniques such as low-rank adaptation (LoRA)~\citep{hu2021lora}. 
However, several challenges need to be addressed to make this approach both scalable and compositional: (1) multiple expert modules significantly increase the number of parameters; (2) expert modules are often prone to overfitting; and (3) flexible composition of these experts is still an open problem.

To overcome these limitations, we first propose \svdacro, a novel parameter-efficient fine-tuning (PEFT) method to obtain effective building blocks for self-adaptation.
\svdacro works by extracting and selectively tuning only the singular values within the model's weight matrices.
By focusing on this essential and principled parameterization, our approach mitigates the risk of overfitting, drastically reduces computational demands, and allows for inherent compositionality.

We then introduce our full \implname framework, which entails a two-pass inference mechanism to produce dynamically adapted weights targeted for the test-time conditions (Figure~\ref{fig:cover}).
We design three different adaptation strategies that can be used within \implname, which we show provide monotonic performance benefits with increasing access to the test-time conditions.
We evaluate \svdacro and the full \implname framework through extensive experiments across a diverse range of LLMs and tasks.
\svdacro outperforms traditional efficient fine-tuning methods like LoRA on domain-specific datasets with far fewer parameters. 
\implname further improves performance, even for out-of-distribution tasks like visual QA. 
Our analysis even shows that \implname allows the reuse of \svdacro experts across different LLMs. In summary, our key technical contributions are: 
\vspace{-2mm}
\begin{itemize}
\item The development of \implname as a pivotal self-adaptation framework for LLMs, providing a blueprint to adapt the behavior of LLMs from a growing set of pre-trained skills.
\item The introduction of \svdacro, a novel PEFT method trainable with RL on small datasets, producing compact expert vectors with inherent compositionality.
\item The implementation of three adaptation strategies, effectively dispatching \svdacro-trained experts with properties designed to cope with different deployment scenarios.
\end{itemize}

\vspace{-2mm}
\section{Hybrid Retrieval Strategy}
\label{appendix_sec:preliminaries}

A list of $k$ passages by merging returned passages from both a $\mathbf{C} = \left \{c_1, c_2, \ldots, c_C \right \}$ index of textual passages and a $\mathbf{T} = \left \{t_1, t_2, \ldots,t_T: t_j = \left ( s_j, p_j, o_j \right ) \right \}$ index representing a set of triples associated with the passages in $\mathbf{C}$, using Reciprocal Rank Fusion (RRF) \cite{Cormack2009}, can be obtained, as follows:
\begin{align}
h^k_{\text{base}}\left( \mathbf{q}, {\mathbf{C} \cup \mathbf{T}} \right) 
    &= \text{RRF} \Big ( h^k_{\text{base}}\left( \mathbf{q}, {\mathbf{C}}\right), \nonumber \\
    &\quad\quad h^k_{\text{base}}\left( \mathbf{q}, {\mathbf{T}} \right) \Big ),
    \label{eq:triple_passage_index_retrieve}
\end{align}
% we multiply $k$ for the $T$ index by 5.
where $h^k_{\text{base}}\left( \mathbf{q}, {\mathbf{C}}\right)$ and $h^k_{\text{base}}\left( \mathbf{q}, {\mathbf{T}}\right)$ are the passages retrieved from $\mathbf{C}$ and $\mathbf{T}$, after a base retrieval step on each index separately. In this case, each triple $\in h^k_{\text{base}}\left( \mathbf{q}, {\mathbf{T}} \right)$ is mapped to its corresponding passage, ensuring that top-$k$ unique passages are returned after considering the triple scores in $\mathbf{T}$.


\iffalse
Given an input query $\mathbf{q'}$ \pascual{Shouldn't be use q instead of $\mathbf{q'}$?}, a \textit{baseline} retrieval step includes selecting the most relevant passages, using a combination of hybrid retrieval steps. Each hybrid retrieval search step returns top-$k$ items from an index of interest $\mathbf{R} = \left \{r_1, \ldots, r_R \right \}$ s.t. $\mathbf{R} \in \left (\mathbf{C} \cup \mathbf{T} \right ) \setminus \left (\mathbf{C} \cap \mathbf{T} \right )$ \pascual{Isn't it unnecessary to include $(\mathbf{C} \cap \mathbf{T})$? (Almost) by definition there won't be any intersection, although I guess a passage could be just a triple? Note that in the following sections we only use $\mathbf{C} \cup \mathbf{T}$} by aggregating the results of semantic search and $\text{score}_{\text{BM25}}$ using Reciprocal Rank Fusion (RRF) \pascual{Should we explain what the RRF function does, or should be cite a paper that does so?}, as follows: 
\begin{align}
h^k_{\text{hybrid}}\left( \mathbf{q'}, {\mathbf{R}}\right ) = \text{RRF}\left(h^k_{\text{dense}}, h^k_{\text{BM25}}\right),
\end{align}
where $h^k_{\text{dense}} \subseteq \mathbf{R}$ and $h^k_{\text{BM25}}\subseteq \mathbf{R}$ are functions
returning sets of items from $\mathbf{R}$, in descending order according to $\text{score}_{\text{dense}}$ and $\text{score}_{\text{BM25}}$ respectively, s.t. 
$\text{score}_{\text{dense}}  \left( \mathbf{q'}, r_j \right ) \geq\text{score}_{\text{dense}}  \left( \mathbf{q'}, r_{j+1} \right )$ $\forall r_j \in h^k_{\text{dense}}$ and $j \in \left [ 1, k - 1\right ]$ and $\text{score}_{\text{BM25}}  \left( \mathbf{q'}, r_j \right ) \geq\text{score}_{\text{BM25}}  \left( \mathbf{q'}, r_{j+1} \right )$ $\forall r_j \in h^k_{\text{BM25}}$ and $j \in \left [ 1, k - 1\right ]$ \pascual{Maybe we can simplify all this math by just saying the lists are ordered in descending order of their respective scores}.
\fi

\section{Retrieval with Synchronised Graph Expansion}
\label{sec:graph_retrieval}

\def\Tqinit{\mathbf{T}_\mathbf{q}}


\begin{figure}[thbp]
  \includegraphics[width=\columnwidth]{figures/gear-sys-fig.pdf}
  \caption{\label{fig:system_diagram}System Architecture}
\end{figure}

% Start: Zhili --------------------------


Given an input query $\mathbf{q}$, let $\mathbf{C}_\mathbf{q}' = h^k_{\text{base}}\left( \mathbf{q}, {\mathbf{C}}\right )$  be a list of passages returned by the base retriever\footnote{The choice of a base retriever within our framework is flexible, without requiring any multi-hop capabilities.}.
Given this initially retrieved list of passages, $\mathbf{C}_\mathbf{q}'$, our goal is to derive relevant multi-hop contexts (passages) by retrieving a sub-graph of triples that interconnect their source passages. There are two challenges for materialising such sub-graph retrieval: \begin{inparaenum}[(i)]\item how to locate initial triples (i.e. starting nodes) $\Tqinit$, and \item how to expand the graph based on initial triples while reducing the search space\end{inparaenum}. The following sections address these challenges respectively, within \gear.



\subsection{Knowledge Synchronisation}
\label{subsection:knowledge_syncro}
\def\linkTriple{\texttt{tripleLink}}

We describe a knowledge \textbf{Sync}hronisation (\textbf{Sync}) process for locating initial nodes for graph expansion. We first employ an LLM to \texttt{read} $\mathbf{C}_\mathbf{q}'$ (see Appendix~\ref{subsec:online_retrieval_prompts}) and summarise knowledge triples that can support answering the current query $\mathbf{q}$, as defined:
\begin{align}
    \mathbf{T}_\mathbf{q}' = \texttt{read}\left (\mathbf{C}_\mathbf{q}', \mathbf{q}\right ).
    \label{eq:proximal_read}
\end{align}
$\mathbf{T}_\mathbf{q}'$ is a collection of triples to which we refer as \textit{proximal triples}. Initial nodes $\Tqinit$ for graph expansion can then be identified by linking each triple in $\mathbf{T}_\mathbf{q}'$ to a triple in $\mathbf{T}$, using the \linkTriple{} function:
\begin{align}
    \Tqinit =\left \{t_i | t_i = \linkTriple(t_i') ~ \forall t_i' \in \mathbf{T}_\mathbf{q}'\right \}.
\end{align}
The implementation of \linkTriple{} can vary. However, in this paper we consider it to be simply retrieving the most similar triple from $\mathbf{T}$.



\begin{algorithm}[ht]
\textbf{Input:} $\mathbf{q}$: query \\
\hspace*{3em} $b$: beam size \\
\hspace*{3em} $l$: maximum length \\
\hspace*{3em} $\mathrm{score}(\cdot, \cdot)$: scoring function \\
\hspace*{3em} $\{t_1, t_2, \ldots, t_n\}$: initial triples \\
\hspace*{3em} $\gamma$: hyperparameter for diversity


\begin{algorithmic}[1]
\State $B_0 \gets [\;]$
\For{$t \in \{t_1, t_2, ..., t_n\}$}
    \State $s \gets \mathrm{score}(\mathbf{q}, [t])$
    \State $B_0.\mathrm{add}(\langle s, [t] \rangle)$
\EndFor

\State $B_0 \gets \mathrm{top}(B_0, b)$


\For{$i \in \{1, \dots, l - 1\}$}
    \State $B \gets [\;]$
    
    \For{$\langle s, T \rangle \in B_{i-1}$}
        \State $V \gets [\;]$

        \For{$t \in \mathrm{get\_neighbours}(T.\mathrm{last}())$}
            \If{$\mathrm{exists}(t, B_{i-1})$}
                \State \textbf{continue}
            \EndIf
            
            \State $s' \gets s + \mathrm{score}(\mathbf{q}, T \circ t)$ ~ \texttt{\# concat} 
            \State $V.\mathrm{add}(\langle s', T \circ t \rangle)$
        \EndFor

        \State $\mathrm{sort}(V, \mathrm{descending})$

        \For{$n \in \{0, \dots, V.\mathrm{length()} - 1\}$}
            \State $\langle s', T \circ t \rangle \gets V[n]$
            \State $s' \gets s' \times e^{- \frac{\mathrm{min}(n, \gamma)}{\gamma}}$
            \State $B.\mathrm{add}(\langle s', T \circ t \rangle)$
        \EndFor
        
    \EndFor
    \State $B_i \gets \mathrm{top}(B, b)$
    
\EndFor

\State \Return $B_i$
\end{algorithmic}

\caption{Diverse Triple Beam Search}
\label{alg:beam_search}
\end{algorithm}

\subsection{Diverse Triple Beam Search}

We borrow the idea of constructing reasoning triple chains \cite{Fang2024} for expanding the graph, and present a retrieval algorithm: \textit{Diverse Triple Beam Search} (see Alg.~\ref{alg:beam_search}). 

We maintain top-$b$ sequences (beams) of triples and the scores at each step are determined by a scoring function. In this paper, we focus on leveraging a dense embedding model to compute the cosine similarity between embeddings of the query and a candidate sequence of triples, leaving other implementations of the scoring function for future work (see Section~\ref{sec:limitations}).

Considering all possible triple extensions at each step, in a Viterbi decoding fashion, would be intractable due to the size of $\mathbf{T}$. Consequently, we define the neighbourhood of a triple as the set of triples with shared head or tail entities (i.e. $\mathrm{get\_neighbours}$ in Alg.~\ref{alg:beam_search}). During each expansion step, we only consider neighbours of the last triple in the sequence, and avoid selecting previously visited triples (i.e. $\mathrm{exists}$ in Alg.~\ref{alg:beam_search}) to further reduce the search space.

While regular beam search can reduce the search space, it is prone to producing high-likelihood sequences that differ only slightly from one another \cite{Ippolito2019, Vijayakumar2018}. Our algorithm increases the diversity across beams to improve the recall for retrieval. In detail, for each beam, we sort candidate sequences extended from that beam in descending order, and weight their scores based on their relative positions. Candidate sequences that are ranked lower, within a beam, will receive smaller weights. Consequently, the resulting top-$b$ beams at each step are less likely to share the same starting sequence. 

The top-$b$ returned sequences are flattened in a breadth-first order. Each triple in the resulting list is then mapped to its source passage. This alignment between triples and passages is described in more detail in Section~\ref{sec:preliminaries}. Let $\widetilde{\mathbf{C}}_\mathbf{q}$ be the list of unique passages after alignment. The output of our graph expansion is then given by the Reciprocal Rank Fusion (RRF) \cite{Cormack2009} of $\widetilde{\mathbf{C}}_\mathbf{q}$ and the initial $\mathbf{C}_\mathbf{q}'$ list of passages :
\begin{align}
    \mathbf{C}_{\mathbf{q}} = \mathrm{RRF}\left(\widetilde{\mathbf{C}}_\mathbf{q}, \mathbf{C}_\mathbf{q}'\right ).
\end{align}
We refer to this graph-based method of retrieving relevant passages as \textbf{Sync}ronised \textbf{G}raph \textbf{E}xpansion (\textbf{SyncGE}).


\section{Multi-step Extension}


While SyncGE can enhance a base retriever with multi-hop context, some queries inherently require multiple steps to gather all necessary evidence. We materialise \gear by incorporating an agent with multi-turn capabilities, capable of interacting with the graph-retriever described above. We focus on:
\begin{itemize}
\item maintaining a gist memory of proximal knowledge obtained throughout the different steps 
\item incorporating a similar synchronisation process 
that summarises retrieved passages in proximal triples to be stored in this multi-turn gist memory
\item determining if additional steps are needed for answering the original input question
\end{itemize}
%
Within this multi-turn setting, the original input question $\mathbf{q}$ is iteratively decomposed into simpler queries: $\mathbf{q}^{(1)}, \ldots, \mathbf{q}^{(n)}$, where $\mathbf{q}^{(1)} = \mathbf{q}$ and $n \in \mathbb{N}$ represents the number of the current step.
For each query $\mathbf{q}^{(n)}$, we use the graph retrieval method introduced in Section~\ref{sec:graph_retrieval} in order to retrieve relevant passages $\mathbf{C}_{\mathbf{q}^{(n)}}$.



\subsection{Gist Memory Constructor}
To facilitate the multi-step capabilities of our agent, we introduce a \textit{gist memory}, $\mathcal{G}^{(n)}$, which is used for storing knowledge as an array of proximal triples. At the beginning of the first iteration, the gist memory is empty. During the $n$-th iteration, similar to the knowledge synchronisation module explained in Section~\ref{subsection:knowledge_syncro}, we employ an LLM to read a collection of retrieved paragraphs $\mathbf{C}_{\mathbf{q}^{(n)}}$ and summarise their content with proximal triples:

\begin{align}
\mathbf{T}_{\mathbf{q}^{(n)}}^{\mathcal{G}} = 
\begin{cases} 
    \texttt{read}\left(\mathbf{C}_{\mathbf{q}^{(n)}}, \mathbf{q} \right), & \text{if } n = 1 \\
    \texttt{read}\left(\mathbf{C}_{\mathbf{q}^{(n)}}, \mathbf{q}\textcolor{blue}{, \mathcal{G}^{(n-1)}} \right), & \text{if } n \geq 2
\end{cases}
\label{eq:proximal_read_agent}
\end{align}


Apart from the first iteration where Eq.~\ref{eq:proximal_read} and ~\ref{eq:proximal_read_agent} are identical, the inclusion of the memory in the \texttt{read} operation differentiates the construction of proximal triples produced at the subsequent steps compared to the ones from Eq.~\ref{eq:proximal_read}. $\mathcal{G}^{(n)}$ maintains the aggregated content of proximal triples s.t. 
\begin{align}
\mathcal{G}^{(n)} = \left[ \mathbf{T}_{\mathbf{q}^{(1)}}^{\mathcal{G}}  \circ \cdots \circ \mathbf{T}_{\mathbf{q}^{(n)}}^{\mathcal{G}} \right],
\end{align}where $\circ$ defines the concatenation operation. The triple memory serves as a concise representation of all the accumulated evidence, up to the $n$-th step. 

We believe the process introduced by the \texttt{read} step along with the information storage paradigm served by the gist memory, aligns well with the communication between the hippocampus and neocortex. The combination of the two establishes the synergetic behaviour between our graph retriever and the LLM that we seek to achieve within \gear.



\subsection{Reasoning for Termination}
After $\mathcal{G}^{(n)}$ is updated, we check the sufficiency of the accumulated evidence, within it, for answering the original question. This is achieved with the following LLM reasoning step:
\begin{align}
\mathbf{a}^{(n)}, \mathbf{r}^{(n)}   = \texttt{reason}(\mathcal{G}^{(n)}, \mathbf{q}),
\end{align}
% We can also call it 'sufficiency' instead of 'answerability'. I do not really have a preference.
where $\mathbf{a}^{(n)}$ denotes the query's answerability given the available evidence in $\mathcal{G}^{(n)}$, and $\mathbf{r}^{(n)}$ represents the reasoning behind this determination. When the query is deemed answerable, the system concludes its iterative process.



\subsection{Query Re-writing}
The query re-writing process leverages an LLM that incorporates three key inputs: the original query $\mathbf{q}$, the accumulated memory, and crucially, the reasoning output $\mathbf{r}^{(n)}$ from the previous step. This process can be formally expressed as:
\begin{align}
\mathbf{q}^{(n+1)} = \texttt{rewrite}\left (\mathcal{G}^{(n)}, \mathbf{q}, \mathbf{r}^{(n)} \right),
\end{align}
where $\mathbf{q}^{(n+1)}$ represents the updated query, which serves as input for the retriever in the next iteration.\\
\subsection{After Termination}
\gear aims to return a single ranked list of passages. Given the final gist memory $\mathcal{G}^{(n)}$ upon termination, we link each proximal triple in $\mathcal{G}^{(n)}$ to a list of passages as follows:
\begin{align}
    \mathbf{C}_{t_j} = \texttt{passageLink}\left(t_j, k\right),
\end{align}
where $j \in \left \{1, \dots, \vert\mathcal{G}^{(n)}\vert \right \}$. Similar to \texttt{tripleLink}, \texttt{passageLink} is implemented by retrieving passages with a triple as the query (see Appendix~\ref{appendixpara:passage_link}). The final list of passages returned by \gear is the RRF of the resulting linked passages and passages retrieved across iterations:
\begin{align}
\mathbf{C}_\mathbf{q}^{(n)} = \mathrm{RRF}\big(&\mathbf{C}_{t_1}, \ldots,\mathbf{C}_{t_{\vert\mathcal{G}^{(n)}\vert}}, \nonumber\\
    &\mathbf{C}_{\mathbf{q}^{(1)}}, \ldots, \mathbf{C}_{\mathbf{q}^{(n)}} \big).
\end{align}

All relevant prompts for the \texttt{read}, \texttt{reason} and \texttt{rewrite} steps are provided in Appendix~\ref{subsec:online_retrieval_prompts}.

\section{Experiments}
\label{sec:experiments}
\begin{figure}[h]
\centering
\includegraphics[width=\textwidth]{figures/train_nll_softmax_vs_sigmoid_v4.pdf}
\caption{Train losses comparing $\sigmoidattn$ with $\softmaxattn$.}
\label{fig:summary_nll}
\end{figure}
To empirically validate $\sigmoidattn$, we evaluate across several domains: supervised image classification using vision transformers \citep{DBLP:conf/iclr/DosovitskiyB0WZ21}, self-supervised image representation learning with SimCLR \citep{DBLP:conf/icml/ChenK0H20, DBLP:conf/icml/ZhaiLLBR0GS23}, Bootstrap Your Own Latent (BYOL) \citep{DBLP:conf/nips/GrillSATRBDPGAP20, DBLP:conf/nips/BusbridgeRALDCW23} and Masked AutoEncoders (MAE) \citep{DBLP:conf/cvpr/HeCXLDG22} as well as automatic speech recognition (ASR) \citep{synnaeve2019end,conformer} and auto-regressive language modeling (LM) \citep{DBLP:conf/nips/BrownMRSKDNSSAA20}. We also validate sequence length generalization on TED-LIUM v3~\citep{hernandez2018ted} for ASR and in small scale synthetic experiments in \cref{sec:a_se_pair_repeat_prob}.
Across all these domains and algorithms, we demonstrate that $\sigmoidattn$ matches the performance of $\softmaxattn$ (\cref{fig:summary_nll,fig:test_top1_results}), while offering training and inference speed-ups as highlighted in \cref{sec:FlashSigmoidHardwareAwareImplementation}. Empirically we make the following observations:
\begin{enumerate}[itemsep=0pt,leftmargin=*]
    \item $\sigmoidattn$ is effective for vision tasks without a bias (except MAE), but relies on LayerScale to match the performance of the baseline $\softmaxattn$ (\cref{fig:imagenet_top_1_ablations}-a) in a hyper-parameter free manner.\footnote{\Cref{sec:layerscale_free_sigmoid} demonstrates that supervised vision tasks using $\sigmoidattn$ without LayerScale can match baseline $\softmaxattn$ performance by relying on \emph{learnable} scalar bias and temperature: $\{b, t\} \in \mathbb{R}$.} All results presented for $\softmaxattn$ also fairly add LayerScale unless specified.
    \item LM and ASR are sensitive to the initial norm $|| \sigma(\mQ \mK^T/\sqrt{d_{qk}}) \mV ||$. Modulation is required via (a) relative positional embeddings like ALiBi \citep{DBLP:conf/iclr/PressSL22}, which reduces the initial attention norm by shifting logit mass to the zero regime under $\sigmoidattn$, (b) appropriate initialization of $b$ to achieve the same effect -- enabling usage of any positional embedding.
\end{enumerate}

\begin{figure}[htbp]
    \centering
    \begin{minipage}{0.48\textwidth}
        \centering
        \includegraphics[width=\textwidth]{figures/attn_norm_seed1000001_softmax_rope_vs_softmax_alibi_vs_sigmoid_sincos.png}    
        \captionsetup{justification=centering}
        \caption{$\sigmoidattn$ with SinCos.}
        \label{fig:rope_vs_sincos}
    \end{minipage}\hfill
    \begin{minipage}{0.48\textwidth}
        \centering        
        \includegraphics[width=\textwidth]{figures/attn_norm_seed1000001_softmax_rope_vs_softmax_alibi_vs_sigmoid_rope.png}
        \captionsetup{justification=centering}
        \caption{$\sigmoidattn$ with RoPE.}
        \label{fig:rope_vs_rope}
    \end{minipage}
    \hfill
    \begin{minipage}{0.48\textwidth}
        \centering
        \includegraphics[width=\textwidth]{figures/attn_norm_seed1000001_softmax_rope_vs_softmax_alibi_vs_sigmoid_alibi.png}
        \captionsetup{justification=centering}
        \caption{$\sigmoidattn$ with ALiBi.}
        \label{fig:rope_vs_alibi}
    \end{minipage}\hfill
    \begin{minipage}{0.48\textwidth}
        \centering        
        \includegraphics[width=\textwidth]{figures/attn_norm_seed1000001_softmax_rope_vs_softmax_alibi_vs_sigmoid_rope_b=-10.png}
        \captionsetup{justification=centering}
        \caption{$\sigmoidattn$ with RoPE, $b=-10$.}
        \label{fig:rope_vs_rope_b-10}
    \end{minipage}  
    \vspace{-0.4cm}
\end{figure}

\subsection{Ablations}
\label{sec:ablations}
We begin with ablations to dissect the benefits of each of our introduced components. To gain intuition about $\sigmoidattn$, we developed a research-friendly auto-regressive (AR) LM training framework to measure all components of attention and validate the effects of LayerScale, LayerNorm applied to Q and K (QK norm), different positional embedding techniques, and initialization values for $b$.
\begin{figure}[h]
    \centering
    \begin{minipage}[t]{0.48\textwidth}
        \centering
        \includegraphics[width=\textwidth]{figures/lines=activation-cols=layerscale_with_log_n_or_max3std.pdf} 
        \caption{LR sensitivity LayerScale ablation.}
        \label{fig:layerscale_ablation}
    \end{minipage}%
    \hfill
    \begin{minipage}[t]{0.48\textwidth}
        \centering
        \includegraphics[width=\textwidth]{figures/lines=activation-cols=qknorm_with_log_n_or_max3std.pdf}
        \caption{LR sensitivity QK norm ablation.}
        \label{fig:qk_norm_ablation}
    \end{minipage}
\end{figure}
\begin{figure}[h]
    \centering
    \vspace{-0.2cm}
    \includegraphics[width=\textwidth]{figures/imagenet_ablations_top1.pdf}
    \caption{ImageNet1k ViT-B/16 classification. (a) $\sigmoidattn$ is robust without QK norm (+LayerScale, -QKNorm). Removing LayerScale reduces accuracy by 1.0\% (-LayerScale, +/-QKNorm). $n^{-\alpha}$ normalization \citep{wortsman2023replacing} underperforms without LayerScale. (b) $\sigmoidattn$ multi-query attention (MQA) \citep{DBLP:journals/corr/abs-1911-02150} with one head matches multi-head attention (MHA). (c) Sigmoid with LayerScale and QK norm performs comparably to other activations, except TanH. ReLU$^2$ \citep{DBLP:conf/icml/HuaDLL22} underperforms without LayerScale and QK norm.}
    \label{fig:imagenet_top_1_ablations}
    \vspace{-0.4cm}
\end{figure}
\paragraph{Mitigating Large Attention Norms} We train a single layer AR transformer block (E=3072, D\_FF=12288) on the realnews split of C4 \citep{DBLP:journals/jmlr/RaffelSRLNMZLL20}. We train for $2^{16}$ steps using a batch size of 6 and max sequence length of 4096 using a single cycle cosine learning rate (LR) schedule without weight decay. $\sigmoidattn$ initially underperformed $\softmaxattn$ when using absolute sinusoidal (SinCos) (\cref{fig:rope_vs_sincos}) or relative (\cref{fig:rope_vs_rope}) positional embeddings (PE), which we attribute to high initial attention Frobenius norms, $\lVert \sigma(\mQ \mK^T / \sqrt{d}) \mV \rVert$. A corresponding evolution of the attention distribution and sparsity can be seen in Appendix \cref{fig:attn_evolve} and \cref{fig:attn_metric_evolve} on a synthetic task.
To address these larger attention norms, we propose: (a) using ALiBi \citep{DBLP:conf/iclr/PressSL22} whose relative bias moves initial attention logit mass to the zero region under the sigmoid activation, producing equivalent train negative log-likelihoods (\cref{fig:rope_vs_alibi}); or (b) set the attention logit bias $b$ to a negative offset proportional to the sequence length, $b \propto -\ln n$ (see \cref{sec:attn_bias_ablation} for an ablation on $b$). This enables the usage of other PE techniques like RoPE~\citep{DBLP:journals/ijon/SuALPBL24} (\cref{fig:rope_vs_rope_b-10}). 
\paragraph{LayerScale} To validate the need for LayerScale, we follow \citet{DBLP:journals/corr/abs-2309-14322} to quantify the impact on stability.
All models are trained with RoPE with $b \propto -\ln n$, using AdamW  \citep{loshchilov2017decoupled} on the 
realnews split of C4 
with $(\beta_1,\beta_2)=(0.9, 0.95)$, $\eps=10^{-8}$,  $wd=0$, 
batch size 24, maximum token sequence length of 512 from the T5 tokenizer \citep{DBLP:journals/jmlr/RaffelSRLNMZLL20}, cosine LR schedule of $2^{14}$ steps including a linear warmup of $2^{10}$ steps. 
Models have 
$n_{\text{heads}}=\kappa$,
$n_{\text{layers}}=2\times \kappa$,
$d_{\text{model}}=64\times \kappa$ and
$d_{\text{feed-forward}}=256\times\kappa$
for a scaling value $\kappa\in\{1,2,4,8,16\}$
leading to models with $\{2.2, 4.9,15.0,67.0,440.0\}M$ trainable non-embedding parameters.
Following \citet{DBLP:journals/corr/abs-2309-14322},
we sweep learning rates
$\eta\in \{3\times 10^{-4}, 1\times 10^{-3}, 3\times 10^{-3}, 1\times 10^{-2}, 3\times 10^{-2}, 1\times 10^{-1}, 3\times 10^{-1}\}$.
LR sensitivity is defined as 
$\mathbb E_{\eta\in[a,b]}\left[\min(\ell(\mathcal A(\eta)),\ell_0)-\ell^*\right]$
where $\ell(\mathcal A(\eta))$ is the loss achieved by the learning algorithm $\mathcal A$ with LR $\eta$,
$\ell_0$ is the loss at initialization, and
$\ell^*$ is the loss achieved by the best LR.
LayerScale is initialized at $10^{-4}$. 
Unlike vision tasks, where LayerScale \emph{improves performance} (\cref{fig:imagenet_top_1_ablations}-a), in LM, we observe that $\softmaxattn$ slightly benefits from LayerScale, while the performance of $\sigmoidattn$ remains largely unaffected.
\paragraph{Stability with QK Norm} \Cref{thm:regularity} indicates that the Jacobian of $\sigmoidattn$ has favorable properties compared to $\softmaxattn$. We explore this by repeating the analysis of \citet{DBLP:journals/corr/abs-2309-14322}, as described in the LayerScale analysis, to investigate the impact of QK norm \citep{DBLP:conf/icml/0001DMPHGSCGAJB23}. For language modeling, both $\sigmoidattn$ and $\softmaxattn$ exhibit sensitivity to learning rate changes without QK norm. However, incorporating QK norm significantly stabilizes performance (\cref{fig:qk_norm_ablation}). In vision tasks, $\sigmoidattn$ demonstrates robustness with and without QK norm (\cref{fig:imagenet_top_1_ablations}-a) and without the need for $n^{-\alpha}$ normalization from \citet{wortsman2023replacing}.\footnote{We ablate multiplicative sequence length scaling in more detail in \cref{sec:appendix_normalization}.}
\paragraph{Multi-query attention (MQA)} In \cref{fig:imagenet_top_1_ablations}-b we explore MQA \citep{DBLP:journals/corr/abs-1911-02150} for vision using only one head for $\{ \mK, \mV \}$. We find that both $\sigmoidattn$ and $\softmaxattn$ perform equally well with or without multiple heads even at the small scale of ViT-B/16.
\paragraph{Activation Function Ablations} As in \citet{wortsman2023replacing}, various activation functions, when combined with LayerScale and QK norm, perform equally well for vision tasks (\cref{fig:imagenet_top_1_ablations}-c). However, for sequence-critical tasks like ASR, activation functions such as ReLU pose instabilities and underperform. In the same figure, we also compare to the ReLU$^2$ proposal from \citet{DBLP:conf/icml/HuaDLL22} and find that it underperforms without LayerScale and QK norm.
\subsection{Supervised Image Classification}
\label{sec:supervised_image_classification}
Vision transformers \citep{DBLP:conf/iclr/DosovitskiyB0WZ21} extend transformers  \citep{DBLP:conf/nips/VaswaniSPUJGKP17} to treat $K \times K$ image grids as disparate tokens. All tokens are refined through sequential layers of self-attention, pooled using a CLS token or global average pooling layer, and optimized using the negative log likelihood, $\ln p(\vy|\vx)$. We train ViT-B/16 models using $\mathbb{R}^{224 \times 224 \times 3}$ images for 300 epochs using the recipe provided in \cref{sec:appendix_vision_hyperparams}. We use the same set of training hyper-parameters for both $\softmaxattn$ and $\sigmoidattn$, changing only the activation function between trials. The train negative log-likelihood is reported in \cref{fig:summary_nll} and the test top-1\% is reported in \cref{fig:test_top1_results}. We find that $\sigmoidattn$ matches both the training dynamics and the evaluation performance of $\softmaxattn$.
\subsection{Self-Supervised Image Representation Learning}
\label{sec:ssl}
Self-supervised representation learning (SSL) exploits vast quantities of unlabeled data to learn semantic representations based on inductive biases such as augmentation invariance (SimCLR \cite{DBLP:conf/icml/ChenK0H20}, BYOL \citep{DBLP:conf/nips/GrillSATRBDPGAP20}) or reconstruction from compressed representations (MAE \citep{DBLP:conf/cvpr/HeCXLDG22}). We employ vision transformer training recipes from \cite{DBLP:conf/icml/ZhaiLLBR0GS23} and \cite{DBLP:conf/nips/BusbridgeRALDCW23} (\cref{sec:appendix_vision_hyperparams}) for SimCLR and BYOL. As with supervised learning, we use the same set of training hyper-parameters for both $\softmaxattn$ and $\sigmoidattn$, changing only the activation function between trials. \Cref{fig:summary_nll} reports the train losses, and \cref{fig:test_top1_results} highlights the linear probe and finetuned test top-1\%. Despite the diverse training objectives in SSL, $\sigmoidattn$ matches $\softmaxattn$ while improving training and inference throughput (\cref{sec:FlashSigmoidHardwareAwareImplementation}).
\subsection{Automatic Speech Recognition (ASR)}
\label{sec:asr}
\begin{table}[t!]
\centering
\caption{Word error rate (\%) on LibriSpeech test sets and TED-LIUM v3~\citep{hernandez2018ted} (``TED'', joint validation and test sets split according to  duration) for transformer (255M params) with either $\softmaxattn$ or $\sigmoidattn$ (LayerScale and QK norm are used with $b=-\log n$) trained on LibriSpeech 960h data (mean duration is 10-15s). Hyper-parameters are in~\cref{sec:asr_hps}.}
\label{tab:asr-results}
\begin{center}
\begin{scriptsize}
\begin{sc}
\resizebox{\columnwidth}{!}{%
\begin{tabular}{lc|rr|rrrr}
\toprule
 attn & PE & test-clean & test-other & ted 0-10s & ted 10-20s & ted 20-30s & ted 30s+  \\
\midrule 
softmax & \multirow{7}{*}{CAPE} & 2.3 & 5.7 & 12.4 & 10.5 & 11.9 & 9.1 \\
 sigmoid &  & 2.4 & 5.5 & 12.4 & 10.3 & 12.3 & 9.7 \\
 \,\,\,\, - QK norm &  & \multicolumn{6}{c}{unstable, gradient norm and loss spikes} \\
 \,\,\,\, - LayerScale &  & 2.5 & 6.1 & 13.6 & 11.5 & 13.4 & 8.9 \\
 sigmoid ($b=-10$, learnable) &  & 2.3 & 5.5 & 12.1 & 10.5 & 13.0 & 9.3 \\
 sigmoid ($b=-5$ in $Q$, learnable) &  & 2.3 & 5.4 & 12.2 & 10.8 & 12.4 & 9.9 \\
 \,\,\,\, - QK norm &  & \multicolumn{6}{c}{unstable, gradient norm and loss spikes} \\

\midrule
softmax & \multirow{5}{*}{RoPE} & 2.2 & 5.5 & 12.7 & 10.6 & 12.8 & 9.5 \\
 sigmoid &  & 2.3 & 5.4 & 12.3 & 10.1 & 12.3 & 8.6 \\
 sigmoid ($b=-10$, learnable) &  & 2.2 & 5.2 & 12.4 & 10.5 & 12.3 & 21.8 \\
 \,\,\,\, + $\alpha=1$ &  & 2.7 & 6.6 & 14.1 & 12.0 & 14.5 & 14.9 \\
 sigmoid ($b=-5$ in $Q$, learnable) &  & \multicolumn{6}{c}{unstable, gradient norm and loss spikes} \\
\midrule
 softmax & \multirow{5}{*}{ALiBi} & 2.2 & 5.4 & 12.3 & 10.7 & 12.1 & 8.6 \\
 sigmoid &  & 2.3 & 5.1 & 12.3 & 10.5 & 12.6 & 9.1 \\
 sigmoid ($b=-10$, learnable) &  & 2.2 & 5.2 & 12.4 & 10.4 & 11.7 & 9.1 \\
 \,\, + $\alpha=1$ &  & 2.6 & 6.6 & 13.9 & 11.9 & 14.2 & 8.6 \\
 sigmoid ($b=-5$ in $Q$, learnable) &  & 2.2 & 5.2 & 12.1 & 10.4 & 12.0 & 8.2 \\
\bottomrule
\vspace{-0.4cm}
\end{tabular}
}
\end{sc}
\end{scriptsize}
\end{center}
\end{table}
We benchmark ASR using LibriSpeech data \citep{DBLP:conf/icassp/PanayotovCPK15} on 100h and 960h settings of paired speech and text transcriptions. Our PyTorch implementations of encoder-based vanilla transformer~\citep{synnaeve2019end} and conformer \citep{DBLP:conf/interspeech/GulatiQCPZYHWZW20} are trained with Connectionist Temporal Classification (CTC) \citep{DBLP:conf/icml/GravesFGS06} w/ BF16 mixed precision, w/o QK norm and w/o LayerScale. After extensively tuning $\softmaxattn$ baselines, we switch to $\sigmoidattn$ per \cref{eq:sigmoid_attn} without any other changes. We investigate the effects of post/pre-LayerNorm, model depth, optimizer type, small data regime, and connection to local attention, with details in~\cref{sec:asr_hps}.

Our main findings are: i) CAPE~\citep{DBLP:conf/nips/LikhomanenkoXSC21} PE is the most unstable for $\sigmoidattn$; ii) post-LayerNorm models with $\softmaxattn$ are hard to match with stable $\sigmoidattn$; iii) w/o QK norm $\sigmoidattn$ is unstable and significant spikes happen in both gradient norms and training loss; iv) LayerScale is needed for generalization; v) learnable bias $b=-10$ gives no loss and gradient norms spikes while matching the $\softmaxattn$ (which does not benefit from the improved throughput of \textsc{FlashSigmoid}); vi) adding a learnable bias, $b=-5$, to $Q$ instead of the attention logits also solves the initial large attention norms for CAPE and ALiBi but not for RoPE; vii) $b=-\log n$ gives rare (2-5 times) marginal gradient norms spikes with smooth loss while matching $\softmaxattn$.


\Cref{tab:asr-results} shows the main result for pre-LayerNorm  transformers with CAPE, RoPE, and ALiBi, where $\sigmoidattn$ uses LayerScale, QK norm, $b=-\log n$, and no sequence normalization. The bias is ablated with learnable bias (one per layer) in attention or $Q$ with or without sequence normalization. $\sigmoidattn$ is stabilized with bias while matching $\softmaxattn$, and $b=-\log n$ works well. In most cases, bias allows generalization to longer sequences without sequence normalization, except for RoPE where it helps for longer sequences but hurts overall performance.









\subsection{Autoregressive Large Language Modeling}
\label{sec:llm}

\newcolumntype{R}[2]{%
    >{\adjustbox{angle=#1,lap=\width-(#2)}\bgroup}%
    l%
    <{\egroup}%
}
\newcommand*\rotdiag{\multicolumn{1}{R{30}{1em}}}%

\begin{table}[t]
\centering
\caption{1B LLM English evaluation.}
\label{tab:lm_results}
\begin{sc}
\begin{scriptsize}
\bgroup
\setlength{\tabcolsep}{.35em}
\begin{tabular}{@{}lllllllllllllll@{}}
\toprule
Model   & \makecell{Seq.\\Len.} & \makecell{ARC\\Easy} & \makecell{ARC\\Challenge} & \makecell{Hella-\\swag} & Piqa & Sciq & \makecell{Wino-\\grande} & \makecell{Lambada\\OpenAI} & \makecell{TriviaQA\\(1-shot)} & \makecell{WebQS\\(1-shot)} & AVG & \makecell{Step\\time (s)} \\ \midrule
Softmax (ALiBi) & 2k & 62.2       &     26.8           &    42.4       &  59.0    &   72.3   &     88.1       &     58.4           &      19.9             &    15.4            &    49.4   & 0.38   \\
Sigmoid (ALiBi) & 2k &  62.8       &      28.8         &    42.5       &  59.7    &   70.3   &     88.6       &      59.7          &       19.1            &   13.8             &       49.5  & 0.34   \\
\midrule
Softmax (RoPE) & 4k & 63.3       &     29.3           &    43.3       &  58.1    &   71.3   &     86.9       &     58.8           &  20.4             &    15.6            &    49.7   & 0.84   \\
Softmax (ALiBi) & 4k & 62.6       &     27.7           &    42.4       &  58.6    &   71.1   &     88.2       &     58.6           &      18.9             &    14.7            &    49.2   & 0.84   \\
Sigmoid (ALiBi) & 4k &  60.5       &      27.3         &    41.3       &  57.8    &   70.5   &     87.0       &      57.6          &       18.9            &   12.6             &       48.2  & 0.67   \\ \bottomrule
\end{tabular}
\egroup
\end{scriptsize}
\end{sc}
\vspace{-0.4cm}
\end{table}

We initially iterated at the 85M scale, as it served as a proxy for larger scale training. Our findings show that: i) attention bias is required for stability, which can be learnable, but setting it to $-\log(n)$, where $n$ is the maximum training sequence length of 4096, works well and is faster; ii) RoPE is more challenging to stabilize; iii) the final setting exhibits smooth loss curves, but still shows gradient norm fluctuations. We then turn our attention to validating $\sigmoidattn$ at scale.

We train a 1B language model using the Llama2 \citep{touvron2023llama} recipe with ALiBi instead of RoPE positional embedding, and the RedPajama \citep{together2023redpajama} dataset (see \cref{sec:llm_appendix}). At sequence length 4096, $\sigmoidattn$ achieves a \textbf{1.23}$\mathbf{\times}$ step-time improvement over $\softmaxattn$ in JAX without \textsc{FlashAttention} (\cref{tab:lm_results}). All LLMs are trained using the AXLearn framework, which include the recipe and $\sigmoidattn$ implementation.\footnote{https://github.com/apple/axlearn}

$\softmaxattn$ and $\sigmoidattn$ have matching train and validation NLL at 85M (\cref{fig:85m_4k_nll}) and at 1B scale when using 2048 sequence length (\cref{fig:summary_nll}). However, a slight disparity is observed at 1B scale when using 4096 sequence length, which we leave for future investigation (more details in \cref{sec:llm_appendix}).


Hyperbolic embeddings embed hierarchical information with high
fidelity and few dimensions. We explored the limits of this approach
by describing scalable, high quality algorithms. We hope the
techniques here encourage more follow-on work on the exciting
techniques of \citet{fb, ucl}. As future work, we hope to explore how
hyperbolic embeddings can be most effectively incorporated into downstream
tasks and applications.


\newpage

\section{Ethics Statement}
A main objective of our work is to improve the ability to classify and forecast time series, which has real-world applications in many fields. These applications may have high stakes, such as classifying abnormalities in medical time series. In these situations, incorrect predictions may lead to harmful patient outcomes. It is thus critical to understand that while we aim to improve time series modeling towards these applications, we do not solve these problems. Further analysis and development into where models fail in time series modeling is necessary, including potentials intersections with research directions such as robustness and model biases when aiming to deploy machine learning models in real world applications.  
% 
% In many applications, such as classifying abnormal medical time series, incorrect predictions may lead to harmful patient outcomes. For many such applications, it is critical to carry out a deeper analysis and understanding of model failures, including an investigation into potential model biases, where the model may perform well on average but underperform on certain subpopulations. 
\section{Reproducibility}
We include code for the main results in Table~\ref{tab:informer-s-long} at \href{https://github.com/HazyResearch/spacetime}{https://github.com/HazyResearch/spacetime}. We provide training hyperparameters and dataset details for each benchmark in Appendix~\ref{app:exp_details}, discussing the Informer forecasting benchmark in Appendix~\ref{app:informer_details}, the Monash forecasting benchmark in Appendix~\ref{app:monash_details}, and the ECG and speech audio classification benchmarks in Appendix~\ref{app:classification_exps}.
%
We provide proofs for all propositions and algorithm complexities in  Appendix~\ref{appendix:theory}. 

\section{Acknowledgements}
We thank Albert Gu, Yining Chen, Dan Fu, Ke Alexander Wang, and Rose Wang for helpful discussions and feedback. We also gratefully acknowledge the support of NIH under No. U54EB020405 (Mobilize), NSF under Nos. CCF1763315 (Beyond Sparsity), CCF1563078 (Volume to Velocity), and 1937301 (RTML); US DEVCOM ARL under No. W911NF-21-2-0251 (Interactive Human-AI Teaming); ONR under No. N000141712266 (Unifying Weak Supervision); ONR N00014-20-1-2480: Understanding and Applying Non-Euclidean Geometry in Machine Learning; N000142012275 (NEPTUNE); NXP, Xilinx, LETI-CEA, Intel, IBM, Microsoft, NEC, Toshiba, TSMC, ARM, Hitachi, BASF, Accenture, Ericsson, Qualcomm, Analog Devices, Google Cloud, Salesforce, Total, the HAI-GCP Cloud Credits for Research program,  the Stanford Data Science Initiative (SDSI), and members of the Stanford DAWN project: Facebook, Google, and VMWare. The U.S. Government is authorized to reproduce and distribute reprints for Governmental purposes notwithstanding any copyright notation thereon. Any opinions, findings, and conclusions or recommendations expressed in this material are those of the authors and do not necessarily reflect the views, policies, or endorsements, either expressed or implied, of NIH, ONR, or the U.S. Government.



%
\bibliographystyle{abbrvnat}
\bibliography{main.bib}
%
\newpage
%
\begin{center}
    \huge\bf{Appendix:\\ Effectively Modeling Time Series with \\Simple Discrete State Spaces}
\end{center}

\appendix
\addcontentsline{toc}{section}{}
\part{}
\parttoc
%
\section{Related Work}
\label{sec:related_work}

\paragraph{Attention variants and distributed attention}
Ever since attention became popular with the Transformer
architecture~\citep{vaswani2017attention}, there has been a large body of work
on approximating attention to scale it to longer sequences.
These approximation methods can generally be categorized into two classes:
sparse and low-rank.
Sparse attention only computes some entries of the attention matrix ($\mathrm{softmax}(\vQ
\vK^T)$) and assumes that other entries are zero.
Different methods have different ways of choosing which entries should be zero,
either with a fixed pattern~\citep{child2019generating}, with a sliding
window~\citep{beltagy2020longformer}, or with a dynamic pattern through
hashing~\citep{kitaev2020reformer} or routing~\citep{roy2020efficient}.
The low-rank approach instead assumes that the attention matrix has a low-rank
structure, and apply a pointwise nonlinearity to the query and
key~\citep{katharopoulos2020transformers} with random
projection~\citep{choromanski2021rethinking, peng2021random, xiong2021nystromformer}.
One can also combine the sparse and low-rank approximation for better
quality~\citep{zaheer2020bigbird,scatterbrain}.
However, these approximation methods typically do not offer the same model
quality as standard attention~\citep{tay2020efficient}, and so most large-scale
models do not employ these techniques.

There are other variants of attention aimed at reducing the size of the KV cache
to improve inference efficiency. Multi-query attention~\citep{shazeer2019fast} and grouped query
attention~\citep{ainslie2023gqa} tie different heads of $\vK$ and $\vV$, and
multiple query heads interact with the same key and value head.
Multi-head latent attention~\citep{deepseekv2} parameterizes the $\vK$ and $\vV$
as low-rank projections of a shared matrix to further reduce the KV cache size.
However, all of these approaches do not change the core computation
$\mathrm{softmax}(\vQ \vK^T) \vV$ during training and simply change how $\vQ, \vK, \vV$ are
obtained.
As a result, any efficiency or accuracy improvement to the standard attention
computation benefits these methods.

To extend to even longer context, attention computation can be distributed
across multiple GPUs.
Methods such as Ring attention~\citep{liu2023ring,liu2024world} and
variants~\citep{brandon2023striped} can reach a context length of up to 1
million.
They use \fa (or \faa) as a primitive, and so the improvement from \fat would
benefit these distributed attention methods as well.

\paragraph{Alternative architectures}
Motivated by the limitations of attention, a variety of alternative
architectures have been proposed.
They build on the connection between linear
attention~\citep{katharopoulos2020transformers} and recurrent neural networks
(RNNs).
RWKV~\citep{peng2023rwkv}, H3~\citep{dao2023hungry}, MEGA~\citep{ma2023mega},
Retnet~\citep{sun2023retentive}  enhance the expressivity of the simple
cumulative sum in linear attention with more sophisticated recurrences.
Mamba~\citep{gu2023mamba} and xLSTM~\citep{beck2024xlstm} use learnable
weighting for the recurrence and can match the quality of Transformers in
language modeling at small or medium scale.
These approaches can be connected to generalizations of linear attention through
the lens of the structure of the token-mixing matrix~\citep{dao2024transformers}.
These models have started to see some traction, seeing usage in some medium to
large-scale models such as Jamba~\citep{jamba}, Zamba~\citep{zamba},
Megalodon~\citep{ma2024megalodon}, and Mamba2-hybrid~\citep{waleffe2024empirical}.
For the highest quality, these SSM- and RNN-based models still employ
many layers of attention.
We expect that techniques to speed up attention presented in this work will be
useful to speedup these alternative architectures.

\paragraph{Low-precision attention}
Quantization is a promising approach to speed up attention, but they have mostly
focused on reducing the space for KV cache for inference efficiency.
QuIP~\citep{chee2024quip} and QuIP\#\citep{tseng2024quip} use incoherent processing to reduce the quantization,
and we adapted this technique for FP8 \fat.
Recent work suggests that for inference the KV cache is highly compressible down to 4-, 3-, or
even 2-bits~\citep{hooper2024kvquant, liu2024kivi}.
However, quantization during training is still challenging as higher precision
is typically required for stable training.

\paragraph{Hardware-aware Algorithms}
Our work presented in this paper focuses on the micro-architecture
specific tuning to leverage new instruction sets and adopt a natively
asynchronous programming model. There are other orthogonal axes for
hardware-aware algorithm co-design being explored.
A recent example of this is LeanAttention~\citep{sanovar2024-leanattention},
which recognizes the poor GPU occupancy and high memory bandwidth requirements
of the sequential token generation phase as primary bottlenecks for inference
and optimizes it via a smarter load balancing strategy similar to Stream-K
load balancing~\citep{streamk} to achieve nearly peak occupancy.
There is a large literature on optimizing GEMM for specific hardware that employs
many of the same techniques.
As an example, \citet{abdel2016batched} presents a high performance batched GEMM kernel on
K40c Graphics Processing Units (GPU) for both fixed and variable sizes,
proposing specialized GEMM designs
and a comprehensive autotuning process to deliver state-of-the-art 
performance.


\section{Theoretical Properties of Sigmoid Attention}
\label{sec:theory}
We analyze $\sigmoidattn$, with two objectives: (1) showing that a transformer architecture remains a universal function approximator when $\sigmoidattn$ replaces $\softmaxattn$, and (2) recovering a measure of regularity of $\sigmoidattn$ by computing its Lipschitz constant.

\subsection{Are Transformers with Sigmoid Attention Universal Approximators?}
\label{sec:ufa}
\cite{Yun_UAP} demonstrate that classical transformers can approximate continuous sequence-to-sequence functions to arbitrary precision, a property known as the \emph{Universal Approximation Property} (UAP). UAP is highly desirable as it provides proof of an architecture's generalizability and representation capability.
As $\sigmoidattn$ modifies the transformer architecture, it is crucial to theoretically guarantee that this modification does not impact the representation capability and that UAP is retained. We provide this guarantee with the following theorem.
\begin{theorem}[UAP for $\sigmoidattn$]
    \label{thm::UAP}
    We denote with $\mathcal{T}^{h,d_v,r}_{\sigma}$ the class of transformer networks obtainable by combining an arbitrary number of $\sigmoidattn$ layers (each of $h$ heads of dimension $d_v$) followed by FFN layers of hidden dimension $r$.
    For any given continuous, permutation-equivariant function $f:\Omega\subset\mathbb{R}^{n\times d}\to\mathbb{R}^{n\times d}$ with compact support $\Omega$, and for any arbitrarily small error $\varepsilon$, there exists a transformer network $g\in\mathcal{T}_\sigma^{4,1,4}$ such that
    \begin{equation}
        \left(\int_{\Omega}\|f(\bb{X})-g(\bb{X})\|^p_p d\bb{X}\right)\leq\varepsilon,\qquad\text{for}\quad 1\leq p<\infty.
    \end{equation}
\end{theorem}
\Cref{thm::UAP} is the exact counterpart of \cite[Thm.~2]{Yun_UAP}, which shows UAP for classical transformers. Our proof largely follows the same path, an outline of the original proof provided in \cref{app:UAP_proof}. Here, we present an overview of the main adaptations required to prove \cref{thm::UAP} for $\sigmoidattn$, with further details in \cref{sec::proof_modified_sigmoid,sec::proof_contextual_mapping_top}.

\paragraph{Sigmoid Attention layers can implement contextual mappings:} A key step in proving \cref{thm::UAP} is showing that, even with $\sigmoidattn$, a sequence of transformer blocks can implement a \emph{Contextual Mapping} \cite[Def.~3.1]{Yun_UAP}. A contextual mapping characterizes a function that maps each input sequence element to an output \emph{uniquely} dependent on the \emph{whole} sequence. This property allows a transformer to capture and store global context within each token, even if each layer only performs pairwise comparisons. Subsequent layers can then use this global information to map individual tokens to the correct output, ultimately approximating any arbitrary sequence-to-sequence function.

In \cite{Yun_UAP}, the contextual mapping is assembled by modifying individual transformer blocks: each block is tuned to react to a specific input token. By stacking a sequence of these blocks, a transformer can be turned into an accumulator, mapping a given input token sequence to a unique global index. This outcome is achieved via a \emph{selective shift layer} \cite[App.~B.5]{Yun_UAP}:
\begin{equation}
    \Psi(\bb{X};b,b')_{i,1}\coloneqq \begin{cases}
        \max_k \bb{X}_{k,1}-\min_k\bb{X}_{k,1}&\text{if}\quad b<\bb{X}_{i,1}<b'\\
        0&\text{otherwise},
    \end{cases}
    \label{eqn::shift_operation_original}
\end{equation}
and can be approximated using classic attention.
Although $\sigmoidattn$ cannot directly approximate~\cref{eqn::shift_operation_original}, our accumulator definition relies on an equivalent selective shift operation:
\begin{equation}
    \Psi_\sigma(\bb{X};b,b')_{i,1}\coloneqq\begin{cases}
        \sum_{k:\bb{X}_{k,1}> b'} \bb{X}_{k,1} &\text{if}\quad b<\bb{X}_{i,1}<b' \\
        0 &\text{otherwise},
    \end{cases}
    \label{eqn::shift_operation_ours}
\end{equation}
which can be approximated by $\sigmoidattn$ (described in \cref{sec::proof_modified_sigmoid}). In~\cref{sec::proof_contextual_mapping}, we show that~\cref{eqn::shift_operation_ours} shares similar properties with~\cref{eqn::shift_operation_original}, allowing us to use the original proof framework in \cite{Yun_UAP} and demonstrate that UAP holds in our case as well.

Our proof is largely equivalent to that in \cite{Yun_UAP}, with two relevant differences: to approximate \cref{eqn::shift_operation_ours}, we require $\sigmoidattn$ with \textit{at least four heads} and shifts included in both query and key definitions. In contrast, $\softmaxattn$ requires \textit{at least two heads} to approximate~\cref{eqn::shift_operation_original}, with shifts only in the query definition. However, this is primarily a theoretical requirement for the proof and does not affect performance. Notably, the total number of parameters required by both architectures for the approximation follows the same tight scaling of \cite{Yun_UAP}.






\subsection{Regularity of Sigmoid Attention}
\label{sec:regularity}
As with any layer in a neural network, the regularity of $\sigmoidattn$ is important to study, as it gives insights into the robustness of the corresponding network and the ease of optimizing it.
The most standard way to quantify the regularity of a layer function $\phi$ is to compute its \emph{Lipschitz constant} over a set $\mathcal{X}$, that is a constant $C>0$ such that for all $\mX, \mY\in \mathcal{X}$, it holds $\|\phi(\mX) - \phi(\mY)\|\leq C \|\mX - \mY\|$, where $\|\cdot\|$ is the standard Frobenius norm.
The \emph{local} Lipschitz constant is the spectral norm of the Jacobian of $\phi$ at $\mX$.
The two are related: the Lipschitz constant of $\phi$ over $\mathcal{X}$ is the greatest local Lipschitz constant for all $\mX\in \mathcal{X}$.
We turn to the theorem giving the regularity of $\sigmoidattn$:
\begin{theorem}
\label{thm:regularity}
    Define $A = \{\langle \mW_q \vx_i \mW_k \vx_j\rangle|,\enspace i, j\in \{1,\dots,n\}\}\subset\mathbb{R}$ the set of attention weights,  and the scaled activation norms $\sigma_{\infty} = n\times\sup_{u\in A} |\sigma(u)|$ and $\sigma'_{\infty} = n\times \sup_{u\in A} |\sigma'(u)|$.
    Then, the Jacobian of $\sigmoidattn$ at $\mX = (\vx_1, \dots, \vx_n)$ has a spectral norm of at most:
    \begin{equation}
        \|\mW_v\|_2\left(\sigma_{\infty} + 2\sigma'_{\infty} \|\mW_q^T \mW_k\|_2\left(\frac1n\sum_{i=1}^n\|\vx_i\|_2^2\right)\right).
    \end{equation}
\end{theorem}
The proof is found in \cref{app:lipschitz_proof}.
In $\sigmoidattn$, if we assume that the attention weights $\langle \mW_q \vx_i, \mW_k \vx_j\rangle$ are all bounded by a constant $\mu$ --- this is true, e.g., if the activations are bounded --- we get $\sigma_{\infty}\leq \exp(\mu)$ and $\sigma'_{\infty}\leq\exp(\mu)$ thanks to the choice of $b = -\log(n)$.
The bound in \cref{thm:regularity} depends only on the \emph{average} squared-norm of the input sequence $\vx_i$, while classical results for the study of attention all rely on the largest value of $\|\vx_i\|^2_2$~\citep{kim2021lipschitz,castin2023understanding}. 
This is another consequence of the simplicity of sigmoid attention and is due to the removal of the normalizing constant in $\softmaxattn$.
Our result implies that if all $\vx_i$ are within a ball of radius $R$ then the Lipschitz constant of $\sigmoidattn$ grows at most like $R^2$, but it is stronger since we can apply this to unbounded distributions $\vx_i$; it matters only that the second moment is bounded.
This result contrasts sharply with the bounds obtained for $\softmaxattn$: \citet[Thm.~3.4.]{castin2023understanding} show that there exists a sequence $\mX = (\vx_1, \dots, \vx_n)$ with $\|\vx_i\|_2\leq R$ for all $i$ such that the spectral norm of the Jacobian of $\attn$ at $\mX$ is at least $cR^2\exp(cR^2)$ for some constant $c>0$.
On the other hand, our bound scales in $R^2$: this means that the local Lipschitz constant of $\sigmoidattn$ is much lower than the worst local Lipschitz constant of $\softmaxattn$.

% \section{Additional Details and Experiments}
% %

% \subsection{\ourmethod{} can approximate the frequency response of digital filters}

% We experimentally verify whether \ourmethod{} can approximate the input--output map of digital filter admitting a state--space representation, with improved generalization over baseline models given test inputs of unseen frequencies.

% We generate a dataset of $1028$ sinusoidal signals of length $200$
% \[
% x(t) = \sin{(2\pi \omega t})
% \]
% where $\omega \in [2, 40]~\bigcup~[50, 100]$ in the training set and $\omega \in (40, 50)$ in the test set. The outputs are obtained by filtering $x$, i.e., $y = \mathcal{F}(x)$  where $\mathcal{F}$ is in the family of digital filters. 

% We introduce common various sequence-to-sequence layers or models as baselines: the original S4 diagonal plus low--rank \citep{gu2021efficiently}, a single layer LSTM, a single 1d convolution (Conv1d), a dense linear layer (NLinear), a single self--attention layer. All models are trained for $800$ epochs with batch size $256$, learning rate $10^{-3}$ and Adam. We repeat this experiment for digital filters of different orders \citep{oppenheim1999discrete}. The results are shown in Figure \ref{fig:dsp_synthetic}. \ourmethod{} learns to match the frequency response of the target filter, producing the correct output for inputs at test frequencies. 

% %
% \paragraph{State as memory}
% %
% In deep state space models, $x\in\R^n$ is the state of a system with a recurrence defined as some matrix $\zA$ (not necessarily in companion form). Given an initial state $x$, for how many recurrence steps can the system \textit{remember} it? We reason about $x$ as model memory to highlight its role as a latent vector of sufficient information to predict an output as $y = cx$.
% %

% If the recurrence is defined as a forward shift $\zS$, $\zS^n = \0$ and thus the model contains information about the initial condition only for for $n$ steps. This notion can be made formal by noting that if the recurrence is defined as contractive operator i.e. eigenvalue $|\lambda(\zA)_{\text{max}}| < 1$, \textbf{or} $\zA$ is nilpotent, after convergence to the fixed--point $x^* = \zA x^*$ (which exists by Banach's fixed--point theorem) there is no information left about the initial condition: all inputs eventually converge to the same fixed--point. If the recurrence is full--rank and not a contraction, the state will never reach a fixed--point and less can be said about the state as a latent vector of fading memory.
% %

% Keeping the recurrence at the edge of stability (maximum eigenvalue close to $1$) ensures a longer transient phase and thus a longer effective memory. 
% %

% \subsection{{\color{red!70}Initialization for Long--Range Memory}}
% %
% A crucial ingredient for the success of deep SSMs for long--range sequences is an initialization that promotes diversity in the temporal scales of the input signals attended to by different heads of the model. 

% For example, ${\tt S4D}$ 


% \[
% x^*: x^* = {\tt shift}(x) + px[-1]    
% \]

%




% In example, it can be verified (setting the initial condition $x[0] = \0$ for notational convenience) using basic properties of the Z--transform.

% [CIT some ref textbook]:
%
% \begin{center}
%     \begin{tabular}{cc}
%     State--Space Recurrence & Readout \\[4pt]
%     %
%     $z X[z] = \zA X[z] + B U[z]$ & $Y[z] = C X[z] + D U[z]$ \\[4pt]
%     $X[z] = (z\zI - \zA)^{-1} + B U[z]$ & $Y[z] = C (z\zI - \zA)^{-1} B U[z] + DU[z]$ \\[3pt]
%     \end{tabular}
% \end{center}
% %
% leading to the transfer function
% %
% \[
% \frac{Y[z]}{U[z]} = C (z\zI - \zA)^{-1} B + D
% \]
% %

% \begin{itemize}
%     \item Discuss approximation of filters
%     \item Discuss numerical considerations
% \end{itemize}

% \begin{prop}[Stable Companion]
% %
% \end{prop}
% %
% \paragraph{Companions and diagonals}
% %
% \section{Numerical Implementation}
% %
% \paragraph{Kernel}
% %
% \section{Approximating Digital Filters}
%


% \begin{table}[t]
%     \centering
%     \begin{tabular}{cc|ccccccc}
%     \toprule
%     %
%     Filter & Order & \ourmethod & S4 & Conv1D & LSTM & NLinear & Transformer \\
%     %
%     \midrule
%     Butterworth & $2$ & $0.0055$ & $0.0118$ & $0.0112$ & $0.0115$ & $1.8420$ & $0.5535$ \\
%     & $3$ & $0.0057$ & $0.3499$ & $0.0449$ & $0.0231$ & $1.7085$ & $0.6639$ \\
%     & $10$ & $0.0039$ & $0.8077$ & $0.4747$ & $0.2753$  & $1.5162$ & $0.7191$ \\
%     \midrule
%     Chebyshev $1$ & $2$ & $0.0187$ & $0.0480$ & $0.0558$ & $0.0285$ & $1.9313$ & $0.2452$ \\
%     & $3$ & $0.0055$ & $0.0467$ & $0.0615$ & $0.0178$ & $1.8077$ & $0.4028$ \\
%     & $10$ & $0.0620$ & $0.6670$ & $0.1961$ & $0.1463$ & $1.5069$ & $0.7925$ \\
%     \midrule
%     Chebyshev $2$ & $2$ & $0.0112$ & $0.0121$ & $0.0067$ & $0.0019$ & $0.4101$ & $0.0030$ \\
%     & $3$ & $0.0201$ & $0.0110$ & $0.0771$ & $0.0102$ & $0.4261$ & $0.0088$\\
%     & $10$ & $0.0063$ &  $0.6209$ & $0.3361$ & $0.1911$ & $1.5584$ & $0.7936$\\
%     \midrule
%     Elliptic & $2$ & $0.0001$ & $0.0300$ & $0.0565$ & $0.0236$ & $1.9150$ & $0.2445$ \\
%     & $3$ & $0.0671$ & $0.0868$ & $0.0551$ & $0.0171$ & $1.8782$  & $0.4198$ \\
%     & $10$ & $0.0622$ & $0.0909$ & $0.1352$ & $0.1344$ & $1.4901$ & $0.7368$ \\
%     %
%     \bottomrule
%     %
%     \end{tabular}
%     \caption{Comparing sequence models on the task of approximating the input--output map defined by digital filters of different orders. Test RMSE on held-out inputs at unseen frequencies.}
%     \label{tab:my_label}
% \end{table}
% %
% \begin{figure}
%     \centering
%     \includegraphics[width=0.99\columnwidth]{_ICLR2023_paper/figures/dsp_SpaceTime.pdf}
%     \includegraphics[width=0.99\columnwidth]{_ICLR2023_paper/figures/dsp_S4.pdf}
%     \includegraphics[width=0.99\columnwidth]{_ICLR2023_paper/figures/dsp_Conv1d.pdf}
%     \includegraphics[width=0.99\columnwidth]{_ICLR2023_paper/figures/dsp_LSTM.pdf}
%     \includegraphics[width=0.99\columnwidth]{_ICLR2023_paper/figures/dsp_NLinear.pdf}
%     \includegraphics[width=0.99\columnwidth]{_ICLR2023_paper/figures/dsp_Transformer.pdf}
%     \vspace{-2mm}
%     \caption{Testing the capability of different sequence--to--sequence models to approximate the input--output map of digital filters. In blue, we show the output signal filtered by each model. The ground--truth digital filter is a Butterworth of order $10$.}
%     \label{fig:dsp_synthetic}
% \end{figure}
% %


% \begin{sidewaystable}[t]
%     \small
%     \centering
%     \begin{tabular}{c|c|cccccc|ccccc}
%     \toprule
%     %
%     Dataset  & \ourmethod & SES & Theta & TBATS & ETS & (DHR-)ARIMA & PR & CatBoost & DeepAR & N-BEATS & WaveNet & Transformer \\
%     %
%     \midrule
%     M1 Yearly & \underline{135508.3} & 193829.5 & 171458.1 & \textbf{116850.9} & 167739.0 & 175343.8 & 152038.7 & 237644.5 & 173075.1 & 192489.8 & 312821.8 & 182850.6 \\
    
%     M1 Quarterly & \underline{2200.3} & 2545.7&	2282.7	&2673.9&	2408.5&	2538.5&	1909.3&	2161.0& 2313.3&	2267.3&	2271.7&	2231.5  \\    
    
%     M1 Monthly & 2601.1 & 2725.8 & 2564.9 & 2594.5 & 2264.0 & 2450.6 & 2478.8 & 2461.7 & 2202.2 & \textbf{2183.4} & 2578.9 & 3129.8  \\

%     M3 Yearly & 1412.4 & 1172.9&	1106.1&	1386.3	&1189.2	&1662.2	&1181.8&	1341.7&		1157.9&	1117.4&	1147.6&	1084.8 \\
    
%     M3 Quarterly & 676.1 & 670.6&	567.7&	653.6&	598.7&	650.8&	605.5&	698.0&	606.6&	582.8&	606.8&	819.2\\ 
    
%     M3 Monthly & 897.12 & 893.9	&754.0&	765.2	&755.3&	790.8&	830.0&	874.2&	873.7&	796.9&	845.3&	948.4 \\
    
%     M3 Other & 265.56 & 309.7	&242.1&	217.0&	224.1	&220.8	&262.3&	349.9&	277.7&	248.5&	277.0	&271.0 \\
    
%     M4 Quarterly & 718.2 & 732.8&	673.2&	672.7&	674.3&	710.0&	711.9&	714.2&	700.3&	684.7&	697.0&	739.1 \\
    
%     M4 Monthly & 1092.2 & 755.5	&683.7&	743.4&	705.7&	702.1&	720.5&	734.8&	740.3&	705.2&	787.9&	902.4\\ 
    
%     M4 Weekly & \textbf{348.3}& 412.6	&405.2&	356.7&	408.5	&386.3&	350.3&	420.8&	422.2&	330.8&	437.3&	456.9\\ 
    
%     M4 Daily &\textbf{183.2}& 209.8&	210.4&	\underline{208.4}&	230.0	&212.6&	213.0&	263.1	&343.5&	221.7&	220.5&	233.6\\ 
    
%     M4 Hourly & \textbf{255.2} & 1476.8	&1483.7	&469.9&	3830.4&	1563.1&	\underline{313.0} &	344.6&	1095.1&	501.2&	468.1&	391.2\\
    
%     Tourism Yearly & \textbf{74799.2}& 106665.2&	99914.2&	105799.4&	104700.5&	106082.6&	89645.6	&87489.0&	78470.7&	78241.7	&77581.3&	80089.3\\
    
%     Tourism Quarterly & 11608.32& 15000.0&	9254.6&	12001.5	&10812.3	&12564.8	&11746.9	&12788.0	&11762.0&	11306.0	&11546.6	&11724.1 \\
    
%     Tourism Monthly & 3181.2& 7039.4&	2702.0	&3661.5	&2543.0	&3132.4&	2739.4&	3102.8&	2359.9&	2596.2&	2694.2&	2660.1\\
    
%     Pedestrian & 69.6& 228.1&	228.2	&261.3&	278.3&	820.3&	61.8&	60.8&	65.8&	99.3&	68.0&	70.2\\
    
%     Weather & \textbf{2.7}& 2.9	&3.3&	2.9&	3.0&	3.1	&9.1&	3.1&	\textbf{2.7}	&3.1&	3.0	&2.8\\
    
%     NN5 Weekly & \textbf{16.9}& 18.8&	18.7&	18.5&	18.8&	18.6&	18.6&	18.7&	18.5&	17.4&	24.2&	24.0 \\
    
%     Solar $10$ min &7.4& 7.2&	7.2	&10.7&	7.2	&5.6&	7.2	&8.7&	7.2	&6.6&	8.0&	7.2\\
    
%     Solar Weekly &1423.7& 1331.3&	1341.6&	1049.0&	1264.4&	967.9&	1168.2&	1754.2&	873.6&	1307.8&	2569.3&	693.8\\
    
%     Electricity Hourly & \textbf{475.1} & 1026.3&	1026.4&	743.4&	1524.9&	1082.4&	689.9&	582.7&	\underline{478.0} &	510.9&	489.9&	514.7\\
    
%     Electricity Weekly &37802.2& 77067.9&	76935.6&	28039.7	&70369.0&	32594.8&	47802.1&	37289.7	&53100.3&	35576.8	&63916.9&	78894.7\\
    
%     Fred-MD &3743.6& 3103.0&	3898.7&	2295.7&	2341.7	&3312.5&	9736.9&	2679.4&	4638.7&	2813.0&	2779.5&	5098.9\\
    
%     Traffic Hourly &0.03& 0.04&	0.04	&0.05&	0.04	&0.04&	0.03	&0.03&	0.02&	0.02	&0.03&	0.02\\
    
%     Traffic Weekly &\textbf{1.3}& 1.5&	1.5	&1.5&	1.5	&1.5&	1.5&	1.5	&1.5&	1.4	&1.6&	1.9\\
    
%     Hospital &40.1& 26.6&	22.6&	21.3&	22.0	&23.7&	23.5&	23.5&	22.0&	24.2&	23.4&	40.5\\
    
%     Covid &490.1& 403.4&	370.1&	113.0&	102.1&	100.5&	394.1&	607.9&	230.5&	186.5&	1135.4&	480.0\\
    
%     Saugeen & \textbf{24.0} & 39.8	&39.8	&42.6&	50.4&	43.2&	47.7&	\underline{39.3} &	45.3&	48.9&	43.0&	49.1\\
    
%     US Births &630.2& 1369.5&	735.5&	606.5&	607.2&	705.5&	732.1&	618.4&	684.0&	627.7&	768.8&	686.5\\
    
%     Sunspot &3.1& 5.0	&5.0&	3.0	&5.0&	3.0	&4.0&	2.4	&1.1&	14.5&	0.7	&0.5\\
    
%     Car Parts & 0.64 & 0.71&	0.65	&0.71&	0.71&	0.71&	\underline{0.58}&	0.71&	\textbf{0.50}&	1.0&	\underline{0.58}&	0.5\\
    
%     Vehicle Trips &30.4& 36.5&	37.4&	25.7&	37.6&	35.0&	31.7&	27.3&	26.5&	33.6&	29.0&	33.0\\
%     \bottomrule
%     %
%     \end{tabular}
%     \caption{Monash forecasting. We report test RMSE of \ourmethod for each dataset (best result selected via validation RMSE, average of $3$ runs).}
%     \label{tab:monash}
% \end{sidewaystable}


% \begin{table}[t]
%     \small
%     \centering
%     \begin{tabular}{c|cccc}
%     \toprule
%     %
%     Dataset  & Ours & S4S & S4D & S4DPLR  \\
%     %
%     \midrule
%     M1 Yearly & {\color{red!50}$154770.78$} & {\color{red!50}$177978.63$} & $161463.53$ &  \\
%     M1 Quarterly \\
%     M1 Monthly \\
%     M3 Yearly \\
%     M3 Quarterly \\ 
%     M3 Other \\ 
%     M4 Monthly \\ 
%     M4 Weekly \\ 
%     M4 Daily \\ 
%     Tourism Yearly \\
%     Tourism Monthly \\
%     Dominick \\
%     Pedestrian \\
%     Weather
%     NN5 \\
%     Solar $10$ min & $6.72$ &\\
%     Solar Weekly \\
%     Electricity Hourly \\
%     Electricity Weekly \\
%     Fred-MD \\
%     Traffic Hourly \\
%     Traffic Weekly \\
%     Hospital \\
%     Covid \\
%     Saugeen \\
%     US Births \\
%     \bottomrule
%     %
%     \end{tabular}
%     \caption{Monash forecasting. We report average test RMSE of \ourmethod for each dataset ($5$ runs each).}
%     \label{tab:my_label}
% \end{table}


% \begin{sidewaystable}[t]
%     \small
%     \centering
%     \begin{tabular}{c|c|cccccc|ccccc}
%     \toprule
%     %
%     Dataset  & Ours & SES & Theta & TBATS & ETS & (DHR-)ARIMA & PR & CatBoost & FFNN & DeepAR & N-BEATS & WaveNet & Transformer \\
%     %
%     \midrule
%     M1 Yearly & $2$ & $11$ & $6$ & $1$ & $5$ & $8$ & $3$ & $12$ & $4$ & $7$ & $10$ & $13$ & $9$ \\
%     M1 Quarterly \\
%     M1 Monthly \\
%     M3 Yearly \\
%     M3 Quarterly \\ 
%     M3 Other \\ 
%     M4 Monthly \\ 
%     M4 Weekly \\ 
%     M4 Daily \\ 
%     Tourism Yearly \\
%     Tourism Monthly \\
%     Dominick \\
%     Pedestrian \\
%     Weather
%     NN5 \\
%     Solar $10$ min & $6.72$ &\\
%     Solar Weekly \\
%     Electricity Hourly \\
%     Electricity Weekly \\
%     Fred-MD \\
%     Traffic Hourly \\
%     Traffic Weekly \\
%     Hospital \\
%     Covid \\
%     Saugeen \\
%     US Births \\
%     \bottomrule
%     %
%     \end{tabular}
%     \caption{Monash foreacasting. Model rankings (based on test RMSE).}
%     \label{tab:my_label}
% \end{sidewaystable}
\section{Experiment Details}
\label{sec:experiment_details}

\subsection{Model Configurations and Hyperparameters}

We summarize the details required to replicate our experiments below.

\subsubsection{Image Classification}

\textbf{Baseline Model:} For dense models, we use standard implementations of
ViT~\citep{dosovitskiy2020image}, MLP-Mixer{tolstikhin2021mlp} from the
\texttt{timm} library and from the T2T-ViT codebase~\citep{yuan2021tokens}.

The Monarch version of these models simply swap out the dense weight matrices in the attention blocks (projection matrices) and in the FFN block (linear layers) with Monarch matrices.
We set the number of blocks in the block-diagonal matrices to 4.
We also reduce the amount of regularization (stochastic depth) as our Monarch models are smaller than the dense models.

We adopt the hyperparameters (optimizer, learning rate, learning rate
scheduler) from~\citet{yuan2021tokens}.
Details are in~\cref{table:imagenet_hparams}.

We measure the wall-clock training time on V100 GPUs.

\begin{table}[!htbp]
 \caption{Configuration of the ImageNet experiment}   
\centering
\resizebox{0.8\linewidth}{!}{
\noindent\begin{tabular}{@{}c||ccccccc@{}}
  \specialrule{.15em}{.05em}{.05em}
Model&\multicolumn{1}{c}{Optimizer}&\multicolumn{1}{c}{Weight Decay}&\multicolumn{1}{c}{Learning Rate}&\multicolumn{1}{c}{Drop Path}&\multicolumn{1}{c}{Warmup/Epoch}\\
  \specialrule{.15em}{.05em}{.05em}
ViT-Small& AdamW & 0.05 & 0.001 & 0.1& 5/300 \\
Monarch-ViT-Small& AdamW & 0.05 & 0.001 &0& 5/300 \\
ViT-Base& AdamW & 0.05 & 0.001 &0.1& 5/300 \\
Monarch-ViT-Base& AdamW & 0.05 & 0.001 &0& 5/300 \\
  \specialrule{.15em}{.05em}{.05em}
Mixer-Small &AdamW& 0.1 &0.001&0.1& 5/300 \\
Monarch-Mixer-Small &AdamW&0.1 &0.001& 0 & 5/300 \\
Mixer-Base &AdamW& 0.1 &0.001&0.1& 5/300 \\
Monarch-Mixer-Base &AdamW &0.1 &0.001& 0 & 5/300 \\
  \specialrule{.15em}{.05em}{.05em}
\end{tabular}
}
\label{table:imagenet_hparams}
\end{table}

We follow the naming convention in the Vision Transformer paper and MLP-Mixer paper. In particular, ViT-S and ViT-B refers to the small and base ViT models respectively, and 16 refers to the patch size of 16x16. The MLP-Mixer models follow the same convention.

\subsubsection{Language Modeling}
For dense models, we use standard implementations of
GPT-2~\citep{radford2019language} from Huggingface \texttt{transformers} library and from Nvidia's Megatron-LM repo. 
We follow the training recipe of the Megatron-LM repo.

The Monarch version of these models simply swap out the dense weight matrices in the attention blocks (projection matrices) and in the FFN block (linear layers) with Monarch matrices.
We set the number of blocks in the block-diagonal matrices to 4.
We also reduce the regularization strength (dropout) as our model is smaller.

We report the hyperparameters used in~\cref{table:wt103} and~\cref{table:owt}.
We use an effective batch size of 512, and use gradient accumulation to fit into available GPU memory.

We measure the wall-clock training time on V100 GPUs.
\begin{table}[!h]
    \vspace{-0.5cm}
\centering
\caption{Configuration of the WikiText-103 experiments}
\resizebox{0.8\linewidth}{!}{
\noindent\begin{tabular}{@{}c||ccccccc@{}}
  \specialrule{.15em}{.05em}{.05em}
Model&\multicolumn{1}{c}{Optimizer}&\multicolumn{1}{c}{Weight Decay}&\multicolumn{1}{c}{Learning Rate}&\multicolumn{1}{c}{Dropout}&\multicolumn{1}{c}{Warmup/Epoch}\\
  \specialrule{.15em}{.05em}{.05em}
GPT-2-small& AdamW & 0.1 & 6e-4 & 0.1& 10/100 \\
Monarch-GPT-2-small& AdamW & 0.1 & 6e-4 & 0.0 & 10/100 \\
GPT-2-medium& AdamW & 0.1 & 1.5e-4 & 0.1& 10/100 \\
Monarch-GPT-2-medium & AdamW & 0.1 & 1.5e-4 & 0.0 & 10/100 \\
  \specialrule{.15em}{.05em}{.05em}
\end{tabular}
}
\label{table:wt103}
\end{table}

\begin{table}[!h]
\vspace{-0.5cm}
\centering
\caption{Configuration of the OpenWebText experiments}
\resizebox{0.8\linewidth}{!}{
\noindent\begin{tabular}{@{}c||ccccccc@{}}
  \specialrule{.15em}{.05em}{.05em}
Model&\multicolumn{1}{c}{Optimizer}&\multicolumn{1}{c}{Weight Decay}&\multicolumn{1}{c}{Learning Rate}&\multicolumn{1}{c}{Dropout}&\multicolumn{1}{c}{Warmup/Total iterations}\\
  \specialrule{.15em}{.05em}{.05em}
GPT-2-Small& AdamW & 0.1 & 6e-4 & 0.1& 4k/400k \\
Monarch-GPT-2-Small & AdamW & 0.1 & 6e-4 & 0.0 & 4k/400k \\
GPT-2-Medium& AdamW & 0.1 & 1.5e-4 & 0.1& 4k/400k \\
Monarch-GPT-2-Medium & AdamW & 0.1 & 1.5e-4 & 0.0 & 4k/400k \\
  \specialrule{.15em}{.05em}{.05em}
\end{tabular}
}
\label{table:owt}
\end{table}


\subsection{Details for PDE Solving}
We adopt the experiment setting and data generation of Navier-Stokes Equation from FNO~\citep{li2020fourier}. It considers the 2-d Navier-Stokes equation for a viscous, incompressible fliud in vorticity form on the unit tortus:
\begin{align}
    \partial_{t} w(x, t) + u(x, t) \cdot \nabla w(x, t) & = v \Delta w(x, t) + f(x), & x \in (0, 1)^2, t \in (0, T] \\
    \nabla w(x, t) & = 0, & x \in (0, 1)^2, t \in (0, T] \\
    w(x, 0) & = w_0(x), & x \in (0, 1)^2 \\
\end{align}
where $u \in C([, T0])$;$H_{per}((0, 1)^2; \mathbb{R}^2))$ for any $r>0$ is the velocity field, $w=\nabla \times u$ is the vorticity, $w_0 \in L^2_{per}((0, 1)^2; \mathbb{R})$ is the initial vorticity, $v \in \mathbb{R_{+}}$ is the viscosity coefficient, and $f \in L_{per}^2((0, 1)^2; \mathbb{R})$ is the forcing function. 
$T$ represents the time interval since it is time-dependent equation. $v$ represents the viscosity. N represents the number of training pairs or data. \cref{table:pde} shows the results for viscosities $v=1e-3, 1e-4, 1e-5$, $T=50, 30, 20$ respectively and use $N=1000$. 

\subsection{Details for GPT-2 Downstream Tasks}
We train Pixelfly-GPT2-small on a larger scale dataset, OpenWebText, and evaluate the downstream quality on zero-shot generation and classification tasks from~\citep{zhao2021calibrate}, achieving comparable and even better performance to the dense model. Specifically, the datasets contains five popular classification tasks: SST2, Trec, CB, Agnews, and Dbpedia. We also adapated the calibrated metric from~\citep{zhao2021calibrate} for evaluation. Results for each individual task are shown in~\cref{table:gpt_finetune_full}. 

\begin{table}[h]
  \small
  \centering
  \vspace{-3mm}
  \caption{\label{table:gpt_finetune_full}The performance (accuracy) of GPT-2-medium trained with Monarch reverse sparsification and with conventional dense training on text classification benchmarks.}
  \setlength{\tabcolsep}{5pt}
  \vspace{1em}
   \resizebox{0.7\linewidth}{!}{
  \begin{tabular}{@{}c||ccccc@{}}
    \specialrule{.15em}{.05em}{.05em}
    Model&\multicolumn{1}{c}{OpenWebText (ppl)}&\multicolumn{1}{c}{Speedup}& \multicolumn{1}{c}{Classification (avg acc)} \\
    \specialrule{.15em}{.05em}{.05em}
    GPT-2m& 68.3 & 37.0 & 10.7 & 52.0 & 26.6\\
    Monarch-GPT-2m& 72 & 38.6 & 12.5 & 47.3 & 23.0 \\
    \specialrule{.15em}{.05em}{.05em}
  \end{tabular}
  }
  \vspace{-3mm}
\end{table}

\subsection{Details for BERT Pretraining}
\label{subsec:bert_details}

We follow the training procedure and hyperparameters of the reference
implementation from Nvidia Deep Learning examples
(\url{https://github.com/NVIDIA/DeepLearningExamples}).
In particular, we use the LAMB optimizer with learning rate 4e-3.
We use as large a minibatch size as possible that still fits in the GPU memory
(A100-40GB), and use gradient accumulation to reach an effective batch size of
64k sequences for phase 1 (maximum sequence length 128) and 32k for phase 2
(maximum sequence legnth 512).
We train is mixed precision (fp16 and fp32).

We use all the optimizations that were in Nvidia's BERT implementation
in MLPerf 1.1:
\begin{enumerate}
  \item Only compute the prediction scores (last layer) for masked tokens as
  the outputs of other tokens are not used to compute the masked language
  modeling loss.
  \item Remove padding tokens and only compute the attention for non-padding
  tokens.
  \item Use a fused CUDA kernel (FMHA) that combines 4 steps into one kernel: computes
  $Q K^T$, take softmax, apply dropout, multiply by $V$, where $Q, K, V$ are the
  query, key, and value respectively.
  \item Fuse matrix multiplication and adding bias into one CUDA kernel in the feed-forward network
  (FFN) layers. The gradient of the bias is also fused with the matrix
  multiplication the backward pass.
  \item Fuse matrix multiplication and adding bias into one CUDA kernel in the
  attention output projection.
  \item Fuse dropout and adding residual in the residual connection at the end
  on the attention and FFN blocks.
\end{enumerate}

We train with DeepSpeed~\citep{rasley2020deepspeed} ZeRO optimizer stage 1 to
shard the optimizer states, thus reducing GPU memory usage and allowing us to
use larger batch sizes.
For the Nvidia MLPerf implementation, we report the speed for both Apex's
automatic mix-precision (AMP) level O2 (as in the original implementation), and
DeepSpeed ZeRO optimizer.

\subsection{Accelerated Multi-coil MRI Reconstruction}
\label{sec:experiment_details_mri}

\subsubsection{Background}
In multi-coil MRI, multiple receiver coils (i.e. sensors) acquire complex-valued measurements in the spatial frequency (a.k.a. \textit{k-space}) domain. These measurements are modulated by the spatially-varying sensitivity maps, which characterize the sensitivity of each coil to the imaging target. In accelerated MRI, scan times are reduced by decreasing the number of samples acquired in k-space. Because the data is sampled below the Nyquist rate, reconstructing the underlying image is an ill-posed problem.

The forward problem for accelerated multi-coil MRI can be written as the matrix equation
\begin{equation*}
    y = \Omega\boldsymbol{F}\boldsymbol{S}x + \epsilon
\end{equation*}
where $\Omega$ is the binary undersampling mask that indexes acquired samples in k-space, $y$ is the vectorized measured signal in k-space, $\boldsymbol{F}$ is the discrete Fourier transform matrix, $\boldsymbol{S}$ is the receiver coil sensitivity maps,  $x$ is the ground-truth signal in image-space, and $\epsilon$ is additive complex Gaussian noise. The acceleration factor is given by $R = \frac{\sum_i^{|N|} \Omega_i}{|\Omega|}$.

\subsubsection{Experimental Details}

\paragraph{Dataset.} We benchmark our method on the SKM-TEA Raw Data Track, which consists of dual-echo 3D MRI scans \citep{desai2021skm}. Scans are accelerated using Poisson Disc undersampling masks distributed with the dataset. During training, Poisson Disc masks are generated, cached, and applied to mask the k-space data to simulate accelerated scans.

\paragraph{Matrix Shape.} Like all matrices, Monarch matrices have an explicit shape constraint, which is a limitation of these matrices for MRI reconstruction tasks. Thus, the SKM-TEA dataset was filtered to include scans of shape $512 \times 512 \times 160$, which is the most frequently occuring scan shape. A total of 3 scans were dropped from the original 155 scans in the dataset. Our method and all baselines were trained on this filtered dataset.

\begin{table}[!ht]
    \vspace{-0.5cm}
\centering
\caption{Baseline configurations of the SKM-TEA MRI reconstruction experiments.}
\resizebox{0.6\linewidth}{!}{
\noindent\begin{tabular}{c||cccccc}
  \specialrule{.15em}{.05em}{.05em}
Model & Params & Optimizer & Weight Decay & Learning Rate & Epoch \\
  \specialrule{.15em}{.05em}{.05em}
SENSE & --- & --- & --- & --- & --- \\
U-Net &  7.8M & Adam & 1e-4 & 1e-3 & 20 \\
mSENSE  & 57.5K & Adam & 1e-4 & 1e-3 & 20 \\
  \specialrule{.15em}{.05em}{.05em}
\end{tabular}
}
\label{table:skmtea-config}
\end{table}


\paragraph{Baselines.} We compare our method to two baselines, SENSE and U-Net. Parameter count and hyperparameters are available in Table \ref{table:skmtea-config}.
\begin{itemize}
    \item \textit{SENSE}: SENSE performs a linear combination of the images acquired on each coil \citep{pruessmann1999sense}. Here, the inverse fsat Fourier transform (IFFT) is applied to the acquired k-space for each coil. The resulting images are combined into a single complex image by weighting each coil image by corresponding coil sensitivity maps. In accelerated MRI, the unsampled frequencies are zero-valued; thus, SENSE produces a \textit{zero-filled image}. Note, SENSE does not require any training.
    \item \textit{U-Net}: U-Net is a popular fully convolutional neural network baseline for MRI reconstruction \citep{ronneberger2015u}. We use the default implementation and hyperparameters used by \citet{desai2021skm} to benchmark the SKM-TEA dataset. In this approach, the SENSE-reconstructed zero-filled image is mapped to SENSE-reconstructed ground truth images.
\end{itemize}

\paragraph{Monarch-SENSE (mSENSE):} We propose a modification to the SENSE method, in which the (IFFT) is parameterized by a factorized Monarch matrix. This matrix is initialized to the IFFT but, unlike SENSE, is learnable. While mSENSE is trainable, it has 137x fewer trainable parameters than U-Net.

\paragraph{Metrics:} We evaluate reconstruction performance using peak signal-to-noise ratio (pSNR) and structural similarity (SSIM) on both echoes (echo1 - E1, echo2 - E2) separately. Both metrics were computed on the 3D volume of each echo.

\paragraph{Extended Results.} We provide sample reconstructions of SENSE, mSENSE, and U-Net in data-limited settings for first (Fig.~\ref{fig:mri-data-limited-echo1}) and second (Fig.~\ref{fig:mri-data-limited-echo2}) echoes. Both SENSE and U-Net reconstructed images have aliasing artifacts. Due to the random Poisson Disc undersampling pattern, these artifacts are incoherent, causing them to manifest as blurring around fine structures and edges. In contrast, mSENSE can recover these structures with higher fidelity. Even in the second echo, which has lower signal-to-noise ratio (SNR) than the first echo, mSENSE does not overblur the image.

\begin{figure}
    \centering
    \includegraphics[width=0.9\linewidth]{figures/sample-mri-echo1.pdf}
    \vspace{-1em}
    \caption{Sample reconstructions at 2x acceleration for the first echo in the SKM-TEA dataset using SENSE, Monarch-SENSE (mSENSE), and U-Net. Both mSENSE and U-Net are trained with 1 training scan. SENSE is an untrained method.}
    \label{fig:mri-data-limited-echo1}
\end{figure}

\begin{figure}
    \centering
    \includegraphics[width=6in]{figures/sample-mri-echo2.pdf}
    \vspace{-1em}
    \caption{Sample reconstructions at 2x acceleration for the second echo in the SKM-TEA dataset using SENSE, Monarch SENSE (mSENSE), and U-Net. Both mSENSE and U-Net are trained with 1 training scan. SENSE is an untrained method.}
    \label{fig:mri-data-limited-echo2}
\end{figure}







\section{Extended experimental results} \label{app:exp_results}

\subsection{Expressivity on digital filters}

% {\ourmethod{} can approximate the frequency response of digital filters}

We experimentally verify whether \ourmethod{} can approximate the input--output map of digital filter admitting a state--space representation, with improved generalization over baseline models given test inputs of unseen frequencies.

We generate a dataset of $1028$ sinusoidal signals of length $200$
\[
x(t) = \sin{(2\pi \omega t})
\]
where $\omega \in [2, 40]~\bigcup~[50, 100]$ in the training set and $\omega \in (40, 50)$ in the test set. The outputs are obtained by filtering $x$, i.e., $y = \mathcal{F}(x)$  where $\mathcal{F}$ is in the family of digital filters. 

We introduce common various sequence-to-sequence layers or models as baselines: the original S4 diagonal plus low--rank \citep{gu2021efficiently}, a single-layer LSTM, a single 1d convolution (Conv1d), a dense linear layer (NLinear), a single self--attention layer. All models are trained for $800$ epochs with batch size $256$, learning rate $10^{-3}$ and Adam. We repeat this experiment for digital filters of different orders \citep{oppenheim1999discrete}. The results are shown in Figure \ref{fig:dsp_synthetic}. \ourmethod{} learns to match the frequency response of the target filter, producing the correct output for inputs at test frequencies. 


\begin{table}[h]
\caption{Comparing sequence models on the task of approximating the input--output map defined by digital filters of different orders. Test RMSE on held-out inputs at unseen frequencies.}
    \label{tab:my_label}
    \centering
    \begin{tabular}{cc|ccccccc}
    \toprule
    %
    Filter & Order & \ourmethod & S4 & Conv1D & LSTM & NLinear & Transformer \\
    %
    \midrule
    Butterworth & $2$ & $0.0055$ & $0.0118$ & $0.0112$ & $0.0115$ & $1.8420$ & $0.5535$ \\
    & $3$ & $0.0057$ & $0.3499$ & $0.0449$ & $0.0231$ & $1.7085$ & $0.6639$ \\
    & $10$ & $0.0039$ & $0.8077$ & $0.4747$ & $0.2753$  & $1.5162$ & $0.7191$ \\
    \midrule
    Chebyshev $1$ & $2$ & $0.0187$ & $0.0480$ & $0.0558$ & $0.0285$ & $1.9313$ & $0.2452$ \\
    & $3$ & $0.0055$ & $0.0467$ & $0.0615$ & $0.0178$ & $1.8077$ & $0.4028$ \\
    & $10$ & $0.0620$ & $0.6670$ & $0.1961$ & $0.1463$ & $1.5069$ & $0.7925$ \\
    \midrule
    Chebyshev $2$ & $2$ & $0.0112$ & $0.0121$ & $0.0067$ & $0.0019$ & $0.4101$ & $0.0030$ \\
    & $3$ & $0.0201$ & $0.0110$ & $0.0771$ & $0.0102$ & $0.4261$ & $0.0088$\\
    & $10$ & $0.0063$ &  $0.6209$ & $0.3361$ & $0.1911$ & $1.5584$ & $0.7936$\\
    \midrule
    Elliptic & $2$ & $0.0001$ & $0.0300$ & $0.0565$ & $0.0236$ & $1.9150$ & $0.2445$ \\
    & $3$ & $0.0671$ & $0.0868$ & $0.0551$ & $0.0171$ & $1.8782$  & $0.4198$ \\
    & $10$ & $0.0622$ & $0.0909$ & $0.1352$ & $0.1344$ & $1.4901$ & $0.7368$ \\
    %
    \bottomrule
    %
    \end{tabular}
    
\end{table}
%
% \begin{figure}[h]
%     \centering
%     \includegraphics[width=0.99\columnwidth]{_ICLR2023_paper/figures/dsp_SpaceTime.pdf}
%     \includegraphics[width=0.99\columnwidth]{_ICLR2023_paper/figures/dsp_S4.pdf}
%     \includegraphics[width=0.99\columnwidth]{_ICLR2023_paper/figures/dsp_Conv1d.pdf}
%     \includegraphics[width=0.99\columnwidth]{_ICLR2023_paper/figures/dsp_LSTM.pdf}
%     \includegraphics[width=0.99\columnwidth]{_ICLR2023_paper/figures/dsp_NLinear.pdf}
%     \includegraphics[width=0.99\columnwidth]{_ICLR2023_paper/figures/dsp_Transformer.pdf}
%     \caption{Testing the capability of different sequence--to--sequence models to approximate the input--output map of digital filters. In blue, we show the output signal filtered by each model. The ground--truth digital filter is a Butterworth of order $10$.}
%     \label{fig:dsp_synthetic}
% \end{figure}
% %




\subsection{Informer Forecasting}\label{appendix:informer_extended}

\textbf{Univariate long horizon forecasts with Informer splits.} Beyond the ETT datasets and horizons evaluated on in Table~\ref{tab:informer-s-long-original}, we also compare \ourmethod{} to alternative time series methods on the complete datasets and horizons used in the original Informer paper~\citep{zhou2021informer}. We compare against recent architectures which similarly evaluate on these settings, including ETSFormer~\citep{woo2022etsformer}, SCINet~\citep{liu2021time}, and Yformer~\citep{madhusudhanan2021yformer}, and other comparison methods found in the Informer paper, such as Reformer~\citep{kitaev2020reformer} and ARIMA.
\ourmethod{} obtains best results on 20 out of 25 settings, the most of any method.

\begin{table}[!t]
\caption{ \textbf{Univariate forecasting} results on Informer datasets. \textbf{Best} results in \textbf{bold}. \ourmethod{} obtains best MSE on 19 out of 25 and best MAE on 20 out of 25 dataset and horizon tasks.}
\resizebox{\linewidth}{!}{
\begin{tabular}{@{}c|cbcbcbcbcbcbcbcbcbcbcbcbc@{}}
% \begin{tabular}{@{}c|ccccccccccccccccccccccccc@{}}
\toprule
\multicolumn{2}{c}{Methods}        & \multicolumn{2}{c}{\textbf{\ourmethod{}}} & \multicolumn{2}{c}{ETSFormer}   & \multicolumn{2}{c}{SCINet} & \multicolumn{2}{c}{S4}          & \multicolumn{2}{c}{Yformer}     & \multicolumn{2}{c}{Informer} & \multicolumn{2}{c}{LogTrans} & \multicolumn{2}{c}{Reformer} & \multicolumn{2}{c}{N-BEATS} & \multicolumn{2}{c}{DeepAR} & \multicolumn{2}{c}{ARIMA} & \multicolumn{2}{c}{Prophet} \\ \midrule
\multicolumn{2}{c}{Metric}         & MSE                & MAE               & MSE            & MAE            & MSE          & MAE         & MSE            & MAE            & MSE            & MAE            & MSE           & MAE          & MSE           & MAE          & MSE           & MAE          & MSE          & MAE          & MSE          & MAE         & MSE         & MAE         & MSE          & MAE          \\ \midrule
\parbox[t]{2mm}{\multirow{5}{*}{\rotatebox[origin=c]{90}{ETTh1}}}   & 24  & \textbf{0.026}     & \textbf{0.124}    & 0.030          & 0.132          & 0.031        & 0.132       & 0.061          & 0.191          & 0.082          & 0.230          & 0.098         & 0.247        & 0.103         & 0.259        & 0.222         & 0.389        & 0.042        & 0.156        & 0.107        & 0.280       & 0.108       & 0.284       & 0.115        & 0.275        \\
\multicolumn{1}{c|}{}        & 48  & \textbf{0.038}     & \textbf{0.153}    & 0.041          & 0.154          & 0.051        & 0.173       & 0.079          & 0.220          & 0.139          & 0.308          & 0.158         & 0.319        & 0.167         & 0.328        & 0.284         & 0.445        & 0.065        & 0.200        & 0.162        & 0.327       & 0.175       & 0.424       & 0.168        & 0.330        \\
\multicolumn{1}{c|}{}        & 168 & 0.066              & 0.209             & \textbf{0.065} & \textbf{0.203} & 0.081        & 0.222       & 0.104          & 0.258          & 0.111          & 0.268          & 0.183         & 0.346        & 0.207         & 0.375        & 1.522         & 1.191        & 0.106        & 0.255        & 0.239        & 0.422       & 0.396       & 0.504       & 1.224        & 0.763        \\
\multicolumn{1}{c|}{}        & 336 & \textbf{0.069}     & \textbf{0.212}    & 0.071          & 0.215          & 0.094        & 0.242       & 0.080          & 0.229          & 0.195          & 0.365          & 0.222         & 0.387        & 0.230         & 0.398        & 1.860         & 1.124        & 0.127        & 0.284        & 0.445        & 0.552       & 0.468       & 0.593       & 1.549        & 1.820        \\
\multicolumn{1}{c|}{}        & 720 & \textbf{0.075}     & \textbf{0.226}    & 0.079          & 0.227          & 0.176        & 0.343       & 0.116          & 0.271          & 0.226          & 0.394          & 0.269         & 0.435        & 0.273         & 0.463        & 2.112         & 1.436        & 0.269        & 0.422        & 0.658        & 0.707       & 0.659       & 0.766       & 2.735        & 3.253        \\ \midrule
\parbox[t]{2mm}{\multirow{5}{*}{\rotatebox[origin=c]{90}{ETTh2}}}   & 24  & \textbf{0.064}     & \textbf{0.189}    & 0.087          & 0.232          & 0.070        & 0.194       & 0.095          & 0.234          & 0.082          & 0.221          & 0.093         & 0.240        & 0.102         & 0.255        & 0.263         & 0.437        & 0.078        & 0.210        & 0.098        & 0.263       & 3.554       & 0.445       & 0.199        & 0.381        \\
\multicolumn{1}{c|}{}        & 48  & \textbf{0.095}     & \textbf{0.230}    & 0.112          & 0.263          & 0.102        & 0.242       & 0.191          & 0.346          & 0.172          & 0.334          & 0.155         & 0.314        & 0.169         & 0.348        & 0.458         & 0.545        & 0.123        & 0.271        & 0.163        & 0.341       & 3.190       & 0.474       & 0.304        & 0.462        \\
\multicolumn{1}{c|}{}        & 168 & \textbf{0.144}     & \textbf{0.300}    & 0.169          & 0.325          & 0.157        & 0.311       & 0.167          & 0.333          & 0.174          & 0.337          & 0.232         & 0.389        & 0.246         & 0.422        & 1.029         & 0.879        & 0.244        & 0.393        & 0.255        & 0.414       & 2.800       & 0.595       & 2.145        & 1.068        \\
\multicolumn{1}{c|}{}        & 336 & \textbf{0.169}     & \textbf{0.333}    & 0.216          & 0.379          & 0.177        & 0.340       & 0.189          & 0.361          & 0.224          & 0.391          & 0.263         & 0.417        & 0.267         & 0.437        & 1.668         & 1.228        & 0.270        & 0.418        & 0.604        & 0.607       & 2.753       & 0.738       & 2.096        & 2.543        \\
\multicolumn{1}{c|}{}        & 720 & 0.188              & \textbf{0.352}    & 0.226          & 0.385          & 0.253        & 0.403       & \textbf{0.187} & 0.358          & 0.211          & 0.382          & 0.277         & 0.431        & 0.303         & 0.493        & 2.030         & 1.721        & 0.281        & 0.432        & 0.429        & 0.580       & 2.878       & 1.044       & 3.355        & 4.664        \\ \midrule
\parbox[t]{2mm}{\multirow{5}{*}{\rotatebox[origin=c]{90}{ETTm1}}}   & 24  & \textbf{0.010}     & \textbf{0.074}    & 0.013          & 0.084          & 0.019        & 0.088       & 0.024          & 0.117          & 0.024          & 0.118          & 0.030         & 0.137        & 0.065         & 0.202        & 0.095         & 0.228        & 0.031        & 0.117        & 0.091        & 0.243       & 0.090       & 0.206       & 0.120        & 0.290        \\
\multicolumn{1}{c|}{}        & 48  & \textbf{0.019}     & \textbf{0.101}    & 0.020          & 0.107          & 0.045        & 0.143       & 0.051          & 0.174          & 0.048          & 0.173          & 0.069         & 0.203        & 0.078         & 0.220        & 0.249         & 0.390        & 0.056        & 0.168        & 0.219        & 0.362       & 0.179       & 0.306       & 0.133        & 0.305        \\
\multicolumn{1}{c|}{}        & 96  & \textbf{0.026}     & \textbf{0.121}    & 0.030          & 0.132          & 0.072        & 0.198       & 0.086          & 0.229          & 0.143          & 0.311          & 0.194         & 0.372        & 0.199         & 0.386        & 0.920         & 0.767        & 0.095        & 0.234        & 0.364        & 0.496       & 0.272       & 0.399       & 0.194        & 0.396        \\
\multicolumn{1}{c|}{}        & 288 & \textbf{0.051}     & \textbf{0.176}    & 0.053          & 0.179          & 0.117        & 0.266       & 0.160          & 0.327          & 0.150          & 0.316          & 0.401         & 0.554        & 0.411         & 0.572        & 1.108         & 1.245        & 0.157        & 0.311        & 0.948        & 0.795       & 0.462       & 0.558       & 0.452        & 0.574        \\
\multicolumn{1}{c|}{}        & 672 & 0.078              & 0.220             & \textbf{0.075} & \textbf{0.214} & 0.180        & 0.328       & 0.292          & 0.466          & 0.305          & 0.476          & 0.512         & 0.644        & 0.598         & 0.702        & 1.793         & 1.528        & 0.207        & 0.370        & 2.437        & 1.352       & 0.639       & 0.697       & 2.747        & 1.174        \\ \midrule
\parbox[t]{2mm}{\multirow{5}{*}{\rotatebox[origin=c]{90}{Weather}}} & 24  & \textbf{0.088}     & \textbf{0.205}    & -              & -              & -            & -           & 0.125          & 0.254          & -              & -              & 0.117         & 0.251        & 0.136         & 0.279        & 0.231         & 0.401        & -            & -            & 0.128        & 0.274       & 0.219       & 0.355       & 0.302        & 0.433        \\
\multicolumn{1}{c|}{}        & 48  & \textbf{0.134}     & \textbf{0.258}    & -              & -              & -            & -           & 0.181          & 0.305          & -              & -              & 0.178         & 0.318        & 0.206         & 0.356        & 0.328         & 0.423        & -            & -            & 0.203        & 0.353       & 0.273       & 0.409       & 0.445        & 0.536        \\
\multicolumn{1}{c|}{}        & 168 & 0.221              & 0.349             & -              & -              & -            & -           & \textbf{0.198} & \textbf{0.333} & -              & -              & 0.266         & 0.398        & 0.309         & 0.439        & 0.654         & 0.634        & -            & -            & 0.293        & 0.451       & 0.503       & 0.599       & 2.441        & 1.142        \\
\multicolumn{1}{c|}{}        & 336 & \textbf{0.268}     & \textbf{0.380}    & -              & -              & -            & -           & 0.300          & 0.417          & -              & -              & 0.297         & 0.416        & 0.359         & 0.484        & 1.792         & 1.093        & -            & -            & 0.585        & 0.644       & 0.728       & 0.730       & 1.987        & 2.468        \\
\multicolumn{1}{c|}{}        & 720 & 0.345              & 0.451             & -              & -              & -            & -           & \textbf{0.245} & \textbf{0.375} & -              & -              & 0.359         & 0.466        & 0.388         & 0.499        & 2.087         & 1.534        & -            & -            & 0.499        & 0.596       & 1.062       & 0.943       & 3.859        & 1.144        \\ \midrule
\parbox[t]{2mm}{\multirow{5}{*}{\rotatebox[origin=c]{90}{ECL} }}    & 48  & \textbf{0.184}     & \textbf{0.306}    & -              & -              & -            & -           & 0.222          & 0.350          & 0.194          & 0.322          & 0.239         & 0.359        & 0.280         & 0.429        & 0.971         & 0.884        & -            & -            & 0.204        & 0.357       & 0.879       & 0.764       & 0.524        & 0.595        \\
\multicolumn{1}{c|}{}        & 168 & \textbf{0.250}     & \textbf{0.353}    & -              & -              & -            & -           & 0.331          & 0.421          & 0.260          & 0.361          & 0.447         & 0.503        & 0.454         & 0.529        & 1.671         & 1.587        & -            & -            & 0.315        & 0.436       & 1.032       & 0.833       & 2.725        & 1.273        \\
\multicolumn{1}{c|}{}        & 336 & 0.288              & 0.382             & -              & -              & -            & -           & 0.328          & 0.422          & \textbf{0.269} & \textbf{0.375} & 0.489         & 0.528        & 0.514         & 0.563        & 3.528         & 2.196        & -            & -            & 0.414        & 0.519       & 1.136       & 0.876       & 2.246        & 3.077        \\
\multicolumn{1}{c|}{}        & 720 & \textbf{0.355}     & \textbf{0.446}    & -              & -              & -            & -           & 0.428          & 0.494          & 0.427          & 0.479          & 0.540         & 0.571        & 0.558         & 0.609        & 4.891         & 4.047        & -            & -            & 0.563        & 0.595       & 1.251       & 0.933       & 4.243        & 1.415        \\
\multicolumn{1}{c|}{}        & 960 & \textbf{0.393}     & \textbf{0.478}    & -              & -              & -            & -           & 0.432          & 0.497          & 0.595          & 0.573          & 0.582         & 0.608        & 0.624         & 0.645        & 7.019         & 5.105        & -            & -            & 0.657        & 0.683       & 1.370       & 0.982       & 6.901        & 4.260        \\ \midrule
\multicolumn{2}{c}{Count}          & \textbf{19}        & \textbf{20}       & 2              & 2              & 0            & 0           & 3              & 2              & 1              & 1              & 0             & 0            & 0             & 0            & 0             & 0            & 0            & 0            & 0            & 0           & 0           & 0           & 0            & 0            \\ \bottomrule
\end{tabular}
}
\label{tab:informer-s-long-original}
\end{table}


% \begin{table}[!t]
% \caption{ \textbf{Univariate forecasting} results on Informer datasets. \textbf{Best} results in \textbf{bold}. \ourmethod{} obtains best MSE on 19 out of 25 and best MAE on 20 out of 25 dataset and horizon tasks.}
% \resizebox{\linewidth}{!}{
% \begin{tabular}{@{}c|cbcbcbcbcbcbcbcbcbcbcbcbc@{}}
% \toprule
% \multicolumn{2}{c}{Methods}                         & \multicolumn{2}{c}{\textbf{\ourmethod{}}}   & \multicolumn{2}{c}{ETSFormer}   & \multicolumn{2}{c}{SCINet} & \multicolumn{2}{c}{S4}          & \multicolumn{2}{c}{Yformer}     & \multicolumn{2}{c}{Informer} & \multicolumn{2}{c}{LogTrans} & \multicolumn{2}{c}{Reformer} & \multicolumn{2}{c}{DeepAR} & \multicolumn{2}{c}{ARIMA} & \multicolumn{2}{c}{Prophet} \\ \midrule
% \multicolumn{2}{c}{Metric}                          & MSE            & MAE            & MSE            & MAE            & MSE          & MAE         & MSE            & MAE            & MSE            & MAE            & MSE           & MAE          & MSE           & MAE          & MSE           & MAE          & MSE          & MAE         & MSE         & MAE         & MSE          & MAE          \\ \midrule
% \parbox[t]{2mm}{\multirow{5}{*}{\rotatebox[origin=c]{90}{ETTh1}}}   & 24  & \textbf{0.026} & \textbf{0.124} & 0.030          & 0.132          & 0.031        & 0.132       & 0.061          & 0.191          & 0.082          & 0.230          & 0.098         & 0.247        & 0.103         & 0.259        & 0.222         & 0.389        & 0.107        & 0.280       & 0.108       & 0.284       & 0.115        & 0.275        \\
% \multicolumn{1}{c|}{}                         & 48  & \textbf{0.038} & \textbf{0.153} & 0.041          & 0.154          & 0.051        & 0.173       & 0.079          & 0.220          & 0.139          & 0.308          & 0.158         & 0.319        & 0.167         & 0.328        & 0.284         & 0.445        & 0.162        & 0.327       & 0.175       & 0.424       & 0.168        & 0.330        \\
% \multicolumn{1}{c|}{}                         & 168 & 0.066          & 0.209          & \textbf{0.065} & \textbf{0.203} & 0.081        & 0.222       & 0.104          & 0.258          & 0.111          & 0.268          & 0.183         & 0.346        & 0.207         & 0.375        & 1.522         & 1.191        & 0.239        & 0.422       & 0.396       & 0.504       & 1.224        & 0.763        \\
% \multicolumn{1}{c|}{}                         & 336 & \textbf{0.069} & \textbf{0.212} & 0.071          & 0.215          & 0.094        & 0.242       & 0.080          & 0.229          & 0.195          & 0.365          & 0.222         & 0.387        & 0.230         & 0.398        & 1.860         & 1.124        & 0.445        & 0.552       & 0.468       & 0.593       & 1.549        & 1.820        \\
% \multicolumn{1}{c|}{}                         & 720 & \textbf{0.075} & \textbf{0.226} & 0.079          & 0.227          & 0.176        & 0.343       & 0.116          & 0.271          & 0.226          & 0.394          & 0.269         & 0.435        & 0.273         & 0.463        & 2.112         & 1.436        & 0.658        & 0.707       & 0.659       & 0.766       & 2.735        & 3.253        \\ \midrule
% \parbox[t]{2mm}{\multirow{5}{*}{\rotatebox[origin=c]{90}{ETTh2}}}   & 24  & \textbf{0.064} & \textbf{0.189} & 0.087          & 0.232          & 0.070        & 0.194       & 0.095          & 0.234          & 0.082          & 0.221          & 0.093         & 0.240        & 0.102         & 0.255        & 0.263         & 0.437        & 0.098        & 0.263       & 3.554       & 0.445       & 0.199        & 0.381        \\
% \multicolumn{1}{c|}{}                         & 48  & \textbf{0.095} & \textbf{0.230} & 0.112          & 0.263          & 0.102        & 0.242       & 0.191          & 0.346          & 0.172          & 0.334          & 0.155         & 0.314        & 0.169         & 0.348        & 0.458         & 0.545        & 0.163        & 0.341       & 3.190       & 0.474       & 0.304        & 0.462        \\
% \multicolumn{1}{c|}{}                         & 168 & \textbf{0.144} & \textbf{0.300} & 0.169          & 0.325          & 0.157        & 0.311       & 0.167          & 0.333          & 0.174          & 0.337          & 0.232         & 0.389        & 0.246         & 0.422        & 1.029         & 0.879        & 0.255        & 0.414       & 2.800       & 0.595       & 2.145        & 1.068        \\
% \multicolumn{1}{c|}{}                         & 336 & \textbf{0.169} & \textbf{0.333} & 0.216          & 0.379          & 0.177        & 0.340       & 0.189          & 0.361          & 0.224          & 0.391          & 0.263         & 0.417        & 0.267         & 0.437        & 1.668         & 1.228        & 0.604        & 0.607       & 2.753       & 0.738       & 2.096        & 2.543        \\
% \multicolumn{1}{c|}{}                         & 720 & \textbf{0.188} & \textbf{0.352} & 0.226          & 0.385          & 0.253        & 0.403       & 0.187          & 0.358          & 0.211          & 0.382          & 0.277         & 0.431        & 0.303         & 0.493        & 2.030         & 1.721        & 0.429        & 0.580       & 2.878       & 1.044       & 3.355        & 4.664        \\ \midrule
% \parbox[t]{2mm}{\multirow{5}{*}{\rotatebox[origin=c]{90}{ETTm1}}}   & 24  & \textbf{0.010} & \textbf{0.074} & 0.013          & 0.084          & 0.019        & 0.088       & 0.024          & 0.117          & 0.024          & 0.118          & 0.030         & 0.137        & 0.065         & 0.202        & 0.095         & 0.228        & 0.091        & 0.243       & 0.090       & 0.206       & 0.120        & 0.290        \\
% \multicolumn{1}{c|}{}                         & 48  & \textbf{0.019} & \textbf{0.101} & 0.020          & 0.107          & 0.045        & 0.143       & 0.051          & 0.174          & 0.048          & 0.173          & 0.069         & 0.203        & 0.078         & 0.220        & 0.249         & 0.390        & 0.219        & 0.362       & 0.179       & 0.306       & 0.133        & 0.305        \\
% \multicolumn{1}{c|}{}                         & 96  & \textbf{0.026} & \textbf{0.121} & 0.030          & 0.132          & 0.072        & 0.198       & 0.086          & 0.229          & 0.143          & 0.311          & 0.194         & 0.372        & 0.199         & 0.386        & 0.920         & 0.767        & 0.364        & 0.496       & 0.272       & 0.399       & 0.194        & 0.396        \\
% \multicolumn{1}{c|}{}                         & 288 & \textbf{0.051} & \textbf{0.176} & 0.053          & 0.179          & 0.117        & 0.266       & 0.160          & 0.327          & 0.150          & 0.316          & 0.401         & 0.554        & 0.411         & 0.572        & 1.108         & 1.245        & 0.948        & 0.795       & 0.462       & 0.558       & 0.452        & 0.574        \\
% \multicolumn{1}{c|}{}                         & 672 & 0.078          & 0.220          & \textbf{0.075} & \textbf{0.214} & 0.180        & 0.328       & 0.292          & 0.466          & 0.305          & 0.476          & 0.512         & 0.644        & 0.598         & 0.702        & 1.793         & 1.528        & 2.437        & 1.352       & 0.639       & 0.697       & 2.747        & 1.174        \\ \midrule
% \parbox[t]{2mm}{\multirow{5}{*}{\rotatebox[origin=c]{90}{Weather}}}  & 24  & \textbf{0.088} & \textbf{0.205} & -              & -              & -            & -           & 0.125          & 0.254          & -              & -              & 0.117         & 0.251        & 0.136         & 0.279        & 0.231         & 0.401        & 0.128        & 0.274       & 0.219       & 0.355       & 0.302        & 0.433        \\
% \multicolumn{1}{c|}{}                         & 48  & \textbf{0.134} & \textbf{0.258} & -              & -              & -            & -           & 0.181          & 0.305          & -              & -              & 0.178         & 0.318        & 0.206         & 0.356        & 0.328         & 0.423        & 0.203        & 0.353       & 0.273       & 0.409       & 0.445        & 0.536        \\
% \multicolumn{1}{c|}{}                         & 168 & 0.221          & 0.349          & -              & -              & -            & -           & \textbf{0.198} & \textbf{0.333} & -              & -              & 0.266         & 0.398        & 0.309         & 0.439        & 0.654         & 0.634        & 0.293        & 0.451       & 0.503       & 0.599       & 2.441        & 1.142        \\
% \multicolumn{1}{c|}{}                         & 336 & \textbf{0.268} & \textbf{0.380} & -              & -              & -            & -           & 0.300          & 0.417          & -              & -              & 0.297         & 0.416        & 0.359         & 0.484        & 1.792         & 1.093        & 0.585        & 0.644       & 0.728       & 0.730       & 1.987        & 2.468        \\
% \multicolumn{1}{c|}{}                         & 720 & 0.345          & 0.451          & -              & -              & -            & -           & \textbf{0.245} & \textbf{0.375} & -              & -              & 0.359         & 0.466        & 0.388         & 0.499        & 2.087         & 1.534        & 0.499        & 0.596       & 1.062       & 0.943       & 3.859        & 1.144        \\ \midrule
% \parbox[t]{2mm}{\multirow{5}{*}{\rotatebox[origin=c]{90}{ECL}}}      & 48  & \textbf{0.184} & \textbf{0.306} & -              & -              & -            & -           & 0.222          & 0.350          & 0.194          & 0.322          & 0.239         & 0.359        & 0.280         & 0.429        & 0.971         & 0.884        & 0.204        & 0.357       & 0.879       & 0.764       & 0.524        & 0.595        \\
% \multicolumn{1}{c|}{}                         & 168 & \textbf{0.250} & \textbf{0.353} & -              & -              & -            & -           & 0.331          & 0.421          & 0.260          & 0.361          & 0.447         & 0.503        & 0.454         & 0.529        & 1.671         & 1.587        & 0.315        & 0.436       & 1.032       & 0.833       & 2.725        & 1.273        \\
% \multicolumn{1}{c|}{}                         & 336 & 0.288          & 0.382          & -              & -              & -            & -           & 0.328          & 0.422          & \textbf{0.269} & \textbf{0.375} & 0.489         & 0.528        & 0.514         & 0.563        & 3.528         & 2.196        & 0.414        & 0.519       & 1.136       & 0.876       & 2.246        & 3.077        \\
% \multicolumn{1}{c|}{}                         & 720 & \textbf{0.355} & \textbf{0.446} & -              & -              & -            & -           & 0.428          & 0.494          & 0.427          & 0.479          & 0.540         & 0.571        & 0.558         & 0.609        & 4.891         & 4.047        & 0.563        & 0.595       & 1.251       & 0.933       & 4.243        & 1.415        \\
% \multicolumn{1}{c|}{}                         & 960 & \textbf{0.393} & \textbf{0.478} & -              & -              & -            & -           & 0.432          & 0.497          & 0.595          & 0.573          & 0.582         & 0.608        & 0.624         & 0.645        & 7.019         & 5.105        & 0.657        & 0.683       & 1.370       & 0.982       & 6.901        & 4.260        \\ \midrule
% \multicolumn{2}{c}{Count}                           & \textbf{20}             & \textbf{20}             & 2              & 2              & 0            & 0           & 2              & 2              & 1              & 1              & 0             & 0            & 0             & 0            & 0             & 0            & 0            & 0           & 0           & 0           & 0            & 0            \\ \bottomrule
% \end{tabular}
% }
% \label{tab:informer-s-long-original}
% \end{table}




\textbf{Multivariate signals.} We additionally compare the performance of \ourmethod{} to state-of-the-art comparison methods on ETT multivariate settings. We focus on horizon length $720$, the longest evaluated in prior works. In Table \ref{tab:informer-m-long}, we find \ourmethod{} is competitive with NLinear, which achieves best performance among compparison methods. \ourmethod{} also notably outperforming S4 by large margins, supporting the companion matrix representation once more.   

% Please add the following required packages to your document preamble:
% \usepackage{booktabs}
\begin{table}[!t]
\caption{\textbf{Multivariate forecasting} results on Informer datasets. \textbf{Best} results in \textbf{bold}. \ourmethod{} obtains MSE and MAE competitive with NLinear, the prior state-of-the-art.}
\resizebox{\linewidth}{!}{
\begin{tabular}{@{}ccbcbcbcbcbcbcbc@{}}
\toprule
\multicolumn{2}{c}{Methods}      & \multicolumn{2}{c}{\ourmethod{}}   & \multicolumn{2}{c}{NLinear}     & \multicolumn{2}{c}{FiLM} & \multicolumn{2}{c}{S4} & \multicolumn{2}{c}{FEDformer} & \multicolumn{2}{c}{Autoformer} & \multicolumn{2}{c}{Informer} \\ \midrule
\multicolumn{2}{c}{Metric}       & MSE            & MAE            & MSE            & MAE            & MSE         & MAE        & MSE        & MAE       & MSE           & MAE           & MSE            & MAE           & MSE           & MAE          \\ \midrule
\multicolumn{1}{c|}{ETTh1} & 720 & 0.499          & 0.480           & \textbf{0.440}  & \textbf{0.453} & 0.465       & 0.472      & 1.074      & 0.814     & 0.506         & 0.507         & 0.514          & 0.512         & 1.181         & 0.865        \\
\multicolumn{1}{c|}{ETTh2} & 720 & 0.402          & \textbf{0.434} & \textbf{0.394} & 0.436          & 0.439       & 0.456      & 2.973      & 1.333     & 0.463         & 0.474         & 0.515          & 0.511         & 3.647         & 1.625        \\
\multicolumn{1}{c|}{ETTm1} & 720 & \textbf{0.408} & \textbf{0.415} & 0.433          & 0.422          & 0.420       & 0.420      & 0.738      & 0.655     & 0.543         & 0.49          & 0.671          & 0.561         & 1.166         & 0.823        \\
\multicolumn{1}{c|}{ETTm2} & 720 & \textbf{0.358} & \textbf{0.378} & 0.368 & 0.384 & 0.393       & 0.422      & 2.074      & 1.074     & 0.421         & 0.415         & 0.433          & 0.432         & 3.379         & 1.338        \\ \bottomrule
\end{tabular}
}
\label{tab:informer-m-long}
\end{table}


\subsection{Monash Forecasting} \label{app:monash_exps}

We report the results across all datasets in Table \ref{tab:monash}. We also investigate the performance of models by aggregating datasets based on common characteristics. Concretely, we generate sets of tasks\footnote{A task can belong to multiple splits, resulting in overlapping splits. For example, a task can involve both long context as well as long forecasting horizon.} based on the following properties: 
\begin{itemize}
    \item \textit{Large dataset:} the dataset contains more than $2000$ effective training samples.
    \item \textit{Long context:} the models are provided a context of length greater than $20$ as input.
    \item \textit{Long horizon:} and the models are asked to forecast longer than $20$ steps in the future.
\end{itemize} 
Figure \ref{fig:monash_rankings} shows the average $x/13$ model ranking in terms of test RMSE across splits. We contextualize \ourmethod{} results with best classical and deep learning methods (TBATS and DeepAR). \ourmethod{} relative performance is noticeably higher when context and forecasting horizons are longer, and when a larger number of samples is provided during training. 


%\begin{wrapfigure}[23]{r}{0.55\textwidth}
\begin{figure}[ht]
    \centering
    \includegraphics[width=0.5\linewidth]{_ICLR2023_paper/figures/monash_rankings2.pdf}
    \caption{ Relative test RMSE rankings ($*/13$ models) across different slices of the $33$ datasets in the Monash repository \citep{godahewa2021monash}. \ourmethod{} sets best overall ranking across all tasks and is significantly more accurate on tasks involving long forecast horizon and larger number of training samples.}
    \label{fig:monash_rankings}
\end{figure}
%\end{wrapfigure}


\subsection{ECG Classification}\label{appendix:ecg_results}

In addition to our results table in the main paper, we also provide the mean and standard deviations of the two models we ran in house (\ourmethod{} and S4) in Table \ref{tab:ecg_stats}.

\begin{table}[H]
    \centering
    \caption{ \textbf{ECG statement classification} on PTB-XL (100 Hz version). We report the mean and standard deviation over three random seeds for the three methods we ran in house.}
    \label{tab:ecg_stats}
    \begin{tabular}{@{}lcccccc@{}}
\toprule
Task AUROC & All            & Diag           & Sub-diag       & Super-diag     & Form           & Rhythm         \\ \midrule
\ourmethod{}            & $93.6 (0.13)$   & $94.1 (0.12)$ & $93.3(0.34)$ & $92.9 (0.09)$ & $88.3(0.63)$          & $96.7 (0.05)$    \\
S4    & $93.8 (0.38)$ & $93.9 (0.15)$  & $92.9 (0.11)$ & $93.1 (0.07)$    & $89.5 (0.66)$  & $97.7 (0.04)$ \\
Transformer    & $85.7 (0.30)$ & $87.6 (0.41)$  & $88.2 (0.20)$ & $88.7 (0.28)$    & $77.1 (0.45)$  & $83.1 (0.72)$ \\
\bottomrule
\end{tabular}
    
\end{table}


\subsection{Efficiency Results}

We additionally empirically validate that \ourmethod{} trains in near-linear time with horizon sequence length. We also use synthetic data, scaling horizon from $1 - 1000$. 

\begin{figure}[H]
  \centering
    \includegraphics[width=0.5\textwidth]{_ICLR2023_paper/figures/speed_benchmark_horizon.pdf}
  \caption{Training wall-clock time versus horizon length for \ourmethod{}, S4, LSTM, and Transformer. }
  \label{fig:horizon_efficiency} 
\end{figure}


\subsection{\ourmethod{} Ablations}\label{appendix:ablations}
To better understand how the proposed \ourmethod{} SSMs lead to the improved empirical performance, we include ablations on the individual closed-loop forecasting SSM (Section~\ref{sec:forecasting_ssm}) and preprocessing SSMs (Section~\ref{sec:preprocessing_ssms}). 

\subsubsection{Closed-loop Forecasting SSM}\label{appendix:ablations_closed_loop_ssm}
To study how the closed-loop SSM improves long horizon forecasting accuracy, we remove the closed-loop SSM component in our default \ourmethod{} forecasting architecture (\cf{} Appendix~\ref{appendix:architectures}, and compare the default \ourmethod{} with one without any closed-loop SSMs on Informer forecasting tasks. For models without closed-loop SSMs, we replace the last layer with the standard ``open-loop'' SSM framework in Section~\ref{sec:method_spacetime_layer}), and keep all other layers the same. Finally, for baseline comparison against another SSM without the closed-loop component, we compare against S4. 

In Table~\ref{tab:ablation_results_closed_loop_ssm}, we report standardized MSE on Informer ETT datasets. Adding the closed-loop SSM consistently improves forecasting accuracy, on average lowering relative MSE by 33.2\%. Meanwhile, even without the closed-loop SSM, \ourmethod{} outperforms S4, again suggesting that the companion matrix parameterization is beneficial for autoregressive time series forecasting. 

\begin{table}[H]
    \centering
    \caption{ \textbf{Closed-loop SSM Ablation} We ablate the closed-loop SSM component in \ourmethod{}, comparing against the prior S4 SSM on four Informer time series forecasting tasks. Removing the closed-loop SSM consistently hurts forecasting accuracy for \ourmethod{}. }
    \label{tab:ablation_results_closed_loop_ssm}
    \begin{tabular}{@{}lbcbcbcbc@{}}
\toprule
                            & \multicolumn{2}{c}{ETTh1 (720)} & \multicolumn{2}{c}{ETTh2 (720)} & \multicolumn{2}{c}{ETTm1 (720)} & \multicolumn{2}{c}{ETTm2 (720)} \\ \cmidrule(l){2-9} 
Method / Ablation                      & MSE            & MAE            & MSE            & MAE            & MSE            & MAE            & MSE            & MAE            \\ \midrule
\ourmethod{}                & 0.076          & 0.222          & 0.188          & 0.352          & 0.074          & 0.213          & 0.166          & 0.318          \\
\ourmethod{} No Closed-loop & 0.114          & 0.271          & 0.278          & 0.431          & 0.156          & 0.310          & 0.213          & 0.365          \\
S4 (No Closed-loop)         & 0.190          & 0.355          & 0.630          & 0.662          & 0.254          & 0.433          & 0.482          & 0.567          \\ \bottomrule
\end{tabular}
    
\end{table}

\subsubsection{Preprocessing SSM}\label{appendix:ablations_preprocessing_ssm}

To study how the preprocessing SSM improves long horizon forecasting accuracy, we next compare how \ourmethod{} performs with and without the weight-initializing preprocessing SSMs introduced in Section~\ref{sec:preprocessing_ssms}. We compare the default \ourmethod{} architecture (Table~\ref{tab:spacetime_forecasting_arch} with (1) replacing the preprocessing SSMs with randomly initialized default companion SSMs, and (2) removing the preprocessing SSMs altogether. For the former, we preserve the number of layers, but now train the first-layer SSM weights. For the latter, there is one-less layer, but the same number of trainable parameters (as we fix and freeze the weights for each preprocessing SSM). 


In Table~\ref{tab:ablation_results_preprocessing_ssm}, we report standardized MSE on Informer ETT datasets. We find fixing the first layer SSMs of a \ourmethod{} network to preprocessing SSMs consistently improves forecasting performance, achieving 4.55\% lower MSE on average than the ablation with just trainable companion matrices. Including the preprocessing layer also improves MSE by 9.26\% on average compared to removing the layer altogether. These results suggest that preprocessing SSMs are beneficial for time series forecasting, \eg{} by performing classic time series modeling techniques on input data. Unlike other approaches, \ourmethod{} is able to flexibly and naturally incorporate these operations into its network layers via simple weight initializations of the same general companion SSM structure. 


\begin{table}[H]
    \centering
    \caption{ \textbf{Preprocessing SSM Ablation} We ablate the preprocessing SSM layer in \ourmethod{}, comparing against either replacing the SSMs with companion SSMs (Companion) or removing the layer (Removed). Including preprocessing SSMs consistently improves forecasting accuracy. }
    \label{tab:ablation_results_preprocessing_ssm}
\resizebox{\linewidth}{!}{
\begin{tabular}{@{}lbcbcbcbc@{}}
\toprule
                                           & \multicolumn{2}{c}{ETTh1 (720)} & \multicolumn{2}{c}{ETTh2 (720)} & \multicolumn{2}{c}{ETTm1 (720)} & \multicolumn{2}{c}{ETTm2 (720)} \\ \cmidrule(l){2-9} 
Method / Ablation                          & MSE            & MAE            & MSE            & MAE            & MSE            & MAE            & MSE            & MAE            \\ \midrule
SpaceTime                                  & 0.076          & 0.222          & 0.188          & 0.352          & 0.074          & 0.213          & 0.166          & 0.318          \\
SpaceTime No Preprocessing (Companion)     & 0.076          & 0.224          & 0.194          & 0.358          & 0.079          & 0.218          & 0.182          & 0.336          \\
SpaceTime No Preprocessing (Removed) & 0.078          & 0.227          & 0.204          & 0.367          & 0.087          & 0.232          & 0.188          & 0.326          \\ \bottomrule
\end{tabular}
}    
\end{table}


\subsection{\ourmethod{} Architectures}\label{appendix:architectures}

We provide the specific \ourmethod{} architecture configurations used for forecasting and classification tasks. Each configuration follows the general architecture presented in Section~\ref{sec:expressive_ssm_layer} and Figure~\ref{fig:arch_overview}, and consists of repeated Multi-SSM \ourmethod{} layers. We first provide additional details on specific instantiations of the companion SSMs we use in our models, \eg{} how we instantiate preprocessing SSMs to recover specific techniques (Section~\ref{sec:preprocessing_ssms}). We then include the layer-specific details of the number and type of SSM used in each network. 

\subsubsection{Specific SSM parameterizations}\label{appendix:specific_ssm_parameterizations}
In Section~\ref{sec:expressive_ssm_with_companion}, we described the general form of the companion SSM used in this work. By default, for any individual SSM we learn the $a$ column in $\zA$ and the vectors $\zB, \zC$ as trainable parameters in a neural net module. We refer to these SSMs specifically as \textbf{companion SSMs}. 

In addition, as discussed in Sections~\ref{sec:expressive_ssm_with_companion} and ~\ref{sec:preprocessing_ssms}, we can also fix $a$, $\zB$, or $\zC$ to specific values to recover useful operations when computing the SSM outputs. We describe specific instantiations of the companion SSM used in our models below (with dimensionality referring to one SSM). 

\header{Shift SSM}
We fix the $\boldsymbol{a}$ vector in the companion state matrix $\zA \in \mathbb{R}^{d \times d}$ to the $\boldsymbol{0}$ vector $\in \mathbb{R}^d$, such that $\zA$ is the shift matrix (see Eq.~\ref{eq:shift_matrix_example} for an example). This is a generalization of a 1-D ``sliding window'' convolution with fixed kernel size equal to SSM state dimension $d$. To see how, note that if $\zB$ is also fixed to the first basis vector $\boldsymbol{e_1} \in \mathbb{R}^{d \times 1}$, then this exactly recovers a 1-D convolution with kernel determined by $\zC$.

\header{Differencing SSM}
As a specific version of the preprocessing SSM discussed in Section~\ref{sec:preprocessing_ssms}, we fix $\boldsymbol{a} = \boldsymbol{0}$, $\zB = \boldsymbol{e_1}$, and set $\zC$ to recover various order differencing when computing the SSM, \ie{}
% \begin{align}
% \zC = \bmatrix 1 & -1 & 0 & \ldots & 0}\;\;\;\text{(1st-order differencing)}
% \end{align}
\begin{align}
    \zC &= 
    \begin{bmatrix}
    1 & \phantom{-}0 & 0 & \phantom{-}0 & 0 & \ldots & 0 \\
    \end{bmatrix}
    \;\;\;\;\;\;
    \begin{matrix}
    \hfill\text{(0-order differencing, \ie{} an identity function)} \\
    \end{matrix} \\
    \zC &= 
    \begin{bmatrix}
    1 & -1 & 0 & \phantom{-}0 & 0 & \ldots & 0 \\
    \end{bmatrix}
    \;\;\;\;\;\;
    \begin{matrix}
    \hfill\text{(1st-order differencing)} \\
    \end{matrix} \\
    \zC &= 
    \begin{bmatrix}
    1 & -2 & 1 & \phantom{-}0 & 0 & \ldots & 0 \\
    \end{bmatrix}
    \;\;\;\;\;\;
    \begin{matrix}
    \hfill\text{ (2nd-order differencing)} \\
    \end{matrix} \\
    \zC &= 
    \begin{bmatrix}
    1 & -3 & 3 & -1 & 0 & \ldots & 0 \\
    \end{bmatrix}
    \;\;\;\;\;\;
    \begin{matrix}
    \hfill\text{ (3rd-order differencing)} \\
    \end{matrix}
\end{align}
 In this work, we only use the above 0, 1st, 2nd, or 3rd-order differencing instantiations. With multiple differencing SSMs in a multi-SSM \ourmethod{} layer, we initialize differencing SSMs by running through the orders repeatedly in sequence. For example, given five differencing SSMs, the first four SSMs perform 0, 1st, 2nd, and 3rd-order differencing respectively, while the fifth performs 0-order differencing again.

\header{Moving Average Residual (MA residual) SSM}
As another version of the preprocessing SSM, we can fix $\boldsymbol{a} = \boldsymbol{0}$, $\zB = \boldsymbol{e_1}$, and set $\zC$ such that the SSM outputs sample residuals from a moving average applied over the input sequence. For an $n$-order moving average, we compute outputs with $\zC$ specified as
\begin{align}
    \zC &= 
    \begin{bmatrix}
    1 - 1/n, & -1/n, & \ldots & -1/n, & 0 & \ldots & 0 \\
    \end{bmatrix}
    \;\;\;\;\;\;
    \begin{matrix}
    \hfill\text{($n$-order moving average residual)} \\
    \end{matrix}
\end{align}
For each MA residual SSM, we randomly initialize the order by uniform-randomly sampling an integer in the range $[4, d]$, where $d$ is again the state-space dimension size (recall $\zC \in \mathbb{R}^{1 \times d}$). We pick $4$ as a heuristic which was not finetuned; we leave additional optimization here for further work.

\subsubsection{Task-specific \ourmethod{} Architectures}\label{appendix:specific_spacetime_architectures}

Here we provide layer-level details on the \ourmethod{} networks used in this work. For each task, we describe number of layers, number of SSMs per layer, state-space dimension (fixed for all SSMs in a network), and which SSMs are used in each layer. 

Expanding on this last detail, as previously discussed in Section~\ref{sec:method_spacetime_layer}, in each \ourmethod{} layer we can specify multiple SSMs in each layer, computing their outputs in parallel to produce a multidimensional output that is fed as the input to the next \ourmethod{} layer. The ``types'' of SSMs do not all have to be the same per layer, and we list the type (companion, shift, differencing, MA residual) and closed-loop designation (standard, closed-loop) of the SSMs in each layer below.

For an additional visual overview of a \ourmethod{} network, please refer back to Figure~\ref{fig:arch_overview}.

\header{Forecasting: Informer and Monash}
We describe the architecture in Table~\ref{tab:spacetime_forecasting_arch}. We treat the first \ourmethod{} layer as ``preprocessing'' layer, which performs differencing and moving average residual operations on the input sequence. We treat the last \ourmethod{} layer as a ``forecasting'' layer, which autoregressively outputs future horizon predictions given the second-to-last layer's outputs as an input sequence.

% \[
% \begin{aligned}
%     \begin{matrix}
%     \text{\phantom{-}} \\
%     \end{matrix}
%     \\
%     &
%     \begin{bmatrix}
%     \text{\phantom{-} Differencing \phantom{-}}   \\ \text{(standard)} \\
%     \end{bmatrix}
%     \times 192 
%     \\
%     \begin{matrix}
%     \text{\phantom{-}} \\
%     \end{matrix}
%     \\
%     &
%     &
%     \begin{bmatrix}
%     \text{\phantom{-} MA Residual\phantom{-} }   \\ \text{(standard)} \\
%     \end{bmatrix}
%      \times 64 
%      \\
%      \begin{matrix}
%     \text{\phantom{-}} \\
%     \end{matrix}
%     \\
% \end{aligned}
% \]

% \[
% \begin{aligned}
%     \begin{matrix}
%     \text{\phantom{-}} \\
%     \end{matrix}
%     \\
%     &
%     \begin{bmatrix}
%     \text{\phantom{-} Differencing \phantom{-}}   \\ \text{(standard)} \\
%     \end{bmatrix}
%     \times 256 
%     \\
%      \begin{matrix}
%     \text{\phantom{-}} \\
%     \end{matrix}
%     \\
% \end{aligned}
% \]


% \[
% \begin{aligned}
%     \begin{matrix}
%     \text{\phantom{-}} \\
%     \end{matrix}
%     \\
%     &
%     \begin{bmatrix}
%     \text{\phantom{-} Differencing \phantom{-}}   \\ \text{(standard)} \\
%     \end{bmatrix}
%     \times 192 
%     \\
%      \begin{matrix}
%     \text{\phantom{-}} \\
%     \end{matrix}
%     \\
% \end{aligned}
% \]

% \[
% \begin{aligned}
%     \begin{matrix}
%     \text{\phantom{-}} \\
%     \end{matrix}
%     \\
%     \begin{bmatrix}
%     \text{\phantom{-} Companion \phantom{-}}   \\ \text{(standard)} \\
%     \end{bmatrix}
%     \times 256 
%     \\
%     \begin{matrix}
%     \text{\phantom{-}} \\
%     \end{matrix}
%     \\
% \end{aligned}
% \]


% \[
% \begin{aligned}
%     \begin{matrix}
%     \text{\phantom{-}} \\
%     \end{matrix}
%     \\
%     \begin{bmatrix}
%     \text{\phantom{-} Shift \phantom{-}}   \\ \text{(standard)} \\
%     \end{bmatrix}
%     \times 256 
%     \\
%     \begin{matrix}
%     \text{\phantom{-}} \\
%     \end{matrix}
%     \\
% \end{aligned}
% \]


% \[
% \begin{aligned}
%     \begin{matrix}
%     \text{\phantom{-}} \\
%     \end{matrix}
%     \\
%     &
%     \begin{bmatrix}
%     \text{\phantom{-} Companion \phantom{-}}   \\ \text{(standard)} \\
%     \end{bmatrix}
%     \times 128 
%     \\
%     \begin{matrix}
%     \text{\phantom{-}} \\
%     \end{matrix}
%     \\
%     &
%     &
%     \begin{bmatrix}
%     \text{\phantom{-} Shift \phantom{-} }   \\ \text{(standard)} \\
%     \end{bmatrix}
%      \times 128  
%      \\
%      \begin{matrix}
%     \text{\phantom{-}} \\
%     \end{matrix}
%     \\
% \end{aligned}
% \]

% \[
% \begin{aligned}
%     \begin{matrix}
%     \text{\phantom{-}} \\
%     \end{matrix}
%     \\
%     \begin{bmatrix}
%     \text{\phantom{-} Companion \phantom{-} }   \\ \text{(closed-loop)} \\
%     \end{bmatrix}
%     \times 128 
%     \\
%     \begin{matrix}
%     \text{\phantom{-}} \\
%     \end{matrix}
%     \\
% \end{aligned}
% \]


\header{Classification: ECG}
We describe the architectures for each ECG classification task in Tables~\ref{tab:spacetime_ecg_superdiag}--\ref{tab:spacetime_ecg_all}. For all models, we use state-space dimension $d = 64$. As described in the experiments, for classification we compute logits with a mean pooling over the output sequence, where pooling is computed over the sequence length. 

\header{Classification: Speech Audio}
We describe the architecture for the Speech Audio task in Table~\ref{tab:spacetime_speech}. We use state-space dimension $d = 1024$. As described in the experiments, for classification we compute logits with a mean pooling over the output sequence, where pooling is computed over the sequence length. 


\begin{table}[]
\centering
\caption{\ourmethod{} forecasting architecture. For all SSMs, we keep state-space dimension $d = 128$. Repeated Identity denotes repeating the input to match the number of SSMs in the next layer, \ie{} 128 SSMs in this case. For each forecasting task, $d'$ denotes time series samples' number of features, $\ell$ denotes the lag size (number of past samples given as input), and $h$ denotes the horizon size (number of future samples to be predicted).}
\label{tab:spacetime_forecasting_arch}
\begin{tabular}{@{}c|c|c|c@{}}
Layer                             & Details                                                                                                                                                                                                                                                                                                                                                                                                                                                                                     & Input Size        & Output Size       \\ \midrule
\multicolumn{1}{c|}{Decoder}     & Linear                                                                                                                                                                                                                                                                                                                                                                                                                                                                                     & $128 \times \ell$ &  $d' \times h$   \\ \midrule
\multicolumn{1}{c|}{SSM Layer 3} & \begin{math}\begin{aligned}    \begin{matrix}    \text{\phantom{-}} \\    \end{matrix}    \\    \begin{bmatrix}    \text{\phantom{-} Companion \phantom{-} } \\ \text{(closed-loop)} \\    \end{bmatrix}    \times 128     \\    \begin{matrix}    \text{\phantom{-}} \\    \end{matrix}    \\\end{aligned}\end{math}                                                                                                                                                                        & $128 \times \ell$ & $128 \times \ell$ \\ \midrule
\multicolumn{1}{c|}{SSM Layer 2} & \begin{math}\begin{aligned}    \begin{matrix}    \text{\phantom{-}} \\    \end{matrix}    \\    \begin{bmatrix}    \text{\phantom{-} Companion \phantom{-}}  \\ \text{(standard)} \\    \end{bmatrix}    \times 128     \\    \begin{matrix}    \text{\phantom{-}} \\    \end{matrix}    \\\end{aligned}\end{math}                                                                                                                                                                             & $128 \times \ell$ & $128 \times \ell$ \\ \midrule
\multicolumn{1}{c|}{\text{SSM Layer 1}} & \begin{math}\begin{aligned}    \begin{matrix}    \text{\phantom{-}} \\    \end{matrix}    \\    &    \begin{bmatrix}    \text{\phantom{-} Differencing \phantom{-}}  \\ \text{(standard)} \\    \end{bmatrix}    \times 64     \\    \begin{matrix}    \text{\phantom{-}} \\    \end{matrix}    \\    &    \begin{bmatrix}    \text{\phantom{-} MA Residual\phantom{-} }  \\ \text{(standard)} \\    \end{bmatrix}     \times 64      \\     \begin{matrix}    \text{\phantom{-}} \\    \end{matrix}    \\\end{aligned}\end{math} & $128 \times \ell$ & $128 \times \ell$ \\ \midrule
\multicolumn{1}{c|}{Encoder}     & Repeated Identity                                                                                                                                                                                                                                                                                                                                                                                                                                                                                   & $d' \times \ell$   & $128 \times \ell$ \\ \bottomrule
\end{tabular}
\end{table}


\begin{table}[]
\centering
\caption{\ourmethod{} architecture for ECG SuperDiagnostic classification. For all SSMs, we keep state-space dimension $d = 64$. Input samples have $d' = 12$ features and are length $\ell = 1000$ time-steps long. The number of classes $c = 5$.}
\label{tab:spacetime_ecg_superdiag}
\begin{tabular}{@{}c|c|c|c@{}}
Layer       & Details                                                                                                                                                                                                                                                                                                                 & Input Size        & Output Size       \\ \midrule
Classifier  & Mean Pooling                                                                                                                                                                                                                                                                                                            & $c \times \ell$   & $c \times 1$      \\ \midrule
Decoder     & Linear                                                                                                                                                                                                                                                                                                                  & $256 \times \ell$ & $c \times \ell$   \\ \midrule
SSM Layer 5 & \begin{math}\begin{aligned}    \begin{matrix}    \text{\phantom{-}} \\    \end{matrix}    \\    \begin{bmatrix}    \text{\phantom{-} Companion \phantom{-}}  \\ \text{(standard)} \\    \end{bmatrix}    \times 256     \\    \begin{matrix}    \text{\phantom{-}} \\    \end{matrix}    \\\end{aligned}\end{math}          & $256 \times \ell$ & $256 \times \ell$ \\ \midrule
SSM Layer 4 & \begin{math}\begin{aligned}    \begin{matrix}    \text{\phantom{-}} \\    \end{matrix}    \\    \begin{bmatrix}    \text{\phantom{-} Companion \phantom{-}} \\ \text{(standard)} \\    \end{bmatrix}    \times 256     \\    \begin{matrix}    \text{\phantom{-}} \\    \end{matrix}    \\\end{aligned}\end{math}          & $256 \times \ell$ & $256 \times \ell$ \\ \midrule
SSM Layer 3 & \begin{math}\begin{aligned}    \begin{matrix}    \text{\phantom{-}} \\    \end{matrix}    \\    \begin{bmatrix}    \text{\phantom{-} Companion \phantom{-}} \\ \text{(standard)} \\    \end{bmatrix}    \times 256     \\    \begin{matrix}    \text{\phantom{-}} \\    \end{matrix}    \\\end{aligned}\end{math}          & $256 \times \ell$ & $256 \times \ell$ \\ \midrule
SSM Layer 2 & \begin{math}\begin{aligned}    \begin{matrix}    \text{\phantom{-}} \\    \end{matrix}    \\    \begin{bmatrix}    \text{\phantom{-} Companion \phantom{-}} \\ \text{(standard)} \\    \end{bmatrix}    \times 256     \\    \begin{matrix}    \text{\phantom{-}} \\    \end{matrix}    \\\end{aligned}\end{math}          & $256 \times \ell$ & $256 \times \ell$ \\ \midrule
SSM Layer 1 & \begin{math}\begin{aligned}    \begin{matrix}    \text{\phantom{-}} \\    \end{matrix}    \\    &    \begin{bmatrix}    \text{\phantom{-} Differencing \phantom{-}} \\ \text{(standard)} \\    \end{bmatrix}    \times 256     \\     \begin{matrix}    \text{\phantom{-}} \\    \end{matrix}    \\\end{aligned}\end{math} & 256 $\times \ell$ & 256 $\times \ell$ \\ \midrule
Encoder     & Linear & $d' \times \ell$  & $256 \times \ell$ \\ \bottomrule
\end{tabular}
\end{table}

\begin{table}[]
\centering
\caption{\ourmethod{} architecture for ECG SubDiagnostic classification. For all SSMs, we keep state-space dimension $d = 64$. Input samples have $d' = 12$ features and are length $\ell = 1000$ time-steps long. The number of classes $c = 23$.}
\label{tab:spacetime_ecg_subdiag}
\begin{tabular}{@{}c|c|c|c@{}}
Layer       & Details                                                                                                                                                                                                                                                                                                                 & Input Size        & Output Size       \\ \midrule
Classifier  & Mean Pooling                                                                                                                                                                                                                                                                                                            & $c \times \ell$   & $c \times 1$      \\ \midrule
Decoder     & Linear                                                                                                                                                                                                                                                                                                                  & $256 \times \ell$ & $c \times \ell$   \\ \midrule
SSM Layer 5 & \begin{math}\begin{aligned}    \begin{matrix}    \text{\phantom{-}} \\    \end{matrix}    \\    \begin{bmatrix}    \text{\phantom{-} Shift \phantom{-}} \\ \text{(standard)} \\    \end{bmatrix}    \times 256     \\    \begin{matrix}    \text{\phantom{-}} \\    \end{matrix}    \\\end{aligned}\end{math}              & $256 \times \ell$ & $256 \times \ell$ \\ \midrule
SSM Layer 4 & \begin{math}\begin{aligned}    \begin{matrix}    \text{\phantom{-}} \\    \end{matrix}    \\    \begin{bmatrix}    \text{\phantom{-} Shift \phantom{-}}  \\ \text{(standard)} \\    \end{bmatrix}    \times 256     \\    \begin{matrix}    \text{\phantom{-}} \\    \end{matrix}    \\\end{aligned}\end{math}              & $256 \times \ell$ & $256 \times \ell$ \\ \midrule
SSM Layer 3 & \begin{math}\begin{aligned}    \begin{matrix}    \text{\phantom{-}} \\    \end{matrix}    \\    \begin{bmatrix}    \text{\phantom{-} Shift \phantom{-}} \\ \text{(standard)} \\    \end{bmatrix}    \times 256     \\    \begin{matrix}    \text{\phantom{-}} \\    \end{matrix}    \\\end{aligned}\end{math}              & $256 \times \ell$ & $256 \times \ell$ \\ \midrule
SSM Layer 2 & \begin{math}\begin{aligned}    \begin{matrix}    \text{\phantom{-}} \\    \end{matrix}    \\    \begin{bmatrix}    \text{\phantom{-} Shift \phantom{-}} \\ \text{(standard)} \\    \end{bmatrix}    \times 256     \\    \begin{matrix}    \text{\phantom{-}} \\    \end{matrix}    \\\end{aligned}\end{math}              & $256 \times \ell$ & $256 \times \ell$ \\ \midrule
SSM Layer 1 & \begin{math}\begin{aligned}    \begin{matrix}    \text{\phantom{-}} \\    \end{matrix}    \\    &    \begin{bmatrix}    \text{\phantom{-} Differencing \phantom{-}} \\ \text{(standard)} \\    \end{bmatrix}    \times 256     \\     \begin{matrix}    \text{\phantom{-}} \\    \end{matrix}    \\\end{aligned}\end{math} & $256 \times \ell$ & $256 \times \ell$ \\ \midrule
Encoder     & Linear                                                                                                                                                                                                                                                                                                                  & $d' \times \ell$  & $256 \times \ell$ \\ \bottomrule
\end{tabular}
\end{table}

\begin{table}[]
\centering
\caption{\ourmethod{} architecture for ECG Diagnostic classification. For all SSMs, we keep state-space dimension $d = 64$. Input samples have $d' = 12$ features and are length $\ell = 1000$ time-steps long. The number of classes $c = 44$.}
\label{tab:spacetime_ecg_diag}
\begin{tabular}{@{}c|c|c|c@{}}
Layer       & Details                                                                                                                                                                                                                                                                                                                 & Input Size        & Output Size       \\ \midrule
Classifier  & Mean Pooling                                                                                                                                                                                                                                                                                                            & $c \times \ell$   & $c \times 1$      \\ \midrule
Decoder     & Linear                                                                                                                                                                                                                                                                                                                  & $256 \times \ell$ & $c \times \ell$   \\ \midrule
SSM Layer 5 & \begin{math}\begin{aligned}    \begin{matrix}    \text{\phantom{-}} \\    \end{matrix}    \\    \begin{bmatrix}    \text{\phantom{-} Shift \phantom{-}} \\ \text{(standard)} \\    \end{bmatrix}    \times 256     \\    \begin{matrix}    \text{\phantom{-}} \\    \end{matrix}    \\\end{aligned}\end{math}              & $256 \times \ell$ & $256 \times \ell$ \\ \midrule
SSM Layer 4 & \begin{math}\begin{aligned}    \begin{matrix}    \text{\phantom{-}} \\    \end{matrix}    \\    \begin{bmatrix}    \text{\phantom{-} Shift \phantom{-}}  \\ \text{(standard)} \\    \end{bmatrix}    \times 256     \\    \begin{matrix}    \text{\phantom{-}} \\    \end{matrix}    \\\end{aligned}\end{math}              & $256 \times \ell$ & $256 \times \ell$ \\ \midrule
SSM Layer 3 & \begin{math}\begin{aligned}    \begin{matrix}    \text{\phantom{-}} \\    \end{matrix}    \\    \begin{bmatrix}    \text{\phantom{-} Shift \phantom{-}}  \\ \text{(standard)} \\    \end{bmatrix}    \times 256     \\    \begin{matrix}    \text{\phantom{-}} \\    \end{matrix}    \\\end{aligned}\end{math}              & $256 \times \ell$ & $256 \times \ell$ \\ \midrule
SSM Layer 2 & \begin{math}\begin{aligned}    \begin{matrix}    \text{\phantom{-}} \\    \end{matrix}    \\    \begin{bmatrix}    \text{\phantom{-} Shift \phantom{-}} \\ \text{(standard)} \\    \end{bmatrix}    \times 256     \\    \begin{matrix}    \text{\phantom{-}} \\    \end{matrix}    \\\end{aligned}\end{math}              & $256 \times \ell$ & $256 \times \ell$ \\ \midrule
SSM Layer 1 & \begin{math}\begin{aligned}    \begin{matrix}    \text{\phantom{-}} \\    \end{matrix}    \\    &    \begin{bmatrix}    \text{\phantom{-} Differencing \phantom{-}} \\ \text{(standard)} \\    \end{bmatrix}    \times 256     \\     \begin{matrix}    \text{\phantom{-}} \\    \end{matrix}    \\\end{aligned}\end{math} & $256 \times \ell$ & $256 \times \ell$ \\ \midrule
Encoder     & Linear                                                                                                                                                                                                                                                                                                                  & $d' \times \ell$  & $256 \times \ell$ \\ \bottomrule
\end{tabular}
\end{table}


\begin{table}[]
\centering
\caption{\ourmethod{} architecture for ECG Form classification. For all SSMs, we keep state-space dimension $d = 64$. Input samples have $d' = 12$ features and are length $\ell = 1000$ time-steps long. The number of classes $c = 19$.}
\label{tab:spacetime_ecg_form}
\begin{tabular}{@{}c|c|c|c@{}}
Layer       & Details                                                                                                                                                                                                                                                                                                                                                                                                                                                                                                                                      & Input Size        & Output Size       \\ \midrule
Classifier  & Mean Pooling                                                                                                                                                                                                                                                                                                                                                                                                                                                                                                                                 & $c \times \ell$   & $c \times 1$      \\ \midrule
Decoder     & Linear                                                                                                                                                                                                                                                                                                                                                                                                                                                                                                                                       & $256 \times \ell$ & $c \times \ell$   \\ \midrule
SSM Layer 5 & \begin{math}\begin{aligned}    \begin{matrix}    \text{\phantom{-}} \\    \end{matrix}    \\    \begin{bmatrix}    \text{\phantom{-} Companion \phantom{-}}  \\ \text{(standard)} \\    \end{bmatrix}    \times 256     \\    \begin{matrix}    \text{\phantom{-}} \\    \end{matrix}    \\\end{aligned}\end{math}                                                                                                                                                                                                                               & $256 \times \ell$ & $256 \times \ell$ \\ \midrule
SSM Layer 4 & \begin{math}\begin{aligned}    \begin{matrix}    \text{\phantom{-}} \\    \end{matrix}    \\    \begin{bmatrix}    \text{\phantom{-} Companion \phantom{-}}   \\ \text{(standard)} \\    \end{bmatrix}    \times 256     \\    \begin{matrix}    \text{\phantom{-}} \\    \end{matrix}    \\\end{aligned}\end{math}                                                                                                                                                                                                                               & $256 \times \ell$ & $256 \times \ell$ \\ \midrule
SSM Layer 3 & \begin{math}\begin{aligned}    \begin{matrix}    \text{\phantom{-}} \\    \end{matrix}    \\    \begin{bmatrix}    \text{\phantom{-} Companion \phantom{-}}   \\ \text{(standard)} \\    \end{bmatrix}    \times 256     \\    \begin{matrix}    \text{\phantom{-}} \\    \end{matrix}    \\\end{aligned}\end{math}                                                                                                                                                                                                                               & $256 \times \ell$ & $256 \times \ell$ \\ \midrule
SSM Layer 2 & \begin{math}\begin{aligned}    \begin{matrix}    \text{\phantom{-}} \\    \end{matrix}    \\    \begin{bmatrix}    \text{\phantom{-} Companion \phantom{-}}   \\ \text{(standard)} \\    \end{bmatrix}    \times 256     \\    \begin{matrix}    \text{\phantom{-}} \\    \end{matrix}    \\\end{aligned}\end{math}                                                                                                                                                                                                                               & $256 \times \ell$ & $256 \times \ell$ \\ \midrule
SSM Layer 1 & \begin{math}\begin{aligned}    \begin{matrix}    \text{\phantom{-}} \\    \end{matrix}    \\    &    \begin{bmatrix}    \text{\phantom{-} Differencing \phantom{-}}   \\ \text{(standard)} \\    \end{bmatrix}    \times 192     \\    \begin{matrix}    \text{\phantom{-}} \\    \end{matrix}    \\    &    \begin{bmatrix}    \text{\phantom{-} MA Residual\phantom{-} }   \\ \text{(standard)} \\    \end{bmatrix}     \times 64      \\     \begin{matrix}    \text{\phantom{-}} \\    \end{matrix}    \\\end{aligned}\end{math} & $256 \times \ell$ & $256 \times \ell$ \\ \midrule
Encoder     & Linear                                                                                                                                                                                                                                                                                                                                                                                                                                                                                                                                       & $d' \times \ell$  & $256 \times \ell$ \\ \bottomrule
\end{tabular}
\end{table}



\begin{table}[]
\centering
\caption{\ourmethod{} architecture for ECG Rhythm classification. For all SSMs, we keep state-space dimension $d = 64$. Input samples have $d' = 12$ features and are length $\ell = 1000$ time-steps long. The number of classes $c = 12$.}
\label{tab:spacetime_ecg_rhythm}
\begin{tabular}{@{}c|c|c|c@{}}
Layer       & Details                                                                                                                                                                                                                                                                                                                                                                                                                                                                                                                                & Input Size        & Output Size       \\ \midrule
Classifier  & Mean Pooling                                                                                                                                                                                                                                                                                                                                                                                                                                                                                                                           & $c \times \ell$   & $c \times 1$      \\ \midrule
Decoder     & Linear                                                                                                                                                                                                                                                                                                                                                                                                                                                                                                                                 & $256 \times \ell$ & $c \times \ell$   \\ \midrule
SSM Layer 5 & \begin{math}\begin{aligned}    \begin{matrix}    \text{\phantom{-}} \\    \end{matrix}    \\    &    \begin{bmatrix}    \text{\phantom{-} Companion \phantom{-}}   \\ \text{(standard)} \\    \end{bmatrix}    \times 128     \\    \begin{matrix}    \text{\phantom{-}} \\    \end{matrix}    \\    &    \begin{bmatrix}    \text{\phantom{-} Shift \phantom{-} }   \\ \text{(standard)} \\    \end{bmatrix}     \times 128       \\     \begin{matrix}    \text{\phantom{-}} \\    \end{matrix}    \\\end{aligned}\end{math} & $256 \times \ell$ & $256 \times \ell$ \\ \midrule
SSM Layer 4 & \begin{math}\begin{aligned}    \begin{matrix}    \text{\phantom{-}} \\    \end{matrix}    \\    &    \begin{bmatrix}    \text{\phantom{-} Companion \phantom{-}}   \\ \text{(standard)} \\    \end{bmatrix}    \times 128     \\    \begin{matrix}    \text{\phantom{-}} \\    \end{matrix}    \\    &    \begin{bmatrix}    \text{\phantom{-} Shift \phantom{-} }   \\ \text{(standard)} \\    \end{bmatrix}     \times 128       \\     \begin{matrix}    \text{\phantom{-}} \\    \end{matrix}    \\\end{aligned}\end{math} & $256 \times \ell$ & $256 \times \ell$ \\ \midrule
SSM Layer 3 & \begin{math}\begin{aligned}    \begin{matrix}    \text{\phantom{-}} \\    \end{matrix}    \\    &    \begin{bmatrix}    \text{\phantom{-} Companion \phantom{-}}   \\ \text{(standard)} \\    \end{bmatrix}    \times 128     \\    \begin{matrix}    \text{\phantom{-}} \\    \end{matrix}    \\    &    \begin{bmatrix}    \text{\phantom{-} Shift \phantom{-} }   \\ \text{(standard)} \\    \end{bmatrix}     \times 128       \\     \begin{matrix}    \text{\phantom{-}} \\    \end{matrix}    \\\end{aligned}\end{math} & $256 \times \ell$ & $256 \times \ell$ \\ \midrule
SSM Layer 2 & \begin{math}\begin{aligned}    \begin{matrix}    \text{\phantom{-}} \\    \end{matrix}    \\    &    \begin{bmatrix}    \text{\phantom{-} Companion \phantom{-}}   \\ \text{(standard)} \\    \end{bmatrix}    \times 128     \\    \begin{matrix}    \text{\phantom{-}} \\    \end{matrix}    \\    &    \begin{bmatrix}    \text{\phantom{-} Shift \phantom{-} }   \\ \text{(standard)} \\    \end{bmatrix}     \times 128       \\     \begin{matrix}    \text{\phantom{-}} \\    \end{matrix}    \\\end{aligned}\end{math} & $256 \times \ell$ & $256 \times \ell$ \\ \midrule
SSM Layer 1 & \begin{math}\begin{aligned}    \begin{matrix}    \text{\phantom{-}} \\    \end{matrix}    \\    &    \begin{bmatrix}    \text{\phantom{-} Differencing \phantom{-}}   \\ \text{(standard)} \\    \end{bmatrix}    \times 256     \\     \begin{matrix}    \text{\phantom{-}} \\    \end{matrix}    \\\end{aligned}\end{math}                                                                                                                                                                                                                & $256 \times \ell$ & $256 \times \ell$ \\ \midrule
Encoder     & Linear                                                                                                                                                                                                                                                                                                                                                                                                                                                                                                                                 & $d' \times \ell$  & $256 \times \ell$ \\ \bottomrule
\end{tabular}
\end{table}


\begin{table}[]
\centering
\caption{\ourmethod{} architecture for ECG All classification. For all SSMs, we keep state-space dimension $d = 64$. Input samples have $d' = 12$ features and are length $\ell = 1000$ time-steps long. The number of classes $c = 71$.}
\label{tab:spacetime_ecg_all}
\begin{tabular}{@{}c|c|c|c@{}}
Layer       & Details                                                                                                                                                                                                                                                                                                                                                                                                                                                                                                                                      & Input Size        & Output Size       \\ \midrule
Classifier  & Mean Pooling                                                                                                                                                                                                                                                                                                                                                                                                                                                                                                                                 & $c \times \ell$   & $c \times 1$      \\ \midrule
Decoder     & Linear                                                                                                                                                                                                                                                                                                                                                                                                                                                                                                                                       & $256 \times \ell$ & $c \times \ell$   \\ \midrule
SSM Layer 5 & \begin{math}\begin{aligned} \begin{matrix} \text{\phantom{-}} \\ \end{matrix} \\ \begin{bmatrix} \text{\phantom{-} Shift \phantom{-}}   \\ \text{(standard)} \\ \end{bmatrix} \times 256 \\ \begin{matrix} \text{\phantom{-}} \\ \end{matrix} \\\end{aligned}\end{math}                                                                                                                                                                                                                                                                           & $256 \times \ell$ & $256 \times \ell$ \\ \midrule
SSM Layer 4 & \begin{math}\begin{aligned} \begin{matrix} \text{\phantom{-}} \\ \end{matrix} \\ \begin{bmatrix} \text{\phantom{-} Shift \phantom{-}}   \\ \text{(standard)} \\ \end{bmatrix} \times 256 \\ \begin{matrix} \text{\phantom{-}} \\ \end{matrix} \\\end{aligned}\end{math}                                                                                                                                                                                                                                                                           & $256 \times \ell$ & $256 \times \ell$ \\ \midrule
SSM Layer 3 & \begin{math}\begin{aligned} \begin{matrix} \text{\phantom{-}} \\ \end{matrix} \\ \begin{bmatrix} \text{\phantom{-} Shift \phantom{-}}   \\ \text{(standard)} \\ \end{bmatrix} \times 256 \\ \begin{matrix} \text{\phantom{-}} \\ \end{matrix} \\\end{aligned}\end{math}                                                                                                                                                                                                                                                                           & $256 \times \ell$ & $256 \times \ell$ \\ \midrule
SSM Layer 2 & \begin{math}\begin{aligned}    \begin{matrix}    \text{\phantom{-}} \\    \end{matrix}    \\    \begin{bmatrix}    \text{\phantom{-} Shift \phantom{-}}   \\ \text{(standard)} \\    \end{bmatrix}    \times 256     \\    \begin{matrix}    \text{\phantom{-}} \\    \end{matrix}    \\\end{aligned}\end{math}                                                                                                                                                                                                                                   & $256 \times \ell$ & $256 \times \ell$ \\ \midrule
SSM Layer 1 & \begin{math}\begin{aligned}    \begin{matrix}    \text{\phantom{-}} \\    \end{matrix}    \\    &    \begin{bmatrix}    \text{\phantom{-} Differencing \phantom{-}}   \\ \text{(standard)} \\    \end{bmatrix}    \times 192     \\    \begin{matrix}    \text{\phantom{-}} \\    \end{matrix}    \\    &    \begin{bmatrix}    \text{\phantom{-} MA Residual\phantom{-} }   \\ \text{(standard)} \\    \end{bmatrix}     \times 64      \\     \begin{matrix}    \text{\phantom{-}} \\    \end{matrix}    \\\end{aligned}\end{math} & $256 \times \ell$ & $256 \times \ell$ \\ \midrule
Encoder     & Linear                                                                                                                                                                                                                                                                                                                                                                                                                                                                                                                                       & $d' \times \ell$  & $256 \times \ell$ \\ \bottomrule
\end{tabular}
\end{table}


\begin{table}[]
\centering
\caption{\ourmethod{} architecture for Speech Audio classification. For all SSMs, we keep state-space dimension $d = 1024$. Input samples have $d' = 1$ features and are length $\ell = 16000$ time-steps long. The number of classes $c = 10$.}
\label{tab:spacetime_speech}
\begin{tabular}{@{}c|c|c|c@{}}
Layer       & Details                                                                                                                                                                                                                                                                                                        & Input Size                              & Output Size                             \\ \midrule
Classifier  & Mean Pooling                                                                                                                                                                                                                                                                                                   & \cellcolor[HTML]{FFFFFF}$c \times \ell$ & $c \times 1$                            \\ \midrule
Decoder     & Linear                                                                                                                                                                                                                                                                                                         & $256 \times \ell$                       & \cellcolor[HTML]{FFFFFF}$c \times \ell$ \\ \midrule
SSM Layer 6 & \begin{math}\begin{aligned}    \begin{matrix}    \text{\phantom{-}} \\    \end{matrix}    \\    \begin{bmatrix}    \text{\phantom{-} Companion \phantom{-}}   \\ \text{(standard)} \\    \end{bmatrix}    \times 256     \\    \begin{matrix}    \text{\phantom{-}} \\    \end{matrix}    \\\end{aligned}\end{math} & $256 \times \ell$                       & $256 \times \ell$                       \\ \midrule
SSM Layer 5 & \begin{math}\begin{aligned}    \begin{matrix}    \text{\phantom{-}} \\    \end{matrix}    \\    \begin{bmatrix}    \text{\phantom{-} Companion \phantom{-}}   \\ \text{(standard)} \\    \end{bmatrix}    \times 256     \\    \begin{matrix}    \text{\phantom{-}} \\    \end{matrix}    \\\end{aligned}\end{math} & $256 \times \ell$                       & $256 \times \ell$                       \\ \midrule
SSM Layer 4 & \begin{math}\begin{aligned}    \begin{matrix}    \text{\phantom{-}} \\    \end{matrix}    \\    \begin{bmatrix}    \text{\phantom{-} Companion \phantom{-}}   \\ \text{(standard)} \\    \end{bmatrix}    \times 256     \\    \begin{matrix}    \text{\phantom{-}} \\    \end{matrix}    \\\end{aligned}\end{math} & $256 \times \ell$                       & $256 \times \ell$                       \\ \midrule
SSM Layer 3 & \begin{math}\begin{aligned}    \begin{matrix}    \text{\phantom{-}} \\    \end{matrix}    \\    \begin{bmatrix}    \text{\phantom{-} Companion \phantom{-}}   \\ \text{(standard)} \\    \end{bmatrix}    \times 256     \\    \begin{matrix}    \text{\phantom{-}} \\    \end{matrix}    \\\end{aligned}\end{math} & $256 \times \ell$                       & $256 \times \ell$                       \\ \midrule
SSM Layer 2 & \begin{math}\begin{aligned}    \begin{matrix}    \text{\phantom{-}} \\    \end{matrix}    \\    \begin{bmatrix}    \text{\phantom{-} Companion \phantom{-}}   \\ \text{(standard)} \\    \end{bmatrix}    \times 256     \\    \begin{matrix}    \text{\phantom{-}} \\    \end{matrix}    \\\end{aligned}\end{math} & $256 \times \ell$                       & $256 \times \ell$                       \\ \midrule
SSM Layer 1 & \begin{math}\begin{aligned}    \begin{matrix}    \text{\phantom{-}} \\    \end{matrix}    \\    \begin{bmatrix}    \text{\phantom{-} Companion \phantom{-}}   \\ \text{(standard)} \\    \end{bmatrix}    \times 256     \\    \begin{matrix}    \text{\phantom{-}} \\    \end{matrix}    \\\end{aligned}\end{math} & $256 \times \ell$                       & $256 \times \ell$                       \\ \midrule
Encoder     & Linear                                                                                                                                                                                                                                                                                                         & $d' \times \ell$                        & $256 \times \ell$                       \\ \bottomrule
\end{tabular}
\end{table}




\begin{figure}[h]
    \centering
    \includegraphics[scale=0.35]{_ICLR2023_paper/figures/dsp_SpaceTime.pdf}
    \includegraphics[scale=0.35]{_ICLR2023_paper/figures/dsp_S4.pdf}
    \includegraphics[scale=0.35]{_ICLR2023_paper/figures/dsp_Conv1d.pdf}
    \includegraphics[scale=0.35]{_ICLR2023_paper/figures/dsp_LSTM.pdf}
    \includegraphics[scale=0.35]{_ICLR2023_paper/figures/dsp_NLinear.pdf}
    \includegraphics[scale=0.35]{_ICLR2023_paper/figures/dsp_Transformer.pdf}
    \caption{Testing the capability of different sequence--to--sequence models to approximate the input--output map of digital filters. In blue, we show the output signal filtered by each model. The ground--truth digital filter is a Butterworth of order $10$.}
    \label{fig:dsp_synthetic}
\end{figure}
%





\begin{sidewaystable}[h]
    \tiny
    \centering
    \caption{Monash forecasting. Test RMSE of \ourmethod for each dataset (best result selected via validation RMSE, average of $3$ runs).}
    \label{tab:monash}
    \begin{tabular}{c|c|cccccc|ccccc}
    \toprule
    %
    Dataset  & \ourmethod & SES & Theta & TBATS & ETS & (DHR-)ARIMA & PR & CatBoost & DeepAR & N-BEATS & WaveNet & Transformer \\
    %
    \midrule
    M1 Yearly & \underline{135508.3} & 193829.5 & 171458.1 & \textbf{116850.9} & 167739.0 & 175343.8 & 152038.7 & 237644.5 & 173075.1 & 192489.8 & 312821.8 & 182850.6 \\
    
    M1 Quarterly & \underline{2200.3} & 2545.7&	2282.7	&2673.9&	2408.5&	2538.5&	1909.3&	2161.0& 2313.3&	2267.3&	2271.7&	2231.5  \\    
    
    M1 Monthly & 2601.1 & 2725.8 & 2564.9 & 2594.5 & 2264.0 & 2450.6 & 2478.8 & 2461.7 & 2202.2 & \textbf{2183.4} & 2578.9 & 3129.8  \\

    M3 Yearly & 1412.4 & 1172.9&	1106.1&	1386.3	&1189.2	&1662.2	&1181.8&	1341.7&		1157.9&	1117.4&	1147.6&	1084.8 \\
    
    M3 Quarterly & 676.1 & 670.6&	567.7&	653.6&	598.7&	650.8&	605.5&	698.0&	606.6&	582.8&	606.8&	819.2\\ 
    
    M3 Monthly & 897.12 & 893.9	&754.0&	765.2	&755.3&	790.8&	830.0&	874.2&	873.7&	796.9&	845.3&	948.4 \\
    
    M3 Other & 265.56 & 309.7	&242.1&	217.0&	224.1	&220.8	&262.3&	349.9&	277.7&	248.5&	277.0	&271.0 \\
    
    M4 Quarterly & 718.2 & 732.8&	673.2&	672.7&	674.3&	710.0&	711.9&	714.2&	700.3&	684.7&	697.0&	739.1 \\
    
    M4 Monthly & 1092.2 & 755.5	&683.7&	743.4&	705.7&	702.1&	720.5&	734.8&	740.3&	705.2&	787.9&	902.4\\ 
    
    M4 Weekly & \textbf{348.3}& 412.6	&405.2&	356.7&	408.5	&386.3&	350.3&	420.8&	422.2&	330.8&	437.3&	456.9\\ 
    
    M4 Daily &\textbf{183.2}& 209.8&	210.4&	\underline{208.4}&	230.0	&212.6&	213.0&	263.1	&343.5&	221.7&	220.5&	233.6\\ 
    
    M4 Hourly & \textbf{255.2} & 1476.8	&1483.7	&469.9&	3830.4&	1563.1&	\underline{313.0} &	344.6&	1095.1&	501.2&	468.1&	391.2\\
    
    Tourism Yearly & \textbf{74799.2}& 106665.2&	99914.2&	105799.4&	104700.5&	106082.6&	89645.6	&87489.0&	78470.7&	78241.7	&77581.3&	80089.3\\
    
    Tourism Quarterly & 11608.32& 15000.0&	9254.6&	12001.5	&10812.3	&12564.8	&11746.9	&12788.0	&11762.0&	11306.0	&11546.6	&11724.1 \\
    
    Tourism Monthly & 3181.2& 7039.4&	2702.0	&3661.5	&2543.0	&3132.4&	2739.4&	3102.8&	2359.9&	2596.2&	2694.2&	2660.1\\
    
    Pedestrian & 69.6& 228.1&	228.2	&261.3&	278.3&	820.3&	61.8&	60.8&	65.8&	99.3&	68.0&	70.2\\
    
    Weather & \textbf{2.7}& 2.9	&3.3&	2.9&	3.0&	3.1	&9.1&	3.1&	\textbf{2.7}	&3.1&	3.0	&2.8\\
    
    NN5 Weekly & \textbf{16.9}& 18.8&	18.7&	18.5&	18.8&	18.6&	18.6&	18.7&	18.5&	17.4&	24.2&	24.0 \\
    
    Solar $10$ min &7.4& 7.2&	7.2	&10.7&	7.2	&5.6&	7.2	&8.7&	7.2	&6.6&	8.0&	7.2\\
    
    Solar Weekly &1423.7& 1331.3&	1341.6&	1049.0&	1264.4&	967.9&	1168.2&	1754.2&	873.6&	1307.8&	2569.3&	693.8\\
    
    Electricity Hourly & \textbf{475.1} & 1026.3&	1026.4&	743.4&	1524.9&	1082.4&	689.9&	582.7&	\underline{478.0} &	510.9&	489.9&	514.7\\
    
    Electricity Weekly &37802.2& 77067.9&	76935.6&	28039.7	&70369.0&	32594.8&	47802.1&	37289.7	&53100.3&	35576.8	&63916.9&	78894.7\\
    
    Fred-MD &3743.6& 3103.0&	3898.7&	2295.7&	2341.7	&3312.5&	9736.9&	2679.4&	4638.7&	2813.0&	2779.5&	5098.9\\
    
    Traffic Hourly &0.03& 0.04&	0.04	&0.05&	0.04	&0.04&	0.03	&0.03&	0.02&	0.02	&0.03&	0.02\\
    
    Traffic Weekly &\textbf{1.3}& 1.5&	1.5	&1.5&	1.5	&1.5&	1.5&	1.5	&1.5&	1.4	&1.6&	1.9\\
    
    Hospital &40.1& 26.6&	22.6&	21.3&	22.0	&23.7&	23.5&	23.5&	22.0&	24.2&	23.4&	40.5\\
    
    Covid &490.1& 403.4&	370.1&	113.0&	102.1&	100.5&	394.1&	607.9&	230.5&	186.5&	1135.4&	480.0\\
    
    Saugeen & \textbf{24.0} & 39.8	&39.8	&42.6&	50.4&	43.2&	47.7&	\underline{39.3} &	45.3&	48.9&	43.0&	49.1\\
    
    US Births &630.2& 1369.5&	735.5&	606.5&	607.2&	705.5&	732.1&	618.4&	684.0&	627.7&	768.8&	686.5\\
    
    Sunspot &3.1& 5.0	&5.0&	3.0	&5.0&	3.0	&4.0&	2.4	&1.1&	14.5&	0.7	&0.5\\
    
    Car Parts & 0.64 & 0.71&	0.65	&0.71&	0.71&	0.71&	\underline{0.58}&	0.71&	\textbf{0.50}&	1.0&	\underline{0.58}&	0.5\\
    
    Vehicle Trips &30.4& 36.5&	37.4&	25.7&	37.6&	35.0&	31.7&	27.3&	26.5&	33.6&	29.0&	33.0\\
    \bottomrule
    %
    \end{tabular}
\end{sidewaystable}

%
\end{document}