%!TEX root = main.tex
We propose a new point of view for regularizing deep neural networks by using 
the norm of a reproducing kernel Hilbert space (RKHS). Even though this norm cannot be computed,
it admits upper and lower approximations leading to various practical strategies. 
Specifically, this perspective (i) provides a common umbrella for many existing regularization principles, including spectral norm and gradient penalties,
or adversarial training, (ii) leads to new effective regularization penalties, and
(iii) suggests hybrid strategies combining lower and upper bounds to get better approximations of the RKHS norm.
We experimentally show this approach
to be effective when learning on small datasets, or to obtain
adversarially robust models.
