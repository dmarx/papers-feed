\documentclass{article}
\usepackage{iclr2025/iclr2025_conference}

\theoremstyle{plain}
\newtheorem{theorem}{Theorem}
\newtheorem{lemma}{Lemma}[section]
\newtheorem{proposition}[lemma]{Proposition}
\newtheorem{corollary}[lemma]{Corollary}
\theoremstyle{definition}
\newtheorem{definition}[lemma]{Definition}
\newtheorem{assumption}[lemma]{Assumption}
\theoremstyle{remark}
\newtheorem{remark}[lemma]{Remark}


\newcommand{\diag}{\mathrm{diag}}
\newcommand{\norm}[1]{\left\|{#1}\right\|} %
\newcommand{\rank}{\mathrm{rank}}
\newcommand{\B}{\mathcal{B}}
\newcommand{\M}{\mathcal{M}}
\newcommand{\BS}{\B^*}
\newcommand{\BBS}{\B\B^*}
\newcommand{\BSB}{\B^*\B}
\newcommand{\MMS}{\M\M^*}
\newcommand{\MSM}{\M^*\M}
\newcommand{\BF}{\mathcal{B}\mathcal{F}}
\newcommand{\BD}{\mathcal{B}\mathcal{D}}
\newcommand{\DB}{\mathcal{D}\mathcal{B}}
\newcommand{\ind}[1]{^{(#1)}}

\newcommand{\vzero}{\mathbf{0}}
\newcommand{\vone}{\mathbf{1}}
\newcommand{\va}{\mathbf{a}}
\newcommand{\vb}{\mathbf{b}}
\newcommand{\vc}{\mathbf{c}}
\newcommand{\vd}{\mathbf{d}}
\newcommand{\ve}{\mathbf{e}}
\newcommand{\vf}{\mathbf{f}}
\newcommand{\vg}{\mathbf{g}}
\newcommand{\vh}{\mathbf{h}}
\newcommand{\vi}{\mathbf{i}}
\newcommand{\vj}{\mathbf{j}}
\newcommand{\vk}{\mathbf{k}}
\newcommand{\vl}{\mathbf{l}}
\newcommand{\vm}{\mathbf{m}}
\newcommand{\vn}{\mathbf{n}}
\newcommand{\vo}{\mathbf{o}}
\newcommand{\vp}{\mathbf{p}}
\newcommand{\vq}{\mathbf{q}}
\newcommand{\vr}{\mathbf{r}}
\newcommand{\vs}{\mathbf{s}}
\newcommand{\vt}{\mathbf{t}}
\newcommand{\vu}{\mathbf{u}}
\newcommand{\vv}{\mathbf{v}}
\newcommand{\vw}{\mathbf{w}}
\newcommand{\vx}{\mathbf{x}}
\newcommand{\vy}{\mathbf{y}}
\newcommand{\vz}{\mathbf{z}}
\newcommand{\vA}{\mathbf{A}}
\newcommand{\vB}{\mathbf{B}}
\newcommand{\vC}{\mathbf{C}}
\newcommand{\vD}{\mathbf{D}}
\newcommand{\vE}{\mathbf{E}}
\newcommand{\vF}{\mathbf{F}}
\newcommand{\vG}{\mathbf{G}}
\newcommand{\vH}{\mathbf{H}}
\newcommand{\vI}{\mathbf{I}}
\newcommand{\vJ}{\mathbf{J}}
\newcommand{\vK}{\mathbf{K}}
\newcommand{\vL}{\mathbf{L}}
\newcommand{\vM}{\mathbf{M}}
\newcommand{\vN}{\mathbf{N}}
\newcommand{\vO}{\mathbf{O}}
\newcommand{\vP}{\mathbf{P}}
\newcommand{\vQ}{\mathbf{Q}}
\newcommand{\vR}{\mathbf{R}}
\newcommand{\vS}{\mathbf{S}}
\newcommand{\vT}{\mathbf{T}}
\newcommand{\vU}{\mathbf{U}}
\newcommand{\vV}{\mathbf{V}}
\newcommand{\vW}{\mathbf{W}}
\newcommand{\vX}{\mathbf{X}}
\newcommand{\vY}{\mathbf{Y}}
\newcommand{\vZ}{\mathbf{Z}}

\newcommand{\cA}{\mathcal{A}}
\newcommand{\cB}{\mathcal{B}}
\newcommand{\cC}{\mathcal{C}}
\newcommand{\cD}{\mathcal{D}}
\newcommand{\cF}{\mathcal{F}}
\newcommand{\cG}{\mathcal{G}}
\newcommand{\cH}{\mathcal{H}}
\newcommand{\cI}{\mathcal{I}}
\newcommand{\cK}{\mathcal{K}}
\newcommand{\cL}{\mathcal{L}}
\newcommand{\cM}{\mathcal{M}}
\newcommand{\cN}{\mathcal{N}}
\newcommand{\cP}{\mathcal{P}}
\newcommand{\cS}{\mathcal{S}}
\newcommand{\cT}{\mathcal{T}}
\newcommand{\cX}{\mathcal{X}}

\newcommand{\R}{\mathbb{R}}
\newcommand{\F}{\mathbb{F}}

\newcommand{\Z}{\mathbb{Z}}
\newcommand{\ep}{\epsilon}
\newcommand{\g}{\gamma}
\newcommand{\Y}{\infty}
\newcommand{\f}[2]{\dfrac{#1}{#2}}
\newcommand{\ff}[2]{\tfrac{#1}{#2}}
\newcommand{\lm}[2]{\lim_{#1\rightarrow #2}}
\newcommand{\de}{\delta}
\newcommand{\T}{\theta}
\newcommand{\tm}{\times}
\newcommand{\su}[2]{\mathlarger{\sum\limits_{#1}^{#2}}}
\newcommand{\pd}[2]{\mathlarger{\prod\limits_{#1}^{#2}}}
\renewcommand{\sec}[1]{\section*{#1}}
\newcommand{\st}[1]{\subsection*{#1}}
\newcommand{\sst}[1]{\subsubsection*{#1}}
\renewcommand{\b}{\textbf}
\newcommand{\lessim}{\lesssim}
\newcommand{\E}{\mathbb{E}}
\newcommand{\p}{\partial}
\newcommand{\lt}{\left(}
\newcommand{\rt}{\right)}
\newcommand{\Lt}{\left[}
\newcommand{\Rt}{\right]}
\newcommand{\A}{\alpha}
\renewcommand{\b}{\beta}
\newcommand{\I}[2]{\mathlarger{\int_{#1}^{#2}}}
\newcommand{\G}{\nabla}
\newcommand{\Om}{\Omega}
\newcommand{\y}{\tau}
\newcommand{\K}{\mathcal{K}}
\newcommand{\C}{\mathbb{C}}
\newcommand{\om}{\omega}
\newcommand{\D}{\Delta}
\newcommand{\N}{\mathcal{N}}
\newcommand{\ra}{\rightarrow}
\newcommand{\Ra}{\Rightarrow}
\newcommand{\floor}[1]{\left\lfloor#1\right\rfloor}
\newcommand{\ceil}[1]{\left\lceil#1\right\rceil}
\newcommand{\ip}[1]{\left\langle#1\right\rangle}
\renewcommand{\mod}{\text{ mod }}
\newcommand{\sign}{\text{sign}}
\newcommand{\defeq}{:=}
\renewcommand{\a}{\bar{a}}
\newcommand{\MM}{\widetilde{\vM}}

\DeclareMathOperator*{\argmin}{argmin}

% \usepackage{preprint}
\usepackage[utf8]{inputenc} % allow utf-8 input
\usepackage[T1]{fontenc}    % use 8-bit T1 fonts
\usepackage{url}            % simple URL typesetting
\usepackage{booktabs}       % professional-quality tables
\usepackage{amsfonts}       % blackboard math symbols
\usepackage{nicefrac}       % compact symbols for 1/2, etc.
\usepackage{microtype}      % microtypography
%\usepackage[dvipsnames]{xcolor}         % colors
\usepackage{colortbl}
\usepackage{amsthm}
\usepackage{thmtools}
\usepackage{bbm}
\usepackage{mathtools}
% \usepackage{algorithm}
% \usepackage{algorithmic}
\usepackage{multirow}
\usepackage{hhline}
% \usepackage[table,xcdraw]{xcolor}
% \usepackage{xcolor,colortbl}
% \definecolor{refcolor}{rgb}{0.2549, 0.4117, 0.8823}
% \usepackage[allcolors=refcolor,colorlinks=true]{hyperref}

\usepackage{caption}
\usepackage{subcaption}
\usepackage{wrapfig}
\usepackage{enumitem,kantlipsum}
\usepackage{tabularx}
\usepackage{amssymb}
\usepackage[textsize=meduim]{todonotes}

\iclrfinalcopy % uncomment this line for final submission


\newcommand{\methodfull}{\textbf{\underline{LaT}}ent \textbf{\underline{R}}easoning \textbf{\underline{O}}ptimization (\textbf{LaTRO})}

\newcommand{\zuxin}[1]{\textcolor{teal}{[zuxin: #1]}}

\newcommand{\yy}[1]{{\textcolor{blue}{(Yihao: #1)}}}

% \title{Unlocking Latent Reasoning Capabilities in Language Models via Self-Rewarding}
\title{Language Models are Hidden Reasoners:\\Unlocking Latent Reasoning Capabilities via Self-Rewarding}
\author{Haolin Chen, Yihao Feng, Zuxin Liu, Weiran Yao, Akshara Prabhakar, \\
\textbf{Shelby Heinecke, Ricky Ho, Phil Mui, Silvio Savarese, Caiming Xiong, Huan Wang\thanks{Corresponding author, \texttt{huan.wang@salesforce.com}.}} \\
Salesforce AI Research \\
}
\date{July 2024}

\begin{document}

\maketitle
\begin{abstract}
    Large language models (LLMs) have shown impressive capabilities, but still struggle with complex reasoning tasks requiring multiple steps. While prompt-based methods like Chain-of-Thought (CoT) can improve LLM reasoning at inference time, optimizing reasoning capabilities during training remains challenging. We introduce \methodfull, a principled framework that formulates reasoning as sampling from a latent distribution and optimizes it via variational approaches. LaTRO enables LLMs to concurrently improve both their reasoning process and ability to evaluate reasoning quality, without requiring external feedback or reward models. We validate LaTRO through experiments on GSM8K and ARC-Challenge datasets using multiple model architectures. On GSM8K, LaTRO improves zero-shot accuracy by an average of 12.5\% over base models and 9.6\% over supervised fine-tuning across Phi-3.5-mini, Mistral-7B, and Llama-3.1-8B. Our findings suggest that pre-trained LLMs possess latent reasoning capabilities that can be unlocked and enhanced through our proposed optimization approach in a self-improvement manner. The code of LaTRO is available at \url{https://github.com/SalesforceAIResearch/LaTRO}.
\end{abstract}

\section{Introduction}

The development of large language models (LLMs) with enhanced reasoning capabilities has emerged as a crucial area of research. Despite their impressive advances, the inherent next-token prediction mechanism of LLMs makes it challenging for these models to solve complex problems requiring multiple reasoning steps \citep{wang2022self, huang2023large}. For instance, LLMs often struggle to directly provide accurate solutions to mathematical problems or even simple puzzles like counting specific letters in a word.
Consequently, researchers have explored various prompting strategies that guide LLMs to generate reasoning trajectories or rationales—sequences of tokens that build a step-by-step progression toward an answer. Techniques such as Chain-of-Thought (CoT) \citep{DBLP:conf/nips/Wei0SBIXCLZ22}, Tree-of-Thought (ToT) \citep{yao2024tree}, and Program-of-Thought (PoT) \citep{DBLP:journals/tmlr/ChenM0C23} prompting methods exemplify this approach.

\begin{figure}[htbp]
    \begin{tikzpicture}[scale=1]
        \node[anchor=north west] (query) at (0,0) {
            \begin{tcolorbox}[colback=white, colframe=gray, width=2.5cm, arc=1mm, auto outer arc, fontupper=\color{black}\scriptsize, boxsep=0.5mm, left=0.5mm, right=0.5mm, top=0.5mm, bottom=0.5mm]
                \centering
                \textbf{Question} $\px$
            \end{tcolorbox}
        };
        \node[anchor=west] (model) at ([shift={(0.3, 0)}]query.east){
            \begin{tcolorbox}[colback=blue!5!white, colframe=main, width=1.5cm, height=1cm, arc=1mm, valign=center, auto outer arc, fontupper=\color{black}\scriptsize, boxsep=0.5mm, left=0.5mm, right=0.5mm, top=0.5mm, bottom=0.5mm]
                \centering
                Language Model $\llm$
            \end{tcolorbox}
        };
        \draw[->] (query.east)--(model.west);
        \node[anchor=west] (rationales) at ([shift={(0.3, 0)}]model.east) {
            \begin{tcolorbox}[colback=white, colframe=gray, width=2.5cm, arc=1mm, valign=top, auto outer arc, fontupper=\color{black}\scriptsize, boxsep=0.5mm, left=0.5mm, right=0.5mm, top=0.5mm, bottom=0.5mm]
                \centering
                \textbf{Sampled Rationale}
                
                $\pz_1, \pz_2, \ldots, \pz_K$
                
            \end{tcolorbox}
        };
        \draw[->] (model.east) --(rationales.west);
        \node[right=0.3cm of rationales.east, anchor=west] (rewards) {
            \begin{tcolorbox}[colback=white, colframe=gray, width=3cm, arc=1mm, valign=top, auto outer arc, fontupper=\color{black}\scriptsize, boxsep=0.5mm, left=0.5mm, right=0.5mm, top=0.5mm, bottom=0.5mm]
            \centering
            \textbf{Self-reward}: Compute the likelihood of $\llm$ generating $\py$ after observing $\px$ and $\pz$.
            \end{tcolorbox}
        };
        \node[below=0.3cm of rewards.south, anchor=north] (groundtruth){
            \begin{tcolorbox}[colback=white, colframe=gray, width=2.5cm, arc=1mm, valign=top, auto outer arc, fontupper=\color{black}\scriptsize, boxsep=0.5mm, left=0.5mm, right=0.5mm, top=0.5mm, bottom=0.5mm]
            \centering
            \textbf{Groundtruth $\py$}
            \end{tcolorbox}
        };

        \draw[->] (rationales.east) --  (rewards.west);
        \draw[->] (groundtruth.north) -- (rewards.south);
        \draw[->] (model.north) -- ([yshift=0.3cm]model.north) -- ([yshift=0.25cm]rewards.north) -- (rewards.north);
        \node[anchor=west] (model_2) at ([shift={(0.5, 0)}]rewards.east){
            \begin{tcolorbox}[colback=blue!5!white, colframe=main, width=1.5cm, height=1cm, arc=1mm, valign=center, auto outer arc, fontupper=\color{black}\scriptsize, boxsep=0.5mm, left=0.5mm, right=0.5mm, top=0.5mm, bottom=0.5mm]
                \centering
                Language Model $\llm$
            \end{tcolorbox}
        };
        \draw[->] (rewards.east) -- node[below]{\tiny update}(model_2.west);
        \node[anchor=north west] (example) at (0, -2.5) {
            \begin{tcolorbox}[colback=white, colframe=gray, width=\textwidth, arc=1mm, valign=top, auto outer arc, fontupper=\color{black}\footnotesize, boxsep=0.5mm, left=0.5mm, right=0.5mm, top=0.5mm, bottom=0.5mm]
                \textbf{Question}: A robe takes 2 bolts of blue fiber and half that much white fiber. How many bolts does it take?

                \textbf{Groundtruth}: The answer is 3.

                
                \textbf{Sampled Rationale 1 (correct ✅, higher likelihood)}: It takes 2/2 = 1 bolt of white fiber. 2 + 1 = 3. So, it takes a total of 3 bolts of fiber.

                \textbf{Sampled Rationale 2 (incorrect ❌, lower likelihood)}: We need 2 bolts of blue and 2 bolts of white fiber. In total, it is 2 + 2 = 4.

            \end{tcolorbox}
        };
    \end{tikzpicture}
    \caption{\small Overview of LaTRO with an example question from GSM8K \citep{cobbe2021training}. LaTRO treats reasoning trajectories as latent variables and optimizes the underlying distribution through self-rewarding. Given a question, the language model generates multiple reasoning rationales, evaluates their likelihood of producing the correct answer, and updates its parameters to favor high-quality rationales. This iterative process allows the model to improve both its ability to generate good reasoning paths and to evaluate the quality of those paths.}
    \label{fig:overview}
    \vspace{-0.2cm}
\end{figure}

Recent progress has also focused on inference-time techniques to enhance the reasoning abilities of LLMs \citep{wu2024empirical, brown2024large}, as observed in the OpenAI o1 model \citep{openai_learning_2024}. These methods have demonstrated remarkable performance in diverse reasoning tasks, including mathematics \citep{cobbe2021training, trinh2024solving, luo2024improve}, coding \citep{jimenez2023swe, guo2024deepseek, zhang2024diversity}, and scientific problem-solving \citep{rein2023gpqa}. Notable inference-time methods, such as CoT with Self-Consistency (CoT-SC) \citep{DBLP:conf/iclr/0002WSLCNCZ23} and CoT-Decoding \citep{wang2024chain}, extend the CoT approach by generating multiple reasoning paths and selecting the most consistent one. Additionally, techniques like ReAct \citep{DBLP:conf/iclr/YaoZYDSN023} and Reflexion \citep{DBLP:conf/nips/ShinnCGNY23} integrate reasoning into LLM agent loops, further enhancing their problem-solving capabilities.

Despite the promising results at inference time, improving the reasoning abilities of LLMs during their training phase remains a challenging problem. Several obstacles impede progress in this area. Firstly, there is a scarcity of high-quality reasoning data for complex problems, limiting the applicability of traditional supervised fine-tuning (SFT) approaches \citep{zelikman2022star}. Moreover, when such data is available, SFT on deterministic reasoning paths may result in a lack of diversity in problem-solving strategies, potentially causing over-confidence issues and performance degradation \citep{cobbe2021training}, especially in domains needing multiple valid approaches, such as mathematical proofs and coding.
Alternatively, improving reasoning through reinforcement learning from human feedback (RLHF) presents its own challenges \citep{havrilla2024teaching,luo2024improve}. Developing a reward model that accurately evaluates the quality and validity of reasoning paths is a formidable task, susceptible to distribution shifts and biased evaluations. 

Self-improvement approaches like STaR (Self-Taught Reasoner) \citep{zelikman2022star} and Quiet-STaR \citep{zelikman2024quiet} have shown promise in enhancing language models' reasoning capabilities without external feedback. 
However, STaR relies on task-specific few-shot examples to bootstrap its reasoning process, which can limit its generalizability across diverse tasks. While Quiet-STaR attempts to overcome this by inferring implicit rationales across arbitrary text, it does not directly optimize the reasoning process itself. Through these findings, we observe that \textit{pretrained LLMs already possess innate reasoning capabilities but just have not been fully activated or utilized}, inspiring us to propose our approach.

Our proposed method, \methodfull, addresses the limitations of previous approaches by formulating reasoning as sampling from a latent distribution and optimizing it through a principled variational framework. As illustrated in Fig. \ref{fig:overview}, LaTRO enables language models to concurrently improve both their reasoning process and ability to evaluate reasoning quality, without requiring task-specific few-shot examples or external reward models.
Key contributions of LaTRO include:
\begin{enumerate}[left=4pt]
\item A theoretical formulation connecting LLM reasoning optimization to latent variable models;
\item A self-rewarding mechanism leveraging the model's own probability estimates;
% \item The ability to compress long reasoning processes to shorter ones during training
\item Significant performance gains across multiple model architectures and reasoning tasks, demonstrating LaTRO's effectiveness in unlocking latent reasoning capabilities of language models.
\end{enumerate}
Our findings suggest that pre-trained LLMs are not only capable reasoners but also possess the potential to act as explicit reward models for evaluating reasoning paths. We term this approach of utilizing explicit reward functions induced by LLMs themselves as "self-rewarding." Empirically, LaTRO outperforms both baseline models and supervised fine-tuning approaches on reasoning tasks like GSM8K, while also demonstrating the capacity to compress reasoning processes and shift computational burdens from inference to training time.


\section{Related work}
\paragraph{Prompt-based LLM Reasoning}
Prompt-based reasoning methods prove to be effective across various domains, 
such as math problem-solving~\citep{polu2020generative,hendrycks2021measuring,DBLP:journals/corr/abs-2110-14168}, logical reasoning~\citep{sprague2024cot} and agentic tasks~\citep{DBLP:conf/iclr/YaoZYDSN023, DBLP:conf/nips/ShinnCGNY23,yao2023retroformer}. Chain-of-Thoughts or CoT~\citep{DBLP:conf/nips/Wei0SBIXCLZ22} is the pioneering work that prompts LLMs to decompose challenging tasks into smaller reasoning steps. After that, two primary research directions further improved reasoning capabilities during inference. One direction searched over the reasoning trajectories against a process-based verifier, or reward model~\citep{yao2024tree,DBLP:conf/aaai/BestaBKGPGGLNNH24,lightman2023let}. For example, tree-of-thoughts~\citep{yao2024tree} explored over thoughts by depth-first search (DFS), breadth-first search (BFS) or beam search. The other approach used a critic model to provide verbal feedback, iteratively refining the responses with that feedback~\citep{saunders2022self,DBLP:conf/nips/ShinnCGNY23,yao2023retroformer,NEURIPS2023_91edff07}. 

\paragraph{Self-Rewarding for LLM Reasoning}

Reasoning capabilities in LLMs can be enhanced in post-training through self-rewarding and reinforcement learning. The Self-Taught Reasoner, or STaR~\citep{zelikman2022star} introduced a bootstrapping technique that allows LLMs to generate rationales and fine-tune itself with self-generated reasoning paths. Quiet-STaR~\citep{zelikman2024quiet} extended this by training LLMs to infer implicit rationales across arbitrary text, enhancing both reasoning and predictive abilities without task-specific fine-tuning. Reinforced Fine-Tuning, or ReFT~\citep{trung2024reft} took this further by leveraging reinforcement learning to improve generalization in reasoning tasks like math problem-solving, enabling LLMs to learn from multiple reasoning paths. Self-correction capabilities in LLMs can also be reinforced through self-generated data~\citep{kumar2024training}. Lastly, \citet{hoffman2024training,hu2023amortizing} formulated the reasoning process as latent variable models, aligning LLMs towards more accurate reasoning with fewer annotated data.



\section{Background anMotivulation}
\label{sec:background}
We start by briefly introducing reasoning techniques (\textit{e.g.}, chain-of-thought \citep{DBLP:conf/nips/Wei0SBIXCLZ22}, ReAct \citep{DBLP:conf/iclr/YaoZYDSN023}, \textit{etc}).
%good reasoning rationales can improve the ability of autoregressive-based LLMs to produce correct responses to complex tasks.
Given a user query $\px$, the standard procedure to sample the response $\py$ is to leverage an autoregressive pretrained LLMs $\pi_\theta$ (parameterized by $\theta$): $\py \sim \pi_\theta (\cdot ~|~ \px)$.
As for prompt-based reasoning techniques such as chain-of-thought \citep{DBLP:conf/nips/Wei0SBIXCLZ22},
the LLM $\pi_\theta$ is firstly asked to generate thoughts (\textit{a.k.a} reasoning rationale) before generating the answers to the response:
\begin{align*}
    \px^\prime := \texttt{Reason}(\px) &= \px \oplus \texttt{Prompt}~\texttt{Template}~\texttt{of}~\texttt{Thought},\\
    ~~~ \pz \sim \pi_\theta &(\cdot ~| ~\px^\prime) ,~~~~\py \sim \pi_\theta (\cdot ~| ~\px^\prime \oplus \pz)\,,
\end{align*}
where $\pz$ is the thought or the reasoning rationale path, $\oplus$ indicates the concatenate operator, and the prompt template of the thought can be some hint prompt such as ``\texttt{Let's think s
tep by step}'' \footnote{We omit the difference between $x^\prime$ and $x$ for convenience in the latter notation.}.Empirically, people observe that there is a higher chance for the LLM $\pi_\theta$ to generate the desired answer $\py$ following the above procedure than directly sampling the response $\py \sim \pi_\theta(\cdot ~| ~\px)$.  
From a statistical perspective, we hypothesize that good reasoning rationales can significantly improve the probability of generating good answers $\py$: $~\exists~ \pz, ~~\textit{s.t.}~ \pi_\theta(\py |~\px \oplus \pz) \gg \pi_\theta (\py | ~\px)$.
\begin{wrapfigure}{r}{0.40\textwidth}
    \centering
    \includegraphics[width=\linewidth]{figures/logprob_diff.pdf}
    \vspace{-1em}
    \caption{\footnotesize{Average negative log probabilities of  LLMs to generate correct responses.}}
    %The red bar indicates the mean negative log probability that a model correctly answers after only the question, whereas the gray bar indicates the mean negative log probability after the model sees both the question and the golden rationale (lower values imply higher likelihood to answer correctly). \yy{@haolin, add more explanations here, and what's y axis.} consider make this a minipage or a wrap figure}
    \label{fig:logprob_comparison}
    \vspace{-3em}
\end{wrapfigure}

To validate the hypothesis, we check the probability of the correct answers $\py$ on pretrained LLMs with or without reasoning rationales.
In \Cref{fig:logprob_comparison},
we visualize the negative log probability of the correct answers on three different LLMs on GSM8K dataset \citep{cobbe2021training}.
We have observed that when the LLMs are conditioned on the reasoning rationales, the probability of the correct answer is remarkably larger than without reasoning rationales.
This suggests that good reasoning rationales help the LLMs generate desired answers for complex tasks by increasing their probability, giving them a higher chance of generating the correct answers. 

\iffalse
The first question we would like to address is: Does LLM possess hidden reasoning capabilities or understand the semantics of reasoning?

We observe that, given a query-rationale-response triplet $(x,z,y)$, an autoregressive LLM is much more likely to produce the correct response $y$ when conditioned on both query $x$ and rationale $z$ than when conditioned on the query alone.

\fi 

%According to \Cref{fig:logprob_comparison}, the probability of generating correct responses in GSM8K increases markedly in three distinct LLMs when a golden reasoning trajectory is provided. This suggests that the log probability LLMs assign to correct answers when conditioned on the input could act as a natural reward function. This finding motivates us to develop a framework that optimizes the rationale $z$ using self-rewarding. We start by formulating prompt-based reasoning methods as a latent variable model.

%\subsection{Prompt-based reasoning as a latent variable model}
%\label{sec:formulation}
\iffalse
Given the user query, a prompt-based LLM reasoning algorithm is a function $\reason$ that maps the user query to the actual input $x$ model receives using a prompt template, such that $x$ induces an intermediate rationale $z$ and an intermediate distribution $\llm(\cdot | x\oplus z)$ to generate the answer:
\[y \sim \llm(\cdot|x\oplus z),\]
where $z\sim\llm(\cdot|x)$. For example, in CoT or ReAct \citep{DBLP:conf/iclr/YaoZYDSN023}, the LLM is asked to provide thoughts before generating the answer or action:
\[x=\reason(\texttt{query}) = \texttt{query}\oplus \text{prompt template for thoughts}.\]
We hope, and empirical evidence shown in \Cref{fig:logprob_comparison} suggests that by routing to first generate rationale $z$, at the end $\llm$ generates the desired answer $y$ with higher probabilities:
$
\exists z, \text{such that }\llm(y|x\oplus z) \geq \llm(y|x).
$
The reasoning process can be seen as sampling the rationale $z$ from a latent distribution $\llm(\cdot|x)$. With this reasoning process, we assume that: 
\begin{assumption}\label{ass:ll}
    The latent reasoning process will increase the probability of generating the groundtruth response $y^*$, that is
    \begin{equation}
        \label{eqn:assumption}
        \E_{z\sim\llm(\cdot|x)}\llm(y^*|x\oplus z) \gg \llm(y^*|x).    
    \end{equation}
\end{assumption}
\fi 


% \begin{remark}
%     When $\reason$ is CoT, the expectation in \Cref{eqn:assumption} is the limiting distribution of CoT-SC \citep{DBLP:conf/iclr/0002WSLCNCZ23}. \haolincomment{write a short proof here when time allows}
% \end{remark}
%Notice that we have the following relationship between the left hand side of \cref{eqn:assumption} and 

%We may wonder whether it is doable to further improve the reasoning ability of the LLMs by leveraging the above observation. 
%We may want to improve the reasoning ability of existing LLMs further, based on the observations. 
The above observation inspires us that we can potentially further improve the reasoning ability of existing LLMs. 
One may find some surrogate objective to enhance the quality of the reasoning rationales or improve the ability of LLMs to leverage good reasoning rationales. 
In the following \Cref{prop:cot-sc},
we show that \textit{Self-Consistency} Chain-of-Thought (CoT-SC) \citep{DBLP:conf/iclr/0002WSLCNCZ23}, which takes a majority vote of multiple reasoning rationales to improve the reasoning ability,
approximates some surrogate objective.
%the popular inference-time methods CoT-SC \citep{DBLP:conf/iclr/0002WSLCNCZ23} or major voting.

%\yy{todo: 1) add explanation of cot-sc; 2) add motivation of moving testing / inference to training. }

%Besides simple COT \citep{DBLP:conf/nips/Wei0SBIXCLZ22} which directly leverage, 
\begin{proposition}\label{prop:cot-sc}
    Denote the user query, model response, and reasoning rationale by $\px, \py,\pz$, respectively. 
    The distribution of the majority vote answer of the $K$ reasoning rationales obtained by CoT-SC approximates $p_{M}(\py | \px):= \E_{\pz\sim\llm(\cdot|~\px)}[\llm(\py|~\px\oplus \pz)]$, as $K \rightarrow \infty$.
    %When we use CoT to obtain reasoning rationales, 
    %the distribution of the 
    %When $\reason$ is CoT, $\E_{z\sim\llm(\cdot|x)}\llm(\cdot|x\oplus z)$ is the limiting distribution of CoT-SC as the number of sampled trajectories $K\rightarrow\infty$.
\end{proposition}
\begin{proof}
    Given a user query $\px$, CoT-SC essentially follows the procedure: 1) Sample $K$ \textit{i.i.d} reasoning rationales together with model responses:
    $(\pz_k, \py_k) \sim \pi_\theta (\cdot | \px),~1 \leq k \leq K$.
    2) Take the majority vote of $(\py_1,\ldots, \py_K)$.
    For a specific response $\py$, its frequency can be calculated as $F(\py):= \frac{1}{K}\sum^K_{k=1} \mathbbm{1} \{\py_k = \py\} $, where $\mathbbm{1}$ is the indicator function. Then the expectation of $F(\py)$ is 
   % Self-consistency essentially does the following: 1. Sample $K$ i.i.d. trajectories of (rationale, response): $(z_1,y_1),\ldots, (z_K, y_K)$ from $\llm(\cdot|x)$; 2. Take the majority vote of $y_i$s. Then for a specific response $y$, we have its frequency $F(y):= \frac{1}{K}\sum^K_{i=1} \mathbbm{1} \{y_i = y\} $, where $\mathbbm{1}$ is the indicator function. Then, the expectation of this frequency becomes
    \begin{equation*}
        \begin{split}
            \E_{\py_1,\ldots,\py_K} F(\py) &= \frac{1}{K}\sum^K_{k=1} \E_{\py_i} \mathbbm{1} \{\py_i = y\} = \frac{1}{K}\sum^K_{i=1} \prob_{\py_i\sim \llm(\cdot |\px\oplus \pz_i)}[\py_i = \py] \\
            &= \frac{1}{K}\sum^K_{i=1} \llm(\py| \px\oplus \pz_i) \overset{K\rightarrow\infty}{\longrightarrow} \E_{\pz\sim \llm(\cdot|\px)}\llm(\py|\px\oplus \pz).
        \end{split}
    \end{equation*}
\end{proof}
CoT-SC essentially leverages $p_{M}(\py | \px):= \int\llm(\pz|\px)\llm(\cdot|~\px\oplus \pz)d\pz$  
to obtain reasoning rationales and produce final correct answers.
Inspired by the conclusion, 
we could leverage surrogate objectives like $\E_{\pz\sim\llm(\cdot|~\px)}[\phi(\llm(\py|~\px\oplus \pz))]$ to further enhance the reasoning ability of LLMs, where $\phi$ is some monotonic transform such as logarithm ($\log(\cdot)$).
Further, we could also optimize the parameters of LLMs to enhance the reasoning abilities of existing LLMs during the training, so we can obtain LLMs with better reasoning abilities with the same inference time budget. 
In the following sections, we introduce the idea of optimizing LLMs to improve reasoning abilities without external feedback, by proposing a principled framework containing the surrogate objective.


\section{Optimizing the reasoning process}

In this section, we describe how to optimize the reasoning rationale without external feedback. Specifically,
we introduce the objective for optimizing the reasoning rationale in \Cref{sec:latro} from a variational perspective of LLM training;
we derive the gradient estimation for the new objective in \Cref{sec:gradient}, and discuss the sampling procedure together with reward shaping in \Cref{sec:sampling_control}. 
We summarize the proposed algorithm, \methodfull\ in \Cref{alg:latro_pseudocode}, and illustrate the overall procedure in \Cref{fig:overview}.
 
\subsection{Latent Reasoning Optimization: A Variational Approach}
\label{sec:latro}
Suppose we are given a golden dataset $\train_{\text{Gold}}:=\{(\px_i, \py_i)\}_{i=1}^{N}$ consisting of $N$ query and answer pairs, 
where $(\px, \py)$ denotes the query and the answer respectively.
A standard finetuning procedure to fit the LLM $\pi_{\theta}$ to the dataset $\train_{\text{Gold}}$ is by likelihood maximization:
\begin{align}
    \max_{\theta} \E_{(\px, \py)\sim \train_{\text{Gold}}}\left[ \log \pi_\theta (\py~|~\px)\right]\,, \label{equ:standard_obj}
\end{align}
where $\theta$ are the parameters of the LLM $\pi_\theta$ to optimize. 
Based on the discussion in \Cref{sec:background}, it is more feasible to optimize $\pi_\theta$ with additional reasoning rationale path $\pz$, 
compared with standard finetuning objective in \Cref{equ:standard_obj}.
Hence, we can introduce another ``reasoner'' $q(\pz | \px)$ to sample the latent reasoning rationales that can help the optimization procedure of $\pi_\theta$.
This is achievable by optimizing the following lower bound:
\begin{align}
    \log \pi_\theta(\py | \px) &=  \log \int \pi_\theta(\py~|~\px \oplus \pz) \pi_0(\pz ~| ~\px) d\pz \notag \\ %\, ~~\text{//~~\textcolor{red}{sampling objective of self-consistency cot}}\\
    & = \log \int \pi_\theta(\py~|~\px \oplus \pz) \frac{q(\pz | \px)}{q(\pz | \px)}\pi_{0}(\pz | \px)d\pz \notag \\  
    & \geq \max_{q(\pz | \px)} \E_{q(\pz | \px)}\big[\log \pi_{\theta}(\py | ~\px\oplus \pz)\big]  - \KL[q(\pz |\px)|| \pi_0(\pz | \px)]\,, \label{equ:lower_bound}
    %& = \underbrace{\E_{q_{\phi}(z | x)}\big[\log p_{\theta}(y | x\oplus z)\big]}_{\text{reward of $z$}} - \underbrace{\KL[q_\phi (z |x)|| p_0(z | x)]}_{\text{regularization }}\,. \tageq 
\end{align}
where $\pi_0$ is a prior reference LLM that regularizes the ``reasoner'' $q(\pz | \px)$, and the lower bound is achieved via Jensen's inequality \citep{higgins2017beta}. 
Based on the literature of variational Bayes~\citep{kingma2013auto}, one can either learn and optimize $q(\pz|\px)$ via variational Expectation Maximization (EM)~\citep{abdolmaleki2018maximum,liu2022constrained}, 
or introduce another parameterized LLM $q_{\phi}(\pz|\px)$ and optimize $\phi$ to amortize the cost.
Additionally, from the discussion in \Cref{sec:background}, we know $\pi_\theta$ itself can also serve as a naive ``reasoner'', since $\pi_\theta$ is an autoregressive LLM.

To simplify the learning procedure, we propose to use $\pi_\theta$ as the ``reasoner'' $q(\pz | \px)$. As a result, 
we can jointly learn one single LLM $\pi_\theta$, that is capable of generating good reasoning rationale together with providing correct answers given the query and its own generated reasoning rationale. To be more specific, we can define the learning objective as follows:
\begin{align}
    \max_{\theta} J(\theta) :=  \E_{(\px, \py)\sim \train_{\text{Gold}}} \bigg[\E_{\pz \sim \pi_{\theta}(\cdot | \px)}\big[\underbrace{\log \pi_{\theta}(\py | ~\px\oplus \pz)}_{R_\theta(\pz, \py, \px)}\big]  - \KL[\pi_\theta(\pz |\px)|| \pi_0(\pz | \px)]  \bigg]\,, \label{equ:our_obj}
\end{align}
where we specify the reference LLM $\pi_{0}$ to be the original $\pi_{\theta}$ before the optimization. 
Furthermore, $\log \pi_{\theta}(\py | ~\px\oplus \pz)$ in \Cref{equ:our_obj} can be viewed as the reward function $R_\theta(\pz, \py, \px)$ to evaluate the quality of the rationale $\pz $ given the pair $(\px, \py)$, since the reasoning rationale $\pz$ with higher likelihood $\log \pi_{\theta}(\py | ~\px\oplus \pz)$ indicates that it would provide a higher probability for the model to answer the question correctly. 

\textbf{Remark}~~~ By substituting $\log \pi_{\theta}(\py | ~\px\oplus \pz)$ with $R_\theta(\pz, \py, \px)$, \Cref{equ:our_obj} exactly recovers the standard optimization objective defined in offline RL \citep{levine2020offline}, RLHF~\citep{ouyang2022training,rafailov2024direct} literature. 
Though \Cref{equ:our_obj} unifies the learning procedure of the ``reasoner'' $\pi_{\theta}(\pz | \px)$ and the ``reward'' function $R_\theta(\pz, \py, \px):= \log \pi_{\theta}(\py | ~\px\oplus \pz)$, we can break down these two procedures to analyze them separately.
When we fix  $R_\theta(\pz, \py, \px)$ and optimize the ``reasoner'' $\pi_{\theta}(\pz | \px)$, 
the procedure can be interpreted as \emph{self-improvement} learning, where we improve $\pi_{\theta}(\pz | \px)$ on self-generated synthetic reasoning rationale. 
When we fix $\pi_{\theta}(\pz | \px)$ and optimize $R_\theta(\pz, \py, \px)$, the procedure can be interpreted as \emph{self-reward} learning, where we learn the self-reward function $\log \pi_{\theta}(\py | ~\px\oplus \pz)$.
The procedure can also be considered finetuning optimization given the learned reasoning rationale and query. 
Fortunately, we can naturally enjoy the benefits of these two self-learning procedures with the new reasoning finetuning objective.

\iffalse 
Given an LLM $\llm$, and a golden dataset $\train_{\text{Gold}}$ containing the pairs of (prompt-templated) queries and desired response $(x,y)$ with no rationale to connect them in between, we want to optimize the LLM reasoning process so that it can give the desired response of the query. 
Specifically, we want to solve the following optimization problem:
\begin{align}
    \label{eqn:latro_problem}
    \min_{\theta} \E_{(x, y)\sim \train_{\text{Gold}}}\left[-\E_{z \sim \pi_{\theta}(\cdot | x)}\left[\log \pi_{\theta}(y~|x \oplus z)\right]\right]\,.
\end{align}
The LaTRO problem \cref{eqn:latro_problem} can be seen as maximizing the reward of self-synthesized rationales $z\sim \llm(\cdot|x)$ with the self-reward function: $\log \llm(y|x\oplus z)$.


For simplicity, we restrict ourselves to the CoT template: $x = \texttt{query} \oplus \text{Let's think step by step}$.\footnote{Generalization to other templates such as ReAct is straightforward.}
Let $R(x, y;\theta)$ be the per-sample loss of \cref{eqn:latro_problem}. The gradient of $R$ is given by:
\begin{align*}
    &\nabla_{\theta}R(x, y;\theta) = - \nabla_{\theta}\E_{z \sim \pi_{\theta}(\cdot | x)}\left[\log \pi_{\theta}(y~|~x \oplus z)\right] \\
    & = - \nabla_{\theta}\int \pi_\theta(z~|~ x)\log \pi_{\theta}(y~|~x \oplus z)dz \\
    & = - \left(\int \left(\nabla_{\theta}\pi_\theta(z | x)\right)\cdot \log \pi_{\theta}(y~|~x \oplus z)dz ~+~ \int \pi_{\theta}(z | x)\left(\nabla_{\theta}\log \pi_\theta(y | x \oplus z)\right)dz \right) \\
    & = - \left(\E_{\pi_\theta(z | x)}\left[\nabla_\theta \log \pi_\theta(z | x)\cdot \log \pi_{\theta}(y~|~x \oplus z)\right] + \E_{\pi_\theta(z | x)}\left[\nabla_{\theta}\log \pi_\theta(y | x \oplus z)\right]\right) \tageq \label{eqn:gradient} \\
\end{align*}
Notice that the first term in \cref{eqn:gradient} is essentially a policy gradient (PG): $\llm$ is the policy model and $\llm(z|x)$ can be seen as the action distribution given state $x$ and $-\log \llm(y|x\oplus z)$ is then the reward the agent receives after taking the action $z$, while the second term comes from maximum likelihood estimation (MLE) given the query $x$, response $y$, with a latent rationale variable $z$. Thus we obtain:
\begin{equation}
    \label{eqn:gradient_decomposition}
    \nabla_\theta R(x, y;\theta) = \nabla_\theta \Ls^{PG}(x,y ;\theta) + \nabla_\theta \Ls^{MLE}(x,y;\theta)
\end{equation}

Next, we show the relationship between our \methodfull (LaTRO) problem \labelcref{eqn:latro_problem} and the direct supervised fine-tuning on the pairs $(x,y)$.
\paragraph{Interpretation as Optimizing ELBO}
Suppose we are given a LLM $p_\theta$ with limited capabilities, and we want to optimize:
\begin{align}
    \max_\theta \log p_{\theta}(y | x)\,.
\end{align}
By introducing and marginalizing a latent variable $z \sim p(\cdot |x)$, we know
\begin{equation}
    \label{eqn:elbo}
    \begin{split}
            \log p_\theta(y | x) &= \log \E_{z \sim p(\cdot | x)} p_\theta(z \oplus y ~| x) \\
            & = \log \int p_\theta(y|x \oplus z) p(z | x) dz\,,
    \end{split}
\end{equation}
where $p(z | x)$ can be arbitrary distribution, and we can choose a fronzen copy of the original LLM, denoted as $p_0(z | x)$. Next, we introduce another distribution $q_\phi$ as our ``reasoner":
\begin{align*}
    \log p_\theta(y | x) &=  \log \int p_\theta(y|x \oplus z) p_0(z | x) dz \\ %\, ~~\text{//~~\textcolor{red}{sampling objective of self-consistency cot}}\\
    & = \log \int p_\theta(y|x\oplus z) \frac{q_\phi(z | x)}{q_\phi(z | x)}p_{0}(z | x)dz \\  
    & \geq \E_{q_{\phi}(z | x)}\big[\log p_{\theta}(y | x\oplus z) - \log q_\phi(z | x) + \log p_0(z | x)\big] \\
    & = \underbrace{\E_{q_{\phi}(z | x)}\big[\log p_{\theta}(y | x\oplus z)\big]}_{\text{reward of $z$}} - \underbrace{\KL[q_\phi (z |x)|| p_0(z | x)]}_{\text{regularization }}\,. \tageq \label{eqn:latro_objective}
\end{align*}
We see that not only does \cref{eqn:latro_objective} fit in a reward modeling framework, but also it recovers the original objective \labelcref{eqn:latro_problem} with a Kullback-Leibler (KL) regularization when $p_\theta = q_\phi = \llm$. In this case, we fine-tune the entire LLM for better reasoning. There are a few other realizations of \cref{eqn:latro_objective} that we will discuss in \Cref{sec:conclusion_future_work} and will defer to future investigation. For clarity, we use \cref{eqn:latro_objective} as our objective function in the rest of the paper and distinguish the policy gradient and the MLE updates, while setting $p_\theta$ and $q_\phi$ to be the same LLM $\llm$ in implementation.
% \haolincomment{and $r(a, x) = \E_{q_{\theta}(z | x)}\big[\log p_{\theta}(a | x\oplus z)\big]$ becomes the reward model of generating answer $a$ with $x$. Can we also use this to optimize with respect to $x$, in the sense of finding the best prompting tokens?}

\fi 

\subsection{Gradient estimation for LaTRO}
\label{sec:gradient}
From previous RL literature, we know that estimating $\nabla_{\theta} J(\theta)$ in \Cref{equ:our_obj} involves the use of policy gradient methods, 
which usually suffers from high variances with the naive REINFORCE estimators~\citep{williams1992simple}. Inspired by the recent work on policy gradient for LLMs~\citep{ahmadian2024back}, 
we also leverage the REINFORCE Leave-One-Out (RLOO)~\citep{DBLP:conf/iclr/KoolHW19a} to optimize the ``reasoner'' $\pi_\theta(\pz |\px)$, 
where we can achieve lower variances of gradient estimation by sampling multiple rationales. 
We summarize the empirical gradient estimation for solving LaTRO in \Cref{prop:ge}.


\begin{proposition}(LaTRO Gradient Estimation) \label{prop:ge}
    Suppose we are given a set of training data $\train_{\text{Gold}}:= \{\px_i, \py_i\}_{i=1}^{N}$, we sample $K$ \textit{i.i.d} reasoning rationales $\pz_1^{(i)}, \pz_2^{(i)}, \ldots, \pz_K^{(i)} \sim \pi_\theta(\cdot | \px_i)$ for each query and answer pair $(\px_i, \py_i)$. The empirical gradient estimator for $\nabla_{\theta} J(\theta)$ is expressed as 
    \begin{align}
       &~~~~~~~~\nabla_{\theta} \widehat{J}(\theta) := \frac{1}{NK}\sum_{i=1}^{N}\sum_{k=1}^{K}\bigg( \nabla_\theta \log \pi_{\theta}(\pz_k^{(i)} ~|~ \px_i)\cdot A_k^{(i)} + \nabla_\theta \log \pi_\theta (\py_i ~|~ \px_i \oplus \pz_k^{(i)} )  \bigg)\,, \label{equ:gradient} \\ 
       &\text{with} ~A_k^{(i)} = r(\pz_k^{(i)}) - \frac{1}{K-1}\sum_{j \neq k}^{K} r(\pz_j^{(i)})\,, r(\pz_k^{(i)}) := \log \pi_\theta(\py_i~|~\px_i \oplus \pz_{k}^{(i)}) - \beta \log \frac{\pi_{\theta}(\pz_k^{(i)} ~|~\px_i )}{\pi_{0}(\pz_k^{(i)} ~|~ \px_i)} \,,\notag 
    \end{align}
where $\beta \geq 0$ is the coefficient to control the KL penalty. The proof can be found in \Cref{app:proof1}. 
\end{proposition}
The first gradient term in \Cref{equ:gradient} serves as policy gradient to improve
the ability of the LLM $\pi_\theta$ to generate high-quality reasoning rationales,
and $\log \pi_\theta (\py | \px \oplus \pz)$ can be viewed as the evaluator for reasoning rationale, which is further used to calculate the advantages. 
The second gradient term in \Cref{equ:gradient},
which is the gradient of supervised finetuning loss, essentially helps the LLM $\pi_\theta$ to leverage the reasoning rationales to produce correct answers. 

\iffalse 

Set $\Ls(x, y; \phi, \theta) := -\E_{q_{\phi}(z | x)}\big[\log p_{\theta}(y | x\oplus z)\big]$. Next, we introduce the method to estimate the gradient of $\Ls(x,y;\phi, \theta)$ such that the model is optimized via policy gradient. Following the same derivation as \cref{eqn:gradient_decomposition}, we obtain that
\begin{equation}
    \begin{split}
        \nabla_\phi \Ls(x,y;\phi, \theta) & = \nabla_\phi \Ls^{PG}_\phi = -\E_{q_\phi(z|x)} \left[\nabla_\phi \log q_\phi(z|x) \cdot \log p_\theta(y | x\oplus z)\right] \\
        \nabla_\theta \Ls(x,y;\phi, \theta) & = \nabla_\theta \Ls^{MLE}_\theta = - \E_{q_\phi(z|x)} \left[\nabla_\theta \log p_\theta(y|x\oplus z)\right]
    \end{split}
\end{equation}

For the first policy gradient term: $\nabla_\phi \Ls^{PG}(\phi)$, a straightforward way is Monte Carlo (MC) sampling, where for each training sample, we sample $K$ rationales $z_1,\ldots, z_K \sim q_\phi(\cdot|x)$ and compute the average gradients as an estimate.
Alternatively, an immediate improvement is to use the leave-one-out REINFORCE estimator, or RLOO~\citep{DBLP:conf/iclr/KoolHW19a}) to reduce the variance. Let $r(z)$ be the reward function of the reasoning process $z$. RLOO estimates the baseline in the advantage function for sample $i$ using the rest $K-1$ independent samples, yielding the following gradient estimation:
\begin{equation}
    \label{eqn:pg_estimator}
    \hat{\nabla}_\phi \Ls^{PG}_\phi =  -\frac{1}{K} \sum^K_{i=1}\nabla_\phi \log q_\phi(z_i|x) \cdot A_i,
\end{equation}
where $A_i := r(z_i) - \frac{1}{K-1} \sum_{j\neq i} r(z_j)$ is the advantage of sample $i$ and $r(z) = \log p_\theta(y|x\oplus z)$. Notice that in \cref{eqn:latro_objective}, there is an extra KL divergence term. Instead of optimizing w.r.t the divergence, we omit it as a penalty term that doesn't need gradient computation in the advantage term in \cref{eqn:pg_estimator}. That said, $r(z)$ is given by:
\begin{equation}
\label{eqn:rationale_reward}
r(z) := \log p_\theta(y|x \oplus z) - \beta \log \frac{q_\phi(z|x)}{p_0(z|x)},  
\end{equation}
where $\beta \geq 0$ is a hyperparameter controlling how much KL penalty is added.
For the second term, it is simply from maximizing the log likelihood of producing the correct answer $y$, given the query $x$ and the rationale $z$. We therefore take the average gradient as an estimate:
\begin{equation}
    \label{eqn:mle_estimator}
    \hat \nabla_\theta \Ls^{MLE}_{\theta} = -\frac{1}{K} \sum^K_{i=1} \nabla_\theta \log p_\theta (y|x\oplus z_i)
\end{equation}

\fi 

\subsection{Practical Considerations}
\label{sec:sampling_control}
To reduce computation overhead and better control the sampling of reasoning rationales during training, we limit their maximum token length to $L$. The rationale ends either at the \texttt{EOS} token or at the start of a predefined answer template (e.g., "The answer is"). We then use the truncated rationale $\pz$, along with the query $\px$ and the answer $\pz$, for further computation.

We also encourage the LLM to finish its reasoning process with $L$ tokens. 
%given the limit of $L$ tokens, 
Inspired by the implementation of the RLOO trainer in the TRL library~\citep{vonwerra2022trl} , we introduce a constant penalty for rationales truncated by the maximum token length $L$. This penalty encourages the generation of rationales that fit within the specified token limit.

\begin{algorithm}[t]
    \caption{\methodfull}
    \label{alg:latro_pseudocode}
    \KwIn{Language model $\llm$, learning rate $\eta$, KL penalty factor $\beta$, MC sample size $K$, maximum generation length $L$, sample temperature $T$, number of epochs $M$, training dataset $\train_{\text{Gold}}$.}
    \KwOut{An optimized language model $\llm$.}
    \hrulefill\\
    \Fn{$\mathtt{generate}(\pi, \px, K, L, T)$}{
        Given an autoregressive language model $\pi$, input $x$, sample $K$ sequences of length $L$ from\\the distribution $\pi(\cdot | \px)$ at temperature $T$.
        
        \Return $K$ sampled sequences
    }
    \vspace{0.1cm}
    % \Fn{$\mathtt{truncate}(\px, t)$}{
    %     Given the input sequence $\px$, truncation pattern $t$, truncate $\px$ at the first occurence of $t$.
        
    %     \Return truncated sequence
    % }
    % \vspace{0.1cm}
    Intialize reference language model $\pi_0$ as $\llm$

    \For{epoch \KwTo $\Range(M)$}{
        \For{$\px_i, \py_i$ \KwTo $\train_{\text{Gold}}$}{
            $\pz_1^{(i)},\ldots,\pz_K^{(i)} \gets \Generate(\llm, \px_i, K, L, T)$ \\
            % $\pz_1^{(i)},\ldots,\pz_K^{(i)} \gets \Truncate(\pz_i^{(j)}, t)$ \textbf{for} $k$ \KwTo $\Range(K)$ \\
            % $r(z_1), \ldots, r(z_K) \gets -\log p_\theta(y|x \oplus z_i) + \beta \log \frac{q_\phi(z_i|x)}{p_0(z_i|x)}$ \\
            % \For{$i$ \KwTo $\Range(K)$}{
            %     \If{$z_i$ \textnormal{is not finished}}{
            %         $r(z_i) \gets \gamma \frac{1}{K} \sum^K_{j=1} r(z_j)$
            %     }
            % }
            % $A_1, \ldots, A_K \gets r(z_i) - \frac{1}{K-1} \sum^K_{j=1}r(z_j)$ \\
            % $\hat{\nabla}_{\phi} \Ls^{PG}_{\phi} \gets  -\frac{1}{K} \sum^K_{i=1} A_i \cdot \nabla_{\phi} \log q_\phi(z_i| x)$ \\
            % $\hat{\nabla}_\theta\Ls^{MLE}_{\theta}\gets -\frac{1}{K} \sum^K_{i=1} \nabla_\theta \log p_\theta (y|x\oplus z_i)$ \\
            % $\phi \gets \phi + \eta \hat{\nabla}_{\phi} \Ls^{PG}_{\phi}$ \\
            % $\theta \gets \theta + \eta \hat{\nabla}_\theta\Ls^{MLE}_{\theta}$ \\
        }
        Estimate $\nabla_{\theta} \widehat{J}(\theta)$ with \Cref{prop:ge}\\
        $\theta \gets \theta + \eta \nabla_{\theta} \widehat{J}(\theta)$ \\
    }
    \Return $\llm$
\end{algorithm}


\section{Experiments}
\subsection{Setup}
We evaluate the performance of the proposed method across two datasets: a mathematical reasoning dataset (GSM8K, \citet{cobbe2021training}) and a logical reasoning dataset (ARC-Challenge, \citet{DBLP:conf/naacl/TalmorHLB19}). The sizes of the datasets are listed in \Cref{tab:datasets}.

\begin{wraptable}{r}{0.47\textwidth}
    \caption{Size of the datasets}
    \vspace{-1em}
    \label{tab:datasets}
    \begin{center}        
    \def\arraystretch{1}
    \begin{tabular}{c|cc}
        \toprule
        \bf Name & \bf Training & \bf Evaluation \\
        \hline
        GSM8K & 7473 & 1319 \\
        ARC-Challenge & 1119 & 1172 \\
        \bottomrule
    \end{tabular}
    \vspace{-1em}
    \end{center}
\end{wraptable}

\paragraph{Training.}
For each dataset, we fine-tune three base models: Phi-3.5-mini-instruct~\citep{DBLP:journals/corr/abs-2404-14219}, Mistral-7B-Instruct-v0.3~\citep{DBLP:journals/corr/abs-2310-06825}, and Meta-Llama-3.1-8B-Instruct~\citep{DBLP:journals/corr/abs-2407-21783}, abbreviated as Phi-3.5, Mistral-7B, and Llama-3.1-8B, respectively. We provide two baseline comparisons: the base model and the supervised fine-tuned (SFT) model. For GSM8K, LaTRO fine-tuning excludes golden rationales from the solutions in the training set, while the SFT model is trained using golden rationales. For ARC-Challenge, as suggested in~\citep{zheng2024large}, the model is trained to generate answers to the text of multiple-choice questions rather than selecting labels. Since no golden rationales are available for ARC-Challenge, the SFT model is trained to directly generate answers.

\paragraph{Evaluation.}
For GSM8K, we evaluate all models with CoT prompting, and for ARC-Challenge, we evaluate the SFT baseline with direct answer generation, while the base model and the LaTRO fine-tuned model with CoT prompting. All evaluations are conducted with zero-shot prompts. We report both greedy decoding (GD) results and self-consistency (with temperature $T=1$) results. We choose a self-consistency sample size $k=8$ (maj@8) in \Cref{tab:main_experiment} after observing that more than 8 samples did not bring further performance improvement (see \Cref{fig:ablation}~(b) for details).

\paragraph{Implementation.} LaTRO is implemented on the high level as in \Cref{alg:latro_pseudocode}, with additional engineering techniques as discussed in \cref{sec:sampling_control}. LaTRO is implemented using the widely recognized transformers \citep{wolf-etal-2020-transformers} and TRL \citep{vonwerra2022trl} libraries, with PyTorch \citep{Ansel2024PyTorch2F} as backend. DeepSpeed ZeRO \citep{DBLP:conf/kdd/RasleyRRH20} is used in stage 3, along with Flash Attention 2 \citep{DBLP:conf/nips/DaoFERR22} to enhance training efficiency. The models were trained on a machine 
 equipped with 8xH100 80GB GPUs, using bfloat16 precision. %Hyperparameter selections are detailed as follows.
\paragraph{Hyperparameters.} AdamW optimizer with a learning rate of $5\times 10^{-7}$, no warm-up steps, and a linear decay strategy is used. The Monte Carlo (MC) sample size $K=16$ and the batch size of the data loader $3$ are predetermined, resulting in an effective batch size of $48$. Gradient accumulation steps and training batch size are subsequently adjusted to prevent out-of-memory errors during training.
A temperature of $T=1$ is used for MC sampling, and a penalty factor $\gamma=2$ is applied for incomplete rationales. The KL penalty is set at $\beta=0.05$ for GSM8K and $0.25$ for ARC-Challenge. Except for the results presented in \Cref{sec:ablation}, the maximum generation length is maintained at $L=500$. We train all models up to six epochs for GSM8K, and 12 epochs for ARC-Challenge. The checkpoint with best test accuracy is chosen.

For the SFT baseline experiments, we use a batch size of $32$ and adjust the learning rate to ensure that the evaluation loss decreases and finally converges. All SFT baselines are trained for a maximum of 12 epochs. The checkpoint with the best test accuracy is selected. 

In addition to the main quantitative results, we conduct ablation studies on two factors: 1. The maximum generation length $L$, where we study the effects of tuning $L$ in both training and inference times; 2. The self-consistency samples $k$, where we explore to what extent LaTRO can still benefit from inference-time scaling.

The main quantitative results, qualitative analysis of sample responses, and results of the ablation study are presented in \Cref{sec:results,sec:analysis,,sec:ablation}, respectively. Additional details on our prompt templates and more samples can be found in \Cref{sec:templates,sec:sample}.

\subsection{Results}
\label{sec:results}

In this subsection, we present evaluation results that demonstrate how effectively LaTRO enhances the reasoning abilities of LLMs on downstream datasets. The detailed results are provided in\Cref{tab:main_experiment}.

For the GSM8K dataset, LaTRO fine-tuned models outperform all base models by up to $19.5\%$ (Mistral-7B, $47.8\%\rightarrow 67.3\%$) and show an average improvement of $12.5\%$ across the three models examined with greedy decoding. The greatest improvement margin is observed for Mistral-7B, while the smallest is seen for Llama-3.1-8B, consistent with our initial findings in \Cref{fig:logprob_comparison}, where Mistral-7B exhibited the lowest log probability for directly answering questions and Llama-3.1-8B exhibited the highest. With self-consistency, the improvements are by up to $16.5\%$ (Phi-3.5, $74.0\% \rightarrow 90.5\%$) and the average improvement is $13.1\%$. Furthermore, LaTRO models demonstrate superior performance relative to SFT baselines, with an average improvement of $9.6\%$ for greedy decoding and $13.2\%$ for self-consistency. It is worth noting that for the SFT baseline of Llama-3.1-8B, overfitting on the test set is still observed after tuning the learning rate.

For ARC-Challenge, LaTRO fine-tuned models still outperform the baselines, though with a smaller margin. When using greedy decoding, the improvements over the base models are up to $1.6\%$ with an average increase of $1\%$. We see more increment with self-consistency, where the improvement margins are on average $2.4\%$. Comparing to SFT baslines, we find that all three models are very sensitive when fine-tuning to directly generate the answer of ARC-Challenge questions. They perform even inferior to the unoptimized base models. When using greedy decoding, the improvements of LaTRO fine-tuned models over the SFT baselines are on an average of $5.2\%$, and by up to $6\%$ (Llama-3.1-8B). In the case of self-consistency, LaTRO performs better than the base models by an average of $2.4\%$, and surpasses the SFT models by an average of $8.1\%$.
On the less surprising results compared to GSM8K, we conjecture that for ARC-Challenge, the models are already good at producing the answer either directly or through CoT prompting. Hence, further optimization of the reasoning process did not yield significant improvement.

\begin{table}[t]
    \caption{Zero-shot accuracy (\%) comparison between LaTRO and the baselines on GSM8K and ARC-Challenge datasets. The models are fine-tuned on corresponding training datasets. The base model are marked with ``N/A'' in the training method. GD stands for greedy decoding at inference time and maj@8 stands for self-consistency with 8 samples. The models are evaluated by default using CoT, except that $\dagger$ indicates the direct answer generation is applied during evaluation. }
    \label{tab:main_experiment}
    \begin{center}
    %\renewcommand{\arraystretch}{1.35}
    \def\arraystretch{1.3}
    \begin{tabular}{lllcc}
        \toprule
        \bf Base Model & \bf Training Method & \bf Inference Method & \bf GSM8K  & \bf ARC-Challenge \\ 
        \hline
        \multirow{6}{*}{Phi-3.5} & \multirow{2}{*}{N/A} & GD & 72.9 & 85.1 \\
        & & maj@8 & 74.0 & 86.0 \\
        & \multirow{2}{*}{SFT} & GD & 75.8 & 81.0\textsuperscript{$\dagger$} \\
        & & maj@8 & 77.1 & 80.5\textsuperscript{$\dagger$} \\
        & \multirow{2}{*}{LaTRO} & GD & \bf 87.6 & \bf 86.4 \\
        & & maj@8 & \bf 90.5 & \bf 87.5\\
        \hline
        \multirow{6}{*}{Mistral-7B} & \multirow{2}{*}{N/A} & GD & 47.8 & 74.1 \\
        & & maj@8 & 58.2 & 74.1 \\
        & \multirow{2}{*}{SFT} & GD & 57.2 & 70.0\textsuperscript{$\dagger$} \\
        & & maj@8 & 59.9 & 70.6\textsuperscript{$\dagger$} \\
        & \multirow{2}{*}{LaTRO} & GD& \bf 67.3 & \bf 74.3 \\
        & & maj@8 & \bf 73.8 & \bf 78.9 \\
        \hline
        \multirow{6}{*}{Llama-3.1-8B} & N/A & GD & 76.8 & 81.4\\
        & & maj@8 & 79.7 & 84.4 \\
        &  \multirow{2}{*}{SFT} & GD & 73.2 & 77.0\textsuperscript{$\dagger$} \\
        & & maj@8 & 74.7 & 76.4\textsuperscript{$\dagger$} \\
        & \multirow{2}{*}{LaTRO} & GD& \bf 80.1 & \bf 83.0 \\
        & & maj@8 & \bf 87.0 & \bf 85.3\\
        \bottomrule
    \end{tabular}
    \end{center}
    
\end{table}


\begin{figure}[t]
    \centering
     \begin{tabular}{cc}
     \includegraphics[width=.47\linewidth]{figures/ablation_study_token_length.pdf} &
     \includegraphics[width=.47\linewidth]{figures/self_consistency.pdf} \\
     \small{(a) Zero-shot accuracy with different $L$.} & \small{(b) Zero-shot maj@$k$ accuracy with different $k$.}
     \end{tabular}
    \caption{Ablation study results on GSM8K with base model Phi-3.5. In (a), the $x$-axis represents various maximum token length $L$ of reasoning rationales, $y$-axis is the accuracy, and the plot shows the zero-shot performance v.s. various maximum token lengths for different methods. In (b), the $x$-axis represents the \# of sampled reasoning rationales, the $y$-axis is the accuracy, and the plot shows the zero-shot performance v.s. the  \# of reasoning rationales used in the majority vote.}
    \label{fig:ablation}
\end{figure}



\subsection{Ablation Study}
\label{sec:ablation}

In this subsection, we present our ablation study on the effect of different parameters in LaTRO. For consistency, we fix the base model to Phi-3.5 and the dataset to GSM8K throughout the ablation experiments.

% \subsubsection{How many tokens are enough?}
\paragraph{How many tokens are enough?} \citet{DBLP:conf/iclr/0001LZ024} demonstrated that when the input length is $n$, a transformer model with a hidden size of $O(\log n)$ can solve problems equivalent to Boolean circuits of size $m$, using $m$ CoT steps. However, the empirical determination of sufficient CoT tokens for optimal performance remains underexplored. In this section, we report zero-shot accuracy with generation length $L$ ranging from 200 to 1000 tokens at inference time. Additionally, a Phi-3.5 model is fine-tuned with $L=200$ for comparison. We distinguish two LaTRO fine-tuned models, referred to as LaTRO and $\text{LaTRO}_{200}$. As shown in~\Cref{fig:ablation}(a) accuracy gains plateau when $L\geq 500$, suggesting 500 tokens might suffice for grade school math problems. In contrast, limiting $L$ to 200 reduces accuracy, unless the model is trained accordingly. Interestingly, LaTRO significantly improves performance under this constraint by training the model to generate more concise rationales. This finding implies LaTRO can also train LLMs to condense rationales for limited inference-time compute. A sample from $\text{LaTRO}_{200}$ is provided in \Cref{sec:sample} for qualitative analysis.

\paragraph{Is inference-time scaling still beneficial?} While LaTRO shifts much of the reasoning computation to training time, it can still benefit from the inference-time scaling. To explore this, We evaluate the models again using self-consistency with the number of samples $k$ ranging from 1 to 32. As shown in~\Cref{fig:ablation} (b), with $k=8$ rationales and majority voting, LaTRO achieves 90\% accuracy on GSM8K. This indicates that LaTRO equipped with self-consistency can further enhance performance, even with LaTRO's focus on training-time optimization.


\begin{figure}[ht]
    \centering
    \begin{bluetextbox}
        \setstretch{1.1}
        \textbf{Question}: Frankie and Binkie went bowling together. Frankie's score was 15 better more than twice as high as Binkie's. If Binkie bowled a score of 90, what was Frankie's score? Let's think step by step.

        \textbf{Response (Base model)}: Let's break this problem down into steps:
        
        1. We know that the difference between Frankie's score (F) and Binkie's score (B) is 15. So we can express Frankie's score as \textcolor{red}{F = B + 15}
        
        2. We also know that Frankie's score is twice as high as Binkie's score. So we can write another equation: F = 2B
        
        3. Now, we have two equations with two unknowns. Let's solve for Frankie's score (F) by setting the two equations equal to each other:
        - 2B = B + 15
        - Multiply both sides by 1 to get rid of the fraction: 2B - B = 15
        - Simplify: B = 15 (since subtracting B from 2B results in 15)
        
        4. Now that we know Binkie's score, we can find Frankie's score using either of the equations we created earlier. Let's use the first one (F = B + 15):
        - F = 15 + 15
        - Frankie's score is 30 more than Binkie's score, which is 15 in this case. So Frankie's score is 15 + 15 = 30 + 15 = 45
        
        The final answer is that Frankie's score was 45.

        \textbf{Reponse (SFT)}: Twice as many as Binkie's 90 score is equal to 2*90= <<2*90=180>>180. Therefore, \textcolor{red}{180-(180-15)=165} was the number Frankie bowled. The answer is 165

        \textbf{Response (LaTRO)}: Here's how to solve the problem:
        1. We know that Binkie's score is 90.
        2. The problem states that Frankie's score was 15 better more than twice as high as Binkie's score. This means that Frankie's score is 15 + (2 * 90) = 15 + 180 = 195.

    \end{bluetextbox}
    \caption{Sample responses of a GSM8K question from Mistral-7B models, the errors are highlighted in red.}
    \label{fig:samples_mistral}
\end{figure}



\subsection{Case Study}
\label{sec:analysis}
 We take a closer look at the responses generated by the LaTRO fine-tuned models. 
 We select a question from GSM8K and compare the responses from the base, the SFT model, and the LaTRO finetuned model. 
 We choose the set of responses from the Mistral-7B models that we evaluated. As can be seen in \Cref{fig:samples_mistral}, the base model not only generates a lengthy response, it also makes a logical mistake at the first step, where the correct equation to establish here is ``F = 2B + 15''. The SFT model simplifies the answer and makes the first step correct. However, in the second step it first makes a wrong equation, then makes an arithmetic error when evaluating this equation. Further, LaTRO can give a concise and correct answer. We include more sample responses in \Cref{sec:sample}.


%\section{Conclusion, limitation and future work}
\label{sec:conclusion_future_work}

\section{Conclusion}
\label{sec:conclusion}


In conclusion, this work introduces LaTRO, a principled framework for optimizing language models' reasoning capabilities without external feedback or reward models. By formulating reasoning as sampling from a latent distribution and leveraging self-rewarding, LaTRO enables models to concurrently improve both their reasoning process and ability to evaluate reasoning quality. Our extensive experiments across multiple model architectures and tasks demonstrate significant performance gains, with LaTRO outperforming baseline models and supervised fine-tuning approaches. These findings suggest that pre-trained LLMs possess latent reasoning capabilities that can be unlocked through our proposed optimization approach, representing a significant step towards creating more intelligent systems that can self-evolve their problem-solving capabilities.

While LaTRO shows promising results, there are some limitations to consider. The computational cost of sampling multiple rationales during training could be prohibitive for very large models. Future work could explore ways to reduce this computational overhead, such as using more efficient sampling techniques or adaptive rationale generation. Other promising directions include investigating the applicability of LaTRO to a wider range of reasoning tasks beyond math and science problems, and exploring how to 
conduct multi-step reasoning learning to to enhance reasoning capabilities further. Despite these limitations, our contributions advance both the state-of-the-art in LLM reasoning capabilities and provide valuable insights into the nature of LLM alignment and its potential for self-improvement.

\iffalse
\paragraph{Limitation}
\label{sec:limitaion}

\paragraph{Future work}
\label{sec:future_work}

\begin{itemize}[left=4pt]
    % \item \textbf{Full Fine-tuning}: Choose both $q_\phi$ and $p_\theta$ to be the current LLM for optimization as $\pi_\theta$, then we can recover the original objective \labelcref{eqn:latro_problem}, and $p_0$ can be the previous LLM to make sure we are not optimizing too far.
    \item \textbf{Adapter Optimization}: Choose $q_\phi$ to be a lightweight adapter, e.g. LoRA \citep{DBLP:conf/iclr/HuSWALWWC22} that controls the reasoning process, or even an implicit reasoning adapter \citep{DBLP:journals/corr/abs-2311-01460}, and $p_\theta = p_0 = \llm$ to be a fixed LLM.
    \item \textbf{Sampling Control}: Choose $q = p_\theta = p_0 = \llm$ to be the same, fixed LLM while $\phi$ denotes the parameters of an inference-time algorithm to control the sampling process.
    \item \textbf{Prompt Optimization}: Fix all $q_\phi = p_\theta = p_0 = \llm$ while optimizing the prompt $\reason$.
    \item \textbf{Teacher Model} $p$: since $p$ can be arbitrary distribution, we can use a stronger teacher model that is better at reasoning instead of the original LLM.
\end{itemize}

\fi 

% \clearpage


\bibliography{iclr2025/iclr2025_conference}
\bibliographystyle{iclr2025/iclr2025_conference}

\clearpage
\newpage

\appendix
\section{Additional Details on Our Theoretical Framework}
\subsection{Proof of \Cref{prop:ge}}
\label{app:proof1}
\begin{proof}
We restate the objective as follows:
\begin{align*}
    J(\theta) &:= \E_{(\px, \py)\sim \train_{\text{Gold}}} \bigg[\E_{\pz \sim \pi_{\theta}(\cdot | \px)}\big[\log \pi_{\theta}(\py | ~\px\oplus \pz)\big]  - \beta\KL[\pi_\theta(\pz |\px)|| \pi_0(\pz | \px)]  \bigg]\,, \\
    & =\E_{(\px, \py) \sim \mathcal{D}_{\text{Gold}}}\big[\E_{\pi_\theta(\pz|\px)} [\log \pi_\theta (\py | \px\oplus \pz) -\beta \log \pi_\theta (\pz | \px) + \beta \log \pi_{0}(\pz |\px)] \,\big]\,,
\end{align*}
where $\beta > 0$ is a positive coefficient to control the regularization strength.  
We take the gradient \textit{w.r.t} $\theta$ at each sample pair $(\px, \py)$, and we get 
\begin{align*}
    \nabla_\theta J(\theta; \px, \py) &:= \nabla_\theta \int (\pi_\theta(\pz | \px))( \log \pi_\theta (\py | \px\oplus \pz) -\beta \log \pi_\theta (\pz | \px) + \beta\log \pi_{0}(\pz |\px) )d\pz \\
    &= \E_{\pi_\theta(\pz|\px)}\left[\nabla_\theta \log \pi_\theta (\pz | \px)\left(\log \pi_\theta(\py | \px \oplus \pz) - \beta \log \frac{\pi_\theta(\pz | \px)}{\pi_{0}(\pz | \px)}\right)\right] \\
    &~~~~~~~~ + \E_{\pi_\theta(\pz |\px)}\left[\nabla_\theta \log \pi_\theta (\py | \px \oplus \pz) - \beta \nabla_\theta\log\pi_\theta (\pz |\px)\right]\,.
\end{align*}
We further define $r(\pz) := \log \pi_\theta (\py | \px \oplus \pz) - \beta \log \frac{\pi_{\theta}(\pz | \px)}{\pi_{0}(\pz | \px)}$, and use the fact that $\E_{\pi_\theta(\pz |\px)}[\nabla_\theta \log \pi_\theta(\pz | \px)] = \int \pi_\theta(\pz | \px)\frac{\nabla_\theta \pi_\theta(\pz | \px)}{\pi_\theta(\pz | \px)}d\pz = \nabla_\theta\int \pi_\theta(\pz | \px)d\pz = 0$. we obtain the final gradient as
\begin{align*}
    \nabla_\theta J(\theta; \px, \py) = \E_{\pi(\pz | \px)}\left[\nabla_\theta \log\pi_\theta(\pz | \px) \cdot r(\pz) + \nabla_\theta \log \pi_\theta (\py | \px \oplus \pz)\right]\,.
\end{align*}
And when we use RLOO estimator with empirical samples, we can replace above gradient estimation with empirical samples, which gives us the following result:
\begin{align*}
       &~~~~~~~~\nabla_{\theta} \widehat{J}(\theta) := \frac{1}{NK}\sum_{i=1}^{N}\sum_{k=1}^{K}\bigg( \nabla_\theta \log \pi_{\theta}(\pz_k^{(i)} ~|~ \px_i)\cdot A_k^{(i)} + \nabla_\theta \log \pi_\theta (\py_i ~|~ \px_i \oplus \pz_k^{(i)} )  \bigg)\,, \\
       &\text{with} ~A_k^{(i)} = r(\pz_k^{(i)}) - \frac{1}{K-1}\sum_{j \neq k}^{K} r(\pz_j^{(i)})\,, r(\pz_k^{(i)}) := \log \pi_\theta(\py_i~|~\px_i \oplus \pz_{k}^{(i)}) - \beta \log \frac{\pi_{\theta}(\pz_k^{(i)} ~|~\px_i)}{\pi_{0}(\pz_k^{(i)}  ~|~\px_i)} \,,
    \end{align*}
which completes the proof.
\end{proof}

\section{Prompt Templates}
\label{sec:templates}
In this section, we list all prompt templates used in our experiments. For the GSM8K dataset, we used a minimum prompt template. For the ARC-Challenge dataset, we use a CoT template for the base model and LaTRO fine-tuning, and another template direct answer generation in SFT.
\begin{figure}[ht]
    \centering
    \begin{bluetextbox}
    \texttt{\{question\}} Let's think step by step.
    \end{bluetextbox}
    \caption{CoT template for GSM8K}
    \label{fig:gsm8k_template}
\end{figure}

\begin{figure}[ht]
    \centering
    \begin{bluetextbox}
        \setstretch{1.1}
        Question: \texttt{\{question\}}
        
        Options: \texttt{\{choices["text"]\}}
        
        Let's think step by step first then print your answer like 'The answer is (one of the options)'.""",
    \end{bluetextbox}
    \caption{CoT template for ARC-Challenge}
    \label{fig:template_arc_cot}
\end{figure}

\begin{figure}[ht]
    \centering
    \begin{bluetextbox}
    \setstretch{1.1}
    The output MUST strictly follow the structure and format described below:

    1. The answer should be strictly one of the options given in the question
    
    2. Do not include any text after you choose an option.
    
    **Example:**
    
    Question: (question here)
    
    Options: [option1, option2, option3]

    The answer is: (one of the options).

    Question: \texttt{\{question\}}
    
    Options: \texttt{\{choices["text"]\}}
    \end{bluetextbox}
    \caption{Direct answer generation template for ARC-Challenge}
    \label{fig:template_arc_direct}
\end{figure}

\newpage
\section{Sample Responses}
\label{sec:sample}

\begin{figure}[ht]
    \centering
    \begin{bluetextbox}
        \setstretch{1.1}
        \textbf{Question}: Kylar went to the store to buy glasses for his new apartment. One glass costs \$5, but every second glass costs only 60\% of the price. Kylar wants to buy 16 glasses. How much does he need to pay for them? Let's think step by step.

        \textbf{Response (Base model)}:
        
        Step 1: Determine the cost of one regular glass and one discounted glass. One regular glass costs \$5. Every second glass costs 60\% of the price of a regular glass, so: 60\% of $5 = 0.60 * $5 = \$3
        
        Step 2: Calculate the total number of regular and discounted glasses. Kylar wants to buy 16 glasses in total. Since every second glass is discounted, we can divide the total number of glasses by 2 to find the number of discounted glasses: 16 glasses / 2 = 8 discounted glasses Now, subtract the number of discounted glasses from the total number of glasses to find the number of regular glasses: 16 glasses - 8 discounted glasses = 8 regular glasses Step \textcolor{red}{(did not finish)}

        \textbf{Response ($\text{LaTRO}_{200}$)}: Let's break it down:
        
        1. Kylar needs to buy 16 glasses.
        
        2. Every second glass costs 60\% of the original price (\$5), which is \$3.
        
        3. So, for every pair of glasses (one full price and one discounted), Kylar pays \$5 + \$3 = \$8.
        
        4. Since he needs 16 glasses, he buys 8 pairs. 5. Therefore, Kylar needs to pay 8 pairs * \$8 per pair = \$64. So, Kylar needs to pay \$64 for the 16 glasses. The answer is 64.

    \end{bluetextbox}
    
    \caption{Sample responses of a GSM8K question, from Phi-3.5 models, maximum generation length $L=200$. The base model does not finish the generation, while the LaTRO model generates a short and correct answer.}
    \label{fig:sample_latro_200_phi}
\end{figure}

\begin{figure}
    \centering
    \begin{bluetextbox}
        \setstretch{1.1}
        \textbf{Question}: Water, ice, and steam all have different temperatures. What is the order from coldest to hottest?
        
        \textbf{Options}: ['ice, water, steam', 'ice, steam, water','steam, ice, water','steam, water, ice']

        \textbf{Response (Base model)}: To determine the order from coldest to hottest, let's evaluate each option.

        - Ice: It is the solid form of water and is typically around 0°C (32°F).
        
        - Water: It is the liquid form of water and is usually around 100°C (212°F) or 0°C (32°F) at standard pressure and temperature conditions, respectively.
        
        - Steam: It is the gaseous form of water and is typically at a temperature equal to or greater than 100°C (212°F) at standard pressure conditions, depending on the pressure and other factors.

        Option 1: 'ice, water, steam' - \textcolor{red}{This order seems incorrect} since steam is typically hotter than water at standard conditions.
        
        Option 2: 'ice, steam, water' - This order is incorrect because ice is the coldest, but the relative order of water and steam is opposite of what we know.
        
        Option 3:'steam, ice, water' - This order is incorrect since ice is colder than steam.
        
        Option 4:'steam, water, ice' - This order seems the most plausible, considering that steam is hotter than water and water is hotter than ice.

        The answer is 'steam, water, ice'.
        
    \textbf{Response (SFT)}: \textcolor{red}{water, ice, steam}

    \textbf{Response (LaTRO)}:
    To determine the order from coldest to hottest, let's analyze each option:
    
    1. Ice has a solid form and a fixed temperature, which is typically around 0°C (32°F) at standard pressure.
    
    2. Water is liquid and exists at a temperature of 0°C (32°F) at standard pressure, melting point of ice.
    
    3. Steam is the gaseous state of water and is typically around 100°C (212°F) at standard pressure.
    
    Based on this information, we can now order them from coldest to hottest:
    
    First, we have ice, as it is the coldest.
    Then, we have water at 0°C, which is the middle temperature.
    Lastly, we have steam, which is the hottest.
    
    So, the order from coldest to hottest is 'ice, water, steam'.
    
    The answer is ['ice, water, steam'].

        \end{bluetextbox}
    
    \caption{Sample responses of an ARC-Challenge question, from Llama-3.1 models, the errors are highlighted in red. The base model shows knowledge about basic physics but makes a logical error on the order.}
    \label{fig:sample_arc_llama}
\end{figure}

\end{document}
