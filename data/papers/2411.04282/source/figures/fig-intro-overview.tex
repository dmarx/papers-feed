\begin{figure}[htbp]
    % \centering
    \begin{tikzpicture}
        \node (query) at (0,0) {
            \begin{tcolorbox}[colback=blue!5!white, colframe=gray, width=5cm, arc=1mm, auto outer arc, fontupper=\color{black}\tiny, boxsep=0.5mm, left=0.5mm, right=0.5mm, top=0.5mm, bottom=0.5mm]
                Query $x$: A robe takes 2 bolts of blue fiber and half that much white fiber. How many bolts in total does it take? Let's think step by step.
            \end{tcolorbox}
        };
        \node[below=1cm of query.south west, anchor=north west] (rationales) {
            \begin{tcolorbox}[colback=white, colframe=gray, width=5cm, arc=1mm, valign=bottom, auto outer arc, fontupper=\color{black}\tiny, boxsep=0.5mm, left=0.5mm, right=0.5mm, top=0.5mm, bottom=0.5mm]
                \centering
                \begin{tikzpicture}
                    \node (rationale1) at (0, 0){
                        \begin{tcolorbox}[colback=green!20, colframe=green!20, width=4.5cm, arc=1mm, auto outer arc, fontupper=\color{black}\tiny, boxsep=0.5mm, left=0.5mm, right=0.5mm, top=0.5mm, bottom=0.5mm]
                            Rationale $z_1$: A robe takes 2 bolts of blue fiber. It takes half that much white fiber, so it takes 2/2 = 1 bolt of white fiber. To find the total number of bolts, we add the number of blue bolts and the number of white bolts: 2 + 1 = 3. So, it takes a total of 3 bolts of fiber.
                        \end{tcolorbox}
                    };
                    \node[anchor=west] (rationale2) at ([shift={(0,-0.75cm)}]rationale1.south west) {
                        \begin{tcolorbox}[colback=green!20, colframe=green!20, width=4.5cm, arc=1mm, auto outer arc, fontupper=\color{black}\tiny, boxsep=0.5mm, left=0.5mm, right=0.5mm, top=0.5mm, bottom=0.5mm]
                            Rationale $z_2$: To make a robe, it takes 2 bolts of blue fiber. It takes half that much white fiber, so it takes $\frac{\text{1}}{\text{2}}\times$2 = 1 bolt of white fiber. So in total, it takes 2 bolts of blue fiber and 1 bolt of white fiber. 2 + 1 = 3.                   
                        \end{tcolorbox}
                    };
                    \node[anchor=west] (rationale3) at  ([shift={(0,-0.4cm)}]rationale2.south west){
                        \begin{tcolorbox}[colback=red!20, colframe=red!20, width=4.5cm, arc=1mm, auto outer arc, fontupper=\color{black}\tiny, boxsep=0.5mm, left=0.5mm, right=0.5mm, top=0.5mm, bottom=0.5mm]
                            Rationale $z_3$: The 2 bolts of blue fiber take 2 bolts of white fiber\texttt{<|eot|>}\texttt{<|eot|>}\texttt{<|eot|>}...
                        \end{tcolorbox}
                    };
                \end{tikzpicture}
                Sampled rationales
            \end{tcolorbox}
        };
        \draw[->,dotted] (query.south)--node[left](sample_arrow){\scriptsize Sample}(rationales.north);
        \node[right=0.2cm of sample_arrow.east, anchor=west] (model_q) {
            \begin{tcolorbox}[colback=blue!5!white, colframe=main, width=2.2cm, height=0.6cm, arc=1mm, valign=center, auto outer arc, fontupper=\color{black}\tiny, boxsep=0.5mm, left=0.5mm, right=0.5mm, top=0.5mm, bottom=0.5mm]
                \centering
                Reasoning model $q_\phi$
            \end{tcolorbox}
        };
        \node (groundtruth) at (4.5, 0){
            \begin{tcolorbox}[colback=blue!5!white, colframe=gray, width=3.5cm, arc=1mm, auto outer arc, fontupper=\color{black}\tiny, boxsep=0.5mm, left=0.5mm, right=0.5mm, top=0.5mm, bottom=0.5mm]
                \centering                
                Groundtruth $y$: The answer is 3.
            \end{tcolorbox}
        };
        \node (model_p) at (4.5, -1) {
            \begin{tcolorbox}[colback=blue!5!white, colframe=main, width=2cm, height=0.6cm, arc=1mm, valign=center, auto outer arc, fontupper=\color{black}\tiny, boxsep=0.5mm, left=0.5mm, right=0.5mm, top=0.5mm, bottom=0.5mm]
                \centering
                Reward model $p_\theta$
            \end{tcolorbox}
        };
        \draw[->,dotted] (groundtruth.south) -- (model_p.north);
        \node[below=0.18cm of model_p.south, anchor=north] (rewards) {
            \begin{tcolorbox}[colback=white, colframe=gray, width=3cm, arc=1mm, valign=bottom, auto outer arc, fontupper=\color{black}\tiny, boxsep=0.5mm, left=0.5mm, right=0.5mm, top=0.5mm, bottom=0.5mm]
                \centering
                \begin{tikzpicture}
                \node (reward1) at (0,0) {
                    \begin{tcolorbox}[colback=green!20, colframe=green!20, width=2.2cm, arc=1mm, auto outer arc, valign=center, fontupper=\color{black}, boxsep=0.5mm, left=0.5mm, right=0.5mm, top=0.5mm, bottom=0.5mm]
                        \centering
                        ✅ \tiny Reward 1: -16
                    \end{tcolorbox}
                };
                \node (reward2) at (0,-1) {
                    \begin{tcolorbox}[colback=green!20, colframe=green!20, width=2.2cm, arc=1mm, auto outer arc, valign=center, fontupper=\color{black}, boxsep=0.5mm, left=0.5mm, right=0.5mm, top=0.5mm, bottom=0.5mm]
                        \centering
                        ✅ \tiny Reward 2: -12
                    \end{tcolorbox}
                };
                \node (reward3) at (0,-2) {
                    \begin{tcolorbox}[colback=red!20, colframe=red!20, width=2.2cm, arc=1mm, auto outer arc, valign=center, fontupper=\color{black}, boxsep=0.5mm, left=0.5mm, right=0.5mm, top=0.5mm, bottom=0.5mm]
                        \centering
                        ❌ \tiny Reward 3: -256
                    \end{tcolorbox}
                };
                \end{tikzpicture}
                Rewards $\log p_\theta(y|x\oplus z)$
            \end{tcolorbox}
        };
        \draw[->,dotted] (model_p.south) -- (rewards.north);
        \node (mle_loss) at (8, -4) {
            \begin{tcolorbox}[colback=blue!5!white, colframe=main, width=2.5cm, height=0.75cm, arc=1mm, valign=center, auto outer arc, fontupper=\color{black}\tiny, boxsep=0.5mm, left=0.5mm, right=0.5mm, top=0.5mm, bottom=0.5mm]
                \centering
                MLE loss $\Ls^{MLE}$
            \end{tcolorbox}
        };
        \draw[->,dotted] (rewards.east)--(mle_loss.west);
        \node[above=2.5cm of mle_loss.north west, anchor=west] (rloo) {
            \begin{tcolorbox}[colback=white, colframe=gray, width=2.7cm, arc=1mm, valign=bottom, auto outer arc, fontupper=\color{black}\tiny, boxsep=0.5mm, left=0.5mm, right=0.5mm, top=0.5mm, bottom=0.5mm]
                \centering
                \begin{tikzpicture}
                    \node (advantage1) at (0, 0) {
                        \begin{tcolorbox}[colback=green!20, colframe=green!20, width=2.2cm, arc=1mm, auto outer arc, fontupper=\color{black}\tiny, boxsep=0.5mm, left=0.5mm, right=0.5mm, top=0.5mm, bottom=0.5mm]
                        ✅ Advantage 1: 118
                        \end{tcolorbox}
                    };
                    \node[anchor=west] (advantage2) at (0, -1) {
                        \begin{tcolorbox}[colback=green!20, colframe=green!20, width=2.2cm, arc=1mm, auto outer arc, fontupper=\color{black}\tiny, boxsep=0.5mm, left=0.5mm, right=0.5mm, top=0.5mm, bottom=0.5mm]
                        ✅ Advantage 2: 124
                        \end{tcolorbox}
                    };
                    \node[anchor=west] (advantage3) at (0, -2)  ([shift={(0,0)}]rationale2.south west){
                        \begin{tcolorbox}[colback=red!20, colframe=red!20, width=2.2cm, arc=1mm, auto outer arc, valign=center, fontupper=\color{black}, boxsep=0.5mm, left=0.5mm, right=0.5mm, top=0.5mm, bottom=0.5mm]
                        ❌ Advantage 3: -242
                        \end{tcolorbox}
                    };
                \end{tikzpicture}
                Leave-one-out advantage
            \end{tcolorbox}
        };
        \draw[->,dotted] (rewards.east)--(rloo.west);
        \node[right=0cm of rloo.east] {$\times$};
        \node[right=0.5cm of rloo.east] (likelihood) {
            \begin{tcolorbox}[colback=white, colframe=gray, width=2.7cm, arc=1mm, valign=bottom, auto outer arc, fontupper=\color{black}\tiny, boxsep=0.5mm, left=0.5mm, right=0.5mm, top=0.5mm, bottom=0.5mm]
                \centering
                Log likelihood of rationales $\log q_\phi(z|x)$
            \end{tcolorbox}
        };
        \node (pg_loss) at (11.55, -4) {
            \begin{tcolorbox}[colback=blue!5!white, colframe=main, width=2.5cm, height=0.75cm, arc=1mm, valign=center, auto outer arc, fontupper=\color{black}\tiny, boxsep=0.5mm, left=0.5mm, right=0.5mm, top=0.5mm, bottom=0.5mm]
                \centering
                PG loss $\Ls^{PG}$
            \end{tcolorbox}
        };
        \draw[->,dotted] (likelihood.south)--(pg_loss.north);
        \node[below=1cm of pg_loss.south, anchor=center] (model_q2) {
            \begin{tcolorbox}[colback=blue!5!white, colframe=main, width=2.2cm, height=0.6cm, arc=1mm, valign=center, auto outer arc, fontupper=\color{black}\tiny, boxsep=0.5mm, left=0.5mm, right=0.5mm, top=0.5mm, bottom=0.5mm]
                \centering
                Reasoning model $q_\phi$
            \end{tcolorbox}
        };
        \draw[->, dotted] (pg_loss.south)--node[left](backprop1){\scriptsize Backprop}(model_q2.north);
        \node[below=1cm of mle_loss.south, anchor=center] (model_p2) {
            \begin{tcolorbox}[colback=blue!5!white, colframe=main, width=2cm, height=0.6cm, arc=1mm, valign=center, auto outer arc, fontupper=\color{black}\tiny, boxsep=0.5mm, left=0.5mm, right=0.5mm, top=0.5mm, bottom=0.5mm]
                \centering
                Reward model $p_\theta$
            \end{tcolorbox}
        };
        \draw[->, dotted] (mle_loss.south)--node[left](backprop2){\scriptsize Backprop}(model_p2.north);
    \end{tikzpicture}
    \caption{Overview of Latent Reasoning Optimization using an example query from GSM8K \citep{DBLP:journals/corr/abs-2110-14168}.}
    \label{fig:enter-label}
\end{figure}