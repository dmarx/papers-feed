
\documentclass{article}

\usepackage{microtype}
\usepackage{graphicx}
\usepackage{subcaption}

\usepackage{booktabs} %

\usepackage{hyperref}


\newcommand{\theHalgorithm}{\arabic{algorithm}}


\usepackage[accepted]{icml2024}

\usepackage{amsmath}
\usepackage{amssymb}
\usepackage{mathtools}
\usepackage{amsthm}

\usepackage[capitalize,noabbrev]{cleveref}


\theoremstyle{plain}
\newtheorem{theorem}{Theorem}[section]
\newtheorem{proposition}[theorem]{Proposition}
\newtheorem{lemma}[theorem]{Lemma}
\newtheorem{corollary}[theorem]{Corollary}
\theoremstyle{definition}
\newtheorem{definition}[theorem]{Definition}
\newtheorem{assumption}[theorem]{Assumption}
\theoremstyle{remark}
\newtheorem{remark}[theorem]{Remark}

\theoremstyle{plain}
\newtheorem{theorem}{Theorem}
\newtheorem{lemma}{Lemma}[section]
\newtheorem{proposition}[lemma]{Proposition}
\newtheorem{corollary}[lemma]{Corollary}
\theoremstyle{definition}
\newtheorem{definition}[lemma]{Definition}
\newtheorem{assumption}[lemma]{Assumption}
\theoremstyle{remark}
\newtheorem{remark}[lemma]{Remark}


\newcommand{\diag}{\mathrm{diag}}
\newcommand{\norm}[1]{\left\|{#1}\right\|} %
\newcommand{\rank}{\mathrm{rank}}
\newcommand{\B}{\mathcal{B}}
\newcommand{\M}{\mathcal{M}}
\newcommand{\BS}{\B^*}
\newcommand{\BBS}{\B\B^*}
\newcommand{\BSB}{\B^*\B}
\newcommand{\MMS}{\M\M^*}
\newcommand{\MSM}{\M^*\M}
\newcommand{\BF}{\mathcal{B}\mathcal{F}}
\newcommand{\BD}{\mathcal{B}\mathcal{D}}
\newcommand{\DB}{\mathcal{D}\mathcal{B}}
\newcommand{\ind}[1]{^{(#1)}}

\newcommand{\vzero}{\mathbf{0}}
\newcommand{\vone}{\mathbf{1}}
\newcommand{\va}{\mathbf{a}}
\newcommand{\vb}{\mathbf{b}}
\newcommand{\vc}{\mathbf{c}}
\newcommand{\vd}{\mathbf{d}}
\newcommand{\ve}{\mathbf{e}}
\newcommand{\vf}{\mathbf{f}}
\newcommand{\vg}{\mathbf{g}}
\newcommand{\vh}{\mathbf{h}}
\newcommand{\vi}{\mathbf{i}}
\newcommand{\vj}{\mathbf{j}}
\newcommand{\vk}{\mathbf{k}}
\newcommand{\vl}{\mathbf{l}}
\newcommand{\vm}{\mathbf{m}}
\newcommand{\vn}{\mathbf{n}}
\newcommand{\vo}{\mathbf{o}}
\newcommand{\vp}{\mathbf{p}}
\newcommand{\vq}{\mathbf{q}}
\newcommand{\vr}{\mathbf{r}}
\newcommand{\vs}{\mathbf{s}}
\newcommand{\vt}{\mathbf{t}}
\newcommand{\vu}{\mathbf{u}}
\newcommand{\vv}{\mathbf{v}}
\newcommand{\vw}{\mathbf{w}}
\newcommand{\vx}{\mathbf{x}}
\newcommand{\vy}{\mathbf{y}}
\newcommand{\vz}{\mathbf{z}}
\newcommand{\vA}{\mathbf{A}}
\newcommand{\vB}{\mathbf{B}}
\newcommand{\vC}{\mathbf{C}}
\newcommand{\vD}{\mathbf{D}}
\newcommand{\vE}{\mathbf{E}}
\newcommand{\vF}{\mathbf{F}}
\newcommand{\vG}{\mathbf{G}}
\newcommand{\vH}{\mathbf{H}}
\newcommand{\vI}{\mathbf{I}}
\newcommand{\vJ}{\mathbf{J}}
\newcommand{\vK}{\mathbf{K}}
\newcommand{\vL}{\mathbf{L}}
\newcommand{\vM}{\mathbf{M}}
\newcommand{\vN}{\mathbf{N}}
\newcommand{\vO}{\mathbf{O}}
\newcommand{\vP}{\mathbf{P}}
\newcommand{\vQ}{\mathbf{Q}}
\newcommand{\vR}{\mathbf{R}}
\newcommand{\vS}{\mathbf{S}}
\newcommand{\vT}{\mathbf{T}}
\newcommand{\vU}{\mathbf{U}}
\newcommand{\vV}{\mathbf{V}}
\newcommand{\vW}{\mathbf{W}}
\newcommand{\vX}{\mathbf{X}}
\newcommand{\vY}{\mathbf{Y}}
\newcommand{\vZ}{\mathbf{Z}}

\newcommand{\cA}{\mathcal{A}}
\newcommand{\cB}{\mathcal{B}}
\newcommand{\cC}{\mathcal{C}}
\newcommand{\cD}{\mathcal{D}}
\newcommand{\cF}{\mathcal{F}}
\newcommand{\cG}{\mathcal{G}}
\newcommand{\cH}{\mathcal{H}}
\newcommand{\cI}{\mathcal{I}}
\newcommand{\cK}{\mathcal{K}}
\newcommand{\cL}{\mathcal{L}}
\newcommand{\cM}{\mathcal{M}}
\newcommand{\cN}{\mathcal{N}}
\newcommand{\cP}{\mathcal{P}}
\newcommand{\cS}{\mathcal{S}}
\newcommand{\cT}{\mathcal{T}}
\newcommand{\cX}{\mathcal{X}}

\newcommand{\R}{\mathbb{R}}
\newcommand{\F}{\mathbb{F}}

\newcommand{\Z}{\mathbb{Z}}
\newcommand{\ep}{\epsilon}
\newcommand{\g}{\gamma}
\newcommand{\Y}{\infty}
\newcommand{\f}[2]{\dfrac{#1}{#2}}
\newcommand{\ff}[2]{\tfrac{#1}{#2}}
\newcommand{\lm}[2]{\lim_{#1\rightarrow #2}}
\newcommand{\de}{\delta}
\newcommand{\T}{\theta}
\newcommand{\tm}{\times}
\newcommand{\su}[2]{\mathlarger{\sum\limits_{#1}^{#2}}}
\newcommand{\pd}[2]{\mathlarger{\prod\limits_{#1}^{#2}}}
\renewcommand{\sec}[1]{\section*{#1}}
\newcommand{\st}[1]{\subsection*{#1}}
\newcommand{\sst}[1]{\subsubsection*{#1}}
\renewcommand{\b}{\textbf}
\newcommand{\lessim}{\lesssim}
\newcommand{\E}{\mathbb{E}}
\newcommand{\p}{\partial}
\newcommand{\lt}{\left(}
\newcommand{\rt}{\right)}
\newcommand{\Lt}{\left[}
\newcommand{\Rt}{\right]}
\newcommand{\A}{\alpha}
\renewcommand{\b}{\beta}
\newcommand{\I}[2]{\mathlarger{\int_{#1}^{#2}}}
\newcommand{\G}{\nabla}
\newcommand{\Om}{\Omega}
\newcommand{\y}{\tau}
\newcommand{\K}{\mathcal{K}}
\newcommand{\C}{\mathbb{C}}
\newcommand{\om}{\omega}
\newcommand{\D}{\Delta}
\newcommand{\N}{\mathcal{N}}
\newcommand{\ra}{\rightarrow}
\newcommand{\Ra}{\Rightarrow}
\newcommand{\floor}[1]{\left\lfloor#1\right\rfloor}
\newcommand{\ceil}[1]{\left\lceil#1\right\rceil}
\newcommand{\ip}[1]{\left\langle#1\right\rangle}
\renewcommand{\mod}{\text{ mod }}
\newcommand{\sign}{\text{sign}}
\newcommand{\defeq}{:=}
\renewcommand{\a}{\bar{a}}
\newcommand{\MM}{\widetilde{\vM}}

\DeclareMathOperator*{\argmin}{argmin}

\newcommand{\ours}
{\textsc{Medusa}\xspace}

\usepackage[textsize=tiny]{todonotes}


\icmltitlerunning{\ours: Simple LLM Inference Acceleration Framework with Multiple Decoding Heads}

\begin{document}

\twocolumn[
\icmltitle{\ours: Simple LLM Inference Acceleration Framework with Multiple Decoding Heads}



\icmlsetsymbol{equal}{*}

\begin{icmlauthorlist}
\icmlauthor{Tianle Cai}{equal,princeton,together}
\icmlauthor{Yuhong Li}{equal,uiuc}
\icmlauthor{Zhengyang Geng}{cmu}
\icmlauthor{Hongwu Peng}{uconn}
\icmlauthor{Jason D. Lee}{princeton}
\icmlauthor{Deming Chen}{uiuc}
\icmlauthor{Tri Dao}{princeton,together}
\end{icmlauthorlist}

\icmlaffiliation{princeton}{Princeton University}
\icmlaffiliation{together}{Together AI}
\icmlaffiliation{uiuc}{University of Illinois Urbana-Champaign}
\icmlaffiliation{cmu}{Carnegie Mellon University}
\icmlaffiliation{uconn}{University of Connecticut}

\icmlcorrespondingauthor{Tianle Cai}{tianle.cai@princeton.edu}
\icmlcorrespondingauthor{Yuhong Li}{leeyh@illinois.edu}

\icmlkeywords{Machine Learning, ICML}

\vskip 0.3in
]



\printAffiliationsAndNotice{\icmlEqualContribution} %

\begin{abstract}
\textcolor{black}{
Large Language Models (LLMs) employ auto-regressive decoding that requires sequential computation, with each step reliant on the previous one's output. This creates a bottleneck as each step necessitates moving the full model parameters from High-Bandwidth Memory (HBM) to the accelerator's cache.
}
While methods such as speculative decoding have been suggested to address this issue, their implementation is impeded by the challenges associated with acquiring and maintaining a separate draft model. 
In this paper, we present \ours, an efficient method that augments LLM inference by adding extra decoding heads to predict multiple subsequent tokens in parallel. Using a \emph{tree-based attention mechanism}, \ours constructs multiple candidate continuations and verifies them simultaneously in each decoding step. By leveraging parallel processing, 
\ours substantially \textcolor{black}{reduces} the number of decoding steps required.
We present two levels of fine-tuning procedures for \ours to meet the needs of different use cases: \textbf{\ours-1}: \ours is directly fine-tuned on top of a \emph{frozen} backbone LLM, enabling lossless inference acceleration. \textbf{\ours-2}: \ours is fine-tuned together with the backbone LLM, enabling better prediction accuracy of \ours heads and higher speedup but needing a special training recipe that preserves the model's capabilities. 
Moreover, we propose several extensions that improve or expand the utility of \ours, including a \emph{self-distillation} to handle situations where no training data is available and a \emph{typical acceptance scheme} to boost the acceptance rate while maintaining generation quality.
We evaluate \ours on models of various sizes and training procedures. Our experiments demonstrate that \ours-1 can achieve over 2.2$\times$ speedup without compromising generation quality, while \ours-2 further improves the speedup to 2.3-2.8$\times$.
\end{abstract}

\section{Introduction}
%
Neural network (NN) learning has underpinned state of the art empirical
results in numerous applied machine learning tasks (see for
instance~\cite{krizhevsky2012imagenet,lecun2015deep}). Nonetheless, neural
network learning remains rather poorly understood in several regards.
Notably, it remains unclear why training algorithms find good weights, how
learning is impacted by the network architecture and activations, what is
the role of random weight initialization, and how to choose a concrete
optimization procedure for a given architecture.

We start by analyzing the expressive power of NNs subsequent to the random
weight initialization. The motivation is the empirical success of training
algorithms despite inherent computational intractability, and the fact that
they optimize highly non-convex objectives with potentially many local minima.
Our key result shows that random initialization already positions learning
algorithms at a good starting point. We define an object termed a {\em
computation skeleton} that describes a distilled structure of feed-forward
networks. A skeleton induces a family of network architectures along with a
hypothesis class $\ch$ of functions obtained by certain non-linear
compositions according to the skeleton's structure.  We show that the
representation generated by random initialization is sufficiently rich to
approximately express the functions in $\ch$. Concretely, all functions in
$\ch$ can be approximated by tuning the weights of the last layer, which is
a convex optimization task.

In addition to explaining in part the success in finding good weights, our
study provides an appealing perspective on neural network learning.  We
establish a tight connection between network architectures and their dual
kernel spaces. This connection generalizes several previous constructions
(see Sec~\ref{sec:related}). As we demonstrate, our dual view gives rise to
design principles for NNs, supporting current practice and suggesting
new ideas. We outline below a few points.

\begin{itemize}

\item Duals of convolutional networks appear a more suitable fit for
	vision and acoustic tasks than those of fully connected networks.

\item Our framework surfaces a principled initialization scheme. It is
	very similar to common practice, but incorporates a small correction.

\item By modifying the activation functions, two consecutive fully connected
	layers can be replaced with one while preserving the network's dual kernel.

\item The ReLU activation, i.e. $x \mapsto \max(x,0)$, possesses favorable
	properties. Its dual kernel is expressive, and it can be well approximated by
	random initialization, even when the initialization's scale is moderately
	changed.

\item As the number of layers in a fully connected network becomes very
	large, its dual kernel converges to a degenerate form for any non-linear
	activation.

\item Our result suggests that optimizing the weights of the last layer can
	serve as a convex proxy for choosing among different architectures prior
	to training. This idea was advocated and tested empirically
	in~\cite{saxe2011random}.

\end{itemize}

\section{Retrieval with Synchronised Graph Expansion}
\label{sec:graph_retrieval}

\def\Tqinit{\mathbf{T}_\mathbf{q}}


\begin{figure}[thbp]
  \includegraphics[width=\columnwidth]{figures/gear-sys-fig.pdf}
  \caption{\label{fig:system_diagram}System Architecture}
\end{figure}

% Start: Zhili --------------------------


Given an input query $\mathbf{q}$, let $\mathbf{C}_\mathbf{q}' = h^k_{\text{base}}\left( \mathbf{q}, {\mathbf{C}}\right )$  be a list of passages returned by the base retriever\footnote{The choice of a base retriever within our framework is flexible, without requiring any multi-hop capabilities.}.
Given this initially retrieved list of passages, $\mathbf{C}_\mathbf{q}'$, our goal is to derive relevant multi-hop contexts (passages) by retrieving a sub-graph of triples that interconnect their source passages. There are two challenges for materialising such sub-graph retrieval: \begin{inparaenum}[(i)]\item how to locate initial triples (i.e. starting nodes) $\Tqinit$, and \item how to expand the graph based on initial triples while reducing the search space\end{inparaenum}. The following sections address these challenges respectively, within \gear.



\subsection{Knowledge Synchronisation}
\label{subsection:knowledge_syncro}
\def\linkTriple{\texttt{tripleLink}}

We describe a knowledge \textbf{Sync}hronisation (\textbf{Sync}) process for locating initial nodes for graph expansion. We first employ an LLM to \texttt{read} $\mathbf{C}_\mathbf{q}'$ (see Appendix~\ref{subsec:online_retrieval_prompts}) and summarise knowledge triples that can support answering the current query $\mathbf{q}$, as defined:
\begin{align}
    \mathbf{T}_\mathbf{q}' = \texttt{read}\left (\mathbf{C}_\mathbf{q}', \mathbf{q}\right ).
    \label{eq:proximal_read}
\end{align}
$\mathbf{T}_\mathbf{q}'$ is a collection of triples to which we refer as \textit{proximal triples}. Initial nodes $\Tqinit$ for graph expansion can then be identified by linking each triple in $\mathbf{T}_\mathbf{q}'$ to a triple in $\mathbf{T}$, using the \linkTriple{} function:
\begin{align}
    \Tqinit =\left \{t_i | t_i = \linkTriple(t_i') ~ \forall t_i' \in \mathbf{T}_\mathbf{q}'\right \}.
\end{align}
The implementation of \linkTriple{} can vary. However, in this paper we consider it to be simply retrieving the most similar triple from $\mathbf{T}$.



\begin{algorithm}[ht]
\textbf{Input:} $\mathbf{q}$: query \\
\hspace*{3em} $b$: beam size \\
\hspace*{3em} $l$: maximum length \\
\hspace*{3em} $\mathrm{score}(\cdot, \cdot)$: scoring function \\
\hspace*{3em} $\{t_1, t_2, \ldots, t_n\}$: initial triples \\
\hspace*{3em} $\gamma$: hyperparameter for diversity


\begin{algorithmic}[1]
\State $B_0 \gets [\;]$
\For{$t \in \{t_1, t_2, ..., t_n\}$}
    \State $s \gets \mathrm{score}(\mathbf{q}, [t])$
    \State $B_0.\mathrm{add}(\langle s, [t] \rangle)$
\EndFor

\State $B_0 \gets \mathrm{top}(B_0, b)$


\For{$i \in \{1, \dots, l - 1\}$}
    \State $B \gets [\;]$
    
    \For{$\langle s, T \rangle \in B_{i-1}$}
        \State $V \gets [\;]$

        \For{$t \in \mathrm{get\_neighbours}(T.\mathrm{last}())$}
            \If{$\mathrm{exists}(t, B_{i-1})$}
                \State \textbf{continue}
            \EndIf
            
            \State $s' \gets s + \mathrm{score}(\mathbf{q}, T \circ t)$ ~ \texttt{\# concat} 
            \State $V.\mathrm{add}(\langle s', T \circ t \rangle)$
        \EndFor

        \State $\mathrm{sort}(V, \mathrm{descending})$

        \For{$n \in \{0, \dots, V.\mathrm{length()} - 1\}$}
            \State $\langle s', T \circ t \rangle \gets V[n]$
            \State $s' \gets s' \times e^{- \frac{\mathrm{min}(n, \gamma)}{\gamma}}$
            \State $B.\mathrm{add}(\langle s', T \circ t \rangle)$
        \EndFor
        
    \EndFor
    \State $B_i \gets \mathrm{top}(B, b)$
    
\EndFor

\State \Return $B_i$
\end{algorithmic}

\caption{Diverse Triple Beam Search}
\label{alg:beam_search}
\end{algorithm}

\subsection{Diverse Triple Beam Search}

We borrow the idea of constructing reasoning triple chains \cite{Fang2024} for expanding the graph, and present a retrieval algorithm: \textit{Diverse Triple Beam Search} (see Alg.~\ref{alg:beam_search}). 

We maintain top-$b$ sequences (beams) of triples and the scores at each step are determined by a scoring function. In this paper, we focus on leveraging a dense embedding model to compute the cosine similarity between embeddings of the query and a candidate sequence of triples, leaving other implementations of the scoring function for future work (see Section~\ref{sec:limitations}).

Considering all possible triple extensions at each step, in a Viterbi decoding fashion, would be intractable due to the size of $\mathbf{T}$. Consequently, we define the neighbourhood of a triple as the set of triples with shared head or tail entities (i.e. $\mathrm{get\_neighbours}$ in Alg.~\ref{alg:beam_search}). During each expansion step, we only consider neighbours of the last triple in the sequence, and avoid selecting previously visited triples (i.e. $\mathrm{exists}$ in Alg.~\ref{alg:beam_search}) to further reduce the search space.

While regular beam search can reduce the search space, it is prone to producing high-likelihood sequences that differ only slightly from one another \cite{Ippolito2019, Vijayakumar2018}. Our algorithm increases the diversity across beams to improve the recall for retrieval. In detail, for each beam, we sort candidate sequences extended from that beam in descending order, and weight their scores based on their relative positions. Candidate sequences that are ranked lower, within a beam, will receive smaller weights. Consequently, the resulting top-$b$ beams at each step are less likely to share the same starting sequence. 

The top-$b$ returned sequences are flattened in a breadth-first order. Each triple in the resulting list is then mapped to its source passage. This alignment between triples and passages is described in more detail in Section~\ref{sec:preliminaries}. Let $\widetilde{\mathbf{C}}_\mathbf{q}$ be the list of unique passages after alignment. The output of our graph expansion is then given by the Reciprocal Rank Fusion (RRF) \cite{Cormack2009} of $\widetilde{\mathbf{C}}_\mathbf{q}$ and the initial $\mathbf{C}_\mathbf{q}'$ list of passages :
\begin{align}
    \mathbf{C}_{\mathbf{q}} = \mathrm{RRF}\left(\widetilde{\mathbf{C}}_\mathbf{q}, \mathbf{C}_\mathbf{q}'\right ).
\end{align}
We refer to this graph-based method of retrieving relevant passages as \textbf{Sync}ronised \textbf{G}raph \textbf{E}xpansion (\textbf{SyncGE}).


\section{Multi-step Extension}


While SyncGE can enhance a base retriever with multi-hop context, some queries inherently require multiple steps to gather all necessary evidence. We materialise \gear by incorporating an agent with multi-turn capabilities, capable of interacting with the graph-retriever described above. We focus on:
\begin{itemize}
\item maintaining a gist memory of proximal knowledge obtained throughout the different steps 
\item incorporating a similar synchronisation process 
that summarises retrieved passages in proximal triples to be stored in this multi-turn gist memory
\item determining if additional steps are needed for answering the original input question
\end{itemize}
%
Within this multi-turn setting, the original input question $\mathbf{q}$ is iteratively decomposed into simpler queries: $\mathbf{q}^{(1)}, \ldots, \mathbf{q}^{(n)}$, where $\mathbf{q}^{(1)} = \mathbf{q}$ and $n \in \mathbb{N}$ represents the number of the current step.
For each query $\mathbf{q}^{(n)}$, we use the graph retrieval method introduced in Section~\ref{sec:graph_retrieval} in order to retrieve relevant passages $\mathbf{C}_{\mathbf{q}^{(n)}}$.



\subsection{Gist Memory Constructor}
To facilitate the multi-step capabilities of our agent, we introduce a \textit{gist memory}, $\mathcal{G}^{(n)}$, which is used for storing knowledge as an array of proximal triples. At the beginning of the first iteration, the gist memory is empty. During the $n$-th iteration, similar to the knowledge synchronisation module explained in Section~\ref{subsection:knowledge_syncro}, we employ an LLM to read a collection of retrieved paragraphs $\mathbf{C}_{\mathbf{q}^{(n)}}$ and summarise their content with proximal triples:

\begin{align}
\mathbf{T}_{\mathbf{q}^{(n)}}^{\mathcal{G}} = 
\begin{cases} 
    \texttt{read}\left(\mathbf{C}_{\mathbf{q}^{(n)}}, \mathbf{q} \right), & \text{if } n = 1 \\
    \texttt{read}\left(\mathbf{C}_{\mathbf{q}^{(n)}}, \mathbf{q}\textcolor{blue}{, \mathcal{G}^{(n-1)}} \right), & \text{if } n \geq 2
\end{cases}
\label{eq:proximal_read_agent}
\end{align}


Apart from the first iteration where Eq.~\ref{eq:proximal_read} and ~\ref{eq:proximal_read_agent} are identical, the inclusion of the memory in the \texttt{read} operation differentiates the construction of proximal triples produced at the subsequent steps compared to the ones from Eq.~\ref{eq:proximal_read}. $\mathcal{G}^{(n)}$ maintains the aggregated content of proximal triples s.t. 
\begin{align}
\mathcal{G}^{(n)} = \left[ \mathbf{T}_{\mathbf{q}^{(1)}}^{\mathcal{G}}  \circ \cdots \circ \mathbf{T}_{\mathbf{q}^{(n)}}^{\mathcal{G}} \right],
\end{align}where $\circ$ defines the concatenation operation. The triple memory serves as a concise representation of all the accumulated evidence, up to the $n$-th step. 

We believe the process introduced by the \texttt{read} step along with the information storage paradigm served by the gist memory, aligns well with the communication between the hippocampus and neocortex. The combination of the two establishes the synergetic behaviour between our graph retriever and the LLM that we seek to achieve within \gear.



\subsection{Reasoning for Termination}
After $\mathcal{G}^{(n)}$ is updated, we check the sufficiency of the accumulated evidence, within it, for answering the original question. This is achieved with the following LLM reasoning step:
\begin{align}
\mathbf{a}^{(n)}, \mathbf{r}^{(n)}   = \texttt{reason}(\mathcal{G}^{(n)}, \mathbf{q}),
\end{align}
% We can also call it 'sufficiency' instead of 'answerability'. I do not really have a preference.
where $\mathbf{a}^{(n)}$ denotes the query's answerability given the available evidence in $\mathcal{G}^{(n)}$, and $\mathbf{r}^{(n)}$ represents the reasoning behind this determination. When the query is deemed answerable, the system concludes its iterative process.



\subsection{Query Re-writing}
The query re-writing process leverages an LLM that incorporates three key inputs: the original query $\mathbf{q}$, the accumulated memory, and crucially, the reasoning output $\mathbf{r}^{(n)}$ from the previous step. This process can be formally expressed as:
\begin{align}
\mathbf{q}^{(n+1)} = \texttt{rewrite}\left (\mathcal{G}^{(n)}, \mathbf{q}, \mathbf{r}^{(n)} \right),
\end{align}
where $\mathbf{q}^{(n+1)}$ represents the updated query, which serves as input for the retriever in the next iteration.\\
\subsection{After Termination}
\gear aims to return a single ranked list of passages. Given the final gist memory $\mathcal{G}^{(n)}$ upon termination, we link each proximal triple in $\mathcal{G}^{(n)}$ to a list of passages as follows:
\begin{align}
    \mathbf{C}_{t_j} = \texttt{passageLink}\left(t_j, k\right),
\end{align}
where $j \in \left \{1, \dots, \vert\mathcal{G}^{(n)}\vert \right \}$. Similar to \texttt{tripleLink}, \texttt{passageLink} is implemented by retrieving passages with a triple as the query (see Appendix~\ref{appendixpara:passage_link}). The final list of passages returned by \gear is the RRF of the resulting linked passages and passages retrieved across iterations:
\begin{align}
\mathbf{C}_\mathbf{q}^{(n)} = \mathrm{RRF}\big(&\mathbf{C}_{t_1}, \ldots,\mathbf{C}_{t_{\vert\mathcal{G}^{(n)}\vert}}, \nonumber\\
    &\mathbf{C}_{\mathbf{q}^{(1)}}, \ldots, \mathbf{C}_{\mathbf{q}^{(n)}} \big).
\end{align}

All relevant prompts for the \texttt{read}, \texttt{reason} and \texttt{rewrite} steps are provided in Appendix~\ref{subsec:online_retrieval_prompts}.

\section{Experiments}
\label{sec:experiments}
\begin{figure}[h]
\centering
\includegraphics[width=\textwidth]{figures/train_nll_softmax_vs_sigmoid_v4.pdf}
\caption{Train losses comparing $\sigmoidattn$ with $\softmaxattn$.}
\label{fig:summary_nll}
\end{figure}
To empirically validate $\sigmoidattn$, we evaluate across several domains: supervised image classification using vision transformers \citep{DBLP:conf/iclr/DosovitskiyB0WZ21}, self-supervised image representation learning with SimCLR \citep{DBLP:conf/icml/ChenK0H20, DBLP:conf/icml/ZhaiLLBR0GS23}, Bootstrap Your Own Latent (BYOL) \citep{DBLP:conf/nips/GrillSATRBDPGAP20, DBLP:conf/nips/BusbridgeRALDCW23} and Masked AutoEncoders (MAE) \citep{DBLP:conf/cvpr/HeCXLDG22} as well as automatic speech recognition (ASR) \citep{synnaeve2019end,conformer} and auto-regressive language modeling (LM) \citep{DBLP:conf/nips/BrownMRSKDNSSAA20}. We also validate sequence length generalization on TED-LIUM v3~\citep{hernandez2018ted} for ASR and in small scale synthetic experiments in \cref{sec:a_se_pair_repeat_prob}.
Across all these domains and algorithms, we demonstrate that $\sigmoidattn$ matches the performance of $\softmaxattn$ (\cref{fig:summary_nll,fig:test_top1_results}), while offering training and inference speed-ups as highlighted in \cref{sec:FlashSigmoidHardwareAwareImplementation}. Empirically we make the following observations:
\begin{enumerate}[itemsep=0pt,leftmargin=*]
    \item $\sigmoidattn$ is effective for vision tasks without a bias (except MAE), but relies on LayerScale to match the performance of the baseline $\softmaxattn$ (\cref{fig:imagenet_top_1_ablations}-a) in a hyper-parameter free manner.\footnote{\Cref{sec:layerscale_free_sigmoid} demonstrates that supervised vision tasks using $\sigmoidattn$ without LayerScale can match baseline $\softmaxattn$ performance by relying on \emph{learnable} scalar bias and temperature: $\{b, t\} \in \mathbb{R}$.} All results presented for $\softmaxattn$ also fairly add LayerScale unless specified.
    \item LM and ASR are sensitive to the initial norm $|| \sigma(\mQ \mK^T/\sqrt{d_{qk}}) \mV ||$. Modulation is required via (a) relative positional embeddings like ALiBi \citep{DBLP:conf/iclr/PressSL22}, which reduces the initial attention norm by shifting logit mass to the zero regime under $\sigmoidattn$, (b) appropriate initialization of $b$ to achieve the same effect -- enabling usage of any positional embedding.
\end{enumerate}

\begin{figure}[htbp]
    \centering
    \begin{minipage}{0.48\textwidth}
        \centering
        \includegraphics[width=\textwidth]{figures/attn_norm_seed1000001_softmax_rope_vs_softmax_alibi_vs_sigmoid_sincos.png}    
        \captionsetup{justification=centering}
        \caption{$\sigmoidattn$ with SinCos.}
        \label{fig:rope_vs_sincos}
    \end{minipage}\hfill
    \begin{minipage}{0.48\textwidth}
        \centering        
        \includegraphics[width=\textwidth]{figures/attn_norm_seed1000001_softmax_rope_vs_softmax_alibi_vs_sigmoid_rope.png}
        \captionsetup{justification=centering}
        \caption{$\sigmoidattn$ with RoPE.}
        \label{fig:rope_vs_rope}
    \end{minipage}
    \hfill
    \begin{minipage}{0.48\textwidth}
        \centering
        \includegraphics[width=\textwidth]{figures/attn_norm_seed1000001_softmax_rope_vs_softmax_alibi_vs_sigmoid_alibi.png}
        \captionsetup{justification=centering}
        \caption{$\sigmoidattn$ with ALiBi.}
        \label{fig:rope_vs_alibi}
    \end{minipage}\hfill
    \begin{minipage}{0.48\textwidth}
        \centering        
        \includegraphics[width=\textwidth]{figures/attn_norm_seed1000001_softmax_rope_vs_softmax_alibi_vs_sigmoid_rope_b=-10.png}
        \captionsetup{justification=centering}
        \caption{$\sigmoidattn$ with RoPE, $b=-10$.}
        \label{fig:rope_vs_rope_b-10}
    \end{minipage}  
    \vspace{-0.4cm}
\end{figure}

\subsection{Ablations}
\label{sec:ablations}
We begin with ablations to dissect the benefits of each of our introduced components. To gain intuition about $\sigmoidattn$, we developed a research-friendly auto-regressive (AR) LM training framework to measure all components of attention and validate the effects of LayerScale, LayerNorm applied to Q and K (QK norm), different positional embedding techniques, and initialization values for $b$.
\begin{figure}[h]
    \centering
    \begin{minipage}[t]{0.48\textwidth}
        \centering
        \includegraphics[width=\textwidth]{figures/lines=activation-cols=layerscale_with_log_n_or_max3std.pdf} 
        \caption{LR sensitivity LayerScale ablation.}
        \label{fig:layerscale_ablation}
    \end{minipage}%
    \hfill
    \begin{minipage}[t]{0.48\textwidth}
        \centering
        \includegraphics[width=\textwidth]{figures/lines=activation-cols=qknorm_with_log_n_or_max3std.pdf}
        \caption{LR sensitivity QK norm ablation.}
        \label{fig:qk_norm_ablation}
    \end{minipage}
\end{figure}
\begin{figure}[h]
    \centering
    \vspace{-0.2cm}
    \includegraphics[width=\textwidth]{figures/imagenet_ablations_top1.pdf}
    \caption{ImageNet1k ViT-B/16 classification. (a) $\sigmoidattn$ is robust without QK norm (+LayerScale, -QKNorm). Removing LayerScale reduces accuracy by 1.0\% (-LayerScale, +/-QKNorm). $n^{-\alpha}$ normalization \citep{wortsman2023replacing} underperforms without LayerScale. (b) $\sigmoidattn$ multi-query attention (MQA) \citep{DBLP:journals/corr/abs-1911-02150} with one head matches multi-head attention (MHA). (c) Sigmoid with LayerScale and QK norm performs comparably to other activations, except TanH. ReLU$^2$ \citep{DBLP:conf/icml/HuaDLL22} underperforms without LayerScale and QK norm.}
    \label{fig:imagenet_top_1_ablations}
    \vspace{-0.4cm}
\end{figure}
\paragraph{Mitigating Large Attention Norms} We train a single layer AR transformer block (E=3072, D\_FF=12288) on the realnews split of C4 \citep{DBLP:journals/jmlr/RaffelSRLNMZLL20}. We train for $2^{16}$ steps using a batch size of 6 and max sequence length of 4096 using a single cycle cosine learning rate (LR) schedule without weight decay. $\sigmoidattn$ initially underperformed $\softmaxattn$ when using absolute sinusoidal (SinCos) (\cref{fig:rope_vs_sincos}) or relative (\cref{fig:rope_vs_rope}) positional embeddings (PE), which we attribute to high initial attention Frobenius norms, $\lVert \sigma(\mQ \mK^T / \sqrt{d}) \mV \rVert$. A corresponding evolution of the attention distribution and sparsity can be seen in Appendix \cref{fig:attn_evolve} and \cref{fig:attn_metric_evolve} on a synthetic task.
To address these larger attention norms, we propose: (a) using ALiBi \citep{DBLP:conf/iclr/PressSL22} whose relative bias moves initial attention logit mass to the zero region under the sigmoid activation, producing equivalent train negative log-likelihoods (\cref{fig:rope_vs_alibi}); or (b) set the attention logit bias $b$ to a negative offset proportional to the sequence length, $b \propto -\ln n$ (see \cref{sec:attn_bias_ablation} for an ablation on $b$). This enables the usage of other PE techniques like RoPE~\citep{DBLP:journals/ijon/SuALPBL24} (\cref{fig:rope_vs_rope_b-10}). 
\paragraph{LayerScale} To validate the need for LayerScale, we follow \citet{DBLP:journals/corr/abs-2309-14322} to quantify the impact on stability.
All models are trained with RoPE with $b \propto -\ln n$, using AdamW  \citep{loshchilov2017decoupled} on the 
realnews split of C4 
with $(\beta_1,\beta_2)=(0.9, 0.95)$, $\eps=10^{-8}$,  $wd=0$, 
batch size 24, maximum token sequence length of 512 from the T5 tokenizer \citep{DBLP:journals/jmlr/RaffelSRLNMZLL20}, cosine LR schedule of $2^{14}$ steps including a linear warmup of $2^{10}$ steps. 
Models have 
$n_{\text{heads}}=\kappa$,
$n_{\text{layers}}=2\times \kappa$,
$d_{\text{model}}=64\times \kappa$ and
$d_{\text{feed-forward}}=256\times\kappa$
for a scaling value $\kappa\in\{1,2,4,8,16\}$
leading to models with $\{2.2, 4.9,15.0,67.0,440.0\}M$ trainable non-embedding parameters.
Following \citet{DBLP:journals/corr/abs-2309-14322},
we sweep learning rates
$\eta\in \{3\times 10^{-4}, 1\times 10^{-3}, 3\times 10^{-3}, 1\times 10^{-2}, 3\times 10^{-2}, 1\times 10^{-1}, 3\times 10^{-1}\}$.
LR sensitivity is defined as 
$\mathbb E_{\eta\in[a,b]}\left[\min(\ell(\mathcal A(\eta)),\ell_0)-\ell^*\right]$
where $\ell(\mathcal A(\eta))$ is the loss achieved by the learning algorithm $\mathcal A$ with LR $\eta$,
$\ell_0$ is the loss at initialization, and
$\ell^*$ is the loss achieved by the best LR.
LayerScale is initialized at $10^{-4}$. 
Unlike vision tasks, where LayerScale \emph{improves performance} (\cref{fig:imagenet_top_1_ablations}-a), in LM, we observe that $\softmaxattn$ slightly benefits from LayerScale, while the performance of $\sigmoidattn$ remains largely unaffected.
\paragraph{Stability with QK Norm} \Cref{thm:regularity} indicates that the Jacobian of $\sigmoidattn$ has favorable properties compared to $\softmaxattn$. We explore this by repeating the analysis of \citet{DBLP:journals/corr/abs-2309-14322}, as described in the LayerScale analysis, to investigate the impact of QK norm \citep{DBLP:conf/icml/0001DMPHGSCGAJB23}. For language modeling, both $\sigmoidattn$ and $\softmaxattn$ exhibit sensitivity to learning rate changes without QK norm. However, incorporating QK norm significantly stabilizes performance (\cref{fig:qk_norm_ablation}). In vision tasks, $\sigmoidattn$ demonstrates robustness with and without QK norm (\cref{fig:imagenet_top_1_ablations}-a) and without the need for $n^{-\alpha}$ normalization from \citet{wortsman2023replacing}.\footnote{We ablate multiplicative sequence length scaling in more detail in \cref{sec:appendix_normalization}.}
\paragraph{Multi-query attention (MQA)} In \cref{fig:imagenet_top_1_ablations}-b we explore MQA \citep{DBLP:journals/corr/abs-1911-02150} for vision using only one head for $\{ \mK, \mV \}$. We find that both $\sigmoidattn$ and $\softmaxattn$ perform equally well with or without multiple heads even at the small scale of ViT-B/16.
\paragraph{Activation Function Ablations} As in \citet{wortsman2023replacing}, various activation functions, when combined with LayerScale and QK norm, perform equally well for vision tasks (\cref{fig:imagenet_top_1_ablations}-c). However, for sequence-critical tasks like ASR, activation functions such as ReLU pose instabilities and underperform. In the same figure, we also compare to the ReLU$^2$ proposal from \citet{DBLP:conf/icml/HuaDLL22} and find that it underperforms without LayerScale and QK norm.
\subsection{Supervised Image Classification}
\label{sec:supervised_image_classification}
Vision transformers \citep{DBLP:conf/iclr/DosovitskiyB0WZ21} extend transformers  \citep{DBLP:conf/nips/VaswaniSPUJGKP17} to treat $K \times K$ image grids as disparate tokens. All tokens are refined through sequential layers of self-attention, pooled using a CLS token or global average pooling layer, and optimized using the negative log likelihood, $\ln p(\vy|\vx)$. We train ViT-B/16 models using $\mathbb{R}^{224 \times 224 \times 3}$ images for 300 epochs using the recipe provided in \cref{sec:appendix_vision_hyperparams}. We use the same set of training hyper-parameters for both $\softmaxattn$ and $\sigmoidattn$, changing only the activation function between trials. The train negative log-likelihood is reported in \cref{fig:summary_nll} and the test top-1\% is reported in \cref{fig:test_top1_results}. We find that $\sigmoidattn$ matches both the training dynamics and the evaluation performance of $\softmaxattn$.
\subsection{Self-Supervised Image Representation Learning}
\label{sec:ssl}
Self-supervised representation learning (SSL) exploits vast quantities of unlabeled data to learn semantic representations based on inductive biases such as augmentation invariance (SimCLR \cite{DBLP:conf/icml/ChenK0H20}, BYOL \citep{DBLP:conf/nips/GrillSATRBDPGAP20}) or reconstruction from compressed representations (MAE \citep{DBLP:conf/cvpr/HeCXLDG22}). We employ vision transformer training recipes from \cite{DBLP:conf/icml/ZhaiLLBR0GS23} and \cite{DBLP:conf/nips/BusbridgeRALDCW23} (\cref{sec:appendix_vision_hyperparams}) for SimCLR and BYOL. As with supervised learning, we use the same set of training hyper-parameters for both $\softmaxattn$ and $\sigmoidattn$, changing only the activation function between trials. \Cref{fig:summary_nll} reports the train losses, and \cref{fig:test_top1_results} highlights the linear probe and finetuned test top-1\%. Despite the diverse training objectives in SSL, $\sigmoidattn$ matches $\softmaxattn$ while improving training and inference throughput (\cref{sec:FlashSigmoidHardwareAwareImplementation}).
\subsection{Automatic Speech Recognition (ASR)}
\label{sec:asr}
\begin{table}[t!]
\centering
\caption{Word error rate (\%) on LibriSpeech test sets and TED-LIUM v3~\citep{hernandez2018ted} (``TED'', joint validation and test sets split according to  duration) for transformer (255M params) with either $\softmaxattn$ or $\sigmoidattn$ (LayerScale and QK norm are used with $b=-\log n$) trained on LibriSpeech 960h data (mean duration is 10-15s). Hyper-parameters are in~\cref{sec:asr_hps}.}
\label{tab:asr-results}
\begin{center}
\begin{scriptsize}
\begin{sc}
\resizebox{\columnwidth}{!}{%
\begin{tabular}{lc|rr|rrrr}
\toprule
 attn & PE & test-clean & test-other & ted 0-10s & ted 10-20s & ted 20-30s & ted 30s+  \\
\midrule 
softmax & \multirow{7}{*}{CAPE} & 2.3 & 5.7 & 12.4 & 10.5 & 11.9 & 9.1 \\
 sigmoid &  & 2.4 & 5.5 & 12.4 & 10.3 & 12.3 & 9.7 \\
 \,\,\,\, - QK norm &  & \multicolumn{6}{c}{unstable, gradient norm and loss spikes} \\
 \,\,\,\, - LayerScale &  & 2.5 & 6.1 & 13.6 & 11.5 & 13.4 & 8.9 \\
 sigmoid ($b=-10$, learnable) &  & 2.3 & 5.5 & 12.1 & 10.5 & 13.0 & 9.3 \\
 sigmoid ($b=-5$ in $Q$, learnable) &  & 2.3 & 5.4 & 12.2 & 10.8 & 12.4 & 9.9 \\
 \,\,\,\, - QK norm &  & \multicolumn{6}{c}{unstable, gradient norm and loss spikes} \\

\midrule
softmax & \multirow{5}{*}{RoPE} & 2.2 & 5.5 & 12.7 & 10.6 & 12.8 & 9.5 \\
 sigmoid &  & 2.3 & 5.4 & 12.3 & 10.1 & 12.3 & 8.6 \\
 sigmoid ($b=-10$, learnable) &  & 2.2 & 5.2 & 12.4 & 10.5 & 12.3 & 21.8 \\
 \,\,\,\, + $\alpha=1$ &  & 2.7 & 6.6 & 14.1 & 12.0 & 14.5 & 14.9 \\
 sigmoid ($b=-5$ in $Q$, learnable) &  & \multicolumn{6}{c}{unstable, gradient norm and loss spikes} \\
\midrule
 softmax & \multirow{5}{*}{ALiBi} & 2.2 & 5.4 & 12.3 & 10.7 & 12.1 & 8.6 \\
 sigmoid &  & 2.3 & 5.1 & 12.3 & 10.5 & 12.6 & 9.1 \\
 sigmoid ($b=-10$, learnable) &  & 2.2 & 5.2 & 12.4 & 10.4 & 11.7 & 9.1 \\
 \,\, + $\alpha=1$ &  & 2.6 & 6.6 & 13.9 & 11.9 & 14.2 & 8.6 \\
 sigmoid ($b=-5$ in $Q$, learnable) &  & 2.2 & 5.2 & 12.1 & 10.4 & 12.0 & 8.2 \\
\bottomrule
\vspace{-0.4cm}
\end{tabular}
}
\end{sc}
\end{scriptsize}
\end{center}
\end{table}
We benchmark ASR using LibriSpeech data \citep{DBLP:conf/icassp/PanayotovCPK15} on 100h and 960h settings of paired speech and text transcriptions. Our PyTorch implementations of encoder-based vanilla transformer~\citep{synnaeve2019end} and conformer \citep{DBLP:conf/interspeech/GulatiQCPZYHWZW20} are trained with Connectionist Temporal Classification (CTC) \citep{DBLP:conf/icml/GravesFGS06} w/ BF16 mixed precision, w/o QK norm and w/o LayerScale. After extensively tuning $\softmaxattn$ baselines, we switch to $\sigmoidattn$ per \cref{eq:sigmoid_attn} without any other changes. We investigate the effects of post/pre-LayerNorm, model depth, optimizer type, small data regime, and connection to local attention, with details in~\cref{sec:asr_hps}.

Our main findings are: i) CAPE~\citep{DBLP:conf/nips/LikhomanenkoXSC21} PE is the most unstable for $\sigmoidattn$; ii) post-LayerNorm models with $\softmaxattn$ are hard to match with stable $\sigmoidattn$; iii) w/o QK norm $\sigmoidattn$ is unstable and significant spikes happen in both gradient norms and training loss; iv) LayerScale is needed for generalization; v) learnable bias $b=-10$ gives no loss and gradient norms spikes while matching the $\softmaxattn$ (which does not benefit from the improved throughput of \textsc{FlashSigmoid}); vi) adding a learnable bias, $b=-5$, to $Q$ instead of the attention logits also solves the initial large attention norms for CAPE and ALiBi but not for RoPE; vii) $b=-\log n$ gives rare (2-5 times) marginal gradient norms spikes with smooth loss while matching $\softmaxattn$.


\Cref{tab:asr-results} shows the main result for pre-LayerNorm  transformers with CAPE, RoPE, and ALiBi, where $\sigmoidattn$ uses LayerScale, QK norm, $b=-\log n$, and no sequence normalization. The bias is ablated with learnable bias (one per layer) in attention or $Q$ with or without sequence normalization. $\sigmoidattn$ is stabilized with bias while matching $\softmaxattn$, and $b=-\log n$ works well. In most cases, bias allows generalization to longer sequences without sequence normalization, except for RoPE where it helps for longer sequences but hurts overall performance.









\subsection{Autoregressive Large Language Modeling}
\label{sec:llm}

\newcolumntype{R}[2]{%
    >{\adjustbox{angle=#1,lap=\width-(#2)}\bgroup}%
    l%
    <{\egroup}%
}
\newcommand*\rotdiag{\multicolumn{1}{R{30}{1em}}}%

\begin{table}[t]
\centering
\caption{1B LLM English evaluation.}
\label{tab:lm_results}
\begin{sc}
\begin{scriptsize}
\bgroup
\setlength{\tabcolsep}{.35em}
\begin{tabular}{@{}lllllllllllllll@{}}
\toprule
Model   & \makecell{Seq.\\Len.} & \makecell{ARC\\Easy} & \makecell{ARC\\Challenge} & \makecell{Hella-\\swag} & Piqa & Sciq & \makecell{Wino-\\grande} & \makecell{Lambada\\OpenAI} & \makecell{TriviaQA\\(1-shot)} & \makecell{WebQS\\(1-shot)} & AVG & \makecell{Step\\time (s)} \\ \midrule
Softmax (ALiBi) & 2k & 62.2       &     26.8           &    42.4       &  59.0    &   72.3   &     88.1       &     58.4           &      19.9             &    15.4            &    49.4   & 0.38   \\
Sigmoid (ALiBi) & 2k &  62.8       &      28.8         &    42.5       &  59.7    &   70.3   &     88.6       &      59.7          &       19.1            &   13.8             &       49.5  & 0.34   \\
\midrule
Softmax (RoPE) & 4k & 63.3       &     29.3           &    43.3       &  58.1    &   71.3   &     86.9       &     58.8           &  20.4             &    15.6            &    49.7   & 0.84   \\
Softmax (ALiBi) & 4k & 62.6       &     27.7           &    42.4       &  58.6    &   71.1   &     88.2       &     58.6           &      18.9             &    14.7            &    49.2   & 0.84   \\
Sigmoid (ALiBi) & 4k &  60.5       &      27.3         &    41.3       &  57.8    &   70.5   &     87.0       &      57.6          &       18.9            &   12.6             &       48.2  & 0.67   \\ \bottomrule
\end{tabular}
\egroup
\end{scriptsize}
\end{sc}
\vspace{-0.4cm}
\end{table}

We initially iterated at the 85M scale, as it served as a proxy for larger scale training. Our findings show that: i) attention bias is required for stability, which can be learnable, but setting it to $-\log(n)$, where $n$ is the maximum training sequence length of 4096, works well and is faster; ii) RoPE is more challenging to stabilize; iii) the final setting exhibits smooth loss curves, but still shows gradient norm fluctuations. We then turn our attention to validating $\sigmoidattn$ at scale.

We train a 1B language model using the Llama2 \citep{touvron2023llama} recipe with ALiBi instead of RoPE positional embedding, and the RedPajama \citep{together2023redpajama} dataset (see \cref{sec:llm_appendix}). At sequence length 4096, $\sigmoidattn$ achieves a \textbf{1.23}$\mathbf{\times}$ step-time improvement over $\softmaxattn$ in JAX without \textsc{FlashAttention} (\cref{tab:lm_results}). All LLMs are trained using the AXLearn framework, which include the recipe and $\sigmoidattn$ implementation.\footnote{https://github.com/apple/axlearn}

$\softmaxattn$ and $\sigmoidattn$ have matching train and validation NLL at 85M (\cref{fig:85m_4k_nll}) and at 1B scale when using 2048 sequence length (\cref{fig:summary_nll}). However, a slight disparity is observed at 1B scale when using 4096 sequence length, which we leave for future investigation (more details in \cref{sec:llm_appendix}).

\bibliography{icml/medusa_icml}
\bibliographystyle{icml2024}


\newpage
\appendix
\onecolumn
\section{Related Work}
\label{sec:related_work}

\paragraph{Attention variants and distributed attention}
Ever since attention became popular with the Transformer
architecture~\citep{vaswani2017attention}, there has been a large body of work
on approximating attention to scale it to longer sequences.
These approximation methods can generally be categorized into two classes:
sparse and low-rank.
Sparse attention only computes some entries of the attention matrix ($\mathrm{softmax}(\vQ
\vK^T)$) and assumes that other entries are zero.
Different methods have different ways of choosing which entries should be zero,
either with a fixed pattern~\citep{child2019generating}, with a sliding
window~\citep{beltagy2020longformer}, or with a dynamic pattern through
hashing~\citep{kitaev2020reformer} or routing~\citep{roy2020efficient}.
The low-rank approach instead assumes that the attention matrix has a low-rank
structure, and apply a pointwise nonlinearity to the query and
key~\citep{katharopoulos2020transformers} with random
projection~\citep{choromanski2021rethinking, peng2021random, xiong2021nystromformer}.
One can also combine the sparse and low-rank approximation for better
quality~\citep{zaheer2020bigbird,scatterbrain}.
However, these approximation methods typically do not offer the same model
quality as standard attention~\citep{tay2020efficient}, and so most large-scale
models do not employ these techniques.

There are other variants of attention aimed at reducing the size of the KV cache
to improve inference efficiency. Multi-query attention~\citep{shazeer2019fast} and grouped query
attention~\citep{ainslie2023gqa} tie different heads of $\vK$ and $\vV$, and
multiple query heads interact with the same key and value head.
Multi-head latent attention~\citep{deepseekv2} parameterizes the $\vK$ and $\vV$
as low-rank projections of a shared matrix to further reduce the KV cache size.
However, all of these approaches do not change the core computation
$\mathrm{softmax}(\vQ \vK^T) \vV$ during training and simply change how $\vQ, \vK, \vV$ are
obtained.
As a result, any efficiency or accuracy improvement to the standard attention
computation benefits these methods.

To extend to even longer context, attention computation can be distributed
across multiple GPUs.
Methods such as Ring attention~\citep{liu2023ring,liu2024world} and
variants~\citep{brandon2023striped} can reach a context length of up to 1
million.
They use \fa (or \faa) as a primitive, and so the improvement from \fat would
benefit these distributed attention methods as well.

\paragraph{Alternative architectures}
Motivated by the limitations of attention, a variety of alternative
architectures have been proposed.
They build on the connection between linear
attention~\citep{katharopoulos2020transformers} and recurrent neural networks
(RNNs).
RWKV~\citep{peng2023rwkv}, H3~\citep{dao2023hungry}, MEGA~\citep{ma2023mega},
Retnet~\citep{sun2023retentive}  enhance the expressivity of the simple
cumulative sum in linear attention with more sophisticated recurrences.
Mamba~\citep{gu2023mamba} and xLSTM~\citep{beck2024xlstm} use learnable
weighting for the recurrence and can match the quality of Transformers in
language modeling at small or medium scale.
These approaches can be connected to generalizations of linear attention through
the lens of the structure of the token-mixing matrix~\citep{dao2024transformers}.
These models have started to see some traction, seeing usage in some medium to
large-scale models such as Jamba~\citep{jamba}, Zamba~\citep{zamba},
Megalodon~\citep{ma2024megalodon}, and Mamba2-hybrid~\citep{waleffe2024empirical}.
For the highest quality, these SSM- and RNN-based models still employ
many layers of attention.
We expect that techniques to speed up attention presented in this work will be
useful to speedup these alternative architectures.

\paragraph{Low-precision attention}
Quantization is a promising approach to speed up attention, but they have mostly
focused on reducing the space for KV cache for inference efficiency.
QuIP~\citep{chee2024quip} and QuIP\#\citep{tseng2024quip} use incoherent processing to reduce the quantization,
and we adapted this technique for FP8 \fat.
Recent work suggests that for inference the KV cache is highly compressible down to 4-, 3-, or
even 2-bits~\citep{hooper2024kvquant, liu2024kivi}.
However, quantization during training is still challenging as higher precision
is typically required for stable training.

\paragraph{Hardware-aware Algorithms}
Our work presented in this paper focuses on the micro-architecture
specific tuning to leverage new instruction sets and adopt a natively
asynchronous programming model. There are other orthogonal axes for
hardware-aware algorithm co-design being explored.
A recent example of this is LeanAttention~\citep{sanovar2024-leanattention},
which recognizes the poor GPU occupancy and high memory bandwidth requirements
of the sequential token generation phase as primary bottlenecks for inference
and optimizes it via a smarter load balancing strategy similar to Stream-K
load balancing~\citep{streamk} to achieve nearly peak occupancy.
There is a large literature on optimizing GEMM for specific hardware that employs
many of the same techniques.
As an example, \citet{abdel2016batched} presents a high performance batched GEMM kernel on
K40c Graphics Processing Units (GPU) for both fixed and variable sizes,
proposing specialized GEMM designs
and a comprehensive autotuning process to deliver state-of-the-art 
performance.



\section{Experiment Settings}\label{appendix:experiment_settings}
\subsection{Common Terms}
We clarify three commonly used terms:
a) Acceleration rate: This refers to the average number of tokens decoded per decoding step. In a standard auto-regressive model, this rate is 1.0.
b) Overhead: This is used to characterize the per decoding step overhead compared to classic decoding, and is calculated by dividing the average per step latency of the \ours models by that of the vanilla model.
c) Speedup: This refers to the wall-time acceleration rate. 
Following these definitions, we have the relation: Speedup = Acceleration rate / Overhead.
\subsection{Shared Settings} 
For all the experiments, we use the Axolotl~\citep{axolotl2023} framework for training. We use a cosine learning rate scheduler with warmup and use 8-bit AdamW~\citep{dettmers20218bit} optimizer. We train $5$ \ours heads with $1$ layer and set $\lambda_k$ in Eq.~\eqref{eq:loss_medusa_1} to be $0.8^k$. For \ours-2, we use either LoRA~\citep{hu2021lora} or QLoRA~\citep{dettmers2023qlora} for fine-tuning and set the learning rate of \ours heads to be $4$ times larger than the backbone model. LoRA is applied to all the linear layers of the backbone model, including the language model head. The rank of LoRA adapter is set to $32$, and $\alpha$ is set to $16$. A dropout of $0.05$ is added to the LoRA adapter. 

\subsection{\ours-1 v.s. \ours-2 on Vicuna 7B and 13B} 
We use a global batch size of $64$ and a peak learning rate of $5e^{-4}$ for the backbone and $2e^{-3}$ for \ours heads and warmup for $40$ steps. We use $4$-bit quantized backbone models for both models. We first train the models with \ours-1 and use these trained models as initialization to train \ours-2. We employ QLoRA for \ours-2 and the  $\lambda_0$ in Eq.~\eqref{eq:loss_medusa_2} is set to be $0.2$.
\subsection{ Training with Self-Distillation on Vicuna-33B and Zephyr-7B} 
We use \ours-2 for both models instead of using a two-stage training procedure. We use a sine schedule for the $\theta_0$ to gradually increase the value to its peak at the end of the training. We find this approach is equally effective. We set the peak learning rate of the backbone LoRA adapter to be $1e^{-4}$ and the warmup steps to be $20$ since the self-distillation loss is relatively small. We set the $\lambda_0$ in Eq.~\eqref{eq:loss_medusa_2} to be $0.01$.
\section{Visualization of optimized tree attention}\label{appendix:sparse_tree}
Fig.~\ref{fig:sparse_tree} illustrates the structure of a sparsely constructed tree for the \ours-2 Vicuna-7B model. This tree structure extends four levels deep, indicating the engagement of four \ours heads in the computation. The tree is initially formed through a Cartesian product approach and subsequently refined by pruning based on the statistical expectations of the top-k predictions from each \ours head measured on the Alpaca-eval dataset~\cite{dubois2023alpacafarm}. The tree's lean towards the left visually represents the algorithm's preference for nodes with higher probabilities on each head.
\begin{figure*}[h]
    \centering
    \includegraphics[width=0.6\textwidth]{sparse_tree.pdf}
    \caption{Visualization of a sparse tree setting for \ours-2 Vicuna-7B. The tree has \textcolor{black}{64 nodes representing candidate tokens} and a depth of 4 which indicates 4 \ours heads involved in calculation. Each node indicates a token from a top-k prediction of a \ours head, and the edges show the connections between them. The red lines highlight the path that correctly predicts the future tokens.}
    \label{fig:sparse_tree}
\end{figure*}

\section{Results of Speculative Decoding}\label{appendix:spec}

In this study, speculative decoding was applied to Vicuna models~\citep{vicuna2023} with varying sizes, specifically 7B, 13B, and 33B. The preliminary framework utilized open-source models such as Llama-68M and 160M~\citep{miao2023specinfer}, alongside Tiny-Llama~\citep{zhang2024tinyllama} and Tiny-Vicuna~\citep{tiny_vicuna_1b}, fine-tuned from Tiny-Llama with the Vicuna-style instructional tuning strategy. Due to the proprietary nature of speculative decoding methods~\citep{chen2023accelerating, leviathan2022fast}, open-source alternatives\footnote{\href{https://github.com/feifeibear/LLMSpeculativeSampling}{https://github.com/feifeibear/LLMSpeculativeSampling}} were deployed for evaluation. Additionally, we utilize \verb|torch.compile()| to accelerate the inference speed of draft models.

Our results shown in Fig.~\ref{fig:speculative_decoding}, reveal that the optimal settings of the draft model vary with the Vicuna model sizes. Specifically, the Llama-68M, with a setting of the draft token number $\gamma=4$, yielded the best performance for Vicuna-7B, while the same draft model with $\gamma=3$ was most effective for Vicuna-13B. For the larger Vicuna-33B, the Tiny-Vicuna \textcolor{black}{(Vicuna-1B)}, with $\gamma=3$, provided the greatest acceleration. These results suggest that the choice and setting of the drafting model should be tailored to the size of the LLMs, presenting an area for further exploration in the field.



\begin{figure*}[h]
     \centering
     \begin{subfigure}[b]{0.32\textwidth}
         \centering
         \includegraphics[width=\textwidth]{spec_7b.pdf}
         \caption{Vicuna-7B}
         \label{fig:spec7b}
     \end{subfigure}
     \begin{subfigure}[b]{0.32\textwidth}
         \centering
         \includegraphics[width=\textwidth]{spec_13b.pdf}
         \caption{Vicuna-13B}
         \label{fig:spec13b}
     \end{subfigure}
    \begin{subfigure}[b]{0.32\textwidth}
         \centering
         \includegraphics[width=\textwidth]{spec_33b.pdf}
         \caption{Vicuna-33B}
         \label{fig:spec33b}
     \end{subfigure}
        \caption{Inference speed of various models using speculative decoding on MT-Bench. Baseline model speeds are presented by grey dotted lines for comparison. $\gamma$ denotes the draft token number.}
        \label{fig:speculative_decoding}
\end{figure*}

\section{Additional Results for All Models}\label{appendix:add_results}
We show speedup on various models in Fig.~\ref{fig:speedup_model_wild}.
\begin{figure}[h]
    \centering
    \includegraphics[width=0.45\textwidth]{speedup_model_wild_wide.pdf}
    \caption{Speedup of various models with \ours-2. \ours-2 shows significant speed improvement over all the models, while models trained with self-distillation \textcolor{black}{(Zephyr-7B, Vicuna-13/33B)} have weaker speedup due to the trade-off between preserving quality and boosting speed.}
    \label{fig:speedup_model_wild}
\end{figure}


\section{Additional Results on AlpacalEval Dataset}
We conduct further experiments on the AlpacaEval~\citep{alpaca_eval} dataset. \ours-2 achieves consistent speedup similar to the results on MT-Bench.
\begin{table}[h]
    \centering
    \begin{tabular}{llrrrr}
    \toprule
     & Model & Base speed (tokens/s) & \ours speed (tokens/s) & Acc. rate & Speedup \\
    \midrule
     & Vicuna-7b & 37.07 & 106.76 & 3.23 & 2.88 \\
     & Vicuna-13b & 29.01 & 91.54 & 3.28 & 3.16 \\
     & Vicuna-33b & 17.87 & 40.43 & 2.85 & 2.26 \\
     & Zephyr-7b & 34.21 & 99.50 & 3.08 & 2.91 \\
    \bottomrule
    \end{tabular}
    \caption{Speedup results on AlpacaEval~\citep{alpaca_eval} dataset.}
    \label{tab:alpaca_eval_speedup}
\end{table}

\section{Exploration and Modeling of Hardware Constraints and \ours}~\label{sec:roofline}


We explore the hardware constraints, specifically memory-bandwidth bound, and their impact on \ours-style parallel decoding by incorporating a simplified \textcolor{black}{Llama-series} model.
First, we \textcolor{black}{identify} that the operators involving matrix multiplications, such as linear layers and attention matrix multiplications, are the primary sources of overhead. We profile the performance of FLOP/s vs. Operational Intensity \textcolor{black}{which is the ratio of FLOP/s to bandwidth (bytes/s)}, across various GPUs, including the A100-80GB-PCIe, A40, and A6000.
Next, we examine the changes in FLOP/s vs. Operational Intensity when using \ours for different operators.
Finally, we apply a straightforward analytical model to calculate acceleration rates and combine it with hardware benchmarks. This provides insights into the effects under different model sizes, sequence lengths, and batch sizes.

\subsection{Roofline Model of Operators}
We present an analysis of the roofline model for various operators in large language models (LLMs), specifically focusing on Llama-7B, Llama-13B, and Llama-33B~\cite{touvron2023llama}. These models were benchmarked on different GPUs, including the A100-80GB-PCIe, A40, and A6000. We looked into the three categories of matrix multiplication operators since they represent the primary sources of computational overhead in these models. Our study follows the report~\cite{chen2023transformer} which investigates the effectiveness of batch size but ours focuses more on decoding and parallel decoding.

Table~\ref{tab:complexity} details the computation and space complexity for each operator during the prefill, decoding, and \ours decoding phases. The operators include the linear layers for query, key, and value matrices ($XW_{Q}$, $XW_{K}$, $XW_{V}$), the attention matrix multiplications ($QK^T$, $PV$), and the up/gate/down linear layers ($XW_{u}$, $XW_{g}$, $XW_{d}$).
$b$ stands for the batch size, $s$ stands for the sequence length, $h$ stands for the hidden dimension, $i$ stands for the intermediate dimension, $n$ stands for the number of attention heads, $d$ stands for the head dimension and $q$ stands for the candidate length for \ours.
For more details of these operators please refer to the articles~\cite{touvron2023llama, chen2023transformer}.


\begin{table}[h]
\centering
\caption{Computational and space complexity of the main operators in different phases. \textcolor{black}{The table is based on Table 2 in the report~\cite{chen2023transformer}.}}

\scriptsize
\begin{tabular}{lcccc}
\toprule
 \textbf{Operator} & \textbf{Input Shape} & \textbf{Output Shape} & \textbf{Comp. Complexity} & \textbf{Space Complexity} \\ \midrule
 \textbf{Prefill} \\ \midrule
 $XW_{Q}$, $XW_{K}$, $XW_{V}$ & $(b, s, h)$ & $(b, s, h)$ & $O(bsh^2)$ & $O(2bsh + h^2)$ \\ \midrule
  $QK^T$ & $(b, n, s, d),(b, n, s, d)$ & $(b, n, s, s)$ & $O(bs^2nd)$ & $O(2bsnd + bs^2n)$ \\ 
  $PV$ &$(b, n, s, s),(b, n, s, d)$&$(b, n, s, d)$&& \\ \midrule
  $XW_{u}$, $XW_{g}$ & $(b, s, h)$ & $(b, s, i)$ & $O(bshi)$ & $O(bs(h + i) + hi)$ \\ 
  $XW_{d}$&$(b, s, i)$&$(b, s, h)$&&\\ \midrule
   \textbf{Decoding} \\ \midrule
$XW_{Q}$, $XW_{K}$, $XW_{V}$ & $(b, 1, h)$ & $(b, 1, h)$ & $O(bh^2)$ & $O(2bh + h^2)$ \\ \midrule
  $QK^T$ & $(b, n, 1, d), (b, n, s, d)$ & $(b, n, s, 1)$ & $O(bsnd)$ & $O(bsn + bsnd + bnd)$ \\ 
  $PV$ & $(b, n, s, 1), (b, n, 1, d)$ & $(b, n, 1, d)$ & &  \\ \midrule
  $XW_{u}$, $XW_{g}$ & $(b, 1, h)$ & $(b, 1, i)$ & $O(bhi)$ & $O(b(h + i) + hi)$ \\
  $XW_{d}$ & $(b, 1, i)$ & $(b, 1, h)$ &  & \\\midrule
   \textbf{Parallel decoding} \\ \midrule
 $XW_{Q}$, $XW_{K}$, $XW_{V}$ & $(b, q, h)$ & $(b, q, h)$ & $O(bqh^2)$ & $O(2bqh + h^2)$ \\ \midrule
  $QK^T$ & $(b, n, q, d), (b, n, s, d)$ & $(b, n, s, q)$ & $O(bsqnd)$ & $O(bsqn + b(s+q)nd)$ \\ 
    $PV$ & $(b, n, s, q), (b, n, q, d)$ & $(b, n, q, d)$ & &  \\ \midrule
  $XW_{u}$, $XW_{g}$ & $(b, q, h)$ & $(b, q, i)$ & $O(bqhi)$ & $O(bq(h + i) + hi)$ \\
  $XW_{d}$ & $(b, q, i)$ & $(b, q, h)$ &  \\ \bottomrule
\end{tabular}
\label{tab:complexity}
\end{table}

Figures~\ref{fig:llama7b-roofline-a100}-\ref{fig:llama33b-roofline-a6000} show the benchmark of three categories of operators on different models (7/13/33B) under various settings. To evaluate each operator's performance and throughput, we chose the combination of settings including batch sizes from 1 to 64 in powers of 2 and sequence lengths from 128 to 8192 in powers of 2 \textcolor{black}{(49 settings for each operator)}. 
From all the figures, we observe that the datapoints of each operator in the prefill and decoding stages cluster at very similar positions across all GPUs and for various model sizes. 



During the prefill phase, increasing the batch size changes the FLOP/s of the attention matrix multiplications (see \texttt{`qk/pv init`}) but does not affect the Operational Intensity (refer to the vertical dashed arrow in Fig. 9). 
In contrast, increasing the sequence length impacts both FLOP/s and Operational Intensity in the prefill phase (refer to the diagonal dashed arrow in Fig. 9).
During the decoding phase, the attention matrix multiplications are significantly limited by memory bandwidth. Despite an increase in FLOP/s with changes in batch size and sequence length, the Operational Intensity remains nearly unchanged (see \texttt{`qk/pv ar`}). This indicates suboptimal resource utilization in the self-attention mechanism.

The linear layers in the prefill phase are mostly compute-bound (see \texttt{`qkv mlp init`} and \texttt{`up/gate/down init`}). During the decoding phase, the datapoints of the linear layer form a line with the same slope as the GPU’s memory bandwidth (see \texttt{`qkv mlp ar`} and \texttt{`up/gate/down ar`}). This indicates the linear layers in the decoding stage are also bounded by memory bandwidth. Increasing the batch size improves the achieved FLOP/s and Operational Intensity under memory bandwidth constraints through better parallelism. Note that linear layers only process the new token and are independent of sequence length (See `Decoding` section in Table~\ref{tab:complexity}).

\begin{figure}[h]
    \centering
    \includegraphics[width=0.8\textwidth]{llama7b-roofline-a100.pdf}
    \caption{The figure shows the relationship between FLOP/s and Operational Intensity for all benchmarked datapoints of Llama-7B operators on A100-80GB-PCIe. The dashed lines represent the HBM bandwidth limit (1,935GB/s) and the peak performance limit (312 TFLOP/s)~\cite{nvidia_a100_datasheet}. `\texttt{qkv mlp}' stands for the linear layers projecting hidden features to query/key/value features. `\texttt{up/gate/down}' stands for the linear layers following the attention block. `\texttt{qk/pv}' stands for the two steps of attention matrix multiplications. `\texttt{ar}' stands for the decoding (autoregressive) and `\texttt{init}' stands for the prefill phase.}
    \label{fig:llama7b-roofline-a100}
\end{figure}

\begin{figure}[h]
    \centering
    \includegraphics[width=0.8\textwidth]{llama13b-roofline-a100.pdf}
    \caption{Llama-13B operators on A100-80GB-PCIe.}
    \label{fig:llama13b-roofline-a100}
\end{figure}

\begin{figure}[h]
    \centering
    \includegraphics[width=0.8\textwidth]{llama33b-roofline-a100.pdf}
    \caption{Llama-33B operators on A100-80GB-PCIe.}
    \label{fig:llama33b-roofline-a100}
\end{figure}

\begin{figure}[h]
    \centering
    \includegraphics[width=0.8\textwidth]{llama7b-roofline-a40.pdf}
    \caption{Llama-7B operators on A40.}
    \label{fig:llama7b-roofline-a40}
\end{figure}


\begin{figure}[h]
    \centering
    \includegraphics[width=0.8\textwidth]{llama13b-roofline-a40.pdf}
    \caption{Llama-13B operators on A40.}
    \label{fig:llama13b-roofline-a40}
\end{figure}


\begin{figure}[h]
    \centering
    \includegraphics[width=0.8\textwidth]{llama33b-roofline-a40.pdf}
    \caption{Llama-33B operators on A40.}
    \label{fig:llama33b-roofline-a40}
\end{figure}



\begin{figure}[h]
    \centering
    \includegraphics[width=0.8\textwidth]{llama7b-roofline-a6000.pdf}
    \caption{Llama-7B operators on A6000.}
    \label{fig:llama7b-roofline-a6000}
\end{figure}

\begin{figure}[h]
    \centering
    \includegraphics[width=0.8\textwidth]{llama13b-roofline-a6000.pdf}
    \caption{Llama-13B operators on A6000.}
    \label{fig:llama13b-roofline-a6000}
\end{figure}

\begin{figure}[h]
    \centering
    \includegraphics[width=0.8\textwidth]{llama33b-roofline-a6000.pdf}
    \caption{Llama-33B operators on A6000.}
    \label{fig:llama33b-roofline-a6000}
\end{figure}

\clearpage



\subsection{FLOP/s vs. Operational Intensity Variations in \ours}

We investigate how Medusa can change Operational Intensity and elevate the FLOP/s.
We choose Llama 33B on A100-80GB-PCIe as the setting. 

First, we examine the attention matrix multiplication. Fig.~\ref{fig:llama33b-spec-bs16} and Table~\ref{tab:llama33b-spec-bs16} illustrate the effects of \ours while keeping the batch size fixed at 16. We observe increased FLOP/s and Operational Intensity as more candidate tokens are added (original decoding results are plotted as grey dots). This indicates that \ours can leverage additional candidate tokens to improve computational throughput. Compared to regular decoding, \ours achieves 44$\times$ FLOP/s and 41$\times$ Operational Intensity under the setting of batch size 16 and sequence length 1024 with 64 candidate tokens.
 Fig.~\ref{fig:llama33b-spec-seq1024} and Table~\ref{tab:llama33b-spec-seq1024} illustrate the effects of \ours decoding while keeping the sequence length fixed at 1024. Increasing the batch size does not improve Operational Intensity in this scenario. 

Next, we examine the linear layer, focusing on the up/gate/down linear layers. The results are shown in Fig.~\ref{fig:llama33b-spec--mlp-bsall} and Table~\ref{tab:llama33b-spec--mlp-bsall}. \textcolor{black}{Since the linear layers in the decoding phase only process the future tokens while the past tokens are cached, they are independent of the sequence length.} We vary the batch size to observe the effects. As \ours increases the number of candidate tokens with the increasing batch size, we observe a shift from a memory-bandwidth-bound region to a computation-bound region. This shift demonstrates how \ours can transition the performance characteristics of the linear layers from being limited by memory bandwidth to being limited by computational capacity.

\begin{figure}[h]
    \centering
    \includegraphics[width=0.8\textwidth]{llama33b-spec-bs16.pdf}
    \caption{FLOP/s vs. Operational Intensity of attention matrix multiplication with batch size 16.}
    \label{fig:llama33b-spec-bs16}
\end{figure}

\begin{figure}[h]
    \centering
    \includegraphics[width=0.8\textwidth]{llama33b-spec-seq1024.pdf}
    \caption{FLOP/s vs. Operational Intensity of attention matrix multiplication with sequence length 1024.}
    \label{fig:llama33b-spec-seq1024}
\end{figure}

\begin{figure}[h]
    \centering
    \includegraphics[width=0.8\textwidth]{llama33b-spec-mlp-bsall.pdf}
    \caption{FLOP/s vs. Operational Intensity of Linear layers.}
    \label{fig:llama33b-spec--mlp-bsall}
\end{figure}

\clearpage


\begin{table}[h]
\centering
\scriptsize
\begin{tabular}{lcccccccc}

\toprule
Seq. Length & \multicolumn{8}{c}{Number of Candidate Tokens} \\
\midrule
 & 1 & 16 & 32 & 48 & 64 & 80 & 96 & 112 \\
\midrule
128  & 0.54 \& 0.98 & 7.87 \& 12.8 & 14.73 \& 21.33 & 19.78 \& 27.43 & 25.25 \& 32.0 & 28.63 \& 35.56 & 32.58 \& 38.4 & 36.57 \& 40.73 \\
256  & 0.75 \& 0.99 & 11.2 \& 13.47 & 21.29 \& 23.27 & 28.69 \& 30.72 & 36.59 \& 36.57 & 41.2 \& 41.29 & 45.99 \& 45.18 & 52.33 \& 48.43 \\
512  & 1.02 \& 0.99 & 14.69 \& 13.84 & 27.47 \& 24.38 & 37.35 \& 32.68 & 47.09 \& 39.38 & 52.24 \& 44.91 & 59.55 \& 49.55 & 66.35 \& 53.49 \\
1024  & 1.24 \& 0.99 & 17.42 \& 14.03 & 32.15 \& 24.98 & 43.89 \& 33.76 & 54.8 \& 40.96 & 60.19 \& 46.97 & 68.28 \& 52.07 & 75.45 \& 56.44 \\
2048 & 1.39 \& 0.99 & 19.03 \& 14.12 & 35.05 \& 25.28 & 48.03 \& 34.32 & 59.66 \& 41.8 & 63.91 \& 48.08 & 72.83 \& 53.43 & 80.05 \& 58.04 \\
4096 & 1.48 \& 0.99 & 19.8 \& 14.17 & 36.59 \& 25.44 & 50.4 \& 34.61 & 62.29 \& 42.23 & 65.84 \& 48.65 & 74.86 \& 54.13 & 82.06 \& 58.87 \\
8192 & 1.53 \& 0.99 & 20.08 \& 14.2 & 36.89 \& 25.52 & 50.44 \& 34.76 & 62.11 \& 42.45 & 67.5 \& 48.94 & 76.97 \& 54.49 & 84.5 \& 59.3 \\
\bottomrule
\end{tabular}
\caption{
TFLOP/s \& Operational Intensity of attention matrix multiplication with batch size 16 for Llama 33B on an A100 80GB PCIe.}
\label{tab:llama33b-spec-bs16}
\end{table}


\begin{table}[h]
\centering
\scriptsize
\begin{tabular}{lcccccccc}
\toprule
Batch Size & \multicolumn{8}{c}{Number of Candidate Tokens} \\
\midrule
 & 1 & 16 & 32 & 48 & 64 & 80 & 96 & 112 \\
\midrule
1  & 0.37 \& 0.99 & 5.22 \& 14.03 & 10.15 \& 24.98 & 15.02 \& 33.76 & 19.79 \& 40.96 & 21.52 \& 46.97 & 25.65 \& 52.07 & 29.4 \& 56.44 \\
2  & 0.54 \& 0.99 & 8.25 \& 14.03 & 16.0 \& 24.98 & 21.62 \& 33.76 & 28.24 \& 40.96 & 31.84 \& 46.97 & 37.49 \& 52.07 & 43.04 \& 56.44 \\
4  & 0.75 \& 0.99 & 11.41 \& 14.03 & 21.97 \& 24.98 & 30.02 \& 33.76 & 38.71 \& 40.96 & 43.41 \& 46.97 & 50.06 \& 52.07 & 56.77 \& 56.44 \\
8  & 1.02 \& 0.99 & 14.78 \& 14.03 & 27.78 \& 24.98 & 38.09 \& 33.76 & 47.99 \& 40.96 & 53.32 \& 46.97 & 61.0 \& 52.07 & 68.11 \& 56.44 \\
16 & 1.24 \& 0.99 & 17.42 \& 14.03 & 32.15 \& 24.98 & 43.89 \& 33.76 & 54.8 \& 40.96 & 60.19 \& 46.97 & 68.28 \& 52.07 & 75.45 \& 56.44 \\
32 & 1.39 \& 0.99 & 18.89 \& 14.03 & 34.67 \& 24.98 & 47.57 \& 33.76 & 58.89 \& 40.96 & 63.61 \& 46.97 & 72.17 \& 52.07 & 79.21 \& 56.44 \\
64 & 1.48 \& 0.99 & 19.58 \& 14.03 & 35.87 \& 24.98 & 49.45 \& 33.76 & 61.13 \& 40.96 & 64.84 \& 46.97 & 73.73 \& 52.07 & 81.02 \& 56.44 \\

\bottomrule
\end{tabular}
\caption{
TFLOP/s \& Operational Intensity of attention matrix multiplication with sequence length 1024 for Llama 33B on an A100 80GB PCIe.}
\label{tab:llama33b-spec-seq1024}
\end{table}

\begin{table}[h]
\centering
\tiny
\begin{tabular}{lcccccccc}
\toprule
Batch Size & \multicolumn{8}{c}{Number of Candidate Tokens} \\
\midrule
 & 1 & 16 & 32 & 48 & 64 & 80 & 96 & 112 \\
\midrule
1  & 1.26 \& 1.0 & 19.95 \& 15.95 & 39.69 \& 31.79 & 58.4 \& 47.53 & 76.57 \& 63.17 & 94.4 \& 78.7 & 111.91 \& 94.14 & 128.64 \& 109.47 \\
2  & 2.51 \& 2.0 & 39.66 \& 31.79 & 76.53 \& 63.17 & 112.05 \& 94.14 & 145.73 \& 124.71 & 130.67 \& 154.89 & 129.1 \& 184.69 & 148.56 \& 214.12 \\
4  & 5.03 \& 4.0 & 76.44 \& 63.17 & 145.8 \& 124.71 & 128.85 \& 184.69 & 167.85 \& 243.17 & 201.19 \& 300.21 & 236.93 \& 355.85 & 195.91 \& 410.14 \\
8  & 10.06 \& 7.99 & 145.72 \& 124.71 & 168.26 \& 243.17 & 236.83 \& 355.85 & 221.11 \& 463.14 & 207.79 \& 565.44 & 236.95 \& 663.07 & 227.8 \& 756.36 \\
16 & 19.96 \& 15.95 & 168.35 \& 243.17 & 221.41 \& 463.14 & 237.5 \& 663.07 & 224.71 \& 845.59 & 232.49 \& 1012.87 & 241.12 \& 1166.74 & 229.25 \& 1308.76 \\
32 & 39.69 \& 31.79 & 221.74 \& 463.14 & 224.88 \& 845.59 & 241.33 \& 1166.74 & 239.02 \& 1440.25 & 245.83 \& 1675.97 & 243.55 \& 1881.24 & 240.33 \& 2061.59 \\
64 & 76.57 \& 63.17 & 225.19 \& 845.59 & 239.2 \& 1440.25 & 243.26 \& 1881.24 & 246.16 \& 2221.31 & 246.91 \& 2491.55 & 244.52 \& 2711.46 & 246.14 \& 2893.91 \\
\bottomrule
\end{tabular}
\caption{
TFLOP/s \& Operational Intensity of linear layers (up/gate/down) for Llama 33B on an A100 80GB PCIe.
}\label{tab:llama33b-spec--mlp-bsall}
\end{table}

\subsection{Predicting \ours Performance}

We further employ a straightforward analytical model \textcolor{black}{for} the acceleration rate. The ablation study results in Sec.~\ref{section:config of tree} indicate that the acceleration rate can be approximated by a simple logarithmic function. Using the results from Fig.~\ref{fig:sparse_acc}, we model the curve as $\texttt{acc\_rate} = 0.477 \log(\texttt{num\_candidate})$. We simulate the latency of one simplified block of the Llama-7B model (sequentially processing $XW_Q$, $XW_K$, $XW_V$, $QK^T$, $PV$, $XW_u$, $XW_g$, $XW_d$) by first fixing the batch size at 1 and the sequence length at 1024.
\textcolor{black}{
The candidate tokens are processed parallelly by constructing the tree attention described in Section~\ref{sec:tree_attention}. We omit the latency of the post-processing steps including verification and acceptance for \ours since they introduce marginal overhead.
}
Fig.~\ref{fig:llama7b-sim-bs1-seq1024} illustrates the simulated acceleration rate and speedup for different numbers of candidate tokens under these settings. As the number of candidate tokens increases, both the acceleration rate and speedup initially show improvements. However, beyond 64, the speedup starts to decline, indicating diminishing returns with further increases in candidate length. This aligns with the experimental results in Fig.~\ref{fig:sparse_speed} and suggests that there is an optimal range for the numbers of candidate tokens where \ours provides the most significant performance gains.

We plot the simulated speedup under different batch size settings with a fixed sequence length of 1024 in Fig.~\ref{fig:llama7b-sim-bs1-allbs}. The results indicate that when the batch size exceeds 32, the speedup decreases and may even have a negative effect. This occurs because the linear layers shift from being memory-bandwidth-bound to computationally bound.

We conduct another experiment using a batch size of 4 and different sequence lengths. As shown in Fig.~\ref{fig:llama7b-sim-allseq}, the optimal number of candidate tokens remains relatively consistent across different sequence lengths. However, as the sequence length increases, the overall performance decreases. This performance drop is primarily due to the overhead from attention matrix multiplication, while the linear layer computation remains constant \textcolor{black}{since the computation of linear layers is independent of the sequence length.}

Our simulations show that the optimal number of candidate tokens is key for model scaling with \ours, as benefits decrease beyond a certain range. Initially, increasing batch size improves performance through parallelism, but too large a batch size shifts linear layers from memory-bandwidth-bound to compute-bound, reducing speedup. Longer sequences increase attention matrix multiplication overhead, lowering performance, and emphasizing the need to optimize attention mechanisms. Effective model scaling requires balancing the number of candidate tokens, adjusting batch sizes to avoid compute-bound transitions, and enhancing attention mechanisms for longer sequences. These strategies ensure better resource utilization and higher performance, demonstrating the value of simulations in predicting performance and guiding acceleration strategy design.

\begin{figure}[h]
    \centering
    \includegraphics[width=0.8\textwidth]{llama7b-sim-bs1-seq1024.pdf}
    \caption{Simulated acceleration rate, speedup, and normalized latency ablation using different numbers of candidate tokens under the setting of batch size 1 and sequence length 1024 for Llama-7B on an A100 80GB PCIe.}
    \label{fig:llama7b-sim-bs1-seq1024}
\end{figure}

\begin{figure}[h]
    \centering
    \includegraphics[width=0.8\textwidth]{llama7b-sim-allbs.pdf}
    \caption{Simulated speedup with sequence length 1024 for Llama-7B.}
    \label{fig:llama7b-sim-bs1-allbs}
\end{figure}

\begin{figure}[h]
    \centering
    \includegraphics[width=0.8\textwidth]{llama7b-sim-allseq.pdf}
    \caption{Simulated speedup with batch size 4 for Llama-7B.}
    \label{fig:llama7b-sim-allseq}
\end{figure}
\end{document}