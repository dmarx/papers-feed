\begin{table}
   \caption{
   \textbf{Comparing different strategies for computing the thickness of the Frosting layer.} We compare the rendering performance (PSNR $\uparrow$) in synthetic and real scenes depending on how we compute and refine the thickness of the Frosting layer. Specifically, we first show that using an adaptive thickness improves performance over a constant thickness. Even though using a large constant thickness improves performance in scenes with very fuzzy materials like Shelly~\cite{wang-siggraphasia2023-adaptive-shells}, this lowers performance in scenes with flat surfaces and generates artifacts when editing the scene, as shown in Figure~\ref{fig:artifacts}. By contrast, our method adapts automatically to the type of surfaces. We also demonstrate that refining the thickness using the unconstrained Gaussians is necessary to achieve top performance.}
  \label{tab:ablation-proposal-intervals}
  \centering
  {\scriptsize
  \begin{tabular}{@{}lccccc@{}}
    \toprule
     \multicolumn{1}{c}{} & \multicolumn{1}{c}{\emph{\> Shelly \>}} & \multicolumn{3}{c}{\emph{Mip-NeRF~360}} \\
    \cmidrule(r){2-2} \cmidrule(r){3-5}
     & Average & Indoor & Outdoor & Average & \\
     %
    %\midrule
    %\multicolumn{4}{l}{\textbf{No mesh (except Frosting)}} \\
    \midrule
    Constant thickness (Small)  & 39.03 & \cellcolor{yellow!25}30.36 & 25.50 & \cellcolor{yellow!25}28.28 &  \\
    Constant thickness (Medium)  & \cellcolor{yellow!25}39.67 & 30.19 & \cellcolor{yellow!25}25.54 & 28.20 &  \\
    Constant thickness (Large) & \cellcolor{red!25}\textbf{40.00} & 30.06 & 25.48 & 28.10 &\\
    Using Regularized Gaussians only ($\delta^{\text{in/out}}=\epsilon^{\text{in/out}}$) & 39.34 & \cellcolor{orange!25}30.42 & \cellcolor{red!25}\textbf{25.57} & \cellcolor{orange!25}28.34 &  \\
     Using Regularized and Unconstrained Gaussians (Full method)  & \cellcolor{orange!25}39.84 & \cellcolor{red!25}\textbf{30.49} & \cellcolor{red!25}\textbf{25.57} & \cellcolor{red!25}\textbf{28.38} & \\
    \bottomrule
  \end{tabular}
  }
\end{table}