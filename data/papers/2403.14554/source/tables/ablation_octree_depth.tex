\begin{table}
   \caption{
   \textbf{Ablation for two different depth computation methods used for the octree in Poisson surface reconstruction.} We compare the rendering performance between using a predefined depth with a high value as in~\cite{guedon2023sugar} and our automatically computed depth. Our technique systematically selects an optimal depth depending on the complexity of the scene, avoiding artifacts in the mesh, and resulting in equivalent or better rendering performance with a much smaller average number of triangles.
   }
  \label{tab:ablation-octree-depth}
  \centering
  {\scriptsize
  \begin{tabular}{@{}lccccccccc@{}}
    \toprule
     \multicolumn{1}{c}{} & \multicolumn{4}{c}{\emph{NeRFSynthetic}} & \multicolumn{4}{c}{\emph{Shelly}} \\
     \cmidrule(r){2-5} \cmidrule(r){6-9}
      & PSNR $\uparrow$ & SSIM $\uparrow$ & LPIPS $\downarrow$ & $N_{tri}$ $\downarrow$  & PSNR $\uparrow$ & SSIM $\uparrow$ & LPIPS $\downarrow$ & $N_{tri}$ $\downarrow$ & \\
     %
    \midrule
    Predefined depth $D$ = 10~\cite{guedon2023sugar} & 31.63 & 0.959 & 0.041 & >1 M & \cellcolor{red!25}\textbf{39.85} & 0.975 & 0.035 & 939 K &\\
    Depth $\bar{D} \leq 10$ & \cellcolor{red!25}\textbf{33.03} & \cellcolor{red!25}\textbf{0.967} & \cellcolor{red!25}\textbf{0.029} & \cellcolor{red!25}\textbf{863 K} & 39.84 & \cellcolor{red!25}\textbf{0.977} & \cellcolor{red!25}\textbf{0.033} & \cellcolor{red!25}\textbf{203 K} &\\
    \bottomrule
  \end{tabular}
  }
\end{table}