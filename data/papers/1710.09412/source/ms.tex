\pdfoutput=1

\documentclass{article}
\usepackage{iclr2018}
\usepackage{times}
\usepackage{float}
\usepackage{hyperref}
\usepackage{url}
\usepackage{graphicx}
\usepackage{amsmath}
\usepackage{amsfonts}
\usepackage{subcaption}
\usepackage{listings}
\usepackage{lipsum}
\usepackage{microtype}
\usepackage{booktabs}
\usepackage{multirow}
\usepackage{siunitx}
\usepackage{hyphenat}

\usepackage{pifont}% http://ctan.org/pkg/pifont
\newcommand{\cmark}{\ding{51}}%
\newcommand{\xmark}{\ding{55}}%

\newcommand{\mixup}{\emph{mixup}}
\newcommand{\todo}[1]{{\color{red} TODO: {#1}}}
\DeclareMathOperator*{\E}{\mathbb{E}}
\newcommand*\samethanks[1][\value{footnote}]{\footnotemark[#1]}

\lstdefinestyle{mypython}{
  language=python,
  breaklines=true,
  basicstyle=\fontsize{8.5}{13}\selectfont\ttfamily,
  keywordstyle=\bfseries\color{green!40!black},
}

\iclrfinalcopy

\title{\mixup{}: Beyond Empirical Risk Minimization}

\author{%
Hongyi Zhang \\
MIT
\And
Moustapha Cisse, Yann N. Dauphin, David Lopez-Paz\thanks{Alphabetical order.} \\
FAIR
}

\begin{document}
    \maketitle
    \begin{abstract}
        Large deep neural networks are powerful, but exhibit undesirable
        behaviors such as memorization and sensitivity to adversarial examples.
        In this work, we propose \mixup{}, a simple learning principle to
        alleviate these issues. In essence, \mixup{} trains a neural network on
        convex combinations of pairs of examples and their labels.  By doing
        so, \mixup{} regularizes the neural network to favor simple linear
        behavior in-between training examples.  Our experiments on the
        ImageNet-2012, CIFAR-10, CIFAR-100, Google commands and UCI datasets
        show that \mixup{} improves the generalization of state-of-the-art
        neural network architectures.  We also find that \mixup{} reduces the
        memorization of corrupt labels, increases the robustness to adversarial
        examples, and stabilizes the training of generative adversarial
        networks.
    \end{abstract}
    \section{Introduction}
\label{sec:introduction}

\begin{wrapfigure}{r}{0.5\textwidth}
\vspace{-6mm}
\begin{center}
    \includegraphics[width=0.5\textwidth]{images/cover.pdf}
  \end{center}
  \vspace{-4mm}
  \caption{\textbf{Overview of \implname.} In training, we tune the singular values of the weight matrices to generate a set of ``expert'' vectors specializing in different tasks. In inference, a two-pass process is adopted where the first applies the expert and the second generates the answer.}
  \label{fig:cover}
  \vspace{-4mm}
\end{wrapfigure}

Self-adaptive large language models (LLMs) would represent a significant advancement in artificial intelligence, enabling real-time adaptation to various tasks and contexts.
While compositionality and scalability are crucial for effective adaptation, current LLM training methodologies fall short of achieving both these properties simultaneously.
Our research aims to present a solution to address these gaps.

In principle, the first step toward achieving self-adaptive LLMs can be realized through the development of specialized expert modules, each fine-tuned~\citep{kaplan2020scaling} via techniques such as low-rank adaptation (LoRA)~\citep{hu2021lora}. 
However, several challenges need to be addressed to make this approach both scalable and compositional: (1) multiple expert modules significantly increase the number of parameters; (2) expert modules are often prone to overfitting; and (3) flexible composition of these experts is still an open problem.

To overcome these limitations, we first propose \svdacro, a novel parameter-efficient fine-tuning (PEFT) method to obtain effective building blocks for self-adaptation.
\svdacro works by extracting and selectively tuning only the singular values within the model's weight matrices.
By focusing on this essential and principled parameterization, our approach mitigates the risk of overfitting, drastically reduces computational demands, and allows for inherent compositionality.

We then introduce our full \implname framework, which entails a two-pass inference mechanism to produce dynamically adapted weights targeted for the test-time conditions (Figure~\ref{fig:cover}).
We design three different adaptation strategies that can be used within \implname, which we show provide monotonic performance benefits with increasing access to the test-time conditions.
We evaluate \svdacro and the full \implname framework through extensive experiments across a diverse range of LLMs and tasks.
\svdacro outperforms traditional efficient fine-tuning methods like LoRA on domain-specific datasets with far fewer parameters. 
\implname further improves performance, even for out-of-distribution tasks like visual QA. 
Our analysis even shows that \implname allows the reuse of \svdacro experts across different LLMs. In summary, our key technical contributions are: 
\vspace{-2mm}
\begin{itemize}
\item The development of \implname as a pivotal self-adaptation framework for LLMs, providing a blueprint to adapt the behavior of LLMs from a growing set of pre-trained skills.
\item The introduction of \svdacro, a novel PEFT method trainable with RL on small datasets, producing compact expert vectors with inherent compositionality.
\item The implementation of three adaptation strategies, effectively dispatching \svdacro-trained experts with properties designed to cope with different deployment scenarios.
\end{itemize}

\vspace{-2mm}
    \section{Retrieval with Synchronised Graph Expansion}
\label{sec:graph_retrieval}

\def\Tqinit{\mathbf{T}_\mathbf{q}}


\begin{figure}[thbp]
  \includegraphics[width=\columnwidth]{figures/gear-sys-fig.pdf}
  \caption{\label{fig:system_diagram}System Architecture}
\end{figure}

% Start: Zhili --------------------------


Given an input query $\mathbf{q}$, let $\mathbf{C}_\mathbf{q}' = h^k_{\text{base}}\left( \mathbf{q}, {\mathbf{C}}\right )$  be a list of passages returned by the base retriever\footnote{The choice of a base retriever within our framework is flexible, without requiring any multi-hop capabilities.}.
Given this initially retrieved list of passages, $\mathbf{C}_\mathbf{q}'$, our goal is to derive relevant multi-hop contexts (passages) by retrieving a sub-graph of triples that interconnect their source passages. There are two challenges for materialising such sub-graph retrieval: \begin{inparaenum}[(i)]\item how to locate initial triples (i.e. starting nodes) $\Tqinit$, and \item how to expand the graph based on initial triples while reducing the search space\end{inparaenum}. The following sections address these challenges respectively, within \gear.



\subsection{Knowledge Synchronisation}
\label{subsection:knowledge_syncro}
\def\linkTriple{\texttt{tripleLink}}

We describe a knowledge \textbf{Sync}hronisation (\textbf{Sync}) process for locating initial nodes for graph expansion. We first employ an LLM to \texttt{read} $\mathbf{C}_\mathbf{q}'$ (see Appendix~\ref{subsec:online_retrieval_prompts}) and summarise knowledge triples that can support answering the current query $\mathbf{q}$, as defined:
\begin{align}
    \mathbf{T}_\mathbf{q}' = \texttt{read}\left (\mathbf{C}_\mathbf{q}', \mathbf{q}\right ).
    \label{eq:proximal_read}
\end{align}
$\mathbf{T}_\mathbf{q}'$ is a collection of triples to which we refer as \textit{proximal triples}. Initial nodes $\Tqinit$ for graph expansion can then be identified by linking each triple in $\mathbf{T}_\mathbf{q}'$ to a triple in $\mathbf{T}$, using the \linkTriple{} function:
\begin{align}
    \Tqinit =\left \{t_i | t_i = \linkTriple(t_i') ~ \forall t_i' \in \mathbf{T}_\mathbf{q}'\right \}.
\end{align}
The implementation of \linkTriple{} can vary. However, in this paper we consider it to be simply retrieving the most similar triple from $\mathbf{T}$.



\begin{algorithm}[ht]
\textbf{Input:} $\mathbf{q}$: query \\
\hspace*{3em} $b$: beam size \\
\hspace*{3em} $l$: maximum length \\
\hspace*{3em} $\mathrm{score}(\cdot, \cdot)$: scoring function \\
\hspace*{3em} $\{t_1, t_2, \ldots, t_n\}$: initial triples \\
\hspace*{3em} $\gamma$: hyperparameter for diversity


\begin{algorithmic}[1]
\State $B_0 \gets [\;]$
\For{$t \in \{t_1, t_2, ..., t_n\}$}
    \State $s \gets \mathrm{score}(\mathbf{q}, [t])$
    \State $B_0.\mathrm{add}(\langle s, [t] \rangle)$
\EndFor

\State $B_0 \gets \mathrm{top}(B_0, b)$


\For{$i \in \{1, \dots, l - 1\}$}
    \State $B \gets [\;]$
    
    \For{$\langle s, T \rangle \in B_{i-1}$}
        \State $V \gets [\;]$

        \For{$t \in \mathrm{get\_neighbours}(T.\mathrm{last}())$}
            \If{$\mathrm{exists}(t, B_{i-1})$}
                \State \textbf{continue}
            \EndIf
            
            \State $s' \gets s + \mathrm{score}(\mathbf{q}, T \circ t)$ ~ \texttt{\# concat} 
            \State $V.\mathrm{add}(\langle s', T \circ t \rangle)$
        \EndFor

        \State $\mathrm{sort}(V, \mathrm{descending})$

        \For{$n \in \{0, \dots, V.\mathrm{length()} - 1\}$}
            \State $\langle s', T \circ t \rangle \gets V[n]$
            \State $s' \gets s' \times e^{- \frac{\mathrm{min}(n, \gamma)}{\gamma}}$
            \State $B.\mathrm{add}(\langle s', T \circ t \rangle)$
        \EndFor
        
    \EndFor
    \State $B_i \gets \mathrm{top}(B, b)$
    
\EndFor

\State \Return $B_i$
\end{algorithmic}

\caption{Diverse Triple Beam Search}
\label{alg:beam_search}
\end{algorithm}

\subsection{Diverse Triple Beam Search}

We borrow the idea of constructing reasoning triple chains \cite{Fang2024} for expanding the graph, and present a retrieval algorithm: \textit{Diverse Triple Beam Search} (see Alg.~\ref{alg:beam_search}). 

We maintain top-$b$ sequences (beams) of triples and the scores at each step are determined by a scoring function. In this paper, we focus on leveraging a dense embedding model to compute the cosine similarity between embeddings of the query and a candidate sequence of triples, leaving other implementations of the scoring function for future work (see Section~\ref{sec:limitations}).

Considering all possible triple extensions at each step, in a Viterbi decoding fashion, would be intractable due to the size of $\mathbf{T}$. Consequently, we define the neighbourhood of a triple as the set of triples with shared head or tail entities (i.e. $\mathrm{get\_neighbours}$ in Alg.~\ref{alg:beam_search}). During each expansion step, we only consider neighbours of the last triple in the sequence, and avoid selecting previously visited triples (i.e. $\mathrm{exists}$ in Alg.~\ref{alg:beam_search}) to further reduce the search space.

While regular beam search can reduce the search space, it is prone to producing high-likelihood sequences that differ only slightly from one another \cite{Ippolito2019, Vijayakumar2018}. Our algorithm increases the diversity across beams to improve the recall for retrieval. In detail, for each beam, we sort candidate sequences extended from that beam in descending order, and weight their scores based on their relative positions. Candidate sequences that are ranked lower, within a beam, will receive smaller weights. Consequently, the resulting top-$b$ beams at each step are less likely to share the same starting sequence. 

The top-$b$ returned sequences are flattened in a breadth-first order. Each triple in the resulting list is then mapped to its source passage. This alignment between triples and passages is described in more detail in Section~\ref{sec:preliminaries}. Let $\widetilde{\mathbf{C}}_\mathbf{q}$ be the list of unique passages after alignment. The output of our graph expansion is then given by the Reciprocal Rank Fusion (RRF) \cite{Cormack2009} of $\widetilde{\mathbf{C}}_\mathbf{q}$ and the initial $\mathbf{C}_\mathbf{q}'$ list of passages :
\begin{align}
    \mathbf{C}_{\mathbf{q}} = \mathrm{RRF}\left(\widetilde{\mathbf{C}}_\mathbf{q}, \mathbf{C}_\mathbf{q}'\right ).
\end{align}
We refer to this graph-based method of retrieving relevant passages as \textbf{Sync}ronised \textbf{G}raph \textbf{E}xpansion (\textbf{SyncGE}).


\section{Multi-step Extension}


While SyncGE can enhance a base retriever with multi-hop context, some queries inherently require multiple steps to gather all necessary evidence. We materialise \gear by incorporating an agent with multi-turn capabilities, capable of interacting with the graph-retriever described above. We focus on:
\begin{itemize}
\item maintaining a gist memory of proximal knowledge obtained throughout the different steps 
\item incorporating a similar synchronisation process 
that summarises retrieved passages in proximal triples to be stored in this multi-turn gist memory
\item determining if additional steps are needed for answering the original input question
\end{itemize}
%
Within this multi-turn setting, the original input question $\mathbf{q}$ is iteratively decomposed into simpler queries: $\mathbf{q}^{(1)}, \ldots, \mathbf{q}^{(n)}$, where $\mathbf{q}^{(1)} = \mathbf{q}$ and $n \in \mathbb{N}$ represents the number of the current step.
For each query $\mathbf{q}^{(n)}$, we use the graph retrieval method introduced in Section~\ref{sec:graph_retrieval} in order to retrieve relevant passages $\mathbf{C}_{\mathbf{q}^{(n)}}$.



\subsection{Gist Memory Constructor}
To facilitate the multi-step capabilities of our agent, we introduce a \textit{gist memory}, $\mathcal{G}^{(n)}$, which is used for storing knowledge as an array of proximal triples. At the beginning of the first iteration, the gist memory is empty. During the $n$-th iteration, similar to the knowledge synchronisation module explained in Section~\ref{subsection:knowledge_syncro}, we employ an LLM to read a collection of retrieved paragraphs $\mathbf{C}_{\mathbf{q}^{(n)}}$ and summarise their content with proximal triples:

\begin{align}
\mathbf{T}_{\mathbf{q}^{(n)}}^{\mathcal{G}} = 
\begin{cases} 
    \texttt{read}\left(\mathbf{C}_{\mathbf{q}^{(n)}}, \mathbf{q} \right), & \text{if } n = 1 \\
    \texttt{read}\left(\mathbf{C}_{\mathbf{q}^{(n)}}, \mathbf{q}\textcolor{blue}{, \mathcal{G}^{(n-1)}} \right), & \text{if } n \geq 2
\end{cases}
\label{eq:proximal_read_agent}
\end{align}


Apart from the first iteration where Eq.~\ref{eq:proximal_read} and ~\ref{eq:proximal_read_agent} are identical, the inclusion of the memory in the \texttt{read} operation differentiates the construction of proximal triples produced at the subsequent steps compared to the ones from Eq.~\ref{eq:proximal_read}. $\mathcal{G}^{(n)}$ maintains the aggregated content of proximal triples s.t. 
\begin{align}
\mathcal{G}^{(n)} = \left[ \mathbf{T}_{\mathbf{q}^{(1)}}^{\mathcal{G}}  \circ \cdots \circ \mathbf{T}_{\mathbf{q}^{(n)}}^{\mathcal{G}} \right],
\end{align}where $\circ$ defines the concatenation operation. The triple memory serves as a concise representation of all the accumulated evidence, up to the $n$-th step. 

We believe the process introduced by the \texttt{read} step along with the information storage paradigm served by the gist memory, aligns well with the communication between the hippocampus and neocortex. The combination of the two establishes the synergetic behaviour between our graph retriever and the LLM that we seek to achieve within \gear.



\subsection{Reasoning for Termination}
After $\mathcal{G}^{(n)}$ is updated, we check the sufficiency of the accumulated evidence, within it, for answering the original question. This is achieved with the following LLM reasoning step:
\begin{align}
\mathbf{a}^{(n)}, \mathbf{r}^{(n)}   = \texttt{reason}(\mathcal{G}^{(n)}, \mathbf{q}),
\end{align}
% We can also call it 'sufficiency' instead of 'answerability'. I do not really have a preference.
where $\mathbf{a}^{(n)}$ denotes the query's answerability given the available evidence in $\mathcal{G}^{(n)}$, and $\mathbf{r}^{(n)}$ represents the reasoning behind this determination. When the query is deemed answerable, the system concludes its iterative process.



\subsection{Query Re-writing}
The query re-writing process leverages an LLM that incorporates three key inputs: the original query $\mathbf{q}$, the accumulated memory, and crucially, the reasoning output $\mathbf{r}^{(n)}$ from the previous step. This process can be formally expressed as:
\begin{align}
\mathbf{q}^{(n+1)} = \texttt{rewrite}\left (\mathcal{G}^{(n)}, \mathbf{q}, \mathbf{r}^{(n)} \right),
\end{align}
where $\mathbf{q}^{(n+1)}$ represents the updated query, which serves as input for the retriever in the next iteration.\\
\subsection{After Termination}
\gear aims to return a single ranked list of passages. Given the final gist memory $\mathcal{G}^{(n)}$ upon termination, we link each proximal triple in $\mathcal{G}^{(n)}$ to a list of passages as follows:
\begin{align}
    \mathbf{C}_{t_j} = \texttt{passageLink}\left(t_j, k\right),
\end{align}
where $j \in \left \{1, \dots, \vert\mathcal{G}^{(n)}\vert \right \}$. Similar to \texttt{tripleLink}, \texttt{passageLink} is implemented by retrieving passages with a triple as the query (see Appendix~\ref{appendixpara:passage_link}). The final list of passages returned by \gear is the RRF of the resulting linked passages and passages retrieved across iterations:
\begin{align}
\mathbf{C}_\mathbf{q}^{(n)} = \mathrm{RRF}\big(&\mathbf{C}_{t_1}, \ldots,\mathbf{C}_{t_{\vert\mathcal{G}^{(n)}\vert}}, \nonumber\\
    &\mathbf{C}_{\mathbf{q}^{(1)}}, \ldots, \mathbf{C}_{\mathbf{q}^{(n)}} \big).
\end{align}

All relevant prompts for the \texttt{read}, \texttt{reason} and \texttt{rewrite} steps are provided in Appendix~\ref{subsec:online_retrieval_prompts}.

    \section{Experiments}
\label{sec:experiments}
\begin{figure}[h]
\centering
\includegraphics[width=\textwidth]{figures/train_nll_softmax_vs_sigmoid_v4.pdf}
\caption{Train losses comparing $\sigmoidattn$ with $\softmaxattn$.}
\label{fig:summary_nll}
\end{figure}
To empirically validate $\sigmoidattn$, we evaluate across several domains: supervised image classification using vision transformers \citep{DBLP:conf/iclr/DosovitskiyB0WZ21}, self-supervised image representation learning with SimCLR \citep{DBLP:conf/icml/ChenK0H20, DBLP:conf/icml/ZhaiLLBR0GS23}, Bootstrap Your Own Latent (BYOL) \citep{DBLP:conf/nips/GrillSATRBDPGAP20, DBLP:conf/nips/BusbridgeRALDCW23} and Masked AutoEncoders (MAE) \citep{DBLP:conf/cvpr/HeCXLDG22} as well as automatic speech recognition (ASR) \citep{synnaeve2019end,conformer} and auto-regressive language modeling (LM) \citep{DBLP:conf/nips/BrownMRSKDNSSAA20}. We also validate sequence length generalization on TED-LIUM v3~\citep{hernandez2018ted} for ASR and in small scale synthetic experiments in \cref{sec:a_se_pair_repeat_prob}.
Across all these domains and algorithms, we demonstrate that $\sigmoidattn$ matches the performance of $\softmaxattn$ (\cref{fig:summary_nll,fig:test_top1_results}), while offering training and inference speed-ups as highlighted in \cref{sec:FlashSigmoidHardwareAwareImplementation}. Empirically we make the following observations:
\begin{enumerate}[itemsep=0pt,leftmargin=*]
    \item $\sigmoidattn$ is effective for vision tasks without a bias (except MAE), but relies on LayerScale to match the performance of the baseline $\softmaxattn$ (\cref{fig:imagenet_top_1_ablations}-a) in a hyper-parameter free manner.\footnote{\Cref{sec:layerscale_free_sigmoid} demonstrates that supervised vision tasks using $\sigmoidattn$ without LayerScale can match baseline $\softmaxattn$ performance by relying on \emph{learnable} scalar bias and temperature: $\{b, t\} \in \mathbb{R}$.} All results presented for $\softmaxattn$ also fairly add LayerScale unless specified.
    \item LM and ASR are sensitive to the initial norm $|| \sigma(\mQ \mK^T/\sqrt{d_{qk}}) \mV ||$. Modulation is required via (a) relative positional embeddings like ALiBi \citep{DBLP:conf/iclr/PressSL22}, which reduces the initial attention norm by shifting logit mass to the zero regime under $\sigmoidattn$, (b) appropriate initialization of $b$ to achieve the same effect -- enabling usage of any positional embedding.
\end{enumerate}

\begin{figure}[htbp]
    \centering
    \begin{minipage}{0.48\textwidth}
        \centering
        \includegraphics[width=\textwidth]{figures/attn_norm_seed1000001_softmax_rope_vs_softmax_alibi_vs_sigmoid_sincos.png}    
        \captionsetup{justification=centering}
        \caption{$\sigmoidattn$ with SinCos.}
        \label{fig:rope_vs_sincos}
    \end{minipage}\hfill
    \begin{minipage}{0.48\textwidth}
        \centering        
        \includegraphics[width=\textwidth]{figures/attn_norm_seed1000001_softmax_rope_vs_softmax_alibi_vs_sigmoid_rope.png}
        \captionsetup{justification=centering}
        \caption{$\sigmoidattn$ with RoPE.}
        \label{fig:rope_vs_rope}
    \end{minipage}
    \hfill
    \begin{minipage}{0.48\textwidth}
        \centering
        \includegraphics[width=\textwidth]{figures/attn_norm_seed1000001_softmax_rope_vs_softmax_alibi_vs_sigmoid_alibi.png}
        \captionsetup{justification=centering}
        \caption{$\sigmoidattn$ with ALiBi.}
        \label{fig:rope_vs_alibi}
    \end{minipage}\hfill
    \begin{minipage}{0.48\textwidth}
        \centering        
        \includegraphics[width=\textwidth]{figures/attn_norm_seed1000001_softmax_rope_vs_softmax_alibi_vs_sigmoid_rope_b=-10.png}
        \captionsetup{justification=centering}
        \caption{$\sigmoidattn$ with RoPE, $b=-10$.}
        \label{fig:rope_vs_rope_b-10}
    \end{minipage}  
    \vspace{-0.4cm}
\end{figure}

\subsection{Ablations}
\label{sec:ablations}
We begin with ablations to dissect the benefits of each of our introduced components. To gain intuition about $\sigmoidattn$, we developed a research-friendly auto-regressive (AR) LM training framework to measure all components of attention and validate the effects of LayerScale, LayerNorm applied to Q and K (QK norm), different positional embedding techniques, and initialization values for $b$.
\begin{figure}[h]
    \centering
    \begin{minipage}[t]{0.48\textwidth}
        \centering
        \includegraphics[width=\textwidth]{figures/lines=activation-cols=layerscale_with_log_n_or_max3std.pdf} 
        \caption{LR sensitivity LayerScale ablation.}
        \label{fig:layerscale_ablation}
    \end{minipage}%
    \hfill
    \begin{minipage}[t]{0.48\textwidth}
        \centering
        \includegraphics[width=\textwidth]{figures/lines=activation-cols=qknorm_with_log_n_or_max3std.pdf}
        \caption{LR sensitivity QK norm ablation.}
        \label{fig:qk_norm_ablation}
    \end{minipage}
\end{figure}
\begin{figure}[h]
    \centering
    \vspace{-0.2cm}
    \includegraphics[width=\textwidth]{figures/imagenet_ablations_top1.pdf}
    \caption{ImageNet1k ViT-B/16 classification. (a) $\sigmoidattn$ is robust without QK norm (+LayerScale, -QKNorm). Removing LayerScale reduces accuracy by 1.0\% (-LayerScale, +/-QKNorm). $n^{-\alpha}$ normalization \citep{wortsman2023replacing} underperforms without LayerScale. (b) $\sigmoidattn$ multi-query attention (MQA) \citep{DBLP:journals/corr/abs-1911-02150} with one head matches multi-head attention (MHA). (c) Sigmoid with LayerScale and QK norm performs comparably to other activations, except TanH. ReLU$^2$ \citep{DBLP:conf/icml/HuaDLL22} underperforms without LayerScale and QK norm.}
    \label{fig:imagenet_top_1_ablations}
    \vspace{-0.4cm}
\end{figure}
\paragraph{Mitigating Large Attention Norms} We train a single layer AR transformer block (E=3072, D\_FF=12288) on the realnews split of C4 \citep{DBLP:journals/jmlr/RaffelSRLNMZLL20}. We train for $2^{16}$ steps using a batch size of 6 and max sequence length of 4096 using a single cycle cosine learning rate (LR) schedule without weight decay. $\sigmoidattn$ initially underperformed $\softmaxattn$ when using absolute sinusoidal (SinCos) (\cref{fig:rope_vs_sincos}) or relative (\cref{fig:rope_vs_rope}) positional embeddings (PE), which we attribute to high initial attention Frobenius norms, $\lVert \sigma(\mQ \mK^T / \sqrt{d}) \mV \rVert$. A corresponding evolution of the attention distribution and sparsity can be seen in Appendix \cref{fig:attn_evolve} and \cref{fig:attn_metric_evolve} on a synthetic task.
To address these larger attention norms, we propose: (a) using ALiBi \citep{DBLP:conf/iclr/PressSL22} whose relative bias moves initial attention logit mass to the zero region under the sigmoid activation, producing equivalent train negative log-likelihoods (\cref{fig:rope_vs_alibi}); or (b) set the attention logit bias $b$ to a negative offset proportional to the sequence length, $b \propto -\ln n$ (see \cref{sec:attn_bias_ablation} for an ablation on $b$). This enables the usage of other PE techniques like RoPE~\citep{DBLP:journals/ijon/SuALPBL24} (\cref{fig:rope_vs_rope_b-10}). 
\paragraph{LayerScale} To validate the need for LayerScale, we follow \citet{DBLP:journals/corr/abs-2309-14322} to quantify the impact on stability.
All models are trained with RoPE with $b \propto -\ln n$, using AdamW  \citep{loshchilov2017decoupled} on the 
realnews split of C4 
with $(\beta_1,\beta_2)=(0.9, 0.95)$, $\eps=10^{-8}$,  $wd=0$, 
batch size 24, maximum token sequence length of 512 from the T5 tokenizer \citep{DBLP:journals/jmlr/RaffelSRLNMZLL20}, cosine LR schedule of $2^{14}$ steps including a linear warmup of $2^{10}$ steps. 
Models have 
$n_{\text{heads}}=\kappa$,
$n_{\text{layers}}=2\times \kappa$,
$d_{\text{model}}=64\times \kappa$ and
$d_{\text{feed-forward}}=256\times\kappa$
for a scaling value $\kappa\in\{1,2,4,8,16\}$
leading to models with $\{2.2, 4.9,15.0,67.0,440.0\}M$ trainable non-embedding parameters.
Following \citet{DBLP:journals/corr/abs-2309-14322},
we sweep learning rates
$\eta\in \{3\times 10^{-4}, 1\times 10^{-3}, 3\times 10^{-3}, 1\times 10^{-2}, 3\times 10^{-2}, 1\times 10^{-1}, 3\times 10^{-1}\}$.
LR sensitivity is defined as 
$\mathbb E_{\eta\in[a,b]}\left[\min(\ell(\mathcal A(\eta)),\ell_0)-\ell^*\right]$
where $\ell(\mathcal A(\eta))$ is the loss achieved by the learning algorithm $\mathcal A$ with LR $\eta$,
$\ell_0$ is the loss at initialization, and
$\ell^*$ is the loss achieved by the best LR.
LayerScale is initialized at $10^{-4}$. 
Unlike vision tasks, where LayerScale \emph{improves performance} (\cref{fig:imagenet_top_1_ablations}-a), in LM, we observe that $\softmaxattn$ slightly benefits from LayerScale, while the performance of $\sigmoidattn$ remains largely unaffected.
\paragraph{Stability with QK Norm} \Cref{thm:regularity} indicates that the Jacobian of $\sigmoidattn$ has favorable properties compared to $\softmaxattn$. We explore this by repeating the analysis of \citet{DBLP:journals/corr/abs-2309-14322}, as described in the LayerScale analysis, to investigate the impact of QK norm \citep{DBLP:conf/icml/0001DMPHGSCGAJB23}. For language modeling, both $\sigmoidattn$ and $\softmaxattn$ exhibit sensitivity to learning rate changes without QK norm. However, incorporating QK norm significantly stabilizes performance (\cref{fig:qk_norm_ablation}). In vision tasks, $\sigmoidattn$ demonstrates robustness with and without QK norm (\cref{fig:imagenet_top_1_ablations}-a) and without the need for $n^{-\alpha}$ normalization from \citet{wortsman2023replacing}.\footnote{We ablate multiplicative sequence length scaling in more detail in \cref{sec:appendix_normalization}.}
\paragraph{Multi-query attention (MQA)} In \cref{fig:imagenet_top_1_ablations}-b we explore MQA \citep{DBLP:journals/corr/abs-1911-02150} for vision using only one head for $\{ \mK, \mV \}$. We find that both $\sigmoidattn$ and $\softmaxattn$ perform equally well with or without multiple heads even at the small scale of ViT-B/16.
\paragraph{Activation Function Ablations} As in \citet{wortsman2023replacing}, various activation functions, when combined with LayerScale and QK norm, perform equally well for vision tasks (\cref{fig:imagenet_top_1_ablations}-c). However, for sequence-critical tasks like ASR, activation functions such as ReLU pose instabilities and underperform. In the same figure, we also compare to the ReLU$^2$ proposal from \citet{DBLP:conf/icml/HuaDLL22} and find that it underperforms without LayerScale and QK norm.
\subsection{Supervised Image Classification}
\label{sec:supervised_image_classification}
Vision transformers \citep{DBLP:conf/iclr/DosovitskiyB0WZ21} extend transformers  \citep{DBLP:conf/nips/VaswaniSPUJGKP17} to treat $K \times K$ image grids as disparate tokens. All tokens are refined through sequential layers of self-attention, pooled using a CLS token or global average pooling layer, and optimized using the negative log likelihood, $\ln p(\vy|\vx)$. We train ViT-B/16 models using $\mathbb{R}^{224 \times 224 \times 3}$ images for 300 epochs using the recipe provided in \cref{sec:appendix_vision_hyperparams}. We use the same set of training hyper-parameters for both $\softmaxattn$ and $\sigmoidattn$, changing only the activation function between trials. The train negative log-likelihood is reported in \cref{fig:summary_nll} and the test top-1\% is reported in \cref{fig:test_top1_results}. We find that $\sigmoidattn$ matches both the training dynamics and the evaluation performance of $\softmaxattn$.
\subsection{Self-Supervised Image Representation Learning}
\label{sec:ssl}
Self-supervised representation learning (SSL) exploits vast quantities of unlabeled data to learn semantic representations based on inductive biases such as augmentation invariance (SimCLR \cite{DBLP:conf/icml/ChenK0H20}, BYOL \citep{DBLP:conf/nips/GrillSATRBDPGAP20}) or reconstruction from compressed representations (MAE \citep{DBLP:conf/cvpr/HeCXLDG22}). We employ vision transformer training recipes from \cite{DBLP:conf/icml/ZhaiLLBR0GS23} and \cite{DBLP:conf/nips/BusbridgeRALDCW23} (\cref{sec:appendix_vision_hyperparams}) for SimCLR and BYOL. As with supervised learning, we use the same set of training hyper-parameters for both $\softmaxattn$ and $\sigmoidattn$, changing only the activation function between trials. \Cref{fig:summary_nll} reports the train losses, and \cref{fig:test_top1_results} highlights the linear probe and finetuned test top-1\%. Despite the diverse training objectives in SSL, $\sigmoidattn$ matches $\softmaxattn$ while improving training and inference throughput (\cref{sec:FlashSigmoidHardwareAwareImplementation}).
\subsection{Automatic Speech Recognition (ASR)}
\label{sec:asr}
\begin{table}[t!]
\centering
\caption{Word error rate (\%) on LibriSpeech test sets and TED-LIUM v3~\citep{hernandez2018ted} (``TED'', joint validation and test sets split according to  duration) for transformer (255M params) with either $\softmaxattn$ or $\sigmoidattn$ (LayerScale and QK norm are used with $b=-\log n$) trained on LibriSpeech 960h data (mean duration is 10-15s). Hyper-parameters are in~\cref{sec:asr_hps}.}
\label{tab:asr-results}
\begin{center}
\begin{scriptsize}
\begin{sc}
\resizebox{\columnwidth}{!}{%
\begin{tabular}{lc|rr|rrrr}
\toprule
 attn & PE & test-clean & test-other & ted 0-10s & ted 10-20s & ted 20-30s & ted 30s+  \\
\midrule 
softmax & \multirow{7}{*}{CAPE} & 2.3 & 5.7 & 12.4 & 10.5 & 11.9 & 9.1 \\
 sigmoid &  & 2.4 & 5.5 & 12.4 & 10.3 & 12.3 & 9.7 \\
 \,\,\,\, - QK norm &  & \multicolumn{6}{c}{unstable, gradient norm and loss spikes} \\
 \,\,\,\, - LayerScale &  & 2.5 & 6.1 & 13.6 & 11.5 & 13.4 & 8.9 \\
 sigmoid ($b=-10$, learnable) &  & 2.3 & 5.5 & 12.1 & 10.5 & 13.0 & 9.3 \\
 sigmoid ($b=-5$ in $Q$, learnable) &  & 2.3 & 5.4 & 12.2 & 10.8 & 12.4 & 9.9 \\
 \,\,\,\, - QK norm &  & \multicolumn{6}{c}{unstable, gradient norm and loss spikes} \\

\midrule
softmax & \multirow{5}{*}{RoPE} & 2.2 & 5.5 & 12.7 & 10.6 & 12.8 & 9.5 \\
 sigmoid &  & 2.3 & 5.4 & 12.3 & 10.1 & 12.3 & 8.6 \\
 sigmoid ($b=-10$, learnable) &  & 2.2 & 5.2 & 12.4 & 10.5 & 12.3 & 21.8 \\
 \,\,\,\, + $\alpha=1$ &  & 2.7 & 6.6 & 14.1 & 12.0 & 14.5 & 14.9 \\
 sigmoid ($b=-5$ in $Q$, learnable) &  & \multicolumn{6}{c}{unstable, gradient norm and loss spikes} \\
\midrule
 softmax & \multirow{5}{*}{ALiBi} & 2.2 & 5.4 & 12.3 & 10.7 & 12.1 & 8.6 \\
 sigmoid &  & 2.3 & 5.1 & 12.3 & 10.5 & 12.6 & 9.1 \\
 sigmoid ($b=-10$, learnable) &  & 2.2 & 5.2 & 12.4 & 10.4 & 11.7 & 9.1 \\
 \,\, + $\alpha=1$ &  & 2.6 & 6.6 & 13.9 & 11.9 & 14.2 & 8.6 \\
 sigmoid ($b=-5$ in $Q$, learnable) &  & 2.2 & 5.2 & 12.1 & 10.4 & 12.0 & 8.2 \\
\bottomrule
\vspace{-0.4cm}
\end{tabular}
}
\end{sc}
\end{scriptsize}
\end{center}
\end{table}
We benchmark ASR using LibriSpeech data \citep{DBLP:conf/icassp/PanayotovCPK15} on 100h and 960h settings of paired speech and text transcriptions. Our PyTorch implementations of encoder-based vanilla transformer~\citep{synnaeve2019end} and conformer \citep{DBLP:conf/interspeech/GulatiQCPZYHWZW20} are trained with Connectionist Temporal Classification (CTC) \citep{DBLP:conf/icml/GravesFGS06} w/ BF16 mixed precision, w/o QK norm and w/o LayerScale. After extensively tuning $\softmaxattn$ baselines, we switch to $\sigmoidattn$ per \cref{eq:sigmoid_attn} without any other changes. We investigate the effects of post/pre-LayerNorm, model depth, optimizer type, small data regime, and connection to local attention, with details in~\cref{sec:asr_hps}.

Our main findings are: i) CAPE~\citep{DBLP:conf/nips/LikhomanenkoXSC21} PE is the most unstable for $\sigmoidattn$; ii) post-LayerNorm models with $\softmaxattn$ are hard to match with stable $\sigmoidattn$; iii) w/o QK norm $\sigmoidattn$ is unstable and significant spikes happen in both gradient norms and training loss; iv) LayerScale is needed for generalization; v) learnable bias $b=-10$ gives no loss and gradient norms spikes while matching the $\softmaxattn$ (which does not benefit from the improved throughput of \textsc{FlashSigmoid}); vi) adding a learnable bias, $b=-5$, to $Q$ instead of the attention logits also solves the initial large attention norms for CAPE and ALiBi but not for RoPE; vii) $b=-\log n$ gives rare (2-5 times) marginal gradient norms spikes with smooth loss while matching $\softmaxattn$.


\Cref{tab:asr-results} shows the main result for pre-LayerNorm  transformers with CAPE, RoPE, and ALiBi, where $\sigmoidattn$ uses LayerScale, QK norm, $b=-\log n$, and no sequence normalization. The bias is ablated with learnable bias (one per layer) in attention or $Q$ with or without sequence normalization. $\sigmoidattn$ is stabilized with bias while matching $\softmaxattn$, and $b=-\log n$ works well. In most cases, bias allows generalization to longer sequences without sequence normalization, except for RoPE where it helps for longer sequences but hurts overall performance.









\subsection{Autoregressive Large Language Modeling}
\label{sec:llm}

\newcolumntype{R}[2]{%
    >{\adjustbox{angle=#1,lap=\width-(#2)}\bgroup}%
    l%
    <{\egroup}%
}
\newcommand*\rotdiag{\multicolumn{1}{R{30}{1em}}}%

\begin{table}[t]
\centering
\caption{1B LLM English evaluation.}
\label{tab:lm_results}
\begin{sc}
\begin{scriptsize}
\bgroup
\setlength{\tabcolsep}{.35em}
\begin{tabular}{@{}lllllllllllllll@{}}
\toprule
Model   & \makecell{Seq.\\Len.} & \makecell{ARC\\Easy} & \makecell{ARC\\Challenge} & \makecell{Hella-\\swag} & Piqa & Sciq & \makecell{Wino-\\grande} & \makecell{Lambada\\OpenAI} & \makecell{TriviaQA\\(1-shot)} & \makecell{WebQS\\(1-shot)} & AVG & \makecell{Step\\time (s)} \\ \midrule
Softmax (ALiBi) & 2k & 62.2       &     26.8           &    42.4       &  59.0    &   72.3   &     88.1       &     58.4           &      19.9             &    15.4            &    49.4   & 0.38   \\
Sigmoid (ALiBi) & 2k &  62.8       &      28.8         &    42.5       &  59.7    &   70.3   &     88.6       &      59.7          &       19.1            &   13.8             &       49.5  & 0.34   \\
\midrule
Softmax (RoPE) & 4k & 63.3       &     29.3           &    43.3       &  58.1    &   71.3   &     86.9       &     58.8           &  20.4             &    15.6            &    49.7   & 0.84   \\
Softmax (ALiBi) & 4k & 62.6       &     27.7           &    42.4       &  58.6    &   71.1   &     88.2       &     58.6           &      18.9             &    14.7            &    49.2   & 0.84   \\
Sigmoid (ALiBi) & 4k &  60.5       &      27.3         &    41.3       &  57.8    &   70.5   &     87.0       &      57.6          &       18.9            &   12.6             &       48.2  & 0.67   \\ \bottomrule
\end{tabular}
\egroup
\end{scriptsize}
\end{sc}
\vspace{-0.4cm}
\end{table}

We initially iterated at the 85M scale, as it served as a proxy for larger scale training. Our findings show that: i) attention bias is required for stability, which can be learnable, but setting it to $-\log(n)$, where $n$ is the maximum training sequence length of 4096, works well and is faster; ii) RoPE is more challenging to stabilize; iii) the final setting exhibits smooth loss curves, but still shows gradient norm fluctuations. We then turn our attention to validating $\sigmoidattn$ at scale.

We train a 1B language model using the Llama2 \citep{touvron2023llama} recipe with ALiBi instead of RoPE positional embedding, and the RedPajama \citep{together2023redpajama} dataset (see \cref{sec:llm_appendix}). At sequence length 4096, $\sigmoidattn$ achieves a \textbf{1.23}$\mathbf{\times}$ step-time improvement over $\softmaxattn$ in JAX without \textsc{FlashAttention} (\cref{tab:lm_results}). All LLMs are trained using the AXLearn framework, which include the recipe and $\sigmoidattn$ implementation.\footnote{https://github.com/apple/axlearn}

$\softmaxattn$ and $\sigmoidattn$ have matching train and validation NLL at 85M (\cref{fig:85m_4k_nll}) and at 1B scale when using 2048 sequence length (\cref{fig:summary_nll}). However, a slight disparity is observed at 1B scale when using 4096 sequence length, which we leave for future investigation (more details in \cref{sec:llm_appendix}).

	\vspace{-5pt}
\section{Related Work}
\label{sec:related}
\vspace{-8pt}
\noindent\textbf{Self-Supervised Learning.} 
Recent SSL approaches have shown performance comparable to their supervised learning equivalents~\cite{caron2020unsupervised, caron2021emerging, chen2020simple, he2020momentum, grill2020bootstrap, zbontar2021barlow, bardes2021vicreg, chen2020improved}. In a nutshell, most of these methods use image augmentation techniques to generate correlated views (positives) from a sample, and then learn a model that is invariant to these augmentations by enforcing the network to output similar representations for the positives. Initially, contrastive learning, based on instance discrimination~\cite{wu2018unsupervised} using noise-contrastive estimation~\cite{gutmann2010noise, oord2018representation}, was a popular strategy~\cite{chen2020simple, he2020momentum}. However, this learning paradigm 
requires large batch sizes or memory banks. A few methods that use a negative-free cosine similarity loss~\cite{grill2020bootstrap, chen2021exploring} have addressed such issues.

Concurrently, clustering-based methods (SwAV~\cite{caron2020unsupervised}, DeepCluster v2~\cite{caron2020unsupervised, caron2018deep} and DINO~\cite{caron2021emerging}) have also been proposed. They do not operate on the features directly, and instead compare positives through a cross-entropy loss using cluster prototypes as a proxy. Redundancy reduction-based methods have also been popular\cite{ermolov2021whitening, zbontar2021barlow, bardes2021vicreg}. Among them, BarlowTwins \cite{zbontar2021barlow} considers an objective function measuring the cross-correlation matrix between the features, and VicReg\cite{bardes2021vicreg} uses a mix of variance, invariance and covariance
regularizations. Methods such as~\cite{dwibedi2021little} have explored the use of nearest-neighbour retrieval and divide and conquer~\cite{tian2021divide}. However, none of these works studied the ability of SSL methods to learn continually and adaptively.


\noindent\textbf{Continual Learning.} A plethora of methods have been developed to counteract catastrophic forgetting~\cite{kirkpatrick2017overcoming, Rusu16progressive, shin2017continual, Lopez-Paz17, Chaudhry19, serra2018overcoming, chaudhry2018riemannian, Aljundi17, Zenke17, buzzega2020dark, fini2020online, douillard2020podnet, wu2019large, castro2018end, rebuffi2017icarl, hou2019learning, prabhu2020gdumb, Li17learning, ostapenko2019learning, cha2021co2l, Robins95}. Following~\cite{de2019continual}, these works can be organized into three macro-categories: replay-based~\cite{ostapenko2019learning, Robins95, rebuffi2017icarl, buzzega2020dark, Chaudhry19, Lopez-Paz17}, regularization-based~\cite{fini2020online, Li17learning, shin2017continual, kirkpatrick2017overcoming, Zenke17, Aljundi17, castro2018end, douillard2020podnet, hou2019learning, chaudhry2018riemannian, wu2019large, cha2021co2l}, and parameter isolation
methods~\cite{Rusu16progressive, serra2018overcoming}. All these works evaluate the effectiveness of CL methods using a linear classifier learned
sequentially over time. However, this evaluation does not reflect an important aspect, \textit{i.e.}, the internal dynamics of the hidden representations. Moreover, most CL methods tend to rely on supervision in order to mitigate catastrophic forgetting. A few of them can be adapted for the unsupervised setting, although their effectiveness is greatly reduced (see discussion in Sec.~\ref{sec:cassowary}, Sec.~\ref{sec:experiments} and the supplementary material). 

Works such as~\cite{rao2019continual, achille2018life, smith2019unsupervised} laid the foundations of unsupervised CL, but their studies are severely limited to digit-like datasets, \emph{e.g.}, MNIST and Omniglot, and the proposed methods are unfit for large-scale scenarios. Recently, \cite{gallardo2021self, caccia2021special} explored self-supervised pretraining for supervised continual learning with online and few-shot tasks, and \cite{cha2021co2l} presented a supervised contrastive CL approach. Two concurrent works~\cite{lin2021continual, madaan2021rethinking} have also attempted to address CSSL recently. The former~\cite{lin2021continual} extends~\cite{cha2021co2l} to the unsupervised setting, but is specifically designed for contrastive SSL, such as~\cite{chen2020simple,he2020momentum}, and lacks generalizability to other popular SSL paradigms. The latter~\cite{madaan2021rethinking} is also limited as it only shows small-scale experiments in the class-incremental setting and considers just two SSL methods. In contrast, we present a general framework for CSSL with superior performance, conduct large-scale experiments on three challenging settings, thereby presenting a deeper analysis of CSSL.
    
Hyperbolic embeddings embed hierarchical information with high
fidelity and few dimensions. We explored the limits of this approach
by describing scalable, high quality algorithms. We hope the
techniques here encourage more follow-on work on the exciting
techniques of \citet{fb, ucl}. As future work, we hope to explore how
hyperbolic embeddings can be most effectively incorporated into downstream
tasks and applications.

    {\small
        \bibliography{mixup}
        \bibliographystyle{iclr2018}}
\end{document}
