\section{Method}
%%%%%%%%% Figure: Overall framework
\begin{figure*}[t]
  \centering
   \includegraphics[width=0.85\linewidth]{figures/PoolFormer_overall_architecture.pdf}
   \vspace{-4mm}
   \caption{\textbf{(a) The overall framework of \modelname{}.} Similar to \cite{resnet, pvt, swin}, \modelname{} adopts hierarchical architecture with 4 stages. For a model with L \modelname{} blocks, stage [1, 2, 3, 4] have [L/6, L/6, L/2, L/6] blocks, respectively. The feature dimension $D_i$ of stage $i$ is shown in the figure. \textbf{(b) The architecture of \modelname{} block.} Compared with Transformer block, it replaces attention with extremely simple non-parametric operator, pooling, to conduct only basic token mixing.}
   \label{fig:overall_architecture}
\end{figure*}


%%%%%%%%% Algorithm: Pooling
\begin{algorithm}[t]
\caption{Pooling for PoolFormer, PyTorch-like Code}
\label{alg:code}
\definecolor{codeblue}{rgb}{0.25,0.5,0.5}
\definecolor{codekw}{rgb}{0.85, 0.18, 0.50}
\lstset{
  backgroundcolor=\color{white},
  basicstyle=\fontsize{7.5pt}{7.5pt}\ttfamily\selectfont,
  columns=fullflexible,
  breaklines=true,
  captionpos=b,
  commentstyle=\fontsize{7.5pt}{7.5pt}\color{codeblue},
  keywordstyle=\fontsize{7.5pt}{7.5pt}\color{codekw},
}
\begin{lstlisting}[language=python]
import torch.nn as nn

class Pooling(nn.Module):
    def __init__(self, pool_size=3):
        super().__init__()
        self.pool = nn.AvgPool2d(
            pool_size, stride=1, 
            padding=pool_size//2, 
            count_include_pad=False,
        )
    def forward(self, x):
        """
        [B, C, H, W] = x.shape
        Subtraction of the input itself is added 
        since the block already has a 
        residual connection.
        """
        return self.pool(x) - x
\end{lstlisting}
\end{algorithm}

\subsection{MetaFormer}
We present the core concept ``MetaFormer" for this work at first. As shown in Figure \ref{fig:first_figure}, abstracted from Transformers \cite{transformer}, 
MetaFormer is a general architecture where the token mixer is not specified while the other components are kept the same as Transformers. The input $I$ is first processed by input embedding, such as  patch embedding for ViTs \cite{vit},
\begin{equation}
    X = \mathrm{InputEmb}(I),
\end{equation}
where  $X \in \mathbb{R}^{N \times C}$ denotes the embedding tokens with sequence length $N$ and embedding dimension $C$. 


Then, embedding tokens are fed to repeated MetaFormer blocks, each of which includes two residual sub-blocks. Specifically, the first sub-block mainly contains a token mixer to communicate information among tokens and this sub-block can be expressed as
\begin{equation}
    Y = \mathrm{TokenMixer}(\mathrm{Norm}(X)) + X,
\end{equation}
where $\mathrm{Norm}(\cdot)$ denotes the normalization such as Layer Normalization \cite{layer_norm} or Batch Normalization \cite{batch_norm}; $\mathrm{TokenMixer}(\cdot)$ means a module mainly working for mixing token information. It is implemented by various attention mechanism in recent vision Transformer models  \cite{vit,refiner,t2t} or spatial MLP in MLP-like models \cite{mlp-mixer, resmlp}. Note that the main function of the token mixer is to propagate token information although some token mixers can also mix channels, like attention. 


The second sub-block primarily consists of a two-layered MLP with non-linear activation, 
\begin{equation}
    Z = \sigma(\mathrm{Norm}(Y)W_1)W_2 + Y,
\end{equation}
where $W_1 \in \mathbb{R}^{C \times rC}$ and $W_2 \in \mathbb{R}^{rC \times C}$ are learnable parameters with MLP expansion ratio $r$; $\sigma(\cdot)$ is a non-linear activation function, such as GELU \cite{gelu} or ReLU \cite{relu}. 

\myPara{Instantiations of MetaFormer} MetaFormer describes a general architecture 
% a general architecture that is powerful at solving computer vision tasks.  
with which different models can be obtained immediately by specifying the concrete design of the token mixers. 
As shown in Figure \ref{fig:first_figure}(a), if the token mixer is specified as attention or spatial MLP, MetaFormer then becomes a Transformer or MLP-like model respectively. 

\subsection{PoolFormer}
From the introduction of Transformers \cite{transformer}, lots of works attach much importance to the attention and focus on designing various attention-based token mixer components. In contrast, these works pay little attention to the general architecture, \ie, the MetaFormer.


In this work, we argue that this MetaFormer general architecture contributes mostly to the success of the recent Transformer and MLP-like models. 
To demonstrate it, we deliberately employ an embarrassingly simple operator, pooling, as the token mixer. This operator has no learnable parameters and it just makes each token averagely aggregate its nearby token features. 


Since this work is targeted at vision tasks,  we assume the input is in channel-first data format, \ie,  $T \in \mathbb{R}^{C \times H \times W}$. The pooling operator can be expressed as
\begin{equation}
\label{eq:pool}
    T'_{:, i, j} =  \frac{1}{K \times K} \sum_{p,q=1}^{K}T_{:, i+p-\frac{K+1}{2}, i+q-\frac{K+1}{2}} - T_{:, i, j},
\end{equation}
where $K$ is the pooling size. Since the MetaFormer block already has a residual connection, subtraction of the input itself is added in Equation (\ref{eq:pool}). The PyTorch-like code of the pooling is shown in Algorithm \ref{alg:code}.


As well known, self-attention and spatial MLP have computational complexity quadratic to the number of tokens to mix. Even worse, spatial MLPs bring much more parameters when handling longer sequences. As a result, self-attention and spatial MLPs usually can only process hundreds of tokens. In contrast, the pooling needs a computational complexity linear to the sequence length without any learnable parameters.  Thus, we take advantage of pooling by adopting a hierarchical structure similar to traditional CNNs \cite{alexnet, vgg, resnet} and recent hierarchical Transformer variants \cite{swin, pvt}. Figure \ref{fig:overall_architecture} shows the overall framework of PoolFormer. Specifically, PoolFormer has 4 stages with $\frac{H}{4} \times \frac{W}{4}$, $\frac{H}{8} \times \frac{W}{8}$, $\frac{H}{16} \times \frac{W}{16}$, and $\frac{H}{32} \times \frac{W}{32}$ tokens respectively, where $H$ and $W$ represent the width and height of the input image. There are two groups of embedding size: 1) small-sized models with embedding dimensions of 64, 128, 320, and 512 responding to the four stages; 2) medium-sized models with embedding dimensions 96, 192, 384, and 768. Assuming there are $L$ PoolFormer blocks in total, stages 1, 2, 3, and 4 will contain $L/6$, $L/6$, $L/2$, and $L/6$ PoolFormer blocks respectively. The MLP expansion ratio is set as 4. According to the above simple model scaling rule, we obtain 5 different model sizes of PoolFormer and their hyper-parameters are shown in Table \ref{tab:model}.


%%%%%%%%% Table: Model Configurations
\begin{table}[t]
\footnotesize
\centering
\setlength{\tabcolsep}{2pt}
% \scalebox{0.65}{\newcommand{\blockc}[4]{
$\begin{bmatrix}
	\begin{array}{l}
	R_1=#1 \\
	N_1=#2 \\
	E_1=#3 \\
	\end{array}
\end{bmatrix} \times #4$
}

\newcommand{\sblock}[3]{
$\begin{matrix}
E_{#1}=#2 \\
L_{#1}=#3 \\
\end{matrix}$
}

\newcommand{\poollayer}{
Pooling Size & \multicolumn{5}{c}{$3 \times 3$, stride 1}\\
\cline{4-9}
}

\newcommand{\stitle}[6]{
\multirow{5}{*}{#1} & \multirow{5}{*}{\scalebox{1}{$\frac{H}{#2}\times \frac{W}{#2}$}} & \multirow{2}{*}{\tabincell{c}{Patch \\ Embedding}} & Patch Size & \multicolumn{5}{c}{$#3 \times #3$, stride $#4$} \\
\cline{4-9}
    &    &    & Embed. Dim. & \multicolumn{3}{c|}{$#5$} & \multicolumn{2}{c}{$#6$} \\
\cline{3-9}
& & \multirow{3}{*}{\tabincell{c}{\modelname{}\\Block}} 
}

\begingroup
\renewcommand{\arraystretch}{1.1}
\begin{tabular}{c|c|c|c|c|c|c|c|c}
\toprule
  \multirow{2}{*}{Stage} & \multirow{2}{*}{\#Tokens} & \multicolumn{2}{c|}{\multirow{2}{*}{Layer Specification}} & \multicolumn{5}{c}{\modelname{}} \\
\cline{5-9}
 & & \multicolumn{2}{c|}{} & S12 & S24 & S36 & M36 & M48 \\
\whline
\stitle{1}{4}{7}{4}{64}{96}    & \poollayer
 & & & MLP Ratio & \multicolumn{5}{c}{4} \\
\cline{4-9}
 & & & \# Block & 2 & 4 & 6 & 6 & 8 \\
\hline
\stitle{2}{8}{3}{2}{128}{192}  & \poollayer
 & & & MLP Ratio & \multicolumn{5}{c}{4} \\
 \cline{4-9}
 & & & \# Block & 2 & 4 & 6 & 6 & 8 \\
\hline
\stitle{3}{16}{3}{2}{320}{384} & \poollayer
 & & & MLP Ratio & \multicolumn{5}{c}{4} \\
\cline{4-9}
 & & & \# Block & 6 & 12 & 18 & 18 & 24 \\
\hline
\stitle{4}{32}{3}{2}{512}{768} & \poollayer
 & & & MLP Ratio & \multicolumn{5}{c}{4} \\
 \cline{4-9}
 & & & \# Block & 2 & 4 & 6 & 6 & 8 \\
\hline
\multicolumn{4}{c|}{Parameters~(M)}& 11.9 & 21.4              &  30.8            &  56.1              &  73.4    \\
\hline
\multicolumn{4}{c|}{MACs~(G)}     & 1.8  &  3.4              &  5.0             &  8.8              &  11.6    \\
\bottomrule
\end{tabular}
\endgroup
}
\newcommand{\blockc}[4]{
$\begin{bmatrix}
	\begin{array}{l}
	R_1=#1 \\
	N_1=#2 \\
	E_1=#3 \\
	\end{array}
\end{bmatrix} \times #4$
}

\newcommand{\sblock}[3]{
$\begin{matrix}
E_{#1}=#2 \\
L_{#1}=#3 \\
\end{matrix}$
}

\newcommand{\poollayer}{
Pooling Size & \multicolumn{5}{c}{$3 \times 3$, stride 1}\\
\cline{4-9}
}

\newcommand{\stitle}[6]{
\multirow{5}{*}{#1} & \multirow{5}{*}{\scalebox{1}{$\frac{H}{#2}\times \frac{W}{#2}$}} & \multirow{2}{*}{\tabincell{c}{Patch \\ Embedding}} & Patch Size & \multicolumn{5}{c}{$#3 \times #3$, stride $#4$} \\
\cline{4-9}
    &    &    & Embed. Dim. & \multicolumn{3}{c|}{$#5$} & \multicolumn{2}{c}{$#6$} \\
\cline{3-9}
& & \multirow{3}{*}{\tabincell{c}{\modelname{}\\Block}} 
}

\begingroup
\renewcommand{\arraystretch}{1.1}
\begin{tabular}{c|c|c|c|c|c|c|c|c}
\toprule
  \multirow{2}{*}{Stage} & \multirow{2}{*}{\#Tokens} & \multicolumn{2}{c|}{\multirow{2}{*}{Layer Specification}} & \multicolumn{5}{c}{\modelname{}} \\
\cline{5-9}
 & & \multicolumn{2}{c|}{} & S12 & S24 & S36 & M36 & M48 \\
\whline
\stitle{1}{4}{7}{4}{64}{96}    & \poollayer
 & & & MLP Ratio & \multicolumn{5}{c}{4} \\
\cline{4-9}
 & & & \# Block & 2 & 4 & 6 & 6 & 8 \\
\hline
\stitle{2}{8}{3}{2}{128}{192}  & \poollayer
 & & & MLP Ratio & \multicolumn{5}{c}{4} \\
 \cline{4-9}
 & & & \# Block & 2 & 4 & 6 & 6 & 8 \\
\hline
\stitle{3}{16}{3}{2}{320}{384} & \poollayer
 & & & MLP Ratio & \multicolumn{5}{c}{4} \\
\cline{4-9}
 & & & \# Block & 6 & 12 & 18 & 18 & 24 \\
\hline
\stitle{4}{32}{3}{2}{512}{768} & \poollayer
 & & & MLP Ratio & \multicolumn{5}{c}{4} \\
 \cline{4-9}
 & & & \# Block & 2 & 4 & 6 & 6 & 8 \\
\hline
\multicolumn{4}{c|}{Parameters~(M)}& 11.9 & 21.4              &  30.8            &  56.1              &  73.4    \\
\hline
\multicolumn{4}{c|}{MACs~(G)}     & 1.8  &  3.4              &  5.0             &  8.8              &  11.6    \\
\bottomrule
\end{tabular}
\endgroup

\vspace{-3mm}
\caption{\textbf{ Configurations of different PoolFormer models.} There are two groups of embedding dimensions, \ie, small size with [64, 128, 320, 512] dimensions and medium size with [96, 196, 384, 768]. Notation ``S24" means the model is in small size of embedding dimensions with 24 PoolFormer blocks in total. The numbers of MACs are counted by \texttt{fvcore}\cite{fvcore} library.
}
\label{tab:model}
\vspace{-4mm}
\end{table}
