\section{Dataset Licenses \& Terms}
\label{app:licenses-and-terms}

\paragraph{Detailed taxonomy}
We code the legal restrictions placed on use of datasets along two axes. First, we identify whether a dataset's license permits commercial use (``Commercial'' in Table \ref{tab:license_terms_breakdown}), only non-commercial / academic use (``NC / Acad''), or does not clearly specify what is permitted (``Unspecified''). 
The latter category includes datasets for which we were unable to locate a license.
Datasets which are in the public domain and not subject to a license are counted as commercially usable.
Second, we annotate the contractual or terms-of-use restrictions placed on dataset use by the source of each dataset. There are four levels, defined in Table \ref{tab:license_terms_breakdown}. Note that the Model Closed status can only apply to datasets that are AI-generated, at least in part.
Some datasets can carry both Model Closed and Source Closed status, but we count the Model Closed first for simplicity.

\begin{table*}[t!]
\centering
\begin{adjustbox}{width=1.0\textwidth}
\begin{tabular}{p{0.19\textwidth}|p{0.81\textwidth}}
\toprule
\textsc{Label} & \textsc{Definition} \\
\midrule
\textsc{Model Closed} & A model used to generate part or all of the dataset prohibits using its outputs commercially, to develop a competing AI model, or in general. \\
\textsc{Source Closed} & The source has a license or terms that prohibits use of the data, either commercially, from being crawled, to develop AI, or in general. \\
\textsc{Unspecified} & No information can be found relevant to restrictions, or lack thereof, for this source. \\
\textsc{Unrestricted} & The source has a commercially permissive license, such as CC BY, or explicitly states the data is open for broad use. \\
\bottomrule
\end{tabular}
\end{adjustbox}
\caption{\textbf{The taxonomy used to determine use restrictions on each dataset source.} Each source in a dataset is examined and fit into one of these categories. The dataset Terms are then labelled according to the strictest terms across the sources, with Model Closed and Source Closed considered stricter than Unspecified which is in turn stricter than Unrestricted.}
\label{tab:terms-taxonomy}
\end{table*}


\paragraph{Detailed breakdown}
Tables \ref{tab:license_terms_breakdown} and \ref{tab:license_terms_breakdown_count} present crosstabs of these two dimensions, according to respectively the total amount of content and the number of datasets. The most notable finding, as discussed in the main text, is the frequency of clashing restriction status between licenses and terms. By amount of content, fully 73.0\% of text content, 55.0\% of speech content, and 21.6\% of video content is subject to a license permitting commercial use but also to terms restrictions forbidding it, or the reverse. The absolute level of restrictions is also high, with < 0.1\% of text content, 5.4\% of speech content, and 0.6\% of video content usable for commercial purposes under both licenses and terms.

\begin{table*}[!htb]
\centering
\begin{adjustbox}{width=0.9\textwidth}
\begin{tabular}{l|rrr|r}
\toprule
\textsc{License / Terms} & \textsc{Restricted} & \textsc{Unspecified} & \textsc{Unrestricted} & \textsc{Total} \\
\midrule
\multicolumn{5}{c}{\textbf{\emph{Text Collections}}} \\
\midrule
\textsc{NC/Acad} & 96.0 & 0.0 & 0.0 & 96.0 \\
\textsc{Unspecified} & 2.3 & 0.1 & 0.0 & 2.4 \\
\textsc{Commercial} &1.5 & 0.0 & 0.0 & 1.6 \\
\midrule
\textsc{Total} & 99.8 & 0.1 & 0.1 & \\
\midrule
\multicolumn{5}{c}{\textbf{\emph{Text Datasets}}} \\
\midrule
\textsc{NC/Acad} & 21.1 & 0.0 & 0.0 & 21.2 \\
\textsc{Unspecified} & 5.7 & 0.1 & 0.0 & 5.7 \\
\textsc{Commercial} & 73.0 & 0.0 & 0.0 & 73.1 \\
\midrule
\textsc{Total} & 99.8 & 0.1 & 0.1 & \\
\midrule
\multicolumn{5}{c}{\textbf{\emph{Speech Datasets}}} \\
\midrule
\textsc{NC/Acad} & 23.9 & 1.4 & 0.8 & 26.2 \\
\textsc{Unspecified} & 0.5 & 0.0 & 0.4 & 0.9 \\
\textsc{Commercial} & 54.2 & 13.3 & 5.4 & 73.0 \\
\midrule
\textsc{Total} & 78.6 & 14.7 & 6.7 & \\
\midrule
\multicolumn{5}{c}{\textbf{\emph{Video Datasets}}} \\
\midrule
\textsc{NC/Acad} & 33.7 & 0.0 & 0.1 & 33.8 \\
\textsc{Unspecified} & 43.9 & 0.1 & 0.1 & 44.1 \\
\textsc{Commercial} & 21.5 & 0.0 & 0.6 & 22.1 \\
\midrule
\textsc{Total} & 99.1 & 0.1 & 0.8 & \\
\bottomrule
\end{tabular}
\end{adjustbox}
\caption{\textbf{A breakdown of the percentage of license and terms restrictions across datasets}, by total tokens or hours of content. The much higher frequency of restrictions at the collection level is because we consider a collection's license or terms status to be the most restrictive of those for its datasets. Note that percentages may not add to exactly 100\% because of rounding. 
}
\label{tab:license_terms_breakdown}
\vspace{-2mm}
\end{table*}


\begin{table*}[!htb]
\centering
\begin{adjustbox}{width=0.9\textwidth}
\begin{tabular}{l|rrr|r}
\toprule
\textsc{License / Terms} & \textsc{Restricted} & \textsc{Unspecified} & \textsc{Unrestricted} & \textsc{Total} \\
\midrule
\multicolumn{5}{c}{\textbf{\emph{Text Collections}}} \\
\midrule
\textsc{NC/Acad} & 84.5 & 0.0 & 0.3 & 84.8 \\
\textsc{Unspecified} & 1.5 & 7.5 & 0.0 & 8.9 \\
\textsc{Commercial} & 1.5 & 0.2 & 4.5 & 6.3 \\
\midrule
\textsc{Total} & 87.5 & 7.7 & 4.8 &  \\
\midrule
\multicolumn{5}{c}{\textbf{\emph{Text Datasets}}} \\
\midrule
\textsc{NC/Acad} & 25.0 & 0.0 & 0.3 & 25.3 \\
\textsc{Unspecified} & 17.3 & 1.2 & 0.0 & 18.5 \\
\textsc{Commercial} & 45.2 & 6.5 & 4.5 & 56.2 \\
\midrule
\textsc{Total} & 87.5 & 7.7 & 4.8 &  \\
\midrule
\multicolumn{5}{c}{\textbf{\emph{Speech Datasets}}} \\
\midrule
\textsc{NC/Acad} & 9.5 & 9.5 & 13.7 & 32.6 \\
\textsc{Unspecified} & 6.3 & 0.0 & 7.4 & 13.7 \\
\textsc{Commercial} & 7.4 & 18.9 & 27.4 & 53.7 \\
\midrule
\textsc{Total} & 23.2 & 28.4 & 48.4 &  \\
\midrule
\multicolumn{5}{c}{\textbf{\emph{Video Datasets}}} \\
\midrule
\textsc{NC/Acad} & 22.1 & 0.0 & 9.6 & 31.7 \\
\textsc{Unspecified} & 23.1 & 1.0 & 11.5 & 35.6 \\
\textsc{Commercial} & 25.0 & 0.0 & 7.7 & 32.7 \\
\midrule
\textsc{Total} & 70.2 & 1.0 & 28.8 & \\
\bottomrule
\end{tabular}
\end{adjustbox}
\caption{\textbf{A breakdown of the percentage of license and terms restrictions} by dataset count. The much higher frequency of restrictions at the collection level is because we consider a collection's license or terms status to be the most restrictive of those for its datasets. Note that percentages may not add to exactly 100\% because of rounding.
}
\label{tab:license_terms_breakdown_count}
\vspace{-2mm}
\end{table*}
