\begin{table*}[t!]
\centering
% \footnotesize
\renewcommand{\arraystretch}{1.5}
\begin{adjustbox}{width=\textwidth}
\begin{tabular}{p{3cm} | p{16cm}}
\toprule
\textsc{Attribute} & \textsc{Details}  \\
\midrule

\textbf{Aggregator Links} & Instances of the dataset on GitHub, HuggingFace, Papers with Code, ArXiv, and Semantic Scholar are manually linked. This allows artifact, usage, and licensing metadata to be cross-referenced. \\ \rowcolor[gray]{0.9}
\textbf{Sources} & The original sources from which the data was collected. This includes websites or databases, human annotation, and model-generated content, wherever disclosed.  \\
\textbf{Licenses} & The dataset licenses and license URLs found on associated GitHub, HuggingFace, website, and academic paper pages. We also use \citet{longpre2023data} guidelines to classify licenses by use type (non-commercial, unspecified, all uses permitted), attribution, and copyleft restrictions.  \\ \rowcolor[gray]{0.9}
\textbf{Creators} & The organizations attributed with curating the dataset (not the underlying data). These organizations are categorized by country and by industry vs academia.  \\
\textbf{Tasks} & The specific tasks associated with the supervised datasets.  \\ \rowcolor[gray]{0.9}
\textbf{Size Metrics} & The size of the dataset, measured in dialog turns and tokens for text, the number of hours for video and speech, as well as the number of speakers for speech.  \\
% \midrule
\textbf{Languages} & The languages and language families represented in the text or speech. \\


\bottomrule
\end{tabular}
\end{adjustbox}
% \captionsetup{justification=centering}
\caption{The \textbf{list of dataset attributes}, their collection process, and the modalities they were collected for (includes only attributes with 2+ modalities). All attributes are manually collected. Annotation guidelines are available in \cref{app:attribute-annotation}.}
\label{tab:attributes}
\vspace{-2mm}
\end{table*}

