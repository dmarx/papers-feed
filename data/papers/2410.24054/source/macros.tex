\renewcommand{\epsilon}{\varepsilon}

\newcommand{\defeq}{=\vcentcolon}
\newcommand{\given}{\,|\,}

% Linear algebra
\newcommand{\norm}[1]{\left\lVert#1\right\rVert}
\newcommand{\trace}{\text{tr}}
\newcommand{\diag}{\text{diag}}

% MathCal
\newcommand{\A}{\mathcal{A}}
\newcommand{\B}{\mathcal{B}}
\newcommand{\C}{\mathcal{C}}
%% Note: we changed this macro for this specific paper
\newcommand{\D}{\mathscr{D}}
\newcommand{\F}{\mathcal{F}}
\newcommand{\G}{\mathcal{G}}
\renewcommand{\L}{\mathcal{L}}
\newcommand{\N}{\mathcal{N}}
\renewcommand{\O}{\mathcal{O}}
\newcommand{\R}{\mathcal{R}}
\newcommand{\Q}{\mathcal{Q}}
\newcommand{\X}{\mathcal{X}}
\newcommand{\J}{\mathcal{J}}
\newcommand{\T}{\mathcal{T}}

\newcommand{\Qfull}{\mathcal{Q}_{\text{Full}}}
\newcommand{\Qfact}{\mathcal{Q}_{\text{Fact}}}

\newcommand{\alphat}{\alpha^{(t)}}
\newcommand{\alphatplus}{\alpha^{(t+1)}}

% Blackboard fonts
\renewcommand{\SS}{\mathbb{S}}
\newcommand{\XX}{\mathbb{X}}
%% Note: we changed this macro for this specific paper
%\newcommand{\ZZ}{\reals^D}
\newcommand{\ZZ}{\mathbb{Z}}

% Script fonts
\newcommand{\sX}{\mathscr{X}}

% Bold fonts
\newcommand{\bfx}{\mathbf{x}}
\newcommand{\bfb}{\mathbf{b}}

% Expectation etc.
\newcommand{\E}{\mathbb{E}}
\newcommand{\Var}{\text{Var}}
\newcommand{\Cov}{\text{Cov}}
\newcommand{\KL}{\text{KL}}

% Commonly used sets
\newcommand{\nats}{\mathbb{N}}
\newcommand{\ints}{\mathbb{Q}}
\newcommand{\reals}{\mathbb{R}}

% Basic definitions of math envs
\newtheorem{theorem}{Theorem}[section]
\newtheorem{definition}[theorem]{Definition}
\newtheorem{proposition}[theorem]{Proposition}
\newtheorem{property}[theorem]{Property}
\newtheorem{lemma}[theorem]{Lemma}
\newtheorem{corollary}[theorem]{Corollary}
\newtheorem{example}[theorem]{Example}
\newtheorem{remark}[theorem]{Remark}
\newtheorem{assumption}[theorem]{Assumption}

% Unnumbered math envs
%\newtheorem*{theorem*}{Theorem}
%\newtheorem*{proposition*}{Proposition}
%\newtheorem*{lemma*}{Lemma}
%\newtheorem*{corollary*}{Corollary}
%


%%%%%%%%%%%%%%%%%%%%%%%%%%%%%%%%%%%%%%%%%%%%%%%%%%%%%%%%%%%%%%%%%%%%%%%%%%%%%%
% Paper-specific

\usepackage{thmtools}


\newcommand{\red}[1]{\textcolor[rgb]{0.635,0.0780,0.1840}{#1}}
\newcommand{\blue}[1]{\textcolor[rgb]{0,0.33,0.56}{#1}}
\newcommand{\darkblue}[1]{\textcolor[rgb]{0,0.33,0.86}{#1}}
\newcommand{\green}[1]{\textcolor[rgb]{0,0.65,0.33}{#1}}
\definecolor{shade}{rgb}{0.9,0.9,0.9}


%\declaretheoremstyle[
%spaceabove=6pt, spacebelow=6pt,
%postheadspace=1em,
%headfont=\normalfont\bfseries,
%notefont=\mdseries, notebraces={(}{)},
%bodyfont=\normalfont,
%%qed=\qedsymbol
%]{mystyle}
%
%% Proof style
%\declaretheoremstyle[
%    spaceabove=-6pt,  spacebelow=6pt,
%    headfont=\normalfont\bfseries,
%    bodyfont = \normalfont,
%    postheadspace=1em,
%    qed=$\blacksquare$,
%    headpunct={:}]{myproofstyle} %<---- change this name
%\declaretheorem[name={Proof}, style=myproofstyle, unnumbered]{Proof}
%
%
%\declaretheorem[
%  style=mystyle,
%  name=Theorem,
%  numberwithin=section,
%  sharenumber=theorem
%]{rtheorem}
%
%\declaretheorem[
%  shaded={
%      rulecolor=black,
%      textwidth=0.98\linewidth,
%      rulewidth=1pt,
%      bgcolor=gray!15,
%      margin=5pt,
%  },
%  style=mystyle,
%  name=Theorem,
%  numberwithin=section,
%  sharenumber=theorem
%]{ftheorem}
%
%
%\declaretheorem[
%  shaded={
%      rulecolor=black,
%      textwidth=0.98\linewidth,
%      rulewidth=1pt,
%      bgcolor=gray!15,
%      margin=5pt,
%  },
%  style=mystyle,
%  name=Lemma,
%  %numberwithin=section
%  sharenumber=theorem
%]{flemma}
%
%\declaretheorem[
%  shaded={
%      rulecolor=black,
%      textwidth=0.98\linewidth,
%      rulewidth=1pt,
%      bgcolor=gray!15,
%      margin=5pt,
%  },
%  style=mystyle,
%  name=Definition,
%  %numberwithin=section
%  sharenumber=theorem
%]{fdefinition}
%
%\declaretheorem[
%  shaded={
%      rulecolor=black,
%      textwidth=0.98\linewidth,
%      rulewidth=1pt,
%      bgcolor=gray!10,
%      margin=5pt,
%  },
%  style=mystyle,
%  name=Corollary,
%  %numberwithin=section
%  sharenumber=theorem
%]{fcorollary}
%
%\declaretheorem[
%  shaded={
%      rulecolor=black,
%      textwidth=0.98\linewidth,
%      rulewidth=1pt,
%      bgcolor=gray!10,
%      margin=5pt,
%  },
%  style=mystyle,
%  name=Proposition,
%  %numberwithin=section
%  sharenumber=theorem
%]{fproposition}




%\DeclareUnicodeCharacter{03BC}{$\mu$}
%\DeclareUnicodeCharacter{03A3}{$\Sigma$}
%\DeclareUnicodeCharacter{0393}{$\Gamma$}
%\DeclareUnicodeCharacter{03BB}{$\lambda$}
%
%% What we want to call \Sigma_p and \mu_p
%\newcommand{\Sigmap}{\Sigma_*}
%\newcommand{\mup}{\mu_*}
%
%%\usepackage{eqparbox}
%%\renewcommand{\algorithmiccomment}[1]{\hfill\eqparbox{COMMENT}{\# #1}}
%
%\newcommand{\suggest}[1]{\red{#1}}
%
%\renewcommand{\vec}[1]{#1} % {\bm{#1}}
%\newcommand{\mat}[1]{#1} % {\mathbf{#1}}
%


\declaretheorem[
  shaded={
      rulecolor=gray!15,
      textwidth=0.98\linewidth,
      rulewidth=1pt,
      bgcolor=gray!10,
      margin=5pt,
  },
  style=mystyle,
  name=Property,
  %numberwithin=section
  sharenumber=theorem
]{fproperty}

