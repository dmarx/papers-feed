%\documentclass[main]{subfiles}

%\begin{document}

\section{Calculation of moments}
\label{app:moments}

In this appendix we show how to calculate the low-order moments of a density
constructed from the Cartesian product of orthogonal function expansions.
In particular, we assume that the density is over~$\mathbb{R}^D$ and of the form
\begin{equation}
    q(z_1,z_2,\ldots,z_D) = \left(\sum_{k_1=1}^{K_1} \cdots \sum_{k_D=1}^{K_D} \alpha_{k_1 k_2 \ldots k_D}\phi_{k_1}(z_1)\phi_{k_2}(z_2)\cdots\phi_{k_D}(z_D)\right)^2,
    \label{eq:joint}
\end{equation}
where $\{\phi_{k}(\cdot)\}_{k=1}^\infty$ are orthogonal functions on $\mathbb{R}$ and where the coefficients are properly normalized so that the density integrates to one. For such a density, we show that the calculation of first and second-order moments boils down to evaluating \textit{one-dimensional} integrals of the form
\begin{align}
\mu_{ij} &= \int_{-\infty}^\infty\! \phi_i(z)\phi_j(z)\, z\, dz,  \label{eq:momint1} \\
\nu_{ij} &= \int_{-\infty}^\infty\! \phi_i(z)\phi_j(z)\, z^2\, dz. \label{eq:momint2}
\end{align}
We also show how to evaluate these integrals specifically for the orthogonal family of weighted Hermite polynomials.

First we consider how to calculate moments such as $\mathbb{E}_q[z_d^p]$, where $p\in\{1,2\}$, and without loss of generality we focus on calculating $\mathbb{E}_q[z_1^p]$.
We start from the joint distribution in \Cref{eq:joint}
and proceed by marginalizing over the variables $(z_2,z_3,\ldots,z_D)$.
%
Exploiting orthogonality, we find that
\begin{align}
  \mathbb{E}_q[z_1^p]
    &= \int\! q(z_1,z_2,\ldots,z_D)\, z_1^p\, dz_1\, dz_2\, \ldots dz_D, \\
    &= \int\! \left(\sum_{k_1=1}^{K_1} \cdots \sum_{k_D=1}^{K_D} \alpha_{k_1 k_2 \ldots k_D}\phi_{k_1}(z_1)\phi_{k_2}(z_2)\cdots\phi_{k_D}(z_D)\right)^2\! z_1^p\, dz_1\, dz_2\, \ldots dz_D, \\
   &= \sum_{k_1,k'_1=1}^{K_1} \left[\sum_{k_2=1}^{K_2} \cdots \sum_{k_D=1}^{K_D} \alpha_{k_1 k_2 \ldots k_D} \alpha_{k'_1 k_2 \ldots k_D}\right] \int\! \phi_{k_1}(z_1)\phi_{k'_1}(z_1)\, z_1^p\, dz_1. \label{eq:marginal1}
\end{align}
We can rewrite this expression more compactly as a quadratic form over integrals
of the form in \Crefrange{eq:momint1}{eq:momint2}.
To this end, we define the coefficients
\begin{equation}
    A_{ij} = \sum_{k_2=1}^{K_2}\cdots\sum_{k_D=1}^{K_D} \alpha_{i k_2 \ldots k_D} \alpha_{j k_2
    \ldots k_D},
    \label{eq:Aij}
\end{equation}
which simply encapsulate the bracketed term in \Cref{eq:marginal1}.
Note that there are $K_1^2$ of these coefficients, each of which can be
computed in $\O(K_2 K_3\ldots K_D)$.
With this shorthand, we can write
\begin{align}
    \mathbb{E}_q[z_1] &= \sum_{i,j=1}^{K_1} A_{ij} \mu_{ij}, \label{eq:Amu}\\
    \mathbb{E}_q[z_1^2] & = \sum_{i,j=1}^{K_1} A_{ij} \nu_{ij}, \label{eq:Anu}
\end{align}
where $\mu_{ij}$ and $\nu_{ij}$ are the integrals defined in
\Crefrange{eq:momint1}{eq:momint2}.
Thus the problem has been reduced to a weighted sum of one-dimensional integrals.

A similar calculation gives the result we need for correlations. Again, without loss of generality, we focus on calculating $\mathbb{E}_q[z_1 z_2]$.
Analogous to \Cref{eq:Aij}, we define the tensor of coefficients
\begin{equation}
B_{ijk\ell} = \sum_{k_3=1}^{K_3}\cdots\sum_{k_D=1}^{K_D} \alpha_{i k k_3 \ldots k_D} \alpha_{j \ell k_3 \ldots k_D},
\end{equation}
which arises from marginalizing over the variables $(z_3,z_4,\ldots,z_D)$.
%
There are $K_1^2 K_2^2$ of these coefficients, each of which can be computed
in $\O(K_3 K_4\ldots K_D)$. With this shorthand, we can write
\begin{equation}
    \mathbb{E}_q[z_1 z_2] = \sum_{i,j=1}^{K_1} \sum_{k,\ell=1}^{K_2} B_{ijk\ell}\mu_{ij}\mu_{k\ell}.
    \label{eq:Bmumu}
\end{equation}
where $\mu_{ij}$ is again the integral defined in \Cref{eq:momint1}).
Thus the problem has been reduced to a weighted sum of (the product of)
one-dimensional integrals.

Finally, we show how to evaluate the integrals in \Crefrange{eq:momint1}{eq:momint2}
for the specific case of orthogonal function expansions with weighted Hermite polynomials;
{similar computations apply in the case of Legendre polynomials.}
Recall in this case that
\begin{equation}
\phi_{k+1}(z) = \left(\sqrt{2\pi}k!\right)^{-\frac{1}{2}}\left(e^{-\frac{1}{2}z^2}\right)^{\frac{1}{2}}\,\text{H}_{k}(z),
\label{eq:phiH-def}
\end{equation}
where $\text{H}_k(z)$ are the \emph{probabilist's} Hermite polynomials given by
\begin{equation}
\text{H}_k(z) = (-1)^k e^{\frac{z^2}{2}} \frac{d^k}{dz^k}\left[e^{-\frac{z^2}{2}}\right].
\end{equation}
To evaluate the integrals for this particular family, we can exploit the following recursions that are satisfied by Hermite polynomials:
\begin{align}
  H_{k+1}(z) &= zH_k(z) - H'_k(z), \label{eq:Hrecur1} \\
  H'_k(z) &= kH_{k-1}(z). \label{eq:Hrecur2}
\end{align}
Eliminating the derivatives $H'_k(z)$ in \Crefrange{eq:Hrecur1}{eq:Hrecur2},
we see that $zH_k(z) = H_{k+1}(z) + kH_{k-1}(z)$.
We can then substitute \Cref{eq:phiH-def} to obtain a recursion
for the orthogonal basis functions themselves:
\begin{equation}
z\phi_k(z) = \sqrt{k}\phi_{k+1}(z) + \sqrt{k\!-\!1}\phi_{k-1}(z).
\label{eq:phi-recur}
\end{equation}
With the above recursion, we can now read off these integrals from the property of orthogonality.
For example, starting from \Cref{eq:momint1}, we find that
\begin{align}
\mu_{ij}
  &= \int_{-\infty}^\infty\! \phi_i(z)\phi_j(z)\, z\, dz, \\
  &= \int_{-\infty}^\infty\! \phi_i(z) \left[\sqrt{j}\phi_{j+1}(z) + \sqrt{j\!-\!1}\phi_{j-1}(z)\right]\, dz, \\
  &= \delta_{i,j+1}\sqrt{j} + \delta_{i,j-1}\sqrt{i}, \label{eq:mom1ij}
\end{align}
where $\delta_{ij}$ is the Kronecker delta function.
Next we consider the integral in \Cref{eq:momint2},
which involves a power of $z^2$ in the integrand. In this case we can make repeated use of the recursion:
\begin{align}
\nu_{ij}
  &= \int_{-\infty}^\infty\! \phi_i(z)\phi_j(z)\, z^2\, dz, \\
  &= \int_{-\infty}^\infty\! \left[\sqrt{i}\phi_{i+1}(z) + \sqrt{i\!-\!1}\phi_{i-1}(z)\right]\, \left[\sqrt{j}\phi_{j+1}(z) + \sqrt{j\!-\!1}\phi_{j-1}(z)\right]\, dz, \\
  &= \delta_{ij}\left[\sqrt{ij}+\sqrt{(i\!-\!1)(j\!-\!1)}\right]\,
      +\, \delta_{i-1,j+1}\sqrt{j(j\!+\!1)}\,
      +\, \delta_{j-1,i+1}\sqrt{i(i\!+\!1)}.
      \label{eq:mom2ij}
\end{align}
Note that the matrices in \Cref{eq:mom1ij} and \Cref{eq:mom2ij}
can be computed for whatever size is required by the orthogonal basis function
expansion in \Cref{eq:joint}.
Once these matrices are computed, it is a simple matter of
substitution\footnote{With further bookkeeping, one can also exploit the
\textit{sparsity} of $\mu_{ij}$ and $\nu_{ij}$ to derive more efficient
calculations of these moments.}
to compute the moments $\mathbb{E}_q[z_1]$, $\mathbb{E}_q[z_1^2]$, and
$\mathbb{E}_q[z_1 z_2]$ from \Crefrange{eq:Amu}{eq:Anu} and \Cref{eq:Bmumu}.
Finally, we can compute other low-order moments
(such as $\mathbb{E}_q[z_5]$ or $\mathbb{E}_q[z_3 z_7]$) by an appropriate
permutation of indices.

%\end{document}
