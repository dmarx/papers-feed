\section{Lipschitzness of Sigmoid Attention}
\label{app:lipschitz_proof}
In the following, we report the proof for the recovering the Lipschitzness constant associated with $\sigmoidattn$, as stated in \cref{thm:regularity}.

Letting $A = W_q^T W_k$, and calling $\sigma_{ij} = \sigma(\langle W_qx_i, W_kx_j\rangle)$ and $\sigma'_{ij} = \sigma'(\langle W_qx_i, W_kx_j\rangle)$, we find that the Jacobian of $\phi$ in the direction $(\delta_1, \dots, \delta_n)$ for the sample $x_i$ is given by:
\begin{equation}
    \mathrm{Jac}_i = \left(\sum_{j=1}^n\sigma'_{ij}x_jx_j^TA^T\right)\delta_i +  \sum_{j=1}^n\left(\sigma'_{ij}x_jx_i^TA + \sigma_{ij}I_p\right)\delta_j,
\end{equation}
We see that this Jacobian is the sum of two terms. To control its norm, we can control each norm individually. 

The first term, $\left(\sum_{j=1}^n\sigma'_{ij}x_jx_j^TA^T\right)\delta_i$ is of the form $U_i\delta_i$ with $U_i$ a matrix. 
Its squared-norm is therefore:
\begin{equation}
    \sum_{i=1}^n \|U_i\delta_i\|^2 \leq \max_{i}\|U_i\|_2^2 \|\delta\|_F .
\end{equation}
Hence, its squared spectral norm is bounded by $\max_{i}\|U_i\|_2^2$.

We now let $\sigma'_{\infty}$ be a bound on $n\times |\sigma'|$; 
We have: 
\begin{align}
\|U_i\|_2 & \leq \sum_{j=1}^n\|\sigma'_{ij}x_jx_j^{\top}A\|_2 \\
& \leq \sigma'_{\infty}\|A\|_2\frac1n \sum_{j=1}^n \|x_j\|^2 \\
&\leq \sigma'_{\infty}\|A\|_2 \mathbb{E}[\|x_j\|^2].
\label{eqn:lip_bound_first}
\end{align}
We see that if the points $x_i$ have norm $\leq R$, then the Jacobian grows at most like $R^2$, because it is ``quadratic'' in $x$.
However, we see that the quadratic term is likely to be mitigated by the $\sigma'(a_{ij})$ term that goes to $0$ if $a_{ij}$ is large.

The second term, $\sum_{j=1}^n\left(\sigma'_{ij}x_jx_i^TA + \sigma_{ij}I_p\right)\delta_j$, is the sum of two terms. Here, too, we use the triangular inequality to control their norm individually. We get:
\begin{align}
    \|\sum_{j=1}^n\sigma_{ij}\delta_j\|^2 &=\|\delta^T \sigma_i\|^2 \\
    &\leq \|\delta\|_F^2 \|\sigma_i\|^2,
\end{align}
where $\sigma_i\in\mathbb{R}^p$ is the $i$-th column of $\sigma_{ij}$, and $\delta\in \mathbb{R}^{n\times p}$.
and by summing, letting $\sigma_{\infty}$ an upper bound on $n\times |\sigma(x)|$:
\begin{equation}
    \sum_{i=1}^n\|\sum_{j=1}^n\sigma_{ij}\delta_j\|^2 \leq \sigma_{\infty}^2 \|\delta\|_F^2 .
\end{equation}
So that $\sigma_{\infty}$ upper bounds the spectral norm of the last term.

For the final term, $\sum_{j=1}^n\sigma'_{ij}x_jx_i^TA\delta_j$, define $\hat{\delta} = \delta A^T$. We get:
\begin{equation}
    \sum_{j=1}^n\sigma'_{ij}x_jx_i^TA\delta_j = \sum_{j=1}^n\sigma'_{ij}\langle x_i,\hat{\delta}_j\rangle x_j .
\end{equation}
Hence, letting $M$ the matrix of entries $M_{ij} = \sigma'_{ij}\langle x_i,\hat{\delta}_j\rangle$, we see that the previous term is simply $x^TM_i^T$, so that we get the upper bound on the norm of the term:
\begin{equation}
    \sum_{i=1}^n \|x^TM_i^T\|^2 \leq \|x\|_F^2 \|M\|_F^2 
\end{equation}
and $\|M\|_F^2 = \sum_{ij}(\sigma'_{ij})^2\langle x_i,\hat{\delta}_j\rangle^2 \leq \frac{1}{n^2} \sigma'_{\infty}\|x\|_F^2\|A\|_2^2\|\delta\|_F^2$, giving overall:
\begin{equation}
    \sqrt{\sum_{i=1}^n \|x^TM_i^T\|^2} \leq \sigma'_{\infty} \|A\|_2\mathbb{E}[\|x_j\|^2]\|\delta\|_F .
\end{equation}
Notice how this quantity matches the one in \cref{eqn:lip_bound_first}.

Finally, summing all together gives:
\begin{equation}
    \|\mathrm{Jac}\|_2\leq 2\sigma'_{\infty} \|A\|_2\mathbb{E}[\|x_j\|^2] + \sigma_{\infty},
\end{equation}
which completes the proof.

\textbf{Remark}: The previous upper bound might not be tight. Indeed, intuitively, if the $x_i$ are large, then the term $\sigma'_{ij}$ should be exponentially small (provided, of course, that $W_qx_i$ and $W_kx_j$ are not orthogonal), which would even remove the dependency on the variance in the sigmoid attention. 

