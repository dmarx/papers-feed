\documentclass[11pt]{report}
\usepackage[margin=2cm]{geometry}
\usepackage{graphicx}
\usepackage{float}
\usepackage{times}

\usepackage[dvipsnames]{xcolor}

\newcommand{\Gap}{\texorpdfstring{\hfill}{}}
\newcommand{\Rec}{\texorpdfstring{{\small\emph{\color{blue}{\fbox{High Leverage}}}}}{}}
\newcommand{\HighRisk}{\texorpdfstring{{\small\emph{\color{orange}{\fbox{Uncertain Impact}}}}}{}}
\newcommand{\Longterm}{\texorpdfstring{{\small\emph{\color{OliveGreen}{\fbox{Long-term}}}}}{}}

\begin{document}

\section{Finance\texorpdfstring{\hfill\textit{by Alexandra Luccioni}}{}}
\label{sec:finance}

The rise and fall of financial markets is linked to many events, both sporadic (e.g.~the 2008 global financial crisis) and cyclical (e.g.~the price of gas over the years), with profits and losses that can be measured in the billions of dollars and can have global consequences. Climate change poses a substantial financial risks to global assets measured in the trillions of dollars \cite{dietz2016}, and it is hard to forecast where, how, or when climate change will impact the stock price of a given company, or even the debt of an entire nation. While financial analysts and investors focus on pricing risk and forecasting potential earnings, the majority of the current financial system is based on quarterly or yearly performance. This fails to incentivize the prediction of medium or long-term risks, which include most climate change-related exposures such as physical impacts on assets or distribution chains, legislative impacts on profit generation, and indirect market consequences such as supply and demand\footnote{For further reading regarding the impact of climate change on financial markets, see~\cite{boissinot2016, battiston2017,campiglio2018}.}.

\paragraph*{Climate investment}
\label{sec:climate-investment}\mbox{}\\
\emph{Climate investment}, the current dominant approach in climate finance, involves investing money in low-carbon assets~\cite{eyraud2013}. The dominant ways in which major financial institutions take this approach are by creating ``green'' financial indexes that focus on low-carbon energy, clean technology, and/or environmental services~\cite{diaz2017} or by designing carbon-neutral investment portfolios that remove or under-weight companies with relatively high carbon footprints \cite{gianfrate2018}. This investment strategy is creating major shifts in certain sectors of the market (e.g.~utilities and energy) towards renewable energy alternatives, which are seen as having a greater growth potential than traditional energy sources such as oil and gas~\cite{bergmann2006}. While this approach currently does not utilize ML directly, we see the potential in applying deep learning both for portfolio selection (based on features of the stocks involved) and investment timing (using historical patterns to predict future demand), to maximize both the impact and scope of climate investment strategies.

\paragraph*{Climate analytics}\Gap\Rec\label{sec:climate-analytics}\mbox{}\\
The other main approach to climate finance is \emph{climate analytics}, which aims to predict the financial effects of climate change, and is still gaining momentum in the mainstream financial community~\cite{eyraud2013}.
Since this is a predictive approach to addressing climate change from a financial perspective, it is one where ML can potentially have greater impact. 
Climate analytics involves analyzing investment portfolios, funds, and companies in order to pinpoint areas with heightened risk due to climate change, such as timber companies that could be bankrupted by wildfires or water extraction initiatives that could see their sources polluted by shifting landscapes. Approaches used in this field include:~natural language processing techniques for identifying climate risks and investment opportunities in disclosures made by companies~\cite{stanny2008} as well as for analyzing the evolution of climate coverage in the media to dynamically hedge climate change risk~\cite{engle2019}; econometric approaches for developing arbitrage strategies that take advantage of the carbon risk factor in financial markets~\cite{andersson2016}; and ML approaches for forecasting the price of carbon in emission exchanges\footnote{Carbon pricing, e.g.~via CO$_2$~cap-and-trade or a carbon tax, is a commonly-suggested policy approach for getting firms to price future climate change impacts into their financial calculations. For an introduction to these topics, see \cite{pizer2006choosing} and also \S\ref{subsec:markets}.}~\cite{zhu2017, zhou2018}. 

To date, the field of climate finance has been largely neglected within the larger scope of financial research and analysis. This leaves many directions for improvement, such as (1) improving existing traditional portfolio optimization approaches; (2) in-depth modeling of variables linked to climate risk; (3) designing a statistical climate factor that can be used to project the variation of stock prices given a compound set of events; and (4) identifying direct and indirect climate risk exposure in annual company reports. ML plays a central role in these strategies, and can be a powerful tool in leveraging the financial sector to mitigate climate change and in reducing the financial impacts of climate change on society.
\end{document}

