\documentclass[11pt]{report}
\usepackage[margin=2cm]{geometry}
\usepackage{graphicx}
\usepackage{float}
\usepackage{times}
\usepackage[dvipsnames]{xcolor}

\begin{document}
\part*{Conclusion}
\addcontentsline{toc}{part}{Conclusion}

Machine learning, like any technology, does not always make the world a better place --- but it can. In the fight against climate change, we have seen that ML has significant contributions to offer across domain areas. ML can enable automatic monitoring through remote sensing (e.g.~by pinpointing deforestation, gathering data on buildings, and assessing damage after disasters). It can accelerate the process of scientific discovery (e.g.~by suggesting new materials for batteries, construction, and carbon capture). ML can optimize systems to improve efficiency (e.g.~by consolidating freight, designing carbon markets, and reducing food waste). And it can accelerate computationally expensive physical simulations through hybrid modeling (e.g.~climate models and energy scheduling models). These and other cross-cutting themes are shown in Table \ref{tab:themes}. We emphasize that in each application, ML is only one part of the solution; it is a tool that enables other tools across fields.

Applying machine learning to tackle climate change has the potential both to benefit society and to advance the field of machine learning. Many of the problems we have discussed here highlight cutting-edge areas of ML, such as interpretability, causality, and uncertainty quantification. Moreover, meaningful action on climate problems requires dialogue with fields within and outside computer science and can lead to interdisciplinary methodological innovations, such as improved physics-constrained ML techniques.

The nature of climate-relevant data poses challenges and opportunities. For many of the applications we identify, data can be proprietary or include sensitive personal information. Where datasets exist, they may not be organized with a specific task in mind, unlike typical ML benchmarks that have a clear objective. Datasets may include information from  heterogeneous sources, which must be integrated using domain knowledge. Moreover, the available data may not be representative of global use cases. For example, forecasting weather or electricity demand in the US, where data are abundant, is very different from doing so in India, where data can be scarce. Tools from transfer learning and domain adaptation will likely prove essential in low-data settings. For some tasks, it may also be feasible to augment learning with carefully simulated data. Of course, the best option if possible is always more real data; we strongly encourage public and private entities to release datasets and to solicit involvement from the ML community.

\begin{table}
\begin{small}
\begin{center}
\begin{tabular}{l l l l l l l l l l }  \toprule
     \multicolumn{2}{l}{ }
         & \small{\rotatebox{90}{\parbox{2.4cm}{Accelerated\\experimentation}}}
         & \small{\rotatebox{90}{\parbox{2.4cm}{Control systems}}}
         & \small{\rotatebox{90}{\parbox{2.4cm}{Forecasting}}}
         & \small{\rotatebox{90}{\parbox{2.4cm}{Human\\interaction}}}
         & \small{\rotatebox{90}{\parbox{2.4cm}{Hybrid physical\\models}}}
         & \small{\rotatebox{90}{\parbox{2.4cm}{Predictive\\maintenance}}}
         & \small{\rotatebox{90}{\parbox{2.4cm}{Remote sensing}}}
         & \small{\rotatebox{90}{\parbox{2.4cm}{System\\optimization}}}
 %        & \small{\rotatebox{90}{\parbox{2.7cm}{Human interaction}}}
    \\ \midrule
    \rowcolor{ccai-yellow}
    \multicolumn{2}{l}{1 \hyperref[sec:electricity-systems]{Electricity systems}} 
         & %Accelerated\\experimentation
         & %Control systems
         & %Forecasting
         &
         & %Hybrid physical\\models
         & %Predictive\\maintenance}         
         & %Remote sensing
         &\\ %System optimization
    & \hyperref[sec:electricity-lowCarbon]{Enabling low-carbon electricity}
         & $\bullet$%Accelerated\\experimentation
         & $\bullet$%Control systems
         & $\bullet$%Forecasting
         &
         & $\bullet$%Hybrid physical\\models
         & $\bullet$%Predictive\\maintenance}         
         & $\bullet$%Remote sensing
         & $\bullet$\\%System optimization
    & \hyperref[sec:electricity-currentSystemImpact]{Reducing current-system impacts}
         & %Accelerated\\experimentation
         & %Control systems
         & $\bullet$%Forecasting
         &
         & %Hybrid physical\\models
         & $\bullet$%Predictive\\maintenance}         
         & $\bullet$%Remote sensing
         & \\%System optimization
    & \hyperref[sec:electricity-developing]{Ensuring global impact}
         & %Accelerated\\experimentation
         & %Control systems
         & $\bullet$%Forecasting
         &
         & $\bullet$%Hybrid physical\\models
         & %Predictive\\maintenance}         
         & $\bullet$%Remote sensing
         & \\%System optimization
    \rowcolor{ccai-yellow}
    \multicolumn{2}{l}{2 \hyperref[sec:transportation]{Transportation}} 
         & %Accelerated\\experimentation
         & %Control systems
         & %Forecasting
         &
         & %Hybrid physical\\models
         & %Predictive\\maintenance}         
         & %Remote sensing
         & \\%System optimization
    & \hyperref[sec:TReducing]{Reducing transport activity}
         & $\bullet$%Accelerated\\experimentation
         & $\bullet$%Control systems
         & $\bullet$%Forecasting
         &
         & %Hybrid physical\\models
         & %Predictive\\maintenance}         
         & $\bullet$%Remote sensing
         & $\bullet$\\%System optimization
   & \hyperref[sec:TEfficient]{Improving vehicle efficiency}
         & $\bullet$%Accelerated\\experimentation
         & $\bullet$%Control systems
         & %Forecasting
         &
         & %Hybrid physical\\models
         & %Predictive\\maintenance}         
         & %Remote sensing
         & $\bullet$ \\ %System optimization
   & \hyperref[sec:TFuels]{Alternative fuels \& electrification}
         & $\bullet$%Accelerated\\experimentation
         & $\bullet$%Control systems
         & $\bullet$%Forecasting
         &
         & %Hybrid physical\\models
         & %Predictive\\maintenance}         
         & %Remote sensing
         &$\bullet$\\ %System optimization
   & \hyperref[sec:modalshift]{Modal shift}
         & %Accelerated\\experimentation
         & $\bullet$%Control systems
         & $\bullet$%Forecasting
         & $\bullet$ % human
         & %Hybrid physical\\models
         & $\bullet$%Predictive\\maintenance}         
         & $\bullet$%Remote sensing
         & $\bullet$\\ %System optimization
    \rowcolor{ccai-yellow}
    \multicolumn{2}{l}{3 \hyperref[sec:buildings-cities]{Buildings and cities}} 
         & %Accelerated\\experimentation
         & %Control systems
         & %Forecasting
         &
         &  %Hybrid physical\\models
         & %Predictive\\maintenance}         
         & %Remote sensing
         &\\ %System optimization
    & \hyperref[sec:indv]{Optimizing buildings}
         & %Accelerated\\experimentation
         & $\bullet$%Control systems
         & $\bullet$%Forecasting
         &
         & $\bullet$%Hybrid physical\\models
         & $\bullet$%Predictive\\maintenance}         
         & %Remote sensing
         &$\bullet$\\ %System optimization
    & \hyperref[sec:distr]{Urban planning}
         & %Accelerated\\experimentation
         & %Control systems
         & %Forecasting
         &
         & %Hybrid physical\\models
         & %Predictive\\maintenance}         
         & $\bullet$%Remote sensing
         &\\ %System optimization
    & \hyperref[sec:cities]{The future of cities}
         & %Accelerated\\experimentation
         & %Control systems
         & %Forecasting
         &
         & %Hybrid physical\\models
         & %Predictive\\maintenance}         
         & %Remote sensing
         &$\bullet$\\ %System optimization
    \rowcolor{ccai-yellow}
    \multicolumn{2}{l}{4 \hyperref[sec:industry]{Industry}} 
         & %Accelerated\\experimentation
         & %Control systems
         & %Forecasting
         &
         & %Hybrid physical\\models
         & %Predictive\\maintenance}         
         & %Remote sensing
         &\\ %System optimization
    & \hyperref[sec:supplychains]{Optimizing supply chains}
         & %Accelerated\\experimentation
         & $\bullet$%Control systems
         &$\bullet$ %Forecasting
         &
         & %Hybrid physical\\models
         & %Predictive\\maintenance}         
         & %Remote sensing
         &$\bullet$\\ %System optimization
    & \hyperref[sec:materialsandconstruction]{Improving materials}
         & $\bullet$%Accelerated\\experimentation
         & %Control systems
         & %Forecasting
         &
         & %Hybrid physical\\models
         & %Predictive\\maintenance}         
         & %Remote sensing
         &\\ %System optimization
    & \hyperref[sec:demandresponse]{Production \& energy}
         & %Accelerated\\experimentation
         & $\bullet$%Control systems
         & %Forecasting
         &
         & %Hybrid physical\\models
         & $\bullet$%Predictive\\maintenance}         
         & %Remote sensing
         &$\bullet$\\ %System optimization
    \rowcolor{ccai-yellow}
    \multicolumn{2}{l}{5 \hyperref[sec:afolu]{Farms \& forests}} 
         & %Accelerated\\experimentation
         & %Control systems
         & %Forecasting
         &
         & %Hybrid physical\\models
         & %Predictive\\maintenance}         
         &%Remote sensing
         &\\ %System optimization
    & \hyperref[sec:emissions-detection]{Remote sensing of emissions}
         & %Accelerated\\experimentation
         & %Control systems
         & %Forecasting
         &
         & %Hybrid physical\\models
         & %Predictive\\maintenance}         
         &$\bullet$ %Remote sensing
         &\\ %System optimization
    & \hyperref[sec:agriculture]{Precision agriculture}
         & %Accelerated\\experimentation
         & $\bullet$%Control systems
         & $\bullet$%Forecasting
         &
         & %Hybrid physical\\models
         & %Predictive\\maintenance}         
         & $\bullet$%Remote sensing
         &\\ %System optimization
    & \hyperref[sec:peatlands]{Monitoring peatlands}
         & %Accelerated\\experimentation
         & %Control systems
         & %Forecasting
         &
         & %Hybrid physical\\models
         & %Predictive\\maintenance}         
         &$\bullet$ %Remote sensing
         &\\ %System optimization
    & \hyperref[sec:forests]{Managing forests}
         & %Accelerated\\experimentation
         & $\bullet$%Control systems
         & $\bullet$%Forecasting
         &
         & %Hybrid physical\\models
         & %Predictive\\maintenance}         
         &$\bullet$ %Remote sensing
         &\\ %System optimization
    \rowcolor{ccai-yellow}
    \multicolumn{2}{l}{6 \hyperref[sec:ccs]{Carbon dioxide removal}}
         & %Accelerated\\experimentation
         & %Control systems
         & %Forecasting
         &
         & %Hybrid physical\\models
         & %Predictive\\maintenance}         
         & %Remote sensing
         &\\ %System optimization
    & \hyperref[sec:ccs]{Direct air capture}
         & $\bullet$%Accelerated\\experimentation
         & %Control systems
         & %Forecasting
         &
         & $\bullet$%Hybrid physical\\models
         & %Predictive\\maintenance}         
         & %Remote sensing
         &\\ %System optimization
    & \hyperref[subsubsec: sequestrativervin]{Sequestering~\cd}
         & %Accelerated\\experimentation
         &%Control systems
         & %Forecasting
         &
         & $\bullet$%Hybrid physical\\models
         & %Predictive\\maintenance}         
         & %Remote sensing
         & \\%System optimization
    \rowcolor{ccai-yellow}\multicolumn{2}{l}{7 \hyperref[sec: climate prediction]{Climate prediction}}
         & %Accelerated\\experimentation
         & %Control systems
         & %Forecasting
         &
         & %Hybrid physical\\models
         & %Predictive\\maintenance}         
         & %Remote sensing
         & \\%System optimization
    & \hyperref[sec:climate-models-params]{Uniting data, ML \& climate science}
         & %Accelerated\\experimentation
         & %Control systems
         & $\bullet$%Forecasting
         &
         & $\bullet$%Hybrid physical\\models
         & %Predictive\\maintenance}         
         & $\bullet$%Remote sensing
         & \\%System optimization
    & \hyperref[sec:models-extreme-events]{Forecasting extreme events}
         & %Accelerated\\experimentation
         & %Control systems
         &$\bullet$ %Forecasting
         &
         & $\bullet$%Hybrid physical\\models
         & %Predictive\\maintenance}         
         & $\bullet$%Remote sensing
         & \\%System optimization
    \rowcolor{ccai-yellow}
    \multicolumn{2}{l}{8 \hyperref[sec:societal-impacts]{Societal impacts}} 
         & %Accelerated\\experimentation
         & %Control systems
         & %Forecasting
         &
         & %Hybrid physical\\models
         & %Predictive\\maintenance}         
         & %Remote sensing
         &\\ %System optimization
    & \hyperref[subsub:ecology]{Ecology}
         & %Accelerated\\experimentation
         & %Control systems
         & %Forecasting
         &
         & %Hybrid physical\\models
         & %Predictive\\maintenance}         
         & $\bullet$%Remote sensing
         &\\ %System optimization
    & \hyperref[subsub:infrastructure]{Infrastructure}
         & %Accelerated\\experimentation
         & %Control systems
         & %Forecasting
         &
         & %Hybrid physical\\models
         & $\bullet$%Predictive\\maintenance}         
         & %Remote sensing
         &$\bullet$\\ %System optimization
    & \hyperref[subsub:social_systems]{Social systems}
         & %Accelerated\\experimentation
         & %Control systems
         & $\bullet$%Forecasting
         & $\bullet$
         & %Hybrid physical\\models
         & %Predictive\\maintenance}         
         & $\bullet$%Remote sensing
         & $\bullet$\\%System optimization
    & \hyperref[subsub:crisis]{Crisis}
         & %Accelerated\\experimentation
         & %Control systems
         & $\bullet$%Forecasting
         &
         & %Hybrid physical\\models
         & %Predictive\\maintenance}         
         & $\bullet$%Remote sensing
         & \\%System optimization
    \rowcolor{ccai-yellow}
    \multicolumn{2}{l}{9 \hyperref[sec:geoengineering]{Solar geoengineering}} 
         & %Accelerated\\experimentation
         & %Control systems
         & %Forecasting
         &
         & %Hybrid physical\\models
         & %Predictive\\maintenance}         
         & %Remote sensing
         & \\%System optimization
    & \hyperref[subsub:better-aerosols]{Understanding \& improving aerosols}
         & %Accelerated\\experimentation
         & $\bullet$%Control systems
         & %Forecasting
         &
         & $\bullet$%Hybrid physical\\models
         & %Predictive\\maintenance}         
         & %Remote sensing
         & \\%System optimization
    & \hyperref[subsub:planetary-control]{Engineering a planetary control system}
         & %Accelerated\\experimentation
         & $\bullet$%Control systems
         & %Forecasting
         &
         & $\bullet$%Hybrid physical\\models
         & %Predictive\\maintenance}         
         & %Remote sensing
         & \\%System optimization
    & \hyperref[subsub:impact-models]{Modeling impacts}
         & %Accelerated\\experimentation
         & %Control systems
         & %Forecasting
         &
         & $\bullet$%Hybrid physical\\models
         & %Predictive\\maintenance}         
         & %Remote sensing
         & \\%System optimization
    \rowcolor{ccai-yellow}
    \multicolumn{2}{l}{10 \hyperref[sec:tools-individuals]{Individual action}} 
         & %Accelerated\\experimentation
         & %Control systems
         & %Forecasting
         & 
         & %Hybrid physical\\models
         & %Predictive\\maintenance}         
         & %Remote sensing
         & \\%System optimization
    & \hyperref[sec:personal_carbon_footprint]{Understanding personal footprint}
         & %Accelerated\\experimentation
         & %Control systems
         & $\bullet$%Forecasting
         & $\bullet$
         & %Hybrid physical\\models
         & %Predictive\\maintenance}         
         & %Remote sensing
         & \\%System optimization
    & \hyperref[sec:behavior_change]{Facilitating behavior change}
         & %Accelerated\\experimentation
         & %Control systems
         & %Forecasting
         & $\bullet$
         & %Hybrid physical\\models
         & %Predictive\\maintenance}         
         & %Remote sensing
         & \\%System optimization
    \rowcolor{ccai-yellow}
    \multicolumn{2}{l}{11 \hyperref[sec:toolsforsociety]{Collective decisions}} 
         & %Accelerated\\experimentation
         & %Control systems
         & %Forecasting
         & 
         & %Hybrid physical\\models
         & %Predictive\\maintenance}         
         & %Remote sensing
         & \\%System optimization
    & \hyperref[sec:coordination]{Modeling social interactions}
         & %Accelerated\\experimentation
         & %Control systems
         & %Forecasting
         & $\bullet$
         & %Hybrid physical\\models
         & %Predictive\\maintenance}         
         & %Remote sensing
         & \\%System optimization
    & \hyperref[sec:decisionmaking]{Informing policy}
         & %Accelerated\\experimentation
         & %Control systems
         & %Forecasting
         & $\bullet$
         & %Hybrid physical\\models
         & %Predictive\\maintenance}         
         & %Remote sensing
         & \\%System optimization
    & \hyperref[subsec:markets]{Designing markets}
         & %Accelerated\\experimentation
         & %Control systems
         & $\bullet$%Forecasting
         &
         & %Hybrid physical\\models
         & %Predictive\\maintenance}         
         & %Remote sensing
         & $\bullet$\\%System optimization
    \rowcolor{ccai-yellow}
    \multicolumn{2}{l}{12 \hyperref[sec:education]{Education}} 
         & %Accelerated\\experimentation
         & %Control systems
         & %Forecasting
         & $\bullet$
         & %Hybrid physical\\models
         & %Predictive\\maintenance}         
         & %Remote sensing
         & \\%System optimization
    \rowcolor{ccai-yellow}
    \multicolumn{2}{l}{13 \hyperref[sec:finance]{Finance}} 
         & %Accelerated\\experimentation
         & %Control systems
         & $\bullet$%Forecasting
         & $\bullet$
         & %Hybrid physical\\models
         & %Predictive\\maintenance}         
         & %Remote sensing
         & \\%System optimization
    \bottomrule
\end{tabular}
\caption{Cross-cutting objectives that are relevant to many climate change domains.}
\label{tab:themes}
\end{center}
\end{small}
\end{table}


For those who want to apply ML to climate change, we provide a roadmap:
\begin{itemize}
    \item \textbf{Learn.} Identify how your skills may be useful -- we hope this paper is a starting point.
    \item \textbf{Collaborate.} Find collaborators, who may be researchers, entrepreneurs, established companies, or policy makers. Every domain discussed here has experts who understand its opportunities and pitfalls, even if they do not necessarily understand ML.
    \item \textbf{Listen.} Listen to what your collaborators and other stakeholders say is needed. Groundbreaking technologies have an impact, but so do well-constructed solutions to mundane problems.
    \item \textbf{Deploy.} Ensure that your work is deployed where its impact can be realized.
\end{itemize}
We call upon the machine learning community to use its skills as part of the global effort against climate change.

\end{document}
