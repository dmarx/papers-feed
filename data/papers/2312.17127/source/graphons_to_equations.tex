\newcommand{\MonadT}{\mathcal{T}}
\newcommand{\MonadP}{\mathcal{P}}
\newcommand{\MonadF}{\mathcal{F}}



\paragraph{Measures on the space of vertices}\label{sec:er-meas}
We define an internal probability measure
(Def.~\ref{def:probability}) on the space~$\VV$ of
vertices, which, we will show, corresponds to the Erd\H{o}s-R\'enyi
graphon.
Fix $α\in[0,1]$, the chance of an edge. 

We define the measure $\nu_α$ of a definable set $S\in 2^\VV$ as follows.
Suppose that $S$ has support $\{a_1,\dots, a_n\}$.
We choose an enumeration of vertices $(v_1,\dots,v_{2^n})$ in $\VV$
(disjoint from $\{a_1,\dots,a_n\}$)
that covers all the $2^n$ possible edge relationships that a vertex
could have with the $a_i$'s. (For example, $v_1$ has no edges to any $a_i$, and $v_{2^n}$ has an edge to
every $a_i$, and the other $v_j$'s have the other possible edge relationships.) Let:
\begin{equation}\label{eqn:graphon-to-measure}
  \nu_α(S) =\sum_{j=1}^{2^n}[v_j\in S] \prod_{i=1}^n \big(α E(v_j,a_i)+(1-α)(1-E(v_j,a_i))\big)\text. 
\end{equation}
%\todo[inline]{Simplify? This looks more complicated than it is.}
\begin{proposition}
  The assignment given in \eqref{eqn:graphon-to-measure}
  is an internal probability measure (Def.~\ref{def:probability}) on $\VV$. 
\end{proposition}
\begin{proof}
The function $\nu_α$ is well-defined: it does not depend on the choice of $v_j$'s (by Prop.~\ref{prop:radonom-definable-sets}), nor on the choice of support (by direct calculation).
% This is the simple calculation:
% Suppose that for $k$, $a_k$ is not in the minimal support of $S$. Then for all $j$, $[v_j \in S]$ is independent on the connectivity $E(v_j, a_k)$. Without loss of generality, let $k = n$. Then, denoting 
% \[
% \pi^{j}_{k} = \prod_{i=1}^k \big(p\cdot E(v_j,a_i)+(1-p)\cdot (1-E(v_j,a_i))\big)
% \] we have
% \begin{align*}
% \nu_α&(S)_{\langle a_1\dots a_n\rangle} = \sum_{j=1}^{2^{n}}[v_j\in S]\cdot \pi^{j}_{n} \\ 
%      &= \sum_{j=1}^{2^{n-1}}[v_j\in S]\cdot p \cdot \pi^{j}_{{n-1}} + \sum_{j=1}^{2^{n-1}}[v_j\in S]\cdot (1-p) \cdot \pi^{j}_{{n-1}} \\
% &= \sum_{j=1}^{2^{n-1}}[v_j\in S]\cdot \pi^{j}_{{n-1}} = \nu_α(S)_{\langle a_1\dots a_{n-1}\rangle}
% \end{align*}
It is equivariant, since for $\sigma \bullet S$, a valid enumeration of vertices is given by $\sigma \bullet v_1, \dots \sigma \bullet v_{2^{n}}$. Also, $\nu(\VV) = 1$, since $\VV$ has empty support. Internal countable additivity follows from the identity $
\left[v_j \in \biguplus_{i=1}^\infty S_i\right] = \sum_{i=1}^{\infty}[v_j \in S_i]
$.
\end{proof}

    \paragraph*{Remark}
The definitions and results of this section appear to be novel. 
    However, the general idea of considering measures on formulas which are invariant to substitutions that permute the variables
goes back to work of Gaifman \cite{MR175755}.
The paper~\cite{MR3515800}
characterizes those countably infinite graphs that can arise with probability~$1$ in that framework;
see \cite{properly-ergodic}
for a discussion of how Gaifman's work connects to
Prop.~\ref{prop:lovasz}.
%%a distribution of a
%exchangeable countable random graph is determined by the probabilities assigned to each finite %subgraph.

\subsection{Nominal Probability Monads}
%\subsubsection{Rudiments of Nominal Probability Theory}
\label{sec:nom-monad}
Since $\RadoNom$ is a Boolean topos with natural numbers object
(Prop.~\ref{prop:radonom-topos}), we can interpret measure-theoretic
notions in the internal language of the topos, as long as they do not
require the axiom of choice. We now spell out the resulting development, without assuming familiarity with topos theory. By doing this, we build new probability
monads on $\RadoNom$. % following development. %nstantiating them to the Rado-nominal setting.

\subsubsection{Finitely Supported Functions and Measures}
Let $X$ and $Y$ be Rado-nominal sets.
The set of all functions $X\to Y$ has an action of $\AutRado$,
given by $(\sigma\bullet f)(x)=\sigma\inv\bullet (f(\sigma\bullet
x))$. The function space $[X\Rightarrow Y]$ comprises those
functions that have finite support under this action. Categorically,
this structure is uniquely determined by the `currying' bijection, natural in~$Z$:
\[
  \RadoNom(Z\times X,Y)\cong \RadoNom(Z,X\Rightarrow Y)\text .
\]
(For example, the powerobject $2^X$ (\S\ref{sec:powersets-and-definiable-sets}) can be regarded as
$[X\Rightarrow 2]$, if we regard a set as its characteristic
function.)

In Def.~\ref{def:probability}, we focused on equivariant
probability measures.
We generalize this to finitely supported measures. For example, pick a
vertex $a\in \VV$. Then, the Dirac measure on $\VV$ (i.e.~$\delta_a(S)=1$ if $a\in S$,
and $\delta_a(S)=0$ if $a\not\in S$) has support $\{a\}$.

\begin{definition}
\label{def:finitely-supported-probability-measure}
For a Rado-nominal set $X$, let $\MonadP (X)$ comprise the
    finitely supported functions $\mu : 2^X\to[0,1]$ that are
    internally countably additive, and satisfy $\mu(X) = 1$.
    This is a Rado-nominal set, as a subset of
    $[2^X\Rightarrow[0,1]]$. Functions in $\MonadP (X)$
    are called \emph{finitely supported probability measures}.
\end{definition}

\subsubsection{Internal Integration}
We revisit some basic integration theory in this nominal setting.
In traditional measure theory, one can define the Lebesgue integral of a \emph{measurable} function ${f : X \to [0,1]}$ by
$
  \int f(x) \mu(\dd x) = \sup \sum_{i = 1}^n r_i \mu(U_i)
$
where the supremum ranges over simple functions ${\sum_i r_i [- \in U_i]}$ with $U_i$ measurable in $X$ and bounded above by $f$ (\S\ref{sec:probthy}).
The same construction works in the internal logic of $\RadoNom$.

Note that the following does not mention $f$ being measurable: since $X$ is considered to have its internal powerset $\sigma$-algebra, finite-supportedness %trivially 
implies `measurability' here.

\begin{proposition}\label{prop:internal-integration}
  Let $\mu\in \MonadP (X)$ be a finitely supported probability measure
  on $X$.
  For any finitely supported function $f\colon X\to [0,1]$,
  the internally-constructed Lebesgue integral $\int f(x)\,\mu(\dd x) \in [0,1]$ exists.
  Moreover, integration is an equivariant map
  \[
    \int : \MonadP (X)\times [X\Rightarrow [0,1]]\to [0,1]
  \]
  which preserves suprema of internally countable monotone sequences in its second argument.
  \end{proposition}
  \begin{proof}
    If $U_1,\ldots,U_n \subseteq X$ are finitely supported, $r_1,\ldots,r_n \in [0,1]$, and $\sum_i r_i [- \in U_i] \leq f$, then by ordinary additivity of $\mu$, we have $\sum r_i \mu(U_i) \in [0,1]$.
    By ordinary real analysis, the supremum of all such values exists and is in $[0,1]$.
    For equivariance, recall that $[0,1]$ is equipped with the trivial action of $\AutRado$. Use the fact that $\sum_i r_i [- \in U_i] \leq f$ if and only if $\sum_i r_i [- \in \sigma\bullet U_i] \leq \sigma \bullet f$.
    The last claim is the monotone convergence theorem internalized to $\RadoNom$.
    % Finally we remark that if $E \subseteq [0,1]$ the support of $f^{-1}(E)$ is bounded by the support of $f$, so that $\int f(x) \mu(\dd x)$ is indeed the supremum of a internally countable sequence of integrals of simple functions.
  \end{proof}

\subsubsection{Kernels and a Monad}
  We can regard a `probability kernel' as a finitely supported function
  $k:X\to \MonadP (Y)$. Equivalently, $k$ is a finitely supported function
  $k:X\times 2^Y\to[0,1]$ that is countably additive and has mass $1$ in its second
  argument. 

  (In traditional measure theory, one would explicitly ask that $k$ is measurable
  in its first argument, but as we observed, finite-supportedness already implies it.)
  
  As usual, probability kernels compose, and this allows us to regard
  them as Kleisli morphisms for a monad (Def.~\ref{def:monad}), defined as follows.
\begin{definition}\label{def:monad-radonom}
  We define the strong monad $\MonadP$ on $\RadoNom$ as follows.
  \begin{itemize}
  \item For a Rado-nominal set $X$, let $\MonadP (X)$ comprise the
    finitely supported probability measures (Def.~\ref{def:finitely-supported-probability-measure}).
  \item
    The unit of the monad $\eta_X:X\to \MonadP (X)$ is the Dirac measure,
    $\eta_X(x)(S)=[x\in S]$.
  \item
    The bind $(\bind) \colon \MonadP (X)\times (X\Rightarrow \MonadP (Y))\to \MonadP (Y)$ is given
    by
    \[
      (\mu \bind k)(S)=\int_X k(x,S)\,\mu(\dd x).
    \]
  \end{itemize}
\end{definition}
We note that this is similar to the `expectations monad'~\cite[Thm.~4]{EPTCS95.12}. 
\subsubsection{Commuting Integrals (Fubini)}
  \label{sec:fubini}
  For measures $\mu_1\in \MonadP (X)$ and $\mu_2\in \MonadP (Y)$, the
  monad structure allows us to define a product measure
  \begin{equation}
    \begin{aligned}
    & \mu_1\otimes\mu_2 = \big(\mu_1\bind (\lambda x.\,\mu_2\bind\,\lambda
    y.\,\eta(x,y))\big)
    \\
    & \int f(x,y)\,(\mu_1\otimes \mu_2)(\dd (x,y)) =
    \int\int f(x,y)\,\mu_2(\dd y)\,\mu_1(\dd x)\text .
  \end{aligned}
\end{equation}
Although this iterated integration is reminiscent of the traditional
approach, in general we cannot reorder integrals (`Fubini does not
hold'). For example, given two measures $\nu_α$ and $\nu_β$ for $α \neq β$ and
$f$ being the characteristic function of the set $\{(x, y): E(x, y)\} \subseteq \VV^2$, we have
\begin{equation}
\begin{aligned}
& \int \int [E(x, y)]\,\nu_α(\dd y)\,\nu_β(\dd x) = \int α\, \nu_β(\dd x) = α  \\
&\;\; {} \neq β = \int β\, \nu_α(\dd y) = \int \int [E(x, y)]\,\nu_β(\dd x)\,\nu_α(\dd y) \text .
\end{aligned}
\end{equation}
However, it does hold when we consider only copies of the same measure.
\begin{proposition}\label{thm:graphon:commutative}
  For $\nu_α\in \MonadP (\VV)$ as in (\ref{eqn:graphon-to-measure}), $\nu_α$ commutes with
  $\nu_α$. That is, for any finitely supported $f:\VV\times\VV\to [0,1]$,
  \[     \int \int f(x,y)\,\nu_α(\dd y)\,\nu_α(\dd x)
    =
       \int\int f(x,y)\,\nu_α(\dd x)\,\nu_α(\dd y)
    \text.\]
\end{proposition}
\begin{proof}[Proof notes]
  By Prop.~\ref{prop:radonom-definable-sets} and \ref{prop:internal-integration}, it suffices to check on the indicator functions of definable subsets of $\VV^2$.
  The indicators of sets $\{(x,y) \mid \Phi(x,y)\}$ where $\Phi(x,y)$ is a disjunction of $x = y$, $x = a$, or $y = a$ for some $a \in \VV$ are seen to have integral $0$ on both sides.
  The remaining possibilities can be reduced to 
  the case where $\Phi_{A,\phi,\psi,\epsilon}(x,y)$ is $(x,y \notin A) \wedge (x \neq y) \wedge (E(x,y) \leftrightarrow \epsilon) \wedge \bigwedge_{a \in A} (E(a,x) \leftrightarrow \phi_a) \wedge (E(a,y) \leftrightarrow \psi_a)$
  where $A \subseteq \VV$ is a finite set, $\epsilon \in \{\bot,\top\}$, and $\phi,\psi \in \{\bot,\top\}^A$.
  This formula corresponds to choosing a two-vertex extension of the finite graph spanned by $A \subseteq \VV$.
  Intuitively, the two double integrals correspond to the two alternative two-step computations of the conditional probability of extending the graph $A$ to this extension according to which of the two vertices is sampled first, and indeed both evaluate to $α^k (1-α)^{2|A|+1 - k}$ where $k = [\epsilon] + \sum_{a \in A} ([\phi_a] + [\psi_a])$.
\end{proof}

\paragraph*{Remark} In traditional measure theory, iterated integrals are defined using product $\sigma$-algebras. Here we have not constructed product $\sigma$-algebras, but rather always take the internal powerset as the $\sigma$-algebra. This allows us to view all the definable sets as measurable on $\VV^n$ (Prop.~\ref{prop:radonom-definable-sets}), which is very useful. We remark that alternative product spaces also arise in non-standard approaches to graphons (see~\cite[\S6]{tao-graphons} for an overview), and also in quasi-Borel spaces~\cite{qbs} for different reasons.

\subsubsection{A Commutative Monad}
  \label{sec:comm-monad}
  We now use Prop.~\ref{thm:graphon:commutative} to build a commutative affine submonad $\MonadP_α$ of the
  monad~$\MonadP$,
  which we will use to model the graph interface for the probabilistic programming language.
  With Prop.~\ref{prop:radonom-topos}, we use the following general result.
  \begin{proposition}%[e.g.~\cite{kammar-mcdermott}, \S2, and \cite{kelly-power}]
    Let $\MonadT$ be a strong monad on a Grothendieck topos. Consider a family of morphisms
    $\{f_i\colon X_i\to \MonadT (Y_i)\}_{i\in I}$.
    \begin{itemize}
    \item There is a least strong submonad $\MonadT_{f}\subseteq \MonadT$ through which all
      $f_i$ factor.
    \item If the morphisms $f_i$ all commute with each other, then
      $\MonadT_f$ is a commutative monad (Def.~\ref{def:affine-monad}).
    \end{itemize}
  \end{proposition}
  \begin{proof}[Proof notes]\newcommand{\Sub}{\mathrm{Sub}}
    Our argument is close to \cite[\S2.3]{kammar-mcdermott} and also
    \cite[Thms.~7.5 \& 12.8]{kammar}.

    We let $\MonadT_f$ be the least subfunctor of $\MonadT$ that contains the images of the $f_i$'s and $\eta$, and is closed under the image of monadic bind ($\bind$).
%
    To show that this exists, we proceed as follows. First, fix a regular cardinal $\lambda>I$ such that $Y_i$'s are all $\lambda$-presentable, such that the topos is locally $\lambda$-presentable (e.g.~\cite{lp-adamek-rosicky}). 
    Consider the poset $\Sub_\lambda(\MonadT)$ of $\lambda$-accessible subfunctors of $\MonadT$. The cardinality bound $\lambda$ ensures it is small. Ordered by pointwise inclusion, this is a complete lattice: the non-empty meets are immediate, and the empty meet requires us to consider the $\lambda$-accessible coreflection of $\MonadT$.

We defined $\MonadT_f$ by a monotone property which we can regard as a monotone operator on this complete lattice $\Sub_\lambda(\MonadT)$, and so the least $\lambda$-accessible subfunctor exists. This is $\MonadT_f$. Concretely, it is a least upper bound of an ordinal indexed chain.
    The chain starts with the functor
    \[\textstyle F_0(Z)=\bigcup_{i\in I,g:Y_i\to Z}\mathsf{image}(\MonadT(g)\circ f_i)\quad\subseteq
      \MonadT(Z)\]
    which is $\lambda$-accessible because the $Y_i$'s are $\lambda$-presentable.
    The chain iteratively closes under the image of monadic bind, until we reach a subfunctor that is a sub\emph{monad} of $\MonadT$. 
    
    To see that $\MonadT_f$ is commutative, we appeal to (transfinite) induction. Say that a subfunctor~$F$ of $\MonadT$ is \emph{commutative} if all morphisms that factor through $F$ commute (Def.~\ref{def:affine-monad}), and then note that the property of being commutative is preserved along the ordinal indexed chain. 
        \end{proof}
  With this in mind, fixing a measure
  $\nu_α$ as in \eqref{eqn:graphon-to-measure},
  we form the least submonad $\MonadP_α$ of $\MonadP$ induced by the morphisms
  \begin{equation} \label{eqn:generators-monad}
    \nu_α:1\to \MonadP (\VV)\qquad
   \tbernoulli:[0,1]\to \MonadP(2) 
  \end{equation}
  where $\tbernoulli(r)=r\cdot \eta(0)+(1-r)\cdot \eta(1)$.
  \begin{corollary}\label{coro:least-submonad-is-comm-affine}
    The least submonad $\MonadP_α$ of the probability monad $\MonadP$ induced by
    the morphisms in \eqref{eqn:generators-monad}
    is a commutative affine monad (Def.~\ref{def:affine-monad}).
  \end{corollary}
\begin{proof}[Proof notes]  It is easy to show that $\tbernoulli$ commutes with every morphism
  $X\to \MonadP (Y)$. Moreover, $\nu_α$ commutes with
  itself (Prop.~\ref{thm:graphon:commutative}).
Finally, $\MonadP_α$ is affine since $\MonadP$ is. \end{proof}
\subsection{Summary and Interpretation}
Fix $\alpha\in [0,1]$. We induce an internal measure~$\nu_α$ on the vertices of the Rado
    graph as explained in \eqref{eqn:graphon-to-measure};
    and build a commutative submonad $\MonadP_α$ of $\MonadP$.
    We can then interpret the graph probabilistic programming
    language.
    We interpret types as Rado-nominal sets:
\begin{equation}\label{eqn:radointerp}    
            \sem{\tbool}  = 2
            \qquad 
            \sem{\tvertex}  = \VV
            \qquad
            \sem{\tunit}  = 1
            \qquad
            \sem{A_1\ast A_2}  = \sem{A_1}\times \sem{A_2}\text.
\end{equation}
    We interpret typed programs $\Gamma\vdash t:A$ as Kleisli morphisms
    \[
      \sem\Gamma\to \MonadP_α(\sem A)
    \]
    i.e. internal probability kernels $\sem\Gamma \times 2^{\sem A}\to [0,1]$. 
    Sequencing (let) is interpreted using the monad structure, with 
    $\sem\tnew:1\to \MonadP_\alpha(\VV)$ and
    $\sem\tedge:\VV\times \VV\to \MonadP_\alpha(2)$ as
    \begin{equation}\label{eqn:rado-interp-b}
      \sem{\tnew()} = \nu_α \qquad
      \sem{\tedge}(v,w) = \eta(E(v,w)) \end{equation}
    \begin{corollary}\label{cor:e-r}
      Consider the interpretation in Rado-nominal sets (\eqref{eqn:radointerp}--~\eqref{eqn:rado-interp-b}).
         If we form the sequence of random graphs
          in \eqref{eqn:programgraph},
          then these correspond to the Erd\H{o}s-R\'enyi graphon.
      \end{corollary}
      \begin{proof}[Proof notes.]
        The semantics interprets ground types as finite sets with discrete $\AutRado$ action -- in which case internal probability kernels correspond to stochastic matrices, agreeing with $\FinStoch$. Thus, the theory is Bernoulli-based.
      To see that the graphon arises, consider for instance when $n=2$, we have:
      \[
          \sem{t_2}(\star)= 
           \int \begin{pmatrix}
              [E(x_1,x_1)], [E(x_1,x_2)]
              \\
              [E(x_2,x_1)], [E(x_2,x_2)]
              \end{pmatrix} (\nu_α \otimes \nu_α)(\dd(x_1, x_2))
      \]
      for $t_2$ as in \eqref{eqn:programgraph}, and therefore
      \[
      \sem{t_2} = 
        \begin{pmatrix}
          \delta_0, &\tbernoulli(α)
          \\
          \tbernoulli(α), &\delta_0
        \end{pmatrix}:\MonadP(2^4)
      \]
      For general $n$, this corresponds to the random graph model $p_{W_α,n}$ for the Erd\H{o}s-R\'enyi graphon~$W_α$.
    \end{proof}

      
%%% Local Variables:
%%% mode: latex
%%% TeX-master: "popl24"
%%% End:
