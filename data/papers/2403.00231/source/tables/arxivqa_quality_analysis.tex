\begin{table*}[t!]
    \centering
    \tiny  
    \begin{tcolorbox}
1. Ensuring Factual Integrity and Clear Presentation:\\
- Factual Alignment: Ensure questions and options are grounded in accurate reflections of the chart data.\\
- Visual Clarity: Maintain high-resolution charts to ensure that all pertinent details are discernible.\\
- Unambiguous Textual Information: Employ precise and unambiguous language to formulate questions and answers, thereby mitigating potential misinterpretations.\\
2. Ensuring Relevance and Integrated Comprehensiveness:\\
- Question and Option Relevance: Charts must align with their questions, and all options should be applicable and relevant to the given data.\\
- Comprehensive Integration: Guarantee the provision of comprehensive information necessary for the interpretation of the chart and the resolution of the question, ensuring a cohesive amalgamation of textual and visual data.\\ 
3. Promoting Fairness and Avoiding Bias:\\
- Equitable Content: Strive for impartiality in the dataset to prevent bias and ensure fair representation of diverse groups and perspectives.\\

Grading Protocol:\\
Each criterion is to be rigorously evaluated for each dataset entry. The assessment is to be conducted on a qualitative scale with three distinct levels: High, Medium, and Low. These levels will denote the degree of conformity to the respective criterion:\\
- 1: High: The dataset entry exhibits exemplary adherence to the evaluation criterion, demonstrating a robust and comprehensive alignment with the specified standard.\\
- 0.5: Medium: The dataset entry meets the evaluation criterion to a moderate extent, indicating a satisfactory but not optimal congruence with the standard.\\
- 0 Low: The dataset entry falls short of the evaluation criterion, signaling a need for significant improvements to meet the standard.
    \end{tcolorbox}
    \caption{Annotator guideline of ArXivQA manual quality examination.}
    \label{tab:quality_analysis_ArXivqa}
\end{table*}

\begin{table}[t!]
    \centering
    \small 
    \begin{tabular}{@{}lc@{}}
         \toprule
         Aspect & Avg Score \\
         \midrule
        Factual Alignment & 0.6975 \\
        Visual Clarity & 0.9925 \\
        Unambiguous Textual Information & 0.9825 \\
        Question and Option Relevance & 0.9375 \\
        Comprehensive Integration & 0.905 \\
        Equitable Content & 1.0 \\
        \midrule
        Score Sum & 5.515 \\
        \bottomrule
    \end{tabular}
    \caption{Manual quality analysis of ArXivQA. The average scores for each aspect are presented.}
    \label{tab:quality_analysis_ArXivqa_result}
\end{table}