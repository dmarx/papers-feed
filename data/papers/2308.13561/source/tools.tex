\section{Recording Tools}
\label{sec:recordingtools}

The primary interface to interact with \AriaDevices{} is a mobile phone companion app. Recordings can be initiated and stopped via the app or the device's capture button.
The app is also used to set the sensor configuration by selecting a recording profile. The profile controls which sensors record, at what frequency and resolution, as well as the output format (e.g., storing images in RAW or JPEG format). 
Multiple recording profiles are available to tailor for different research use cases. Additional profiles are being added as necessary. Once a recording is made on the device, thumbnail previews are available on the companion app for convenient review of the captured data. These functionalities are illustrated in Figure~\ref{fig:companion-app}.


Once the recording is complete, the user can download the recorded data from a \AriaDevice{} via an USB connection to a local machine for further processing. A user can optionally upload their recordings to our Machine Perception Services (MPS), which apply state-of-the-art processing to recover device trajectories, online calibration, a semi-dense point cloud and eye gaze information (see Section~\ref{sec:mps} for details).

All device sensors are recorded in VRS file format~\cite{vrsdocs}.  We selected VRS as the data container because it is an open file format designed to record and playback streams of AR sensor data and because it supports very large file sizes. The VRS files contain streams of time-sorted records generated for each sensor, with one set of sensors per stream.

In order to easily visualize and interact with data, we provide \ProjectAria{} tools as part of an open source repository \cite{ariatools}. This is a set of tools and libraries for accessing, visualizing and manipulating recordings from Aria. The C++/Python toolkit includes VRS Data Provider and Viewer interfaces, enabling researchers to read and visualize \ProjectAria{} sequences, to retrieve and interact with device calibration data, and to read and process the output of the MPS.
More details about these tools are available from documentation website \cite{ariadocs}.
The \ProjectAria{} tools are available to install from PyPI via
{\tt pip install projectaria\_tools} (see \cite{ariatools} for more details).


