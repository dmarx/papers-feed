\section{Privacy Considerations}
\label{sec:privacy}

AR glasses and in general egocentric recording devices such as \AriaDevices{} promise to make technology more accessible, but also pose novel and unique challenges for security and privacy: the more they succeed in being unobtrusive, the more important it becomes to preserve and respect the privacy not only of the wearer, but also that of bystanders and individuals the wearer interacts with. 

A stated goal of  \ProjectAria{} is to pioneer responsible innovation for research leveraging egocentric data and devices, both by establishing guidelines and principles to preserve privacy of wearers and bystanders as well as by building privacy-facilitating features directly into the device where possible. 

Throughout the development of \ProjectAria{} we followed Meta’s Responsible Innovation Principles~\cite{metari}, which express our commitment to building inclusive, privacy-centric products. 

In concert with our principles we have designed and made available privacy-centric hardware and software features to research partners. The \AriaDevice{} has an LED indicator that signals to bystanders when the device is recording raw data. Furthermore, \AriaDevices{} have a privacy switch: When activated during a recording session, the device immediately stops and deletes the current recording. This allows a wearer to immediately and easily fulfill a request of a bystander to delete any audio or video recording that might have been captured of them. 

We require all \ProjectAria{} partners to follow the \ProjectAria{} community guidance \cite{communityguidelines} and to practice responsible research, protecting the privacy of those who wear the research devices, and most importantly, those who do not.
